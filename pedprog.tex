\documentclass[magyar,12pt,a4paper,draft]{report}
\usepackage{graphicx}

\usepackage[T1]{fontenc}
\usepackage[utf8]{inputenc}
\usepackage[magyar]{babel}

\usepackage[colorinlistoftodos,prependcaption,textsize=tiny]{todonotes}

\usepackage{verbatim}
\usepackage{markdown}
\usepackage{booktabs}
\usepackage{longtable}
\usepackage{array}
\usepackage[justification=centering]{caption}
\usepackage{url}
\usepackage[nottoc,numbib]{tocbibind}
\DeclareUnicodeCharacter{2212}{-}
% heylyseglista tablazat miatt
\usepackage{graphicx}
\usepackage{lscape}
\usepackage{natbib}
\pagestyle{headings}
\usepackage{fancyhdr}
\usepackage[yyyymmdd,hhmmss]{datetime}
\pagestyle{fancy}

\fancyhf{}
\rhead{}
\lhead{\leftmark}

\rfoot{Compiled on \today\ at \currenttime}
\cfoot{}
\lfoot{}
\begin{document}
% csinalunk egy feltetelt, hogy a szovegben tudjuk, hogy kerettantervben vagyunk
% vagy a pedprogramban
\newif\ifkerettanterv
\kerettantervfalse

\title{Budapest School pedagógia program}
\author{Budapest School általános iskola tanárai}
\maketitle

Kedves Olvasó,

Ezekre figyelj:
\begin{itemize}
  \item \ref{sec:iskola_celja} a kerettantervbol jott at
\item \ref{sec:tanulasszervezesi_elvek} egy teljesen új szöveg.
\item \ref{sec:milyenek_vagyunk} fejezet árkerült a kozepiskola oldalrol. Milyenek vagyunk. Lehet, hogy egyeztetni kene az elotte levo alapertettekkl.
\item \ref{sec:kapcsolat_reformokkal} kiket tekintunk mintanak? kit meg?
\end{itemize}

\listoftodos[Notes]

\tableofcontents
\newpage
\part{Nevelési program}

\chapter{Az iskola célja és alapelvei}
\label{sec:alapelvek}

\begin{quote}
  Az iskolában folyó nevelő-oktató munka pedagógiai alapelvei, értékei, céljai, feladatai, eszközei, eljárásai.

  Ide kene valami frappans idezet. Talan Sugata Mitratol. Mondjuk ilyen We need to look at learning as the product of educational self-organization. It’s not about making learning happen; it’s about letting it happen.



\end{quote}
\section{Az iskola célja}
\label{sec:iskola_celja}

A Budapest School abban támogatja a gyerekeket, hogy azok az
attitűdök, képességek és szokások alakuljanak ki bennük, amelyek segítségével
boldog, egészséges és a társadalom számára hasznos felnőttekké válhatnak. A
cél, hogy a gyerekek a mai világ szükségleteihez és lehetőségeihez a saját
erősségeik felhasználásával kapcsolódhassanak.	Hogy tanulási útjukat
sajátjukként éljék meg, és felelősséget érezzenek az alakításáért, az újabb és
újabb kihívások megtalálásáért.

Olyan tanulási környezetet kell kialakítani ehhez, ahol a szülők, tanárok és
gyerekek tesznek magukért és egymásért, ahol gyerekeink képesek nehéz
helyzetekben is életük, kapcsolataik és környezetük aktív alakítói lenni,
cselekedeteikkel, tetteikkel elérni a kitűzött céljaikat.

Mindenki kíváncsinak születik, és meg tudja tanulni azt, amit
igazán szeretne. Nincs is másra szükség, csak izgalmas kihívásokra, kérdésekre,
biztonságra, támogatásra és lehetőségekre.

Ennek szellemében az iskolánk elsődleges feladata, hogy a gyerekek közösségét
segítsék abban, hogy sokat és hatékonyan tanuljanak és alkossanak.
\emph{Tanulják azt, amit szeretnének, és azt, amire szükségük van.}


\section{A fejlődésközpontú módszer}
A Budapest School Általános Iskola és Gimnázium mikroiskoláinak legfőbb
pedagógiai alapelve, hogy mindig ott, akkor és az történjen a
gyerekekkel, ami őket a fejlődésükben a leginkább támogatja. Minden, ami
az iskolában történik újra és újra erre az alapkérdésre kell, hogy
visszatérjen. Az segíti a leginkább a gyerekek fejlődését, amit most
csinálunk, vagy változtatnunk kell rajta? A Budapest School
tehát rugalmas és integratív, a gyerekek fejlődéséhez igazodó. Abban
támogatja a gyerekeket, hogy örömmel, harmóniában, mégis folyamatos
kihívásokat keresve fejlődjenek, és így egészséges, boldog és a
társadalom számára hasznos felnőttekké váljanak. A Budapest School
célja ennek megfelelően, hogy a mikroiskoláiba járó gyerekek
megtanuljanak saját erősségeiket fejlesztve egyéni célokat állítani és
azokat később a mai világ adta lehetőségekhez és szükségletekhez
igazítani.

Az oktatás tartalma helyett ezért a Budapest School a tanulás módjára
helyezi a fő hangsúlyt, és ezt támogatandó pedagógiai alapelve, hogy
integratív módon fejlessze és alkalmazza a tanár és gyerekcsoportban
azokat a pedagógiai, pszichológiai és szervezetfejlesztési módszereket,
melyek korszerű módon tudják segíteni a tanulás tanulását, az egyéni és
csoportos fejlődést, a konfliktusok feloldását.

A tanulás tartalmát tekintve a Budapest School saját alternatív
kerettantervére támaszkodik, amely a tanulás rendszerét, annak
folyamatát szabályozza. E dokumentum alapján az állami kerettanterv
tanulási eredménycéljain történő végighaladás mellett a Budapest School
nagy hangsúlyt fektet a gyerekek egyéni tanulási céljaira és a
célállítás módjára.

A gyerekek életkori, fejlettségi, valamint szociális és érzelmi
állapotának figyelemmel kísérésében, valamint az egyéni tanulási célok
kitűzésében és elérésében minden gyereket egy mentor tanár kísér végig
az iskolán. A mentor tanár személye változhat az iskolában töltött évek
alatt, de arra is van lehetőség, hogy ugyanaz a mentor kísérjen végig
egy gyereket az általános iskola első osztályától az érettségiig.


\section{Tanulásszervezési elvek}
\label{sec:tanulasszervezesi_elvek}
A tanulás módja egyénenként változó. Az alábbi elvek azonban minden
korosztályban állandóan befolyásolják és keretezik a Budapest School
pedagógiai programját

\paragraph{A fejlődésközpontú szemlélet}

A gyerekek tanulása fejlődésközpontú szemléletben (growth mindset)
történik. Ami alapján a diákok és a tanárok számára nagyobb fontossággal
bír a tanulásba fektetett erőfeszítésük, mint a képességeik.
Erőfeszítéseik segítségével képességeik fejleszthetők,
megváltoztathatóak. Számukra inspiráló a kihívás, és a hibázás kevésbé
töri le lelkesedésüket.

\paragraph{A tanulás folyamata}

A tanulás egy olyan folyamat, amely különböző állomásokra, rövid célokra
bontható. A tanulási célok állításának folyamata, annak minősége évek
alatt folyamatosan változik, egyre tudatosabbá, pontosabbá, komplexebbé
válhat. Azonban már az első évektől el kell kezdeni annak gyakorlását,
hogy később kialakulhasson az önálló tanulási cél állítás.

\paragraph{A hibázás tisztelete}

A hibázás a gyakorlás és az új megismerésének jele. A hibázás
feszültségmentes kezelése kulcsfontosságú abban, hogy a gyerekek merjék
feszegetni a saját határaikat, hogy magabiztosan dolgozzanak azon, hogy
képességeiket, ismereteiket, vagy gyakorlataikat folyamatosan
fejlesszék. A hibázásból való tanulás fő célja, hogy mindig új hibákat
ismerjünk meg, és a korábbiakra minél jobb megoldásokat találjanak a
gyerekek.

\paragraph{A zárkózottság és fókuszált
tanulás}

A tanulás módja nagyban függhet attól, hogy egy gyerek mennyire
zárkózott, mennyire tud és akar önállóan tanulni. A csoportos munkák
során alapelv, hogy a zárkózott gyerekek is lehetőséget kapjanak, hogy
írásban, csöndesen, vagy kisebb csoportban végezhessék a munkájukat,
mondhassák el ötleteiket. Az egyéni tanulásban minden gyereknek
lehetőséget kell adni arra és segíteni kell abban, hogy önállóan,
fókuszáltan tudjon tanulni.

\paragraph{A mindennapok
nehézségei}

A gyerekek tanulását családi hátterük változása, egyéni problémák,
számos mindennapi esemény befolyásolhatja. Ezek figyelembe vétele a
mindennapokban, a mentor tanárral való bizalmi viszonynak köszönhetően
válik lehetségessé. Ennek a kapcsolatnak az alapjait ezért a partnerség,
az értő figyelem adja.

\paragraph{A trauma feldolgozása}

Egy gyerek traumatizálásának számos oka lehet, az ilyen esetekben
szakember bevonása szükséges, és annak megvizsgálása, hogy mi segítheti
a legjobban a trauma feldolgozását. Ilyen esetekben az egyéni célok
megváltoztatásával, a tanulási ritmusának, ütemének újrakeretezésével
lehet segíteni egy gyereket abban, hogy a tanulás komfortos legyen a
számára.

\paragraph{Egyéni és csoportos
tanulás}

A tanulás egyénileg és csoportokban is történhet. A csoportok
megszervezése mindig azon múlik, hogy az adott tanulási célt mi
szolgálja a legjobban. Ennek megfelelően a gyerekek nem állandó, hanem a
tanulási célokhoz, az érdeklődéshez, a képességi szintekhez alkalmazkodó
rugalmas csoportokban tanulnak.

\paragraph{Esélyegyenlőség a
tanulásban}

A Budapest School kiemelt figyelmet fordít arra, hogy a sajátos nevelési
igényű tanulók is lehetőséget kapjanak a csoportban való munkára.
Tanulásukat amennyiben szükséges, külső szakember bevonásával segíti. A
Budapest School a hátrányos helyzetű tanuló számára is biztosítani
kívánja az elfogadó, fejlesztő környezetet. Az egyenlő bánásmód
megvalósulása érdekében olyan differenciált tanulási környezetet alakít
ki, ami biztosítja a minél nagyobb mértékű inkluzivitást. A tanulási
esélyegyenlőség feltételeinek megvalósulását az egész napos iskola is
nagyban segíti.

\paragraph{Projektek és gyakorlatias
tanulás}

A tanulás három rétege, az ismeretszerzés, a gondolkodás fejlesztése és
a gyakorlatias, aktív alkotás egyszerre jelenik meg a Budapest School
mindennapjaiban. Az alkotó munka rugalmas időkereteket, változó
csoportbontásokat, és a projektmódszerek sokszínű alkalmazását igényli.
A tanulás ilyenkor sokszor csinálássá válik, az ismeret pedig termékké
változik.

\paragraph{Önirányított
tanulás}

A kisgyerekkor élettani jellemzője a kíváncsiság, az igény a
felfedezésre, tapasztalásra. A tanulás számukra egy önvezérelt aktív
folyamat, melynek megtartása és folyamatos fejlesztése a Budapest School
tanárainak legfőbb feladata, hogy ez a későbbiek során rögzült
viselkedésformává válhasson. Két út van, megtanítani a gyerekeket azokra
a képességekre, amelyekre szükségük van, ezzel kockáztatva azt, hogy a
tanulásuk a világ változásával veszít korszerűségükből, vagy abban
segíteni a gyerekeket, hogy megtaníthassák magukat azokra a
képességekre, amelyekre épp az adott élethelyzetükben szükségük van. A
tanulás így élményszerűvé válik, ismeretszerző jellege csökken, és nő az
önálló felfedezés lehetősége.

\paragraph{A tanár szerepe a tanulási
folyamatban}

A Budapest School tanulásszervezői partnerként, a tanulás folyamán
segítő társként vannak jelen a gyerekek életében. A tanulás tanórák
helyett pontos tanulási célokat tartalmazó tanulási modulokból épül fel,
melyek során a tanár az adott cél eléréséhez szükséges eszközöket,
tanulási segédleteket biztosítja. A tanár akkor és annyira segíti a
gyerekeket a saját céljuk elérésében, amennyire azt a gyerek igényli, és
folyamatosan tekintettel van arra, hogy a gyerek saját fejlődési üteme
megvalósulhasson. Ehhez tudatosan kell kezelnie nem csak a gyerekeket
érintő fejlesztési lehetőségeket, hanem azt is pontosan látnia kell,
hogy egy adott tanulási cél elérésének milyen készségszintű, vagy
gyakorlatias alapfeltételei vannak. Ezért a gyerekek tanulási célját
támogatandó segítenie kell abban, hogy a gyakorló idő, a gyakorlatias
alkotó idő és az új ismeret megszerzésének ideje folyamatos egyensúlyban
legyen.

\paragraph{A tanulási rend}

A tanulás délelőtti és délutáni idősávokra tagolódik, melynek pontos
alakítása a gyerekek tanulási igényeitől, fejlettségi szintjétől és
korosztályától is függ. A tanulási rend meghatározásáért a Budapest
School mikroiskoláinak tanulásszervzezői felelnek. A reggeli
beszélgetőkör minden iskola napinditásának alapeleme. Ez a tere annak,
hogy a mikroiskolák tagjai megbeszélhessék közös ügyeiket. A reggeli kör
után után a tanulás trimeszterenként újraszervezett módon tanulási
modulokban történik, melynek során jut idő egyéni és csoportos,
gyakorló, ismeretszerző és gyakorlatias foglalkozásokra is. Az egész
napos iskola lehetőséget biztosít arra, hogy a tanulási egységek között
legyen idő fellélegezni, és felkészülni az újabb modulokra, valamint
arra is, hogy ha egy tanulási egység nagyobb magával ragadja a
gyerekeket, akkor benne maradhassanak és annak megfelelően alakítsák
újra az időrendjüket.


\section{Milyenek is vagyunk}
\label{sec:milyenek_vagyunk}

A tanár, a tanuló, az iskolai közösség és a család szorosan összefüggő, bonyolult kapcsolat-rendszert alkotnak. Fontos számunkra, hogy ebben a kapcsolatban a legfontosabb kérdésekben hasonlóan gondolkodjunk.
\paragraph{A tanuló és a tanulás}
\begin{itemize}
  \item
  A gyermek valamennyi képessége fejleszthető jól megtervezett gyakorlással.
\item
Van olyan helyzet, amikor egy esztergagép használatának az elsajátítása pont annyira fontos lehet, mint jól elvégezni egy barokk vers elemzését.
\item
Az iskolában nem az a legfontosabb, hogy felkészüljünk az érettségire.
\item
Értelmetlen, hogy az érettségin egyedül, mindenféle technológia használata nélkül kell a gyerekeknek feladatot megoldani, amikor a felnőtteknek egyre inkább csapatban és a hálózat segítségével kell boldogulniuk.
\item A Google-kereső segítségével megválaszolni egy kérdést fontosabb képesség, mint fejből emlékezni egy ritkán használt információra.
\end{itemize}

\paragraph{Szülői attitűdök, értékek}
Az iskolában azokkal a szülőkkel szeretnénk együttműködni, akikre érvényesek a következő állítások.

\begin{itemize}
\item Szívesen engedem, hogy gyerekem azt tanulja, amit szeret, vagy amire szerinte szüksége van.
\item A gyerekem az érettségire szinte magától fel tud készülni az internet és korrepetáló tanárok segítségével.
\item Fontosnak tartom, hogy gyerekem törődjön másokkal, mások érzéseivel.
\item Fontosnak tartom, hogy a gyerekem tanáraival jó kapcsolatom alakuljon ki.
\item Az iskolaválasztás közös családi döntés.
\item Előre tudatosan készülök rá, hogy hogyan fogjuk a tinédzser / fiatal felnőtt gyerekemmel újradefiniálni a kapcsolatunkat.
\item Párommal, gyerekeim szüleivel meg tudjuk beszélni, és megállapodásra tudunk jutni arról, hogy milyen iskolát szeretnénk gyerekeinknek.

\end{itemize}

\paragraph{Amivel nem ért egyet az iskola.}

\begin{itemize}
\item  Inkább elkerülöm a nehéz, konfliktusos szituációkat.
\item Sokszor kell veszekedni a gyerekemmel, hogy megcsinálja a kötelező dolgokat.
\item Az iskolai jegyek jól mutatják a gyermekben lévő potenciált a boldogságra, egészségre vagy későbbi teljesítményére.
\item A gyermek matematikai képességei attól nem változnak, hogy matekproblémákon gondolkozik.
\end{itemize}

\section{Kapcsolat a hazai és nemzetközi oktatási
reformokkal}\label{sec:kapcsolat_reformokkal}

A Budapest School programja a hazai és nemzetközi oktatási reformok
kontextusában és a pszichológia, a szociálpszichológia, valamint a
szervezetfejlesztés terén elvégzett kortárs kutatások tükrében válik
könnyebben értelmezhetővé.

Magyarországról több iskola története, működése is nagy hatással volt ránk. Az
alternatív iskolák hagyományát többek között a 90-es évektől a Rogers
Személyközpontú Általános Iskola, a Lauder Javne Iskola, a Kincskereső
Iskola és a Gyermekek Háza iskolák teremtették meg. A megújuló
középiskolák úttörője az Alternatív Közgazdasági Gimnázium és a
Közgadasági Politechnikum voltak. Ezek az iskolák a személyközpontúság,
a gyerekközpontúság hangsúlyozása mellett elkezdték a gyakorlatban
alkalmazni a differenciálás, a kooperetív technikák alkalmazását és
egyes projektmódszertanokat. Programunk kidolgozásában nagy szerepe volt
annak, hogy ezek az iskolák olyan szemléletmódbéli alapokat fektettek
le, amelyek mára alapelvárásként fogalmazdónak meg a szülők oldaláról az
iskolákkal szemben.

Gyakorlati tapasztalatokat a világ más részein is gyűjöttünk. A 21.
Században a Budapest Schoolhoz hasonló kezdeményezések sorra indulnak a
világban. Ezek egyes jegyei a Budapest School pedagógiai programjával
összhangban vannak:

A Wildflower School mikroiskolák hálózatát működteti kisebb
üzlethelyiségekben. A Budapest Schoolhoz hasonlóan célja, hogy falakat
romboljon a gyerekek és a világ között: a magántanulás és az intézményes
tanulás, a tanár és a tudós szerepe, valamint az iskola és környezete
közötti határok elmosása az egyik fő üzenete.
https://wildflowerschools.org/

Hasonlóan az otthontanulás és az  unschooling struktúrált formáját keresi az Amerikai Texasban alapított Acton Academy, ami a szokratikus módszereket (azaz, hogy megbeszéljük közösen), valós projekten keresztüli tanulást, és a gyakornokoskodáshoz hasonló munka közbeni tanulást (,,learning on the job") teszi a megközelítésének középpontjába.
https://www.actonacademy.org/



A High Tech High iskoláiban a gyerekek elsősorban projektmódszertan
alapján tanulnak. A tanulási jogokban való egyenlőség mellett az egyéni
célokra szabott tanulás, a világ alakulásához kapcsolódó tartalmi
elemek, valamint az együttműködés alapú tanulás is megjelenik
pedagógiáukban a Budapest School által is alkalmazott jegyekből.
https://www.hightechhigh.org/

A School21 brit iskola a 21. századi képességek fejlészétését tűzte ki
célul. Ennek jegyében a prezentációs, előadói skillek kiemelt
jelentőségüek. Iskolájuk egyensúlyt akar teremteni a tudásbéli
(akadémiai), a szívbéli (személyiség és jóllét) és a kézzel fogható
(problémamegoldó, alkotó) között. A Budapest School iskoláinak hasonló
módon célja, hogy a tanulás három rétegét, a tudást, a gondolkodást és
az alkotást folyamatos harmóniában tartsa. https://www.school21.org.uk

A Khan Lab School Monterossi módszert keveri az online tanulással. Kevert korosztályú csoportokban, személyreszabott módszerekkel segítik a képességfejlesztést és a projekt alapú munkát.
https://khanlabschool.org/

\section{Pedagógiai és pszichológiai háttér}

Az iskolák gyakrolatias tapasztalata mellett a Budapest School számos
elméletet is kiemelten fontosnak tart.

Ezek közül is oktatási programjának központjában Carol Dweck
fejlődésközpontű szemlélete áll. E mellett nagy hangsúlyt fektetünk az
alábbi elméletek gyakorlati alkalmazására is:

Reformpedagógiai irányzatok elméletei, különös tekintettel:

\begin{itemize}

\item
  Motessori-pedagógia (Maria Montessori),
\item
  kritikai pedagógia (Paulo Freire)
\item
  élménypedagógia (John Dewey)
\item
  felfedeztető tanulás (Jerome Bruner)
\item
  projektmódszer (William Kilpatrick)
\item
  kooperatív tanulás (Spencer Kagan)
\end{itemize}

Pszichológia és szociálpszichológiai kutatások eredményei:

\begin{itemize}

\item
  kognitív interakcionista tanuláselmélet (Jean Piaget)
\item
  személyközpontú pszichológia (Carl Rogers)
\item
  kommunikáció és konfliktuskezelés (Thomas Gordon)
\item
  erőszakmentes kommunikáció (Marshall Rosenberg)
\item
  pozitív pszichológia kutatási eredményei, különös tekintettel:
  flow-elmélet, kreativitás kutatások (Csíkszentmihályi Mihály)
\item
  érzelmi és társas intelligencia (Daniel Goleman)
\item motiváció kutatások, amiket jól összefoglal \citep{pink2011drive}
\item
  hősiesség pszichológiai alapjai (Phil Zimbardo)
\item
  fejlődésfókuszú szemlélet (Carol Dweck)
\end{itemize}

\section{Pedagógia
módszerek}
Budapest School iskola tanárainak feladata, hogy mindig keressék azt a módszert, azt a környezetet, ami az adott gyerekekkel, adott mikroiskolában a leginkább működik. Nem tudjuk előre megmondani, hogy mikor milyen módszert érdemes választani, de azt tudjuk, hogy XXXX





\subsection{Projektmódszer}
Projekt módszert alkalmazó modulok során fő célünk, hogy a gyerekek aktivak és kreatívak legyenek, és ezért a tevékenységek sokszinűségét helyezzük fókuszba. Projektmódszer esetén is bátorítva van minden tanár, hogy a legváltozatosabb módszertárral közelítse a gyerekeket, figyelve arra, hogy a tanár attitűdje, a csoport dinamikája és az aktuális tevékenységek mit kívánnak.

Projektmunka folyamata.

\begin{description}
  \item[Téma, cél] Első lépés, hogy meghatározzuk a projekt témáját vagy célját. Ez jöhet tanártól, a gyerekektől. Fontos, hogy a gyerekek a projekt témáját már önmagában értelmesnek, relevánsnak tartsák.

  \item [Ötletroham]  Egy-egy téma feldolgozását csoportalakítással és ötletrohammal kezdünk. Ennek célja, hogy résztvevők bevonódjanak, illetve megmutassák, hogy nekik milyen elképzeléseik vannak az adott témáról, továbbá milyen produktummal, ereménnyel szeretnék zárni a folyamatot. A létrehozott produktumoknak csak a képzelet szabhat határt. Lehetnek videók, prezentációk, fotók, rapdalok, telefonos applikációk, rajzok, tablók stb.

  \item [Kutatói kérdés] Ezek után úgynevezett kutatói kérdéseket teszünk fel, melyek meghatározzák a vizsgálat irányát. A kérdések feldolgozása a legváltozatosabb módon történhet. Az egyéni munkától kezdve, a forntális instruáláson vagy a kooperatív csoportmunkán keresztűl egészen a dráma- és zenefoglalkozásokig minden hasznosítható a tanár, a csoport és a téma igényeihez mérten.
  \item [Elmélyült csoportmunka] A projekt azon szakasza, amikor a tervek, kutatások alapján az implementáción dolgozik a csapat.
  \item [Prezentáció] A létrehozott produktumokat bemutatására külön hangsúlyt kell fektetni. Ennek több módja is lehet: prezentációk, demonstrációk, plakátok, projekt fesztiválok.
\end{description}

Az iskolai projektek célja egy fejlődésés fókuszú gondolkodásmódú tanár és gyerek számára mindig kettős: egyrészt cél a téma feldolgozása, a produktum létrehozása, másrészt az iskola fő célja, hogy a gyerekek, a csapatok mindig fejleszzék alkotó, együttműködő, probléma megoldó stb. képességüket. Ezért a projekt folyamatára való reflektálás, visszajelzés ugyanolyan olyan fontos, mint maga a cél elérése.

A munka során külön figyelmet kell fordítani arra, hogy mindent dokumentáljanak a résztvevők. Lehetőleg online felületen.
\paragraph{Értékelés} A projekt során több értékelési pontot érdemes beépíteni. A foglalkozások végén a résztvevők visszajeleznek a folyamatra, értékelik a saját, a csoport és tanár munkáját. A folyamat végén az egész projektfolyamatot értékelik, szintén kitérve a saját, a csoport és a tanár munkájára. A produktumok, az eredmény értékelése csoport és egyéni szinten is megtörténik.

Értékelésnél fontos a fejlődésfókuszú gondolkodásmódot fejlesztő visszajelzést adni: inkább a folyamatra magára és ne a végeredményre fókuszáljon a visszajelzés.

\subsection{Önszerveződő tanulási környezet, SOLE}
Sugata Mitra által kialakított módszertan lényege, hogy a tanárok arra bátorítják a gyerekeket, hogy csoportban, az internet segítségével Nagy Kérdéseket válaszoljanak meg. A jó kérdés az, amire nem egyszerű a válasz, sőt lehet, hogy nincs is rá egyfajta választ. Cél, hogy a gyerekek maguk alakítsák a folyamatot, formálják a kérdést és alakítsák a válaszokat.

\begin{itemize}
  \item Tanár kialakítja a teret: körülbelül négy gyerekenként egy számítógép, amit körbe lehet ülni.
  \item Gyerekek maguk formálják a csoportjukat, sőt, még csoportot is válthatnak a munka során. Mozoghatnak, kérdezhetnek, ,,leshetnek'' a másik csoportoktól.
  \item Körülbelül 30-45 perc után a csoportok prezentálják a kutatásuk eredményét.
\end{itemize}



\paragraph{A jó kérdések}
Nagy kérdésekre nincs könnyű válasz. Sokszor nyílt és nehéz kérdések; de még azért előfordulhat, hogy senki sem tudja rá választ. A cél, hogy mély és hosszú beszélgetéseket generáljon a kérdés. Ezek azok a kérdések, amikre érdemes nagyobb elméleteket állítani, amiket jobb csoportban megvitatni, amikről érvelni lehet és kritikusan gondolkodni.

A jó kérdések több témát, területet (tantárgyat) kapcsolnak össze: ,,Mi a hangya'' kérdés például nem érint annyi különböző területet, mint a ,,Mi történne a Földdel, ha minden hangya eltünne''.

\paragraph{Fegyelmezés}
Tanár feladata a folyamat során meghatározni a nagy kérdést, és tartani a kereteket. A cél, hogy a gyerekek maguk szervezzék saját munkájukat, így minimális beavatkozás javasolt. Kezdetben, gyakorló csoportoknál a tanárnak sokszor kell emlékeztetni magát, hogy idővel kialakul a rend. ,,Bízz a folyamatban!''

Amikor a tanár úgy látja, hogy nem megy a munka, akkor csak finoman emlékezteti a csoportokat, hogy lassan jön a prezentáció ideje. Amikor valaki a csoportjáról panaszkodik, akkor elmondhatja, hogy szabad csoportot váltani. Ha valaki zavarja a többieket, akkor megfigyelheti, hogy a gyerekek tudnak-e már konfliktus feloldani. Ha valaki nem vesz rész a munkában, akkor gondolkozhat olyan kérdésen, ami a demotivált gyerekeket is bevonza.

\subsection{Megfontolt gyakorlás}
Ahogy \citep{ericsson2016peak} is kimutatja mindenki tudja, minden készségét, képességét fejleszteni, ha megtervezetten, megfontoltan gyakorolja. Budapest School a hagyományosan készségtárgyként számon tartott ének, rajz, testnevelés és technika témákon kívül nagyon sok mindent kezel készségként: írásbeli érettségi vizsgát tenni magyarból, geopolitikai elemzéseket végezni, hiperbolikus függvényekkel való egyenlet megoldás ugyanúgy értelmezhetőek készségként, mint domináns csoporttagokkal való együttműködés, van egy stresszhelyzetben lenyugtatni magunkat.

A készség és képesség fejlesztés legjobb eszköze a megtervezett gyakorlás: fejlődés érdekében okosan gyakorlunk. A megfontolt gyakorlás jellemzője, hogy

\begin{description}
  \item[Világos és specifikus cél] Fontos, hogy tudjuk, mit gyakorlunk, mit akarunk elérni.  Lehetőleg a cél legyen mérhető és mindenképp realisztikus, elérhető.
  \item[Fókusz] Gyakorlás során egy dologra érdemes figyelni
  \item[Konfortzónán kivül kell lenni] Az edzőnek, tanárnak, trainernek néha érdemes a tanulót kicsit ,,nyomni''. Emlékeztetni, hogy mindig lehet kicsit többet elérni.
  \item[Folyamatos visszajelzés] Nagyon gyakran kap a tanuló visszajelzést, tudja, mindig tudja, hogy mikor és miben fejlődött.
\end{description}


\section{A Budapest School mikroiskola-hálózata}


A Budapest School mikroiskolák hálózataként működik. A mikroiskola a Budapest
School iskolahálózatának legkisebb egysége. Nem egy önálló intézmény, és nem a
telephely szinonimája.

Az egyes mikroiskolákban a gyerekek kevert korosztályú tanulóközösségként,
érdeklődésük és képességeik alapján, együtt és egymástól tanulnak és fejlődnek.
A tanulóközösség fontos célja, hogy biztonságot, támogatást nyújtson, és
\emph{így} segítse a közösség tagjainak a minőségi tanulását. Az egyes
mikroiskolák egymással is kapcsolatban vannak,
egymástól tanulnak, így a gyerekek tanulási helye változhat.

\paragraph{Épületek, tagintézménzek és telephelyek}
A Budapest School iskola egy székhellyel és több tagintézménnyel, illetve
telephellyel működhet. Fontos azonban, hogy egy telephelyen több mikroiskola is
működhet, és egy mikroiskola több telephely adottságait is kihasználhatja.

A 
\ifkerettanterv
kerettanterv
\else
program
\fi tudatos szándéka, hogy az épületet és a
tanulás szervezeti formáját ne kösse össze szorosan. Egy-egy mikroiskola az
épületére úgy gondol, mint egy átmeneti bérleményre vagy egy helyre, amit most
meglátogat.
Változhat, hogy egy-egy épületet éppen melyik mikroiskola használja.

A tanulás helyszínének változtathatósága lehetővé teszi, hogy a
múzeumpedagógiát, a tudományos kutatóközpontokkal való együttműködést, az erdei
iskolák világát, a sportegyesületek tevékenységeit, vagy más külső helyszínen
megvalósuló szakköröket a Budapest School gyerekek számára a mindennapok
integrált részvé tegyük. Fontos kiemelnünk, hogy mindeközben mindenkinek
szüksége van bázisra, biztonságot adó otthonra: ezért van minden Budapest
School gyereknek és tanárnak egy elsődleges helye.
\ifkerettanterv
  \section{A Budapest School fenntartója}
  A Budapest School fenntartójának a nemzeti köznevelésről szóló 2011. évi CXC.
  törvényben (a továbbiakban: Nkt)  szabályozottakon felül feladata
  \begin{itemize}
    \item a Budapest School hálózatának építése, működési struktúrájának
          fejlesztése, az adminisztratív és szabályozási rendszer kialakítása,
          valamint az egyes mikroiskolákban a gyerekek tanulását, fejlődését segítő
          folyamatok megalkotása;
    \item  az adminisztratív és jogi folyamatok kezelése;
    \item  a tanárok kiválasztási folyamatának kezelése és folyamatos tanulásuk
          szervezése;
    \item a minőségbiztosítási és fejlesztési rendszer kialakítása és
          működtetése.
  \end{itemize}

  Azaz a fenntartó nemcsak a fenntartásért, hanem a fenntartható fejlődésért is
  felel, és ebben támogatja az egyes mikroiskolákat.

  \section{A Budapest School mikroiskolái}
\fi
A tanulási folyamat működtetéséért a Budapest School egyes mikroiskolái
felelősek. Az iskolákat tanulásszervező-tanárok (a különböző pedagógus
szerepek kibontása a \ref{sec:tanarok}. fejezetben található) egy csoportja
alkotja és vezeti. Így a
mikroiskolák vezetéséért a tanulásszervező-tanárok felelősek.

Az egyes mikroiskolák különböznek egymástól abban, hogy az oda járó gyerekek
pontosan mit és mikor
tanulnak vagy alkotnak. Azonban a következő alapelvek az összes mikroiskolára
érvényesek.

\paragraph{Az iskoláknak \emph{saját fókuszuk van}.}

Van olyan mikroiskola, amely a fejlesztési célok eléréséhez és az egyéni
célok
mentén már 6 éves gyerekek tanulásánál a robotika eszközeit használja,
másutt drámafoglalkozásokkal fejlesztik 12 éves gyerekek a szövegértésüket
és
éntudatukat. A
mikroiskola-rendszerben rejlik annak a lehetősége, hogy egy adott tanulási
környezetben a hangsúlyok úgy váltakozhassanak a csoport és az egyén
érdeklődését követve, hogy közben a tanulási egyensúly fennmaradjon a
tantárgyak fejlesztési területei között. Az iskolák nemcsak abban térnek el
egymástól, hogy kevert korcsoportban, más korosztályú gyerekek, más
érdeklődések mentén, és ily módon más célokat követve tanulnak, hanem
területileg, regionálisan is eltérőek lehetnek.

\paragraph{Az iskolákban a tanulók nagymértékben befolyásolják, hogy mit és
  hogyan tanulnak és alkotnak.}

A tanárok választási lehetőségeket dolgoznak ki, amikből a gyerekek (a
mentoruk és szüleik segítségével) a saját céljaikat, érdeklődésüket
leginkább
támogató \emph{egyéni tanulási tervet} alkotnak. Az iskolákban (a tanárok
által
meghatározott kereteken belül) megfér egymással több, különböző egyéni
céllal
rendelkező gyerek.

Eltérhet, hogy egy-egy gyerek mit tanul, ezért az is, hogy mikor és hogyan
sajátítja el a szükséges ismereteket: egy közösségben megfér a központi
felvételire fókuszáló 11 éves gyerek, és az olyan is, aki ekkor inkább a
Mine\-craft programozásában akar elmélyülni, ezért más képességek
fejlesztésével
lassabban halad. A tanárok feladata és felelőssége, hogy olyan közösségeket
válogassanak össze, amelyek kellően diverzek, és mégis jól működnek, a
gyerekek
igényeit és a kerettanterv céljait egyaránt megfelelően kielégítik.

\paragraph{A mikroiskolák kevert korosztályos közösségek.}

A Budapest Schoolban a gyerekek nemcsak a saját korcsoportjukban, hanem
kevert korosztályok szerinti csoportokban is tanulnak egy-egy mikroiskolában,
hasonlóan a Montessori-féle kevert korosztályos csoportokhoz, vagy az
\emph{önállósági szintek} (independece level  \citep{indepence_level}) alapján
szervezett tanulócsoportokhoz. A csoportok létrehozásakor arra törekszünk, hogy
olyan gyerekek tanuljanak együtt, akik tudják egymást támogatni a tanulásban.

\paragraph{A mikroiskolák együtt fejlődő közösségek.}

Úgy fejlődnek, mintha első lépésben óvodapedagógusok által vezetett óvodai
csoportok
alakulnának át kéttanítós alsós osztályokká. Majd amikor a tanuláshoz újabb
tanárokra van szükségük, akkor bővül a tanárcsapat. Így jöhet létre akár 50
gyerek és 5-8
tanár közössége. Amikor a fejlődésükhöz újabb tanárra van szükségük a
gyerekeknek -- például egy speciális képesség erősítéséhez --, akkor vagy a
Budapest School másik mikroiskolájából jövő tanártól, vagy egy külsős
szakembertől
tanulhatnak. Amikor az érettségire készülve maguk alkotnak gyakorló
csoportot, akkor a tanulásukat akár már önmaguknak is megszervezhetik.

\paragraph{Különféle tanulási struktúrák jöhetnek létre a mikroiskolákon
  belül.}

A közösséget kisebb csoportokra bonthatjuk, ha a tanulásszervezés ezáltal
hatékonyabb. Egyes modulokban egy-egy projektre szerveződnek a gyerekek,
ilyenkor gyakran az eltérő képességű és életkorú gyerekek is megférnek
egymás
mellett. Más moduloknál a csoportokat általában képességszint alapján hozza
létre a tanár. Ilyen lehet a másodfokú egyenletek megoldóképletét megismerő
csoport, az írni tanulók csoportja, vagy egy angol nyelvű újság
szerkesztésére
és megírására alakult modul, ahol a nyelvismeretnek és a szövegalkotási
képességnek már egy olyan szintjén kell lenni, hogy a projektnek jól
mérhető
kimenete lehessen.

\paragraph{A mikroiskolák diverz, integratív közösségek.} A Budapest School
iskolák társadalmi,
kulturális és gazdasági értelemben is egyik fő céljuknak tartják az
integrációt addig,
amíg az a közösség céljait szolgálja.

\paragraph{A Budapest School mikroiskolái tanuló közösségek.} Mindig, minden
módszer,
folyamat fejleszthető, ezért a tanárok feladata, lehetősége, hogy az aktuális
helyzethez illő legalkalmasabb módszert válasszák a tanulás segítéséhez.

A Budapest School mikroiskolák célja, hogy jól átlátható, követhető és
folyamatosan fejlődő folyamattá váljon a tanulás mind a tanuló, mind a tanár,
mind
a szülő részéről. Kerettantervünk folyamatszabályozást nyújt, nem kimeneti
szabályozást. A kimenet a gyerekek és a közösség képességeitől, céljaitól és
érdeklődésétől,
valamint a társadalmi szabályozási környezettől függ.

\ifkerettanterv
  \input{chapters/kerettanterv/tanari_szerepek}
\fi

\section{A közösségi lét szabályai}

BEVEZETO JON IDE A KOZOSSEGROL

\subsection{Konfliktus, feszültségek kezelése}
Tudjuk, hogy a Budapest School szereplői, a gyerekek, tanárok, szülők,
pedagógia program, kerettanterv, fenntartó között kialakul néha
kialakulnak feszültségek és konfliktusok, mert különbözőek vagyunk,
különbözőek az igényeink.

Akkor is feszültség alakul ki, amikor valamilt szeretnénk elérni, de még
nem tartunk ott. A gyerek még nem érte el korosztályában szokásos
évfolyam szint elvárásait, tanár még nem teljes mértékben a pedagógia
programban meghatározott módon segíti a tanulást, a szülő még nem tudja
elég jól kifejezni igényét, vagy a mikroiskola még nem minden esetben
tudja megszervezni a mindennapi testnevelést úgy, ahogy azt szeretnénk.
Amikor eltérés van a vágyaink, víziónk, célunk és a realitás között,
akkor kialaul az úgynevezett kreatív feszültség.

A Budapest School a feszültségekre/konfliktusokra olyan lehetőségként
tekint, amelyek együttműködésen alapuló megoldása építi a kapcsolatot,
és segíti a fejlődést.

A Budapest Schoolban feszültségnek, konfliktusnak hívunk minden vágy,
ötlet, szándék, cél, viselkedés közötti különbséget, ami valamelyik
félben negatív érzéseket kelt. Ebbe beletartozik az is, ha valaki nem
azt és úgy csinálja, ahogy nekünk erre szükségünk van, vagy ha bármilyen
okból nem érezzük magunkat biztonságban, vagy más univerzális emberei
szükségletünk nem elégül ki.

Feszültség alakulhat ki a gyerekek, szülők és tanárok között bármilyen
relációban és adódhatnak egyéb, belső konfliktusok, nehézségek is akár a
család, akár a Budapest School életében, amelyek kihathatnak a közösségi
kapcsolatainkra.

\subsubsection{A BPS konfliktuskezelés során irányadó általános
elvek}\label{a-bps-konfliktuskezeluxe9s-soruxe1n-iruxe1nyaduxf3-uxe1ltaluxe1nos-elvek}

\paragraph{Tárgyalás}\label{tuxe1rgyaluxe1s}

A Budapest School közösségének valamennyi tagja, a tanárok, gyerekek,
szülők, iskolát képviselő fenntartó vállalja:

\begin{itemize}

\item
  A közösség mindennapjaival kapcsolatos konfliktusok esetén elsőként az
  abban érintett személynek jelez közvetlenül.\\
\item
  Személyes kritikát mindig privát csatornán fogalmazza meg először, ha
  kell, akkor segítő bevonásával.
\item
  Vállaljuk, hogy bármelyik fél jelzése esetén lehetőséget biztosítunk
  arra, hogy a vitás kérdést megbeszélhessük közvetlenül, a folyamatban
  részt veszünk.
\item
  Szakítás, kilépés, lezárás előtt legalább három alkalommal megpróbál
  egyeztetni;
\item
  az egyeztetésre elegendő időt hagy, amely legalább 60 nap vagy --
  amennyiben több időre van szükség -- a másik féllel megállapodott idő;
\item
  Teljes figyelemmel, nyitottsággal, a probléma megoldására fókuszálva
  igyekszik feloldani a konfliktust, és közösen megoldást találni a
  problémára.
\end{itemize}

Összefoglalva: ha problémánk van egymással, akkor azt megbeszéljük. Nem
okozunk egymásnak meglepetést, mert vállaljuk, hogy rögtön elmondjuk
egymásnak konfliktusainkat.

\paragraph{Közvetítő bevonása}\label{kuxf6zvetuxedtux151-bevonuxe1sa}

Ha úgy érezzük, hogy a személyes egyeztetés nem vezetett megoldásra, a
tárgyalást külső segítség bevonásával folytatjuk.

\paragraph{Megoldás keresése}\label{megolduxe1s-keresuxe9se}

Ha a közvetlen egyeztetés, tárgyalás és közvetítő bevonása során sikerül
valamilyen megoldást vagy a megoldáshoz vezető folyamatot egyeztetni, a
felek ezt írásban rögzítik és meghatározzák azt az időtartamot is,
amelyre szükség van ahhoz, hogy megtapasztalhassák, érezhessék, hogy van
esély a feszültség, konfliktus megoldására.

\paragraph{Egyeztetés sikertelen}\label{egyeztetuxe9s-sikertelen}

Ha a felek között az egyeztetés sikertelen volt, vagy a megoldási
javaslat nem működött, felek ezt írásban meg kell, hogy állapítsák. Erre
azért van szükség, hogy egyetértés legyen abban közöttük, hogy értik, a
másik fél sikertelennek érzi az egyeztetést.

\paragraph{Lezárás}\label{lezuxe1ruxe1s}

Az egyeztetés sikertelensége esetén elengedjük egymást. De ez a végső
megoldás.

\subsubsection{Gyerek-gyerek
konfliktus}\label{gyerek-gyerek-konfliktus}

TODO Csoti ORsi?

\subsubsection{Gyerek-iskola, tanár-szülő
konfliktus}\label{gyerek-iskola-tanuxe1r-szuxfclux151-konfliktus}

Minden gyereknek van egy mentora. A szülők számára a mentor az
elsődleges kapocs az iskola felé. Ezért, ha a szülőben jelenik meg egy
feszültség, akkor elsődlegeses a mentornak jelez. Ugyanígy, a mentor
közvetíti a család felé a gyerekkel kapcsolat feszültségeket.

Ha egy gyerek, vagy szülő úgy érzi, hogy egy gyereknek nem jó az iskolai
élménye, például nem tanul eleget, vagy kiközösítik, vagy csak nem
szeret bemenni, akkor feladata, hogy rögtön beszéljen a mentor tanárral.

Előfordulhat, hogy a tanárok vagy az iskola úgy érzik, hogy egy
gyereknek nem tesz jót a Budapest School közössége, vagy a hozzáállása
súlyosan zavarja vagy sérti a Budapest School közösséget vagy azok
tagjait. Az is lehet, hogy a tanárok vagy a Budapest School a szülővel
való kapcsolatot érzik konfliktus vagy feszültség forrásának. Ilyen
esetben ugyanígy le kell folytatni a konfliktuskezelés folyamatát és
megpróbálni feloldani a feszültséget. Ennek sikertelensége esetén az
iskola jelzi a családnak írásban, hogy el szeretne válni.

\subsubsection{Tanár-tanár
konfliktus}\label{tanuxe1r-tanuxe1r-konfliktus}

TODO

\subsubsection{Pedagógia program, kerettanterv be nem tartásával
kapcsolatos
konfliktusok}\label{pedaguxf3gia-program-kerettanterv-be-nem-tartuxe1suxe1val-kapcsolatos-konfliktusok}

Amikor egy tanár, egy gyerek vagy egy mikroiskola nem tartja be a
kerettantervet, a pedagógia programot, a házirendet, vagy egyéb közösen
megalkotott szabályokat, akkor konfliktus alakul ki közte és az iskola
között. Ilyenkor szintén a konfliktuskezelés folyamatát kell
lefolytatni.

\subsubsection{Eszkalálás,
mediálás}\label{eszkaluxe1luxe1s-mediuxe1luxe1s}

Ha nem sikerül a személyes egyeztetés vagy megbeszélés a tanárokkal vagy
az érintett szülővel, vagy már nem is hiszel benne, hogy a feszültséget
így fel tudjátok oldani, akkor mindenképpen küldj egy jelzést TODO. Innen ő fog közvetíteni (mediálni) köztetek.

\subsection{A tanulók és az iskola
szabályai}\label{a-tanuluxf3k-uxe9s-az-iskola-szabuxe1lyai}

A tanulóknak az intézményi döntési folyamatban való részvételi jogának
gyakorlásának rendje

\subsection{Partneri kapcsolatok}\label{partneri-kapcsolatok}

A szülő, a tanuló, a pedagógus és az intézmény partnerei
kapcsolattartásának formái

\begin{quote}
intézmény partnerei kapcsolattartásának formái (csak a nevelési-oktatási
tartalmak relevanciája vonatkozásában, mert egyébként
SZMSZ-kompetencia);
\end{quote}


\ifkerettanterv
  \section{A Budapest School tanárai: a tanulásszervezők, a mentorok és a
    modulvezetők.}
\else
  \section{Különböző tanári szerepek: a tanulásszervező, a mentor és a
    modulvezető}

\fi

\label{sec:tanarok}
\begin{quote}

  Tanulni bárkitől lehet, aki tud olyasmit mutatni, ami felkelti a tanuló
  érdeklődését, és elő tudja segíteni a fejlődését. Tanítani az tud igazán,
  aki tanulni is tud.
\end{quote}
A Budapest Schoolban a gyerekek azokat a felnőtteket tekintik tanáruknak, akik
minőségi időt töltenek velük, és segítik, támogatják vagy vezetik őket a
tanulásukban. Több szerepre bontjuk a tanár fogalmát: a gyerek egy (és csak
egy) felnőtthöz különösen kapcsolódik, a \emph{mentortanárához}, aki rá
különösen figyel. Ezenkívül a gyerek tudja, hogy a mikroiskola mindennapjait
egy tanárcsapat, a \emph{tanulásszervezők} határozzák meg, azaz ők vezetik az
iskolát.  A foglalkozásokon megjelenhetnek más tanárok, a \emph{modulvezetők},
akik egy adott foglalkozást, szakkört, órát tartanak. Néha megjelennek más
felnőttek, akik párban vannak egy másik tanárral: ők az \emph{asszisztensek}, a
\emph{gyakornokok} vagy az \emph{önkéntes segítők}.

Szervezetileg minden mikroiskolának van egy állandó \emph{tanárcsapata}, a
tanulásszervezők. Állandó, mert legalább egy tanévre elköteleződnek, szemben a
modulvezetőkkel, akik lehet, hogy csak egy pár hetes projektre vesznek részt a
munkában.

A tanulásszervezők általában mentorok is, de nem minden esetben. Nem lehet
mentor az, aki a gyerek mikroiskolájában nem tanulásszervező, mert nem lenne
rálátása a mikroiskola történéseire. Egy tanulásszervező lehet több
mikroiskolában is ebben a szerepben, és így mentor is lehet több mikroiskolában.

\paragraph{Mentor}
Minden tanulónak van egy \emph{mentora}, aki az egyéni céljainak
megfogalmazásában és
a fejlődése követésében segíti. Minden mentorhoz több tanuló tartozik, de nem
több mint 12. A mentor együtt dolgozik a Budapest School tanárcsapatával, a
szülőkkel és az általa mentorált gyerekekkel. A mentor segít az általa
mentorált gyereknek, hogy a tantárgyi fejlesztési célok és
a
saját magának megfogalmazott egyéni célok között megtalálja  az egyensúlyt, és segít megalkotni a
gyerek \emph{saját
  tanulási tervét}.

A mentor a kapocs a Budapest School, a szülő és a gyerek között.

\begin{itemize}
  \item Képviseli a Budapest Schoolt, a mikroiskola közösségét.
        \begin{itemize}
          \item Ismeri a Budapest Schoolt, a lehetőségeket, a tanulásszervezés
                folyamatait.
          \item Együtt tanul más Budapest School mentorokkal, együtt dolgozik a
                tanártársaival.
        \end{itemize}

  \item Ismeri, segíti, képviseli a gyereket.
        \begin{itemize}
          \item  Tudja, hol és merre tart mentoráltja, ismeri a képességeit,
                körülményeit, szándékait, vágyait.
          \item    Segít az egyéni célok elérésében, felügyeli a haladást.
          \item    Megerősíti mentoráltjai pszichológiai biztonságérzetét.
          \item   Visszajelzéseket ad a mentoráltjainak.
          \item    Segít abban, hogy az elért célok a portfólióba kerüljenek.
          \item    Összeveti a portfólió tartalmát a tantárgyak fejlesztési
                céljaival.
        \end{itemize}

  \item Együtt dolgozik, gondolkozik a szülőkkel, képviseli igényüket a
        közösség felé.
        \begin{itemize}
          \item Erős partneri kapcsolatot épít ki a szülőkkel, információt oszt meg
                velük.
          \item Segít a gyerekekkel közös célokat állítani.
          \item Szülő számára a mentor az elsődleges kapcsolattartó a különféle
                iskolai ügyekkel kapcsolatban.
        \end{itemize}

\end{itemize}

A mentor egyszerre felelős a mentorált tanuló előrehaladásának segítéséért,
és
közös felelőssége van a mentortársakkal, hogy az iskolában a lehető legtöbbet
tanuljanak a gyerekek. A mentor folyamatosan figyelemmel követi az egyéni
tanulási tervben megfogalmazottakat, és ezzel kapcsolatos visszajelzést ad a
mentoráltnak és a szülőnek.

\paragraph{Tanulásszervező}
Csoportban dolgozó, iskolaszervező, strukturáló tanár. Egy mikroiskola
állandó tanári
csapatát 2-7 tanulásszervező alkotja, akik egyedileg meghatározott szerepek
mentén a mikroiskola mindennapjainak működtetéséért felelnek. Minden mentor
tanulásszervező is. A tanulásszervezők tarthatnak
modulokat, sőt, kívánatos is, hogy dolgozzanak a gyerekekkel, ne csak
szervezzék az életüket.
Ők rendelik meg a külső modulvezetőktől a munkát, ilyen értelemben a
tanulási utak projektmenedzserei.

\paragraph{Modulvezetők}

Bárki lehet modulvezető, aki képes akár egy egyetlen alkalommal történő, vagy
éppen
egy egész trimeszteren át tartó tanulási, alkotási folyamatot vezetni. Ők
általában
az adott tudományos, művészeti, nyelvi vagy bármilyen más terület szakértői.

Modulokat a tanulásszervezők is vezethetik, de külsős, egyedi megbízással
dolgozó szakemberek is megjelennek modulvezetőként. Modulvezető lehet bárki,
akiről az őt megbízó tanárcsapat tudja, hogy képes gyerekek folyamatos
fejlődését és egy tanulási cél felé való haladását segíteni. A moduláris
tanmenettel \aref{sec:modularis_tanmenet}. fejezet foglalkozik.

\section{A pedagógusok feladatai}\label{a-pedaguxf3gusok-feladatai}

\begin{verbatim}
a pedagógusok helyi intézményi feladatai, az osztályfőnöki munka tartalmát, az osztályfőnök feladatai
\end{verbatim}

A Budapest School kerettanterv háromféle tanárt, a tanulásszervező
tanárt, a mentorttanárt és a modulvezetőtanárt különböztet meg.

VAN BENNÜK Közös és mindent performanca review szerint adunk meg



\paragraph{Érzelmi intelligencia (EQ)
ninja}\label{uxe9rzelmi-intelligencia-eq-ninja}

\begin{itemize}

\item
  Gyerekek pszichológiai biztonságérzetét megerősíti.
\item
  Olyan visszajelzéseket ad, amelyek az erőfeszítésekre, a belefektetett
  energiára, munkára és a jövőbeni fejlődésre fókuszálnak (growth
  mindset), pozitív megerősítést alkalmaz, épít a gyerek erősségeire, és
  egyértelműen megfogalmazza, mit tehetne másként.
\item
  Értő figyelemmel van jelen, kedves, nyitott és figyel a gyerekekre.
\end{itemize}

\paragraph{SNI szakértő}\label{sni-szakuxe9rtux151}

\begin{itemize}

\item
  Jól tudja kezelni az egyéni bánásmódot, speciális nevelési igényű
  gyerekeket a csoportban.
\item
  Akinek külső segítségre van szüksége, azoknak a szüleivel ezt
  proaktívan leegyezteti és menedzseli a folyamatot.
\end{itemize}

\paragraph{Change agent}\label{change-agent}

\begin{itemize}

\item
  Nehéz helyzeteken is könnyen továbblendül, vannak módszerei arra
  hogyan töltse magát és ezeket használja is.
\item
  Azt keresi, mire lehet hatása, mit lehet eggyel jobban csinálni és
  ebben akciókat tesz.
\item
  Megünnepli az előrelépéseket, közös sikereket.
\end{itemize}

\subsection{Tanulásszervező}\label{tanuluxe1sszervezux151}

\paragraph{Jó csapattag}\label{juxf3-csapattag}

\begin{itemize}

\item
  Rendszeresen jelen van a tanári csapat megbeszélésein.
\item
  Elérhető telefonon vagy online a csapattal megállapodott kereteken
  belül.
\item
  Feladatokat vállal magára és azokat megbízhatóan, határidőre
  végrehajtja.
\item
  Kooperatív és támogató a közös munkák, ötletelések, megbeszélések
  alatt, képviseli a saját nézőpontját, gondolatait érzéseit, miközben a
  csapat és a többiek igényeire is figyel.
\item
  Kifejezi támogatását, ellenérveit és javaslatait a jobb megoldás
  érdekében.
\item
  A visszajelzést keresi, a kritikát jól fogadja, és megfontolja,
  átgondolja a lehetséges változtatásokat.
\item
  Kollégák fejlődését segíti rendszeres visszajelzésekkel.
\end{itemize}

\paragraph{BPS tag}\label{bps-tag}

\begin{itemize}

\item
  Rendszeresen jelen van heti hétfőkön, havi szombatokon.
\item
  Részt vesz a mikroiskolájának és a Budapest Schoolnak az építésében.
\item
  Proaktívan alakít ki rendszereket, folyamatokat, és a legjobb
  gyakorlatokat megosztja BPS szinten.
\item
  Közös témákban aktív, hozzászól, alakítja a véleményével és tudásával
  a BPS rendszerét.
\end{itemize}

\subsection{Mentor}\label{mentor}

\begin{itemize}
\item
  Az első trimeszter alatt a mentor megismeri mentoráltját,
  személyiségjegyeit, képességeit, érdeklődését, motivációit. Megismeri
  a családot.
\item
  Megállapodik a családdal a kapcsolattartás szabályaiban.
\item
  Bevezeti a családot a Budapest School rendszerébe.
\item
  Trimeszterenként a mentorált gyerekkel és szülőkkel egyetértésben
  kialakítja a mentorált gyerek tanulási céljait.
\item
  Képviseli mentoráltját a többi tanár felé.
\item
  Szülőkkel rendszeresen információt oszt meg, elérhető, asszertíven
  kommunikál, tiszta, mérhető megállapodásokat köt.
\item
  Szülőkkel erős partneri kapcsolatot épít.
\item
  Amikor a gyereknek külső fejlesztésre van szüksége, akkor a családot
  segíti a megfelelő fejlesztő felkeresésében, a külső fejlesztővel
  kapcsolatot tart és konzultál a mentoráltja haladásáról.
\item
  Maximum kéthetente talalkozik mentoráltjával. Követi, tudja, hogy a
  gyerek hogy van iskolában, családban, életben.
\item
  Mentorként nyomon követi, monitorozza mentoráltjai fejlődését és
  szükség esetén továbblendíti, inspirálja őket.
\item
  Mentor biztosítja, hogy a mentorát gyerek portfólió friss legyen.
\item
  Rendszeresen reflektál a mentorált gyerekkel együtt annak tanulási
  céljaira, és haladásáa.
\item
  Segíti a mentorált modulválasztását.
\item
  Visszajelzés gyűjt és ad a mentorált gyerek fejlődésére.
\end{itemize}

\subsection{Modulvezetők}\label{modulvezetux151k}

\begin{itemize}

\item
  Izgalmas, érdekes foglalkozásokat tart, amire felkészül és amiben a
  gyerekeket flowban tudja tartani.
\item
  Amikor a gyerekek vele vannak, akkor figyelnek, fókuszálnak,
  koncentrálnak, dolgoznak, tanulnak.
\item
  Kedvesen és határozottan vezeti a csoportot, figyel arra, hogy
  mindenkit bevonjon.
\end{itemize}

\subsection{A kiemelt figyelmet igénylő
gyerekek}\label{a-kiemelt-figyelmet-iguxe9nylux151-gyerekek}

\begin{verbatim}
A kiemelt figyelmet igénylő tanulókkal kapcsolatos pedagógiai tevékenység helyi rendje

> (ez a korábbi szabályozás terminológiájával élve a következő tematikai egységeket foglalja magába: a beilleszkedési, magatartási nehézségekkel összefüggő; a tehetség, képesség kibontakoztatását és a szociális hátrányok enyhítését segítő tevékenységet és a tanulási kudarcnak kitett tanulók felzárkóztatását segítő programokat, így tehát a sajátos nevelési igényű és a hátrányos helyzetű tanulók integrációjának sajátos programelemei is idetartoznak pl. Officina Bona, IPR);
\end{verbatim}

TODO: max 20\% TODO: mentor tudja, h mi a szitu

A tanulásszervezők, modulvezetők feladata:

\begin{itemize}
\item
  Megfelelő tanulásszervezési formákkal és módokkal biztosítani, hogy a
  tanórákon és a tanórán kívüli tevékenységben érvényesüljön a
  differenciált, az egyéniesített fejlesztés, eltérő képességekhez,
  viselkedéshez való alkalmazkodás.
\item
  Olyan tanulási környezetet, speciális módszerek, tapasztalatszerzési
  lehetőség biztosítása, amelyben sokoldalú szemléltetéssel,
  cselekvéssel, gazdag feladattárral, speciális eszközök alkalmazásával
  valósul meg készség- és képességfejlesztés.
\item
  A pedagógus a tanórai tevékenységek/foglalkozások tervezésébe építse
  be a pedagógiai diagnózisban szereplő javaslatokat.
\item
  A pedagógus a tananyag adaptálásánál, feldolgozásánál vegye figyelembe
  az egyes tanulók fejlettségi szintjét, a támogatás szükséges mértékét.
\item
  Az egyéni haladási ütem biztosítására egyéni fejlesztési és tanulási
  terv készítése, individuális módszerek, technikák alkalmazása.
\item
  A pedagógus működjön együtt a gyermek/tanuló fejlesztésében résztvevő
  szakemberekkel.
\end{itemize}

\section{Saját tanulási célok}
\label{sec:tanulasi_celok}

Minden gyerek megfogalmazza és háromhavonta újrafogalmazza a \emph{saját
      tanulási céljait}: eredményeket, amelyeket el akar érni, képességeket,
amelyeket fejleszteni akar, szokásokat, amelyeket ki akar alakítani. A saját
célok elfogadásakor a gyerek és a mentora a szülőkkel együtt \emph{tanulási
      szerződést} köt.

Csak olyan célok kerülhetnek a saját célok közé, amelyek
minden érintettnek biztonságosak, és amelyek összhangban vannak a tantárgyi
fejlesztési célokkal és tanulási eredményekkel. A szerződésben rögzíthetőek
tanulási eredményekre
vonatkozó megállapodások,
tantárgyi évfolyamszintekre vonatkozó elvárások (pl. ,,\emph{haladjon egy
      évfolyamszintet egy év alatt}'' vagy ,,\emph{készüljön fel emelt szintű
      érettségire}''), és a tantárgyi rendszeren kívüli célok és
feladatok.

Fontos megkötés, hogy a saját tanulási célok legalább a felének
\ifkerettanterv
      \aref{sec:tantargyi_tanulasi_eredmenyek}. fejezetben
\else
      a kerettanterv Tantárgyi tanulási eredmények fejezetében
\fi
felsorolt tanulási eredmények elérésére kell vonatkoznia. A másik
fele szabadon alakítható.

Háromhavonta a tanulásszervezők és a gyerekek megállnak, reflektálnak az elmúlt
időszakra, és a tapasztalatok, valamint az elért célok ismeretében és az új
célok figyelembevételével újratervezik, újraszervezik a foglalkozások rendjét,
tehát azt, hogy mikor és mit csinálnak majd a gyerekek az iskolában.
A mindennapi tevékenység során tapasztalt élmények, alkotások, elvégzett
feladatok, kitöltött vizsgák, tehát mindaz, ami a gyerekekkel történik, bekerül
a portfóliójukba. Még az is, amit nem terveztek meg előre.

A gyerekeket a mentoruk segíti a saját célok kitűzésében, a különböző
választásoknál, a portfólióépítésben, a reflektálásban. A tanulási célok
kitűzése az önirányított tanulás fokozatos fejlődésével és az életkor
előrehaladtával folyamatosan egyre önállóbb tevékenységgé válik. Tanulási
útján, céljai kitűzésében a mentor kíséri végig a gyerekeket.

A Budapest School személyre szabott tanulásszervezésének jellegzetessége, hogy
a gyerekek a saját céljuk irányába haladnak, az adott célhoz az adott
kontextusban leghatékonyabb úton. Tehát mindenki rendelkezik saját célokkal,
még akkor is, ha egy közösség tagjainak céljai a tantárgyi tanulási eredmények
azonossága, vagy a hasonló érdeklődés miatt akár  80\% átfedést mutatnak.

A NAT műveltségi területeiben megfogalmazott követelmények teljesítése is célja
a tanulásnak, a tanulás fő irányítója azonban más. Mi azt kérdezzük a
gyerekektől, hogy \emph{ezenfelül} mi az ő személyes céljuk.

\section{A tanulási szerződés}

A tanulási szerződés az előbbiekben említett gyerek-mentor-szülő közötti
megállapodás, ami rögzíti
\begin{enumerate}
      \item a gyerek, a mentor (iskola) és a szülő igényeit, elvárásait;

            ezek lehetnek: \emph{,,szeretném, ha a gyerekem naponta olvasna''}
            típusú
            folyamatra vonatkozó kérések, vagy erősebb \emph{,,változtatnod
                  kell a
                  viselkedéseden, ha a közösségben akarsz maradni''} igények,
            határok
            megfogalmazása;

      \item a gyerek céljait a következő trimeszterre, vagy a tanév végéig;

      \item a gyerek, mentorok (iskola) és szülő vállalásait, amivel támogatják
            a
            cél
            elérését és a felek igényének elérését.

\end{enumerate}

A tanulási szerződésre jellemző, hogy
\begin{itemize}
      \item A kitűzött célokat minél specifikusabban, mérhetőbben kell
            megfogalmazni.
            Javasolt az OKR  (Objectives and Key Results, azaz	Cél és Kulcs
            Eredmények)
            \citep{okr} vagy a SMART (Specific, Measurable, Achievable,
            Relevant,
            Time-bound, azaz Specifikus,  Mérhető, Elérhető, Releváns és Időhöz
            kötött)
            \citep{wiki:smart} technika alkalmazása, hogy minél specifikusabb,
            teljesíthetőbb, tervezhetőbb és könnyen mérhető célokat tűzzenek
            ki.

      \item A kitűzött célokban való megállapodást követően, megállapodást
            kell
            kötni arról is, hogy ki és mit tesz azért, hogy a gyerek a célokat
            elérje.

      \item A mentor a teljes mikroiskolát (a többi tanárt, a közösséget)
            képviseli
            a
            megállapodás során.
\end{itemize}

A tanulási szerződést néha hívjuk \emph{megállapodásnak} is. A megállapodás és
szerződés szavakat ez a kerettanterv szinonimának tekinti. A \emph{learn\-ing
      con\-tract} az önirányított tanulást hangsúlyozó felnőttképzéssel
foglakozó
irodalomban
bevett szakkifejezés már a 80-as évektől \citep{Malcolm77}. Ennek a magyar
nyelvben inkább a szerződés felel meg. Egy másik szakterületen, a
pszichoterápiás munkában a terápiás szerződések megkötésekor a közös munka
kereteinek kialakítását és fenntarthatóságát hangsúlyozzák
\citep{pszichoterapia}. Erre is utalunk a tanulási szerződés elnevezéssel. Van,
amikor a \emph{hármas szerződés} kifejezést használjuk, hangsúlyozva, hogy mind
a három szereplőnek elfogadhatónak kell tartania a szerződés tartalmát.

\section{Visszajelzés, értékelés}
\label{sec:ertekeles}
Ahhoz, hogy hatékony legyen a tanulás, fejlődés, fontos, hogy a gyerekek,
tanárok és szülők is tudják, hogy
\begin{enumerate}
      \item hol tart most egy gyerek, mit tud most,
      \item hova akar vagy kell eljutni, azaz, mi a célja,
      \item mi kell ahhoz, hogy elérje a célját.
\end{enumerate}
Ezek mellett mindenkinek hinnie kell abban, hogy odafigyeléssel, gyakorlással a
gyerek meg tud tanulni egy konkrét dolgot. Fontos, hogy magas legyen a gyerekek
énhatékonysága,  erős legyen az önbizalmuk, és nem szabad félniük a hibázástól,
a nem-tudástól,
mert a tanulás első lépése, hogy elfogadjuk, hogy valamit nem tudunk. Azaz
fontos, hogy fejlődésfókuszú gondolkodásuk (growth mindset)
\citep{growthmindset} legyen, azaz
\begin{enumerate}
      \setcounter{enumi}{3}
      \item hinniük kell, hogy el tudják érni a céljukat.
\end{enumerate}

Egy visszajelzés, értékelés akkor jó és hasznos, azaz hatékony, ha ebben a négy
dologban segít. Mai tudásunk szerint ehhez:
\begin{itemize}
      \item Rendszeresen visszajelzést kell kapnunk és adnunk.
      \item A tanulási céloknak és visszajelzéseknek minél specifikusabbaknak
            kell
            lenniük (azaz például ne a 8. oszályos \emph{matematikatudást}
            értékeljük,
            hanem hogy mennyire képes valaki \emph{fagráfokat
                  használni
                  feladatmegoldások során}\footnote{Ez a konkrét példa a STEM
                  tantárgy
                  egyik
                  tanulási eredménye.}).
      \item A \emph{,,hol tartok most''} diagnózisnak mindig cselekvésre,
            viselkedésre, aktív tevékenységre kell vonatkoznia. Ne az legyen a
            visszajelzés, hogy \emph{,,ügyes vagy egyenletekből''}, hanem
            \emph{,,gyorsan és
                  pontosan oldottad meg a 4 egyenletet''}. A legjobb, amikor a
            visszajelzés
            konkrét megfigyelésen alapul, és tudni, hogy mikor, hol történt az
            eset:
            \emph{,,amikor társaiddal Minecraftban házat építettél, akkor
                  pontosan
                  kiszámoltad a ház területét''.}
      \item Ha a cél nem a mások legyőzése, akkor a visszajelzés se
            tartalmazzon
            olyan állítást, ami másokhoz hasonlít (így kerüljük a
            \emph{tehetség}
            szót is,
            aminek bevett definíciója szerint az átlagnál jobb képesség). A
            másokhoz való
            szint felmérése akkor (és csak akkor) fontos, amikor a cél egy
            versenyszituációban jó eredményt elérni.

      \item A gyerek legyen részese a visszajelzésnek. Értse, tudja, hogy miért
            kapta
            azt a visszajelzést, a legjobb, ha -- amikor ezt a képességei
            engedik
            -- önmaga
            képes elvégezni a visszajelzést, vagy annak egy részét.
      \item A visszajelzésnek transzparensen hatással kell lennie a
            tanulásszervezésre. Legyen része a folyamatnak, és a gyerek, tanár
            és a
            szülő
            is értse, hogy a visszajelzés alapján mit és hogyan csinálunk
            másképp.
\end{itemize}

\paragraph{Többszintű visszajelzés} A Budapest School iskolákban a gyerekek
többféle visszajelzést kapnak. \begin{enumerate}
      \item Minden modul elvégzése után a modul céljai, témája, fókusza alapján
            a
            modulvezetők visszajelzést adnak a tanulásról, eredményekről,
            viselkedésről.
      \item Trimeszterenként a mentorok visszajelzést adnak arról, hogy a
            gyerek
            általában hogyan haladt a tanulási célok felé.
      \item Ennek része, hogy a tantárgyi tanulási eredmények alapján hogyan
            haladt a
            gyerek a tantárgyak évfolyamszinthez tartozó követelmények
            teljesítésében. Az
            évfolyamok, mint elérhető szintek Budapest School értelmezését
            \aref{sec:evfolyamok}. fejezet tárgyalja.
      \item A mentorok irányításával a gyerekek visszajelzést kapnak arról,
            hogyan
            működnek a közösségben.
\end{enumerate}

\paragraph{Érdemjegyek, osztályzatok helyett értékelő táblázatok} A Budapest
School visszajelzéseinek sokkal részletesebbeknek kell lenniük, mint azt a
tantárgyi érdemjegyek és osztályzatok lehetővé teszik, ezért azok helyett a
kerettanterv
értékelő táblázatokat (angolul rubric) alkalmaz. Az értékelő táblázatban
szerepelnek az értékelés szempontjai és szempontonkénti szintek, rövid
leírásokkal.
Ezek alapján a gyerekek maguk is láthatják, hogy hol tartanak, hogyan
javíthatnak még a munkájukon. A táblázatok formája minden visszajelzés esetén
(értsd modulonként, célonként)
változtatható.

\section{Portfólió}
\label{sec:portfolio}
A modulok eredményeiből, a produktumokból és visszajelzésekből a gyerek és a
mentor portfóliót
állít össze, hogy a tanulás mintázatait észlelhesse, és a
tanárok tudatosabban tudják
a gyereket segíteni a céljai kitalálásában és elérésében. A portfólió a gyerek
céljainak nyomon követését szolgálja, és egyúttal a szülők felé történő
visszajelzés eszköze
is. Minden gyerek portfóliója folyamatosan épül: az tartalmazza az általa
elvégzett feladatokat, projekteket vagy azok dokumentációját, alkotásait,
eredményeit, az esetleges vizsgák eredményeit és a társaitól, tanáraitól kapott
visszajelzéseket. A \emph{portfólió célja}, hogy minden információ meglegyen
ahhoz,
hogy

\begin{itemize}
      \item a gyerek és mentora fel tudja mérni, hogy sikerült-e a kitűzött
            célokat
            elérni, illetve mire van szüksége még a gyereknek új célok
            eléréséhez;

      \item a szülő folyamatosan rálásson a gyereke tanulási útjára;

      \item megítélhető legyen, hogy a tantárgyi követelményekhez
            képest
            hol
            tart a gyerek;

      \item a gyerek a portfólió megtekintésével visszaemlékezhessen a
            tanultakra,
            ismételhessen, tudása elmélyülhessen;

      \item eredményei alapján bizonyítványt lehessen kiállítani.

\end{itemize}

A portfólió folyamatosan frissül, a mindennapi, formális, non-formális és
informális tanulási helyzetek
bármikor adhatnak okot a portfólió frissítésére. Az iskola életében kiemelt
szerepe van a következő eseményeknek.

\begin{enumerate}
      \item Minden \emph{modul végeztével} a portfólióba kerül:

            \begin{enumerate}

                  \item  A képesség elsajátításának, tanulási eredmény
                        elérésének a ténye.
                        Nincs
                        félig elsajátított képesség, tehát már értékelni nem
                        kell. Ha a modul során a gyerek megtanult százas
                        számkörben alapműveleteket
                        végezni, 
                        akkor
                        annyi kerül be a portfólióba, hogy ,,\emph{Szóban és
                              írásban
                              összead, kivon, szoroz és oszt a százas
                              számkörben.}''. Amennyiben a
                        készséget a
                        gyerek és a
                        tanár megítélése alapján nem sikerült megfelelően
                        elsajátítani,
                        úgy a gyakorlás
                        ténye kerül be a portfólióba.
                  \item Az alkotás vagy a projektmunka eredménye, ha a modul
                        célja egy
                        alkotás
                        létrehozása volt.
                  \item A részvétel ténye, ha a jelenlét volt a modul célja
                        (például
                        kirándulás
                        az Országos Kéktúra útvonalán).

            \end{enumerate}
      \item Az elvégzett vizsgák, tudáspróbák, képességfelmérők, diagnózisok
            eredményeit érdemes rögzíteni.

      \item A \emph{kipakolás} célja, hogy a gyerekek a tanároknak, szülőknek
            és
            más érintetteknek bemutassák elvégzett
            munkájukat, azaz
            a portfólióváltozásukat. A kipakolásra való felkészülés
            tulajdonképpen
            a
            portfólió összeállítása, prezentálásra való felkészítése, a
            \emph{portfólió
                  frissítése}.

      \item Társas visszajelzés eredményeként minden gyerek kap visszajelzést a
            társaitól. Ilyenkor összegyűjtik, mit tett a gyerek, ami a többiek
            elismerését
            és háláját kivívta. Ez is releváns adatokkal szolgálhat a
            portfólióhoz.

      \item A gyerek saját értékelése, reflexiója arról, hogyan értékeli, amit
            elért, fontos eleme a portfóliónak.

      \item A tanárok adhatnak kompetenciatanúsítványokat. Ezek
            rövid,
            specifikus visszajelzések, amelyek mutatják, ha valamit a gyerek
            megcsinált,
            valamiben fejlődött.
\end{enumerate}

A mentorok segítenek a gyerekeknek a tanulás módját, folyamatát és eredményeit
bemutatni
portfólióban.

\paragraph{Formai követelmények}
A portfóliónak rendezettnek, hozzáférhetőnek,
elérhetőnek, visszakereshetőnek és könnyen bővíthetőnek kell lennie. Olyan
(technológiai)
megoldást kell a mikroiskoláknak választaniuk, ami alapján
a gyerek, tanár és a szülő \emph{naponta} tudja a portfóliót bővíteni, és akár
\emph{heti rendszerességgel} át tudják tekinteni időrendben, modulonként vagy
tantárgyanként a portfólió bővülését.

A portfólió formátumára nincs egységes megkötés. Minden mikroiskola maga
alakítja ki a gyerekek, tanárok és szülők számára legjobban működő rendszert.
Évfolyamszintlépéshez és osztályzatokra váltáshoz az iskola csak digitális
formában tárolt és a kijelölt tanárok számára online elérhetővé tett portfóliót
fogad el.
\section{A vizsgák rendje}\label{a-vizsguxe1k-rendje}

A tanulmányok alatti vizsgák és az alkalmassági vizsga szabályai; az
iskolai írásbeli, szóbeli, gyakorlati beszámoltatások, az ismeretek
számonkérésének rendje

\begin{quote}
valamint középfokú iskola esetében a szóbeli felvételi vizsga
követelményeit (javító, osztályozó, különbözeti vizsgák
vizsgaszabályzata);
\end{quote}

\section{A felvétel és az
átvétel}\label{a-felvuxe9tel-uxe9s-az-uxe1tvuxe9tel}

a felvétel és az átvétel - Nkt. keretei közötti - helyi szabályai

\chapter{A gyerekek fejlesztése}
\section{A
személyiségfejlesztés}
\begin{verbatim}
  Trv. elvarja: A személyiségfejlesztéssel kapcsolatos
  pedagógiai feladatok
  Itt irjuk le a 4x3 szintet
\end{verbatim}


\subsection{A teljeskörű
egészségfejlesztés}\label{a-teljeskuxf6rux171-eguxe9szsuxe9gfejlesztuxe9s}

\subsubsection{Definíció, cél}\label{definuxedciuxf3-cuxe9l}

Az egészségfejlesztés a WHO meghatározása szerint az a folyamat, ami
képessé teszi az embereket arra, hogy saját egészségüket felügyeljék és
javítsák. Az egészségnevelés pedig változatos kommunikációs formákat
használó, tudatosan létrehozott tanulási lehetőségek összessége, amely
az egészséggel kapcsolatos tudást, ismereteket és életkészségeket bővíti
az egyén és a környezetében élők egészségének előmozdítása érdekében.
Budapest School Általános Iskola és Gimnázium (a továbbiakban e
fejezetben BPS) ezen definíciót teszi magáévá a teljes körű
egészségfejlesztési program összeállításakor és alkalmazásával.

A teljes körű egészségfejlesztési program a BPS közösség életminőségének
javítását szolgáló, a közösséghez tartozók közös akaratát összegző
cselekvési program, melynek közvetlen és közvetett célja az életminőség,
ezen keresztül az egészségi állapot javítása, olyan közösségi
problémakezelési módszer, amely az érintettek aktív részvételére épít.

A teljes körű egészségfejlesztés célja, hogy a BPS-ben eltöltött időben
minden gyerek részesüljön a teljes testi-lelki jóllétét, egészségét
megőrző és hatékonyan fejlesztő, a BPS mindennapjaiban rendszerszerűen
működő egészségfejlesztő tevékenységekben.

\subsubsection{Négy alapfeladat}\label{nuxe9gy-alapfeladat}

A teljes körű iskolai egészségfejlesztés az alábbi négy
egészségfejlesztési alapfeladat rendszeres végzését jelenti - minden
gyerekkel, a tanárok és a szülők, valamint a BPS partneri kapcsolati
hálóban szereplők bevonásával:

\begin{itemize}

\item
  egészséges táplálkozás megvalósítása (elsősorban megfelelő, magas
  minőségű, lehetőleg helyi alapanyagokat használó kiszállítóval való
  megállapodással; helyben főzés esetén az alapanyagok kiválasztásánál
  legyen elsődleges szempont)
\item
  mindennapi testmozgás minden gyereknek (változatos foglalkozásokkal,
  koncentráltan az egészség-javító elemekre, módszerekre, pl.
  tartásjavító torna, tánc, jóga)
\item
  a gyermekek érett személyiséggé válásának elősegítése személyközpontú
  pedagógiai módszerekkel és a művészetek személyiségfejlesztő
  hatékonyságú alkalmazásával (ének, tánc, rajz, mesemondás, népi
  játékok, stb.)
\item
  környezeti, médiatudatossági, fogyasztóvédelmi, balesetvédelmi
  egészségfejlesztési modulok, modulrészletek hatékony (azaz ``bensővé
  váló'') oktatása.
\end{itemize}

\subsubsection{Az egészségfejlesztési ismeretek
témakörei}\label{az-eguxe9szsuxe9gfejlesztuxe9si-ismeretek-tuxe9makuxf6rei}

\begin{itemize}

\item
  Az egészség fogalma
\item
  Az egyén és az őt körülvevő közösség egészsége: felelősségünk
\item
  A környezet egészsége
\item
  Az egészséget befolyásoló tényezők
\item
  A jó egészségi állapot megőrzése
\item
  A betegség fogalma
\item
  Megelőzés
\item
  A táplálkozás és az egészség, betegség kapcsolata
\item
  A testmozgás és az egészség, betegség kapcsolata
\item
  Balesetek, baleset-megelőzés
\item
  A lelki egészség.
\item
  Önismeret, önértékelés, a másikat tiszteletben tartó kommunikáció
  módjai, ennek szerepe a másik önértékelésének segítésében
\item
  A két agyfélteke harmonikus fejlődése
\item
  Az érett, autonóm személyiség jellemzői
\item
  A társas kapcsolatok
\item
  A társadalom élete, a társadalmi együttélés normái
\item
  A gyermek fejlődését elősegítő viszonyulás a gyermekhez - családban,
  iskolában
\item
  A szenvedélybetegségek és megelőzésük (dohányzás, alkohol- és
  drogfogyasztás, játék-szenvedély, internet- és tv-függés)
\item
  Művészeti és sporttevékenységek lelki egészséget, egészséges
  személyiségfejlődést és tanulási eredményességet elősegítő hatásai
\item
  Médiatudatosság, a médiafogyasztás egészségvédő módja
\item
  Az idő és az egészség, bioritmus, időbeosztás
\item
  Tartós egészségkárosodással élő társakkal együttélés, a segítségre
  szorulók segítése
\item
  Önmagunk és egészségi állapotunk ismerete
\item
  A személyes krízishelyzetek felismerése és kezelési stratégiák
  ismerete
\item
  Az idővel való gazdálkodás szerepe
\item
  A rizikóvállalás és határai
\item
  A tanulási környezet alakítása
\item
  A természethez való viszony, az egészséges környezet jelentősége
\end{itemize}

\subsubsection{Indikátorok}\label{indikuxe1torok}

A teljes körű iskolai egészségfejlesztés az alábbi részterületeken
jelentkező hatások révén eredményezi a hatékonyság növekedését:

\begin{itemize}

\item
  a tanulási eredményesség javítása
\item
  a társadalmi befogadás és esélyegyenlőség elősegítése
\item
  a társadalmi kapcsolatok javulása a kortársakkal, szülőkkel,
  tanárokkal
\item
  az önismeret és önbizalom javulása
\item
  az alkalmazkodókészség, a stresszkezelés, a problémamegoldás javulása
\item
  érett, autonóm személyiség kialakulása
\item
  a krónikus, nem fertőző megbetegedések (lelki betegségek,
  szív-érrendszeri, mozgásszervi és daganatos betegségek) elsődleges
  megelőzése
\end{itemize}

\subsubsection{A program végrehajtása - elsősorban a Harmónia (fizikai,
lelki jóllét és kapcsolódás a környezethez) tantárgy
keretében}\label{a-program-vuxe9grehajtuxe1sa---elsux151sorban-a-harmuxf3nia-fizikai-lelki-juxf3lluxe9t-uxe9s-kapcsoluxf3duxe1s-a-kuxf6rnyezethez-tantuxe1rgy-keretuxe9ben}

A Budapest School tanulási koncepciójának középpontjában az egyén, mint
a közösség jól funkcionáló, saját célokkal rendelkező tagja áll. Az
iskolában való fejlődése során elsősorban azt tanulja, hogy miként tud
specifikált saját célokat megfogalmazni, és hogyan tudja ezeket elérni.
Ebben a folyamatban egy mentor segíti a munkáját az iskola kezdetétől a
végéig. Ő figyel arra, hogy a gyerek fizikai és lelki biztonsága és
fejlődése folyamatos legyen, és segíti azokban a helyzetekben, amikor
biztonságérzete vagy stabilitása csökken.

A közösségben jól funkcionáló egyén belső harmóniájához ez a tantárgy a
következő fejlesztési területeket határozza meg:

\begin{itemize}

\item
  Érzelmi és társas intelligencia
\item
  Önismeret és önbizalom
\item
  Konfliktuskezelés
\item
  Rugalmasság (reziliencia)
\item
  Kritikai gondolkodás
\item
  Közösségi szabályok alkotásában való részvétel és azok alkalmazása
\item
  Csapatmunka gyakorlati fejlesztése
\item
  Oldott játék
\item
  Egészséges testi fejlődés
\item
  Saját igényekhez képest megfelelő táplálkozás
\item
  A természettel való kapcsolódás
\item
  Épített falusi és városi környezetben való eligazodás
\item
  A technológia világában felhasználói szintű eligazodás és annak
  harmonikus alkalmazása
\end{itemize}

\paragraph{Közösségben,
csapatban}\label{kuxf6zuxf6ssuxe9gben-csapatban}

A Budapest School egy közösségi iskola, ahol a közösség tagjai egymással
és egymástól tanulnak. A közösségekhez való tartozáshoz, a csapatban
való gondolkodáshoz, és a családban való működéshez szükséges
képességeket leginkább úgy tudjuk fejleszteni, ha azt kezdetektől
megéljük. A közösség belső szabályainak megalkotása és az azokhoz való
kapcsolódás a tanulás folyamatosságának alapfeltétele.

\paragraph{Életképességek (life
skills)}\label{uxe9letkuxe9pessuxe9gek-life-skills}

Szeretnénk, ha gyerekeink általában alkalmazkodóan (adaptívan) és
pozitívan tudnának hozzáállni az élet kihívásaihoz, ha lelki és fizikai
erősségük és rugalmasságuk (rezilienciájuk) megmaradna és fejlődne. A
WHO a következőképpen definiálta (World Health Organization, 1999) az
életképességeket:

\begin{itemize}

\item
  Döntéshozás, problémamegoldás
\item
  Kreatív gondolkodás
\item
  Kommunikáció és interperszonális képességek
\item
  Önismeret, empátia
\item
  Magabiztosság (asszertivitás) és higgadtság
\item
  Terhelhetőség és érzelmek kezelése, stressztűrés
\end{itemize}

\paragraph{Érzelmi intelligencia}\label{uxe9rzelmi-intelligencia}

Sokszor kiemeljük az érzelmi intelligenciát, kihangsúlyozva, hogy
gyerekeinknek többet kell foglalkozniuk az érzelmek felismerésével,
kontrollálásával és kifejezésével, mint szüleinknek kellett.

\paragraph{Szabad mozgás és séta}\label{szabad-mozguxe1s-uxe9s-suxe9ta}

A különböző mozgásformák, sportok és a séta mindennapivá tétele
természetes módon, a gyerekek saját igényei szerint kell hogy történjen.

\paragraph{Gyakorlatias, mindennapi
képességek}\label{gyakorlatias-mindennapi-kuxe9pessuxe9gek}

Ahhoz, hogy gyerekeink önállóan és hatékonyan tudják élni életüket, hogy
a társakhoz való kapcsolódás ne függőség legyen, egy csomó praktikus
mindennapi tudást el kell sajátítaniuk. A gyerekeknek folyamatosan
fejleszteniük kell az élethez szükséges minden- napi tudást a
levélszemét kezeléstől, a facebook profil tudatos használatán át,
egészen a személyi költségvetés készítéséig.

\paragraph{Egészséges
táplálkozás}\label{eguxe9szsuxe9ges-tuxe1pluxe1lkozuxe1s}

Az egészséges táplálkozás tanulható viselkedésforma, melynek alapja nem
csupán a megfelelő élelmiszerek kiválasztása, hanem azok élettani
hatásainak megismerése, és az étkezési szokások alakítása is.

\subsection{A
közösségfejlesztés}\label{a-kuxf6zuxf6ssuxe9gfejlesztuxe9s}

A közösségfejlesztéssel, az iskola szereplőinek együttműködésével
kapcsolatos feladatok

\begin{quote}
csak a nevelési-oktatási tartalmak relevanciája vonatkozásában, mert
egyébként SZMSZ-kompetencia)
\end{quote}

\subsection{Elsősegély-
nyújtás}\label{elsux151seguxe9ly--nyuxfajtuxe1s}

az elsősegély- nyújtási alapismeretek elsajátításával kapcsolatos
iskolai terv



\part{A helyi tanterv}
\chapter{Az iskola helyi tanterve}
\label{az-iskola-helyi-tanterve}

\section{A választott kerettanterv
megnevezése}
\label{a-vuxe1lasztott-kerettanterv-megnevezuxe9se}

Budapest School Általános Iskola és Gimnázium az oktatásért felelős
miniszter által jóváhagyott Budapest School Kerettanterv alapján
működik. Az iskola munkatársai a kerettantervet, a pedagógia programot,
sőt még a szervezeti és működési szabályokat is együtt, egy egységként
állították össze. Ez a Budapest School Módszer, aminek fő célja, hogy a
gyerekek úgy tudják azt és akkor tanulni, amit szeretnek, vagy amire
szükségük van, hogy közben a tanárok, szülők számára is kiszámítható,
tervezhető, biztonságos tanulási környezetet biztosít az iskola.

Az iskola személyre szabható tanulási környezetet biztosít. A
tanulásszervező tanárok, az iskola sok választási lehetőséget kínál a
gyerekek számára, akik a lehetőségekből összeválogatják a saját tanulási
céljaikhoz leginkább illeszkedő saját tanulási programjukat. A
választásban -- az önvezérelt tanulás menedzselésében -- a gyerekek
folyamatos egyéni támogatást kapnak a \emph{mentor} tanáruktól.

A tanulási célokat, a választásokat, az igényeket és az elvárásokat
trimeszterenként a gyerek, a tanár és a szülő egy tanulási szerződésben
rögzíti. Erre mintát ad XX melléklet.

\begin{quote}
Példa: egy 7 éves gyerek, Elemér szeretne önállóbban olvasni, és ennek a
szülők és tanárok is örülnek. Megállapodnak, hogy a cél, hogy a
trimeszter végére a gyermek képes legyen egy minimum 20 oldalas könyvet
önállóan elolvasni. Ezért a mentor vállalja, hogy társaival együtt
olvasóklubot szervez, amikor hetente minimum 30 percet egyénileg
olvasnak a gyerekek. A szülők vállalják, hogy megvásárolják vagy
könyvtárból kikölcsönöznek Elemérnek 3 könyvet a trimeszter során.
Elemér pedig vállalja, hogy hetente 30 percet önállóan olvas.
\end{quote}

\section{Modulok és a tantárgyak}
\marginpar{ez kell ide extran?}
A Budapest School Kerettanterv a három tantárgyat határoz meg: STEM,
KULT és Harmónia tantárgyakkal fedi le a Nemzeti Alaptanterv által
meghatározott tartalmi követelményeket. A gyerekek órarendjében azonban
nem ezek a tantárgyak jelennek meg, hanem specifikusabb, rövidebb,
célorientáltabb egységek: a modulok. A Budapest School modulok
különböznek a megszokott tantárgyi tanóráktól. Abban azonban
hasonlítanak, hogy a moduloknak is mindig van előre meghatározott
fejlesztési célja, azaz, hogy a modul során mit tanul, miben fejlődik
várhatóan a résztvevő gyerek.

\paragraph{A modulok rövidebbek.}

A tantárgyak céljait, folyamatai általában minimum egy tanévre
határozzák meg. A miniszter által kiadott kerettantervekben két éves
egységekben gondolkoznak a szerzők. A Budapest School moduljai lehetnek
egy hetesek is, vagy legfeljebb egy trimeszter hosszúak. A lezárását
követően egy tanulási folyamat egy új modulban tovább folytatódhat.

\paragraph{Tervezettség}\label{tervezettsuxe9g}

A tantárgyi rendszerek a kimeneti követelmények alapján előre
megtervezettek. A Budapest School a gyerekek céljai és érdeklődése
alapján, akár az igény megjelenése utáni héten is indulhatnak.
Természetesen nem kell mindig kitalálni a spanyol viaszt. Ha egy modul
bevált, akkor lehet egy másik gyerekcsoporttal is kipróbálni. De mindig
meg kell tartanunk annak a lehetőségét, hogy más gyerekekkel, más
tanárral, más helyen ugyanaz a modul is másképp sikerül.

\paragraph{Tantárgyközi tudás}\label{tantuxe1rgykuxf6zi-tuduxe1s}

A modulok alapértelmezett tulajdonsága, hogy több tantárgy tartalmából
építkeznek, hisz \emph{minden, mindennel} összefügg. Így minden modul
lefedhet egy, kettő vagy akár három tantárgy tartalmát. A
kerettantervben meghatározott kiegyensúlyozottság elve alapján a
tanároknak úgy kell modult hirdetniük és gyerekeknek választaniuk, hogy
minden tantárgy tartalmával közel egyenlő mértékben találkozzanak.

\paragraph{Modulindítás}\label{modulinduxedtuxe1s}

Modulokként jelennek meg a foglalkozások, a tanórák, a projektek, a
fakultációk, szóval minden olyan tevékenységek, amit a gyerekek az
iskolában csinálhatnak. Az a tanulásszervező tanárok feladata, hogy a
gyerekek minden trimeszter előtt legalább két héttel megismerhessék a
következő trimeszter választható moduljait.

Modul indulásakor mindig elkészül egy modulkiírás, ami tartalmazza a
modul célját, azaz, hogy \emph{miért} csináljuk, hogy \emph{mit} fognak
a gyerekek csinálni a modul során, a tervezett tantárgyi
eredménycélokat, \emph{mikor} és mennyi ideig tart a modul, ha érdekes,
akkor a módszert, azaz \emph{hogyan szervezzük a tanulást}, és a modul
során tapasztalható fejlődés érteklési szempontjait. Példa

\begin{itemize}

\item
  Néptánc Sárival

  \begin{itemize}
  \item
    Cél: közösen táncoljunk, hogy többet mozogjunk és zenei érzéket
    fejlesszük
  \item
    Mit: 8 alkalom, hetente 1 órában 2 népdalra tanuljuk be a
    koreográfiát
  \item
    Tantárgyi célok: Harmnónia, BLA BLA
  \item
    Értékelési szempontok: együtt mozgott társaival, energikusan
    mozgott, követte a zenét, megfelelő ruhában érkezett
  \end{itemize}
\end{itemize}

\section{Óraszámok}\label{uxf3raszuxe1mok}

\begin{quote}
A választott kerettanterv által meghatározott óraszám feletti kötelező
tanórai foglalkozások, továbbá a kerettantervben meghatározottakon felül
a nem kötelező tanórai foglalkozások megtanítandó és elsajátítandó
tananyaga, az ehhez szükséges kötelező, kötelezően választandó vagy
szabadon választható tanórai foglalkozások megnevezése, óraszáma
\end{quote}

Minden gyereknek úgy kell tanulási célokat és tanulási utat, azaz
modulokat választani, hogy három egymásutáni trimeszter óraszámai
alapján mind a három tantárggyal közel egyenlő mértékben foglalkozzon.

A gyerekek trimeszterenként a mentortanáruk segítségével és a szüleikkel
egyetértésben választanak modulokat, így nem tudjuk előre meghatározni,
hogy mit, mekkora óraszámban tanul. Azt azonban tudjuk figyelni, hogy a
kerettanterv kiegyensúlyozottság elve teljesül-e. Modulonként
kiszámolható, hogy egy-egy gyerek egy tantárggyal hány órát
foglalkozott. Ezeket összegezve ki tudjuk számolni, hogy egy gyerek hány
órát fektetett be egy-egy tantárgy tartalmának megismerésébe. Ezek
alapján megállapítható, hogy egyensúlyban vannak-e a gyerek tantárgyi
élményei.

\section{Tankönyvek
kiválasztása}\label{tankuxf6nyvek-kivuxe1lasztuxe1sa}

\begin{quote}
Az oktatásban alkalmazható tankönyvek, tanulmányi segédletek és
taneszközök kiválasztásának elvei (figyelembe véve a tankönyv
térítésmentes igénybevétele biztosításának kötelezettségét).
\end{quote}

Modulvezetők minden esetben maguk választják a modulhoz szükséges
tankönyvek, szoftverek, weboldalak és egyéb eszközöket úgy, hogy

\begin{itemize}

\item
  az a megfelelő legyen annak a csoportnak, ahhoz a célhoz, amit el akar
  érni
\item
  minden esetben legyen mindenki számára elérhető (esetek többségeben
  értsd ingyenes) megoldás
\item
  modulvezetők bátorítva vannak arra, hogy új dolgokat próbáljanak ki,
  és tapasztalataikat az iskola többi tanárával megosszák.
\end{itemize}

Mivel a Budapest School kerettantervének értelmében az egyéni célok
legalább 50\%-át az állami kerettantervben meghatározott fejlesztési
célok közül kell választani, az ehhez szükséges ismeretek megszerzéséhez
a Budapest School az Oktatási Hivatal általi jegyzékben államilag
támogatott OFI által fejlesztett tankönyveket veszi alapul. A Budapest
School tanárcsapatának lehetősége van arra, hogy ettől eltérő, a
mindenkori tankönyvjegyzékben szereplő tankönyvvel segítse el a
kerettantervben meghatározott eredménycélok elérését. És arra is
lehetősége van, hogy egyátalán ne használjon tankönyvet, mert sokszor az
internet elegendő információt tartalmaz.

A Budapest School pedagógiai programjának alapja, hogy a gyerekek egyéni
céljaira szabott tanulási terveket készít. Ennek előfeltétele, hogy a
könyvek használata is ehhez kapcsolódó módon rugalmasan történjen,
minden esetben az adott tanulási modul igényeihez szabva. Ennek
érdekében a program pedagógusai folyamatosam állítják össze a gyerekek
eltérő céljaihoz és képességszintjeihez igazodó differenciált
tevékenységek és feladatsorok rendszerét.

\section{NAT pedagógiai
feladatok}\label{nat-pedaguxf3giai-feladatok}

\begin{quote}
a Nemzeti alaptantervben meghatározott pedagógiai feladatok helyi
megvalósításának részletes szabályai
\end{quote}

\texttt{EZT\ KI\ KELL\ KERESNI}
\texttt{ide\ csak\ azt\ kell\ irni,\ h\ a\ tantargyak\ ezt\ elintezik}

\section{Mindennapos testnevelés}\label{mindennapos-testneveluxe9s}

\begin{quote}
a mindennapos testnevelés, testmozgás megvalósításának módja, ha azt nem
az Nkt. 27. § (11) bekezdésében* meghatározottak szerint szervezik meg,
Az iskola az Nkt. 27. § (11) bekezdésben meghatározottak szerint
szervezi meg, hogy mindennap alkalma legyen a gyerekeknek a
testnevelésre, testmozgásra. Mivel a Budapest School kerettanterv nem
határoz meg külön testnevelés tantárgyat, így minden modul annak
tekintendő, aminek célja a mozgás, az egészséges életmód és a Harmónia
tantárgy témáját (is) feldolgozza.
\end{quote}

\section{Válaszható érettségi
tárgyak}\label{vuxe1laszhatuxf3-uxe9rettsuxe9gi-tuxe1rgyak}

\begin{quote}
középiskola esetén azon választható érettségi vizsgatárgyak megnevezése,
amelyekből a középiskola tanulóinak közép- vagy emelt szintű érettségi
vizsgára való felkészítését az iskola kötelezően vállalja, továbbá annak
meghatározáse, hogy a tanulók milyen helyi tantervi követelmények
teljesítése mellett melyik választható érettségi vizsgatárgyból tehetnek
érettségi vizsgát,
\end{quote}

Az iskola a kötelező középszintű érettségi vizsgatárgyakra való
felkeszítést kötelezően vállalja érettségi felkészítő modulok
szervezésével. (Más iskolákban ezt fakultációnak hívnák, de a Budapest
School iskola nem különbözteti meg a fakultációt a tanórától.) A
választható tantárgyak és az emeltszintű érettségi vizsgára csak akkor
szervez egy mikroiskola modult, ha arra legalább a közösség 10\%-a
igényt tart.

\subsection{középszintű érettségi vizsga
témakörei}\label{kuxf6zuxe9pszintux171-uxe9rettsuxe9gi-vizsga-tuxe9makuxf6rei}

\begin{quote}
középiskola esetén az egyes érettségi vizsgatárgyakból a középszintű
érettségi vizsga témakörei
\end{quote}

\texttt{EZ\ HP\ SZERINT\ LEHET\ HOGY\ NEM\ KELL}

\section{Értékelés,
minősítés}\label{uxe9rtuxe9keluxe9s-minux151suxedtuxe9s}

\begin{quote}
a tanuló tanulmányi munkájának írásban, szóban vagy gyakorlatban történő
ellenőrzési és értékelési módját, diagnosztikus, szummatív, fejlesztő
formáit, valamint a magatartás és szorgalom minősítésének elvei
\end{quote}

A Budapest School iskolában a felnőttek a gyerekek munkáját,
erőfeszítését, részvételét a Budapest School kerettantervének 3.4
fejezetében leírt módon értékeli. Hasonlóan a gyerekek is folyamatosan
értékelik a tanárok munkáját.

Minden modul zárásakor a részvételt, a haladást, a fejlődést és az elért
eredményeket is értekelik a gyerekek és tanárok. Trimeszter zárásakor a
mentorok és mentorált gyerekek értekelik a trimeszter tanulási
eredményeit, és hogy hogy van a gyerek az iskolában, a céljaival, a
közösséggel.

\subsection{Portfólió}\label{portfuxf3liuxf3}

Minden projekt eredménye, minden visszajelzés, minden tudáspróba, szóval
a gyerekek iskolai munkásságának az eredménye a \emph{portfólióba}
kerül. A potfólió alapján megállapítható, hogy a gyerek mit csinált,
amiből következtethetünk arra, hogy mit tud, milyen képességei,
készségei, kompetenciái vannak.

Portfólió fejlesztése, rendbentartása elsősorban a gyerekek feladata. A
tanárok és szülők pedig elsődlegesen arra figyelnek, hogy a gyerekek
megtanulják: nem elég valamit megcsinálni, megtanulni, fontos, hogy
megtanuljuk megmutatni is magukat.

IDE EGY MONDAT: a mentorok meg azert felelosek, hogy a gyerekekek
megcsinaljak.

\subsection{Évfolyamok elismerése, a
bizonyítvány}\label{uxe9vfolyamok-elismeruxe9se-a-bizonyuxedtvuxe1ny}

Ahogy a Budapest School Kerettanterv 4.5 fejezete is meghatározza, a
gyerekek mind a három tantárgyból más évfolyamszinten állhatnak.
Trimeszterenként a gyerekek, szülők kérhetik, hogy a portfóliójuk
alapján évfolyamszintet léphessenek egy vagy több tantárgyból.

Ennek első lépése, hogy a portfólióba bekerüljenek olyan elemek, amik
bizonyítják, hogy a tantárgyi követelményeknek legalább a felét
teljesítették. Ezután a gyerek build a case to level up. Fontos, hogy a
teljes dokumentációnak digitálisan, online elérhetőnek kell lennie úgy,
hogy távolról megítélhető legyen XXX

\subsection{Osztályzatok, vizsga}\label{osztuxe1lyzatok-vizsga}

Ha iskolaváltás vagy továbbtanulás miatt a szülő vagy gyerek kéri, hogy
gyereket félévkor és/vagy év végén osztályzattal minősítsük, és a
portfólia értékelésünk alapján megajánlott osztályzatot nem fogadja el,
a tanuló osztályozó vizsgát tehet. Ebben az esetben a szülő osztályozó
vizsga iránti kérelmet nyújthat be az mentortanárnál írásban is
megerősítve. A tanév alatt beadott írásbeli kérelem beadását követően az
iskolának 20 tanítási nap áll a rendelkezésére a vizsga megszervezésére.
(Ez alatt a tanév elején a tanév rendjében kiírt tanítási napok
értendőek. A tanév elején tanítási szünetnek kiírt napok nem számítanak
tanítási napnak!) Amennyiben a szülő a nyári szünet ideje alatt
kérelmezi az osztályozó vizsga megtartását, úgy azt legkésőbb június
30-ig jeleznie kell (írásban is). Az ilyen esetekben az osztályozó
vizsga augusztus utolsó hetében kerül megtartásra.

\subsubsection{Vizsgák tartalma}
A VIZSGA nem tudáspróba, hanem a portfólió alapján védés.
TODO, FONTOS

\todo[inline]{itt azt kell leírni, hogy a vizsga nem más, mint hogy hozza a portfólióját, és az alapján elmondja, hogy miért kell abban az évfolyamban azt a jegyet kapnia. Az portfolionak legyen lehetoleg digitalis valtozata.}
az egesz vizsa tortenhet online es aszinkron.



\section{A csoportbontások}\label{a-csoportbontuxe1sok}

\begin{quote}
a csoportbontások és az egyéb foglalkozások szervezésének elvei
\end{quote}

A Budapest School iskoláit mikroiskolák közösségeiből hozzuk létre. Így
egy gyerek elsődleges csoportja a mikroiskolájának közzéssége, ami lehet
12 - 50 gyerek. Ezen belül modulonként eltér, hogy milyen
csoportbontásban dolgoznak. Több szinje van a csoportmunkának. 1. A
mikroiskola közössége heti rendszerességgel tarthat iskolagyűlést,
fórumot, HOGY HIVJAK A BREZNOBAN?. Ilyenkor a mikro iskola közössége
együtt van. 2. Modulokra kisebb csoportok jelentkezhetnek. Egy modul
csoportjának rendező elve lehet, hogy 1. egy képességszinten lévő
gyerekek tanuljanak együtt; 2. vagy közös érdeklődés hozza össze a
csoporttagokat; 3. van, hogy direkt a véletlenszerűségben van az
érdekesség, mert keveredni akarunk; 4. kölcsönös szimpátia és vonzalom
lehet a modultagok között: most azért vannak egy csoportban, mert egy
csoportban akartak lenni. 3. Egy-egy foglalkozáson belül is sokszor
csoportot alkotunk, az előző elvek alapján.

\section{Nemzetiségek
megismerése}\label{nemzetisuxe9gek-megismeruxe9se}

\begin{quote}
a nemzetiséghez nem tartozó tanulók részére a településen élő nemzetiség
kultúrájának megismerését szolgáló tananyag,
\end{quote}

\section{az egészségnevelési és környezeti nevelési
elvek}\label{az-eguxe9szsuxe9gneveluxe9si-uxe9s-kuxf6rnyezeti-neveluxe9si-elvek}

\begin{quote}
az egészségnevelési és környezeti nevelési elvek
\end{quote}

\section{Esélyegyenlőség}\label{esuxe9lyegyenlux151suxe9g}

\begin{quote}
a gyermekek, tanulók esélyegyenlőségét szolgáló intézkedések
\end{quote}

\section{Jutalmazás és
értékelés}\label{jutalmazuxe1s-uxe9s-uxe9rtuxe9keluxe9s}

\begin{quote}
a tanuló jutalmazásával összefüggő, a tanuló magatartásának,
szorgalmának értékeléséhez, minősítéséhez kapcsolódó elvek
\end{quote}

\todo[inline]{ legyen growth mindsetet, praise the effort, ugy mint Ha adunk,
akkor adjunk pozitív visszajelzést, fókuszáljunk arra, amit csinált a
gyerek és nem arra, amit elért. jutalmazas bevezeto}

\begin{itemize}

\item
  Fantasztikus, ma egy nagy kihívást választottál!
\item
  Bátran rizikót vállaltál!
\item
  Nagyon jó! Tényleg sokat próbálgattad.
\item
  Kitartóan csináltad, erre nagyon büszke vagyok!
\item
  De jó, valami újat próbáltál ki ma!
\item
  Köszönöm, hogy ma valakinek segítettél.
\item
  Nagyon nagy öröm látni a haladásodat!
\item
  Ne feledd, mindannyian tudunk a hibáinkból tanulni. Örüljünk annak,
  hogy ma valamit jobban tudunk, mint előtte.
\item
  Wow, egy nehéz feladatot oldottál meg!
\item
  Szép munka! Kipróbáltál egy másik módszert.
\end{itemize}

\section{Önkéntes közösségi
szolgálat}\label{uxf6nkuxe9ntes-kuxf6zuxf6ssuxe9gi-szolguxe1lat}

Az Nkt 4. § (15) pontjában definiált közösségi szolgálat is modulként
kerül meghírdetésre, amit 12. évét betöltött gyerek választhat csak.
Közösségi szolgálatként elfogadható, ha a Budapest School iskola egy
másik mikroiskolájában segít a gyerek.



\bibliography{references.bib}{}
\label{sec:bibliographyk}
\bibliographystyle{apalike}

\end{document}
