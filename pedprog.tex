\documentclass[magyar,12pt,a4paper,draft]{report}
\usepackage{graphicx}

\usepackage[T1]{fontenc}
\usepackage[utf8]{inputenc}
\usepackage[magyar]{babel}

\usepackage[colorinlistoftodos,prependcaption,textsize=tiny]{todonotes}

\usepackage{verbatim}
\usepackage{markdown}
\usepackage{booktabs}
\usepackage{longtable}
\usepackage{array}
\usepackage[justification=centering]{caption}
\usepackage{url}
\usepackage[nottoc,numbib]{tocbibind}
\DeclareUnicodeCharacter{2212}{-}
% heylyseglista tablazat miatt
\usepackage{graphicx}
\usepackage{lscape}
\usepackage{natbib}
\pagestyle{headings}
\usepackage{fancyhdr}
\usepackage[yyyymmdd,hhmmss]{datetime}
\pagestyle{fancy}

\fancyhf{}
\rhead{}
\lhead{\leftmark}

\rfoot{Compiled on \today\ at \currenttime}
\cfoot{}
\lfoot{}
\begin{document}
% csinalunk egy feltetelt, hogy a szovegben tudjuk, hogy kerettantervben vagyunk
% vagy a pedprogramban
\newif\ifkerettanterv
\kerettantervfalse

\title{Budapest School pedagógia program}
\author{Budapest School általános iskola tanárai}
\maketitle

Kedves Olvasó,

Ezekre figyelj:
\begin{itemize}
  \item \ref{sec:osztalyozo_vizsga} eleg eros. Tessek megnezni.
  
\end{itemize}

\listoftodos[Notes]

\tableofcontents
\newpage
 
\part{Nevelési program}
\chapter{Iskola célja, alapelvei}
\label{sec:alapelvek}
% vision and goals
\section{Az iskola célja}
\label{sec:iskola_celja}

A Budapest School abban támogatja a gyerekeket, hogy azok az
attitűdök, képességek és szokások alakuljanak ki bennük, amelyek segítségével
boldog, egészséges és a társadalom számára hasznos felnőttekké válhatnak. A
cél, hogy a gyerekek a mai világ szükségleteihez és lehetőségeihez a saját
erősségeik felhasználásával kapcsolódhassanak.	Hogy tanulási útjukat
sajátjukként éljék meg, és felelősséget érezzenek az alakításáért, az újabb és
újabb kihívások megtalálásáért.

Olyan tanulási környezetet kell kialakítani ehhez, ahol a szülők, tanárok és
gyerekek tesznek magukért és egymásért, ahol gyerekeink képesek nehéz
helyzetekben is életük, kapcsolataik és környezetük aktív alakítói lenni,
cselekedeteikkel, tetteikkel elérni a kitűzött céljaikat.

Mindenki kíváncsinak születik, és meg tudja tanulni azt, amit
igazán szeretne. Nincs is másra szükség, csak izgalmas kihívásokra, kérdésekre,
biztonságra, támogatásra és lehetőségekre.

Ennek szellemében az iskolánk elsődleges feladata, hogy a gyerekek közösségét
segítsék abban, hogy sokat és hatékonyan tanuljanak és alkossanak.
\emph{Tanulják azt, amit szeretnének, és azt, amire szükségük van.}

\section{A tanulni tanulás hét pillére}
A tanulás egyénenként változó, és ha vegyes korosztályokban is, mégis egy közösségben valósul meg. \emph{A tanulni tanulás hét pillére} minden korosztályban állandóan befolyásolja és keretezi a Budapest School pedagógiai programját.

\subsection*{Tanulni tanulunk -- fejlődésközpontúak vagyunk}
A Budapest School mikroiskoláiban mindig ott, akkor és annak kell történnie a gyerekekkel, ami őket a fejlődésükben a leginkább támogatja. Minden, ami az iskolában történik, újra és újra erre az alapkérdésre kell hogy visszatérjen. Az segíti a leginkább a gyerekek fejlődését, amit most csinálunk, vagy változtatnunk kell rajta? A Budapest School tehát rugalmas és integratív, a gyerekek fejlődéséhez igazodó.

A gyerekek tanulása \emph{fejlődésközpontú szemléletben}
\citep{growthmindset}\linebreak
(growth mindset) történik.  Erőfeszítéseik segítségével képességeik
fej-\linebreak
leszthetők, megváltoztathatóak. Számukra inspiráló a kihívás, és a hibázás kevésbé töri le lelkesedésüket. A gyerekek és a tanárok számára ezáltal nagyobb fontossággal bír a tanulásba fektetett erőfeszítés és a fejlődés, mint az aktuális képességeik.

A hibázás inkább válik a gyakorlás és az új megismerésének jelzőjévé. Ennek feszültségmentes kezelése kulcsfontosságú abban, hogy a gyerekek merjék feszegetni a saját határaikat, hogy magabiztosan dolgozzanak azon, hogy képességeiket, ismereteiket vagy gyakorlataikat folyamatosan fejlesszék. A hibázásból való tanulás fő célja, hogy mindig új hibákat ismerjenek meg, és a korábbiakra minél jobb megoldásokat találjanak a gyerekek.

\subsection{Tanulni tanulunk -- saját célokat állítanak}
A gyerekek saját erősségeiket fejlesztve saját célokat állítanak, közben céljaikat folyamatosan igazítják a világ adta lehetőségekhez és szükségletekhez.

A tanulás egésze egy olyan folyamatként írható le, amely különböző állomásokra, rövid célokra bontható. A tanulási célok állításának folyamata, annak minősége az évek alatt folyamatosan változik, egyre tudatosabbá, pontosabbá, komplexebbé válik. Azonban már az első évektől el kell kezdeni annak gyakorlását, hogy később kialakulhasson a saját tanulásicél-állítás.

A kisgyerekkor jellemzője a kíváncsiság, az igény a felfedezésre, tapasztalásra. A gyerekek számára  \emph{a tanulás  egy önvezérelt aktív folyamat}, melynek megtartása és folyamatos fejlesztése a Budapest School tanárainak legfőbb feladata. Két út van: vagy megtanítjuk a gyerekeket azokra a képességekre, amelyekre ma szükségük van, ezzel kockáztatva azt, hogy a tanulásuk a világ változásával veszít korszerűségéből, vagy abban segítjük őket, hogy megtaníthassák magukat azokra a képességekre, amelyekre épp az adott élethelyzetükben szükségük van. A tanulás így élményszerűvé válik, ismeretszerző jellege csökken, és nő az önálló felfedezés lehetősége.

\subsection{Tanulni tanulunk -- mindig és mindenhol}
A tanulás a Budapest Schoolban egész nap történik, melynek pontos alakítása a gyerekek tanulási igényeitől, fejlettségi szintjétől és korosztályától is függ. A tanulási rend meghatározásáért a Budapest School mikroiskoláinak tanulásszervezői felelnek.

A tanulás trimeszterenként újraszervezett módon tanulási modulokban történik, melynek során jut idő egyéni és csoportos, gyakorló, ismeretszerző és gyakorlatias foglalkozásokra is. Az egész napos iskola lehetőséget biztosít arra, hogy a tanulási egységek között legyen idő fellélegezni, és felkészülni az újabb modulokra, valamint arra is, hogy ha egy tanulási egység nagyon magával ragadja a gyerekeket, akkor benne maradhassanak és annak megfelelően alakítsák újra az időrendjüket.

A tanulás az iskolában nem ér véget. A tanulás szeretetének kialakulásával folyamatossá válik az ezzel való foglalkozás, így a Budapest School tanulásnak veszi az otthon, szünetekben eltöltött időt is, ahol sokszor hatékonyabb módon tud egy gyerek gyakorolni, kutatni, alkotni, mint az iskolában, amikor társaival van egy közösségben és ezáltal számos más inger is éri.

Az iskolában történő tanulással egyenrangúnak tekintjük az otthon tanulást, az iskolától független iskola utáni programokat, a (nyári) táborokat, a családi utazásokat, a vállalatoknál töltött gyakornoki időt, az egyéni tanulást és projekteket. A tanulás bárhol és bármikor történhet, és célunk, hogy mindenhol és mindig tanuljanak.

\subsection{Tanulni tanulunk -- együtt, egymástól}

A tanulás egyénileg és csoportokban is történhet. A csoportok megszervezése mindig azon múlik, hogy az adott tanulási célt mi szolgálja a legjobban. Ennek megfelelően a gyerekek nem állandó, hanem a tanulási célokhoz, az érdeklődéshez, a képességi szintekhez alkalmazkodó rugalmas csoportokban tanulnak. A tanulás ezáltal kevert korcsoportokban is történhet, akár egy nagy családban. Együtt, egymástól tanulnak a gyerekek, egymást segítik a fejlődésben. Az egymásnak adott fejlesztő visszajelzések, pozitív megerősítések révén folyamatosan alakul ki a tanulás tisztelete és a képességek fejlesztésébe vetett hit.

A közösségben tanulás módja nagyban függhet attól, hogy egy gyerek mennyire zárkózott, mennyire tud és akar önállóan tanulni. A csoportos munkák során alapelv, hogy a zárkózott gyerekek is lehetőséget kapjanak, hogy csöndesen, vagy kisebb csoportban végezhessék a munkájukat, mondhassák el ötleteiket. Az egyéni tanulásban minden gyereknek lehetőséget kell adni arra és segíteni kell abban, hogy önállóan, fókuszáltan tudjon tanulni.\vfill\eject

\subsection{Tanulni tanulunk -- alkotnak és felfedeznek}
A tanulás három rétege, az ismeretszerzés, a gondolkodás fejlesztése és a gyakorlatias, aktív alkotás egyszerre jelenik meg a Budapest School mindennapjaiban. Az alkotó munka rugalmas időkereteket, változó csoportbontásokat, és a projektmódszerek sokszínű alkalmazását igényli. A tanulás ilyenkor sokszor csinálássá válik, az ismeret pedig termékké változik.

A gyerekek önmaguk és a világ számára releváns kérdésekkel foglalkoznak, amihez külső szakértőket is bevonnak, ha szükséges. A tanulás tehát célokhoz és nem tárgyakhoz kötött. A tanulás tartalmát igazítjuk a tanulás céljához, ezáltal az egyes tudományterületek, művészetek, vagy épp mesterségek gyakran keverednek egymással egy-egy modulon belül. Szintén a célhoz igazított tanuláshoz kötődik a kutató-felfedező attitűd, ami a már ismert újramegismerése mellett az ismeretlen felfedezésére, a megválaszolhatatlan megválaszolására irányul.

\subsection{Tanulni tanulunk -- elfogadóak és egymásra figyelnek}

A gyerekek tanulását családi hátterük változása, egyéni problémák, számos mindennapi esemény befolyásolhatja. Ezek figyelembevétele a mindennapokban, a \emph{mentor tanárral} való bizalmi viszonynak köszönhetően válik lehetségessé. Ennek a kapcsolatnak az alapjait ezért a partnerség, az értő figyelem adja.

A Budapest School emellett kiemelt figyelmet fordít arra, hogy a sajátos nevelési igényű tanulók is lehetőséget kapjanak a csoportban való munkára, amennyiben az a közösség számára is hasznos. Tanulásukat, ha szükséges, külső szakember segíti. A Budapest School a hátrányos helyzetű gyerekek számára is biztosítani kívánja az elfogadó, fejlesztő környezetet. Az egyenlő bánásmód megvalósulása érdekében olyan differenciált tanulási környezetet alakít ki, ami biztosítja a minél nagyobb mértékű inkluzivitást.

\subsection{Tanulni tanulunk -- tanulva tanítunk}
A Budapest School \emph{tanulásszervezői} partnerként, a tanulás folyamán segítő társként vannak jelen a gyerekek életében. A tanulás tanórák helyett pontos tanulási célokat tartalmazó tanulási modulokból épül fel, melyek során a tanár az adott cél eléréséhez szükséges eszközöket, tanulási segédleteket biztosítja.

A tanár akkor és annyira segíti a gyerekeket a saját céljuk elérésében, amennyire azt a gyerek igényli, és folyamatosan tekintettel van arra, hogy a gyerek saját fejlődési üteme megvalósulhasson. Ehhez tudatosan kell kezelnie nemcsak a gyerekeket érintő fejlesztési lehetőségeket, hanem azt is pontosan látnia kell, hogy egy adott tanulási cél elérésének milyen készségszintű vagy gyakorlatias alapfeltételei vannak. Ezért a gyerekek tanulási célját támogatandó segítenie kell abban, hogy a gyakorló idő, a gyakorlatias alkotó idő és az új ismeret megszerzésének ideje folyamatos egyensúlyban legyen. A tanár a gyerekekkel együtt fejlődik, saját tanulása a segítői szerepben folytonos.

\section{Emberkép}
\label{sec:gyerekkep}

A világ és benne az egyén állandóan változik, és a változás mégis számos állandóságot jelöl ki. Korábban a technológiai változás évszázadokban volt mérhető, az általunk használt eszközök száma és komplexitása jóval kevesebb volt, mára néhány év, vagy egy pár órás repülőút elegendő ahhoz, hogy olyan környezetbe, olyan emberek, eszközök közé kerüljünk, ahol másként kell alkalmazkodnunk, mint ahogy azt tanultuk. Ebben a világban a saját belső harmóniánk megtalálása, a saját közösségünkhöz, közösségeinkhez való kapcsolódás, a világ működésének megértése és a törekvés arra, hogy tegyünk a fenntartásáért, különösen fontossá vált. Ahhoz, hogy ennek megfeleljünk, a tanári, szülői és a tanulói szerepekben is folyamatos fejlődésre van szükségünk. A jövőbe nem látunk, de egy dolgot biztosan tudunk: bármilyenek is lesznek a jövő kihívásai, nekünk az a fontos, hogy ez a gyerekek számára ne félelmetes és szorongást keltő legyen, hanem lehetőséget, kihívást és örömet okozzon.

A célunk, hogy a fiatalok az iskolában és utána is könnyen megtalálják a saját útjukat. Képesek legyenek önmaguknak célokat állítani, azokat elérni. Képesek legyenek már kisgyerekkortól sajátjuknak megélni a tanulást és ahhoz kapcsolódóan célokat elérni, és fokozatosan tanulják meg azt, hogy egyénileg és csoportosan is tudjanak nagyszabású projekteket véghezvinni. A tanulás három rétege, a tudás megszerzése, az annak felhasználását segítő önálló gondolkodás és az aktív alkotás egyszerre jelenik meg a mindennapokban. Ezek egyensúlyban tartása éppoly fontos, mint az, hogy a Budapest School kerettantervének megfelelően a saját célok legalább 50\%-ban a kerettantervi tanulási eredményekből és további 50\%-ban ott nem felsorolt célokból épüljenek fel. Saját célként a sajátnak megélt célokat értjük.

Ezért \emph{legfontosabb fejlődési dimenziónak az önálló, célorientált tanulási és stratégiai gondolkodást tekintjük}, melynek ütemeit tanulási szakaszokra bontjuk. A társas kapcsolatok tanulási folyamatában a születéstől a felnőttkorig a teljes magára hagyatottság és énközpontúság csecsemőkori állapotából fejlődik az ember önállóan és kreatívan gondolkodó, önmagával és közösségével integránsan élő érett nagykorúig. Ezt az utat Erik Erikson pszichoszociális fejlődési modelljében rögzítette \citep{Erikson91}. A Budapest School ezt a modellt integrálja tanulási struktúrájába, melynek eredményeképp az iskolakezdőtől az érettségizőig négy tanulási szakaszt különít el. Ezen tanulási szakaszok határait az önállósodás során tapasztalt egyéni mérföldkövek jelölik ki, melyek fokmérője a saját cél állításának időbeli, tartalmi előrelépései, valamint az egyén belső szabadságérzete és közösséghez való kapcsolódása. Amellett, hogy az egyes tanulási szakaszok  a célállításban elkülönülnek egymástól, a szintugrás abban is mérhető, hogy mennyire képes egy gyerek az alatta lévő szinten lévő társait segíteni az előrelépésben. A folyamatos kortárs támogatás a szintekben való előrehaladás valós fokmérője a mentor és a szülő visszajelzései mellett.

\paragraph{Első szint, 5--10 év }

Ebben a korban alakul ki a gyerekek logikus gondolkodása, melynek részeként feladatokat tudnak rendszerezni, sorrendeket képesek felállítani és azokat fogalmakhoz társítani. A gondolkodásuk elkezd a befelé fordulóból a társas kapcsolatok irányába nyílni, ezáltal megnyílik a közösségi problémamegoldás lehetősége. Igényük nő a belső és a külső rendezettségre, így elkezdhetnek önmaguk szabályokat alakítani a saját tanulási igényeik kapcsán. Az első szakaszban a gyerekek pszichoszociális fejlődésükkel összhangban a saját maguknak állított cél jelentőségének fogalmát tanulják rövid, eleinte néhány órás, majd a szakasz vége felé néhány napos tervezéssel és erre adott önreflexiókkal és külső megerősítésekkel.  Megtanulják a ma, a tegnap és a holnap fogalmát. Tanulási szerződésük tartalmát eleinte főleg a mentor és a szülő segítségével állítják össze, hogy annak a szakasz végére már teljesen egyedi, önállóan kiválasztott elemei is lehessenek.

Ebben a szakaszban a szabad játéknak még nagy szerepe van, a belső világ tágassága fokozatosan nyílik meg a külvilág felé, és kezd lényegessé válni az azokkal való kommunikáció kiérlelése. Ekkor tanulnak meg a gyerekek írni, olvasni, és megismerik a matematikai alapműveleteket, valamint a mindennapjaikban használatos  geometriai és kombinatorikai alapfogalmakat, mely tudás a mindennapjaikba beépülő önálló tanulási célokhoz kapcsolódóan folyamatosan egyre szükségesebbé válik. Önmagukhoz és környezetükhöz érzékenyen, odafigyeléssel és empátiával fordulnak, megtanulják tiszteletben tartani, hogy társaik másmilyenek, akiket más dolgok is érdekelhetnek, másfajta megoldásaik is lehetnek. \emph{A szakasz végére biztonsággal meg tudnak maguknak tervezni egy néhány napos projektet.}\eject

\paragraph{Második szint, 9--14 év}

A gyerek testi és pszichoszociális fejlődése szempontjából egyaránt kiemelten fontos időszak következik. A korai serdülőkorban alakulnak ki a másodlagos nemi jellegek, melyek ütemükben mind a fiúk és lányok, mind az egyének között jelentős eltéréseket mutathatnak. Az egyéni tanulási tervek összeállításának ezért ebben az időszakban megnő a szerepe.

Az agyi struktúrák jelentős átalakulása is erre az időszakra tehető, a gondolkodás, a figyelem, az emlékezet területén is lehetnek ezen átalakulások miatt nehézségek, amit később, az átmeneti visszaesést követően az agy magasabb minőségű működése követi. Ekkor alapozhatja meg egy gyerek a tervezés, a saját tanulási célok komplexebb fogalmait. Aki ebbe a szakaszba lép, már hetekre előre ki tudja jelölni a saját feladatait. Három év alatt oda akar eljutni, hogy a szakasz végére önállóan képes legyen egy teljes trimeszternyi tanulási tervet összeállítani. Ehhez a hetek számát folyamatosan növelnie kell, miközben a problémák komplexitása is változik. A szülő még mindig megjelenik a célalkotásban, de egyre inkább átalakul a szerepe tanácsadóvá. A mentor az egyéni fejlődés mintázataira figyelve segíti a gyereket abban, hogy fejlettségi szintjének, aktuális testi, lelki változásainak megfelelő módon legyen terhelve.

\emph{A szakasz végén önállóan be tudja mutatni társainak egy pár hetes projekt eredményeit, és mind szövegértése, mind logikus gondolkodása és matematikai alapismeretei olyan szinten állnak, hogy elmélyült alkotó, kutató feladatokat vállalhat a következő szakaszban.} Biztonságos nyelvhasználata a sajátja mellett már egy másik nyelvben is elkezdődik, és értékrendjében önmaga és társai megismerésén túl a környezet és a világ fogalma is tágulni kezd.

\paragraph{Harmadik szint, 12--16 év}

A sajátként megélt tanulási célok eddigi alapozása ebben a szakaszban nyeri el mélyebb funkcióját. A tinédzserkor minden szempontból a gyerek kivirágzásának időpontja, melynek során a helyét kereső, befelé, saját változásaira fókuszáló gyerekből egy, a jövő felé nyitott kamasz válhat. A gondolkodási struktúrák mellett megnő a családon túli társas kapcsolatok szerepe, és ekkor fontos, hogy a családias jelleg, a biztonságos tanulás, a szülővel való konzultáció mint a gyerek számára egyaránt fontos érték, és nem mint kényszer jelenjen meg. Az érvelés, a hipotézis-alapú problémamegoldás, valamint a jelen fókuszált megélése ebben a korban alakul ki, ahogy annak a tudata is, hogy tetteinknek a jövőre nézve nagyobb mértékű következményei lehetnek. Kiemelten fontos ebben a szakaszban a különböző vélemények, információk, megoldási lehetőségek számba\-vétele, annak lehetősége, hogy a gyerek egyénileg ismerhesse fel az eltérő utak közötti különbségeket, és ha teheti, kérdőjelezzen meg akár tudományos hipotéziseket, vagy írjon újra művészeti, kulturális tartalmakat.
Ha a gyerek ezen határok megértésében szabadon fejlődik, akkor lehetősége lesz arra, hogy a megszerzett alapjait olyan kreatív irányokba fordíthassa, melyek saját élete és társadalmunk jövője formálásához egyaránt hasznosak lehetnek. \emph{Ebben az időszakban a saját tanulási célokat egy teljes trimeszterre vonatkozóan egyénileg határozza meg a gyerek, a szülő és a mentor ebben mint konzulens jelennek meg. Meghozza első komolyabb döntéseit arról, hogy milyen területekkel szeretne elmélyültebben foglalkozni, és ehhez kapcsolódóan választott érettségi tantárgyait is kijelöli a szakasz végére.} Előadásmódját, kutatási és alkotói munkáját felnőtt jegyek jellemzik.

\paragraph{Negyedik szint, 15--19 év} Ebben a szakaszban a gyerek fizikai
értelemben teljesen felnőtté válik. Érzelmi és szociális biztonságra azonban kiemelten szüksége van. A szerelem, a szexuális identitás erősödése, az önállósodásra való igény, az addikciók veszélye konfliktusokat szülhet, melyek kezelése különösen fontos ebben az időszakban. Az iskola elsődleges feladata ennek a biztonságos közegnek a megteremtése ebben a felnőttszerű időszakban. Ebben a mentor a szülővel együttműködve tudja segíteni a gyereket. A szakasz célja, hogy a gyerek képessé váljon a saját életét meghatározó, egy-két éves távlatokban mérhető felelős döntések meghozatalára, melyek akár sorsfordító jelentőségűek is lehetnek. Ezek a döntések  vonatkozhatnak egy emelt szintű érettségire való felkészülésre, egy nemzetközi egyetemre való felkészülésre, egy nagyobb komplexitású kutatási vagy alkotói projektre. Fontos azonban a gyerek támogatása abban is, hogy nem kell örök érvényű döntéseket hozni. Ekkorra már megtanult szakaszosan célokat állítani, és tudja, hogy bármikor lesz lehetősége az életben újratervezni. Önállóan készül az érettségire, tanárai segítik, hogy folyamatosan megtalálja a kihívást ebben. Saját céljai szűkebb környezetén túl könnyedén hatással lehetnek már a világra is.

\section{Kapcsolat a hazai és nemzetközi oktatási reformokkal}
\label{sec:kapcsolat_reformokkal}

A Budapest School programja a hazai és nemzetközi oktatási reformok kontextusában és a pszichológia, a szociálpszichológia, valamint a szervezetfejlesztés terén elvégzett kortárs kutatások tükrében válik könnyebben értelmezhetővé.

Magyarországról több iskola története, működése is nagy hatással volt ránk. A 90-es évektől induló alternatív iskolák világát mi a \emph{Rogers Személyközpontú Általános Iskola}, a \emph{Lauder Javne Iskola}, a \emph{Kincskereső Iskola} és a \emph{Gyermekek Háza} alapján ismertük meg. A megújuló középiskolák modelljének mi az \emph{Alternatív Közgazdasági Gimnáziumot} és a \emph{Közgazdasági Politechnikumot} tartjuk. Ezek az iskolák a személyközpontúság, a gyerekközpontúság hangsúlyozása mellett elkezdték a gyakorlatban alkalmazni a partnerség alapú kommunikációt, a differenciálás, a kooperatív technikák módszerét és egyes projektmódszertanokat. A \emph{Zöld Kakas} iskola programjának egyszerűsége és rendszerszintű szemlélete az egyik legnagyobb inspirációt adta számunka.

Programunk kidolgozásában nagy szerepe volt annak, hogy ezek az iskolák olyan szemléletmódbéli alapokat fektettek le, amelyek mára alapelvárásként fogalmazódnak meg a szülők oldaláról az iskolákkal szemben.

Gyakorlati tapasztalatokat a világ más részein is gyűjtöttünk. A 21. században a Budapest Schoolhoz hasonló kezdeményezések sorra indulnak a világban. Ezek egyes jegyei a Budapest School modelljével összhangban vannak:

A \emph{Wildflower School}\footnote{https://wildflowerschools.org/} mikroiskolák hálózatát működteti kisebb üzlethelyiségekben. A Budapest Schoolhoz hasonlóan célja, hogy falakat romboljon a gyerekek és a világ között: a magántanulás és az intézményes tanulás, a tanár és a tudós szerepe, valamint az iskola és környezete közötti határok elmosása az egyik fő üzenete.

Hasonlóan az otthon tanulás és az  \emph{unschooling} strukturált formáját keresi az amerikai Texasban alapított \emph{Acton Academy}\footnote{https://www.actonacademy.org/}, amely a szokratikus módszereket (azaz, hogy megbeszéljük közösen), a valós projekten keresztüli tanulást, és a gyakornokoskodáshoz hasonló munka közbeni tanulást (,,learning on the job'') teszi a megközelítésének középpontjába.

A \emph{High Tech High}\footnote{https://www.hightechhigh.org/} iskoláiban a gyerekek elsősorban projektmódszertan alapján tanulnak. A tanulási jogokban való egyenlőség mellett az egyéni célokra szabott tanulás, a világ alakulásához kapcsolódó tartalmi elemek, valamint az együttműködés-alapú tanulás is megjelenik pedagógiájukban a Budapest School által is alkalmazott jegyekből.

A \emph{School21}\footnote{https://www.school21.org.uk} brit iskola 21.~századi képességek fejlesztését tűzte ki célul. Ezért a prezentációs, előadói skillek kiemelt jelentőségűek. Az iskola egyensúlyt akar teremteni a tudásbéli (akadémiai), a szívbéli (személyiség és jóllét) és a kézzel fogható (problémamegoldó, alkotó) között. A Budapest School iskoláinak hasonló módon célja, hogy a tanulás három rétegét, a tudást, a gondolkodást és az alkotást folyamatos harmóniában tartsa.

A \emph{Khan Lab School}\footnote{https://khanlabschool.org/} a Montessori-módszert keveri az online tanulással. Kevert korosztályú csoportokban, személyre szabott módszerekkel segítik a képességfejlesztést és a projektalapú munkát. 
\newpage
\thispagestyle{empty}


\chapter{Tanulás megközelítése}
% approach to learning
\section{Pedagógiai és pszichológiai háttér}

Az iskolák gyakorlatias tapasztalata mellett a Budapest School számos elméletet is kiemelten fontosnak tart. 

Ezek közül is oktatási programjának központjában Carol Dweck fejlődésközpontú szemlélete \citep{growthmindset} áll. Emellett nagy hangsúlyt fektetünk az alábbi elméletek gyakorlati alkalmazására is:

Reformpedagógiai irányzatok elméletei, különös tekintettel:

\begin{itemize}

      \item
            Montessori-pedagógia (Maria Montessori),
      \item
            kritikai pedagógia (Paulo Freire)
      \item
            élménypedagógia (John Dewey)
      \item
            felfedeztető tanulás (Jerome Bruner)
      \item
            projektmódszer (William Kilpatrick)
      \item
            kooperatív tanulás (Spencer Kagan)
\end{itemize}

Pszichológia és szociálpszichológiai kutatások eredményei:

\begin{itemize}

      \item
            kognitív interakcionista tanuláselmélet (Jean Piaget)
      \item
            személyközpontú pszichológia (Carl Rogers)
      \item
            kommunikáció és konfliktuskezelés (Thomas Gordon)
      \item
            erőszakmentes kommunikáció (Marshall Rosenberg)
      \item
            pozitív pszichológia eredményei, különös tekintettel: flow-elmélet, kreativitáskutatások (Csík\-szent\-mihályi Mihály)
      \item
            érzelmi és társas intelligencia (Peter Salovey, John D. Mayer, Daniel Goleman)
      \item motivációkutatások, amiket jól összefoglal Daniel H. Pink műve \citep{pink2011drive}
      \item
            hősiesség pszichológiai alapjai (Phil Zimbardo)
      \item
            fejlődésfókuszú szemlélet (Carol Dweck)
\end{itemize}
\section{Pedagógia módszerek}
\label{sec:pedagogia_modszerek}

A Budapest School iskola tanárainak feladata, hogy mindig keressék azt a
módszert, azt a környezetet, ami az adott gyerekekkel, adott mikroiskolában a
leginkább működik. A módszer kiválaszatásakor mindig azt kell szem előtt
tartaniuk, hogy az a gyerekek egyéni céljaihoz, a közösség tanulási
lehetőségeihez a leginkább alkalmazkodjon. Nem tudjuk megmondani előre, hogy mikor milyen módszert érdemes választani, de
azt tudjuk, hogy mi alapján keressük a megfelelő technikákat. Vannak olyan módszerek, amelyek a
saját célok lehetőségeinek kitágítását és azok elérését nagyban támogatják.
Ezek alkalmazása javasolt a csoportmunkák és az egyéni gyakorlások alatt.

Azt is tudjuk,
hogy nem baj, ha nem elsőre találtuk meg a megfelelő módszert, hiszen a
próbálkozások során rengeteg új információt nyerünk, amelyek segítségével már
könnyebb megtalálni a valóban megfelelő megoldást. Az alábbiakban néhány a Budapest School számára meghatározó fontosságú módszert
emelünk ki.

\subsection{Projektmódszer}
Projektmódszert alkalmazó modulok során fő célünk, hogy a gyerekek aktívak és
kreatívak legyenek, és ezért a tevékenységek sokszínűségét helyezzük fókuszba.
Projektmódszer esetén is bátorítva van minden tanár, hogy a legváltozatosabb
módszertárral közelítse a gyerekeket, figyelve arra, hogy a tanár attitűdje, a
csoport dinamikája és az aktuális tevékenységek mit kívánnak.

\paragraph{A projektmunka folyamata}

\begin{description}
      \item[Téma, cél] Első lépés, hogy meghatározzuk a projekt témáját vagy
            célját. Ez jöhet a tanártól, a gyerekektől, vagy akár egy szülőtől
            is.
            Fontos,
            hogy a gyerekek a projekt témáját már önmagában értelmesnek,
            relevánsnak
            tartsák.

      \item [Ötletroham]  Egy-egy téma feldolgozását csoportalakítással és
            ötletrohammal kezdjük. Ennek célja, hogy a résztvevők bevonódjanak,
            illetve
            megmutassák, hogy nekik milyen elképzeléseik vannak az adott
            témáról,
            továbbá
            milyen produktummal, eredménnyel szeretnék zárni a folyamatot. A
            létrehozott
            produktumoknak csak a képzelet szabhat határt. Lehetnek videók,
            prezentációk,
            fotók, rapdalok, telefonos applikációk, rajzok, tablók, tudományos
            cikkek stb.

      \item [Kutatói kérdés] Ezek után úgynevezett kutatói kérdéseket teszünk
            fel,
            melyek meghatározzák a vizsgálat irányát. A kérdések feldolgozása a
            legváltozatosabb módon történhet. Az egyéni munkától kezdve, a
            forntális
            instruáláson vagy a kooperatív csoportmunkán keresztül egészen a
            dráma-
            és
            zenefoglalkozásokig minden hasznosítható a tanár, a csoport és a
            téma
            igényeihez mérten.
      \item [Elmélyült csoportmunka] A projekt azon szakasza, amikor a tervek,
            kutatások alapján az implementáción dolgozik a csapat.
      \item [Prezentáció] A létrehozott produktumokat bemutatására külön
            hangsúlyt
            kell fektetni. Ennek több módja is lehet: prezentációk,
            demonstrációk,
            plakátok, projektfesztiválok.
\end{description}

Az iskolai projektek célja egy fejlődésfókuszú tanár és gyerek számára mindig
kettős: egyrészt cél a téma feldolgozása, a produktum létrehozása, másrészt az
iskola fő célja, hogy a gyerekek, a csapatok mindig fejlesszék alkotó,
együttműködő, problémamegoldó képességüket. Ezért a projekt folyamatára való
reflektálás, visszajelzés ugyanolyan fontos, mint maga a cél elérése.

A munka során külön figyelmet kell fordítani arra, hogy mindent dokumentáljanak
a résztvevők. Lehetőleg online felületen.

\paragraph{Értékelés} A projekt során több értékelési pontot érdemes beépíteni.
A foglalkozások végén a résztvevők visszajeleznek a folyamatra, értékelik a
saját, a csoport és tanár munkáját. A folyamat végén az egész projektfolyamatot
értékelik, szintén kitérve a saját, a csoport és a tanár munkájára. A
produktumok, az eredmény értékelése csoport és egyéni szinten is megtörténik.

Értékelés fókuszáljon a folyamatra, és ne (csak) 
az eredményre, hogy fejlessze a fejlődésfókuszú gondolkodásmódot.\citep{growthmindset}


\subsection{Önszerveződő tanulási környezet}
A Sugata Mitra által kialakított módszertan (angolul Self Organizing Learning
Environment) lényege, hogy a tanárok arra
bátorítják a gyerekeket, hogy csoportban, az internet segítségével \emph{Nagy
Kérdéseket} válaszoljanak meg. A jó kérdés az, amire nem egyszerű a válasz, sőt
lehet, hogy nincs is rá egyfajta válasz. Cél, hogy a gyerekek maguk alakítsák
a folyamatot, formálják a kérdést, és találjanak válaszokat.

\begin{itemize}
      \item Tanár kialakítja a teret: körülbelül négy gyerekenként egy
            számítógép,
            amit körbe lehet ülni.
      \item Gyerekek maguk formálják a csoportjukat, sőt, még csoportot is
            válthatnak a munka során. Mozoghatnak, kérdezhetnek, ,,leshetnek''
            
            más
            csoportoktól.
      \item Körülbelül 30-45 perc után a csoportok prezentálják a kutatásuk
            eredményét.
\end{itemize}

\paragraph{A jó kérdések}
A Nagy Kérdésekekre nincs könnyű válasz. Ezek nyílt és nehéz kérdések, és
előfordulhat, hogy senki sem tudja még rájuk a választ. A cél, hogy mély és hosszú
beszélgetéseket generáljanak. Ezek azok a kérdések, amikre érdemes
nagyobb elméleteket állítani, amiket jobb csoportban megvitatni, amikről
érvelni lehet és kritikusan gondolkodni.

A jó kérdések több témát, területet (tantárgyat) kapcsolnak össze: \emph{,,Mi a
hangya''} kérdés például nem érint annyi különböző területet, mint a ,\emph{,Mi
történne a Földdel, ha minden hangya eltűnne''}.

\paragraph{Fegyelmezés nélkül}
A tanár feladata a folyamat során meghatározni a Nagy Kérdést, és tartani a
kereteket. A cél, hogy a gyerekek maguk szervezzék saját munkájukat, így
minimális beavatkozás javasolt a tanár részéről. Kezdetben, gyakorló csoportoknál, a tanárnak
sokszor kell emlékeztetnie magát, hogy idővel kialakul a rend. \emph{,,Bízz a
folyamatban!''
}
Amikor a tanár úgy látja, hogy nem megy a munka, akkor csak finoman emlékezteti
a csoportokat, hogy lassan jön a prezentáció ideje. Amikor valaki a
csoportjáról panaszkodik, akkor elmondhatja, hogy szabad csoportot váltani. Ha
valaki zavarja a többieket, akkor megfigyelheti, hogy a gyerekek tudnak-e már
konfliktust feloldani. Ha valaki nem vesz részt a munkában, akkor gondolkozhat
olyan kérdésen, ami az éppen demotivált gyerekeket is bevonzza.

\subsection{Megfontolt gyakorlás}
Ahogy \citep{ericsson2016peak} is kimutatja, bárki tudja valamennyi készségét,
képességét fejleszteni, ha megtervezetten, megfontoltan gyakorolja. A Budapest
School a hagyományosan készségtárgyként számon tartott ének, rajz, testnevelés
és technika témákon kívül nagyon sok mindent kezel készségként: írásbeli
érettségi vizsgát tenni magyarból, geopolitikai elemzéseket végezni,
hiperbolikus függvényekkel egyenletet megoldani épp annyira értelmezhetőek
készségként, mint a domináns csoporttagokkal való együttműködés, vagy az, hogy
egy stresszhelyzetben lenyugtassuk önmagunkat.

A készség-, és képességfejlesztés legjobb eszköze a megtervezett gyakorlás:
a fejlődés érdekében okosan gyakorlunk. A megfontolt gyakorlás jellemzője, hogy

\begin{description}
      \item[Világos és specifikus cél] Fontos, hogy tudjuk, mit gyakorlunk, mit
            akarunk elérni.  Lehetőleg a cél legyen mérhető és mindenképp
            realisztikus,
            elérhető.
      \item[Fókusz] Gyakorlás során egy dologra érdemes figyelni
      \item[Konfortzónán kivül kell lenni] Az edzőnek, tanárnak, trénernek
            néha
            érdemes a tanulót kicsit ,,nyomni''. Emlékeztetni, hogy mindig
            lehet
            kicsit
            többet elérni.
      \item[Folyamatos visszajelzés] Nagyon gyakran kap a tanuló visszajelzést,
            mindig tudja, hogy mikor és miben fejlődött.
\end{description}

\section{Jutalmazás és értékelés}
\label{jutalmazas}

Fontos alapelv, hogy minél inkább a folyamatot, cselekvést jutalmazzuk, értékeljünk. Tehát arra fókuszáljunk, hogy \emph{mit csinált} a gyerek, és ne arra, hogy \emph{milyen} a gyerek. Azokra a viselkedésmintákra adjunk megerősítő visszajelzést, amit látni szeretnénk a gyerekben később.  Lehetőleg kerüljük a statikus jellemzőkre, személyiségjegyekre vonatkozó értékelést. Így nem azt mondjuk, hogy \emph{,,mindig jó vagy matekból''}, hanem, hogy \emph{,,kitartóan és odafigyelve oldottad meg a feladatot''}. A Budapest School tanárai ezért nem szeretik, ha bármikor elhangzik, hogy \emph{,,tehetséges gyerek vagy''} vagy \emph{,,jaj, de cuki!''}.

Álljon itt néhány példa cselekvésre vontakozó visszajelzésre.

\begin{itemize}

      \item
            Fantasztikus, ma egy nagy kihívást választottál!
      \item
            Bátran vállaltad a rizikót!
      \item
            Nagyon jó! Tényleg sokat próbálgattad.
      \item
            Kitartóan csináltad, erre nagyon büszke vagyok!
      \item
            De jó, valami újat próbáltál ki ma!
      \item
            Köszönöm, hogy ma valakinek segítettél.
      \item
            Nagyon nagy öröm látni a haladásodat!
      \item
            Ne feledd, mindannyian tudunk a hibáinkból tanulni. Örüljünk annak, hogy ma valamit jobban tudunk, mint előtte.
      \item
            Wow, egy nehéz feladatot oldottál meg!
      \item
            Szép munka! Kipróbáltál egy másik módszert.
\end{itemize}

Az iskolában történő folyamatos visszajelzés a mindennapok része. Ezt írja le \aref{sec:ertekeles}. fejezet.

\section{A kiemelt figyelmet igénylő gyerekek}
\label{sec:kiemelt_figyelem}

A kiemelt figyelmet igénylő tanulók támogatásához  elsődlegesen a tanulásszervezőknek, mentoroknak és a modulvezetőknek kell differenciáltan, sokszínűen, türelemme és személyreszabottan megközelíteni a gyerekeket.

\begin{itemize}
      \item Mentoroknak ismerniük kell a gyerek sajátosságait. A szülőnek és a mentornak őszintén, egymást támogatva kell a gyerek egyéni támogatására felkészülni. Ehhez sokszor külső diagnózisra, szakember bevonására és a felnőttek közötti nehéz beszélgetésekre van szükség.
      \item Mentoroknak meg kell osztani a gyerekek sajátos igényeit a többi tanárral, a modulvezetőkkel, hogy ők is fel tudjanak készülni a gyerekek személyreszabott támogatására.
      \item A tanulásszervezés, a modulok, a foglalkozások, szóval minden, ami az iskolában történik  differenciálnak, egyéniesítettnek kell lennie, hogy az eltérő képességekkel bíró gyerekek is tudjanak együtt tanulni. Ahogy egy nagycsaládban is is figyelünk a különböző gyerekek eltérő igényeire, képességszintjeire.
      \item A szemléltetésnek, tevékenységeknek sokoldalúnak kell lennie, sokféle feladatot, speciális eszközöket kell használni. A gyerekek élménye változatos kell hogy legyen.
      \item El kell kérni a szülőktől a diagnózist, beszélni kell róla. Külső szakemberekkel konzultálni kell és a diagnózísban szereplő javaslatokat be kell építeni a mindennapok tervezésébe.
      \item Az egyéni haladási ütem biztosítására egyéni fejlesztési és tanulási tervet kell készíteni.
      \item A tanároknak együtt kell működni a gyermek/tanuló fejlesztésében résztvevő szakemberekkel.
\end{itemize}

Ez sok feladat, ami a tanárokra, a csoportra nagy terhet tud róni. Ezért a mikroiskolákban a kiemelt figyelmet igénylő gyerekek arányát körülbelül 20\% alatt kell tartani, hogy a támogatásukra legyen egyéni figyelem.

A fenti listából a legfontosabb elem: a tanár, gyerek, szülő hármas mellé be kell általában hívni külső, a gyerek igényeit jól értő szakembert támogatónak. A Budapest School iskola csak akkor tud segíteni, ha minden fél tudatosan áll hozzá a kiemelt figyelem szükségeléhez.



% architecture of learning
\chapter{Tanulás szervezése}
\section{A tanulás rendszere és folyamata}

\input{chapters/pedprogram/regi_struktura_bevezeto}
\paragraph{A tanulás rendszerszemléletű megközelítése}
Az oktatás tartalmának előzetes szabályozása helyett a Budapest School a tanulás módjára helyezi a hangsúlyt. Az iskola alapelve, hogy integratív módon folyamatosan keresse és fejlessze a pedagógiai, pszichológiai és szervezetfejlesztési módszereket, amelyek korszerű módon tudják segíteni a tanulás tanulását, az egyéni és csoportos fejlődést, a konfliktusok feloldását.

A tanulás tartalmát tekintve a Budapest School saját alternatív kerettantervére támaszkodik, amely a tanulás rendszerét, annak folyamatát szabályozza. E dokumentum alapján az állami kerettanterv tanulási eredményein történő végighaladás mellett a Budapest School nagy hangsúlyt fektet a gyerekek saját tanulási céljaira és a célállítás módjára.


\section{A mikroiskolák, a Budapest School összevont osztályai}
\label{sec:mikroiskola}

A Budapest School iskolában összevont osztályokban, azaz kevert korosztályú és
maximum 6 évfolyamszintű közösségben tanulnak a gyerekek. Az összevont
osztályokat a Budapest School kerettanterv \emph{mikroiskolának} hívja, ezzel
is
hangsúlyossá téve ezek egyedi jellemzőit:
\begin{itemize}
      \item A mikroiskolákban  tanulásszervező  \emph{tanárcsapatok} vezetik a
            közösségeket. 
      \item A mikroiskolák saját szabályaik, normarendszerük, szokásaik,
            kultúrájuk alakul ki.
      \item A mikroiskolák maguk alakítják saját órarendjüket, modulkínálatukat
            és ebben nem kell más mikroiskolákhoz igazodniuk.
\end{itemize}


\paragraph{A mikroiskolákat tanulásszervezők vezetik.}

A tanulóközösség fontos célja, hogy biztonságot, támogatást nyújtson, és
\emph{így} segítse a közösség tagjainak a minőségi tanulását.

A mikroiskolát az igazgató által kinevezett tanulásszervezők irányítják.
Ők felelnek a tanulás tartalmáért, a modulok meghirdetéséért és a
tanulási eredmények nyomonkövetéséért. Ők döntenek a jelentkező gyerekek
kiválasztásáról és a mikroiskola mint közösség összetételéről.

A különböző tanárszerepeket, a tanulásszervezők, mentorok és modulvezetők kapcsolódását \aref{sec:tanarok}. fejezet mutatja be.

\paragraph{Mikroiskola egy nagy csoport.}

Egy mikroiskola minimális létszáma 6 maximális létszáma 60 fő. Minden
mikroiskolának megfelelő számú olyan tanulásszervezővel kell
rendelkeznie, aki mentortanárként is végzi munkáját.

\paragraph{A mikroiskolák korkülönbségei állandók, a gyerekek együtt nőnek.}
Egy mikroiskolában fő szabály szerint legfeljebb 6 (egymást követő)
évfolyamszintnek megfelelő korosztály tanul együtt. A mikroiskola
korosztályait, és az induláskor meglévő évfolyamokszinteket a jelentkező ill. a
felvett gyerekek életkora alapján határozza meg az alapító.

A mikroiskolák korhatára, mint az összevont osztályok korhatárai, a gyerekekkel
változnak. Ettől eltérően a korhatárokat tágítani és szűkíteni évente egyszer
lehet, és
erről az iskola értesíti a szülőket minden tanévkezdést megelőző február 15-ig.
(pl. Amennyiben ezzel nem sérül a maximum 6 évfolyam elve, a korhatárok
tágíthatóak, vagy ha a legalsó vagy legfelső korosztályba tartozó gyerekek
elmentek, a mikroiskola dönthet a korhatárok szűkítéséről.)
A Budapest School mikroiskolái a 12. évfolyamig tartanak, kivéve ha a
mikroiskola valamilyen okból megszűnik.

\paragraph{A mikroiskola állandó, a gyerekek és tanárok jöhetnek és mehetnek.}
A Budapest School mikroiskolái úgy működnek, mint egy összevont osztály. A
tanulásszervező tanárok vagy gyerekek kilépése a mikroiskola fennállását nem
érinti, helyettük a mikroiskola új tanulásszervező tanárt és
gyereket vehet fel.  A mikroiskola létrehozásakor arra kell törekedni, hogy
olyan
gyerekek tanuljanak együtt, akik támogatni tudják egymást a tanulásban. A
gyerekek a mikroiskola tagjai addig, amíg ott jól tudnak tanulni, és a közösség
és a gyerek kapcsolat gyümölcsöző.

\paragraph{A mikroiskoláknak saját fókuszuk, helyszínük, stílusuk alakulhat
      ki.}
A mikroiskolák nemcsak abban térnek el egymástól, hogy kevert korcsoportban,
más korosztályú gyerekek, más érdeklődések mentén, és ily módon más célokat
követve tanulnak, hanem területileg, regionálisan is eltérőek lehetnek.

A mikroiskola-rendszerben lehetőség van arra, hogy adott tanulási környezetben
úgy váltakozhassanak a hangsúlyok a csoport és az egyén érdeklődését követve,
hogy közben
fennmaradjon a tanulási egyensúly a tantárgyak között.

Van olyan mikroiskola, amely a fejlesztési célok eléréséhez és az egyéni célok
mentén már 6 éves gyerekek tanulásánál a robotika eszközeit használja, másutt
drámafoglalkozásokkal fejlesztik 12 éves gyerekek a szövegértésüket és
éntudatukat.

\paragraph{A mikroiskolákban a tanulók nagymértékben befolyásolják, hogy mit és
      hogyan
      tanulnak és alkotnak.}
A mikroiskolákban (a tanulásszervezők által meghatározott kereteken belül)
megfér egymással több, különböző egyéni céllal rendelkező gyerek addig amíg a
tanulásszervezők minden gyerek számára biztosítani tudják a kerettantervben
megfogalmazott tanulási eredmények eléréset.

A tanulásszervezők feladata és felelőssége, hogy olyan közösségeket
válogassannak össze és építsenek, amelyek kellően diverzek, és mégis jól
működnek. A közösségnek a gyerekek igényeit és a kerettanterv céljait egyaránt
ki kell elégíteni.

A tanulásszervezők választási lehetőségeket kínálnak, (azaz modulokat dolgoznak
ki), amikből a gyerekek (a mentoruk és szüleik segítségével) a saját céljaikat,
érdeklődésüket leginkább támogató egyéni tanulási tervet és utat alkotnak.

Eltérhet, hogy egy-egy gyerek mit tanul, ezért az is, hogy mikor és hogyan
sajátítja el a szükséges ismereteket: egy közösségben megfér a központi
felvételire fókuszáló 11 éves gyerek, és az is, aki ekkor inkább a Minecraft
programozásában akar elmélyedni, ezért más képességek fejlesztésével lassabban
halad.

\paragraph{Kisebb csoportokban tanulhatnak a gyerekek.}
\label{sec:csoportbontasok}

A mikroiskolákban a közösséget kisebb csoportokra bonthatjuk, ha a
tanulásszervezés ezáltal hatékonyabb. Egyes moduloknál a gyerekek egy-egy
projektre szerveződnek, ilyenkor általában az eltérő képességű és életkorú
gyerekek is kitűnően tudnak együtt dolgozni. Más modulok esetén a csoportokat a
tanár képességszint alapján hozza létre. Ilyen csoportok lehetnek a másodfokú
egyenletek megoldóképletét megismerő csoport, az írni tanulók csoportja, vagy
egy angol nyelvű újság szerkesztésére és megírására alakult modul, ahol a
nyelvismeretnek és a szövegalkotási képességnek már egy olyan szintjén kell
állni, hogy a projektnek jól mérhető kimenete lehessen.

\paragraph{A mikroiskolák diverz, integratív közösségek.}
A Budapest School mikroiskolák társadalmi, kulturális és gazdasági értelemben
is diverzek és egyik fő céljuknak tekintik az integrációt addig, mindaddig amíg
az a közösség céljait szolgálja.

\paragraph{A mikroiskolák tanuló közösségek.}
A Budapest School célja, hogy a mikroiskolákban történő tanulás mind a gyerek,
mind a tanár, mind a szülő számára jól átlátható, követhető legyen, és gyerek
és a közösség folyamatosan fejlődjön.
A Budapest School kiemelt elve, hogy ,mindig, minden módszer, folyamat
fejleszthető, ezért a tanárok feladata, lehetősége, hogy az aktuális helyzethez
illő legmegfelelőbb módszert válasszák meg a gyerekek tanulásának segítéséhez.

\paragraph{Mikroiskolát a fenntartó indít és addig él, amíg szolgálja a
      gyerekek tanulását.}
A mikroiskolát a fenntartó indítja. Meghatározza, hogy milyen telephelyen,
tagintézményben, milyen korhatárokkal és létszámokkal
induljon a mikroiskola,. A fenntartó feladata a szükséges épületet, eszközöket
biztosítania. A
mikroiskola alapításához legalább 6 gyerek és egy tanulásszervező (aki
értelemszerűen mentortanár)
szükséges.

A mikroiskola a tanárokon és a gyerekeken is túlmutató közösség, amely akkor is
tovább működik, ha egy tanár vagy gyerek távozik. Tanulásszervező illetve
gyerek távozása esetén a mikroiskola  új tanulásszervező tanárt és gyereket
vesz fel, mindaddig, amíg a mikroiskola számára meghatározott maximális
létszámot el nem érik.

A mikroiskola abban az esetekben megszűnik meg, ha összeolvad egy másik
mikroiskolával vagy
ha a mikroiskola létszáma 6 gyerek és egy mentortanár alá csökken.

\paragraph{Gyerek csatlakozása a mikroiskolához.}
Arról, hogy egy gyerek csatlakozhat-e egy mikroiskolá közösségéhez, a
tanulásszervező
tanárok döntenek, figyelembe véve a gyerek életkorát, a közösségben való
eligazodását, érdeklődését, egyéni fejlődési igényét. Elv egyszerű: minden
gyereket fogadjon be a közösség, amitől a közösség jobban tudja támogatni az
egyének tanulási céljait.\footnote{A mikroiskolához gyerek akkor csatlakozhat,
      ha ő már másik mikroiskola, így az iskola tanulója, vagy az iskola
      igazgatója a
      gyerek felvétele mellett döntött.}

\footnote{Átjárás mikroiskolák között.}
Egy gyerek akkor válthat a Budapest School egyes mikroiskolái között,
amennyiben a fogadó mikroiskola őt elfogadja. Ilyenkor új mentortanárt kell
számára kijelölni.

\section{Moduláris tanmenet és a tanulási eredmények}

\subsection{Modulok -- a tanulásszervezés alapegységei}
\label{sec:modulok}

A \emph{modulok} a tanulásszervezés \emph{alapegységei}: olyan foglalkozások megtervezett sorozata, amelyek során egy meghatározott időn belül a gyerekek valamely képességüket fejlesztik, valamilyen ismeretet elsajátítanak, vagy valamilyen produktumot létrehoznak. A modulok célja sokféle lehet, de kötelező elvárás, hogy a résztvevők a portfóliójukba bejegyzésre érdemes eredményt hozzanak létre, vagyis hogy legyen egyértelmű célja.

A mindennapi tanulás a modulok elvégzésén keresztül történik, ezzel biztosítva, hogy rugalmas keretek között, pontosan megfogalmazott célok mentén, a gyerekek számára érthető, átlátható és sajátnak megélt tartalommal történjen a tanulás.

A tanulási modulokat, vagyis a tanulás tartalmának és formájának alapegységét a tanulásszervezők három kötelező összetevőből állítják össze:

\begin{enumerate}
      \item
            a kerettanterv tantárgyainak tartalmából,
      \item
            a gyerekek, tanárok érdeklődéséből, aktuális tudásából,
      \item
            és a környezetük és a világ aktuális kihívásaiból.
\end{enumerate}

A három komponensből a legelső a legstatikusabb, hiszen a kerettanterv -- összhangban a NAT-al -- meghatározza a tantárgyakat és azok tartalmát, valamint azt, hogy milyen lehetséges eredmények elérését várjuk az ezekben való fejlődéstől. Az egyes modulokban ezek személyre, illetve a csoport igényeire szabhatóak, hiszen az elérhető eredményeket különféle gyakorlati és elméleti tanulási módszerekkel el lehet érni.

A gyerekek és tanárok érdeklődése -- ami a sajátként megélt cél és a minél nagyobb fokú bevonódás alapfeltétele -- alakítja ki a modulok témáját, a projekteket, és a gyerekek egyéni tanulási idejét is meghatározhatja.

Mindemellett a kerettanterv szándéka, hogy a tanárok, gyerekek reagáljanak a környezetükre, a világ aktuális kihívásaira, kérdéseire. A kerettanterv meghatározza például, hogy a gyerek ,,\emph{táblázatkezelővel feladatot old meg}''. Az azonban, hogy a gyerekek milyen táblázatokat szerkesztenek szívesen, csak a modulok összeállításakor és a modulok elvégzése során derül ki. Nagyon hasonló táblázatkezelési képességeket lehet fejleszteni, ha valaki az önvezető autóktól várt csökkenő baleseti halálozási arányról, vagy ha a vegánok számának és a GDP-növekedés alakulásának arányáról készít táblázatot.

A moduláris rendszer fő célja, hogy egyszerre képes legyen alkalmazkodni a menet közben felmerülő tanulási igényekhez, adjon átlátható struktúrát a tanulásnak, és hogy a mikroiskola minél rugalmasabban tudja támogatni\break a tanulást, úgy, hogy a saját, a közösségi és a társadalmi célok harmóniába kerülhessenek.

Ez is mutatja, hogy bár közösek a kereteink, végtelen az elképzelhető modulok (a tanulási utak építőkövei, és így a különböző tanulási utak) száma. Ezért tartja fontosabbnak a kerettanterv annak meghatározását, hogy hogyan kell a modulokat létrehozni, mint azt, hogy a modulokat tételesen felsorolja.

Modulok során a gyerekek tudnak

\begin{itemize}
      \item produktum létrehozására szerveződő projektben részt venni;

      \item felfedezni, feltalálni, kutatni, vizsgálni, azaz kérdésekre választ keresni;

      \item egy jelenséget több nézőpontból megismerni;

      \item valamely képességüket, készségüket fejleszteni;

      \item adott vizsgára gyakorló feladatokkal felkészülni;

      \item közösségi programokban részt venni;

      \item az önismeretükkel, a tudatosságukkal, a testi-lelki jóllétükkel foglalkozni.
\end{itemize}

\subsubsection{A modulok meghirdetése}
\label{sec:modulok_meghirdetese}
A modulok kiválasztása, felkínálása a tanulásszervezők feladata, hiszen ők figyelnek és reagálnak a gyerekek, szülők céljaira és igényeire. A meghirdetett modulokból áll össze a tanulás trimeszterenkénti tanulási rendje.

A tanulásszervezők az egyes modulok tematikáját, azok hosszát és feladatát a gyerekek tanulási céljainak megismerését követően és a kerettantervben meghatározott tantárgyi tanulási eredményeket figyelembe véve határozzák meg.

A nem kötelező modulokba való csatlakozásról a mentor, a szülő és a gyerek közösen dönt, mindig szem előtt tartva, hogy folyamatos előrelépés legyen a már elért egyéni és tantárgyi eredményekben is. Egy modul megkezdésének lehet feltétele egy korábbi modul elvégzése, a gyerek képességszintje, a jelentkezők száma, és lehet egyedüli feltétele a gyerekek érdeklődése.

Egy modulvezető különféle tematikájú modulokat tarthat függően attól, hogy a saját célok, a tantárgyi eredmények mit kívánnak, és a tanulásszervezők, valamint a modulvezetők kapacitása mit enged.

Amikor egy gyerek moduljai befejeződnek, és újat vesz fel, a tanulásszervező feladata a gyereket segíteni abban, hogy az érdeklődési körének, tanulási céljainak, és a soron következő, még el nem ért tantárgyi eredményekben való fejlődéshez megfelelő modulok közül választhasson.

A tanulásszervezők feladata a tantárgyi eredményelvárások nyomon követése is. A modulok kidolgozáshoz és azok megtartásához külsős szakembereket is meghívhatnak, azonban ilyenkor is ők felelnek azért, hogy a modulokkal elérni kívánt tanulási célok teljesüljenek.

\subsubsection{A modulok formátuma}

Egy-egy modul hossza és a modulhoz kapcsolódó foglalkozások száma és gyakorisága változó: egy alkalomtól legfeljebb egy teljesen trimeszteren keresztül tarthat. A modul végén azt a tanulásszervező és a gyerek(ek) lezárják, értékelik és az elért eredményeket rögzítik a (tanulási) portfólióban. Egy modul folytatásaként a következő trimeszterben új modult lehet meghirdetni.

A modulok nemcsak témájukban, céljaikban, időtartamukban, hanem módszertanukban, folyamataikban is különbözhetnek: bizonyos modulokban a felfedeztető (inquiry based) módszer, másokban az ismétlő (repetitív) gyakorlás a célravezető. Így mindig a modul céljához, a tanárok és a gyerekek képességeihez és igényeihez választható a legjobb módszer. Modulonként változhat, hogy a folyamatot a gyerekek vagy a tanárok befolyásolják-e, és milyen mértékben. Két példa az eltérésre:

\begin{enumerate}
      \item Egy digitális kézműves modul célja, hogy építsünk valamit, ami programozható. Annak kitalálása, hogy mit és hogyan építünk, a gyerekek feladata. Itt a modul vezetője csak támogatja a tanulás folyamatát, azaz \emph{facilitál}.

      \item Egy „\emph{A vizuális kommunikáció fejlődése a XX. század második felében}'' modul esetén a tanár előre felépíti a tanmenetet, pl. hogy mely alkotók munkásságát, alkotásokat fogja bemutni, és ezeket a gyerekekkel sorban végigveszi. Ilyenkor is bővülhet azonban a tematika a gyerekek érdeklődése, felvetései mentén.

\end{enumerate}

\subsubsection{A modulok helyszíne}

A tanulás az egyes mikroiskolák helyszínén, egy másik Budapest School mikroiskolában, a tanár által kiválasztott külső helyszínen, vagy akár online, virtuális térben történik. A tanulásra úgy tekintünk, mint az élethez szorosan kapcsolódó holisztikus fejlődési igényre, melynek jegyében az elsődleges szocializációs tértől és formától, a szülői, családi környezettől sem akarjuk a tanulást leválasztani. Az élethosszig tartó tanulás jegyében a tanulás tere az iskolai időszak után és az iskola terein kívül is folytatódik.

A gyerekek több ok miatt is tanulnak az iskolán kívül:

\begin{enumerate}
      \item Modulok vagy modulok foglalkozásai szervezhetők külső helyszínekre, úgymint múzeumokba, erdei iskolákba, parkokba, vállalatokhoz, vagy tölthetik az idejüket „kint a társadalomban''.

      \item Amennyiben ez saját céljuk elérését nem veszélyezteti, és a folyamatos fejlődés biztosított, a mentoruk tudomásával a gyerekek az önirányított tanulás elvére figyelemmel a mikroiskolán kívüli egyéb helyszínen is elvégezhetnek egy modult.
\end{enumerate}

A modul lezárásaként a gyerekek és modulvezetők visszajelzést adnak egymásnak, aminek része, hogy megosztják saját élményeiket, reflektálnak a közös időre, összegyűjtik és értékelik az elért eredményt, és kitérnek az esetleges fejlődési lehetőségekre.

\subsection{Tanulási eredmények -- a formális tanulás alapegységei}
\label{sec:tanulasi_eredmenyek}
A kerettanterv három interdiszciplináris tantárgyat jelöl meg, azok témaköreit, tartalmát és követelményeit \emph{tanulási eredmények} listájaként adja meg, ezzel igazodva az Nkt.~5.~§ (5) pontjához. Tanulási eredmény (learning outcome) lehet a kerettanterv szellemében minden olyan tudás, képesség, kompetencia, attitűd, amit a gyerek egy tanulási folyamat során elsajátított és/vagy ezt demonstrálni tudja. Az eredmény eléréséhez vezető út a modulokon keresztül történik, és a tanulás folyamata történhet az iskolában vagy azon kívül, lehet formális, non-formális vagy informális.

A tanulási eredmények több funkciót látnak el a kerettantervben.

\begin{itemize}

      \item A kerettanterv évfolyamonként meghatározza az adott tantárgy teljesítéséhez elérendő tanulási eredményeket. Egy gyerek akkor  léphet egy tantárgyból évfolyamszintet, ha a tantárgyhoz tartozó követelményeket teljesítette.

      \item A tanulási eredmények a modulok (és így a mindennapokban szervezett foglalkozások, órák stb.) építőelemei. Egy-egy modul célját a  tanulásszervezők az elérendő tanulási eredmények	összeválogatásával és saját célokkal, érdeklődéssel való	kiegészítésével adják meg, figyelembe véve az életkori  sajátosságok, az egymásra épülés és az átjárhatóság  követelményeit.
      \item A tanulási eredmények megfeleltethetőek a miniszter által kiadott kerettantervek tantárgyai (és így a kötelező érettségi tárgyai )  és a NAT műveltségi területeivel, ami biztosítja, hogy a Budapest  School tanulója más rendszerben működő iskolába is illeszkedik.  Vagyis a tanulási eredmények a Budapest School saját interdiszciplináris tantárgyi elvárásain túl, az azokon belüli halmazt képző diszciplináris bontásban is követhetőek, így az	elért eredmények alapján mind a Budapest School tantárgyi  struktúrájával, mind (pl. egy esetleges iskolaváltás esetén) a	miniszter által kiadott kerettantervek tantárgystruktúrájával megfeleltethetőek.
\end{itemize}

\subsection{Modulok és tanulási eredmények}
\label{sec:modulok_es_tanulasi_eredmenyek}
A gyerekek egyik feladata az iskolában, hogy tanulási eredményeket érjenek el. Ezt megtehetik a modulok elvégzésével, vagy más tanulási helyzetekben. A tanulási eredményeket a portfólióban rögzítik. A mentor feladata, hogy folyamatosan kövesse, hogy megfelelő haladás történik-e a portfólióban a tanulási eredmények és a saját célok tekintetében. Az évfolyamszintlépés a portfólióban összegyűlt tanulási eredmények alapján történhet meg.

A modul kecsegtet a gyerekek haladásához releváns tanulási eredményekkel, a gyerekek által meghatározott saját célokkal és olyan kimenettel, amely a portfólióban rögzíthető, legyen az egy alkotás, az elért fizikai vagy szellemi eredmény dokumentációja, vagy egy értékelő visszajelzés. A modulok tehát tartalmaznak tanulási eredményeket, az önálló gondolkodás, szabad alkotás lehetőségét és teret engednek az alkotásra, létrehozásra.

\paragraph{A modulok különféle tanulási eredmények elérését teszik elérhetővé}

Modulok tervezésekor és összeállításakor a tanulásszervezők a modulvezetővel közösen határozzák meg a modul céljait, de azok meghirdetéséért mindig a tanulásszervezők felelnek. A célok között fel kell sorolni, hogy milyen tanulási eredmények elérését várhatják el a gyerekek a modulon való részvételtől.

Például a 6-8 éves gyerekek számára megtervezett ,,\emph{3d nyomtató használata}'' modul során azon kívül, hogy megismerik a 3d nyomtatás folyamatát, a modul célja, hogy a gyerekek számára elérhetővé tegye a ,,\emph{A kockát, téglatestet, gömböt felismeri, és képes létrehozni egyszerű módszerekkel. Ismeri ezeknek a testeknek a jellemzőit.}'' (STEM tantárgy, Matematika tématerület, 3--4 évfolyam) tanulási eredményt is.

Lehetőség van egy modul esetében több tantárgyból való tanulási eredmény kiválasztására, ezzel biztosítva az interdiszciplinaritást, valamint a Budapest School tantárgyi fejlesztési céljaihoz való integrált kapcsolódást.

A tanulási eredmények egy időbeni egymásra épülést feltételeznek, melyben azonban van lehetőség előre- és hátrafele is lépni. Előre, amennyiben a modul meghirdetésekor az arra jelentkező gyerekcsoportnál a megfelelő előkészítés megtörtént, hátra, amennyiben ezt ismétlés/felzárkóztatás jelleggel szükségesnek ítéli a mentor vagy a modult szervező, vezető. Vagyis akkor foglalkozzon egy gyerek a 10~000-es számkörrel, ha a 100-as számkört már begyakorolta. Az egymásra épülésért a modult meghirdető tanulásszervező felel. A példát folytatva a 3d nyomtató használata modul lehetővé teszi, hogy a gyerek elérje a következő eredményeket is: \emph{,,Ismeri a számítógép
      részeinek és perifériáinak funkcióit, azokat önállóan használja.''}
(Harmónia, Informatika, 5--6 évfolyam), és  \emph{,,Használati utasításokat
      értő módon olvas és tart be.''} (Harmónia, Életvitel, 3--4)

\paragraph{Új tanulási eredmények}

A gyerekek olyan tanulási eredményt is elérhetnek, ami a modulok céljai között eredetileg nem volt megadva, mert

\begin{itemize}
      \item lehetőségük van egyénileg is tanulni;

      \item tanulási eredményekkel járnak a projektek, az iskolai lét, a közösségi élet és még számos informális és non-formális tanulási helyzet;

      \item egy modul során is alakulhatnak előre nem tervezett helyzetek, amik hozzásegíthetik a gyerekeket tanulási eredmények eléréséhez.
\end{itemize}

Az újonnan létrejövő tanulási eredmények is bekerülnek a portfólióba.

\paragraph{Tanulási eredmények dokumentációja}

Minden modul dokumentálásra kerül, hogy annak célja, elért eredményei nyilvánosak legyenek a Budapest School valamennyi mikroiskolája számára, és ha szükséges, újra meg lehessen hirdetni. A tanulási eredmények egy, a modulhoz kapcsolódó terv-tény összehasonlítás alapján kerülnek meghatározásra. Az elért eredmények újra elérhetőek, amennyiben a folyamatos fejlődés biztosítva van.

\paragraph{Egységes modulok egyedi alkalmazása}

Egy modul elvégzésével egy-egy gyerek más tanulási eredményt is elérhet.

\begin{itemize}
      \item
            Működhet a differenciálás, tehát nem minden gyerek ugyanazt és ugyanúgy csinálja a foglalkozásokon. Egy modulban tud együtt	tanulni az a gyerek, aki még ,,\emph{Ismeri az írott és nyomtatott  betűket''} eredményért dolgozik, és az, aki ,,\emph{Jelöli helyesen a j	hangot 30--40 begyakorolt szóban''.}
      \item
            A modulnak része lehet testreszabható sáv. Például egy tudományos kísérletező modulban néhány gyerek a rövid távú memória és a	fáradtság kapcsolatáról kutat, a másik csoport az esőzés és a	közlekedési dugók kialakulása közti kapcsolatot vizsgálja. Minden  gyerek elérheti a ,,\emph{Valós folyamatokat képes elemezni a folyamathoz tartozó függvény grafikonja alapján.}'' (forrás, STEM), de a ,,\emph{Környezettudatos közlekedésszemlélet.}'' (forrás, Harmónia)	eredményt is elérheti.
      \item
            Egy-egy gyerek saját tanulási célja érdekében extra lépéseket tehet, és olyan eredményeket is el tud érni, amit mások nem.	Például egy modul végén önálló prezentációt, saját kutatási  tervet, vagy egy kész működő modellt alkothat.
\end{itemize}

\subsubsection{Kötelező tanulási eredmények}
\label{sec:kotelezo_tanulasi_eredmenyek}
A kerettanterv kötelező tanulási eredményként definiálja mindazokat az eredményeket, melyek a kötelező érettségi tárgyak teljesítéséhez szükségesek. Ezeket minden mikroiskola elérhetővé kell, hogy tegye a gyerekek számára a modulok választékában.

Ezek az 1--4 évfolyamszinteken a miniszter által kiadott kerettantervek \emph{Magyar nyelv és irodalom}, \emph{Matematika}, \emph{Idegen nyelv} tantárgyakból származó tanulási eredmények, és 5.~évfolyamszinttől kiegészülnek a \emph{történelem, társadalmi és állampolgári ismeretek} tantárgyak alapján létrehozott tanulási eredményekkel. További kötelező tanulási eredményként jelennek meg 9.~évfolyamtól a választott érettségi tantárgyhoz kapcsolódó eredmények. Ezek a tanulási eredmények megtalálhatók a kerettanterv három tantárgyának elérhető eredményei között.

A mikroiskolában meghirdetett moduloknak a kötelező tanulási eredmények 80\%-át le kell fednie.

\paragraph{Tanulási eredmények kiegyenlítettsége}

Szintén fontos kötöttség, hogy a modulok meghirdetésénél a kerettanterv három tantárgyából egyenlő súllyal (plusz/minusz 20\%) legyenek elérhetőek tanulási eredmények. Emellett a kerettanterv minden tématerületéről (vagyis a tantárgyak eredményeit alkotó diszciplinákból) legalább 20\% tanulási eredményt kell választani, így biztosítva, hogy a NAT minden műveltségterülete megjelenjen a tanárok által lefedett témák között.

\paragraph{Kötelező modulok}
A kerettanterv és a pedagógia program is előírhat kötelező modulokat a mikroiskolák számára. Ilyen például a 11. évfolyamszinten belépő érettségire felkészítő modulok (ld. \ref{sec:erettsegi}. fejezet), a minden mikroiskolára egységes pedagógia program tetszőleges kötelező modult írhat elő. Így lehet biztosítani kötelező tartalmi elemek és foglalkozás -- úgymint testnevelés, elsősegélynyújtás -- elérhetőségét.

\subsubsection{Monitorozás}

Kötelező elérni az eredményeket? Nem tudunk hatalmi szóval tanulásra bírni gyereket, mert lehet, hogy annyira nem akarja, vagy nincs meg hozzá a képessége. A kerettanterv a tanároknak ad keretet. Azonban a fenntartó által üzemeltetett rendszerrel az iskola  monitorozza a haladást, és ha valaki a kötelező tanulási elemekkel nem halad, akkor az iskola erre felhívja a figyelmét. Mivel a többség haladni fog, ezért előre tudja az iskola jelezni, hogy le fog szakadni a többiektől, és túl nagy lesz az évfolyamszint-különbség közöttük. Ezekben az esetekben a mentortanárnak, a gyereknek és a szülőnek reagálnia kell a helyzetre. A fenntartó által működtetett monitorozó és minőségfejlesztő rendszerről \aref{sec:minosegbiztositas} fejezet ír részletesen.

\section{Saját tanulási célok}
\label{sec:tanulasi_celok}

Minden gyerek megfogalmazza és háromhavonta újrafogalmazza a \emph{saját tanulási céljait}: eredményeket, amelyeket el akar érni, képességeket, amelyeket fejleszteni akar, szokásokat, amelyeket ki akar alakítani. A saját célok elfogadásakor a gyerek és a mentora a szülőkkel együtt \emph{tanulási szerződést} köt.

Csak olyan célok kerülhetnek a saját célok közé, amelyek minden érintettnek biztonságosak, és amelyek összhangban vannak a tantárgyi fejlesztési célokkal és tanulási eredményekkel. A szerződésben rögzíthetőek tanulási eredményekre vonatkozó megállapodások, tantárgyi évfolyamszintekre vonatkozó elvárások (pl. ,,\emph{haladjon egy évfolyamszintet egy év alatt}'' vagy ,,\emph{készüljön fel emelt szintű érettségire}''), és a tantárgyi rendszeren kívüli célok és feladatok.

%Fontos megkötés, hogy a saját tanulási célok legalább a felének \ifkerettanterv
%      \aref{sec:tantargyi_tanulasi_eredmenyek}. fejezetben
%\else
%      a kerettanterv Tantárgyi tanulási eredmények fejezetében
%\fi felsorolt tanulási eredmények elérésére kell vonatkoznia. A másik fele szabadon alakítható.

Háromhavonta a tanulásszervezők és a gyerekek megállnak, reflektálnak az elmúlt időszakra, és a tapasztalatok, valamint az elért célok ismeretében és az új célok figyelembevételével újratervezik, újraszervezik a foglalkozások rendjét, tehát azt, hogy mikor és mit csinálnak majd a gyerekek az iskolában. A mindennapi tevékenység során tapasztalt élmények, alkotások, elvégzett feladatok, kitöltött vizsgák, tehát mindaz, ami a gyerekekkel történik, bekerül a portfóliójukba. Még az is, amit nem terveztek meg előre.

A gyerekeket a mentoruk segíti a saját célok kitűzésében, a különböző választásoknál, a portfólióépítésben, a reflektálásban. A tanulási célok kitűzése az önirányított tanulás fokozatos fejlődésével és az életkor előrehaladtával folyamatosan egyre önállóbb tevékenységgé válik. Tanulási útján, céljai kitűzésében a mentor kíséri végig a gyerekeket.

A Budapest School személyre szabott tanulásszervezésének jellegzetessége, hogy a gyerekek a saját céljuk irányába haladnak, az adott célhoz az adott kontextusban leghatékonyabb úton. Tehát mindenki rendelkezik saját célokkal, még akkor is, ha egy közösség tagjainak céljai a tantárgyi tanulási eredmények azonossága, vagy a hasonló érdeklődés miatt akár  80\% átfedést mutatnak.

A NAT műveltségi területeiben és a tantárgyakban megfogalmazott követelmények teljesítése is célja a tanulásnak, a tanulás fő irányítója azonban más. Mi azt kérdezzük a gyerekektől, hogy \emph{ezenfelül} mi az ő személyes céljuk.
\section{A tanulási szerződés}

A tanulási szerződés az előbbiekben említett gyerek-mentor-szülő közötti
megállapodás, ami rögzíti
\begin{enumerate}
      \item a gyerek, a mentor (iskola) és a szülő igényeit, elvárásait;

            ezek lehetnek: \emph{,,szeretném, ha a gyerekem naponta olvasna''}
            típusú
            folyamatra vonatkozó kérések, vagy erősebb \emph{,,változtatnod
                  kell a
                  viselkedéseden, ha a közösségben akarsz maradni''} igények,
            határok
            megfogalmazása;

      \item a gyerek céljait a következő trimeszterre, vagy a tanév végéig;

      \item a gyerek, mentorok (iskola) és szülő vállalásait, amivel támogatják
            a
            cél
            elérését és a felek igényének elérését.

\end{enumerate}

A tanulási szerződésre jellemző, hogy
\begin{itemize}
      \item A kitűzött célokat minél specifikusabban, mérhetőbben kell
            megfogalmazni.
            Javasolt az OKR  (Objectives and Key Results, azaz	Cél és Kulcs
            Eredmények)
            \citep{okr} vagy a SMART (Specific, Measurable, Achievable,
            Relevant,
            Time-bound, azaz Specifikus,  Mérhető, Elérhető, Releváns és Időhöz
            kötött)
            \citep{wiki:smart} technika alkalmazása, hogy minél specifikusabb,
            teljesíthetőbb, tervezhetőbb és könnyen mérhető célokat tűzzenek
            ki.

      \item A kitűzött célokban való megállapodást követően, megállapodást
            kell
            kötni arról is, hogy ki és mit tesz azért, hogy a gyerek a célokat
            elérje.

      \item A mentor a teljes mikroiskolát (a többi tanárt, a közösséget)
            képviseli
            a
            megállapodás során.
\end{itemize}

A tanulási szerződést néha hívjuk \emph{megállapodásnak} is. A megállapodás és
szerződés szavakat ez a kerettanterv szinonimának tekinti. A \emph{learn\-ing
      con\-tract} az önirányított tanulást hangsúlyozó felnőttképzéssel
foglalkozó
irodalomban
bevett szakkifejezés már a 80-as évektől \citep{Malcolm77}. Ennek a magyar
nyelvben inkább a szerződés felel meg. Egy másik szakterületen, a
pszichoterápiás munkában a terápiás szerződések megkötésekor a közös munka
kereteinek kialakítását és fenntarthatóságát hangsúlyozzák
\citep{pszichoterapia}. Erre is utalunk a tanulási szerződés elnevezéssel. Van,
amikor a \emph{hármas szerződés} kifejezést használjuk, hangsúlyozva, hogy mind
a három szereplőnek elfogadhatónak kell tartania a szerződés tartalmát.

\section{Visszajelzés, értékelés}
\label{sec:ertekeles}
Ahhoz, hogy hatékony legyen a tanulás, fejlődés, fontos, hogy a gyerekek, tanárok és szülők is tudják, hogy
\begin{enumerate}
      \item hol tart most egy gyerek, mit tud most,
      \item hova akar vagy kell eljutni, azaz, mi a célja,
      \item mi kell ahhoz, hogy elérje a célját.
\end{enumerate}
Ezek mellett mindenkinek hinnie kell abban, hogy odafigyeléssel, gyakorlással a gyerek meg tud tanulni egy konkrét dolgot. Fontos, hogy magas legyen a gyerekek énhatékonysága,  erős legyen az önbizalmuk, és nem szabad félniük a hibázástól, a nem-tudástól, mert a tanulás első lépése, hogy elfogadjuk, hogy valamit nem tudunk. Azaz fontos, hogy fejlődésfókuszú gondolkodásuk (growth mindset) \citep{growthmindset} legyen, azaz
\begin{enumerate}
      \setcounter{enumi}{3}
      \item hinniük kell, hogy el tudják érni a céljukat.
\end{enumerate}

Egy visszajelzés, értékelés akkor jó és hasznos, azaz hatékony, ha ebben a négy dologban segít. Mai tudásunk szerint ehhez:
\begin{itemize}
      \item Rendszeresen visszajelzést kell kapniuk és adniuk.
      \item A tanulási céloknak és visszajelzéseknek minél specifikusabbaknak kell lenniük (azaz például ne a 8. oszályos \emph{matematikatudást} értékeljük, hanem hogy mennyire képes valaki \emph{fagráfokat használni feladatmegoldások során}\footnote{Ez a konkrét példa a matematika tantárgy egyik tanulási eredménye.}).
      \item A \emph{,,hol tartok most''} diagnózisnak mindig cselekvésre, viselkedésre, aktív tevékenységre kell vonatkoznia. Ne az legyen a visszajelzés, hogy \emph{,,ügyes vagy egyenletekből''}, hanem \emph{,,gyorsan és       pontosan oldottad meg a 4 egyenletet''}. A legjobb, amikor a visszajelzés konkrét megfigyelésen alapul, és tudni, hogy mikor, hol történt az eset: \emph{,,amikor társaiddal Minecraftban házat építettél, akkor       pontosan kiszámoltad a ház területét''.}
      \item Ha a cél nem a mások legyőzése, akkor a visszajelzés se tartalmazzon olyan állítást, ami másokhoz hasonlít (így kerüljük a \emph{tehetség} szót is, aminek bevett definíciója szerint az átlagnál jobb képesség). A másokhoz való szint felmérése akkor (és csak akkor) fontos, amikor a cél egy versenyszituációban jó eredményt elérni.

      \item A gyerek legyen részese a visszajelzésnek. Értse, tudja, hogy miért kapta azt a visszajelzést, a legjobb, ha -- amikor ezt a képességei engedik -- önmaga képes elvégezni a visszajelzést, vagy annak egy részét.
      \item A visszajelzésnek transzparensen hatással kell lennie a tanulásszervezésre. Legyen része a folyamatnak, és a gyerek, tanár és a szülő is értse, hogy a visszajelzés alapján mit és hogyan csinálunk másképp.
\end{itemize}

\paragraph{Többszintű visszajelzés} A Budapest School iskolákban a gyerekek
többféle visszajelzést kapnak. \begin{enumerate}
      \item Minden modul elvégzése után a modul céljai, témája, fókusza alapján a modulvezetők visszajelzést adnak a tanulásról, eredményekről, viselkedésről.
      \item Trimeszterenként a mentorok visszajelzést adnak arról, hogy a gyerek általában hogyan haladt a tanulási célok felé.
      \item Ennek része, hogy a tantárgyi tanulási eredmények alapján hogyan haladt a gyerek a tantárgyak évfolyamszinthez tartozó követelményeinek teljesítésében. Az évfolyamok, mint elérhető szintek Budapest School-értelmezését \aref{sec:evfolyamok}. fejezet tárgyalja.
      \item A mentorok irányításával a gyerekek visszajelzést kapnak arról, hogyan működnek a közösségben.
\end{enumerate}

\paragraph{Érdemjegyek, osztályzatok helyett értékelő táblázatok} A Budapest
School visszajelzéseinek sokkal részletesebbeknek kell lenniük, mint azt a tantárgyi érdemjegyek és osztályzatok lehetővé teszik, ezért azok helyett a kerettanterv értékelő táblázatokat (angolul rubric) alkalmaz. Az értékelő táblázatban szerepelnek az értékelés szempontjai és szempontonkénti szintek, rövid leírásokkal. Ezek alapján a gyerekek maguk is láthatják, hogy hol tartanak, hogyan javíthatnak még a munkájukon. A táblázatok formája minden visszajelzés esetén (értsd modulonként, célonként) változtatható.
\section{Portfólió}
\label{sec:portfolio}
A modulok eredményeiből, a produktumokból és visszajelzésekből a gyerek és a mentor portfóliót állít össze, hogy a tanulás mintázatait észlelhesse, és a tanárok tudatosabban tudják a gyereket segíteni a céljai kitalálásában és elérésében. A portfólió a gyerek eredményeinek nyomon követését is szolgálja, és egyúttal a szülők felé történő visszajelzés eszköze is. Minden gyerek portfóliója folyamatosan épül: az tartalmazza az általa elvégzett feladatokat, projekteket vagy azok dokumentációját, alkotásait, eredményeit, az esetleges vizsgák eredményeit és a társaitól, tanáraitól kapott visszajelzéseket. A \emph{portfólió célja}, hogy minden információ meglegyen ahhoz, hogy

\begin{itemize}
      \item a gyerek és mentora fel tudja mérni, hogy sikerült-e a kitűzött célokat elérni, illetve mire van szüksége még a gyereknek új célok eléréséhez;

      \item a szülő folyamatosan rálásson a gyereke tanulási útjára;

      \item megítélhető legyen, hogy a tantárgyi követelményekhez képest hol tart a gyerek;

      \item a gyerek a portfólió megtekintésével visszaemlékezhessen a tanultakra, ismételhessen, tudása elmélyülhessen;

      \item eredményei alapján bizonyítványt lehessen kiállítani.

\end{itemize}

A portfólió folyamatosan frissül, a mindennapi, formális, non-formális és informális tanulási helyzetek bármikor adhatnak okot a portfólió frissítésére. Az iskola életében kiemelt szerepe van a következő eseményeknek.

\begin{enumerate}
      \item Minden \emph{modul végeztével} a portfólióba kerül:

            \begin{enumerate}

                  \item  A képesség elsajátításának, tanulási eredmény elérésének a ténye. Nincs félig elsajátított képesség, tehát már értékelni nem kell. Ha a modul során a gyerek megtanult százas számkörben alapműveleteket végezni,  akkor annyi kerül be a portfólióba, hogy ,,\emph{Szóban és írásban összead, kivon, szoroz és oszt a százas számkörben}''. Amennyiben a készséget a gyerek és a tanár megítélése alapján nem sikerült megfelelően elsajátítani, úgy a gyakorlás ténye kerül be a portfólióba.
                  \item Az alkotás vagy a projektmunka eredménye, ha a modul célja egy alkotás létrehozása volt.
                  \item A részvétel ténye, ha a jelenlét volt a modul célja (például kirándulás az Országos Kéktúra útvonalán).

            \end{enumerate}
      \item Az elvégzett vizsgák, tudáspróbák, képességfelmérők, diagnózisok eredményeit érdemes rögzíteni.

      \item A \emph{kipakolás} célja, hogy a gyerekek a tanároknak, szülőknek és más érintetteknek bemutassák elvégzett munkájukat, azaz a portfólióváltozásukat. A kipakolásra való felkészülés tulajdonképpen a portfólió összeállítása, prezentálásra való felkészítése, a \emph{portfólió frissítése}.

      \item Társas visszajelzés eredményeként minden gyerek kap visszajelzést a társaitól. Ilyenkor összegyűjtik, mit tett a gyerek, ami a többiek elismerését és háláját kivívta. Ez is releváns adatokkal szolgálhat a portfólióhoz.

      \item A gyerek saját értékelése, reflexiója arról, hogyan értékeli, amit elért, fontos eleme a portfóliónak.

      \item A tanárok adhatnak kompetenciatanúsítványokat. Ezek rövid, specifikus visszajelzések, amelyek mutatják, ha valamit a gyerek megcsinált, valamiben fejlődött.
\end{enumerate}

A mentorok segítenek a gyerekeknek a tanulás módját, folyamatát és eredményeit bemutatni portfólióban.

\paragraph{Formai követelmények}
A portfóliónak rendezettnek, hozzáférhetőnek, elérhetőnek, visszakereshetőnek és könnyen bővíthetőnek kell lennie. Olyan (technológiai) megoldást kell a mikroiskoláknak választaniuk, ami alapján a gyerek, tanár és a szülő \emph{naponta} tudja a portfóliót bővíteni, és akár \emph{heti rendszerességgel} át tudják tekinteni időrendben, modulonként vagy tantárgyanként a portfólió bővülését.

A portfólió formátumára nincs egységes megkötés. Minden mikroiskola maga alakítja ki a gyerekek, tanárok és szülők számára legjobban működő rendszert. Évfolyamszintlépéshez és osztályzatokra váltáshoz az iskola csak digitális formában tárolt és a kijelölt tanárok számára online elérhetővé tett portfóliót fogad el.

\section{A csoportbontások}
\label{sec:csoportbontasok}

A Budapest School iskoláit mikroiskolák közösségeiből hozzuk létre. Így
egy gyerek elsődleges csoportja a mikroiskolájának közzössége, ami lehet
12 - 50 gyerek. Ezen belül modulonként eltér, hogy milyen
csoportbontásban dolgoznak. Több szinje van a csoportmunkának.
\begin{enumerate}

      \item A mikroiskola közössége heti rendszerességgel tarthat
            iskolagyűlést,
            fórumot, plenárist, iskolakonferenciát. Ilyenkor a mikroiskola
            közössége
            dolgozik együtt.
      \item  Modulokra kisebb csoportok jelentkezhetnek. Egy modul
            csoportjának rendezőelve lehet, hogy
            \begin{enumerate}
                  \item egy képességszinten lévő gyerekek tanuljanak együtt;
                  \item a közös érdeklődés hozza össze a
                        csoporttagokat;
                  \item an, hogy direkt a véletlenszerűségben van az
                        érdekesség, mert keveredni akarunk;
                  \item kölcsönös szimpátia és vonzalom
                        lehet a modultagok között: most azért vannak egy
                        csoportban,
                        mert egy
                        csoportban akartak lenni.
            \end{enumerate}
      \item  Egy-egy foglalkozáson belül is sokszor
            csoportot alkotunk, az előző elvek alapján.

\end{enumerate}

Arra is lehetőség van, hogy egy csoport tagjai több mikroiskolából álljanak
össze, ha az támogatja a tanulást és az utazás biztonságosan megoldható.

% ez behuzza a kerettanterves reszt is
\input{chapters/kerettanterv/keretek/bizonyitvany}
\subsection{Kérelmek elbírálása}

A szülő és gyerek évfolyamszintlépés vagy osztályzatra váltás kérelmét bírálók értékelik. Minden esetben legalább három bíráló bírál egy kérelmet: a bírálók közül egy a gyerek mentortanára, egy pedig mindenképp másik mikroiskola tanulásszervezője. A bírálókat a fenntartó választja ki és kéri fel. Ha a mentortanár valamilyen okból nem tudja feladatát elvégezni, akkor helyette az a tanulásszervező lesz a bíráló csapat tagja, aki legtöbb időt töltött az elmúlt két trimeszterben a gyerekkel.

Ha a bírálók közül akár egy is úgy véli, hogy az évfolyamszintlépés vagy osztályzat nem megalapozott, akkor erről indoklással értesítik a gyereket, szüleit és mentortanárát. Ilyenkor a gyerek javíthatja kérelmét és tetszőleges számú esetben ismételheti a folyamatot. Tulajdonképpen az évfolyamszintekről és osztályzatokról mindig olyan döntésnek kell születnie, ami a gyereknek, a szülőknek és a bírálóknak is elfogadható. Ha nem tudnak megállapodni, akkor \aref{sec:konfliktusok_kezelese}. fejezetben leírtak alapján kell keresniük a mindenki számára elfogadható megoldást.

Az évfolyamszintlépésről és az osztályzatra váltásról vagy épp a kérelmek elutasításáról az iskola minden tanára értesítést kap. Ha a nevelőtestület\footnote{NKT szóhasználata.} nem ért egyet a bírálók döntésével, akkor új bírálók kinevezését kérhetik.

A kérelmek elbírálását 20 tanítási nap alatt el kell végezni.\footnote{Ha a kérelem nyári szünetben érkezett, akkor augusztus 31-ig.}

Évfolyamszintlépést az iskola az utolsó évfolyamszintlépéstől számított két tanéven belül automatikusan indít. Így nem fordulhat elő, hogy egy gyerek évfolyamszintjei két éven keresztül nincsenek felülvizsgálva.

Magántanuló és az órák látogatásáról valamilyen okkal felmentett gyerekek évfolyamszintlépését és szükség esetén osztályzatkérelmét pontosan ugyanazon folyamatokkal kell elvégezni, mint a nem magántanuló és nem felmentett gyerekét.

\subsection{Osztályozóvizsga}
% forras: http://www.petroczigabor.hu/cikkek/tanugyigazgatas/Osztalyozo_vizsga.html
A \emph{Osztályzatokra váltás} és \emph{Évfolyamszintlépés} folyamat a gyerek évközi alkotásai, munkája, teljesítménye eredményeként készült portfólió alapján történik, ezért a folyamat megfelel a 20/2012. (VIII.31.) EMMI-rendelet 64.~§ (1) azon elvárásának, hogy a tanuló osztályzatait \emph{,,évközi teljesítménye és érdemjegyei''} alapján kell megállapítani azzal a kikötéssel, hogy a Budapest School-kerettanterv elfogadja az érdemjegyeknél gazdagabb szöveges és értékelőtáblázatok-alapú visszajelzést.

Ha a portfólió alapján nem állapítható meg az évfolyamszintlépéshez és az osztályzathoz szükséges tanulási eredmények megléte, akkor a gyereknek ki kell egészítenie a portfólióját. Akár egy tudáspróbával, vagy online teszttel.

Igazolt és igazolatlan mulasztások miatt az iskolának nem kell az \emph{Osztályzatokra váltás} és \emph{Évfolyamszintlépés} folyamatoktól eltérnie, mert a portfólió miatt mindig értékelhető a gyerek elért teljesítménye.

Amennyiben a gyereket bármely okból felmentették a tanórai foglalkozásokon való részvétele alól, akkor a 20/2012. (VIII.31.) EMMI-rendelet 64.~§ (2)  alapján osztályozóvizsgát kell tennie. A Budapest School iskolában a fent részletezett \emph{Osztályzatokra váltás} és \emph{Évfolyamszintlépés} folyamatok  felelnek meg az osztályozóvizsgának.


\section{A felvétel és az átvétel}
\label{sec:felvetel-atvetel}
Egy mikroiskola közösségéhez bármikor lehet csatlakozni, ha és amikor a
csatlakozó
család ezt szeretné, és ha ettől a közösség minden tagjának valamiért  jobb
lesz vagy nem változik (de rosszabb nem lehet).
Az iskola nem azért fogad be valakit, mert
ezt kell, hanem mert a mikroiskola közössége ezt szeretné.
A családok nem azért csatlakoznak, mert valamit kell találni a gyereknek, hanem
mert
szeretnének a Budapest School egyik mikroiskola közösségéhez tartozni.

A 12 évfolyamos egységes iskolába bármikor lehet csatlakozni, az iskola
normálisnak tartja, hogy a közösség tagjai változnak.
Ezért az iskolában egy iskolát most kezdő 6 éves felvétele, egy 8 éves, az előző
iskoláját nem kedvelő felvételi kérelme, egy 9 éves
külföldről hazaköltöző év közbeni csatlakozása, egy 12 éves ,,gimnáziumba''
jelentkezése és egy 16 éves más városból érkező  között az iskola
számára
a \emph{felvételi és átvételi folyamatot} tekintve nincs különbség.

Kizárólag egy mikroiskola tanulásszervezőinek és a fenntartónak
a\linebreak
hozzájárulása
szükséges ahhoz, hogy
egy család csatlakozzon egy mikroiskolához.

Egy család jelentkezése után legalább három dolognak kell történnie.
\begin{itemize}
      \item A családnak meg kell ismernie a Budapest School alapelveit,
            működését, jellegzetességeit. Az iskolának meg kell mutatnia
            önmagát. A
            családnak meg kell értenie,  és meg kell fogalmaznia, hogy miért
            akarnak
            csatlakozni a közösséghez.
      \item A tanulásszervezőknek meg kell ismerniük a családot, megnézni, hogy
            ,,működik-e a kémia'', tudják-e vállalni a gyerek tanulásának
            támogatását.
      \item A gyereknek időt kell eltöltenie a mikroiskolában, a mindennapokhoz
            minél inkább hasonló körülmények között, hogy mindenki meg tudja
            tapasztalni,
            érezni, hogy milyen lenne együtt és egymástól tanulni.
\end{itemize}

\paragraph{Szempontok a döntéshez}
A mikroiskolák közössége legyen minél inkább diverz és kiegyensúlyozott: kevert
korosztályú, kevert nemi, kevert szociális státuszú, kevert
érdeklődésű, kevert személyiségjegyű csoportok úgy, hogy legyen egy erős,
mindenkit megtartó szociális
hálózat. A mikroiskolák közösségét egyenként kell kiegyensúlyozni.%\eject

Így az is előfordulhat, hogy egy gyerek egy mikroiskola közösségében nem talál
helyet magának, de az iskola egy másik mikroiskolájában igen. Mert a
közösségek különbözőek.\footnote{Legegyszerűbb példa: van, ahol több lányt
      szeretnénk, mint ma, és van, ahol több fiút, és van, ahol ez most nem
      szempont.}

\paragraph{Nincs felvételi vizsga.}
Az iskola nem követel meg se írásbeli (központi), se szóbeli
felvételit, és nem is az előző iskolák osztályzatai alapján dönt. Egyetlen
szempont, hogy jobban tud-e egy mikroközösség működni egy gyerek/család
csatlakozásával. A döntést a mikroiskola tanárai, a fenntartó (vagy
delegáltja) és a család hozzák meg.

%A mikroiskolák folyamatosan veszik fel a gyerekeket. Az iskolában nincs egy nap, amikor a felvéte

\paragraph{Szakítás, távozás, elengedés}
Működésünk része, hogy konfliktusok, kényelmetlenségek, változó körülmények
között, aki egyszer csatlakozott, az egyszer távozhat is a közösségből.
\begin{itemize}
      \item Az iskola, a tanárok és a család alapelvei, értékei közötti
            különbségek okozhatnak annyi és olyan konfliktust, amit már nem
            tudnak a felek
            feloldani.
      \item Van, hogy egy gyerek nem találja meg a helyét, vagy épp valamiért
            elkezd a közösségben ,,nem boldog'' pozícióba kerülni. Vagy épp a
            közösség
            többi tagjának lesz kényelmetlen az együttlét.
      \item Családok élete, vágyai, motivációjuk, körülményei
        változhatnak\linebreak
        úgy,
            hogy épp más közösségben jobb helyet találnának.
\end{itemize}

Bármi legyen is az ok, a távozás, szakítás feszültséggel teli szituáció. Ezért
is fontos, hogy minden fél betartsa  \aref{sec:konfliktusok_kezelese}.
fejezetben leírt konfliktuskezelési szokásokat.

\section{Hiányzások, mulasztások, igazolások}
A Budapest School feladata, hogy olyan környezetet biztosítson a gyerekeknek, amiben boldogak, felszabadultak, magabiztosak és hatékonyak tudnak lenni. Budapest School családok ezért választják ezt az iskolát, áldoznak sok pénzt és időt arra, hogy az iskolában tudjanak tanulni. Ezért az iskola feltételezi, hogy a gyerekek az iskolában akarnak lenni önszántukból.

Sok oka lehet annak, hogy egy gyerek még sincs az iskolában . Például
\begin{itemize}
    \item betegnek, fáradtnak érezheti magát, fizikailag vagy lekileg kimerült, vagy lehet valamilyen fertőző betegsége;
    \item családjával tölt értékes, minőségi időt, mert fejlődését ez szolgálja a leginkább;
    \item utazik, felfedez, külső helszínre szervezett tanulási programokon vesz részt;
    \item egy projektjébe úgy belemerül, hogy érdemesnek találja otthon, fókuszáltan végezni a munkát (home office)
    \item előre nem tervezett esemény miatt nem tud az iskolába menni.
\end{itemize}

A nem iskolában töltött időt a ,,home office'' mintájára ,,otthon tanulásnak'' hívja az iskola, mert feltételezi, hogy a nem iskolában töltött idő is tanulással jár.

Alapelv, hogy \emph{a mentornak, gyereknek és szülőnek meg kell állapodni az otthon tanulásról}. Minden félnek tudnia kell róla, és meg kell különböztetni a tervezett otthontanulást a nem tervezett hiányzástól.

\paragraph{Tervezett otthontanulás} Előre eltervezett módon, valamilyen program miatt nincs a gyerek az iskolában. Ilyenkor a mentor és a gyerek megtervezi az otthontanulás célját, várható eredményeit. A terv létrejöttéért a gyerek és a szülő felelős és minden félnek el kell fogadnia a tervet.

\paragraph{Nem tervezett hiányzás} A gyerek és a mentortanár nem tud előre felkészülni az iskolán kívüli tanulásra, mert a hiányzás előző nap vagy aznap derül ki vagy más okból a megállapodás nem jön létre. A szülő feladata, hogy még ebben az esetben is erről reggel 9 óra előtt értesítse a mentortanárt. Mikroiskolánként eltérhet a preferált kommunikációs eszköz, ezért a tanulásszervezők feladata meghatározni, hogyan kérik az értesítés formáját.

Ha a tervezett otthontanulás elérte a 20 napot, vagy 160 órát, akkor a mentortanár mellett egy másik mentortanár szerepben dolgozó tanulásszervezőnek is meg kell ismernie és el kell fogadnia a tervet. 40 nap felett három mentortanárnak kell együtt elfogadnia a tervet melyek egyike egy másik mikroiskola mentortanára.

Egy tanévben 15 nap, vagy 120 óra, nem megtervezett hiányzást igazolhat a szülő (rögzített és dokumentált módon). Orvos által igazolt betegség, hatósági intézkedés és egyéb alapos indok esetén 20/2012. (VIII.~31.) EMMI rendelet 51.~§~(2) értelmében igazoltnak kell tekinteni a hiányzást.

Igazolatlan hiányzásnak azt az esetet kell tekinteni, amikor a szülő vagy a mentor tanár nem tudott a hiányzásról, nem volt előre megtervezve vagy a 15 napos, 120 órás keret kimerült. Ilyenkor a 20/2012. (VIII.~31.) EMMI rendelet 51.~§~(3) pontja értelmében minden esetben az iskola értesíti a szülőt, és 10 igazolatlan óra után figyelmezteti, hogy a következő igazolatlan után ,,az iskola a gyermekjóléti szolgálat közreműködését igénybe véve megkeresi a tanuló szülőjét" (idézés az EMMI rendeletből). Az iskola megközelítése egyszerű: mivel partneri viszonyban van a tanár, gyerek, szülő ezért meg tudják beszélni a hiányzásokat. Elég tág keretet enged az iskola. Abban az esetben azonban, amikor a gyerek vagy a szülő nem tartja be a kereteket, nem él a partneri viszonnyal, akkor ott valami baj van. Gyorsan kell reagálni.

\subsection{Késések kezelése}
A Budapest School mikroiskolák maguk állítják fel a napirenddel kapcsolatos kereteket: mikor kezdenek, meddig tartanak a struktúrált foglalkozások, mikor vannak a szünetek és hogyan kezdődik újra a nap folyamán a fókuszált munka. A kereteket a tanulásszervező tanárok feladata kialakítani és trimeszterenként kihirdetni.

Fontos, megbeszélendő részlet, hogy hogyan kezeli a közösség a késéseket: mikortól lehet érkezni, mikor kezd a közösség annyira dolgozni, hogy zavaró, amikor valaki belép és megzavarja a folyamatot. Megállapodást köt a közösség, hogy hogyan kívánja kezelni a késéseket, mi segíti a csapatot leginkább a céljai elérésében.

Az iskola nem regisztrálja a késéseket, mert az iskola nem tudhatja, hogy egy-egy késés elfogadható-e a közösségnek vagy nem. Egy színdarab főpróbájáról 5 percet késni mást jelent a közösség számára, mint arról az óráról, ahol mindenki egyedül füllhallgatóval böngészi egy online tananyag számára legrelevánsabb fejezetét.

Ha egy csoportot megzavar valakinek az ismételt késése, akkor konfliktus alakul ki a csoport és a késő vagy a tanár és a késő között. Ezt a típusú konfliktust (is) \aref{sec:konfliktusok_kezelese}.~fejezet szerint kell feloldani.

% innentol vagyunk koldoknezegetosek
\section{Az iskola kormányzása}
\label{sec:az_iskola_kormanyzasa}
Az iskola szervezetét a tagjai együtt alakítják. A szervezet élő, folyamatosan változik, a következő alapelvek mentén:

\begin{itemize}

  \item
        Gyorsan és folyamatosan tanuló, agilis szervezetként az iskola min\-den\-nap jobban támogatja a gyerekek tanulását, mint tegnap.
  \item
        Minden résztvevő stabilitás, biztonság, kiszámíthatóság iránti igénye pont annyira fontos, mint a változás, a javulás, a fejlődés igénye.
  \item
        A gyerekeket leginkább ismerő, hozzájuk legközelebb álló tanárok (gyerekekkel és szülőkkel erős kapcsolatban) minél több helyzetben hoznak döntéseket.
  \item
        Az együttműködés, a partneri kapcsolat, a kölcsönös felhatalmazás a tanárok, adminisztrátorok között is folytonos, nem csak a tanár-gyerek kapcsolatban.
\end{itemize}

Azt szeretnénk, hogy gyerekeink kreatív, környezetüket aktívan alakító, mély partneri kapcsolatokban élő, problémamegoldó, csapatjátékos, folyamatosan tanuló, a világot változtatni tudó felnőttekké váljanak. Ehhez olyan iskolát építünk, ahol a tanárok kreatívak, környezetüket aktívan alakítják, mély partneri kapcsolatban élnek, problémákat oldanak meg, csapatban dolgoznak és folyamatosan tanulnak.

Az iskolát mint szervezetet a \emph{szociokrácia} (sociocracy) szervezeti modell alapján működtetjük, mert így tudjuk leginkább elérni, hogy egyszerre legyen örömteli és hatékony az együttműködésünk. Ez egy olyan, az üzleti világban is kipróbált dinamikus döntéshozatali rendszer, amely egyszerre segíti a harmonikus és örömteli közösségi együttműködést, és jelent garanciát arra, hogy az együttműködés hatékony lesz az egyes csapatokon belül.

\subsection{Döntéshozás}
\label{sec:consent_based}

Az iskola minden csapata (egy mikroiskolát működtető tanulásszervezők, gyerekek egy csapata stb.) a \emph{hozzájáruláson alapuló döntési mechanizmust} (consent based decision making) használja ahhoz, hogy a szervezet gyorsan tudjon döntést hozni, \emph{és} minden tagja hallathassa a hangját.

A csapat valamennyi tagja hozhat javaslatokat, ha működési
hatékonytalanságot, feszültséget, problémát talál, és van rá
megoldása. A javaslat értelmezése után az érintettek mindegyikét meg
kell hallgatni, hogy \emph{elfogadhatónak} tartja-e a javaslatot, azaz
hozzájárul-e a változáshoz, mert \emph{,,elég jónak és biztonságosnak
  találja, hogy kipróbáljuk az új működést''} (\emph{``is this
  good enough for now and safe enough to try?''}). Fontos, hogy
mindenki egyenként hallassa hangját. A javaslat elfogadásra kerül, ha
és amikor minden érintett hozzájárult.

Mindenki kifejezheti a \emph{fenntartásait} (concern), és a csapat feladata ezeket meghallani, és reagálni rájuk. A fenntartás azonban még nem jelenti a javaslat elutasítását, csak fontos információt ad a döntés végrehajtásához.

A javaslatot a csapat nem fogadja el, ha valamelyik csapattag \emph{ellenzi} (objection) azt. Az ellenzés egy én-üzenet, valami olyasmi: ,,ha a csapat ezt a döntést meghozná, akkor mélyen sérülne a csapathoz való elköteleződésem, mert az én igényemet, ami \ldots{}, nem elégíti ki a javaslat. Nekem szükségem van \ldots{}, ezért inkább javaslom, hogy \ldots{}''. Fontos, hogy az ellenvetést megfogalmazó mondja el a saját igényeit, szükségleteit és tegyen új javaslatot, vagy kérjen segítséget, hogy milyen új javaslatot tehetne. Ellenvetés esetén a csapat együtt dolgozik azon, hogy új javaslatot találjon, ami az ellenvetést feloldja, és az eredeti javaslat célja felé viszi a szervezetet.

\paragraph{A hozzájárulás nem konszenzus}

A konszenzus alapú döntések esetén mindenkinek egyet kell értenie abban, hogy a döntés a legjobb, leghelyesebb, leghelyénvalóbb. A Budapest School iskolában azt a kérdést tesszük fel inkább, hogy van-e valakinek ellenvetése és a javaslat kellően biz\-ton\-sá\-gos-e ahhoz, hogy kipróbáljuk. Nem azt a kérdést tesszük fel, hogy mindenki ezt a döntést hozta volna-e és mindenki egyetért-e a döntéssel hanem azt, hogy mindenki tudja-e támogatni a csapat egy másik tagját, és nincs-e olyan ismert kockázat, ami az egyén vagy a szervezet szempontjából nem vállalható fel.

\paragraph{A hozzájárulás nem szavazás}

A Budapest School iskolában nem a többség dönt, és nem az számít, hogy hányan akarnak egy döntés mellé állni. Mindenki hozhat döntést, amit elfogad a csapat minden tagja, azaz egyetlenegy ellenvetés sincs.

\paragraph{Az ellenvetés nem vétó}

A vetójog gyakorlatban a döntés megakadályozását jelenti. Amikor valaki megvétóz egy döntést, akkor azzal a folyamat általában megakad. Az iskola működésében használt ellenvetés egy beszélgetés megindítását jelenti: ,,ezt én így nem tudom támogatni, helyette ezt javaslom inkább''.

\paragraph{A ,,van egy jobb ötletem'', nem ellenvetés}

A szervezetnek nem az a feladata, hogy a legjobb döntéseket hozza, hanem hogy amikor szükséges, akkor javítson a működésén. Ezért minden döntéskor mindenkinek azt kell mérlegelnie először, hogy elfogadható-e neki, hogy azt a bizonyos javaslatot kipróbálja a csapat. Attól, hogy valaki jobb, más javaslatot is tud, attól még először az eredeti javaslatot érdemes kipróbálni és tesztelni.

\paragraph{Minden javaslat csak egy hipotézis}

Amikor valamit változtatunk a szervezet működésén, akkor egy kísérletbe vágunk bele: kipróbáljuk, hogy az új működés tényleg jobb-e, megoldja-e a problémát, feszültséget, kielégí\-ti-e az igényeket. A döntés támogatásakor ezt a próbálkozást támogatjuk.

\paragraph{Nem döntünk mindenről együtt}

A Budapest School iskolá\-ban csak az iskola és a mikroiskola működését megváltoztató kormányzási kérdésekről döntünk együtt.~A Budapest School minden tagja szerepeiből kifolyólag fel van hatalmazva arra, hogy a mindennapi döntéseit maga meghozhassa, ezért nem kell mindent megbeszélnünk. A cél, hogy olyan szerepeket és rendszereket alakítsunk ki, hogy a mindennapi döntéseket mindenki maga meg tudja hozni.

\subsection{Csapatok -- az iskola szervezeti egységei}

A Budapest School csapatai (a szociokrácia terminológiájában a \emph{körök}) önálló csoportok egy jól meghatározott céllal, felruházott felelősséggel, döntési körrel. A csapatok maguk határozzák meg a saját működésüket (policy making), és végzik el a saját feladatukat. Az iskolában azok döntenek együtt, akik együtt dolgoznak, egy csapatban (``those who associate together govern together''). És fontos, hogy akik együtt dolgoznak, jól legyenek egy\-mással.

Azt is tudjuk, hogy akik egy munkát elvégeznek, azok a munka szak\-ér\-tői, ezért ők tudnak arról a legjobban dönteni, hogy hogyan érdemes a munká\-ju\-kat szervezni, alakítani. Nincs főnök, külső szakértő, aki megmondja egy csapatnak, mit és hogyan csináljanak addig, amíg a rájuk felhatalmazott kereteken belül maradnak. Az természetes, hogy minden segítséget, támogatást, információt megkapnak, amire szükségük van. De a kormányzás az ő kezükben van.

\paragraph{Mikroiskola tanulásszervező csapata.}

A Budapest School szervezet állandó csapatai az egy-egy mikroiskolát vezető tanulásszervezők csapata, ami egy \emph{szociokratikus kör}. A mikroiskola gyerekeinek (családjainak) és tanárainak életét meghatározó döntéseket maguk hozzák meg. Így például a napirend, a csoportbontások, a szülői értekezletek tematikája a saját döntéseik alapján alakul ki. Fontos, hogy a csapattagok maguk tudják meghatározni, kivel tudnak és akarnak együtt dolgozni, mikor és mit akarnak csinálni.

\paragraph{Csapatok kapcsolódása}

Egy-egy ember több csapatnak is tagja lehet. Egyrészt munkacsoportok alakulhatnak egy-egy feladat elvégzésére, és a Budapest Schoolban egy ember több részfeladatot is ellát. Másrészt a csapatokat kifejezetten úgy alakítja a közösség, hogy legyenek köztük kapcsolódások, olyan tagok, akik összekötik a csapatokat.

Vannak olyan csapatok, melyek elsődleges célja összekötni a kisebb csapatokat. Például minden mikroiskolai csapat delegál egy képviselőt az iskola közös naptárát létrehozó munka\-csoportba.

\paragraph{Csapatok vezetői}

Minden csapatnak van egy \emph{vezetője} (szociokrácia terminológiában a
\emph{circle leader}). A vezető feladata, hogy mindenki ismerje a csapat célját, a \emph{,,miért létezünk?''} kérdésre a választ, és hogy a csapat működjön: tiszták legyenek a szerepek és megtörténjen az, amiben megállapodott a csapat, és működjenek és fejlődjenek a folyamatok.

A csapatvezető a Budapest School rendszerében nem az, aki
meg-\linebreak
mondja, ki mit csináljon, nem osztja, ellenőrzi vagy felügyeli a feladatokat, nem rúg ki, és nem vesz fel embereket, hanem szolgálja a csapatot (servant leadership) azzal, hogy segíti a megállapodásokat betartani: facilitál, moderál, szintetizál, kísér, kérdez. A csapatvezető megválasztásához, mint minden szerep megválasztásához, a csapat minden tagjának hozzájárulása szükséges.

% Strukturál, összeszervez, koordinál, levezet, emlékeztet.

%Learning architect
%ne legyen bystander
%Ő sem autoritas,  hanem javasol, tematizál, szempontot hoz.

%az, aki arról beszél, hogy a gyerekek igyényeinek milyen struktúrák, keretek, %csoportbontások, modulok, tevékenyslg felelnek meg leginkább 

\subsection{Mi van, amikor egy csapat nem tud döntést hozni,\\ együttműködni?}

Amikor a csapattagok úgy érzik, hogy nem tudnak mindenki számára elfogadható döntéseket hozni, nem haladnak, vagy megjelentek a játszmák, és ezért már nem tudják a csapat célját szolgálni, akkor konfliktus, feszültség alakul ki, aminek feloldásához segítséget hívhatnak be a szervezet töb\-bi részétől. %(lásd .~fejezet).

A csapat folyamatos harmóniájáért folyamatosan dolgozni kell, ahogy az egészségünk megőrzése és a problémák megelőzése is napi rutinná kell hogy váljon. Ezért a Budapest School csapatainak erősen ajánlott a rendszeres visszajelzés, visszatekintés (retrospektív) és a team coaching.

\section{A közösségi lét szabályai}
\label{sec:kozossegi_elet}
A Budapest School egy közösségi iskola: a gyerekek együtt tanulnak és alkotnak, a tanárok csapatokban dolgoznak és a szülők is egy elfogadó közösség részének érezhetiek magukat. A tagok -- a tanárok, a gyerekek, a szülők, és az adminisztrátorok -- azért csatlakoznak a közösséghez, mert itt szeretnének lenni és itt érzik magukat boldognak, egészségesnek és hasznosnak. A közösség azért fogad be új tagokat, hogy nagyobb, erősebb közösség tudjunk lenni.

\paragraph{Alap csoportok} A Budapest School iskola kisebb mikroiskolák
hálózataként működik (lásd \aref{sec:mikroiskola} fejezetet). Ez a gyerekek és tanárok elsődleges közössége: gyerekként ez az a közösség, akikkel együtt járok iskolába, együtt alakítom az iskolámat, tanárként ez az a csapat, akikkel együtt dolgozom, hogy létrehozzuk, tarsuk és fejlesszük a mikroiskola gyerekeinek tanulási környezetét.

A tanulás egysége a modul, amiről \aref{sec:modulok} fejezet részletesen ír. Egy modul csoportjában a közös érdeklődésű és célú gyerekek tanulnak együtt, mert együtt boldogabban, hatékonyabban el tudják érni a céljukat.

\paragraph{Saját szabályok}
\label{sec:sajat_szabalyok}
A mikroiskola és akár egy-egy modul kereteit, szabályait a résztvevők alakítják ki. Pontosabban a tanárok \footnote{Mikroiskola esetén a tanulásszervező tanárok, modulok esetén a modulvezető tanárok és a tanulásszervező tanárok közösen.} felelőssége és feladata, hogy mindenk tanár és gyerek számára elofgadható, betartható, releváns és értelmes szabályok legyenek. Mind a gyerekek, mind a tanárok javasolhatnak új szabályokat, vagy szabályváltoztatást. Minden egyes gyereknek és tanárnak hozzájárulását kell adnia minden javaslathoz, mondván: ,,it's good enough for now and safe enough to try", azaz, ,,elég jónak és biztonságosnak tartom, hogy kiprobáljuk'' a szabályokat.

Ebből következik, hogy minden mikroiskolának saját házirendje, szabályai lehetnek, és modulonként alakulhat, hogy mit, mikor, hogyan és kivel csinálnak a résztvevők.

Szabályok alakításának elsődleges szándéka mindig az legyen, hogy hogy a gyerekek fejlődése, tanulása, alkotása és a közösség működése egyre jobb legyen.

\subsection{Konfliktusok, feszültségek kezelése}
\label{sec:konfliktusok_kezelese}
Tudjuk, hogy a Budapest School szereplői, a gyerekek, a tanárok, a szülők, a pedagógia program, a kerettanterv, a fenntartó, a szomszédok, az állami hivatalok  között feszültségek és konfliktusok alakulhatnak ki, mert különbözőek vagyunk, különbözőek az igényeink. A feszültségekre és a konfliktusokra a Budapest School olyan lehetőségként tekint, amelyek együttműködésen alapuló megoldása építi a kapcsolatot, és segíti a fejlődést.

Minden vágy, ötlet, szándék, cél, viselkedés közötti különbség, ha az valamelyik félben negatív érzéseket kelt, feszültséget és konfliktus okozhat. Ebbe beletartozik az is, ha valaki nem azt és úgy csinálja, ahogy nekünk erre szükségünk van, vagy ha bármilyen okból nem érezzük magunkat biztonságban, vagy más univerzális emberei szükségletünk \citep{rosenberg2003nonviolent} nem elégül ki.

Konfliktus alakulhat ki a gyerekek, szülők és tanárok között bármilyen relációban és adódhatnak egyéb, belső konfliktusok, nehézségek is akár a család, akár a Budapest School életében, amelyek kihathatnak a közösségi kapcsolatainkra.

\subsubsection{A BPS konfliktusok feloldása}

Az iskola a partnerségen alapuló szervezetében nem autoritások, főnökök, hivatalok, bírók oldják meg a konfliktusokat, hanem egyenrangú társak. Az iskola feltételezi, hogy a felek tudnak gondolkodni, következtetni, felelősséget vállalni a döntéseikért és cselekedetikért.\footnote{Kisgyerekeket mentoraik és szüleik reprezentálnak.}

\paragraph{Alapértékek}
Ahhoz, hogy tényleg partneri kapcsolatban, egyenrangú felekként tudjunk konfliktusokat, problémákat megoldani, érdemes közös értékeket elfogadni.
\begin{itemize}
      \item Először a saját változásunkon dolgozunk, mert nem nagyon tudunk mást embert megváltoztatni, csak magunk változásáért lehetünk felelősek.
      \item Felelősséget vállalunk gondolatainkért, hiedelmeinkért, szavainkért és viselkedésünkért.
      \item Nem pletykálunk, szóbeszédet nem terjesztjük.
      \item Nem beszélünk ki embereket a hátuk mögött.
      \item Félreértéseket tisztázunk, konfliktusokat felszínre hozunk.
      \item Személyesen, 1-on-1 beszélünk meg problémákat, másokat nem húzunk be a problémába.
      \item Nem hibáztatunk másokat a problémákért. Amikor mégis, akkor az egy jó alkalom arról gondolkozni, hogy miként vagyunk mi is része a problémának, és kell a megoldás részévé vállnunk.
      \item Erősségekre több figyelmet fordítunk, mint a gyengeségekre, és a lehetőségekről, megoldásokról többet beszélünk, mint a problémákról.
\end{itemize}

\paragraph{Feszültségre felszínre hozása}
Olyan módszereket, folyamatokat, szabályokat, szokásokat kell kialakítani minden közösségben, hogy legyen tere és ideje a feszültségeket előhozni.
\begin{itemize}
      \item Iskolai csoportok rendszeresen kezdik a napjukat egy bejelentkező körrel, ahol van lehetőség a feszültségeket is felhozni.
      \item A csoportok rendszeresen tartanak retrospektív gyűlést, ahol értékelik, mi volt jó és nem annyira jó egy vizsgált időszakban.
      \item Évente legalább kétszer a tanárok egymásnak, a szülők a tanároknak, a gyerekek a tanároknak, a tanárok a gyerekeknek visszajelzést adnak szervezett formában.
      \item A gyerekek a mentorukkal rendszeresen találkoznak, ahol teret kapnak a felmerülő feszültségek.
\end{itemize}

\paragraph{Megbeszélés}

A Budapest School közösségének valamennyi tagja (a tanárok, gyerekek, szülők, adminisztrátorok, iskolát képviselő fenntartó) vállalja:

\begin{itemize}

      \item A közösség mindennapjaival kapcsolatos konfliktusok esetén elsőként az abban érintett személynek jelez közvetlenül.
      \item Személyes kritikát mindig privát csatornán fogalmazza meg először, ha kell, akkor segítő bevonásával.
      \item   Bármelyik fél jelzése esetén lehetőséget biztosít arra, hogy a vitás kérdést megbeszéljék közvetlenül, a folyamatban részt vesz.
      \item
            Szakítás, kilépés, lezárás előtt legalább három alkalommal megpróbál egyeztetni.
      \item
            Az egyeztetésre elegendő időt hagy, amely legalább 30 nap vagy -- amennyiben több időre van szükség -- a másik féllel megállapodott idő.
      \item
            Teljes figyelemmel, nyitottsággal, a probléma megoldására fókuszálva igyekszik feloldani a konfliktust, és közösen megoldást találni a problémára.
\end{itemize}

Összefoglalva: ha problémánk van egymással, akkor azt megbeszéljük. Nem okozunk egymásnak meglepetést, mert vállaljuk, hogy rögtön elmondjuk egymásnak konfliktusainkat.

\paragraph{Közvetítő bevonása}

Ha úgy érezzük, hogy a személyes egyeztetés nem vezetett megoldásra, a tárgyalást külső segítség bevonásával folytatjuk. Ez lehet egy másik csoporttag, egy tanár, vagy egy teljesen külsős meditátor.

Amikor bármely fél közvetítőt kér, akkor a másik ezt elfogadja. Nem mondhatjuk azt, hogy  \emph{,,de hát mi magunk is meg tudjuk oldani a konfliktust''}.

\paragraph{Eszkalálás}

Ha két fél nem tudja megoldani a konfliktust, akkor kérhetnek segítséget a \texttt{segitseg@budapestschool.org} címen, amire 48 órán belül kell válaszolnia. Az email kezeléséért az iskola igazgató felel.

\paragraph{Megállapodások}

Ha az egyeztetés, tárgyalás és közvetítő bevonása során sikerül valamilyen megoldást vagy a megoldáshoz vezető folyamatot egyeztetni, a felek \emph{megállapodnak} abban, hogy ki mit tesz, vagy milyen szabályokat alakítanak ki, illetve, hogy mennyi időt adnak egymásnak, hogy kipróbálják, sikerült-e feloldani a konfliktust. Segíteni szokott a kérdés, hogy \emph{most megállapodtunk vagy csak beszéltünk róla?}, hogy minden fél számára világos legyen a megállapodás.

Nagyobb konfliktusok esetén jó gyakorlat, és bármely fél kérheti, hogy írásban is rögzítsék a megállapodást. Ez lehet egy papirfecni is, vagy egy email. Lényege nem a formátum, hanem  hogy minden fél emlékezzen a megállapodásra.

\paragraph{Egyeztetés sikertelen}

Ha a felek között az egyeztetés sikertelen volt, vagy a megoldási javaslat nem működött, a felek ezt írásban meg kell, hogy állapítsák. Erre azért van szükség, hogy egyetértés legyen abban közöttük, hogy értik, a másik fél sikertelennek érzi az egyeztetést.

\paragraph{Lezárás}

Az egyeztetés sikertelensége esetén elengedjük egymást. De ez a végső megoldás.

\subsection{Kiemelt konfliktusok}

\subsubsection{Gyerek-gyerek konfliktus}

A mindennapokban a gyerekek akarva akaratlanul belecsöppennek olyan hely\-ze\-tek\-be, amikor a közösségben vagy interperszonális kapcsolataik során megborul a mindennapi egyensúly. Ezekben a konfliktushelyzetekben leginkább a felborult egyensúly helyreállítására törekszünk \emph{resztoratív konfliktusfeloldási technikával}.

A resztoratív konfliktusfeloldás alaptézise az, hogy minden ilyen megborult egyensúlyi állapot egy lehetőség valami megújítására, újragondolására. A résztvevők egy külső személy segítségével (általában a jelenlévő tanár) együtt alakítják a megoldást, egészen addig, amíg az eredmény mindenki számára a konfliktus feloldását jelenti, azaz a megborult egyensúly helyreállítását.

A folyamat során a konfliktusban résztvevő összes személy elmondja az érzéseit, meglátását a felmerülő helyzettel kapcsolatban, valamint az én-közléseken túl a szükségleteikről is beszélnek. Ezen szükségletek képezik a megoldás alapját, azaz ezeket egy tető alá hozva feloldhatjuk a fennálló konfliktust. Ilyenkor mindig először azokat a pontokat keressük meg, amelyekben egyetértenek a résztvevők, hiszen ez egy közös alapot szolgáltat arra, hogy a valódi feloldást megtalálhassuk.

Fontos, hogy a beszélgetésben az összes fél hallassa a hangját, és meg is legyen hallgatva. Az értő figyelem kompetenciája is fejlődik ezen módszer alkalmazása során, például, ha az érintett felek elmondják, hogy mit hallottak meg abból, amit a másik elmondott.

Előfordulhat, hogy ez a folyamat nem egyből a konfliktus után indul el: ha a résztvevők beleegyeznek, akkor a beszélgetés elhalasztható, de lehetőleg még aznap történjen meg.

\subsubsection{Gyerek-iskola, tanár-szülő konfliktus}

Minden gyereknek van egy mentora. A szülők számára a mentor az elsődleges kapocs az iskola felé. Ezért, ha a szülőben jelenik meg egy feszültség, akkor elsődlegesen a mentornak jelez. Ugyanígy, a mentor közvetíti a család felé a gyerekkel kapcsolatos feszültségeket.

Ha egy gyerek, vagy szülő úgy érzi, hogy egy gyereknek nem jó az iskolai élménye, például nem tanul eleget, vagy kiközösítik, vagy csak nem szeret bemenni, akkor feladata, hogy rögtön beszéljen a mentor tanárral.

Ha nem sikerül a mentorral megbeszélni a konfliktust és megoldást találni, akkor a szülőnek is lehetősége van segítőt behívni, aki lehet egy másik tanár, másik szülő, az iskolaigazgató vagy bárki, akinek képességeiben bízik.

Előfordulhat, hogy a tanárok vagy az iskola úgy érzik, hogy egy gyereknek nem tesz jót a Budapest School közössége, vagy a hozzáállása súlyosan zavarja vagy sérti a Budapest School közösséget vagy azok tagjait. Az is lehet, hogy a tanárok vagy a Budapest School a szülővel való kapcsolatot érzik konfliktus vagy feszültség forrásának. Ilyen esetben ugyanígy le kell folytatni a konfliktuskezelés folyamatát és megpróbálni feloldani a feszültséget. Ennek sikertelensége esetén az iskola jelzi a családnak írásban, hogy el fog válni.

\subsubsection{Pedagógia program, kerettanterv be nem tartásával kapcsolatos konfliktusok}

Amikor egy tanár, egy gyerek vagy egy mikroiskola nem tartja be a kerettantervet, a pedagógia programot, a házirendet, vagy egyéb közösen megalkotott szabályokat, akkor konfliktus alakul ki közte és az iskola között. Ilyenkor szintén a konfliktuskezelés folyamatát kell lefolytatni.

\subsubsection{Évfolyamszint-lépéssel, osztályzatra való átváltás konfliktusai}
Szülő-iskola konfliktusainak nagy része sok iskolában az osztályzatokkal és a bizonyítvánnyal kapcsolatos. A BUdapest School iskolában a gyerekek maguk kérelmezek az évfolyamszint-lépés és szükség esetén az osztályzatra való átváltást. Maguk tesznek javaslatot arra, hogy mikor és hány évfolyamot lépjenek és hogy milyen osztályzat kerüljön a bionyítványba. A bírálók ezt efogadják vagy elitasítják. Amikor a gyerek/szülő kérelmét a bírálók elutasították, akkor konfliktus alakul ki. Ilyenkor is az előbbeikben leírt elveket, folyamatokat kell alkalmazni.



\chapter{Tanítás mestersége}
% art of teaching

\section{Különböző tanári szerepek}
\label{sec:tanarok}

A Budapest Schoolban a gyerekek azokat a felnőtteket tekintik tanáruknak, akik
minőségi időt töltenek velük, és segítik, támogatják vagy vezetik őket a
tanulásukban. Több szerepre bontjuk a tanár fogalmát: a gyerek egy (és csak
egy) felnőtthöz különösen kapcsolódik, a \emph{mentortanárához}, aki rá
különösen figyel. Ezenkívül a gyerek tudja, hogy a mikroiskola mindennapjait
egy tanárcsapat, a \emph{tanulásszervezők} határozzák meg, ők vezetik az
iskolát.  A foglalkozásokon megjelenhetnek további tanárok, a \emph{modulvezetők},
akik egy adott foglalkozást, szakkört, órát tartanak.

Szervezetileg minden mikroiskolának van egy állandó \emph{tanárcsapata}, a
tanulásszervezők. A tanulásszervezők legalább egy tanévre elköteleződnek, szemben a
modulvezetőkkel, akik lehet, hogy csak egy pár hetes projekt keretében vesznek részt a
munkában.
A tanulásszervezők általában mentorok is, de nem minden esetben. Nem lehet
mentor az, aki a gyerek mikroiskolájában nem tanulásszervező, mert nem lenne
rálátása a mikroiskola történéseire. Egy tanulásszervező lehet több
mikroiskolában is ebben a szerepben, és így mentor is lehet több mikroiskolában.

\paragraph{Mentor}
Minden gyereknek van egy \emph{mentora}, aki a saját céljainak
megfogalmazásában és
a fejlődése követésében segíti. Minden mentorhoz több gyerek tartozik, de nem
több, mint 12. A mentor együtt dolgozik a mikroiskola többi tanulásszervezőjével, a
szülőkkel és az általa mentorált gyerekekkel. A mentor segít az általa
mentorált gyereknek, hogy a tantárgyi fejlesztési célok és
a
saját magának megfogalmazott saját célok között megtalálja  az egyensúlyt, és segít megalkotni a
gyerek \emph{saját
  tanulási tervét}.

A mentor a kapocs a Budapest School, a szülő és a gyerek között.

\begin{itemize}
  \item Képviseli a Budapest Schoolt, a mikroiskola közösségét.
        \begin{itemize}
          \item Ismeri a Budapest Schoolt, a lehetőségeket, a tanulásszervezés
                folyamatait.
          \item Együtt tanul más Budapest School-mentorokkal, együtt dolgozik a
                tanártársaival.
        \end{itemize}

  \item Ismeri, segíti, képviseli a gyereket.
        \begin{itemize}
          \item  Tudja, hol és merre tart a mentoráltja, ismeri a képességeit,
                körülményeit, szándékait, vágyait.
          \item    Segít a saját célok elérésében, felügyeli a haladást.
          \item    Megerősíti a mentoráltjai pszichológiai biztonságérzetét.
          \item   Visszajelzéseket ad a mentoráltjainak.
          \item    Segít abban, hogy az elért célok a portfólióba kerüljenek.
          \item    Összeveti a portfólió tartalmát a tantárgyak fejlesztési
                céljaival.
        \end{itemize}

  \item Együtt dolgozik, gondolkozik a szülőkkel, képviseli igényüket a
        közösség felé.
        \begin{itemize}
          \item Erős partneri kapcsolatot épít ki a szülőkkel, információt oszt meg
                velük.
          \item Segít a gyerekekkel közös célokat állítani.
          \item A szülő számára a mentor az elsődleges kapcsolattartó a különféle
                iskolai ügyekkel kapcsolatban.
        \end{itemize}

\end{itemize}

A mentor egyszerre felelős a mentorált gyerek előrehaladásának segítéséért,
és
közös felelőssége van a mentortársakkal, hogy az iskolában a lehető legtöbbet
tanuljanak a gyerekek. A mentor folyamatosan figyelemmel követi az egyéni
tanulási tervben megfogalmazottakat, és ezzel kapcsolatos visszajelzést ad a
mentoráltnak és a szülőnek.

\paragraph{Tanulásszervező}
Csoportban dolgozó, iskolaszervező, strukturáló tanár. Egy mikroiskola
állandó tanári
csapatát 2--7 tanulásszervező alkotja, akik egyedileg meghatározott szerepek
mentén a mikroiskola mindennapjainak működtetéséért felelnek. Minden mentor
tanulásszervező is. A tanulásszervezők tarthatnak
modulokat, sőt kívánatos is, hogy dolgozzanak a gyerekekkel, ne csak
szervezzék az életüket.
Ők rendelik meg a külső modulvezetőktől a munkát, ilyen értelemben a
tanulási utak projektmenedzserei.

\paragraph{Modulvezetők}

Bárki lehet modulvezető, aki képes akár egy egyetlen alkalommal történő, vagy
éppen
egy egész trimeszteren át tartó tanulási, alkotási folyamatot vezetni. Ők
általában
az adott tudományos, művészeti, nyelvi vagy bármilyen más terület szakértői.

A modulokat a tanulásszervezők is vezethetik, de külsős, egyedi megbízással
dolgozó szakemberek is megjelennek modulvezetőként. Modulvezető lehet bárki,
akiről az őt megbízó tanárcsapat tudja, hogy képes gyerekek folyamatos
fejlődését és egy tanulási cél felé való haladását segíteni. A moduláris
tanmenettel \aref{sec:modulok}. fejezet foglalkozik.

\section{A tanárok feladatai}\label{sec:tanarfeladatok}

A Budapest School szervezetben a felelősségek és feladatok körét egy-egy \emph{szerep} formájában határozza meg a csapat. A szerep által kijelölt területen (domain) belül önállósága van a szerepet betöltő (role keeper) csapattagnak. Egy-egy munkatárs több szerepben is lehet, és a szerepeket és a szerepkiosztást úgy kell változtatni, hogy azok a szervezet céljait, jelen esetben a gyerekek tanulását jobban szolgálják.

Például a mikroiskola teendőinek koordinálása, az iskolai adminisztráció is egy szerep, nem egy önálló munkakör.  Mentortanárnak lenni, az egy másik szerep, és az is keverhető más szereppel, így egy iskolai adminisztrátor is lehet men\-tor, vagy másképp fogalmazva, egy mentortanár kaphat adminisztrációs szerepet is.



\subsection{Tanulásszervező}\label{sec:tanulasszervezo}
\begin{itemize}
    \item Kiszámítható, átlátható rendszert épít, ahol a szülők és a gyerekek is biztonságban, informálva, bevonva érzik magukat.
    \item A tanulási célokkal rezonáló modulokat hirdet meg, szervez le.
    \item A szülők és a gyerekek is érzik, értik, hogy ,,történik'' a tanulás.
    \item A modulvezetőknek megad minden szükséges információt, kontextust, így ők hatékonyan tudják végezni a munkájukat.
    \item Megadja a résztvevőknek az informált választás lehetőségét.
\end{itemize}

\subsection{Mentor}\label{mentor}

\begin{itemize}
    \item
          Az első trimeszter alatt a mentor megismeri mentoráltját, személyiségjegyeit, képességeit, érdeklődését, motivációit. Megismeri a családot.
    \item
          Megállapodik a családdal a kapcsolattartás szabályaiban.
    \item
          Bevezeti a családot a Budapest School rendszerébe.
    \item
          Trimeszterenként a mentorált gyerekkel és szülőkkel egyetértésben kialakítja, majd folyamatosan monitorozza a mentorált gyerek tanulási céljait.
    \item
          Képviseli mentoráltját a többi tanár felé.
    \item
          A szülőkkel rendszeresen információt oszt meg, elérhető, asszertíven kommunikál, tiszta, mérhető megállapodásokat köt.
    \item
          A szülőkkel erős partneri kapcsolatot épít.
    \item
          Amikor a gyereknek külső fejlesztésre, mentorra, tanárra, trénerre van szüksége, akkor a családot segíti a megfelelő segítő felkeresésében, a külsőssel kapcsolatot tart, és konzultál a mentoráltja haladásáról.
    \item
          Minimum kéthetente találkozik mentoráltjával. Követi, tudja, hogy a gyerek hogy van az iskolában, a családban, az életben.
    \item
          Mentorként nyomon követi, monitorozza a mentoráltjai fejlődését, és szükség esetén továbblendíti, inspirálja őket.
    \item
          A mentor biztosítja, hogy a mentorált gyerek portfóliója friss legyen.
    \item
          Rendszeresen reflektál a mentorált gyerekkel együtt annak tanulási céljaira és haladására.
    \item
          Segíti a mentorált modulválasztását.
    \item
          Visszajelzést gyűjt és ad a mentorált gyerek fejlődésére.
\end{itemize}

\subsection{Modulvezetők}\label{modulvezetux151k}

\begin{itemize}

    \item
          Izgalmas, érdekes foglalkozásokat tart, amire felkészül, és amiben a gyerekeket flow-ban tudja tartani.
    \item
          Amikor a gyerekek vele vannak, akkor figyelnek, fókuszálnak, koncentrálnak, dolgoznak, tanulnak.
    \item
          Kedvesen és határozottan vezeti a csoportot, figyel arra, hogy mindenkit bevonjon.
    \item Változatos, gazdag módszertani eszköztárából mindig a foglalkozáshoz megfelelő módszert tudja elővenni.
    \item  A gyerekek kritikai gondolkozását foglalkozásai fejlesztik.
    \item A gyerekek sokat dolgoznak csoportban foglalkozásain.
    
\end{itemize}

\subsection{A közös szerepek}

Minden BPS-tanulásszervező, -modulvezető és -mentor egyszerre  \emph{Csapattag}, \emph{EQ ninja}, \emph{Change Agent} és \emph{SNI-szakértő}. 

\paragraph{Csapattag}

\begin{itemize}

    \item
          Rendszeresen jelen van a tanári csapat megbeszélésein.
    \item
          El lehet érni telefonon vagy online, a csapattal megállapodott kereteken belül.
    \item
          Feladatokat vállal magára, és azokat megbízhatóan, határidőre végrehajtja.
    \item
          Kooperatív és támogató a közös munkák, ötletelések, megbeszélések alatt, képviseli a saját nézőpontját, gondolatait, érzéseit, miközben a csapat és a többiek igényeire is figyel.
    \item
          Kifejezi támogatását, ellenérveit és javaslatait a jobb megoldás érde\-kében.
    \item
          A visszajelzést keresi, a kritikát jól fogadja, és megfontolja, átgondolja a lehetséges változtatásokat.
    \item
          A kollégák fejlődését segíti rendszeres visszajelzésekkel.
\end{itemize}

\paragraph{BPS-tag}

\begin{itemize}

    \item
          Rendszeresen jelen van heti hétfőkön, havi szombatokon.
    \item
          Részt vesz a mikroiskolájának és a Budapest Schoolnak az építésében.
    \item
          Proaktívan alakít ki rendszereket, folyamatokat, és a legjobb gyakorlatokat megosztja BPS-szinten.
    \item
          Közös témákban aktív, hozzászól, alakítja a véleményével és tudásával a BPS rendszerét.
\end{itemize}

\paragraph{Érzelmi intelligencia (EQ) ninja}

\begin{itemize}

    \item
          Gyerekek pszichológiai biztonságérzetét megerősíti.
    \item
          Olyan visszajelzéseket ad, amelyek az erőfeszítésekre, a belefektetett energiára, munkára és a jövőbeni fejlődésre fókuszálnak (growth mindset), pozitív megerősítést alkalmaz, épít a gyerek erősségeire, és egyértelműen megfogalmazza, mit tehetne másként.
    \item
          Értő figyelemmel van jelen, kedves, nyitott, és figyel a gyerekekre.
    \item Munkája során gondoskodik a gyermek személyiségének, képességeinek, kompetenciájának fejlődéséről.
    \item Figyelembe veszi a gyerekek egyéni képességeit, adottságait, fejlődésének ütemét, szociokulturális helyzetét.
    \item Előmozdítja a gyerek erkölcsi fejlődését.
    \item Segíti a gyerekeket jobban működni a csoportban.

\end{itemize}

\paragraph{SNI-szakértő}

\begin{itemize}

    \item
          Jól tudja kezelni az egyéni, különleges bánásmódot, speciális nevelést igénylő gyerekeket a csoportban.
    \item A különleges bánásmódot igénylő gyermekek családjával, a nevelést, oktatást segítő más szakemberekkel együttműködik, hogy a gyerek támogatása minden esetben megtörténjen.
    \item
          Akinek külső segítségre van szüksége, azoknak a szüleivel ezt proaktívan egyezteti, és menedzseli a folyamatot.
\end{itemize}

\paragraph{Change agent}\label{change-agent}

\begin{itemize}

    \item
          Nehéz helyzeteken is könnyen továbblendül, vannak módszerei arra, hogyan töltse magát, és ezeket használja is.
    \item
          Azt keresi, mire lehet hatása, mit lehet eggyel jobban csinálni, és ebben akciókat tesz.
    \item
          Megünnepli az előrelépéseket, közös sikereket.
\end{itemize}




\part{Helyi tanterv}
\chapter{Helyi tanterv}
\label{az-iskola-helyi-tanterve}
% academic and characters

\section{A kerettanterv és a pedagógiai program\hfill\break  kapcsolata}
\label{sec:tanterv-program}
Budapest School Általános Iskola és Gimnázium az oktatásért felelős miniszter által jóváhagyott Budapest School Kerettanterv alapján működik. Az iskola munkatársai a kerettantervet, a pedagógiai programot, sőt még a szervezeti és működési szabályokat is együtt, egy egységként kezelik. Ez a Budapest School Model, aminek fő célja, hogy a gyerekek úgy tudják azt és akkor tanulni, amit szeretnek, vagy amire szükségük van, hogy közben a tanárok, szülők és a teljes társadalmunkat képviselő hivatalok számára is kiszámítható, tervezhető, biztonságos tanulási környezetet biztosít az iskola.

A minimális tantervet, azaz, hogy mit tanulnak mindenkép a gyerekek, a kerettanterv határozza meg kötelezően elérendő tanulási eredmények félévenkénti tantárgyi bontásában. Ettől az iskola nem tér el. Óraszámokban is követi a kerettanterv által medagott minimális óraszámokat. Az iskolánk a gyerekeknek azt a kérdést teszi fel, hogy ezen felül mit akarnak tanulni. Az iskolánk akkor tud igazán segíteni, ha valaki a minimális elváráson túl, valamivel mélyebben, alaposabban akar foglalkozni.

\section{Tantárgyak}
\label{sec:tantargyak}
A Budapest School a ma gyerekeinek kínál olyan oktatást, ami segíti felkészíteni őket a jövő kihívásaira. Információs társadalmunk legnagyobb kihívása az adaptációs képességünk fejlesztése, ez az alapja annak, hogy képesek legyünk eligazodni a folyamatosan változó, komplex világunkban. A tanulásunk célja, hogy boldog, hasznos és egészséges tagjai legyünk a társadalomnak. Iskolánkban a tanulás három rétege, a tudásszerzés, a megtanultakat elmélyítő önálló gondolkodás és az aktív alkotás egyszerre jelennek meg.

A kerettanterv a célok eléréséhez a miniszter által kiadott kerettantervek tantárgyi struktúráját használja  a moduláris tanulás tartalmi keretezéséhez. A keretezésen azt értjük, hogy a tantárgyak tartalma határozza meg, hogy mivel kell mindenképp a Budapest School iskolákban foglalkozni, mit kell mindenképp megtanulni. 
Az egyes modulok ezen tantárgyak tanulási eredményeinek elérését támogatják.

\subsection{Tanulási eredmények -- a formális tanulás alapegységei}
\label{sec:tanulasi_eredmenyek}
A kerettanterv a tantárgyak témaköreit, tartalmát és követelményeit \emph{tanulási eredmények} halmazával adja meg, ezzel igazodva az Nkt.~5.~§ (5) pontjához. A tanulási eredmények (learning out\-comes) tudás, képesség, kompetencia, attitűd kontextusában meghatározott kijelentések arra vonatkozóan, hogy a tanulónak mit kell tudnia, mit kell értenie, és mire legyen képes, miután lezárt egy tanulási folyamatot, függetlenül attól, hogy hol, hogyan, mikor szerezte meg ezeket a kompetenciákat \citep{learning_outcomes}.  Tanulási eredmény a kerettanterv szellemében minden lehet, amit a gyerek egy tanulási folyamat során elsajátított és ezt demonstrálni tudja.

Az eredmény eléréséhez vezető út a modulokon keresztül történik, és a tanulás folyamata történhet az iskolában vagy azon kívül, lehet formális, non-formális vagy informális.
Az egyes modulok különféle tanulási eredmények elérését is támogathatják, ezzel több tantárgy részcéljait is teljesíthetik.

A tanulási eredmények több funkciót látnak el a kerettantervben.

\begin{itemize}

      \item A kerettanterv évfolyamonként meghatározza az adott tantárgy teljesítéséhez elérendő tanulási eredményeket. Egy gyerek akkor  léphet egy tantárgyból évfolyamszintet, ha a tantárgyhoz tartozó követelményeket teljesítette.

      \item A tanulási eredmények a modulok (és így a mindennapokban szervezett foglalkozások, órák stb.) építőelemei. Egy-egy modul célját a  tanulásszervezők az elérendő tanulási eredmények	összeválogatásával és saját célokkal, érdeklődéssel kiegészítve adják meg, figyelembe véve az életkori  sajátosságok, az egymásra épülés és az átjárhatóság  követelményeit.
      \item A tanulási eredmények alapján osztályzatok és évfolyamok egy átlátható és egyszerű számítás segítségével megállapíthatóak, ami biztosítja, hogy a Budapest  School tanulója más rendszerben működő iskolába is át tud menni, és a felvételikre is tud jelentkezni.
\end{itemize}

A kerettanterv tantárgyankénti és félévenkénti bontásban adja meg a továbbhaladáshoz elengedhetetlen tanulási eredmények listáját.

A tantárgyi definíciókhoz a miniszter által kiadott kerettanterv ,,elvárt eredmények a tanulási ciklus végén" fejezetek felsorolásait alakítottuk át egységes nyelvezetre, hogy azok valóban kompetenciákat írjanak.

A tantárgyi specifikációk nem térnek ki rész\-letesen a tematikákra. Ez szabadságot ad a tanároknak arra, hogy a tanmenet tekintetében akár jelentős eltérések legyenek addig, amig a miniszter által kiadott kerettanterv mérhető tanulási eredményei teljesülnek. A tanulási eredmény alapú szabályozás folyamatos visszacsatolást tud adni a tanulónak és a tanároknak, megmutatva, melyik tanulási eredményeket kell még elérni a következő szintre való lépéshez.

\subsection{Tantárgyak szerepe a mindennapokban}
A Budapest School iskoláiban a tantárgyak ugyanúgy kapnak szerepet, mint a NAT által definiált kulcskompetenciák, fejlesztési területek: tanár sose mondja azt a gyerekeknek, hogy „most kezdeményezőképességet és vállalkozói kompetenciát fejlesztünk'', hanem a különböző feladatok elvégzése eredményeképp történik a fejlesztés. A Budapest School iskolákban a tantárgyközi tevékenységek vannak előtérben. A tantárgyak a tanulás tartalmi elemeinek forrásai és keretei: a tanulandó dolgok listájaként működnek. Az, hogy milyen csoportosításban történik a tanulás, az a modulvezetőkre van bízva.

A tantárgyak ezért elsősorban a modulok kiírásakor és azok kimeneti értékelésekor jelennek meg, a mindennapok struktúráját, a napi- és hetirendet azonban a modulok adják. Egyes modulok több tantárgy fejlesztési céljainak is eleget tehetnek, több tantárgy tanulási eredményének elérését is célul tűzhetik ki, összhangban a NAT-tal. A tantárgyaknak ezzel együtt fontos célja, hogy segítse a tanulás tartalmi egyensúlyának fennmaradását. A tanulásszervezők, modulvezetők szakképesítése nem köthető a Budapest School tantárgyaihoz, felelősségük, hogy a saját moduljukban megfelelően tudják szervezni a tanulást, és legfőképp, hogy saját moduljuk megtartására alkalmasak legyenek.

\subsection{Modulok és tanulási eredmények}
\label{sec:modulok_es_tanulasi_eredmenyek}
A gyerekek egyik feladata az iskolában, hogy tanulási eredményeket érjenek el. Ezt megtehetik a modulok elvégzésével, vagy más tanulási helyzetekben. A tanulási eredményeket a portfólióban rögzítik. A mentor feladata, hogy folyamatosan kövesse, hogy megfelelő haladás történik-e a portfólióban a tanulási eredmények és a saját célok tekintetében. Az évfolyamszintlépés a portfólióban összegyűlt tanulási eredmények alapján történhet meg.

A modul kecsegtet a gyerekek haladásához releváns tanulási eredményekkel, a gyerekek által meghatározott saját célokkal és olyan kimenettel, amely a portfólióban rögzíthető, legyen az egy alkotás, az elért fizikai vagy szellemi eredmény dokumentációja, vagy egy értékelő visszajelzés. A modulok tehát tartalmaznak tanulási eredményeket, az önálló gondolkodás, szabad alkotás lehetőségét, és teret engednek az alkotásra, létrehozásra.

\paragraph{A modulok különféle tanulási eredmények elérését teszik elérhetővé}

Modulok tervezésekor és összeállításakor a tanulásszervezők a modulvezetővel közösen határozzák meg a modul céljait, de azok meghirdetéséért mindig a tanulásszervezők felelnek. A célok között fel kell sorolni, hogy milyen tanulási eredmények elérését várhatják el a gyerekek a modulon való részvételtől.

Például a 6--8 éves gyerekek számára megtervezett ,,\emph{3d nyomtató használata}'' modul során azon kívül, hogy megismerik a 3d nyomtatás folyamatát, a modul célja, hogy a gyerekek számára elérhetővé tegye a ,,\emph{Kocka, téglatest jellemzőit ismeri, képes őket létrehozni.}'' (Matematika tantárgy, 4. évfolyam 2. félév) tanulási eredményt is.

Lehetőség van egy modul esetében több tantárgyból való tanulási eredmény kiválasztására, ezzel biztosítva az interdiszciplinaritást, valamint a Budapest School tantárgyi fejlesztési céljaihoz való integrált kapcsolódást.

A tanulási eredmények egy időbeni egymásra épülést feltételeznek, melyben azonban van lehetőség előre- és hátrafele is lépni. Előre, amennyiben a modul meghirdetésekor az arra jelentkező gyerekcsoportnál a megfelelő előkészítés megtörtént, hátra, amennyiben ezt ismétlés/felzárkóztatás jelleggel szükségesnek ítéli a mentor vagy a modult szervező, vezető. Vagyis akkor foglalkozzon egy gyerek a 10~000-es számkörrel, ha a 100-as számkört már begyakorolta. Az egymásra épülésért a modult meghirdető tanulásszervező felel. A példát folytatva a 3d nyomtató használata modul lehetővé teszi, hogy a gyerek elérje a következő eredményeket is: \emph{,,Ismeri a számítógép
      részeinek és perifériáinak funkcióit, azokat önállóan használja.''}
(Harmónia, Informatika, 5. évfolyam 1. félév), és  \emph{,,Használati utasításokat
      értő módon olvas és tart be.''} (Harmónia, Életvitel, 4. évfolyam 2. félév)

\paragraph{Új tanulási eredmények}

A gyerekek olyan tanulási eredményt is elérhetnek, ami a modulok céljai között eredetileg nem volt megadva, mert

\begin{itemize}
      \item lehetőségük van egyénileg is tanulni;

      \item tanulási eredményekkel járnak a projektek, az iskolai lét, a közösségi élet és még számos informális és non-formális tanulási helyzet;

      \item egy modul során is alakulhatnak előre nem tervezett helyzetek, amik hozzásegíthetik a gyerekeket tanulási eredmények eléréséhez.
\end{itemize}

Az újonnan létrejövő tanulási eredmények is bekerülnek a portfólióba.

\paragraph{Tanulási eredmények dokumentációja}

Minden modul dokumentálásra kerül, hogy annak célja, elért eredményei nyilvánosak legyenek a Budapest School valamennyi mikroiskolája számára, és ha szükséges, újra meg lehessen hirdetni. A tanulási eredmények egy, a modulhoz kapcsolódó terv-tény összehasonlítás alapján kerülnek meghatározásra. Az elért eredmények újra elérhetőek, amennyiben a folyamatos fejlődés biztosítva van.

\paragraph{Egységes modulok egyedi alkalmazása}

Egy modul elvégzésével egy-egy gyerek más tanulási eredményt is elérhet.

\begin{itemize}
      \item
            Működhet a differenciálás, tehát nem minden gyerek ugyanazt és ugyanúgy csinálja a foglalkozásokon. Egy modulban tud együtt	tanulni az a gyerek, aki még ,,\emph{Ismeri az írott és nyomtatott  betűket''} eredményért dolgozik, és az, aki ,,\emph{Jelöli helyesen a j	hangot 30--40 begyakorolt szóban''.}
      \item
            A modulnak része lehet testre szabható sáv. Például egy tudományos kísérletező modulban néhány gyerek a rövid távú memória és a	fáradtság kapcsolatáról kutat, a másik csoport az esőzés és a	közlekedési dugók kialakulása közti kapcsolatot vizsgálja. Minden  gyerek elérheti a ,,\emph{valós folyamatokat képes elemezni a folyamathoz tartozó függvény grafikonja alapján}''  (forrás, Matematika) eredményt, de a ,,\emph{környezettudatos közlekedésszemlélet}'' (forrás, Harmónia)	eredményt is elérheti.
      \item
            Egy-egy gyerek saját tanulási célja érdekében extra lépéseket tehet, és olyan eredményeket is el tud érni, amit mások nem.	Például egy modul végén önálló prezentációt, saját kutatási  tervet vagy egy kész működő modellt alkothat.
\end{itemize}

\subsubsection{Kötelező tanulási eredmények}
\label{sec:kotelezo_tanulasi_eredmenyek}
A kerettanterv kötelező tanulási eredményként definiálja mindazokat az eredményeket, melyek a kötelező érettségi tárgyak teljesítéséhez szükségesek. Ezeket minden mikroiskola elérhetővé kell hogy tegye a gyerekek számára a modulok választékában.

Ezek az 1--4. évfolyamszinteken a miniszter által kiadott kerettantervek \emph{Magyar nyelv és irodalom}, \emph{Matematika}, \emph{Idegen nyelv}, \emph{Testnevelés és sport} tantárgyakból származó tantárgyak tanulási eredményei, és 5.~évfolyamszinttől a \emph{Történelem, társadalmi és állampolgári ismeretek} tantárgy eredményei. További kötelező tanulási eredményként jelennek meg 9.~évfolyamtól a választott érettségi tantárgyhoz kapcsolódó eredmények. Ezek a tanulási eredmények megtalálhatók a kerettanterv tantárgyainak elérhető eredményei között.

\paragraph{Kötelező modulok}
A kerettanterv és a pedagógiai program is előírhat kötelező modulokat a mikroiskolák számára. Ilyenek például a 11.~évfolyamszinten belépő érettségire felkészítő modulok, a minden mikroiskolára egységes pedagógiai program tetszőleges kötelező modult írhat elő. Így lehet biztosítani a kötelező tartalmi elemek és foglalkozás -- például elsősegélynyújtás vagy a nemzetiségekkel való ismerkedés -- elérhetőségét.

\subsubsection{Monitorozás}

Kötelező elérni az eredményeket? Nem tudunk hatalmi szóval tanulásra bírni gyereket, mert lehet, hogy annyira nem akarja, vagy nincs meg hozzá a képessége. A kerettanterv a tanároknak ad keretet. Azonban a fenntartó által üzemeltetett rendszerrel az iskola  monitorozza a haladást, és ha valaki a kötelező tanulási elemekkel nem halad, akkor az iskola erre felhívja a figyelmét. Mivel a többség haladni fog, ezért előre tudja az iskola jelezni, hogy le fog szakadni a többiektől, és túl nagy lesz az évfolyamszint-különbség közöttük. Ezekben az esetekben a mentortanárnak, a gyereknek és a szülőnek reagálnia kell a helyzetre. A fenntartó által működtetett monitorozó és minőségfejlesztő rendszerről \aref{sec:minosegbiztositas} fejezet ír részletesen.


\subsection{Tantárgyi struktúra és óraszámok}
\label{sec:tantargyi_struktura}
\paragraph{Heti óraszámok} 
A Budapest School közösségi tanulási élményeket és modulokat szervez a gyerekeknek, egyúttal lehetőséget ad arra, hogy a gyerekek a közösen kialakított szabályaik mentén, a tanulásszervezők felügyeletével a Budapest School székhelyén vagy egyes telephelyein, vagy más, erre alkalmas tanulási környezetben tartózkodjanak. A közösségben együtt töltött idő tanulásnak, fejlődésnek minősül akkor is, ha az nem egy modulhoz kapcsolódik, hanem az ebéd élvezetéhez, vagy épp a parkban a lehulló falevelek neszének megfigyeléséhez.

A gyerekek, a tanítási szüneteket leszámítva, általában naponta 8 órát tartózkodnak az iskolában.\footnote{Ez alól több kivétel lehet: külső foglalkozások, otthoni, egyéni tanulás. Ezekről külön megállapodást kell kötni a mentorral.} Ezekben az időkben vannak a tanítási órák, foglalkozások, szakkörök, műhelyek. Az egyes mikroiskolák ettől 20\%-ban bármelyik irányban eltérhetnek, ha ez segíti a tanulásszervezők munkáját és a gyerekek fejlődését. Így hetente minimum $5 \cdot 8 \cdot 0,8 = 32$ órát, maximum $48$ órát töltenek az iskolában.

Ennek az 1--4.~évfolyamszinten körülbelül a felét, azaz 18--22 órát, majd később 3--4.~évfolyamonszinten 20--26 órát; az 5--12.~évfolyamszinten pedig kétharmad részét, azaz 24--32 órát töltik a gyerekek előre eltervezett módon, azaz modulokkal. A többi időben a tanárok vezetése és felügyelete mellett szabadon alkotnak, játszanak, pihennek, közösségi életet élnek. Ugyanezek a számok a miniszter által kiadott kerettantervekben 1. évfolyamon 25 és a 10. évfolyamon 36 órát tesznek ki.

\paragraph{Modulok és a tantárgyi óraszámok}
A modulok során több tantárgyi tananyagot is érinthetnek a modul résztvevői. Egy modul így több tantárgyi órát is lefed, ráadásul ez óraszám megtakarítással is jár. 5 óra angolul tartott dráma foglalkozás egyszerre számíthat 5 óra \emph{magyar irodalom és nyelvnek} és 5 óra \emph{idegennyelv} órának. A tantárgyköziségből spórolt óraszámokra a kerettanterv óvatos előírást ad: egy egységnyi idő alatt átlagban legalább 1,25 egységnyi tantárgyi óraszámot kell teljesíteni (szemben az előző példában szereplő kettes szorzóval).

A modulok tantárgyi óraszámát a teljes modul hosszára kell számítani, és nem hetente. Egy összevont természettudományi modul, ami érinti a kémiát és a fizikát is, nem kell, hogy minden héten járjon kémia tanulási eredménnyel. Több tantárgyat lefedő modul esetén a tantárgyi óraszámokat úgy kell számolni, hogy először meg kell állapítani, hogy átlagban az idő hány százalékában foglalkozik a modul egy-egy tantárgy anyagával, majd a modul teljes hosszából becsülhető a tantárgyi óraszám. Például egy \emph{tudományos kísérletezés} modul során az idő 30\%-ban foglalkozunk kémiával, 40\%-ban fizikával és 30\%-ában szociálpszichológiával. A modul egy trimeszteren keresztül tart, kéthetente 4 órában. Ebből számolható a modul teljes hossza, ami itt $\frac{12}{2}x4 = 24$ óra. Ez hetente $24x0,3 = 7,2$ óra kémiának felel meg. 
A példa is mutatja, hogy a \emph{kerettanterv megengedi, hogy nem egészszámú óraszámokkal dolgozzon az iskola}. 

\subsection{Tantárgyi óraszámok trimeszterenként}
A Budapest School modulalapú tanulásszervezéséhez jobban illik trimeszterenként megadni az elvárt óraszámokat, mert ahogy \aref{sec:tanev_ritmusa}.~fejezet is mutatja, nem minden hét ugyanolyan az iskolában. A kerettanterv a miniszter által kiadott kerettantervek óraszámait veszi alapul, annak heti óraszámait szorozza fel kilenccel. 

\begin{landscape}
\begin{table}[]
  \begin{tabular}{l|l|l|l|l|l|l|l|l|l|l|l|l}
  
                                                        & \multicolumn{12}{l}{\textbf{Évfolyamszintek}}                                                                                                                                                           \\ \hline
    \textbf{Tantárgyak}                                          & 1                                     & 2           & 3           & 4           & 5           & 6           & 7           & 8           & 9           & 10          & 11          & 12          \\ \hline
    Biológia - egészségtan                              &     &     &     &     &     &     & 18  & 9   &     & 18  & 18  & 18  \\\hline
    Dráma és tánc/Mozgóképkultúra és médiaismeret       &     &     &     &     & 9   &     &     &     & 9   &     &     &     \\ \hline
    Életvitel és gyakorlat                              & 9   & 9   & 9   & 9   &     &     &     &     &     &     &     & 9   \\\hline
    Ének-zene                                           & 18  & 18  & 18  & 18  & 9   & 9   & 9   & 9   & 9   & 9   &     &     \\\hline
    Erkölcstan                                          & 9   & 9   & 9   & 9   & 9   & 9   & 9   & 9   &     &     &     &     \\\hline
    Etika                                               &     &     &     &     &     &     &     &     &     &     & 9   &     \\\hline
    Fizika                                              &     &     &     &     &     &     & 18  & 9   & 18  & 18  & 18  &     \\\hline
    Földrajz                                            &     &     &     &     &     &     & 9   & 18  & 18  & 18  &     &     \\\hline
    I. idegen nyelv                                     &     &     &     & 18  & 27  & 27  & 27  & 27  & 27  & 27  & 27  & 27  \\\hline
    II. idegen nyelv                                    &     &     &     &     &     &     &     &     & 27  & 27  & 27  & 27  \\\hline
    Informatika                                         &     &     &     &     &     & 9   & 9   & 9   & 9   & 9   &     &     \\\hline
    Kémia                                               &     &     &     &     &     &     & 9   & 18  & 18  & 18  &     &     \\\hline
    Környezetismeret                                    & 9   & 9   & 9   & 9   &     &     &     &     &     &     &     &     \\\hline
    Magyar nyelv és irodalom                            & 63  & 63  & 54  & 54  & 36  & 36  & 27  & 36  & 36  & 36  & 36  & 36  \\\hline
    Matematika                                          & 36  & 36  & 36  & 36  & 36  & 27  & 27  & 27  & 27  & 27  & 27  & 27  \\\hline
    Művészetek                                          &     &     &     &     &     &     &     &     &     &     & 18  & 18  \\\hline
    Technika, életvitel és gyakorlat                    &     &     &     &     & 9   & 9   & 9   &     &     &     &     &     \\\hline
    Természetismeret                                    &     &     &     &     & 18  & 18  &     &     &     &     &     &     \\\hline
    Testnevelés és sport                                & 45  & 45  & 45  & 45  & 45  & 45  & 45  & 45  & 45  & 45  & 45  & 45  \\\hline
    Történelem, társadalmi és állampolgársági ismeretek &     &     &     &     & 18  & 18  & 18  & 18  & 18  & 18  & 27  & 27  \\\hline
    Vizuális kultúra                                    & 18  & 18  & 18  & 18  & 9   & 9   & 9   & 9   & 9   & 9   &     &     \\\hline \hline
    \textbf{Összesen}                                   & 207 & 207 & 198 & 216 & 225 & 216 & 243 & 243 & 270 & 279 & 252 & 234
    

  \end{tabular}
  \caption{A minimális óraszámok trimeszterekre számolva. A táblázatban szereplő számok a miniszter által kiadott kerettanterv óraszámai alapján készültek.}  
  \label{tbl:oraszamok}
\end{table}

\end{landscape}
\paragraph{Heti óraszámok}

A Budapest School közösségi tanulási élményeket és modulokat szervez a
gyerekeknek, egyúttal lehetőséget ad arra, hogy a gyerekek a közösen kialakított
szabályaik mentén tanulásszervezők felügyeletével a Budapest School székhelyén
vagy egyes telephelyein, vagy más, erre alkalmas tanulási környezetben
tartózkodjanak. A közösségben együtt töltött idő tanulásnak, fejlődésnek
minősül akkor is, ha az nem egy modulhoz kapcsolódik, hanem az ebéd
élvezetéhez, vagy épp a parkban a lehulló falevelek neszének megfigyeléséhez.

A gyerekek, a tanítási szüneteket leszámítva, általában naponta 8 órát
tartózkodnak az
iskolában.\footnote{Ez alól több kivétel lehet: külső foglalkozások, otthoni,
  egyéni tanulás. Ezekről külön megállapodást kell kötni a mentorral.} Ezekben
az
időkben vannak a tanítási órák, foglalkozások, szakkörök,
műhelyek. Az egyes mikroiskolák ettől 20\%-ban bármelyik irányban eltérhetnek,
ha ez segíti a tanulásszervezők munkáját és a gyerekek fejlődését. Így hetente
minimum $5 \cdot 8 \cdot 0,8 = 32$ órát, maximum 48 órát töltenek az iskolában.

Ennek 1--4 évfolyamszinten körülbelül a felét, azaz 16--24 órát, a 5--12
évfolyamszinten a  kétharmad részét, azaz 21--32 órát
töltik a gyerekek előre eltervezett módon, azaz modulokkal. A többi időben a tanárok
vezetése és felügyelete mellett szabadon alkotnak, játszanak, pihennek,
közösségi életet élnek.

Mivel az elvárt kiegyensúlyozottság miatt mind a három tantárgyra körülbelül
ugyanannyi energiát kell fektetni, így az egyes tantárgyakra a teljes
rendelkezésre álló időkeret egyharmad részét kell számolni. Ettől az iskolák
$\pm$ 20\%-ban eltérhetnek, így kiszámolható, hogy minimum mennyi időt kell
egy-egy gyereknek egy héten egy tantárggyal foglalkoznia. Ezt összegzi
\aref{tbl:oraszamok}. táblázat.

\begin{table}

  \begin{tabular}{ l|l|l }

    \textbf{Tantárgy} & \textbf{1--4 évfolyam}                               & \textbf{5--12 évfolyam}
    \\ \hline
    Harmónia          & $\frac{5 \times 8 \times 0,8}{2} \times \frac{1}{3}
      \times 0.8 =
    4,27$ óra         &
    $\frac{5 \times 8 \times 0,8 \times 2}{3} \times \frac{1}{3} \times 0,8 =
      5,69$
    óra
    \\ \hline
    STEM              & 4,27 óra
                      & 5,69 óra                                                                     \\
    \hline
    KULT              & 4,27 óra
                      & 5,69 óra                                                                     \\
    \hline

  \end{tabular}
  \caption{Az elvárt kiegyensúlyozottság miatt a tantárgyakkal egyenlő
    minimális óraszámban kell foglalkozni.}
  \label{tbl:oraszamok}
\end{table}

Fontos, hogy \emph{egy-egy modul több tantárgy fejlesztési céljaihoz és
  tanulási eredményeihez is kapcsolódhat.}
\section{NAT céljainak támogatása}
\label{sec:nat_celjai}
A Nemzeti alaptantervben szereplő fejlesztési célok elérését és a
kulcskompetenciák fejlődését több minden támogatja:

Egyrészt a tantárgyak lefedik a NAT fejlesztési céljait, kulcskompetenciáit és
műveltségi területeit, mert a jelenleg érvényben lévő, a miniszter által az
\emph{51/2012. (XII. 21.) számú EMMI rendelet I-IV. mellékletében} kiadott
kerettantervek \citep{ofi:kerettanterv} tanulási, tanítási eredményeiből
indultunk ki. Mivel a rendeletben szereplő kerettantervek teljesítik a NAT
feltételeit, így a Budapest School tantárgystruktúrája is teljesíti ezeket.

Másrészt az iskola életében, folyamatában való részvétel már önmagában
biztosítja a kulcskompentenciák fejlődését és a NAT fejlesztési céljainak
teljesülését sok esetben.

A \ref{tbl:nat_fejlesztesi} táblázat bemutatja a NAT fejlesztési területeihez
való kapcsolódást, a
\ref{tbl:nat_kulcs} táblázat pedig az illeszkedési pontokat a NAT
kulcskompetenciáihoz.

\begin{table}

  \begin{tabular}{p{5cm}|>{\raggedright}p{3cm}|p{3cm}}

    \textbf{A NAT fejlesztési céljai}               & \textbf{Tantárgyak}  & \textbf{Struktúra}           \\
    \hline
    Az erkölcsi nevelés                          & KULT, harmónia       & közösség                     \\ \hline
    Nemzeti öntudat, hazafias nevelés            & KULT, harmónia       & projektek                    \\ \hline
    Állampolgárságra, demokráciára nevelés       & KULT, harmónia       & közösség                     \\ \hline
    Az önismeret és a társas kultúra fejlesztése & , harmónia, STEM & saját
    tanulási út, közösség                                                                              \\ \hline
    A családi életre nevelés                     & harmónia             &                              \\ \hline
    A testi és lelki egészségre nevelés          & harmónia             & közösség                     \\ \hline
    Felelősségvállalás másokért, önkéntesség     & harmónia             & közösség, pro\-jek\-tek      \\
    \hline
    Fenntarthatóság, környezettudatosság         & harmónia, STEM       & projektek                    \\ \hline
    Pályaorientáció                              & KULT, harmónia, STEM & saját tanulási út            \\ \hline
    Gazdasági és pénzügyi nevelés                & KULT, harmónia, STEM & projektek                    \\ \hline
    Médiatudatosságra nevelés                    & KULT                 & projektek                    \\ \hline
    A tanulás tanítása                           & KULT, harmónia, STEM & saját tanulási út, mentorság \\

  \end{tabular}
  \caption{A NAT fejlesztési céljainak elérését nemcsak a tantárgyak, hanem az
    iskola struktúrája is támogatja.}
  \label{tbl:nat_fejlesztesi}
\end{table}

A \emph{saját tanulási} út fogalma például önmagában segíti a tanulás
tanulását, hiszen az a gyerek, aki képes önmagának saját célt állítani (mentori
segítséggel), azt elérni, és a folyamatra való reflektálás során képességeit
javítani, az fejleszti a tanulási képességét.

Vagy másik példaként, a Budapest School iskoláiban a \emph{közösség} maga hozza
a működéséhez szükséges szabályokat, folyamatosan alakítja és fejleszti saját
működését a tagok aktív részvételével. Ez az aktív állampolgárságra, a
demokráciára való nevelés Nemzeti alaptantervben előírt céljait is támogatja.

\begin{table}
  \centering
  \begin{tabular}{p{5cm}|>{\raggedright}p{3cm}|p{3cm}}

    \textbf{NAT kulcskompetenciái}                     & \textbf{Tantárgyak}  &
    \textbf{Struktúra}                                                                                        \\ \hline
    Anyanyelvi kommunikáció                            &  KULT                & tanulási szerződés, portfólió \\ \hline
    Idegen nyelvi kommunikáció                         & KULT                 & idegennyelvű modulok          \\ \hline
    Matematikai kompetencia                            & STEM                 &                               \\ \hline
    Természettudományos és technikai kompetencia       & STEM                 & projektek                     \\ \hline
    Digitális kompetencia                              & harmónia, STEM       & digitális portfólió-kezelés   \\ \hline
    Szociális és állampolgári kompetencia              & harmónia             & saját tanulási út,
    közösség                                                                                                  \\ \hline
    Kezdeményezőképesség és vállalkozói kompetencia    & KULT, harmónia, STEM & saját
    tanulási út, közösség                                                                                     \\ \hline
    Esztétikai-művészeti tudatosság és kifejezőkészség & harmónia, KULT       &                               \\ \hline
    A hatékony, önálló tanulás                         & KULT, harmónia, STEM & saját tanulási út,
    mentorság                                                                                                 \\

  \end{tabular}
  \caption{A NAT kulcskompetenciáinak fejlesztését támogatják a tantárgyak és
    az iskola felépítése is.}
  \label{tbl:nat_kulcs}
\end{table}
\section{Érettségire készülés}
\label{sec:erettsegi}

Az iskola a kötelező középszintű érettségi vizsgatárgyakra való
felkeszítést kötelezően vállalja érettségire felkészítő modulok
szervezésével. Lefordítva ezt a miniszter által kiadott kerettantervek alapján
működő iskolák esetén használt terminológiára, az érettségire felkészülés
érettségi tárgyak alapján szervezett fakultáció formájában történik.

A választható tantárgyak és az emeltszintű érettségi vizsgára csak akkor
szervez egy mikroiskola modult, ha arra legalább a közösség 20\%-a és minimum 6
gyerek igényt tart. Abban az esetben, ha minden választható tantárgyat csak
kevesebb, 
mint 20\% vagy 6 gyerek választ, és így a közösség nem tud válaszható érettségi
tárgyat választani, a fenntartó véletlenszerűen sorsol legalább egy választható
tárgyat
a gyerekek által megjelöltből.

Érettségire felkészítő modulokat akkor kell meghirdetni, amikor a gyerekek
elérik a 11. évfolyamszintet minden tárgyból.

Különböző mikroiskolákba járó gyerekek közös modulon készülhetnek az
érettségire. A fenntartó, ha nem tudja maga megszervezni a felkészítő modulokat, akkor más iskolákkal
együttműködve kell, hogy biztosítsa a felkészülési lehetőséget.

\section{A személyiség és egészségfejlesztés}
\label{sec:szemilyesegfejlesztes}

A Budapest School-gyerekek boldogak, egészségesek, hasznosak közösségüknek. Képesek önmaguknak célokat állítani, azokat elérni. Képesek  már kisgyerekkortól sajátjuknak megélni a tanulást és ahhoz kapcsolódóan célokat elérni, és fokozatosan tanulják meg azt, hogy egyénileg és csoportosan is tudnak nagyszabású projekteket véghezvinni. Tesznek a saját egészségükért, jövőjükért, társaikért, kapcsolódnak önmagukhoz és társaikhoz.

A gyerekek személyiségfejődését két szinten támogatja az iskola. Első szintet a mindennapi működés adja, mert az iskola működése önmagában személyiség- és egészségfejlesztő hatással bír. Második szintet az összevont harmónia tantárgy biztosítja, ami holisztikus megközelítésével támogatja a gyerekek fizikai, lelki jóllétét és kapcsolódásukat a
környezethez.

\paragraph{Működésből adódó fejlesztések.}
Az iskola alapműködése, hogy a gyerekek csoportban, közösségben élnek, tanulnak, dolgoznak, ezért ,,természetes'', hogy fejlődik az \emph{empátiájuk, kooperációs, kollaborációs képességük és érzelmi intelligenciájuk}. Alapelvünk: ha minél közelebb áll az iskola működése a jövő hétköznapjaihoz, a családhoz és a munkahelyhez, akkor a boldog családi életre, a sikeres munkahelyre való felkészülést már az iskolában való aktív részvétel önmagában támogatja. Hasonlóan, ahogy támogató, funkcionális, boldog családban felnőtt gyerekek nagyobb valószínűséggel lesznek maguk is egészségesebbek és boldogabbak.

A \emph{fejlődésfókuszú gondolkodásmód} kialakítását  kulcstényezőnek gondoljuk a gyerekeink hosszú távú boldogulásához. Ezért a saját célok által irányított tanulási környezettől kezdve, a jutalmazás, értékelés, visszajelzés módjáig minden az iskolában azt a célt szolgálja, hogy a gyerekek képesek legyenek magukról pozitívan gondolkodni, ami az integráns és egészséges embernek talán egyik legfontosabb jellemzője.

A \emph{teljes körű iskolai egészségfejlesztést} az alábbi négy egészségfejlesztési feladat rendszeres végzése adja:

\begin{itemize}

    \item
          egészséges táplálkozás megvalósítása (elsősorban megfelelő, magas minőségű, lehetőleg helyi alapanyagokból)
    \item
          mindennapi testmozgás minden gyereknek (változatos foglalkozásokkal, koncentráltan az egészségjavító elemekre, módszerekre, pl. tartásjavító torna, tánc, jóga)
    \item
          a gyerekek érett személyiséggé válásának elősegítése személyközpontú pedagógiai módszerekkel és a művészetek személyiségfejlesztő hatékonyságú alkalmazásával (ének, tánc, rajz, mesemondás, népi játékok, stb.)
    \item
          környezeti, médiatudatossági, fogyasztóvédelmi, balesetvédelmi egészségfejlesztési modulok, modulrészletek hatékony (azaz ``bensővé váló'') oktatása
\end{itemize}

\subsection{Egészségügyi felmérés szervezése és hatása a gyerekek életére}
A Budapest School gyerekek megelőző jelleggel rendszeresen iskola orvosi, védőnői és fogorvosi felülvizsgálaton vesznek részt.  Az orvosi, védőnői és fogorvosi vizsgálatot a fenntartó a mindenkori jogszabályokban meghatározott rendszerességgel szervezi meg. Jelenleg ezeket külső helyszínen, megbízott orvossal, fogorvossal és védőnővel szervezi meg a fenntartó.
 
Minden mikroiskola tanulásszervező csapata a tanév megkezdése előtt kijelöli az egészségnapokat: amikor a védőnői, orvosi, fogorvosi és egyéb fizikai és mentális felülvizsgálatokat megszervezi. 20 főig egy, afölött két napot kell megjelölni, és a fenntartóval egyeztetni. Egyeztetni azért kell, hogy a különböző mikroiskolák között ne legyen időpontütközés. Ha a gyerek az egészségnapon hiányzik az iskolából, akkor a szülő feladata a felülvizsgálatot megszerveznie.
 
Ezeken a napokon a gyerekek és a tanárok, iskolaidőben elutaznak a rendelőkbe, felkeresik az orvost, fogorvost, védőnőt. Mivel a gyerekek kivizsgálása feltehetőleg egyesével történik, ezért a ,,többi” gyereknek sokat kell várnia. Ezért a tanárok erre az időre egészség témában mikromodulokat terveznek, amikor a gyerekek olyan tanulási eredményeket érhetnek el, mint a \emph{,,Tisztában van az egészség megőrzésének jelentőségével, és tudja, hogy maga is felelős ezért.''} (5. Évfolyam 1. félév).
 
A felülvizsgálatok eredményeit a gyerekekkel a mentorokkal megbeszélik, és ha szükséges, akkor az eredmények alapján fejlődési célokat fogalmaznak meg. A vizsgálatok eredményeit a gyerekek a portfóliójukban ugyanúgy megőrzik, mint egy tudáspróba eredményét.
 

\section{Mindennapos testmozgás}
\label{sec:mindennapos-testmozgas}

A Budapest School-kerettantervben bemutatott rendszer szerint a ,,mikroiskoláknak saját fókuszuk, helyszínük, stílusuk alakulhat ki''. Ennek részeként a gyerekek mindennapos testmozgását is a mikroiskolák saját identitásuk és lehetőségeik szerint alakítják ki. Az iskola csak annyit ír elő, hogy mindennap legyen minimum 35 perc, aminek elsődleges célja a testmozgás és testnevelés.

Nincs olyan testmozgásforma, amelyet ez a program preferálna, vagy amely kizárt lenne. A mindennapos testmozgás fizikai szükségleteinek megszervezése is változatos módon történhet:
\begin{enumerate}
    \item Egyes mikroiskolák (telephelyek) saját tornatermében, tornaszobájában vagy udvarán.
    \item Szerződés, megállapodás alapján, elérhető közelben lévő más oktatási vagy sportlétesítményben (úszás, korcsolya, torna, foci, falmászás, sípálya).
    \item A szabadban, épített helyszínhez nem kötődő mozgásszervezés (pl. kirándulás, séta, barangolás, felfedező játékok).
    \item A feladatellátási helyeken, magyarul a tantermekben, speciálisan sportra alkalmas helyiséget nem igénylő mozgások esetén (pl. jóga).
\end{enumerate}

A testmozgás minden esetben a mozgást ismerő, abban képesítéssel, végzettséggel vagy megfelelő gyakorlattal rendelkező személy vezetésével vagy felügyelete mellett zajlik. Az iskola feladata monitorozni a mindennapos testnevelés megvalósulását: ki, mikor, hol, milyen testmozgást vezetett a gyerekeknek.

\section{A közösségfejlesztés}
\label{sec:kozossegfejlesztes}
A Budapest School közösségét a tanulásszervezők és a családok alkotják közösen, akiknek közös célja, hogy közösségük a gyerekek fejlődését együtt támogassa. A tanulás holisztikus szemlélete révén a gyerekeknek egyszerre kell jó kapcsolatot ápolniuk saját társaikkal, partnerségben lenni tanáraikkal és meghallani a szülők feléjük irányuló kéréseit.  A tanulás célját a gyerekek maguk határozzák meg, azok az értékek, rituálék viszont, melyek a mindennapjaikat meghatározzák, túlmutatnak az egyénen, és a közösség céljait szolgálják oly módon, hogy az az egyes családok életvitelével is összhangban legyen.

Minden mikroiskola-közösségnek feladata ezért, hogy az együttműködésük, a közösségi szabályaik, a házirendjük meghatározásakor a családok véleményét, javaslatait is kikérjék, és úgy hozzanak döntést, hogy az mindenki számára kellően biztonságos és elég jó döntés legyen.

Ez vonatkozhat a mindennapi eszközök használatára, a közlekedésre, az együtt, tanulással eltöltött egyéni és csoportos idő arányaira, a közösen megtartott ünnepekre, jeles napokra,  mindazokra a kérdésekre, melyek közösségi döntések, és hatással lehetnek az egyén fejlődésére.

\subsection{Elsősegély-nyújtási alapismeretek}
\label{sec:elsosegely}
Ahogy azt a 20/2012 EMMI rendelet 7. § (1) ak) pont előírja, a pedagógia progamunk ismerteti az elsősegély-nyújtási alapismeretek elsajátításával kapcsolatos iskolai tervet.

A cél, hogy a gyerekek és tanárok megtanulják aktívan úgy alakítani környezetüket és viselkedésüket, hogy a balesetek számát minimalizálják, hogy felismerjék, amikor segítségre van szükség, hogy hatékonyan segítsenek és tudjanak segítséget hívni. Ehhez gyakorlásra, a témával kapcsolatos védett időre van szükség. Ezért a Harmónia tantárgy 2., 4., 6. 8., 10. évfolyamszintja csak akkor teljesíthető, ha a gyerek minden második évben elvégez egy minimum négy órás modult, aminek célja,
\begin{itemize}
    \item hogy a gyerekek sajátítsák el a legalapvetőbb és legkorszerűbb elsősegély-nyújtási módokat, azaz tudjanak egymásnak segíteni baj esetén; (nem csak elméletben, hanem gyakorlatban)
    \item sajátítsák el, mikor és hogyan kell mentőt, segítséget hívni;
    \item foglalkozzanak azzal, hogyan tudják környezetüket, csoportjukat, mikroiskolájukat biztonságosabbá tenni és ezt dokumentálják is.
\end{itemize}

A modulszervezője próbáljon meg elsősegély-nyújtási bemutatót szervezni a gyerekeknek az Országos Mentőszolgálat, a Magyar Ifjúsági Vöröskereszt vagy az Ifjúsági Elsősegélynyújtók Országos Egyesületének vagy más, magyar, vagy külföldi képesítést szerzett szakember bevonásával.

Az elsősegély-nyújtási alapismeretek elsajátításával kapcsolatos feladatok megvalósításának elősegítése érdekében:
\begin{itemize}
    \item az iskola kapcsolatot épít ki az Országos Mentőszolgálattal, Magyar Ifjúsági Vöröskereszttel vagy az Ifjúsági Elsősegélynyújtók Országos Egyesületével tanulóink - választásuk szerint - bekapcsolódhatnak az elsősegély-nyújtással kapcsolatos iskolán kívüli vetélkedőkbe.
    \item  minden második évben legalább egyszer a tanároknak lehetőséget biztosít elsősegély tanfolyam látogatására.

\end{itemize}

\section{Út az esélyegyenlőség felé}
\label{sec:ut_az_eselyegyenloseg_fele}
A Budapest School lehetőséget kíván biztosítani arra, hogy a
mikroiskoláiba a társadalom minél szélesebb rétegeiből kerülhessenek be
gyerekek, hogy a családok társadalmi, gazdasági státuszától
függetlenül, kizárólag saját fejlődési útjuk kiteljesedése jegyében
válhassanak a közösség részévé.

Tudjuk, hogy az esélyek sosem egyenlőek, tenni viszont lehet azért, hogy
minél kiegyenlítettebbek legyenek. A következő területeken a családok
kiválasztásakor és az iskolába járó gyerekekkel való partneri
kapcsolatban azon dolgozunk, hogy senki se érezhesse magát hátrányosan
megkülönböztetve testi, szellemi, kulturális, szociális, nemi vagy
hitbéli egyediségei miatt.

\paragraph{A társadalmi státuszban}

  A Budapest School mikroiskoláiba járó családok legalább 30\%-a
  kevesebb hozzájárulást fizet, mint amennyi az iskola működésének
  teljes bekerülési költsége. Az ő számuk, az általuk fizetendő
  hozzájárulás mértéke a támogató családoktól függ. Minél érzékenyebb,
  minél nagyvonalúbb egy közösség azok irányába, akik nem tehetik meg,
  hogy a gyerekük tanulásáért kifizessék az ahhoz szükséges költségeket,
  annál nagyobb mértékű a társadalmi státuszbéli esélyegyenlőség egy
  adott Budapest School mikroiskolában.


\paragraph{A nemek arányában}


  A Budapest School mikroiskoláiban törekszünk a nemek arányának
  kiegyenlítésére. Folyamatosan monitorozzuk a fiú-lány arányt az egyes
  közösségekben, és amennyiben elcsúszik valamely irányba, akkor a
  felvételi során a kiegyenlítés irányába hozunk döntéseket.


\paragraph{A kulturális és vallási egyediségekben}


  A Budapest School mikroiskoláiban meghatározó a családok értékrendje. Ezen
  családok pedig jöhetnek azonos, de jöhetnek diverz kulturális és
  vallási környezetből is. Az ő szempontjaik tiszteletben tartása
  mindaddig kiemelten fontos, amíg az nem veszélyezteti a közösségben
  együtt tanuló gyerekek fejlődését.


\paragraph{A fejlődés sebességében}

  A Budapest School kerettanterve lehetőséget biztosít arra, hogy egy
  gyerek a saját tempójában, a saját maga által kijelölt és komfortos
  tanulási úton haladjon addig, amíg ebben a mentortanárával és a
  szüleivel is közös megállapodást kötnek. A fejlődés sebessége azonban
  nem akadályozhatja a közösségben tanulást. A tanulásszervezők
  felelőssége annak meghatározása, hogy egy, a közösségben lassabban
  fejlődő gyerek mennyiben veszélyezteti, vagy mennyiben segíti a
  tanulásszervezést a közösség egésze szempontjából.


\paragraph{A tanulás tartalmában}


  A Budapest School mikroiskoláiban minden gyereknek van saját célja.
  Ennek hossza, komplexitása minden esetben egyedi, függően a gyerek
  érdeklődésétől, érettségétől, családi helyzetétől, mentális és fizikai
  állapotától. A tanulás tartalmában mindenkinek van lehetősége arra,
  hogy a maga útját járja, ha figyelembe veszi, hogy ezt a közösség
  részeként kell tennie.


\paragraph{A testi és szellemi egyediségben}

  A Budapest School mikroiskoláiban a tanulásszervezők felelőssége annak
  eldöntése, hogy egy adott közösség milyen mértékben tud befogadni
  sérült, vagy saját nevelési igényű gyerekeket. Az ő befogadásukkor és
  a velük való kiemelt foglalkozáskor mindig arra a kérdésre kell
  válaszolni, hogy tudjuk-e garantálni a gyerek fejlődését, elég
  biztonsá\-gos-e a közeg a számára, és az iskolában való szerepe miként
  segíti a többi gyerek fejlődését.


\paragraph{A döntéshozásban}

A Budapest School-döntéshozatal nem többségi és\linebreak
nem konszenzusos
  megállapodások alapján történik. A döntések esélyt adnak arra, hogy
  minden kellően biztonságos, elég jó javaslatot ki lehessen próbálni
  akkor, ha az nem áll szemben a közösen elfogadott célokkal, és nem
  sérti  bármely egyén érdekeit olyan mértékben, ami sérti a
  biztonságérzetét.


\section{Önkéntes közösségi szolgálat}

Az Nkt 4. § (15) pontjában definiált közösségi szolgálat is modulként
kerül meghirdetésre, amit 12. évét betöltött gyerek választhat csak.
Közösségi szolgálatként elfogadható, ha a Budapest School iskola egy
másik mikroiskolájában segít a gyerek.

\section{Nemzetiségek megismerése}
\label{sec:nemzetiseg}

Az iskola környezetében elő nemzetiségek kultúrájának megismerését fontosnak tartja az iskola. Ezért minden második tanévben a településen működő nemzetiségi önkormányzatokkal kapcsolatot vesz, és  velük együttműködve legalább 4 órás modululakt alakít ki. A modulok célja, hogy a modul résztvevői megismerjék a nemzetiségekről azt, amit az önkormányzatok fontosnak tartanak megmutatni a nemzetiségekről.

Minden második évben legalább három önkörmányzattal három különböző modult kínál fel az iskola választásra.\footnote{Amelyik településen ennél több nemzetiségi önkormányzat működik, ott a tanév megkezdése előtt, augusztus folyamán véletlenszerűen sorsoljuk ki az adott évi 3 megismerendő nemzetiséget.}
\section{Környezeti nevelés}
\label{sec:kornyezeti-neveles}
Ahogy az UNESCO 1977-ben definiálta \emph{,,a környezeti nevelés olyan folyamat,
    melynek célja, hogy a világ népessége környezettudatosan gondolkodjék, figyeljen oda a környezetre és minden azzal kapcsolatos problémára. Rendelkezzen az ehhez szükséges tudással, beállítódással, képességekkel, motivációval, valamint mind egyéni, mind közösségi téren eltökélten törekedjék a jelenlegi problémák megoldására és az újabbak megelőzésére.”}

A környezeti nevelés céljainak eléréséhez a tanárok környezettudatossága, rendszerszemlélete és aktív részvételére van szükség. Ezért az iskola vállalja, hogy a tanárainak tanévente legalább 4 órás workshopot szervez a témában, ahol a tanárok arról tudnak is tudnak beszélni, hogyan vitték be a környezeti nevelést a gyerekek mindennapjaiba.

\section{Tankönyvek kiválasztása}
\label{sec:tankonyvek}

Modulvezetők minden esetben maguk választják a modulhoz szükséges tankönyvek, szoftverek, weboldalak és egyéb eszközöket úgy, hogy

\begin{itemize}

      \item
            az a megfelelő legyen annak a csoportnak, ahhoz a célhoz, amit el akar érni
      \item
            minden esetben legyen mindenki számára elérhető (esetek többségeben értsd ingyenes) megoldás
      \item
            modulvezetők bátorítva vannak arra, hogy új dolgokat próbáljanak ki, és tapasztalataikat az iskola többi tanárával megosszák.
\end{itemize}

Mivel a Budapest School kerettantervének értelmében a saját célok legalább 50\%-át az állami kerettantervben meghatározott tanulási eredmények közül kell választani, az ehhez szükséges ismeretek megszerzéséhez a Budapest School az Oktatási Hivatal általi jegyzékben államilag támogatott OFI által fejlesztett tankönyveket veszi alapul. A Budapest School tanárcsapatának lehetősége van arra, hogy ettől eltérő, a mindenkori tankönyvjegyzékben szereplő tankönyvvel segítse el a kerettantervben meghatározott tanulási eredmények elérését. És arra is lehetősége van, hogy egyátalán ne használjon tankönyvet, mert sokszor az internet elegendő információt tartalmaz.

A Budapest School pedagógiai programjának alapja, hogy a gyerekek egyéni céljaira szabott tanulási terveket készít. Ennek előfeltétele, hogy a könyvek használata is ehhez kapcsolódó módon rugalmasan történjen, minden esetben az adott tanulási modul igényeihez szabva. Ennek érdekében a program pedagógusai folyamatosam állítják össze a gyerekek eltérő céljaihoz és képességszintjeihez igazodó differenciált tevékenységek és feladatsorok rendszerét.

A Budapest School iskolában, egy (modul)csoport csak akkor választhat egy tankönyvet, ha az minden család számára elérhető. Ha valamelyik család nem tudja a könyvet magának megvásárolni, akkor a csoport többi tagja megvásárolja neki.



\part{Referenciák}
\chapter{Jogszabályok által elvárt tartalmak indexe}
\begin{itemize}
   

\item \emph{az iskolában folyó nevelő-oktató munka pedagógiai alapelvei, értékei, céljai, feladatai, eszközei, eljárásai: }    
       lásd        \ref{sec:alapelvek}.~fejezet \apageref{sec:alapelvek}.~oldalon
              
\item \emph{a személyiségfejlesztéssel kapcsolatos pedagógiai feladatok: }    
       lásd        \ref{sec:gyerekkep}.~fejezet \apageref{sec:gyerekkep}.~oldalon és 
              \ref{sec:szemilyesegfejlesztes}.~fejezet \apageref{sec:szemilyesegfejlesztes}.~oldalon
              
\item \emph{a közösségfejlesztéssel, az iskola szereplőinek együttműködésével kapcsolatos feladatok: }    
       lásd        \ref{sec:kozossegfejlesztes}.~fejezet \apageref{sec:kozossegfejlesztes}.~oldalon
              
\item \emph{a pedagógusok helyi intézményi feladatai, az osztályfőnöki munka tartalmát, az osztályfőnök feladatai: }   a Budapest Schoolban osztályfőnök szerep külön nincs nevesítve, a tanulászervezők és mentorok feladatai állnak ehhez a legközelebb 
       lásd        \ref{sec:tanarfeladatok}.~fejezet \apageref{sec:tanarfeladatok}.~oldalon
              
\item \emph{Az iskola által alkalmazott jelölés, értékelés érdemjegyre, osztályzatra való átváltásának szabályai. NKT 54. § (4): }    
       lásd        \ref{sec:osztalyzatok}.~fejezet \apageref{sec:osztalyzatok}.~oldalon
              
\item \emph{a kiemelt figyelmet igénylő tanulókkal kapcsolatos pedagógiai tevékenység helyi rendje: }    
       lásd        \ref{sec:kiemelt_figyelem}.~fejezet \apageref{sec:kiemelt_figyelem}.~oldalon
              
\item \emph{a tanulóknak az intézményi döntési folyamatban való részvételi jogai gyakorlásának rendje: }    
       lásd        \ref{sec:sajat_szabalyok}.~fejezet \apageref{sec:sajat_szabalyok}.~oldalon
              
\item \emph{a szülő, a tanuló, a pedagógus és az intézmény partnerei kapcsolattartásának formái: }   mind az alapelvekben, mind a döntéshozásban és konfliktusok kezelésében erre törekszik az iskola 
       lásd        \ref{sec:kozossegi_elet}.~fejezet \apageref{sec:kozossegi_elet}.~oldalon és 
              \ref{sec:sajat_szabalyok}.~fejezet \apageref{sec:sajat_szabalyok}.~oldalon és 
              \ref{sec:konfliktusok_kezelese}.~fejezet \apageref{sec:konfliktusok_kezelese}.~oldalon
              
\item \emph{a tanulmányok alatti vizsgák és az alkalmassági vizsga szabályai; az iskolai írásbeli, szóbeli, gyakorlati beszámoltatások, az ismeretek számonkérésének rendje: }   a Budapest School portfólióalapú értékelést alkalmaz a kerettantervet követve 
       lásd        \ref{sec:evfolyamok_osztalyzatok}.~fejezet \apageref{sec:evfolyamok_osztalyzatok}.~oldalon
              
\item \emph{a felvétel és az átvétel -- Nkt. keretei közötti -- helyi szabályai: }    
       lásd        \ref{sec:felvetel-atvetel}.~fejezet \apageref{sec:felvetel-atvetel}.~oldalon
              
\item \emph{az elsősegély-nyújtási alapismeretek elsajátításával kapcsolatos iskolai terv: }    
       lásd        \ref{sec:elsosegely}.~fejezet \apageref{sec:elsosegely}.~oldalon
              
\item \emph{az iskolában alkalmazott sajátos pedagógiai módszerek, beleértve a projektoktatást: }   a pedagógia program (az elfogadott kerettantervnek megfelelően) a tanárok kísérletezését támogatja, ezért minden módszert előre nem tudunk meghatározni 
       lásd        \ref{sec:pedagogia_modszerek}.~fejezet \apageref{sec:pedagogia_modszerek}.~oldalon
              
\item \emph{a választott kerettanterv megnevezése: }   Budapest School Kerettantenterv 
       lásd        \ref{sec:tanterv-program}.~fejezet \apageref{sec:tanterv-program}.~oldalon
              
\item \emph{a választott kerettanterv által meghatározott óraszám feletti kötelező tanórai foglalkozások, továbbá a kerettantervben meghatározottakon felül a nem kötelező tanórai foglalkozások megtanítandó és elsajátítandó tananyaga, az ehhez szükséges kötelező, kötelezően választandó vagy szabadon választható tanórai foglalkozások megnevezése, óraszáma: }   a Budapest School elfogadott kerettanterve szerint a foglalkozások rendszere (ebből következően az óraszám, tananyag, stb.) speciális, a pedagógiai programra vonatkozó jelen előírást így e kerettanterv rendszerén belül kerül kifejtésre.
 
       lásd        \ref{sec:elsosegely}.~fejezet \apageref{sec:elsosegely}.~oldalon és 
              \ref{sec:nemzetiseg}.~fejezet \apageref{sec:nemzetiseg}.~oldalon és 
              \ref{sec:mindennapos-testmozgas}.~fejezet \apageref{sec:mindennapos-testmozgas}.~oldalon
              
\item \emph{az oktatásban alkalmazható tankönyvek, tanulmányi segédletek és taneszközök kiválasztásának elvei (figyelembe véve a tankönyv térítésmentes igénybevétele biztosításának kötelezettségét): }    
       lásd        \ref{sec:tankonyvek}.~fejezet \apageref{sec:tankonyvek}.~oldalon
              
\item \emph{a Nemzeti alaptantervben meghatározott pedagógiai feladatok helyi megvalósításának részletes szabályai: }    
       lásd        \ref{sec:nat_celjai}.~fejezet \apageref{sec:nat_celjai}.~oldalon
              
\item \emph{a mindennapos testnevelés, testmozgás megvalósításának módja, ha azt nem az Nkt. 27. § (11) bekezdésében meghatározottak szerint szervezik meg: }    
       lásd        \ref{sec:mindennapos-testmozgas}.~fejezet \apageref{sec:mindennapos-testmozgas}.~oldalon
              
\item \emph{a választható tantárgyak, foglalkozások, továbbá ezek esetében a pedagógusválasztás szabályai: }   A Budapest School alapvetése a saját tanulási út és a saját tanulási cél. 
       lásd        \ref{sec:modulok}.~fejezet \apageref{sec:modulok}.~oldalon és 
              \ref{sec:modulok_meghirdetese}.~fejezet \apageref{sec:modulok_meghirdetese}.~oldalon
              
\item \emph{középiskola esetén azon választható érettségi vizsgatárgyak megnevezése, amelyekből a középiskola tanulóinak közép- vagy emelt szintű érettségi vizsgára való felkészítését az iskola kötelezően vállalja, továbbá annak meghatározáse, hogy a tanulók milyen helyi tantervi követelmények teljesítése mellett melyik választható érettségi vizsgatárgyból tehetnek érettségi vizsgát: }    
       lásd        \ref{sec:erettsegi}.~fejezet \apageref{sec:erettsegi}.~oldalon
              
\item \emph{a tanuló tanulmányi munkájának írásban, szóban vagy gyakorlatban történő ellenőrzési és értékelési módját, diagnosztikus, szummatív, fejlesztő formáit, valamint a magatartás és szorgalom minősítésének elvei: }    
       lásd        \ref{sec:ertekeles}.~fejezet \apageref{sec:ertekeles}.~oldalon és 
              \ref{sec:jutalmazas_es_ertekeles}.~fejezet \apageref{sec:jutalmazas_es_ertekeles}.~oldalon
              
\item \emph{a csoportbontások és az egyéb foglalkozások szervezésének elvei: }    
       lásd        \ref{sec:csoportbontasok}.~fejezet \apageref{sec:csoportbontasok}.~oldalon és 
              \ref{sec:csoportok}.~fejezet \apageref{sec:csoportok}.~oldalon
              
\item \emph{a nemzetiséghez nem tartozó tanulók részére a településen élő nemzetiség kultúrájának megismerését szolgáló tananyag: }    
       lásd        \ref{sec:nemzetiseg}.~fejezet \apageref{sec:nemzetiseg}.~oldalon
              
\item \emph{az egészségnevelési és környezeti nevelési elvek: }    
       lásd        \ref{sec:kornyezeti-neveles}.~fejezet \apageref{sec:kornyezeti-neveles}.~oldalon és 
              \ref{sec:mindennapos-testmozgas}.~fejezet \apageref{sec:mindennapos-testmozgas}.~oldalon
              
\item \emph{a gyermekek, tanulók esélyegyenlőségét szolgáló intézkedések: }    
       lásd        \ref{sec:csoportbontasok}.~fejezet \apageref{sec:csoportbontasok}.~oldalon
              
\item \emph{a tanuló jutalmazásával összefüggő, a tanuló magatartásának, szorgalmának értékeléséhez, minősítéséhez kapcsolódó elvek: }   magatartás és szorgalom minősítés nem történik a Budapest Schoolban 
       lásd        \ref{sec:jutalmazas_es_ertekeles}.~fejezet \apageref{sec:jutalmazas_es_ertekeles}.~oldalon
              
\item \emph{intézmény partnerei kapcsolattartásának formái: }    
       lásd        \ref{sec:az_iskola_kormanyzasa}.~fejezet \apageref{sec:az_iskola_kormanyzasa}.~oldalon
              
\item \emph{a teljeskörű egészségfejlesztéssel összefüggő feladatok: }    
       lásd        \ref{sec:szemilyesegfejlesztes}.~fejezet \apageref{sec:szemilyesegfejlesztes}.~oldalon
              
\end{itemize}

\begin{longtable}{p{0.4\textwidth} | p{0.3\textwidth} |p{0.2\textwidth}}

    \textbf{feladat} & megjegyzés              & \textbf{referencia} \\
    \hline

    az iskolában folyó nevelő-oktató munka pedagógiai alapelvei, értékei, céljai, feladatai, eszközei, eljárásai  &   & 
              \ref{sec:alapelvek}.~fejezet \apageref{sec:alapelvek}.~oldalon
              \\ \hline

    a személyiségfejlesztéssel kapcsolatos pedagógiai feladatok  &   & 
              \ref{sec:gyerekkep}.~fejezet \apageref{sec:gyerekkep}.~oldalon és 
              \ref{sec:szemilyesegfejlesztes}.~fejezet \apageref{sec:szemilyesegfejlesztes}.~oldalon
              \\ \hline

    a közösségfejlesztéssel, az iskola szereplőinek együttműködésével kapcsolatos feladatok  &   & 
              \ref{sec:kozossegfejlesztes}.~fejezet \apageref{sec:kozossegfejlesztes}.~oldalon
              \\ \hline

    a pedagógusok helyi intézményi feladatai, az osztályfőnöki munka tartalmát, az osztályfőnök feladatai  &  a Budapest Schoolban osztályfőnök szerep külön nincs nevesítve, a tanulászervezők és mentorok feladatai állnak ehhez a legközelebb & 
              \ref{sec:tanarfeladatok}.~fejezet \apageref{sec:tanarfeladatok}.~oldalon
              \\ \hline

    Az iskola által alkalmazott jelölés, értékelés érdemjegyre, osztályzatra való átváltásának szabályai. NKT 54. § (4)  &   & 
              \ref{sec:osztalyzatok}.~fejezet \apageref{sec:osztalyzatok}.~oldalon
              \\ \hline

    a kiemelt figyelmet igénylő tanulókkal kapcsolatos pedagógiai tevékenység helyi rendje  &   & 
              \ref{sec:kiemelt_figyelem}.~fejezet \apageref{sec:kiemelt_figyelem}.~oldalon
              \\ \hline

    a tanulóknak az intézményi döntési folyamatban való részvételi jogai gyakorlásának rendje  &   & 
              \ref{sec:sajat_szabalyok}.~fejezet \apageref{sec:sajat_szabalyok}.~oldalon
              \\ \hline

    a szülő, a tanuló, a pedagógus és az intézmény partnerei kapcsolattartásának formái  &  mind az alapelvekben, mind a döntéshozásban és konfliktusok kezelésében erre törekszik az iskola & 
              \ref{sec:kozossegi_elet}.~fejezet \apageref{sec:kozossegi_elet}.~oldalon és 
              \ref{sec:sajat_szabalyok}.~fejezet \apageref{sec:sajat_szabalyok}.~oldalon és 
              \ref{sec:konfliktusok_kezelese}.~fejezet \apageref{sec:konfliktusok_kezelese}.~oldalon
              \\ \hline

    a tanulmányok alatti vizsgák és az alkalmassági vizsga szabályai; az iskolai írásbeli, szóbeli, gyakorlati beszámoltatások, az ismeretek számonkérésének rendje  &  a Budapest School portfólióalapú értékelést alkalmaz a kerettantervet követve & 
              \ref{sec:evfolyamok_osztalyzatok}.~fejezet \apageref{sec:evfolyamok_osztalyzatok}.~oldalon
              \\ \hline

    a felvétel és az átvétel -- Nkt. keretei közötti -- helyi szabályai  &   & 
              \ref{sec:felvetel-atvetel}.~fejezet \apageref{sec:felvetel-atvetel}.~oldalon
              \\ \hline

    az elsősegély-nyújtási alapismeretek elsajátításával kapcsolatos iskolai terv  &   & 
              \ref{sec:elsosegely}.~fejezet \apageref{sec:elsosegely}.~oldalon
              \\ \hline

    az iskolában alkalmazott sajátos pedagógiai módszerek, beleértve a projektoktatást  &  a pedagógia program (az elfogadott kerettantervnek megfelelően) a tanárok kísérletezését támogatja, ezért minden módszert előre nem tudunk meghatározni & 
              \ref{sec:pedagogia_modszerek}.~fejezet \apageref{sec:pedagogia_modszerek}.~oldalon
              \\ \hline

    a választott kerettanterv megnevezése  &  Budapest School Kerettantenterv & 
              \ref{sec:tanterv-program}.~fejezet \apageref{sec:tanterv-program}.~oldalon
              \\ \hline

    a választott kerettanterv által meghatározott óraszám feletti kötelező tanórai foglalkozások, továbbá a kerettantervben meghatározottakon felül a nem kötelező tanórai foglalkozások megtanítandó és elsajátítandó tananyaga, az ehhez szükséges kötelező, kötelezően választandó vagy szabadon választható tanórai foglalkozások megnevezése, óraszáma  &  a Budapest School elfogadott kerettanterve szerint a foglalkozások rendszere (ebből következően az óraszám, tananyag, stb.) speciális, a pedagógiai programra vonatkozó jelen előírást így e kerettanterv rendszerén belül kerül kifejtésre.
 & 
              \ref{sec:elsosegely}.~fejezet \apageref{sec:elsosegely}.~oldalon és 
              \ref{sec:nemzetiseg}.~fejezet \apageref{sec:nemzetiseg}.~oldalon és 
              \ref{sec:mindennapos-testmozgas}.~fejezet \apageref{sec:mindennapos-testmozgas}.~oldalon
              \\ \hline

    az oktatásban alkalmazható tankönyvek, tanulmányi segédletek és taneszközök kiválasztásának elvei (figyelembe véve a tankönyv térítésmentes igénybevétele biztosításának kötelezettségét)  &   & 
              \ref{sec:tankonyvek}.~fejezet \apageref{sec:tankonyvek}.~oldalon
              \\ \hline

    a Nemzeti alaptantervben meghatározott pedagógiai feladatok helyi megvalósításának részletes szabályai  &   & 
              \ref{sec:nat_celjai}.~fejezet \apageref{sec:nat_celjai}.~oldalon
              \\ \hline

    a mindennapos testnevelés, testmozgás megvalósításának módja, ha azt nem az Nkt. 27. § (11) bekezdésében meghatározottak szerint szervezik meg  &   & 
              \ref{sec:mindennapos-testmozgas}.~fejezet \apageref{sec:mindennapos-testmozgas}.~oldalon
              \\ \hline

    a választható tantárgyak, foglalkozások, továbbá ezek esetében a pedagógusválasztás szabályai  &  A Budapest School alapvetése a saját tanulási út és a saját tanulási cél. & 
              \ref{sec:modulok}.~fejezet \apageref{sec:modulok}.~oldalon és 
              \ref{sec:modulok_meghirdetese}.~fejezet \apageref{sec:modulok_meghirdetese}.~oldalon
              \\ \hline

    középiskola esetén azon választható érettségi vizsgatárgyak megnevezése, amelyekből a középiskola tanulóinak közép- vagy emelt szintű érettségi vizsgára való felkészítését az iskola kötelezően vállalja, továbbá annak meghatározáse, hogy a tanulók milyen helyi tantervi követelmények teljesítése mellett melyik választható érettségi vizsgatárgyból tehetnek érettségi vizsgát  &   & 
              \ref{sec:erettsegi}.~fejezet \apageref{sec:erettsegi}.~oldalon
              \\ \hline

    a tanuló tanulmányi munkájának írásban, szóban vagy gyakorlatban történő ellenőrzési és értékelési módját, diagnosztikus, szummatív, fejlesztő formáit, valamint a magatartás és szorgalom minősítésének elvei  &   & 
              \ref{sec:ertekeles}.~fejezet \apageref{sec:ertekeles}.~oldalon és 
              \ref{sec:jutalmazas_es_ertekeles}.~fejezet \apageref{sec:jutalmazas_es_ertekeles}.~oldalon
              \\ \hline

    a csoportbontások és az egyéb foglalkozások szervezésének elvei  &   & 
              \ref{sec:csoportbontasok}.~fejezet \apageref{sec:csoportbontasok}.~oldalon és 
              \ref{sec:csoportok}.~fejezet \apageref{sec:csoportok}.~oldalon
              \\ \hline

    a nemzetiséghez nem tartozó tanulók részére a településen élő nemzetiség kultúrájának megismerését szolgáló tananyag  &   & 
              \ref{sec:nemzetiseg}.~fejezet \apageref{sec:nemzetiseg}.~oldalon
              \\ \hline

    az egészségnevelési és környezeti nevelési elvek  &   & 
              \ref{sec:kornyezeti-neveles}.~fejezet \apageref{sec:kornyezeti-neveles}.~oldalon és 
              \ref{sec:mindennapos-testmozgas}.~fejezet \apageref{sec:mindennapos-testmozgas}.~oldalon
              \\ \hline

    a gyermekek, tanulók esélyegyenlőségét szolgáló intézkedések  &   & 
              \ref{sec:csoportbontasok}.~fejezet \apageref{sec:csoportbontasok}.~oldalon
              \\ \hline

    a tanuló jutalmazásával összefüggő, a tanuló magatartásának, szorgalmának értékeléséhez, minősítéséhez kapcsolódó elvek  &  magatartás és szorgalom minősítés nem történik a Budapest Schoolban & 
              \ref{sec:jutalmazas_es_ertekeles}.~fejezet \apageref{sec:jutalmazas_es_ertekeles}.~oldalon
              \\ \hline

    intézmény partnerei kapcsolattartásának formái  &   & 
              \ref{sec:az_iskola_kormanyzasa}.~fejezet \apageref{sec:az_iskola_kormanyzasa}.~oldalon
              \\ \hline

    a teljeskörű egészségfejlesztéssel összefüggő feladatok  &   & 
              \ref{sec:szemilyesegfejlesztes}.~fejezet \apageref{sec:szemilyesegfejlesztes}.~oldalon
              \\ \hline




\end{longtable}



\bibliography{references.bib}{}
\label{sec:bibliographyk}
\bibliographystyle{apalike}

\end{document}
