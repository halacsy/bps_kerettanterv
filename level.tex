\pagestyle{empty}

\noindent
\textit{Prof. Dr. Kásler Miklós Miniszter Úr \\
      részére}
\vspace{0.75cm}

\noindent
\textbf{tárgy}: hiánypótlás

\noindent
\textbf{iktatószám}: 14259/2019/KÖZNEVTART
\vspace{0.75cm}

\noindent
Tisztelt Miniszter Úr!
\vspace{0.75cm}

\noindent
A MiSulink Nonprofit Korlátolt Felelősségű Társaság (székhely: 1118 Budapest,
Breznó köz 8., Cg. 01-09-209663), mint a Budapest School Általános Iskola és
Gimnázium (nyilvántartási szám: 702.) fenntartója az Emberi Erőforrások
Minisztériuma által {\tt 14259/2019/KÖZNEVTART} iktatószámon, 2019. április 5.
napján
hozott hiánypótló végzésben foglaltaknak az alábbiak szerint teszünk eleget.

\paragraph{A beadott kerettanterv változásainak összefoglalója}

\begin{itemize}
      \item  Az eredmények és fejlesztése elkülönítése végett az előző
            változatban használt
            eredménycélok helyett most az egyértelműbb \emph{tanulási eredmény}
            fogalmat
            használja a kerettanterv. Ezt definiálja
            \aref{sec:tanulasi_eredmenyek}.~fejezet.

      \item A tantárgyak tartalmát meghatározó
            \ref{sec:tantargyi_tanulasi_eredmenyek}.~fejezet $1187$ elemű
            lista teljesen
            átdolgozásra került.
            \begin{itemize}
                  \item Tanulási eredményeinek megfogalmazása most egységes:
                        mindenhol a
                        \emph{,,felismeri a háromszöget"}
                        és nem a \emph{,,háromszög felismerése''} vagy
                        \emph{,,háromszög''}
                        megfogalmazás van.
                  \item A tanulási eredmények tantárgyak,
                        tématerületek és
                        pedagógia szakaszok alapján van csoportosítva. Mivel
                        megjelentek a \emph{kötelező tanulási
                              eredmények} a
                        kerettanterv érvényesíteni tudja a magyar iskolarendszer
                        szakaszait.
            \end{itemize}
            A tervezhetőség és a tantárgyak
            közti integráció segítése miatt a modulok tartalmára több megkötést és iránymutatást ad a mostani
            kerettanterv. A legerősebb változás: a jelenlegi kerettanterv
            egyértelművé teszi, hogy a moduloknak a tantárgyak tanulási
            eredményeit le \emph{kell} fedniük (ld.
            \ref{sec:modulok_es_tanulasi_eredmenyek}.~fejezet).
            \emph{Így a kerettanterv biztosítja, hogy a tantárgyi követelményeket
                  a modulok lefedik.}

      \item A kerettanterv kötelező tanulási eredményként definiálja mindazokat
            az
            eredményeket, melyek a kötelező érettségi tárgyak teljesítéséhez
            szükségesek. Ezek az 1--4.~évfolyamszinteken a miniszter által
            kiadott
            kerettantervek
            \emph{magyar nyelv és irodalom}, \emph{matematika}, \emph{idegen
                  nyelv}
            tantárgyakból
            származó tanulási eredmények, és az 5.~évfolyamszinttől kiegészülnek a
            \emph{történelem, társadalmi és állampolgári ismeretek} tantárgyak
            alapján
            létrehozott tanulási eredményekkel. Pontosabb szabályozásért ld.
            \aref{sec:kotelezo_tanulasi_eredmenyek}.~fejezetet.
      \item A 11.~évfolyamtól kötelező modulonként megjelennek a érettségire
            felkészítő
            modulok \aref{sec:erettsegi}.~fejezetben.
      \item Az átjárhatótág transzparens és (szülők számára) egyszerű biztosítása
            érdekében a
            kerettanterv meghatározza, hogy a Nat pedagógiai szakaszainak végén
            hogyan
            kaphat minden gyerek olyan bizonyítványt, amiben
            a miniszter által kiadott kerettanterv
            tantárgyaihoz rendelt osztályzatok szerepelnek
            (ld. \ref{sec:osztalyzatok}.~fejezet).
      \item A kerettanterv \aref{sec:mikroiskola}.~fejezetben egyértelműen
            meghatározza, hogy a Budapest School mikroiskolák
            megegyeznek az Nkt. összevont osztályával. A kerettanterv
            meghatározza egy új mikroiskola
            indításának és a mikroiskolák közötti átjárhatóság feltételeit.
      \item A kerettanterv pontosan fogalmaz a \emph{felső tagozat} tekintetében
            azzal, hogy évfolyamokban határozza meg, ahol szakaszonként eltérő
            szabályok
            érvényesek. Így pl. \apageref{tbl:oraszamok}. oldalon található
            \ref{tbl:oraszamok}. táblázat
            az óraszámokról lefedi az 1--12. évfolyamot.

      \item \Aref{sec:jogszabalyok}.~fejezet kiegészült:
            \begin{itemize}
                  \item \emph{Elfogadott pedagógus végzettség és szakképzettség} c.
                        fejezet a gyerekek életkor szerint bontja
                        az elfogadott pedagógus végzettségeket. A kerettanterv
                        mostani változata sokkal
                        szigorúbban veszi az elfogadott végzettségek halmazát, hogy
                        védje a gyerekek
                        biztonságát és biztosítsa hatékony tanulásukat.
                  \item Új elem a szükséges eszközök biztosításának szabályozása. \emph{Helyiségek
                              bútorzata és egyéb berendezési tárgyai} c. fejezet
                         az
                        iskolában a tanárok és gyerekek számára szabadon elérhető
                        wifit és tantermenként online számítógépet tesz kötelezővé,
                        ezzel
                        kiegészítve a 20/2012. (VIII. 31.) EMMI rendelet 2.II.
                        mellékletét. Ugyanez a fejezet
                        felmenti az iskolát a pedagógusonkénti 1 asztal
                        kötelezettség
                        alól a
                        nevelőtestületi szobában, és hogy tanári asztal legyen minden tanteremben.

            \end{itemize}

      \item Egyéni cél mindenhol egységesen \emph{saját cél} lett, kifejezve, hogy nem
            biztos, hogy egyénenként változik, viszont a gyerek sajátjának kell hogy érezze.
      \item Egységesen \emph{gyereknek} hívja a kerettanterv a tanulót.
\end{itemize}

\noindent
\textbf{Jelen levelünkhöz mellékelten küldjük a hiánypótlásban foglaltaknak
      megfelelően módosított, változásokkal egységes szerkezetbe foglalt
      kerettantervet.}

\vspace{0.75cm}

\noindent
Budapest, 2019. április 16.

\vspace{0.75cm}
\noindent
Tisztelettel,

\vspace{0.75cm}
\noindent
\begin{center}
      \begin{tabular}{p{8cm}}
            \begin{center}
                  \hrulefill \\
                  \textbf{MiSulink Nonprofit Kft.} \\
                  fenntartó\\
                  képviseletében: Halácsy Péter ügyvezető \\
            \end{center}
      \end{tabular}
\end{center}
\newpage