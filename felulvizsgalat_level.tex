\pagestyle{empty}

\noindent
\textit{Prof. Dr. Kásler Miklós Miniszter Úr \\
      részére}
\vspace{0.75cm}

\noindent
\textbf{tárgy}: felülvizsgálat

\noindent
\textbf{iktatószám}: 14259-7/2019/KÖZNEVTART
\vspace{0.75cm}

\noindent
Tisztelt Miniszter Úr!
\vspace{0.75cm}

\noindent
A MiSulink Nonprofit Korlátolt Felelősségű Társaság (székhely: 1118 Budapest,
Breznó köz 8., Cg. 01-09-209663), mint a Budapest School Általános Iskola és
Gimnázium (nyilvántartási szám: 702.) fenntartója az Emberi Erőforrások
Minisztériuma által {\tt 14259-7/2019/KÖZNEVTART} iktatószámon, 2019. április 30.
napján
elfogadott \emph{Budapest School Kerettanterv} módosítását és felülvizsgálatát kérjük.


\paragraph{A kerettanterv változásainak összefoglalója}

\begin{itemize}
      \item A kerettanterv a miniszter által kiadott kerettantervek tantárgyi struktúráját vette át, így a három tantárgy helyett, most húsz tantárgyban csoportosítja a tanulási eredményeket. A tantárgyakat \aref{sec:tantargyak}~.fejezet mutatja be.
      \item A tantárgyakhoz rendelt heti óraszámok a miniszter által kiadott kerettantervben megadott óraszámok 75\%-a \aref{sec:tantargyak}.~fejezetben található \ref{tbl:oraszamok}.~táblázatban.
      \item A kerettanterv a tananyag tartalmakat tanulási eredmények halmazaként adja meg tantárgyanként és félévenkénti bontásban \aref{sec:tantargyi_tanulasi_eredmenyek}.~fejezetben.
      \item A tanév trimeszterenkénti bontása megmaradt, ugyanakkor a kerettanterv egyértelműen kijelöli a helyét, idejét és módját a félévenkénti értékelésnek \aref{sec:tanev_ritmusa}.
      \item A bevezető részben egy ábra szemlélteti, hogyan tud az iskola egyszerre önvezérelt tanulást támogató, személyreszabott tanulási környezetet biztosítani, a törvényi társadalmi elvárásoknak megfelelni és átjárhatóságot biztosítani a gyerekek számára.
      \item \aref{sec:modulok_es_tanulasi_eredmenyek}~.fejezet bemutatja, hogyan kapcsolódnak a modulok a tantárgyakhoz a tanulási eredményeken keresztül.
\end{itemize}


\noindent
\textbf{Jelen levelünkhöz mellékelten küldjük a módosított, változásokkal egységes szerkezetbe foglalt
      kerettantervet.}

\vspace{0.75cm}

\noindent
Budapest, 2019. július 29.

\vspace{0.75cm}
\noindent
Tisztelettel,

\vspace{0.75cm}
\noindent
\begin{center}
      \begin{tabular}{p{8cm}}
            \begin{center}
                  \hrulefill \\
                  \textbf{MiSulink Nonprofit Kft.} \\
                  fenntartó\\
                  képviseletében: Halácsy Péter ügyvezető \\
            \end{center}
      \end{tabular}
\end{center}
\newpage