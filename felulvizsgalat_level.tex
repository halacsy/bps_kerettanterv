\pagestyle{empty}

\noindent
\textit{Prof. Dr. Kásler Miklós Miniszter Úr \\
      részére}
\vspace{0.75cm}

\noindent
\textbf{tárgy}: felülvizsgálat

\noindent
\textbf{iktatószám}: 14259-7/2019/KÖZNEVTART
\vspace{0.75cm}

\noindent
Tisztelt Miniszter Úr!
\vspace{0.75cm}

\noindent
A MiSulink Nonprofit Korlátolt Felelősségű Társaság (székhely: 1118 Budapest, Breznó köz 8., Cg. 01-09-209663), mint a Budapest School Általános Iskola és Gimnázium (nyilvántartási szám: 702.) fenntartója azzal a tiszteletteljes kéréssel fordulunk Önhöz, hogy a mellékelt kerettanterv-módosítást jóváhagyni szíveskedjen!

Tisztelt Miniszter Úr nevében eljárva Dr. Maruzsa Zoltán Helyettes Államtitkár Úr 2019. április 30. napján \texttt{14259-7/2019/KOZNEVTART} számon jóváhagyta a \textit{Budapest School Kerettanterv} elnevezésű kerettantervünket.

A 2019. július 26. napjától hatályos a köznevelési törvényt módosító 2019. évi LXX.  sz. törvény. A módosító törvény következtében az Nkt. 9. § (9a) bekezdéssel egészült ki. Ennek értelmében:

\begin{quote}
,,(9a) A (9) bekezdés alkalmazása során az alternatív kerettanterv csak akkor hagyható jóvá, ha megfelel a következő szempontoknak: \\
a) a Nat-ban meghatározott tananyag tartalmakat tanévenként két félévre bontva kell megjeleníteni a kerettantervekben, összhangban a tanulók félévenkénti értékelésével, \\
b) a Nat-ban foglalt műveltségi területek adaptálása során az iskolák közötti átjárhatóság és a továbbtanulás biztosítása érdekében az alternatív kerettanterv tantárgyi struktúrája legfeljebb harminc százalékban térhet el az oktatásért felelős miniszter által kiadott kerettantervben foglalt tantárgyi struktúrától.” \\
\end{quote}

A fenti előírásoknak a jelen kérelmünkhöz mellékelt módosított, egységes szerkezetbe foglalt kerettantervünkkel - álláspontunk szerint - megfelelünk, röviden összefoglalva az alábbiak miatt. 

\begin{itemize}
      \item A kerettanterv a miniszter által kiadott kerettantervek tantárgyi struktúráját vette át, így a három tantárgy helyett, most húsz tantárgyban csoportosítja a tanulási eredményeket. A tantárgyakat \aref{sec:tantargyak}~.fejezet mutatja be.
      \item A tantárgyakhoz rendelt heti óraszámok a miniszter által kiadott kerettantervben megadott óraszámok 75\%-a \aref{sec:tantargyak}.~fejezetben található \ref{tbl:oraszamok}.~táblázatban.
      \item A kerettanterv a tananyag tartalmakat tanulási eredmények halmazaként adja meg tantárgyanként és félévenkénti bontásban \aref{sec:tantargyi_tanulasi_eredmenyek}.~fejezetben.
      \item A tanév trimeszterenkénti bontása megmaradt, ugyanakkor a kerettanterv egyértelműen kijelöli a helyét, idejét és módját a félévenkénti értékelésnek \aref{sec:tanev_ritmusa}.
      \item A bevezető részben egy ábra szemlélteti, hogyan tud az iskola egyszerre önvezérelt tanulást támogató, személyreszabott tanulási környezetet biztosítani, a törvényi társadalmi elvárásoknak megfelelni és átjárhatóságot biztosítani a gyerekek számára.
      \item \aref{sec:modulok_es_tanulasi_eredmenyek}~.fejezet bemutatja, hogyan kapcsolódnak a modulok a tantárgyakhoz a tanulási eredményeken keresztül.
\end{itemize}


\noindent
\emph{Az Nkt. 99/K. §-ban foglalt kötelezettségünknek eleget téve kérjük a Tisztelt Miniszter Urat, hogy a mellékelt módosított kerettantervet jóváhagyni szíveskedjen!}

\vspace{0.75cm}

\noindent
Budapest, 2019. július 29.

\vspace{0.75cm}
\noindent
Tisztelettel,

\vspace{0.75cm}
\noindent
\begin{center}
      \begin{tabular}{p{8cm}}
            \begin{center}
                  \hrulefill \\
                  \textbf{MiSulink Nonprofit Kft.} \\
                  fenntartó\\
                  képviseletében: Halácsy Péter ügyvezető \\
            \end{center}
      \end{tabular}
\end{center}
\newpage