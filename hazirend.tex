\documentclass{article}
\usepackage[utf8]{inputenc}
\usepackage{t1enc}
\def\magyarOptions{defaults=hu-min}
\usepackage[magyar]{babel}
\title{Budapest School házirend}
\author{Budapest School tanárai}

\date{\today}
% Hint: \title{what ever}, \author{who care} and \date{when ever} could stand 
% before or after the \begin{document} command 
% BUT the \maketitle command MUST come AFTER the \begin{document} command!  
\begin{document}

\maketitle

\tableofcontents
\pagebreak
\begin{abstract}
    A házirend a 20/2012. (VIII. 31.) EMMI rendelet 5.§ által meghatározott tartalmú és formájú házirend. Elsődleges célja, hogy a Budapest School Általános Iskola és Gimnázium indulásához szükséges minimális szabályokat rögzítse. Indulás után az iskola tanárainak a feladata, hogy a házirendet a saját gyakorlatához, kultúrájához alakítsa. A házirend a kerettanterv, pedagógia program és szervezeti működési szabályzat elveivel, értékeivel és gyakorlataival együtt értelmezhető. 
    \end{abstract}


     

\section{Saját szabályok alakítása}
Minden mikroiskolai közösség a tanulásszervező tanárok vezetésével ki  az alábbi házirenden felül ki kell, hogy alakítsa a saját szabályaikat, normarendszerüket, szokásaikat és kultúrájukat. A szabályokat a gyerekekkel közösen kell hozni, és azokat mindenkinek be kell tartani.

Ha valaki, valamilyen szabályt nem tart be, és ez más zavar, akkor a konfliktuskezelés c. fejezetben leírtakat kell alkalmazni.


\section{A gyermek, tanuló távolmaradásának, mulasztásának, késésének igazolására vonatkozó előírások}

A Budapest School feladata, hogy olyan környezetet biztosítson a gyerekeknek, amiben boldogak, felszabadultak, magabiztosak és hatékonyak tudnak lenni. Budapest School családok maguk és önszántukból választják ezt az iskolát, áldoznak sok időt és energiát arra, hogy az iskolában tudjanak tanulni. \emph{Ezért az iskola feltételezi, hogy a gyerekek az iskolában akarnak tanulni önszántukból}.

Sok oka lehet annak, hogy egy gyerek még sincs az iskolában. Például
\begin{itemize}
    \item betegnek, fáradtnak érezheti magát, fizikailag vagy lelkileg kimerült, vagy lehet valamilyen fertőző betegsége;
    \item családjával tölt értékes, minőségi időt, mert fejlődését ez szolgálja a leginkább;
    \item előre nem tervezett esemény miatt nem tud az iskolába menni;
    \item utazik, felfedez, külső helyszínre szervezett tanulási programokon vesz részt;
    \item egy projektjébe úgy belemerül, hogy érdemesnek találja nem az iskolában, fóku\-száltan végezni a munkát.
 \end{itemize}

A fenti példák is két jól elkülöníthető kategóriába sorolhatóak. A \emph{nem tervezett hiányzástól} jól elkülöníthetőek azok az esetek, amikor a gyerek, bár nem az iskolában tartozkodik, mégis szervezett, struktúrált módon biztosított a fejlődése, tanulása. Ezt az esetet a \emph{,,tervezett tanuláscélú távolmaradásnak''} hívja az iskola, mert ezt az időt is tanulásra szánjuk.


Az iskola megközelítése egyszerű: mivel partneri viszonyban van a tanár, gyerek és szülő, ezért az alapértelmezett az, hogy őszintén és nyiltan megbeszélünk dolgokat. Ha valami betegség, kimerülés, rosszullét közbe jön, akkor a szülő meg tudja ítélni, hogy érdemes-e iskolába jönni, vagy ha nagy a baj, akkor úgyis orvoshoz megy a család és hoz igazolást. 

Partneri viszonyból következik, hogy elvárjuk: egyszerűen ne legyenek igazolatlan órák. Ne forduljon elő, hogy egy gyerek nem jön iskolába, egy óráról kimarad és erről nem tudunk.

Elég tág keretet enged az iskola. Abban az esetben azonban, amikor a gyerek vagy a szülő nem tartja be a kereteket, nem él a partneri viszonnyal, akkor ott valami baj van. Gyorsan kell reagálni.

Ha a gyerek hirtelen, nem tervezett módon nem tud iskolába jönni, akkor a szülő feladata, hogy erről reggel 9 óra előtt értesítse a mentortanárt minden esetben. Mikroiskolánként eltérhet a preferált kommunikációs eszköz, ezért a tanulásszervezők feladata meghatározni az értesítés formáját.


\paragraph{Szülő által igazolt hiányzás.} Egy tanévben 15 munkanap, vagy 120 óra, de alkalmanként csak 5 munkanap, nem megtervezett hiányzást igazolhat a szülő, 
(rögzített és dokumentált módon). Orvos által igazolt betegség, hatósági intézkedés és egyéb alapos indok esetén a 20/2012.~(VIII.~31.) EMMI-rendelet 51.~§~(2) értelmében igazoltnak kell tekinteni a hiányzást.

\paragraph{Igazolatlan hiányzásnak} azt az esetet kell tekinteni, amikor a szülő vagy a mentortanár nem tudott a hiányzásról, nem volt előre megtervezve, vagy a 15 napos, 120 órás keret kimerült. Ilyenkor a 20/2012. (VIII.~31.) EMMI-rendelet 51.~§~(3) pontja értelmében minden esetben az iskola értesíti a szülőt, és 10 igazolatlan óra után figyelmezteti, hogy a következő igazolatlan után ,,az iskola a gyermekjóléti szolgálat közreműködését igénybe véve megkeresi a tanuló szülőjét'' (idézés az EMMI-rendeletből). 


\subsection{Késések kezelése}
A Budapest School mikroiskolák maguk állítják fel a napirenddel kapcsolatos kereteket: mikor kezdenek, meddig tartanak a strukturált foglalkozások, mikor vannak a szünetek és hogyan kezdődik újra a nap folyamán a fóku\-szált munka. A kereteket a tanulásszervező tanárok feladata kialakítani és trimeszterenként megkezdése előtt kihirdetni.

Fontos, megbeszélendő részlet, hogy hogyan kezeli a közösség a késéseket: mikortól lehet érkezni, mikor kezd a közösség annyira dolgozni, hogy zavaró, amikor valaki belép és megzavarja a folyamatot. Megállapodást köt a közösség, hogy hogyan kívánja kezelni a késéseket, mi segíti a csapatot leginkább a céljai elérésében.

Az iskola nem regisztrálja a késéseket, mert az iskola nem tudhatja, hogy egy-egy késés elfogadható-e a közösségnek vagy nem. Egy színdarab főpróbájáról 5 percet késni mást jelent a közösség számára, mint arról az óráról, ahol mindenki egyedül füllhallgatóval böngészi egy online tananyag számára legrelevánsabb fejezetét.

Ha egy csoportot megzavar valakinek az ismételt késése, akkor konfliktus alakul ki a csoport és a késő vagy a tanár és a késő között. Ezt a típusú konfliktust (is) a pedagógia program konfliktusok kezelése c.~fejezet szerint kell feloldani.

\section{Ki viheti el a gyereket?}

Gyereket bárki behozhat az iskolába, de elvinni csak az viheti el, akit a gyerek szülei -- a fenntartó által üzemeltetett számítógépes rendszeren keresztül -- felhatalmaztak. Egyedül gyerek csak akkor mehet el az iskolából, ha erre a szülei engedélyt adtak. Ezek a felhatalmazások lehetnek egyszeriek, határidőhöz kötöttek vagy visszavonásig érvényesek.



\section{Tervezett, tanuláscélú távolmaradás rendje}
Az 20/2012. (VIII.~31.) EMMI-rendelet 51.~§~(2) pont felhatalmazása alapján a házirend igazoltnak tekinti azt a \emph{tervezett tanuláscélú távolmaradást}\footnote{Van, amikor a \emph{,,távmunka''} mintájára \emph{,,távtanulásnak''} hívjuk, mert ezt az időt is tanulásra szánjuk.}

Alapelv, hogy \emph{a mentornak, gyereknek és szülőnek előre meg kell állapodni a távolmaradásról}. Minden félnek tudnia kell róla, a szülőnek írásban kell kérnie (lehet számítógépes rendszerek kereszült) meg kell előre tervezni és nem lehet esetleges. Mindenképp meg kell különböztetni a \emph{nem tervezett hiányzástól}.

\emph{Tervezett tanuláscélú távolmaradás} tehát az, amikor előre eltervezett módon, valamilyen program miatt nincs a gyerek az iskolában. Ilyenkor a mentor és a gyerek megtervezi a tanulás célját, várható eredményeit. A terv létrejöttéért a gyerek és a szülő felelős, és minden félnek el kell fogadnia a tervet. Tehát a mentortanárnak hozzá kell járulnia. Ha a mentortanár nem járul hozzá, akkor addig nem kezdhető meg a távolmaradás, amíg a pedagógia program konfliktusok kezelése c.~fejezet alapján megállapodás nem születik.\footnote{Ha mégis, akkor azt nem igazolt hiányzásnak kell tekinteni.}

Ha a Tervezett tanuláscélú távolmaradás elérte a 20 napot, vagy 160 órát, akkor a mentortanár mellett egy másik mentortanár szerepben dolgozó tanulásszervezőnek is meg kell ismernie és el kell fogadnia a tervet. 40 nap felett három mentortanárnak kell együtt elfogadnia a tervet, melyek egyike egy másik mikroiskola mentortanára. Eföllötti távtanulás már egyéni tanrendnek vagy egyéni munkarendnek számít, amire más szabályok vonatkoznak.


\section{A térítési díj, szülői hozzájárulás, visszafizetésére vonatkozó rendelkezések}

Minden mikroiskolánkat az adott közösség finanszírozza a szülői hozzájárulásokból és az állami normatíva rá eső részéből. A vállalt hozzájárulásokat minden hónap 10-ig kell megfizetni a vállalása alapján. Vissza a szülők nem kérnek pénzt, hiszen a tanárok sem adnak vissza fizetést.

\section{A tanuló által előállított termék, dolog, alkotás vagyoni jogára vonatkozó díjazás szabályai}
A tanulók nem kapnak díjazást az iskolában előállított termékek, dolgok és alkotások után.

\section{A szociális ösztöndíj, a szociális támogatás megállapításának és felosztásának elveit, a nem alanyi jogon járó tankönyvtámogatás elvét, az elosztás rendje}
Nem műkdöik az iskolában még szociális ösztöndíj. Minden család maga dönt arról, hogy mennyi hozzájárulást tud az iskolának adni.

A Budapest School közösségeivel egy felelős társadalmat próbálunk modellezni, ahol a közösség tagjai segítik egymást. Jelenleg a gyerekek 10\%-a után nem fizetnek a szülők tanulási hozzájárulást, 20\%-uk fizet nagyjából fele annyit, mint a valós költségek, és további 10\% valamennyivel kevesebbet, mint amennyi az egy gyerekre jutó tényleges költség. Azaz úgy is fogalmazhatunk, hogy a közösség 60\%-a támogatja, hogy az a 40\% is velünk tanulhasson, aki anyagi okokból egyébként kimaradna.

\section{A tanulók véleménynyilvánításának, a tanulók rendszeres tájékoztatásának rendje és formái}
Mivel a tanulók életét érintő döntések a mikroiskola tanári csapatában történnek, ezért nagyon fontos, hogy a tanulóknak jó kapcsolata legyen a tanárokkal, kifejezetetten a mentortanárral.

A mikroiskolákban a bejelentkező körökben, a fórumokon minden gyereknek lehetősége van elmondani, hogy érzi magát, mi a véleménye a közösségi dolgokról. Minden gyerek mentor tanárának a feladata, hogy a gyerek tudjon mindenről, ami őt érinti.


\section{A gyermekek, tanulók jutalmazásának elvei és formái}
A gyerekek erőfeszítéseit, tevékenységeit értékeljük a kerettantervben és a pedagógia programban részletezett módon.

\section{A fegyelmező intézkedések formái és alkalmazásának elvei}
Ha valakinek a viselkedése zavarja a tanárt vagy mást, akkor a pedagógia program konflikutkezelés fejezet alapján kell eljárni. Nem fegyelmező intézkedéseink vannak, hanem megbeszéljük, hogyan tudunk jól együtt működni, és ha zavar támad, akkor helyreállítjuk a kapcsolatot egymás között a resztoratív megközelítés szerint.

\section{Elektronikus napló használata esetén a szülő részéről történő hozzáférés módja}
A szülők mobil alkalmazáson és weboldalon férhetnek hozzá az elektronikus naplóhoz, a \texttt{app.budapestschool.org} címről indulva.

\section{Az osztályozó vizsga tantárgyankénti, évfolyamonkénti követelményeit, a tanulmányok alatti vizsgák tervezett ideje, az osztályozó vizsgára jelentkezés módje és határideje}
Osztályozó vizsga követelményeit a kerettanterv tantárgyi tanulási eredményi rögzítik. Ettől eltérő követelményt nem szabad megállapítani. A keretanttervben és a pedagógia programban ismertetett módon a kijelölt tanárok összevetik a portfólió elemeket az elvárt tanulási eredményekkel. Akkor és csak akkor lehet évfolyamot léptetni, vagy osztályzatot kiadni, ha a gyerek a szükséges eredményeket a portfólió alapján bizonyíthatóan elérte.

Tanulmányok alatt nincsenek előre tervezett vizsgák. Bármikor érdemes szintfelmérőt, tudáspróbát végezni, amikor a tanár vagy a gyerek úgy érzi, hogy egy új szintet ért el a gyerek.

Osztályozó vizsgára jelentkezés módja és határideje a pedagógia programban került rögzítésre.
\section{a tankönyvellátás iskolán belüli szabályai}
A tankönyvellátásnak nincs iskolán belüli szabályai. Az internetet mindenki számára elérhetővé kell tenni az iskolában. 

\section{Az iskolai tanulói munkarend, a tanórai és egyéb foglalkozások rendje}
Az iskolában a tanárok által struktúrált, tervezett időszak reggel 9 és délután 4 óra közé esik az esetek többségében. A foglalkozások hosszát a tanárok határozzák meg a trimeszterek elején, amikor kialakul a modulstruktúra.

Minden gyereknek szükséges szabad játékra és egyéni tanulásra is időt fordítani, legalább hetente 4 órát.

\section{A tanulók tantárgyválasztásával, annak módosításával kapcsolatos eljárási kérdések}
A modulok struktúrája a trimeszter első napján kerül meghirdetésre. Innentől kezdve egy hete van a tanulóknak választani, a választható modulokból. Módosításra akkor van lehetőség, ha minden érintett modulvezető és a mentor tanár hozzájárul a módosításhoz.

\section{Az iskola helyiségei, berendezési tárgyai, eszközei és az iskolához tartozó területek használatának rendje}
Alapszabály: a mikroiskola dolgai a közösség dolgai. Bánjál jól a közös dolgokkal! A közösség, a csoport az osztály alakítsa ki a saját szabályait és azok betartásáról is beszéljen.

\section{Az iskola által szervezett, a pedagógiai program végrehajtásához kapcsolódó iskolán, kollégiumon kívüli rendezvényeken elvárt tanulói magatartás}
Ahogy az angol mondja: play well with others, azaz játssz jól a másokkal.

\section{Házirend megváltoztatása}
A házirendért egy csapat felelős. A csapatba minden mikroiskola csapata delegálhat egy tagot. A házirend csapat vezetője a fenntartó. A házirend módosításra minden gyerek, tanár, szülő tehet javaslatot. Módosítások elfogadását a hozzájárulás-alapú döntéshozással kell elvégezni.
\end{document}