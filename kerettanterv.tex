\documentclass[magyar,12pt,a4paper,oneside,draf]{report}
\usepackage{graphicx}

\usepackage[utf8]{inputenc}
\usepackage[T1]{fontenc}
\usepackage[magyar]{babel}
\usepackage{verbatim}
\usepackage{booktabs}
\usepackage[colorinlistoftodos,prependcaption,textsize=tiny]{todonotes}

\usepackage{longtable}
\usepackage{array}
\usepackage[justification=centering]{caption}
\usepackage{url}
\usepackage[nottoc,numbib]{tocbibind}
\DeclareUnicodeCharacter{2212}{-}
% heylyseglista tablazat miatt
\usepackage{graphicx}
\usepackage{lscape}
\usepackage{natbib}
\begin{document}

% csinalunk egy feltetelt, hogy a szovegben tudjuk, hogy kerettantervben vagyunk
% vagy a pedprogramban
\newif\ifkerettanterv
\kerettantervtrue

\title{Budapest School Kerettanterv}
\author{}
\date{2019. m\'arcius}
\maketitle

\tableofcontents
\newpage

Egyéni cél mindenhol egységesen saját cél lett, kifejezve, hogy nem biztos,
hogy egyéni, viszont a gyerek sajátjának kell hogy érezze.

Bekerültek a kötelező tanulási eredmények
\ref{sec:kotelezo_tanulasi_eredmenyek}. fejezetként.
A Budapest School Általános Iskola és Gimnázium 12 évfolyammal működő nevelési-oktatási intézményként ellátja az általános iskola és a gimnázium feladatait.

Az iskola a miniszter által kiadott kerettantervek tananyagtartalmát kínálja a gyerekeknek. A tanulás szervezését, azaz a pedagógiáját a motiváció \citep{pink2011drive}\footnote{Az irodalomjegyzéket lásd a \pageref{sec:bibliographyk}. oldalon}, a fejlődési szemlélet \citep{growthmindset} és az elmélyült gyakorlás \citep{ericsson2016peak} pszichológiai kutatási eredményei, az  OECD, azaz a Gazdasági Együttműködési és Fejlesztési Szervezet \emph{Az iskolázás a jövőben}  (\emph{Schooling for Tomorrow}) programjának eredményei \citep{2006schooling} és a modern tanulásszervezési paradigmák, mint az önvezérelt \citep{mitra2012beyond} és a személyre szabott \citep{khan2012one} tanulás alapján határozza meg.

Az iskola egyszerre akar a gyerekek számára egy önvezérelt és személyreszabott tanulási környezetet biztosítani, ami képes agilisen reagálni a gyerekek és a környezet igényeire és egy stabil, biztonságos, kiszámítható rendszert is adni, ami biztosítja az iskola és más iskolák közötti átjárhatóságot, a továbbtanulást. Ezt a két szándékot ötvözi a Budapest School modell.

\clearpage
\includepdf{pics/CELOK_EGYENSULYA.JPG}

\paragraph{Egyéni és közösségi szempontok} A Budapest School modellben fontos szempont, hogy
hol, milyen
módon, kitől és mit tanulhat egy Budapest Schoolba járó gyerek.
A gyerekek igényeire gyorsan reagáló mikroiskola egy buborék a gyerek körül. Ebben a körben a Budapest School Modell a tanulás folyamatát szabályozza, a \emph{mit tanulunk?} kérdés mellett a \emph{hogyan szervezzük meg a tanulást?} kérdésre ad választ. A tanulás során a gyerekek a Budapest School Modellben részletezett módon az érdeklődésüknek megfelelően specifikus tanulási egységeket, vagyis modulokat végeznek el, melyek eredményeit a saját portfóliójukban gyűjtik össze. Így a gyerekek tanulási útja a portfólió fejlődésével nyomon követhető, és a portfólió tartalma alapján megállapítható a gyerek aktuális tudása, képessége. Az iskola fő funkciója emellett mindvégig az aktív tanulás, saját fejlődésük kereteinek megtalálása, a folyamatosan újragondolt saját célok állítása, és e célok irányába történő haladás marad.
 
A Budapest School Modell a gyerekek, tanárok és a szülők közös döntésére bízza, hogy a gyerekek mit és hogyan tanulnak a modellben meghatározott kereteken belül. A modell a specifikus célkitűzés-tervezés, a tanulás, és az arra történő reflektálás módját írja le, vagyis a tanulás folyamatát rögzíti, míg annak pontos tartalmában szabadságot enged.

\paragraph{Társadalmi szempontok}  Ezt a szabadságot keretezik a társadalmi normák és jogszabályi elvárások, hogy biztosítva legyen a mikroiskolán kívüli boldogulása is a gyerekeknek. A Budapest School Modell
 a miniszter által kiadott kerettantervekre \citep{ofi:kerettanterv} épül. A helyi tanterv a tantárgyak tartalmát tanulási eredmények halmazaként adja meg. A tanulási eredmények féléves bontása, és a tantárgyi specifikációk lehetővé teszik a személyreszabott portfóliók osztályzatokra váltását egy átlátható folyamaton keresztül.

A Budapest School Modell transzparensé teszi az iskolában működő kétszíntű struktúrát: a gyerekközpontú, személyreszabott, saját tervezésen alapuló mindennapi tanulási élményt folyamatosan leképezzük a miniszter által kiadott kerettantervek tantárgyi struktúrájának, A félévenkénti osztályzatok biztosítják az átjárhatóságot és a továbbtanuláshoz szükséges feltételeket. A kettős rendszer, a személyre szabott belső buborék és a kiszámíthatóságot adó külső kör biztosítja, hogy a 12.~évfolyam végén a gyerekeknek lehetőségük van arra, hogy érettségi vizsgát tegyenek.

\paragraph{Pedagógiai módszerek}
A Budapest School Model semleges a pedagógiai módszerekkel kapcsolatban, a tanár feladatának tekinti, hogy mindig az optimálisnak tűnő tanulási, tanítási, gyakorlási módszert válassza. Ezért ez a modell nem beszél arról, hogy a modulokat (foglalkozásokat, tanórákat) milyen pedagógiai módszer alapján szervezi a tanár.

A Budapest School Model az iskolába járókat gyerekeknek hívja, nem tanulóknak és nem diákoknak. Ennek fő oka, hogy a rendszerünkben a tanárok és a szülők is tanulók, sőt az egész iskola egy tanuló szervezet, így ez a szöveg nem akarja  kizárólag az iskola egyik szereplőjére alkalmazni ezt a szót. Másodsorban a modell hangsúlyozza, hogy a \emph{család} fontos szerepet kap a Budapest School rendszerében: az iskolában a szülők, a gyerekek és a tanárok együttműködésben dolgoznak a fejlődésért. Azok a gyerekek, akik tanulmányaik vége felé felnőtté érnek a Budapest Schoolban, tanulók is maradnak, és talán egy kicsit gyerekek is, ezért a szóhasználaton miattuk sem változtatunk. Az ő esetükben a gyerek az iskolába járó tanulót jelenti.




\chapter{Tanulás szervezése}
A Budapest School Általános Iskola és Gimnázium egy több helyszínen működő tanulási hálózat, amelynek célja, hogy otthonos környezetben, rugalmas, mégis jól szabályozott keretek között integrálja a NAT műveltségi területeit, fejlesztési céljait és kulcskompetenciáit a gyerekek saját tanulási céljaival. A különböző helyszínek, a Budapest School \emph{mikroiskolái}, legfeljebb hat évfolyamot átölelő összevont osztályokként, 6--60 fős tanulási közösségekként működnek (ld.~\ref{sec:mikroiskola}.~fejezet), ahol a gyerekek többfajta csoportbontásban tanulnak attól függően, hogy a Budapest School kerettantervében megfogalmazottaknak megfelelően milyen területeken kell, és milyen területeken akarnak fejlődni.

Az egyes mikroiskolákat \emph{tanulásszervező} tanárok (tanulásszervezők) csapata vezeti. Minden gyereknek van egy kitüntetett tanára, a \emph{mentora}, aki egyéni figyelmével a fejlődésben segíti (ld.~\ref{sec:tanarok}.~fejezet). Minden gyerek a mentortanára segítségével és a szülők aktív részvételével trimeszterenként meghatározza a \emph{saját tanulási céljait}
(ld.~\ref{sec:tanulasi_celok}).

A tanulásszervezők \emph{modulokat} hirdetnek ezen célokból és a kerettanterv tantárgyainak tartalmából. A modulok reflektálnak a mai világ alapvető kérdéseire, integrálják	a tudományterületeket és művészeti ágakat (tantárgyakat), és egyenlő lehetőséget adnak a tudásszerzésre, az önálló gondolkodásra és az alkotásra a gyerekek mindennapjaiban 
(ld.~\ref{sec:modulok}.~fejezet).

A modulok végeztével a gyerekek eredményei bekerülnek saját portfóliójukba (ld.~\ref{sec:portfolio}.~fejezet), melyek tartalmazhatnak önálló vagy csoportos alkotásokat, tudáspróbákat, vizsgafeladatokat, egymás felé történő visszajelzéseket, a fejlődést jól mérő dokumentációkat vagy bármit, amire a gyerek és tanárai büszkék vagy fontosnak tartanak. Erre a portfólióra épül a Budapest School visszajelző és értékelő rendszere (ld.~\ref{sec:ertekeles}). A portfólió alapján ismeri el az iskola az évfolyamok teljesítését (ld.~\ref{sec:evfolyamszintlepes}.~fejezet). Szükség esetén a portfólió alapján kaphatnak a gyerekek osztályzatokat is (ld.~\ref{sec:osztalyzatok}.~fejezet).

A gyerekek mindennapjait meghatározó modulok több műveltségi területet, többféle kompetenciát, több tantárgy anyagát is lefedhetik, és egy tantárgy anyagát több modul is érintheti.
Ezért is mondhatjuk, hogy a Budapest School iskolákban a tantárgyközi tevékenységek vannak előtérben. 
A kerettanterv szándéka, hogy a gyerekek folyamatosan fejlődjenek a világ tudományos megismerésében (STEM), a saját és mások kulturális közegéhez való kapcsolódásban (KULT), valamint a testi-lelki egyensúlyuk fenntartásában (Harmónia), vagyis a \emph{kiemelt tantárgyközi fejlesztési irányelvekben}. 

A modulok fejlesztési irányelvei útmutatóul szolgálnak a tanároknak arra, hogy az elérendő eredményekhez milyen elvek mentén szervezzenek modulokat.


A kerettanterv szerint az a tanárok döntése, hogy a gyerekek kémia órán kísérleteznek, vagy kísérletezés órán foglalkoznak kémiával. A kerettanterv annyit határoz meg, hogy  a 7-10. évfolyamszinten kémia tantárgyhoz kapcsolódóan 17 különböző tanulási eredményt kell elérni, és kísérletezéssel kapcsolatban pedig 15 különböző tanulási eredményt több különböző tantárgyból (amikből csak 3 kapcsolódik a kémia tantárgyhoz).

Tehát a tantárgyak a tanulás tartalmi elemeinek forrása és keretei: a tanulandó dolgok halmazaként működik. Az, hogy milyen csoportosításban történik a tanulás, az a modulvezetőkre van bízva. A gyerekek lehet, hogy csak félévente, az elszámolás időszakában találkoznak a tantárgyak taxonómiájával. Ebben az időszakban veti össze minden gyerek és mentor, hogy amit tanultak, alkottak és amiben fejlődtek hogy viszonyul a társadalom és a törvények elvárásaival, a Nemzeti Alaptantervvel és a kerettantervvel.

A kerettanterv a tantárgyak témaköreit, tartalmát és követelményeit \emph{tanulási eredmények} halmazaként adja meg (ld.~\ref{sec:tanulasi_eredmenyek}). A gyerekek feladata az iskolában, hogy tanulási eredményeket érjenek el és így sajátítsák el a tantárgyak által szabott követelményeket. Tanulási eredményeket modulok elvégzésével (is) lehet elérni, tehát a modulok elsődleges feladata, hogy a tanulási eredményekhez vezető utat mutassák.

Az iskolában -- a kerettanterv szándéka szerint -- egyszerre jelennek meg a miniszter által kiadott kerettantervek tantárgyi elvárásai, a legújabb NKT módosítás szándékai, a gyerekek saját céljai és a mai világra való integrált reflexió. 
\section{A mikroiskolák, a Budapest School összevont osztályai}
\label{sec:mikroiskola}

A Budapest School iskolában összevont osztályokban, azaz kevert korosztályú és maximum 6 évfolyamszintű közösségben tanulnak a gyerekek. Az összevont osztályokat a Budapest School kerettanterv \emph{mikroiskoláknak} hívja, ezzel is hangsúlyossá téve ezek egyedi jellemzőit:
\begin{itemize}
      \item A mikroiskolákban  tanulásszervező	\emph{tanárcsapatok} vezetik a közösségeket.
      \item A mikroiskolák maguk alakítják ki saját szabályaikat, normarendszerüket, szokásaikat és kultúrájukat.
      \item A mikroiskolák maguk alakítják saját órarendjüket, modulkínálatukat, és ebben nem kell más mikroiskolákhoz igazodniuk.
\end{itemize}

\paragraph{A mikroiskolákat tanulásszervezők vezetik.}

A tanulóközösség fontos célja, hogy biztonságot, támogatást nyújtson, és \emph{így} segítse a közösség tagjainak a minőségi tanulását.

A mikroiskolát az igazgató által kinevezett tanulásszervezők irányítják. Ők felelnek a tanulás tartalmáért, a modulok meghirdetéséért és a tanulási eredmények nyomon követéséért. Ők döntenek a jelentkező gyerekek kiválasztásáról és a mikroiskola mint közösség összetételéről.

A különböző tanárszerepeket, a tanulásszervezők, mentorok és modulvezetők kapcsolódását \aref{sec:tanarok}.~fejezet mutatja be.

\paragraph{Mikroiskola egy nagy csoport.}

Egy mikroiskola minimális létszáma 6, maximális létszáma 60 fő. Minden mikroiskolának megfelelő számú olyan tanulásszervezővel kell rendelkeznie, aki mentortanárként is végzi munkáját.

\paragraph{A mikroiskolák korkülönbségei állandók, a gyerekek együtt nőnek.}
Egy mikroiskolában fő szabály szerint legfeljebb 6 (egymást követő) évfolyamszintnek megfelelő korosztály tanul együtt. A mikroiskola korosztályait, és az induláskor meglévő évfolyamszinteket a jelentkező, illetve a felvett gyerekek életkora alapján határozza meg az alapító.

A mikroiskolák korhatára, mint az összevont osztályok korhatárai, a gyerekekkel változnak. Ettől eltérően a korhatárokat tágítani és szűkíteni évente egyszer lehet, és erről az iskola értesíti a szülőket minden tanévkezdést megelőző február 15-ig (például amennyiben ezzel nem sérül a maximum 6 évfolyam elve, a korhatárok tágíthatóak, vagy ha a legalsó vagy legfelső korosztályba tartozó gyerekek elmentek, a mikroiskola dönthet a korhatárok szűkítéséről). A Budapest School mikroiskolái a 12.~évfolyamig tartanak, kivéve, ha a mikroiskola valamilyen okból korábban megszűnik.

\paragraph{A mikroiskola állandó, a gyerekek és tanárok jöhetnek és mehetnek.}
A Budapest School mikroiskolái úgy működnek, mint egy összevont osztály. A tanulásszervező tanárok vagy gyerekek kilépése a mikroiskola fennállását nem érinti, helyettük a mikroiskola új tanulásszervező tanárt és gyereket vehet fel.  A mikroiskola létrehozásakor arra kell törekedni, hogy olyan gyerekek tanuljanak együtt, akik támogatni tudják egymást a tanulásban. A gyerekek a mikroiskola tagjai addig, amíg ott jól tudnak tanulni, és a közösség és a gyerek kapcsolata gyümölcsöző.

\paragraph{A mikroiskoláknak saját fókuszuk, helyszínük, stílusuk alakulhat ki.}
A mikroiskolák nemcsak abban térnek el egymástól, hogy kevert korcsoportban, más korosztályú gyerekek, más érdeklődések mentén, és ily módon más célokat követve tanulnak, hanem területileg, regionálisan is eltérőek lehetnek.

A mikroiskola-rendszerben lehetőség van arra, hogy adott tanulási környezetben úgy váltakozhassanak a hangsúlyok a csoport és az egyén érdeklődését követve, hogy közben fennmaradjon a tanulási egyensúly a tantárgyak között.

Van olyan mikroiskola, amely a fejlesztési célok eléréséhez és a saját célok mentén már 6 éves gyerekek tanulásánál a robotika eszközeit használja, másutt drámafoglalkozásokkal fejlesztik 12 éves gyerekek a szövegértésüket és éntudatukat.

\paragraph{A mikroiskolákban a gyerekek nagymértékben befolyásolják, hogy mit és hogyan tanulnak és alkotnak.}
A mikroiskolákban (a tanulásszervezők által meghatározott kereteken belül) megfér egymással több, különböző saját céllal rendelkező gyerek addig, amíg a tanulásszervezők minden gyerek számára biztosítani tudják a kerettantervben megfogalmazott tanulási eredmények elérését.

A tanulásszervezők feladata és felelőssége, hogy olyan közösségeket építsenek, amelyek kellően diverzek, és mégis jól működnek. A közösségnek a gyerekek igényeit és a kerettanterv céljait egyaránt ki kell elégíteni.

A tanulásszervezők választási lehetőségeket kínálnak (azaz modulokat dolgoznak ki), amikből a gyerekek (a mentoruk és szüleik segítségével) a saját céljaikat, érdeklődésüket leginkább támogató saját tanulási tervet és utat alkotnak.

Eltérhet, hogy egy-egy gyerek mit tanul, ezért az is, hogy mikor és hogyan sajátítja el a szükséges ismereteket: egy közösségben megfér a központi felvételire fókuszáló 11 éves gyerek, és az is, aki ekkor inkább a Minecraft programozásában akar elmélyedni, ezért más képességek fejlesztésével lassabban halad.

\paragraph{Kisebb csoportokban tanulhatnak a gyerekek.}
\label{sec:csoportbontasok}

A mikroiskolákban a közösséget kisebb csoportokra bonthatjuk, ha a tanulásszervezés ezáltal hatékonyabb. Egyes moduloknál a gyerekek egy-egy projektre szerveződnek, ilyenkor általában az eltérő képességű és életkorú gyerekek is kitűnően tudnak együtt dolgozni. Más modulok esetén a csoportokat a tanár képességszint alapján hozza létre. Ilyen csoportok lehetnek a másodfokú egyenletek megoldóképletét megismerő csoport, az írni tanulók csoportja, vagy egy angol nyelvű újság szerkesztésére és megírására alakult modul, ahol a nyelvismeretnek és a szövegalkotási képességnek már egy olyan szintjén kell állni, hogy a projektnek jól mérhető kimenete lehessen.

\paragraph{A mikroiskolák diverz, integratív közösségek.}
A Budapest School mikroiskolák társadalmi, kulturális és gazdasági értelemben is diverzek, és egyik fő céljuknak tekintik az integrációt mindaddig, amíg az a közösség céljait szolgálja.

\paragraph{A mikroiskolák tanuló közösségek.}
A Budapest School célja, hogy a mikroiskolákban történő tanulás mind a gyerek, mind a tanár, mind a szülő számára jól átlátható, követhető legyen, és a gyerek és a közösség folyamatosan fejlődjön. A Budapest School kiemelt elve, hogy mindig, minden módszer, folyamat fejleszthető, ezért a tanárok feladata, lehetősége, hogy az aktuális helyzethez illő legmegfelelőbb módszert válasszák meg a gyerekek tanulásának segítéséhez.

\paragraph{Mikroiskolát a fenntartó indít, és addig él, amíg szolgálja a gyerekek tanulását.}
A mikroiskolát a \emph{fenntartó} indítja el. Meghatározza, hogy a mikroiskola milyen telephelyen, tagintézményben, milyen korhatárokkal és létszámokkal induljon. A fenntartó feladata a szükséges épületet és eszközöket biztosítani. A mikroiskola alapításához legalább 6 gyerek és egy tanulásszervező (aki értelemszerűen mentortanár) szükséges.

A mikroiskola a tanárokon és a gyerekeken is túlmutató közösség, amely akkor is tovább működik, ha egy tanár vagy gyerek távozik. Tanulásszervező vagy gyerek távozása esetén a mikroiskola  új tanulásszervező tanárt, illetve gyereket vesz fel mindaddig, amíg a mikroiskola a számára meghatározott maximális létszámot el nem éri.

A mikroiskola abban az esetben szűnik meg, ha összeolvad egy másik mikroiskolával. Ha a mikroiskola létszáma 6 gyerek vagy egy mentortanár alá csökken és a létszám a következő tanév elejére sem éri el a minimiális szintet, akkor a mikroiskolát másik mikroiskolával kell összeolvasztani.

\paragraph{Csatlakozás a mikroiskolához.}
Arról, hogy egy gyerek csatlakozhat-e egy mikroiskola közösségéhez, a tanulásszervezők döntenek, figyelembe véve a gyerek életkorát, a közösségben való eligazodását, érdeklődését, saját fejlődési igényét. A kiválasztás fő elve,  hogy a közösség fejlődjön minden gyerek csatlakozásával.

Egy gyerek akkor válthat a Budapest School egyes mikroiskolái között, amennyiben a fogadó mikroiskola őt elfogadja. Ilyenkor új mentortanárt kell számára kijelölni.\footnote{Átjárás mikroiskolák között.}

\section{A tanév ritmusa}
\label{sec:tanev_ritmusa}
A tanév három trimeszter ismétlődésével írható le: a tanulási célok tervezése után következik a tanulás, és a ciklust a visszajelzés és értékelés zárja.	Amint egy ciklus véget ér, elkezdődik egy új.


Az egyes trimeszterek átlagosan 12 hétig tartanak úgy, hogy a tanítással eltöltött napok  és a tanítási szünetek mindig az EMMI által kiadott tanév rendjéhez igazodva kerülnek meghatározásra. A trimeszterek első hete mindig a tervezéssel, utolsó hete mindig az értékeléssel telik. Trimeszterenként átlagosan további egy hét a közösség építésével, önálló tanulással  zajlik.

A ciklusok állandósága adja a tanulás irányításához szükséges kereteket. Ezek megtartásáért az egyes mikroiskolák tanulásszervezői felelnek, melynek működését a fenntartó monitorozza. A tanév ritmusát \aref{tbl:tanevritmus}. táblázat mutatja.


\begin{table}
  \centering
  \begin{tabular}{ l|l }
    \textbf{időszak} & \textbf{tevékenység}                \\
    \hline
    Szeptember       &
    közösségépítés                                         \\
                     & saját célok meghatározása           \\
                     & modulok kialakítása és meghirdetése
    \\ \hline

    Október          &
    tanulás, alkotás
    \\ \hline

    November         &
    tanulás, alkotás
    \\ \hline

    December         &
    portfólió frissítése                                   \\
                     & reflexiók                           \\
                     & visszajelzések                      \\
                     & célok felülvizsgálata               \\
                     & modulok változtatása igény esetén
    \\ \hline

    Január           &
    tanulás, alkotás                                       \\
    féléves értékelés kiadása
    \\ \hline

    Február          &
    tanulás, alkotás
    \\ \hline

    Március          &
    portfólió frissítése                                   \\
                     & reflexiók                           \\
                     & visszajelzések                      \\
                     & célok felülvizsgálata               \\
                     & modulok változtatása igény esetén
    \\ \hline

    Április          &
    tanulás, alkotás
    \\ \hline

    Május            &
    tanulás, alkotás
    \\ \hline

    Fél június       &
    évzárás, értékelés, bizonyítványok
  \end{tabular}
  \caption{Egy tanévben háromszor ismételjük a célállítás, tanulás, reflektálás ciklust.}
  \label{tbl:tanevritmus}
\end{table}

A tanév három periódusból áll: ez a felosztás követi az üzleti világ negyedéves tervezését, néhány egyetem trimeszterekre bontását, de leginkább az évszakokat. Minden periódus után értékeljük az elmúlt három hónapot, ünnepeljük az eredményeket, és megtervezzük a következő időszakot.  A trimesztereken belül az egyes mikroiskolák között lehetnek néhány hetes eltérések, melyek a közösség sajátosságait követik.

\paragraph{Féléves és évvégi elszámolás}
\label{sec:feleves_bontas}
A külső rendszerekkel és a törvényeknek való megfelelés miatt a kerettanterv két félévre bontva jeleníti meg a tananyag tartalmakat, és ezzel összhangban az iskola félévenkénti értekelést állít ki automatikusan a portfólió alapján, ami nem más, mint a portfólióba bekerült értékelések összegyűjtése.

A félév vége január vége\footnote{A félév végét miniszteri rendelet évente szabályozza.}, ami a második trimeszterbe esik, ezért a féléves értékelést az első trimeszter végén rögzített állapot szerint adja ki az iskola.  Az 5-12. évfolyamszinteken lévő gyereknek januárban módjuk és lehetőségük van a portfóliójuk frissítésére, ha a féléves értékelés és az osztályzatokra váltás eredménye számukra iskolaváltás, továbbtanulás vagy egyéb okból fontos. Ilyenkor januárban alkalmuk van felkészülni \aref{sec:evfolyamok_osztalyzatok}.~fejezetben részletezett osztályzatokra váltás folyamatra.

Év végén, június hónap fele marad az évvégi zárások és igény szerint az osztályzatokra váltásra való felkészülésre, ami a portfólió frissítését, bővítését, kiegészítését jelenti.
\section{Saját tanulási célok}
\label{sec:tanulasi_celok}

Minden gyerek megfogalmazza és háromhavonta újrafogalmazza a \emph{saját tanulási céljait}: eredményeket, amelyeket el akar érni, képességeket, amelyeket fejleszteni akar, szokásokat, amelyeket ki akar alakítani. A saját célok elfogadásakor a gyerek és a mentora a szülőkkel együtt \emph{tanulási szerződést} köt.

Csak olyan célok kerülhetnek a saját célok közé, amelyek minden érintettnek biztonságosak, és amelyek összhangban vannak a tantárgyi fejlesztési célokkal és tanulási eredményekkel. A szerződésben rögzíthetőek tanulási eredményekre vonatkozó megállapodások, tantárgyi évfolyamszintekre vonatkozó elvárások (pl. ,,\emph{haladjon egy évfolyamszintet egy év alatt}'' vagy ,,\emph{készüljön fel emelt szintű érettségire}''), és a tantárgyi rendszeren kívüli célok és feladatok.

%Fontos megkötés, hogy a saját tanulási célok legalább a felének \ifkerettanterv
%      \aref{sec:tantargyi_tanulasi_eredmenyek}. fejezetben
%\else
%      a kerettanterv Tantárgyi tanulási eredmények fejezetében
%\fi felsorolt tanulási eredmények elérésére kell vonatkoznia. A másik fele szabadon alakítható.

Háromhavonta a tanulásszervezők és a gyerekek megállnak, reflektálnak az elmúlt időszakra, és a tapasztalatok, valamint az elért célok ismeretében és az új célok figyelembevételével újratervezik, újraszervezik a foglalkozások rendjét, tehát azt, hogy mikor és mit csinálnak majd a gyerekek az iskolában. A mindennapi tevékenység során tapasztalt élmények, alkotások, elvégzett feladatok, kitöltött vizsgák, tehát mindaz, ami a gyerekekkel történik, bekerül a portfóliójukba. Még az is, amit nem terveztek meg előre.

A gyerekeket a mentoruk segíti a saját célok kitűzésében, a különböző választásoknál, a portfólióépítésben, a reflektálásban. A tanulási célok kitűzése az önirányított tanulás fokozatos fejlődésével és az életkor előrehaladtával folyamatosan egyre önállóbb tevékenységgé válik. Tanulási útján, céljai kitűzésében a mentor kíséri végig a gyerekeket.

A Budapest School személyre szabott tanulásszervezésének jellegzetessége, hogy a gyerekek a saját céljuk irányába haladnak, az adott célhoz az adott kontextusban leghatékonyabb úton. Tehát mindenki rendelkezik saját célokkal, még akkor is, ha egy közösség tagjainak céljai a tantárgyi tanulási eredmények azonossága, vagy a hasonló érdeklődés miatt akár  80\% átfedést mutatnak.

A NAT műveltségi területeiben megfogalmazott követelmények teljesítése is célja a tanulásnak, a tanulás fő irányítója azonban más. Mi azt kérdezzük a gyerekektől, hogy \emph{ezenfelül} mi az ő személyes céljuk.
\ketoldaltkep{pics/3A_TSZ_SOMA.JPG}{pics/3B_TSZ_FLORA.JPG}
\section{A tanulási szerződés}

A tanulási szerződés az előbbiekben említett gyerek-mentor-szülő közötti megállapodás, ami rögzíti
\begin{enumerate}
      \item a gyerek, a mentor (iskola) és a szülő igényeit, elvárásait;\hfill\break
 ezek lehetnek: \emph{,,szeretném, ha a gyerekem naponta olvasna''} típusú folyamatra vonatkozó kérések, vagy erősebb \emph{,,változtatnod       kell a viselkedéseden, ha a közösségben akarsz maradni''} igények, határok megfogalmazása;

      \item a gyerek céljait a következő trimeszterre, vagy a tanév végéig;

      \item a gyerek, mentorok (iskola) és szülő vállalásait, amivel támogatják a cél elérését és a felek igényének teljesülését.

\end{enumerate}

A tanulási szerződésre jellemző:
\begin{itemize}
      \item A kitűzött célokat minél specifikusabban és mérhetőbben
        kell megfogalmazni. Javasolt az OKR  (Objectives and Key
        Results, azaz	Cél és Kulcs Eredmények) \citep{okr} vagy
        a\linebreak
        SMART (Specific, Measurable, Achievable, Relevant, Time-bound, azaz Specifikus,  Mérhető, Elérhető, Releváns és Időhöz kötött) \citep{wiki:smart} technika alkalmazása, hogy minél specifikusabb, teljesíthetőbb, tervezhetőbb és könnyen mérhető célokat tűzzenek ki.

      \item A kitűzött célokban való megállapodást követően, megállapodást kell kötni arról is, hogy ki és mit tesz azért, hogy a gyerek a célokat elérje.

      \item A mentor a teljes mikroiskolát (a többi tanárt, a közösséget) képviseli a megállapodás során.
\end{itemize}

A tanulási szerződést néha hívjuk \emph{megállapodásnak} is. A
megállapodás és szerződés szavakat az iskola szinonimának
tekinti. A \emph{learn\-ing con\-tract} az önirányított tanulást
hangsúlyozó felnőttképzéssel foglalkozó irodalomban bevett
szakkifejezés már a 80-as évektől (Knowles,\linebreak
\citeyear{Malcolm77}). Ennek a magyar nyelvben inkább a szerződés felel meg. Egy másik szakterületen, a pszichoterápiás munkában a terápiás szerződések megkötésekor a közös munka kereteinek kialakítását és fenntarthatóságát hangsúlyozzák \citep{pszichoterapia}. Erre is utalunk a tanulási szerződés elnevezéssel. Van, amikor a \emph{hármas szerződés} kifejezést használjuk, hangsúlyozva, hogy mind a három szereplőnek elfogadhatónak kell tartania a szerződés tartalmát.

\ifkerettanterv
  \section[A tanulásszervező, a mentor és a
modulvezető]{Különböző tanári szerepek: a tanulásszervező, a mentor és a
    modulvezető}
\label{sec:tanarok}

A Budapest Schoolban a gyerekek azokat a felnőtteket tekintik tanáruknak, akik
minőségi időt töltenek velük, és segítik, támogatják vagy vezetik őket a
tanulásukban. Több szerepre bontjuk a tanár fogalmát: a gyerek egy (és csak
egy) felnőtthöz különösen kapcsolódik, a \emph{mentortanárához}, aki rá
különösen figyel. Ezenkívül a gyerek tudja, hogy a mikroiskola mindennapjait
egy tanárcsapat, a \emph{tanulásszervezők} határozzák meg, ők vezetik az
iskolát.  A foglalkozásokon megjelenhetnek további tanárok, a \emph{modulvezetők},
akik egy adott foglalkozást, szakkört, órát tartanak.

Szervezetileg minden mikroiskolának van egy állandó \emph{tanárcsapata}, a
tanulásszervezők. A tanulásszervezők legalább egy tanévre elköteleződnek, szemben a
modulvezetőkkel, akik lehet, hogy csak egy pár hetes projekt keretében vesznek részt a
munkában.
A tanulásszervezők általában mentorok is, de nem minden esetben. Nem lehet
mentor az, aki a gyerek mikroiskolájában nem tanulásszervező, mert nem lenne
rálátása a mikroiskola történéseire. Egy tanulásszervező lehet több
mikroiskolában is ebben a szerepben, és így mentor is lehet több mikroiskolában.

\paragraph{Mentor}
Minden gyereknek van egy \emph{mentora}, aki a saját céljainak
megfogalmazásában és
a fejlődése követésében segíti. Minden mentorhoz több gyerek tartozik, de nem
több, mint 12. A mentor együtt dolgozik a mikroiskola többi tanulásszervezőjével, a
szülőkkel és az általa mentorált gyerekekkel. A mentor segít az általa
mentorált gyereknek, hogy a tantárgyi fejlesztési célok és
a
saját magának megfogalmazott saját célok között megtalálja  az egyensúlyt, és segít megalkotni a
gyerek \emph{saját
  tanulási tervét}.

A mentor a kapocs a Budapest School, a szülő és a gyerek között.

\begin{itemize}
  \item Képviseli a Budapest Schoolt, a mikroiskola közösségét.
        \begin{itemize}
          \item Ismeri a Budapest Schoolt, a lehetőségeket, a tanulásszervezés
                folyamatait.
          \item Együtt tanul más Budapest School-mentorokkal, együtt dolgozik a
                tanártársaival.
        \end{itemize}

  \item Ismeri, segíti, képviseli a gyereket.
        \begin{itemize}
          \item  Tudja, hol és merre tart a mentoráltja, ismeri a képességeit,
                körülményeit, szándékait, vágyait.
          \item    Segít a saját célok elérésében, felügyeli a haladást.
          \item    Megerősíti a mentoráltjai pszichológiai biztonságérzetét.
          \item   Visszajelzéseket ad a mentoráltjainak.
          \item    Segít abban, hogy az elért célok a portfólióba kerüljenek.
          \item    Összeveti a portfólió tartalmát a tantárgyak fejlesztési
                céljaival.
        \end{itemize}

  \item Együtt dolgozik, gondolkozik a szülőkkel, képviseli igényüket a
        közösség felé.
        \begin{itemize}
          \item Erős partneri kapcsolatot épít ki a szülőkkel, információt oszt meg
                velük.
          \item Segít a gyerekekkel közös célokat állítani.
          \item A szülő számára a mentor az elsődleges kapcsolattartó a különféle
                iskolai ügyekkel kapcsolatban.
        \end{itemize}

\end{itemize}

A mentor egyszerre felelős a mentorált gyerek előrehaladásának segítéséért,
és
közös felelőssége van a mentortársakkal, hogy az iskolában a lehető legtöbbet
tanuljanak a gyerekek. A mentor folyamatosan figyelemmel követi az egyéni
tanulási tervben megfogalmazottakat, és ezzel kapcsolatos visszajelzést ad a
mentoráltnak és a szülőnek.

\paragraph{Tanulásszervező}
Csoportban dolgozó, iskolaszervező, strukturáló tanár. Egy mikroiskola
állandó tanári
csapatát 2--7 tanulásszervező alkotja, akik egyedileg meghatározott szerepek
mentén a mikroiskola mindennapjainak működtetéséért felelnek. Minden mentor
tanulásszervező is. A tanulásszervezők tarthatnak
modulokat, sőt kívánatos is, hogy dolgozzanak a gyerekekkel, ne csak
szervezzék az életüket.
Ők rendelik meg a külső modulvezetőktől a munkát, ilyen értelemben a
tanulási utak projektmenedzserei.

\paragraph{Modulvezetők}

Bárki lehet modulvezető, aki képes akár egy egyetlen alkalommal történő, vagy
éppen
egy egész trimeszteren át tartó tanulási, alkotási folyamatot vezetni. Ők
általában
az adott tudományos, művészeti, nyelvi vagy bármilyen más terület szakértői.

A modulokat a tanulásszervezők is vezethetik, de külsős, egyedi megbízással
dolgozó szakemberek is megjelennek modulvezetőként. Modulvezető lehet bárki,
akiről az őt megbízó tanárcsapat tudja, hogy képes gyerekek folyamatos
fejlődését és egy tanulási cél felé való haladását segíteni. A moduláris
tanmenettel \aref{sec:modulok}. fejezet foglalkozik.
\fi
\section{Moduláris tanmenet és a tanulási eredmények}

\subsection{Modulok -- a tanulásszervezés alapegységei}
\label{sec:modulok}

A \emph{modulok} a tanulásszervezés \emph{alapegységei}: olyan foglalkozások megtervezett sorozata, amelyek során egy meghatározott időn belül a gyerekek valamely képességüket fejlesztik, valamilyen ismeretet elsajátítanak, vagy valamilyen produktumot létrehoznak. A modulok célja sokféle lehet, de kötelező elvárás, hogy a résztvevők a portfóliójukba bejegyzésre érdemes eredményt hozzanak létre, vagyis hogy legyen egyértelmű célja.

A mindennapi tanulás a modulok elvégzésén keresztül történik, ezzel biztosítva, hogy rugalmas keretek között, pontosan megfogalmazott célok mentén, a gyerekek számára érthető, átlátható és sajátnak megélt tartalommal történjen a tanulás.

A tanulási modulokat, vagyis a tanulás tartalmának és formájának alapegységét a tanulásszervezők három kötelező összetevőből állítják össze:

\begin{enumerate}
      \item
            a kerettanterv tantárgyainak tartalmából,
      \item
            a gyerekek, tanárok érdeklődéséből, aktuális tudásából,
      \item
            és a környezetük és a világ aktuális kihívásaiból.
\end{enumerate}

A három komponensből a legelső a legstatikusabb, hiszen a kerettanterv -- összhangban a NAT-tal -- meghatározza a tantárgyakat és azok tartalmát, valamint azt, hogy milyen lehetséges eredmények elérését várjuk az ezekben való fejlődéstől. Az egyes modulokban ezek személyre, illetve a csoport igényeire szabhatóak, hiszen az elérhető eredményeket különféle gyakorlati és elméleti tanulási módszerekkel el lehet érni.

A gyerekek és tanárok érdeklődése -- ami a sajátként megélt cél és a minél nagyobb fokú bevonódás alapfeltétele -- alakítja ki a modulok témáját, a projekteket, és a gyerekek egyéni tanulási idejét is meghatározhatja.

Mindemellett a kerettanterv szándéka, hogy a tanárok, gyerekek reagáljanak a környezetükre, a világ aktuális kihívásaira, kérdéseire. A kerettanterv meghatározza például, hogy a gyerek ,,\emph{táblázatkezelővel feladatot old meg}''. Az azonban, hogy a gyerekek milyen táblázatokat szerkesztenek szívesen, csak a modulok összeállításakor és a modulok elvégzése során derül ki. Nagyon hasonló táblázatkezelési képességeket lehet fejleszteni, ha valaki az önvezető autóktól várt csökkenő baleseti halálozási arányról, vagy ha a vegánok számának és a GDP-növekedés alakulásának arányáról készít táblázatot.

A moduláris rendszer fő célja, hogy egyszerre képes legyen alkalmazkodni a menet közben felmerülő tanulási igényekhez, adjon átlátható struktúrát a tanulásnak, és hogy a mikroiskola minél rugalmasabban tudja támogatni\break a tanulást úgy, hogy a saját, a közösségi és a társadalmi célok harmóniába kerülhessenek.

Ez is mutatja, hogy bár közösek a kereteink, végtelen az elképzelhető modulok (a tanulási utak építőkövei, és így a különböző tanulási utak) száma. Ezért tartja fontosabbnak a kerettanterv annak meghatározását, hogy hogyan kell a modulokat létrehozni, mint azt, hogy a modulokat tételesen felsorolja.

Modulok során a gyerekek tudnak

\begin{itemize}
      \item produktum létrehozására szerveződő projektben részt venni;

      \item felfedezni, feltalálni, kutatni, vizsgálni, azaz kérdésekre választ keresni;

      \item egy jelenséget több nézőpontból megismerni;

      \item valamely képességüket, készségüket fejleszteni;

      \item adott vizsgára gyakorló feladatokkal felkészülni;

      \item közösségi programokban részt venni;

      \item az önismeretükkel, a tudatosságukkal, a testi-lelki jóllétükkel fog\-lal\-kozni.
\end{itemize}

\subsubsection{A modulok meghirdetése}
\label{sec:modulok_meghirdetese}
A modulok kiválasztása, felkínálása a tanulásszervezők feladata, hiszen ők figyelnek és reagálnak a gyerekek, szülők céljaira és igényeire. A meghirdetett modulokból áll össze a tanulás trimeszterenkénti tanulási rendje.

A tanulásszervezők az egyes modulok tematikáját, azok hosszát és feladatát a gyerekek tanulási céljainak megismerését követően és a kerettantervben meghatározott tantárgyi tanulási eredményeket figyelembe véve határozzák meg.

A nem kötelező modulokba való csatlakozásról a mentor, a szülő és a gyerek közösen dönt, mindig szem előtt tartva, hogy folyamatos előrelépés legyen a már elért egyéni és tantárgyi eredményekben is. Egy modul megkezdésének lehet feltétele egy korábbi modul elvégzése, a gyerek képességszintje, a jelentkezők száma, és lehet egyedüli feltétele a gyerekek érdeklődése.

Egy modulvezető különféle tematikájú modulokat tarthat függően attól, hogy a saját célok, a tantárgyi eredmények mit kívánnak, és a tanulásszervezők, valamint a modulvezetők kapacitása mit enged.

Amikor egy gyerek moduljai befejeződnek, és újat vesz fel, a tanulásszervező feladata a gyereket segíteni abban, hogy az érdeklődési körének, tanulási céljainak, és a soron következő, még el nem ért tantárgyi eredményekben való fejlődéshez megfelelő modulok közül választhasson.

A tanulásszervezők feladata a tantárgyi eredményelvárások nyomon követése is. A modulok kidolgozásához és azok megtartásához külsős szakembereket is meghívhatnak, azonban ilyenkor is a tanulásszervezők felelnek azért, hogy a modulokkal elérni kívánt tanulási célok teljesüljenek.

\subsubsection{A modulok formátuma}

Egy-egy modul hossza és a modulhoz kapcsolódó foglalkozások száma és gyakorisága változó: egy alkalomtól legfeljebb egy teljesen trimeszteren keresztül tarthat. A modul végén a tanulásszervező és a gyerek(ek) a modult lezárják, értékelik és az elért eredményeket rögzítik a (tanulási) portfólióban. Egy modul folytatásaként a következő trimeszterben új modult lehet meghirdetni.

A modulok nemcsak témájukban, céljaikban, időtartamukban, hanem módszertanukban, folyamataikban is különbözhetnek: bizonyos modulokban a felfedeztető (inquiry based) módszer, másokban az ismétlő (repetitív) gyakorlás a célravezető. Így mindig a modul céljához, a tanárok és a gyerekek képességeihez és igényeihez választható a legjobb módszer. Modulonként változhat, hogy a folyamatot a gyerekek vagy a tanárok befolyásolják-e, és milyen mértékben. Két példa az eltérésre:

\begin{enumerate}
      \item Egy digitális kézműves modul célja, hogy építsünk valamit, ami programozható. Annak kitalálása, hogy mit és hogyan építünk, a gyerekek feladata. Itt a modul vezetője csak támogatja a tanulás folyamatát, azaz \emph{facilitál}.

      \item Egy „\emph{A vizuális kommunikáció fejlődése a XX. század második felében}'' modul esetén a tanár előre felépíti a tanmenetet, például hogy mely alkotók munkásságát, alkotásokat fogja bemutatni, és ezeket a gyerekekkel sorban végigveszi. Ilyenkor is bővülhet azonban a tematika a gyerekek érdeklődése, felvetései mentén.

\end{enumerate}

\subsubsection{A modulok helyszíne}

A tanulás az egyes mikroiskolák helyszínén, egy másik Budapest School mikroiskolában, a tanár által kiválasztott külső helyszínen, vagy akár online, virtuális térben történik. A tanulásra úgy tekintünk, mint az élethez szorosan kapcsolódó holisztikus fejlődési igényre, melynek jegyében az elsődleges szocializációs tértől és formától, a szülői, családi környezettől sem akarjuk a tanulást leválasztani. Az élethosszig tartó tanulás jegyében a tanulás tere az iskolai időszak után és az iskola terein kívül is folytatódik.

A gyerekek több ok miatt is tanulnak az iskolán kívül:

\begin{enumerate}
      \item Modulok vagy modulok foglalkozásai szervezhetők külső helyszínekre, úgymint múzeumokba, erdei iskolákba, parkokba, vállalatokhoz, vagy tölthetik az idejüket „kint a társadalomban''.

      \item Amennyiben ez saját céljuk elérését nem veszélyezteti, és a folyamatos fejlődés biztosított, a mentoruk tudomásával a gyerekek az önirányított tanulás elvére figyelemmel a mikroiskolán kívüli egyéb helyszínen is elvégezhetnek egy modult.
\end{enumerate}

A modul lezárásaként a gyerekek és modulvezetők visszajelzést adnak egymásnak, aminek része, hogy megosztják saját élményeiket, reflektálnak a közös időre, összegyűjtik és értékelik az elért eredményt, és kitérnek az esetleges fejlődési lehetőségekre.

\subsection{Tanulási eredmények -- a formális tanulás alapegységei}
\label{sec:tanulasi_eredmenyek}
A kerettanterv három interdiszciplináris tantárgyat jelöl meg, azok témaköreit, tartalmát és követelményeit \emph{tanulási eredmények} listájaként adja meg, ezzel igazodva az Nkt.~5.~§ (5) pontjához. Tanulási eredmény (learning out\-come) lehet a kerettanterv szellemében minden olyan tudás, képesség, kompetencia, attitűd, amit a gyerek egy tanulási folyamat során elsajátított és/vagy ezt demonstrálni tudja. Az eredmény eléréséhez vezető út a modulokon keresztül történik, és a tanulás folyamata történhet az iskolában vagy azon kívül, lehet formális, non-formális vagy informális.

A tanulási eredmények több funkciót látnak el a kerettantervben.

\begin{itemize}

      \item A kerettanterv évfolyamonként meghatározza az adott tantárgy teljesítéséhez elérendő tanulási eredményeket. Egy gyerek akkor  léphet egy tantárgyból évfolyamszintet, ha a tantárgyhoz tartozó követelményeket teljesítette.

      \item A tanulási eredmények a modulok (és így a mindennapokban szervezett foglalkozások, órák stb.) építőelemei. Egy-egy modul célját a  tanulásszervezők az elérendő tanulási eredmények	összeválogatásával és saját célokkal, érdeklődéssel való	kiegészítésével adják meg, figyelembe véve az életkori  sajátosságok, az egymásra épülés és az átjárhatóság  követelményeit.
      \item A tanulási eredmények megfeleltethetőek a miniszter által kiadott kerettantervek tantárgyai (és így a kötelező érettségi tárgyai )  és a NAT műveltségi területeivel, ami biztosítja, hogy a Budapest  School tanulója más rendszerben működő iskolába is illeszkedik.  Vagyis a tanulási eredmények a Budapest School saját interdiszciplináris tantárgyi elvárásain túl, az azokon belüli halmazt képző diszciplináris bontásban is követhetőek, így az	elért eredmények alapján mind a Budapest School tantárgyi  struktúrájával, mind (például egy esetleges iskolaváltás esetén) a	miniszter által kiadott kerettantervek tantárgystruktúrájával megfeleltethetőek.
\end{itemize}

\subsection{Modulok és tanulási eredmények}
\label{sec:modulok_es_tanulasi_eredmenyek}
A gyerekek egyik feladata az iskolában, hogy tanulási eredményeket érjenek el. Ezt megtehetik a modulok elvégzésével, vagy más tanulási helyzetekben. A tanulási eredményeket a portfólióban rögzítik. A mentor feladata, hogy folyamatosan kövesse, hogy megfelelő haladás történik-e a portfólióban a tanulási eredmények és a saját célok tekintetében. Az évfolyamszintlépés a portfólióban összegyűlt tanulási eredmények alapján történhet meg.

A modul kecsegtet a gyerekek haladásához releváns tanulási eredményekkel, a gyerekek által meghatározott saját célokkal és olyan kimenettel, amely a portfólióban rögzíthető, legyen az egy alkotás, az elért fizikai vagy szellemi eredmény dokumentációja, vagy egy értékelő visszajelzés. A modulok tehát tartalmaznak tanulási eredményeket, az önálló gondolkodás, szabad alkotás lehetőségét, és teret engednek az alkotásra, létrehozásra.

\paragraph{A modulok különféle tanulási eredmények elérését teszik elérhetővé}

Modulok tervezésekor és összeállításakor a tanulásszervezők a modulvezetővel közösen határozzák meg a modul céljait, de azok meghirdetéséért mindig a tanulásszervezők felelnek. A célok között fel kell sorolni, hogy milyen tanulási eredmények elérését várhatják el a gyerekek a modulon való részvételtől.

Például a 6--8 éves gyerekek számára megtervezett ,,\emph{3d nyomtató használata}'' modul során azon kívül, hogy megismerik a 3d nyomtatás folyamatát, a modul célja, hogy a gyerekek számára elérhetővé tegye a ,,\emph{kockát, téglatestet, gömböt felismeri, és képes létrehozni egyszerű módszerekkel. Ismeri ezeknek a testeknek a jellemzőit}'' (STEM tantárgy, Matematika tématerület, 3--4. évfolyam) tanulási eredményt is.

Lehetőség van egy modul esetében több tantárgyból való tanulási eredmény kiválasztására, ezzel biztosítva az interdiszciplinaritást, valamint a Budapest School tantárgyi fejlesztési céljaihoz való integrált kapcsolódást.

A tanulási eredmények egy időbeni egymásra épülést feltételeznek, melyben azonban van lehetőség előre- és hátrafele is lépni. Előre, amennyiben a modul meghirdetésekor az arra jelentkező gyerekcsoportnál a megfelelő előkészítés megtörtént, hátra, amennyiben ezt ismétlés/felzárkóztatás jelleggel szükségesnek ítéli a mentor vagy a modult szervező, vezető. Vagyis akkor foglalkozzon egy gyerek a 10~000-es számkörrel, ha a 100-as számkört már begyakorolta. Az egymásra épülésért a modult meghirdető tanulásszervező felel. A példát folytatva a 3d nyomtató használata modul lehetővé teszi, hogy a gyerek elérje a következő eredményeket is: \emph{,,Ismeri a számítógép
      részeinek és perifériáinak funkcióit, azokat önállóan használja.''}
(Harmónia, Informatika, 5--6. évfolyam), és  \emph{,,Használati utasításokat
      értő módon olvas és tart be.''} (Harmónia, Életvitel, 3--4.)

\paragraph{Új tanulási eredmények}

A gyerekek olyan tanulási eredményt is elérhetnek, ami a modulok céljai között eredetileg nem volt megadva, mert

\begin{itemize}
      \item lehetőségük van egyénileg is tanulni;

      \item tanulási eredményekkel járnak a projektek, az iskolai lét, a közösségi élet és még számos informális és non-formális tanulási helyzet;

      \item egy modul során is alakulhatnak előre nem tervezett helyzetek, amik hozzásegíthetik a gyerekeket tanulási eredmények eléréséhez.
\end{itemize}

Az újonnan létrejövő tanulási eredmények is bekerülnek a portfólióba.

\paragraph{Tanulási eredmények dokumentációja}

Minden modul dokumentálásra kerül, hogy annak célja, elért eredményei nyilvánosak legyenek a Budapest School valamennyi mikroiskolája számára, és ha szükséges, újra meg lehessen hirdetni. A tanulási eredmények egy, a modulhoz kapcsolódó terv-tény összehasonlítás alapján kerülnek meghatározásra. Az elért eredmények újra elérhetőek, amennyiben a folyamatos fejlődés biztosítva van.

\paragraph{Egységes modulok egyedi alkalmazása}

Egy modul elvégzésével egy-egy gyerek más tanulási eredményt is elérhet.

\begin{itemize}
      \item
            Működhet a differenciálás, tehát nem minden gyerek ugyanazt és\break ugyanúgy csinálja a foglalkozásokon. Egy modulban tud együtt	tanulni az a gyerek, aki még ,,\emph{Ismeri az írott és nyomtatott  betűket''} eredményért dolgozik, és az, aki ,,\emph{Jelöli helyesen a j	hangot 30--40 begyakorolt szóban''.}
      \item
            A modulnak része lehet testre szabható sáv. Például egy tudományos kísérletező modulban néhány gyerek a rövid távú memória és a	fáradtság kapcsolatáról kutat, a másik csoport az esőzés és a	közlekedési dugók kialakulása közti kapcsolatot vizsgálja. Minden  gyerek elérheti a ,,\emph{valós folyamatokat képes elemezni a folyamathoz tartozó függvény grafikonja alapján}''  (forrás, STEM) eredményt, de a ,,\emph{környezettudatos közlekedésszemlélet}'' (forrás, Harmónia)	eredményt is elérheti.
      \item
            Egy-egy gyerek saját tanulási célja érdekében extra lépéseket tehet, és olyan eredményeket is el tud érni, amit mások nem.	Például egy modul végén önálló prezentációt, saját kutatási  tervet, vagy egy kész működő modellt alkothat.
\end{itemize}

\subsubsection{Kötelező tanulási eredmények}
\label{sec:kotelezo_tanulasi_eredmenyek}
A kerettanterv kötelező tanulási eredményként definiálja mindazokat az eredményeket, melyek a kötelező érettségi tárgyak teljesítéséhez szükségesek. Ezeket minden mikroiskola elérhetővé kell hogy tegye a gyerekek számára a modulok választékában.

Ezek az 1--4. évfolyamszinteken a miniszter által kiadott kerettantervek \emph{Magyar nyelv és irodalom}, \emph{Matematika}, \emph{Idegen nyelv} tantárgyakból származó tanulási eredmények, és 5.~évfolyamszinttől kiegészülnek a \emph{történelem, társadalmi és állampolgári ismeretek} tantárgyak alapján létrehozott tanulási eredményekkel. További kötelező tanulási eredményként jelennek meg 9.~évfolyamtól a választott érettségi tantárgyhoz kapcsolódó eredmények. Ezek a tanulási eredmények megtalálhatók a kerettanterv három tantárgyának elérhető eredményei között.

A mikroiskolában meghirdetett moduloknak a kötelező tanulási eredmények 80\%-át le kell fednie.

\paragraph{Tanulási eredmények kiegyenlítettsége}

Szintén fontos kötöttség, hogy a modulok meghirdetésénél a kerettanterv három tantárgyából egyenlő súllyal (plusz/minusz 20\%) legyenek elérhetőek a tanulási eredmények. Emellett a kerettanterv minden tématerületéről (vagyis a tantárgyak eredményeit alkotó diszciplinákból) legalább 20\% tanulási eredményt kell választani, így biztosítva, hogy a NAT minden műveltségterülete megjelenjen a tanárok által lefedett témák között.

\paragraph{Kötelező modulok}
A kerettanterv és a pedagógiai program is előírhat kötelező modulokat a mikroiskolák számára. Ilyenek például a 11.~évfolyamszinten belépő érettségire felkészítő modulok (ld.~\ref{sec:erettsegi}.~fejezet), a minden mikroiskolára egységes pedagógiai program tetszőleges kötelező modult írhat elő. Így lehet biztosítani a kötelező tartalmi elemek és foglalkozás -- úgymint testnevelés, elsősegélynyújtás -- elérhetőségét.

\subsubsection{Monitorozás}

Kötelező elérni az eredményeket? Nem tudunk hatalmi szóval tanulásra bírni gyereket, mert lehet, hogy annyira nem akarja, vagy nincs meg hozzá a képessége. A kerettanterv a tanároknak ad keretet. Azonban a fenntartó által üzemeltetett rendszerrel az iskola  monitorozza a haladást, és ha valaki a kötelező tanulási elemekkel nem halad, akkor az iskola erre felhívja a figyelmét. Mivel a többség haladni fog, ezért előre tudja az iskola jelezni, hogy le fog szakadni a többiektől, és túl nagy lesz az évfolyamszint-különbség közöttük. Ezekben az esetekben a mentortanárnak, a gyereknek és a szülőnek reagálnia kell a helyzetre. A fenntartó által működtetett monitorozó és minőségfejlesztő rendszerről \aref{sec:minosegbiztositas} fejezet ír részletesen.

\section{Visszajelzés, értékelés}
\label{sec:ertekeles}
Ahhoz, hogy hatékony legyen a tanulás, fejlődés, fontos, hogy a gyerekek, tanárok és szülők is tudják, hogy
\begin{enumerate}
      \item hol tart most egy gyerek, mit tud most,
      \item hova akar vagy kell eljutni, azaz, mi a célja,
      \item mi kell ahhoz, hogy elérje a célját.
\end{enumerate}
Ezek mellett mindenkinek hinnie kell abban, hogy odafigyeléssel, gyakorlással a gyerek meg tud tanulni egy konkrét dolgot. Fontos, hogy magas legyen a gyerekek énhatékonysága,  erős legyen az önbizalmuk, és nem szabad félniük a hibázástól, a nem-tudástól, mert a tanulás első lépése, hogy elfogadjuk, hogy valamit nem tudunk. Azaz fontos, hogy fejlődésfókuszú gondolkodásuk (growth mindset) \citep{growthmindset} legyen, azaz
\begin{enumerate}
      \setcounter{enumi}{3}
      \item hinniük kell, hogy el tudják érni a céljukat.
\end{enumerate}

Egy visszajelzés, értékelés akkor jó és hasznos, azaz hatékony, ha ebben a négy dologban segít. Mai tudásunk szerint ehhez:
\begin{itemize}
      \item Rendszeresen visszajelzést kell kapniuk és adniuk.
      \item A tanulási céloknak és visszajelzéseknek minél specifikusabbaknak kell lenniük (azaz például ne a 8. oszályos \emph{matematikatudást} értékeljük, hanem hogy mennyire képes valaki \emph{fagráfokat használni feladatmegoldások során}\footnote{Ez a konkrét példa a matematika tantárgy egyik tanulási eredménye.}).
      \item A \emph{,,hol tartok most''} diagnózisnak mindig cselekvésre, viselkedésre, aktív tevékenységre kell vonatkoznia. Ne az legyen a visszajelzés, hogy \emph{,,ügyes vagy egyenletekből''}, hanem \emph{,,gyorsan és       pontosan oldottad meg a 4 egyenletet''}. A legjobb, amikor a visszajelzés konkrét megfigyelésen alapul, és tudni, hogy mikor, hol történt az eset: \emph{,,amikor társaiddal Minecraftban házat építettél, akkor       pontosan kiszámoltad a ház területét''.}
      \item Ha a cél nem a mások legyőzése, akkor a visszajelzés se tartalmazzon olyan állítást, ami másokhoz hasonlít (így kerüljük a \emph{tehetség} szót is, aminek bevett definíciója szerint az átlagnál jobb képesség). A másokhoz való szint felmérése akkor (és csak akkor) fontos, amikor a cél egy versenyszituációban jó eredményt elérni.

      \item A gyerek legyen részese a visszajelzésnek. Értse, tudja, hogy miért kapta azt a visszajelzést, a legjobb, ha -- amikor ezt a képességei engedik -- önmaga képes elvégezni a visszajelzést, vagy annak egy részét.
      \item A visszajelzésnek transzparensen hatással kell lennie a tanulásszervezésre. Legyen része a folyamatnak, és a gyerek, tanár és a szülő is értse, hogy a visszajelzés alapján mit és hogyan csinálunk másképp.
\end{itemize}

\paragraph{Többszintű visszajelzés} A Budapest School iskolákban a gyerekek
többféle visszajelzést kapnak. \begin{enumerate}
      \item Minden modul elvégzése után a modul céljai, témája, fókusza alapján a modulvezetők visszajelzést adnak a tanulásról, eredményekről, viselkedésről.
      \item Trimeszterenként a mentorok visszajelzést adnak arról, hogy a gyerek általában hogyan haladt a tanulási célok felé.
      \item Ennek része, hogy a tantárgyi tanulási eredmények alapján hogyan haladt a gyerek a tantárgyak évfolyamszinthez tartozó követelményeinek teljesítésében. Az évfolyamok, mint elérhető szintek Budapest School-értelmezését \aref{sec:evfolyamok}. fejezet tárgyalja.
      \item A mentorok irányításával a gyerekek visszajelzést kapnak arról, hogyan működnek a közösségben.
\end{enumerate}

\paragraph{Érdemjegyek, osztályzatok helyett értékelő táblázatok} A Budapest
School visszajelzéseinek sokkal részletesebbeknek kell lenniük, mint azt a tantárgyi érdemjegyek és osztályzatok lehetővé teszik, ezért azok helyett a kerettanterv értékelő táblázatokat (angolul rubric) alkalmaz. Az értékelő táblázatban szerepelnek az értékelés szempontjai és szempontonkénti szintek, rövid leírásokkal. Ezek alapján a gyerekek maguk is láthatják, hogy hol tartanak, hogyan javíthatnak még a munkájukon. A táblázatok formája minden visszajelzés esetén (értsd modulonként, célonként) változtatható.
\section{Portfólió}
\label{sec:portfolio}
A modulok eredményeiből, a produktumokból és visszajelzésekből a gyerek és a
mentor portfóliót
állít össze, hogy a tanulás mintázatait észlelhesse, és a
tanárok tudatosabban tudják
a gyereket segíteni a céljai kitalálásában és elérésében. A portfólió a gyerek
céljainak nyomon követését szolgálja, és egyúttal a szülők felé történő
visszajelzés eszköze
is. Minden gyerek portfóliója folyamatosan épül: az tartalmazza az általa
elvégzett feladatokat, projekteket vagy azok dokumentációját, alkotásait,
eredményeit, az esetleges vizsgák eredményeit és a társaitól, tanáraitól kapott
visszajelzéseket. A \emph{portfólió célja}, hogy minden információ meglegyen
ahhoz,
hogy

\begin{itemize}
      \item a gyerek és mentora fel tudja mérni, hogy sikerült-e a kitűzött
            célokat
            elérni, illetve mire van szüksége még a gyereknek új célok
            eléréséhez;

      \item a szülő folyamatosan rálásson a gyereke tanulási útjára;

      \item megítélhető legyen, hogy a tantárgyi követelményekhez
            képest
            hol
            tart a gyerek;

      \item a gyerek a portfólió megtekintésével visszaemlékezhessen a
            tanultakra,
            ismételhessen, tudása elmélyülhessen;

      \item eredményei alapján bizonyítványt lehessen kiállítani.

\end{itemize}

A portfólió folyamatosan frissül, a mindennapi, formális, non-formális és
informális tanulási helyzetek
bármikor adhatnak okot a portfólió frissítésére. Az iskola életében kiemelt
szerepe van a következő eseményeknek.

\begin{enumerate}
      \item Minden \emph{modul végeztével} a portfólióba kerül:

            \begin{enumerate}

                  \item  A képesség elsajátításának, tanulási eredmény
                        elérésének a ténye.
                        Nincs
                        félig elsajátított képesség, tehát már értékelni nem
                        kell. Ha a modul során a gyerek megtanult százas
                        számkörben alapműveleteket
                        végezni, 
                        akkor
                        annyi kerül be a portfólióba, hogy ,,\emph{Szóban és
                              írásban
                              összead, kivon, szoroz és oszt a százas
                              számkörben.}''. Amennyiben a
                        készséget a
                        gyerek és a
                        tanár megítélése alapján nem sikerült megfelelően
                        elsajátítani,
                        úgy a gyakorlás
                        ténye kerül be a portfólióba.
                  \item Az alkotás vagy a projektmunka eredménye, ha a modul
                        célja egy
                        alkotás
                        létrehozása volt.
                  \item A részvétel ténye, ha a jelenlét volt a modul célja
                        (például
                        kirándulás
                        az Országos Kéktúra útvonalán).

            \end{enumerate}
      \item Az elvégzett vizsgák, tudáspróbák, képességfelmérők, diagnózisok
            eredményeit érdemes rögzíteni.

      \item A \emph{kipakolás} célja, hogy a gyerekek a tanároknak, szülőknek
            és
            más érintetteknek bemutassák elvégzett
            munkájukat, azaz
            a portfólióváltozásukat. A kipakolásra való felkészülés
            tulajdonképpen
            a
            portfólió összeállítása, prezentálásra való felkészítése, a
            \emph{portfólió
                  frissítése}.

      \item Társas visszajelzés eredményeként minden gyerek kap visszajelzést a
            társaitól. Ilyenkor összegyűjtik, mit tett a gyerek, ami a többiek
            elismerését
            és háláját kivívta. Ez is releváns adatokkal szolgálhat a
            portfólióhoz.

      \item A gyerek saját értékelése, reflexiója arról, hogyan értékeli, amit
            elért, fontos eleme a portfóliónak.

      \item A tanárok adhatnak kompetenciatanúsítványokat. Ezek
            rövid,
            specifikus visszajelzések, amelyek mutatják, ha valamit a gyerek
            megcsinált,
            valamiben fejlődött.
\end{enumerate}

A mentorok segítenek a gyerekeknek a tanulás módját, folyamatát és eredményeit
bemutatni
portfólióban.

\paragraph{Formai követelmények}
A portfóliónak rendezettnek, hozzáférhetőnek,
elérhetőnek, visszakereshetőnek és könnyen bővíthetőnek kell lennie. Olyan
(technológiai)
megoldást kell a mikroiskoláknak választaniuk, ami alapján
a gyerek, tanár és a szülő \emph{naponta} tudja a portfóliót bővíteni, és akár
\emph{heti rendszerességgel} át tudják tekinteni időrendben, modulonként vagy
tantárgyanként a portfólió bővülését.

A portfólió formátumára nincs egységes megkötés. Minden mikroiskola maga
alakítja ki a gyerekek, tanárok és szülők számára legjobban működő rendszert.
Évfolyamszintlépéshez és osztályzatokra váltáshoz az iskola csak digitális
formában tárolt és a kijelölt tanárok számára online elérhetővé tett portfóliót
fogad el.

\chapter{Tanulási utak}
\section{Saját tanulási célok}
\label{sec:tanulasi_celok}

Minden gyerek megfogalmazza és háromhavonta újrafogalmazza a \emph{saját
      tanulási céljait}: eredményeket, amelyeket el akar érni, képességeket,
amelyeket fejleszteni akar, szokásokat, amelyeket ki akar alakítani. A saját
célok elfogadásakor a tanuló és a mentora a szülőkkel együtt \emph{tanulási
      szerződést} köt. Csak olyan célok kerülhetnek a saját célok közé, amelyek
minden érintettnek biztonságosak, és amelyek összhangban vannak a tantárgyi
fejlesztési célokkal és tanulási eredményekkel. A szerződésben rögzíthetőek
tanulási eredményekre
vontakozó megállapodások,
tantárgyi évfolyamszintekre vonatkozó elvárások (pl. ,,\emph{haladjon egy
      évfolyamszintet egy év alatt}'' vagy ,,\emph{készüljön fel emeltszintű
      érettségire}''), és a tantárgyi rendszeren kívüli célok és
feladatok.

Fontos megkötés, hogy a saját tanulási célok felének a
\ifkerettanterv
      \ref{sec:tantargyi_tanulasi_eredmenyek}. fejezetben
\else
      a kerettanterv Tantárgyi tanulási eredmények fejezetében
\fi
felsorolt tanulási eredmények elérésére kell vonatkoznia. A másik
fele szabadon alakítható. Az iskola tanulásszervezői a közösségek igényei,
céljai
alapján alakítják ki, hogy pontosan mi történik az iskolákban: mikor és milyen
modulok vannak, hogyan szervezik a mindennapokat a kerettanterv megkötéseit
figyelembe véve.

Háromhavonta a tanulásszervezők és a gyerekek megállnak, reflektálnak az elmúlt
időszakra, és a tapasztalatok, valamint az elért célok ismeretében és az új
célok figyelembevételével újratervezik, újraszervezik a foglalkozások rendjét,
tehát azt, hogy mikor és mit csinálnak majd a gyerekek az iskolában.
A mindennapi tevékenység során tapasztalt élmények, alkotások, elvégzett
feladatok, kitöltött vizsgák, tehát mindaz, ami a gyerekekkel történik, bekerül
a portfóliójukba. Még az is, amit nem terveztek előre.

A gyerekeket a mentoruk segíti a saját célok kitűzésében, a különböző
választásoknál, a portfólióépítésben, a reflektálásban. A tanulási célok
kitűzése az önirányított tanulás fokozatos fejlődésével és az életkor
előrehaladtával folyamatosan egyre önállóbb tevékenységgé válik. Tanulási
útján, céljai kitűzésében a mentor kíséri végig a gyerekeket.

A Budapest School személyre szabott tanulásszervezésének jellegzetessége, hogy
a tanulók a saját egyéni céljuk irányába haladnak, az adott célhoz az adott
kontextusban leghatékonyabb úton. Tehát mindenki rendelkezik saját célokkal,
még akkor is, ha egy közösség tagjainak céljai a tantárgyi tanulási eredmények
azonossága, vagy a hasonló érdeklődés miatt akár  80\% átfedést mutatnak.

A NAT műveltségi területeiben megfogalmazott követelmények teljesítése is célja
a tanulásnak, a tanulás fő irányítója azonban más. Mi azt kérdezzük a
tanulóktól, hogy \emph{ezen felül} mi az ő személyes céljuk.

\section{A tanulási szerződés}

A tanulási szerződés az előbbiekben említett gyerek-mentor-szülő közötti
megállapodás, ami rögzíti
\begin{enumerate}
      \item a gyerek, a mentor (iskola) és a szülő igényeit, elvárásait;

            ezek lehetnek: \emph{,,szeretném, ha a gyerekem naponta olvasna''}
            típusú
            folyamatra vonatkozó kérések, vagy erősebb \emph{,,változtatnod
                  kell a
                  viselkedéseden, ha a közösségben akarsz maradni''} igények,
            határok
            megfogalmazása;

      \item a gyerek céljait a következő trimeszterre, vagy a tanév végéig;

      \item a gyerek, mentorok (iskola) és szülő vállalásait, amivel támogatják
            a
            cél
            elérését és a felek igényének elérését.

\end{enumerate}

A tanulási szerződésre jellemző, hogy
\begin{itemize}
      \item A kitűzött célokat minél specifikusabban, mérhetőbben kell
            megfogalmazni.
            Javasolt az OKR,  (Objectives and Key Results, azaz  Cél és Kulcs
            Eredmények)
            \citep{okr} vagy a SMART (Specific, Measurable, Achievable,
            Relevant,
            Time-bound, azaz Specifikus,  Mérhető, Elérhető, Releváns és Időhöz
            kötött)
            \citep{wiki:smart} technika alkalmazása, hogy minél specifikusabb,
            teljesíthetőbb, tervezhetőbb és könnyen mérhető célokat tűzzenek
            ki.

      \item A kitűzött célokban való megállapodást követően, megállapodást
            kell
            kötni arról is, hogy ki és mit tesz azért, hogy a tanuló a célokat
            elérje.

      \item A mentor a teljes mikroiskolát (a többi tanárt, a közösséget)
            képviseli
            a
            megállapodás során.
\end{itemize}

A tanulási szerződést néha hívjuk \emph{megállapodásnak} is. A megállapodás és
szerződés szavakat ez a kerettanterv szinonímának tekinti. A \emph{learn\-ing
      con\-tract} az önirányított tanulást hangsúlyozó felnőttképzés
irodalomban
bevett szakkifejezés már a 80-as évektől \citep{Malcolm77}. Ennek a magyar
nyelvben inkább a szerződés felel meg. Egy másik szakterületen, a
pszichoterápiás munkában a terápiás szerződések megkötésekor a közös munka
kereteinek kialakítását és fenntarthatóságát hangsúlyozzák
\citep{pszichoterapia}. Erre is utalunk a tanulási szerződés elnevezéssel. Van,
amikor a \emph{hármas szerződés} kifejezést használjuk, hangsúlyozva, hogy mind
a három szereplőnek elfogadhatónak kell tartania a szerződés tartalmát.

Tekintettel arra, hogy a kiskorú mellett minden esetben a szülő (törvényes
képviselő) is aláírója a szerződésnek, a Ptk. 2:14.§ (1) és (3) bekezdése
szerint a jognyilatkozat érvényesen megtehető, a szerződés érvényesen létre
jöhet.

\section{Visszajelzés, értékelés}
\label{sec:ertekeles}
Ahhoz, hogy hatékony legyen a tanulás, fejlődés, érdemes a gyerekeknek,
tanároknak és szülőknek tudnia, hogy
\begin{enumerate}
      \item hol tart most egy gyerek, mit tud most,
      \item hova akar vagy kell eljutni, azaz, mi a célja,
      \item mi kell ahhoz, hogy elérje a célját.
\end{enumerate}
Ezek mellett mindenkinek hinnie kell abban, hogy odafigyeléssel, gyakorlással a
gyerek meg tud tanulni egy konkrét dolgot. Fontos, hogy magas legyen a gyerekek
énhatékonysága,  erős legyen az önbizalmuk, és nem szabad félniük a hibázástól,
a nem-tudástól,
mert a tanulás első lépése, hogy elfogadjuk, hogy valamit nem tudunk. Azaz
fontos, hogy fejlődésfókuszú gondolkodásuk (growth mindset)
\citep{growthmindset} legyen, azaz
\begin{enumerate}
      \setcounter{enumi}{3}
      \item hinniük kell, hogy el tudják érni a céljukat.
\end{enumerate}

Egy visszajelzés, értékelés akkor jó és hasznos, azaz hatékony, ha ebben a négy
dologban segít. Mai tudásunk szerint ehhez:
\begin{itemize}
      \item Rendszeresen visszajelzést kell kapnunk és adnunk.
      \item A tanulási céloknak és visszajelzéseknek minél specifikusabbaknak
            kell
            lenniük (azaz például ne a 8. oszályos \emph{matematikatudást}
            értékeljük,
            hanem például, hogy mennyire képes valaki \emph{fagráfokat
                  használ
                  feladatmegoldások során}\footnote{Ez a konkrét példa a STEM
                  tantárgy
                  egyik
                  tanulási eredmények}).
      \item A \emph{,,hol tartok most''} diagnózisnak mindig cselekvésre,
            viselkedésre, aktív tevékenységre kell vonatkoznia. Ne az legyen a
            visszajelzés, hogy \emph{,,ügyes vagy egyenletekből"}, hanem
            \emph{,,gyorsan és
                  pontosan oldottad meg a 4 egyenletet"}. A legjobb, amikor a
            visszajelzés
            konkrét megfigyelésen alapul, és tudni, hogy mikor, hol történt az
            eset:
            \emph{,,amikor társaiddal Minecraftban házat építettél, akkor
                  pontosan
                  kiszámoltad a ház területét.}
      \item Ha a cél nem a mások legyőzése, akkor a visszajelzés se
            tartalmazzon
            olyan állítást, ami másokhoz hasonlít (így kerüljük a
            \emph{tehetség}
            szót is,
            aminek bevett definíciója szerint az átlagnál jobb képesség). A
            másokhoz való
            szint felmérése akkor (és csak akkor) fontos, amikor a cél egy
            versenyszituációban jó eredményt elérni.

      \item A tanuló legyen részese a visszajelzésnek. Értse, tudja, hogy miért
            kapta
            azt a visszajelzést, a legjobb, ha -- amikor ezt a képességei
            engedik
            -- önmaga
            képes elvégezni a visszajelzést, vagy annak egy részét.
      \item A visszajelzésnek transzparensen hatással kell lennie a
            tanulásszervezésre. Legyen része a folyamatnak, és a gyerek, tanár
            és a
            szülő
            is értse, hogy a visszajelzés alapján mit és hogyan csinálunk
            másképp.
\end{itemize}

\paragraph{Többszintű visszajelzés} A Budapest School iskolákban a gyerekek
többféle visszajelzést kapnak. \begin{enumerate}
      \item Minden modul elvégzése után a modul céljai, témája, fókusza alapján
            a
            modulvezetők visszajelzést adnak a tanulásról, eredményekről,
            viselkedésről.
      \item Trimeszterenként a mentorok visszajelzést adnak arról, hogy a
            tanuló
            általában hogyan haladt a tanulási célok felé.
      \item Ennek része, hogy a tantárgyi tanulási eredmények alapján hogyan haladt a
            tanuló a tantárgyak évfolyam-szinthez tartozó követelmények
            teljesítésében. Az
            évfolyamok, mint elérhető szintek Budapest School értelmezését
            \aref{sec:evfolyamok}. fejezet tárgyalja.
      \item A mentorok irányításával a tanulók visszajelzést kapnak arról,
            hogyan
            működnek a közösségben.
\end{enumerate}

\paragraph{Érdemjegyek, osztályzatok helyett értékelőtáblázatok} A Budapest
School visszajelzéseinek sokkal részletesebbeknek kell lenniük, mint azt a
tantárgyi érdemjegyek és osztályzatok lehetővé teszik, ezért azok helyett a
kerettanterv
értékelőtáblázatokat (angolul rubric) alkalmaz. Az értékelőtáblázatban
szerepelnek az értékelés szempontjai és azok szintjei is, rövid leírásokkal.
Ezek alapján a gyerekek maguk is láthatják, hogy hol tartanak, hogyan
javíthatnak még a munkájukon. A táblázatok formája minden visszajelzés esetén
változtatható (értsd modulonként, célonként).

\section{Portfólió}
\label{sec:portfolio}
A modulok eredményeiből, a produktumokból és visszajelzésekből portfóliót
állít össze a gyerek, a mentor, hogy a tanulás mintázatait észlelhessés, és a
tanárok tudatosabban tudják
a gyereket segíteni a céljai kitalálásában és elérésében. A portfólió a gyerek
céljainak nyomonkövetését szolgálja, és egyúttal a szülői visszajelzés eszköze
is. Minden tanulónak folyamatosan épül a portfóliója: ez tartalmazza az általa
elvégzett feladatokat, projekteket vagy azok dokumentációját, alkotásait,
eredményeit, az esetleges vizsgák eredményeit és a visszajelzéseket társaitól,
tanáraitól. A \emph{portfólió célja}, hogy minden információ meglegyen ahhoz,
hogy

\begin{itemize}
      \item a tanuló és mentora fel tudja mérni, hogy sikerült-e a kitűzött
            célokat
            elérni, illetve mire van szüksége még a tanulónak új célok
            eléréséhez;

      \item a szülő folyamatosan rálásson a gyereke tanulási útjára;

      \item megítélhető legyen, hogy a tantárgyi fejlesztési területekhez
            képest
            hol
            tart a tanuló;

      \item a gyerek a portfólió megtekintésével visszaemlékezhessen a
            tanultakra,
            ismételhessen, tudása elmélyülhessen;

      \item eredményei alapján bizonyítványt lehessen kiállítani.

\end{itemize}
A portfólió alakításának eseményei
\begin{enumerate}
      \item Minden \emph{modul végeztével} a portfólióba kerül:

            \begin{enumerate}

                  \item  A képesség elsajátításának, tanulási eredmény
                        elérésének a ténye.
                        Nincs
                        félig elsajátított képesség, tehát már értékelni nem
                        kell. Ha a modul során a gyerek megtanult százas számkörben alapműveletek
                        végezni, akkor
                        akkor
                        annyi kerül be, a portfólióba, hogy ,,\emph{Szóban és írásban
                              összead, kivon, szoroz és oszt a százas
                              számkörben.}". Amennyiben a
                        készséget a
                        gyerek és a
                        tanár megítélése alapján nem sikerült megfelelően
                        elsajátítani,
                        úgy a gyakorlás
                        ténye kerül be a portfólióba.
                  \item Az alkotás vagy a projektmunka eredménye, ha a modul
                        célja egy
                        alkotás
                        létrehozása volt.
                  \item A részvétel ténye, ha a jelenlét volt a modul célja
                        (például
                        kirándulás
                        az Országos Kéktúra útvonalán).

            \end{enumerate}
      \item Az elvégzett vizsgák, tudáspróbák, képességfelmérők, diagnózisok
            eredményei.

      \item A kipakolás, amelynek célja, hogy a tanulók a tanároknak, szülőknek
            és
            más érintetteknek bemutassák a két kipakolás között elvégzett
            munkájukat, azaz
            a portfólió változásukat. A kipakolásra való felkészülés
            tulajdonképpen
            a
            portfólió összeállítása, prezentálásra való felkészítése, a
            \emph{portfólió
                  frissítése}.

      \item Társas visszajelzés eredményeként minden tanuló kap visszajelzést a
            társaitól. Ilyenkor összegyűjtik, mit tett a tanuló, ami a többiek
            elismerését
            és háláját kivívta.

      \item A tanuló saját értékelése, reflexiója arról, hogyan értékeli, amit
            elért.

      \item A tanárok folyamatosan adnak kompetencia tanúsítványokat. Ezek
            rövid,
            specifikus visszajelzések, amelyek mutatják, ha valamit a tanuló
            megcsinált,
            valamiben fejlődött.
\end{enumerate}

A mentorok segítenek a gyerekeknek a tanulás módját, folyamatát és eredményeit bemutatni
portfólióban.

\paragraph{Formai követelmények}
A portfólió formátumára nincs egységes megkötés. Minden mikroiskola maga
alakítja ki a gyerekek, tanárok és szülők számára legjobban működő rendszert.
Így arról is ők döntenek, hogy digitális vagy analóg formátumban tárolják a
portfóliót. Azonban fontos, hogy a portfóliónak rendezettnek, hozzáférhetőnek,
elérhetőnek, visszakereshetőnek és könnyen bővíthetőnek kell lennie. Olyan
megoldást kell a mikroiskoláknak választaniuk, ami alapján
a gyerek, tanár és a szülő \emph{naponta} tudja a portfóliót bővíteni, és akár
\emph{heti rendszerességgel} át tudják tekinteni időrendben, modulonként vagy
tantárgyanként a portfólió bővülését.


\chapter{A tanulás keretei}

\section{Biztonságos tanulási környezet}

A Budapest School a tanulásszervezés folyamatait befolyásolja, annak folyamatos
fejlesztését tűzte ki célul. A folyamatok kialakításakor és akkor, amikor ezek
a folyamatok valamilyen oknál fogva nem működnek, a következő szempontok
élveznek prioritást:

\begin{enumerate}

  \item Legfontosabb, hogy a tanulás és az iskolában eltöltött idő mindenki
        számára legyen biztonságos fizikai és érzelmi szempontból egyaránt.

  \item Egyensúlyban kell tartani az iskola  tagjainak egyéni fejlődését és
        tanulását és a közösség tagjainak együttműködését, kapcsolódását.

\end{enumerate}
Ez azt is jelenti, hogy nem tartható fenn az az állapot, amikor a közösség
egyik tagjának érdekei felülírják a többiek érdekeit, ahogy az sem, amikor a
közösség valakinek az igényeit nem veszi figyelembe, vagy amikor a közösségi
kapcsolódás felülírja a tanulási célokat. A közösség minden tagja számára
biztonságos és kiegyensúlyozott környezet kialakításáért, és fenntartásáért a
tanulásszervező tanárok a felelősek.

\section{Tantárgyak}
\label{sec:tantargyak}
A Budapest School a ma gyerekeinek kínál olyan oktatást, ami segíti felkészíteni őket a jövő kihívásaira. Információs társadalmunk legnagyobb kihívása az adaptációs képességünk fejlesztése, ez az alapja annak, hogy képesek legyünk eligazodni a folyamatosan változó, komplex világunkban. A tanulásunk célja, hogy boldog, hasznos és egészséges tagjai legyünk a társadalomnak. Iskolánkban a tanulás három rétege, a tudásszerzés, a megtanultakat elmélyítő önálló gondolkodás és az aktív alkotás egyszerre jelennek meg.

A kerettanterv a célok eléréséhez a miniszter által kiadott kerettantervek tantárgyi struktúráját használja  a moduláris tanulás tartalmi keretezéséhez. A keretezésen azt értjük, hogy a tantárgyak tartalma határozza meg, hogy mivel kell mindenképp a Budapest School iskolákban foglalkozni, mit kell mindenképp megtanulni. 
Az egyes modulok ezen tantárgyak tanulási eredményeinek elérését támogatják.


\subsection{Tantárgyi definiciók és a tanulási eredmények}
A tantárgyak tanulási eredmények felsorolásával adnak tartalmi szabályozást. A tanulási eredmények (learning outcomes) tudás, képesség, kompetencia kontextusában meghatározott kijelentések arra vonatkozóan, hogy a tanulónak mit kell tudnia, mit kell értenie, és mire legyen képes, miután lezárt egy tanulási folyamatot, függetlenül attól, hogy hol, hogyan, mikor szerezte meg ezeket a kompetenciákat \citep{learning_outcomes}.  Vagyis  az egyes modulok különféle tanulási eredmények elérését is támogathatják, ezzel több tantárgy részcéljait is teljesíthetik.

A kerettanterv tantárgyankénti és félévenkénti bontásban adja meg a továbbhaladáshoz elengedhetetlen tanulási eredmények listáját.

A tantárgyi definiciókhoz a miniszter által kiadott kerettanterv ,,elvárt eredmények a tanulási ciklus végén" fejezetek felsorolásait alakítottuk át egységes nyelvezetre, hogy azok valóban kompetenciákat írjanak.

A tantárgyi specifikációk nem térnek ki rész\-letesen a tematikákra. Ez szabadságot ad a tanároknak arra, hogy a tanmenet tekintetében akár jelentős eltérések legyenek addig, amig a miniszter által kiadott kerettanterv mérhető tanulási eredményei teljesülnek. A tanulási eredmény alapú szabályozás folyamatos visszacsatolást tud adni a tanulónak és a tanároknak, megmutatva, melyik tanulási eredményeket kell még elérni a következő szintre való lépéshez.

\paragraph{Tantárgyak szerepe a mindennapokban}

A Budapest School iskoláinak tantárgyi leírásai a miniszter által kiadott kerettanterv alapján készültek. Az egyes tanulási moduloknak a portfólióba való elhelyezését követően háromhavonta összevetjük az elért tanulási eredményeket és  a tantárgyi kötelezően választható tanulási eredményeket, hogy ezáltal folyamatosan monitorozni lehessen az iskolai követelmények és a gyerekek egyéni eredményei közötti egyensúlyt.

A Budapest School iskoláiban a tantárgyak ugyanúgy kapnak szerepet, mint a NAT által definiált kulcskompetenciák, fejlesztési területek: tanár sose mondja azt a gyerekeknek, hogy „most kezdeményezőképességet és vállalkozói kompetenciát fejlesztünk'', hanem a különböző feladatok elvégzése eredményeképp történik a fejlesztés. A Budapest School iskolákban a tantárgyközi tevékenységek vannak előtérben. A tantárgyak a tanulás tartalmi elemeinek forrása és keretei: a tanulandó dolgok listájaként működik. Az, hogy milyen csoportosításban történik a tanulás, az a modulvezetőkre van bízva.

A tantárgyak ezért elsősorban a modulok kiírásakor és azok kimeneti értékelésekor jelennek meg, a mindennapok struktúráját, a napi- és hetirendet azonban a modulok adják. Egyes modulok több tantárgy fejlesztési céljainak is eleget tehetnek, több tantárgy tanulási eredményének elérését is célul tűzhetik ki, összhangban a NAT-tal. A tantárgyaknak ezzel együtt fontos célja, hogy segítse a tanulás tartalmi egyensúlyának fennmaradását. A tanulásszervezők, modulvezetők szakképesítése nem köthető a Budapest School tantárgyaihoz, felelősségük, hogy a saját moduljukban megfelelően tudják szervezni a tanulást, és legfőképp, hogy saját moduljuk megtartására alkalmasak legyenek.

\section{NAT céljainak támogatása}
\label{sec:nat_celjai}
A Nemzeti Alaptantervben szereplő fejlesztési célok elérését és a
kulcskompetenciák fejlődését több minden támogatja:

Egyrészt a tantárgyak lefedik a NAT fejlesztési céljait, kulcskompetenciáit és
műveltségi területeit, mert a jelenleg érvényben lévő, a miniszter által a
\emph{51/2012. (XII. 21.) számú EMMI rendelet I-IV. mellékletében} kiadott
kerettantervek \citep{ofi:kerettanterv} tanulási, tanítási eredményeiből
indultunk ki. Mivel a rendeletben szereplő kerettantervek teljesítik a NAT
feltételeit, így a Budapest School tantárgystruktúrája is teljesíti ezeket.

Másrészt az iskola életében, folyamatában való részvétel, már önmagában
biztosítja a kulcskompentenciák fejlődését és a NAT fejlesztési céljainak
teljesülését sok esetben.

A \ref{tbl:nat_fejlesztesi} táblázat bemutatja a NAT fejlesztési területeihez
való kapcsolódást, a
\ref{tbl:nat_kulcs} táblázat pedig az illeszkedési pontokat a NAT
kulcskompetenciáihoz.

\begin{table}

  \begin{tabular}{p{5cm}|>{\raggedright}p{3cm}|p{3cm}}

    \textbf{NAT Fejlesztési célok}               & \textbf{Tantárgyak}  & \textbf{Struktúra}           \\
    \hline
    Az erkölcsi nevelés                          & kult, harmónia       & közösség                     \\ \hline
    Nemzeti öntudat, hazafias nevelés            & kult, harmónia       & projektek                    \\ \hline
    Állampolgárságra, demokráciára nevelés       & kult, harmónia       & közösség                     \\ \hline
    Az önismeret és a társas kultúra fejlesztése & kult, harmónia, stem & saját
    tanulási út, közösség                                                                              \\ \hline
    A családi életre nevelés                     & harmónia             &                              \\ \hline
    A testi és lelki egészségre nevelés          & harmónia             & közösség                     \\ \hline
    Felelősségvállalás másokért, önkéntesség     & harmónia             & közösség, pro\-jek\-tek      \\
    \hline
    Fenntarthatóság, környezettudatosság         & harmónia, stem       & projektek                    \\ \hline
    Pályaorientáció                              & kult, harmónia, stem & saját tanulási út            \\ \hline
    Gazdasági és pénzügyi nevelés                & kult, harmónia, stem & projektek                    \\ \hline
    Médiatudatosságra nevelés                    & kult                 & projektek                    \\ \hline
    A tanulás tanítása                           & kult, harmónia, stem & saját tanulási út, mentorság \\

  \end{tabular}
  \caption{A NAT fejlesztési céljainak elérését nemcsak a tantárgyak, hanem az
    iskola struktúrája is támogatja.}
  \label{tbl:nat_fejlesztesi}
\end{table}

A \emph{saját tanulási} út fogalma például önmagában segíti a tanulás
tanulását, hiszen az a gyerek, aki képes önmagának saját célt állítani (mentor
segítséggel), azt elérni, és a folyamatra való reflektálás során képességeit
javítani, az fejleszti a tanulási képességét.

Vagy másik példaként, a Budapest School iskoláiban a \emph{közösség} maga hozza
a működéséhez szükséges szabályokat, folyamatosan alakítja és fejleszti saját
működését a tagok aktív részvételével. Ez az aktív állampolgárságra, a
demokráciára való nevelés Nemzeti Alaptantervben előírt céljait is támogatja.

\begin{table}
  \centering
  \begin{tabular}{p{5cm}|>{\raggedright}p{3cm}|p{3cm}}

    \textbf{NAT kulcskompetenciái}                     & \textbf{Tantárgyak}  &
    \textbf{Struktúra}                                                                                        \\ \hline
    Anyanyelvi kommunikáció                            & kult                 & tanulási szerződés, portfólió \\ \hline
    Idegen nyelvi kommunikáció                         & kult                 & idegennyelvű modulok          \\ \hline
    Matematikai kompetencia                            & stem                 &                               \\ \hline
    Természettudományos és technikai kompetencia       & stem                 & projektek                     \\ \hline
    Digitális kompetencia                              & harmónia, stem       & digitális portfólió kezelés   \\ \hline
    Szociális és állampolgári kompetencia              & harmónia             & saját tanulási út,
    közösség                                                                                                  \\ \hline
    Kezdeményezőképesség és vállalkozói kompetencia    & kult, harmónia, stem & saját
    tanulási út, közösség                                                                                     \\ \hline
    Esztétikai-művészeti tudatosság és kifejezőkészség & harmónia, kult       &                               \\ \hline
    A hatékony, önálló tanulás                         & kult, harmónia, stem & saját tanulási út,
    mentorság                                                                                                 \\

  \end{tabular}
  \caption{NAT kulcskompetenciáinak fejlesztését támogatják a tantárgyak és
    az iskola felépítése is.}
  \label{tbl:nat_kulcs}
\end{table}

\section{A tanév ritmusa}

A tanév három trimeszter ismétlődésével írható le: a tanulási célok
tervezése után következik a tanulás, és a ciklust a visszajelzés és értékelés
zárja.	Amint egy ciklus véget ér, elkezdődik egy új.

A ciklusok állandósága adja a tanulás irányításához szükséges kereteket. Ezek
megtartásáért az egyes mikroiskolák tanulásszervezői felelnek, melynek működését a
fenntartó monitorozza. A tanév ritmusát \aref{tbl:tanevritmus}. táblázat
mutatja.

\begin{table}
  \centering
  \begin{tabular}{ l|l }
    \textbf{időszak} & \textbf{tevékenység}                \\
    \hline
    Szeptember       &
    közösségépítés                                         \\
                     & saját célok meghatározása           \\
                     & modulok kialakítása és meghirdetése
    \\ \hline

    Október          &
    tanulás, alkotás
    \\ \hline

    November         &
    tanulás, alkotás
    \\ \hline

    December         &
    portfólió frissítése                                   \\
                     & reflexiók                           \\
                     & visszajelzések                      \\
                     & célok felülvizsgálata               \\
                     & modulok változtatása igény esetén
    \\ \hline

    Jánuár           &
    tanulás, alkotás
    \\ \hline

    Február          &
    tanulás, alkotás
    \\ \hline

    Március          &
    portfólió frissítése                                   \\
                     & reflexiók                           \\
                     & visszajelzések                      \\
                     & célok felülvizsgálata               \\
                     & modulok változtatása igény esetén
    \\ \hline

    Április          &
    tanulás, alkotás
    \\ \hline

    Május            &
    tanulás, alkotás
    \\ \hline

    Fél június       &
    évzárás, értékelés, bizonyítványok
  \end{tabular}
  \caption{Egy tanévben háromszor ismételjük a célállítás, tanulás,
    reflektálás ciklust.}
  \label{tbl:tanevritmus}
\end{table}

A tanév három periódusból áll: ez a felosztás követi az üzleti világ negyedéves
tervezését, néhány egyetem trimeszterekre bontását, de leginkább az évszakokat.
Minden periódus után értékeljük az elmúlt három hónapot, ünnepeljük az
eredményeket, és megtervezzük a következő időszakot.  A trimesztereken belül az
egyes mikroiskolák között lehetnek néhány hetes eltérések, melyek a közösség
sajátosságait követik.

\subsection{Félévenkénti bontás}
IDE KELL MAJD VALAMI A VÉGLEGES TORVENYKOR
\section{Érettségire készülés}
\label{sec:erettsegi}

Az iskola a kötelező középszintű érettségi vizsgatárgyakra való felkészítést kötelezően vállalja érettségire felkészítő modulok szervezésével. Lefordítva ezt a miniszter által kiadott kerettantervek alapján működő iskolák esetén használt terminológiára, az érettségire felkészülés érettségi tárgyak alapján szervezett fakultáció formájában történik.

A választható tantárgyak és az emelt szintű érettségi vizsgára csak akkor szervez egy mikroiskola modult, ha arra legalább a közösség 20\%-a és minimum 6 gyerek igényt tart. Abban az esetben, ha minden választható tantárgyat csak kevesebb,  mint 20\% vagy 6 gyerek választ, és így a közösség nem tud választható érettségi tárgyat választani, a fenntartó véletlenszerűen sorsol legalább egy választható tárgyat a gyerekek által megjelöltekből.

Érettségire felkészítő modulokat akkor kell meghirdetni, amikor a gyerekek elérik a 11. évfolyamszintet minden tárgyból.

Különböző mikroiskolákba járó gyerekek közös modulon készülhetnek az érettségire. A fenntartó, ha nem tudja maga megszervezni a felkészítő modulokat, akkor más iskolákkal együttműködve kell hogy biztosítsa a felkészülési lehetőséget.
\paragraph{Heti óraszámok}

A Budapest School közösségi tanulási élményeket és modulokat szervez a
tanulóinak, egyúttal lehetőséget ad arra, hogy a gyerekek a közösen kialakított
szabályaik mentén tanulásszervezők felügyeletével a Budapest School székhelyén
vagy egyes telephelyein, vagy más, erre alkalmas tanulási környezetben
tartózkodjanak. A közösségben együtt töltött idő tanulásnak, fejlődésnek
minősül akkor is, ha az nem egy modulhoz kapcsolódik, hanem az ebéd
élvezetéhez, vagy épp a parkban a lehulló falevelek neszének megfigyeléséhez.

A gyerekek, a tanítási szüneteket leszámítva, naponta 8 órát tartózkodnak az
iskolában. Ezekben az időkben vannak a tanítási órák, foglalkozások, szakkörök,
műhelyek. Az egyes mikroiskolák ettől 20\%-ban bármelyik irányban eltérhetnek,
ha ez segíti a tanulásszervezők munkáját és a gyerekek fejlődését. Így hetente
minimum $5 \cdot 8 \cdot 0.8 = 32$ órát, maximum 48 órát töltenek az iskolában.

Ennek alsó tagozatban körülbelül a felét, a felső tagozatban kétharmad részét
töltik előre eltervezett módon, azaz modulokkal. A többi időben a tanárok
vezetése és felügyelete mellett szabadon alkotnak, játszanak, pihennek,
közösségi életet élnek. Azaz alsó tagozatban 16--24, míg felső tagozatban
21--32 órát töltenek modulokkal.

Mivel az elvárt kiegyensúlyozottság miatt mind a három tantárgyra körülbelül
ugyanannyi energiát kell fektetni, így az egyes tantárgyakra a teljes
rendelkezésre álló időkeret egyharmad részét kell számolni. Ettől az iskolák
$\pm$ 20\%-ban eltérhetnek, így kiszámolható, hogy minimum mennyi időt kell
egy-egy gyereknek egy héten egy tantárggyal foglalkoznia. Ezt összegzi
\aref{tbl:oraszamok}. táblázat.

\begin{table}

  \begin{tabular}{ l|l|l }

    \textbf{Tantárgy} & \textbf{Alsó
    tagozat}          & \textbf{Felső tagozat}                                                      \\ \hline
    Harmónia          & $\frac{5 \times 8 \times 0.8}{2} \times \frac{1}{3} \times 0.8 =
    4.27$ óra         &
    $\frac{5 \times 8 \times 0.8 \times 2}{3} \times \frac{1}{3} \times 0.8 = 5.69$
    óra                                                                                             \\ \hline
    STEM              & 4.27 óra                                                         & 5.69 óra \\ \hline
    KULT              & 4.27 óra                                                         & 5.69 óra \\ \hline

  \end{tabular}
  \caption{Az elvárt kiegyensúlyozottság miatt a tantárgyakkal egyenlő
    minimális óraszámban kell foglalkozni.}
  \label{tbl:oraszamok}
\end{table}

Fontos, hogy \emph{egy-egy modul több tantárgy fejlesztési céljaihoz és
  eredménycéljaihoz is kapcsolódhat.}

\section{Évfolyamok és osztályzatok}
A Budapest School gyerekek saját tanulási célokat tűznek ki, modulokat
választanak, tanulnak, alkotnak, trimeszterenként frissíti a portfóliójukat,
mentorukkal és a modulvezetőkkel értékelik haladásukat, és ha kell,
újraterveznek.

Eközben a gyerekek a tanulás és alkotás eredményeként évfolyamszinteken
lépkednek fel,
első szintről a tizenkettedik szintig tantárgyanként.
Azt, hogy ez hogyan és mikor történik, azaz az évfolyamszintek elismerését -- a
kerettantervvel összhangban -- az iskola transzparens folyamata
szabályozza.

Hivatalos, tantárgyankénti érdemjegyet a gyerekek akkor és csak akkor kapnak,
ha erre iskolaváltás, továbbtanulás, ösztöndíj	    (vagy más külső rendszer)
miatt szükségük van.
Tehát osztályzatok, érdemjegyek és vizsgák nélkül is van lehetőség
évfolyamszinteket lépni.

A vizsga így teljesen átértékelődik a Budapest Schoolban.
Az évfolyamszintek elismeréséhez és (szükség esetén) az érdemjegyek
megállapításához
nem elégséges a pillanatnyi tudást vagy képességet felmérő eseményt szervezni,
hanem
a teljes portfóliót kell értékelni és figyelembe venni.
A portfólió sokkal gazdagabban dokumentálja, hogy egy gyerek mit csinált, mire
volt képes, mint egy szóbeli vagy írásbeli feladatsor: előzetes tudás-,
képességpróbák
mellett tartalmazza az alkotások, projektek, visszajelzések, versenyek, stb.
dokumentációt is.

\subsection{Évfolyamszintek}
\label{sec:evfolyamok}

A Budapest Schoolban az évfolyamokra úgy tekintünk, mint egy szerepjáték
nehézségi
szintjeire \citep{wiki:game_levels}: akkor léphet egy tanuló a következőbe, ha
az
évfolyamhoz köthető tantárgyi tanulási eredményekből eleget összegyűjtött.
Ezeket \aref{sec:tantargyi_tanulasi_eredmenyek} fejezet határozza meg.

A Budapest School évfolyamszintjei eltérnek az iskolák többségében alkalmazott
évfolyamtól. A különbség kihangsúlyozása végett a kerettanterv az évfolyamszint
kifejezést használja. A különbségek:

\begin{itemize}
      \item Egy gyerek tantárgyanként más-más szinten állhat.
      \item Nem biztos, hogy az egy korcsoportba tartozó gyerekek vannak
            ugyanazon az évfolyamszinten.

      \item Nem mindig az egy évfolyamszinten lévők tanulnak együtt,
            előfordulhat, hogy a különböző szinten lévő tanulók tudnak együtt
            és akár
            egymástól is tanulni.

      \item Egy év alatt több évfolyamszintet is lehet lépni.
\end{itemize}

Bár egy gyerek tantárgyanként eltérő szinten állhat , a hivatalos (azaz külső
hivatalok, rendszerek számára értelmezhető) bizonyítványába mindig csak annak
az évfolyamnak az elvégzése kerül be, amelyből mind a három tantárgyhoz
szükséges fejlesztési célt elérte. Formálisabban kifejezve a bizonyítványban a
tantárgyankénti évfolyamszintek minimumát kell rögzíteni.

\subsubsection{Évfolyamszint-lépés}
\label{sec:evfolyamszintlepes}
A tanulók portfóliója alapján megállapítható, hogy egy adott évfolyamhoz
köthető tantárgyi követelményeknek megfelel-e.
Ehhez a tanulók elvégzik a mentoruk segítségével a portfóliójuk (mit csináltak,
mit tanultak, mit tudnak) összehasonlítását \ifkerettanterv
      \aref{sec:tantargyi_tanulasi_eredmenyek}.
      fejezetben
\else
      a kerettanterv \emph{Tantárgyi tanulási eredmények} c. fejezetében
\fi
felsorolt tantárgyankénti bontásban megadott elvárt tanulási eredményekkel.

Ha a kapcsolódás biztosításához szükséges, a gyerekek a portfóliójukat
kiegészíthetik tudáspróbák, tesztek, szabványos vizsgák teljesítésével, melynek
megszervezése az adott mikroiskola tanárközösségének a feladata.

Miután a gyerek (mentora és szülei) segítségével összeállította a portfólióját,
jelzi az iskolának az évfolyamszint lépési kérelmét.

Ezt az iskola által kijelölt tanulásszervezők\footnote{A kérelmeket elbíráló
      tanulásszervzők kijelölését a pedagógia programnak vagy a szervezeti és
      működési szabályzatnak kell meghatároznia.}  
megvizsgálják és elismerik az
évfolyamszinthez szükséges tantárgyi követelmények teljesítését.
Egy tantárgyból egy évfolyam teljesítettnek tekinthető, ha a tantárgyhoz
tartozó tanulási eredmények 50\%-ának elérése a portfólió alapján bizonyítható.

A kérelmet a gyerek digitálisan adja be.
Az elbírálás csak a portfólió alapján történhet, ami egy
online elérhető adatbázisként tartalmaz mindent, ami a döntéshez szükséges
lehet. A döntéshez így a tanulásszervezőknek és a gyereknek nem
kell egy időben és egy helyen lennie. Minden esetben szükséges a portfóliót és
a teljes folyamatot digitálisan rögzíteni.
Az iskolának két évig meg kell őriznie a portfóliót, a kérelmet és a döntéshez
használt minden dokumentációt.

Magántanuló vagy az órák látogatásáról valamilyen okkal
felmentett gyerekek ugyanígy, a portfóliójuk összeállításával és a szintlépés
kérelmezésével kérhetik az évfolyamszintek teljesítésének igazolását.

Amennyiben valamely gyereknél egy adott évfolyamszint tantárgyi követelményei
elismerésre kerülnek, akkor az iskola igazolja, hogy a gyerek az adott tantárgy
vagy tantárgyak évfolyam szerinti követelményeit teljesítette.
Erről igazolást állít ki, és teljesíti a jelentési kötelezettségét az Oktatási
Hivatal felé.

\subsection{Osztályzatokra váltás}
\label{sec:osztalyzatok}
A kerettanterv lehetővé teszi, hogy a gyerekek az érdemjegyek
és osztályzatok
helyett egy több szempontot figyelembe vevő szöveges vagy értékelőtáblázat
(rubric) alapú visszajelzést kapjanak.
A gazdag információtartalmú visszajelzések és portfólió osztályzatra való
átváltására mégis szükség lehet, például iskolaváltás vagy továbbtanulás
esetén.\footnote{Azt az NKT. 54.§ (4) pontja alapján.}

Az átváltás \aref{sec:evfolyamszintlepes}.
fejezetben leírtakhoz
hasonlóan is a portfólió értékelésén alapul.
A gyerek (mentora és szülei segítségével) összeállítja a portfóliót, és
bizonyítja, hogy a portfólió alapján megállapítható a kívánt osztályzat, az
adott tárgyhoz az adott évfolyamszinten.

\paragraph{,,Szokásos'' tantárgyakra leképzés}
A gyerekek (szüleik) kérhetik NAT pedagógiai szakaszainak
végén\footnote{A NAT II.2.1 pontja három pedagógia szakaszt határoz meg: az
      alapfokú nevelés-oktatási két szakasza az 1-4. évfolyamok és 5-8.
      évfolyamok,
      a középfokú nevelés-oktatás szakasz a 9-12. évfolyam.}
a miniszter
által kiadott kerettanterv tantárgyi rendszere szerinti értékelést és az annak
történő megfeleltetést. Ez a tanulási eredmények adatbázisa alapján
egyértelműen elvégezhető, mert minden tanulási eredmény egy tanulási területhez
tartozik, amik pedig egy az egyben kapcsolódnak a miniszter által kiadott
tantárgyakkal.


\section{Minőségfejlesztés, folyamatszabályozás}
\label{sec:minosegbiztositas}
\begin{quote}
      Honnan tudjuk, hogy jól működnek a mikroiskolák? Minden gyerek azt
      tanulja,
      amit szeretne, és amire neki a leginkább szüksége van? Megkérdezzük az
      érintetteket és figyeljük az adatokat.
\end{quote}

A Budapest School iskolában a gyerekek tanulását egy komplex rendszer egyik
elemének tekintjük \citep{barabasi}. Iskoláink egy, a világ felé nyitott
hálózatot alkotnak: az egy közösségben lévő gyerekek, a családok, a velük
foglalkozó tanárok, a helyi környezet, a Budapest School további
mikroiskolái\-nak közössége, az ország és a nemzet állapota, valamint a globális
társadalmi folyamatok is befolyásolják, hogy mi történik egy iskolában.

Hogy egy gyerek épp mit és mennyit tanul, nemcsak a kerettantervtől,
hanem számos más tényezőtől is függ, melyek befolyásolják a fejlődést és a
közösség egymás fejlődésére gyakorolt hatását: többek közt függ a gyerekek
múltjától, aktuális hangulatától és vágyaitól, a tanárok személyiségétől, a
családtól, a csoportdinamikától és a társadalomban történő változásoktól
is.

Ezért a gyermekeink fejlődését, tanulását és boldogságának alakulását egy
\emph{komplex rendszer} működésével modellezhetjük.

A tanulási folyamataink minőségfejlesztésénél a következő szempontokat vesszük
figyelembe:
\begin{enumerate}
      \item  A történéseket, eseményeket, (rész)eredményeket folyamatosan kell
            monitorozni.
      \item  A visszajelzéseket a rendszer minden tagjától folyamatosan
            gyűjteni kell: a gyerekektől, tanároktól, szülőktől és az
            adminisztrátoroktól.
      \item Anomália esetén a helyzetfelismerés, az eltérések okának
            felkutatása a
            cél.
      \item A feltárt hibák alapján a rendszert folyamatosan kell javítani.
\end{enumerate}

A minőségfejlesztés célja az iskola, mint tanuló rendszer folyamatos
fejlesztése. A monitorozás folyamatos,
így az iskola hamar  felismeri az anomáliákat, és kivizsgálás után, ha szükséges,
akkor azt meg tudja szüntetni, tanulni is tud belőle, és
javítja a rendszert.

A fenntartó folyamatosan monitorozza a mikroiskolákat és visszajelzéseket ad,
ami alapján a mikroiskola javítja a saját működési folyamatait. A kerettantervben leírt működést a
fenntartó mérhető és megfigyelhető metrikákra fordítja le, és kidolgozza,
üzemelteti a metrikák 2-3 hónapnyi rendszerességű nyomon követésére alkalmas
rendszerét.

A fenntartónak meg kell figyelnie legalább a következő metrikákat:
\begin{enumerate}
      \item A szülő, a gyerek és a tanár közötti saját célokat megfogalmazó
            hármas
            megállapodások időben megszülettek, nincs olyan gyerek, akinek
            nincs
            elfogadott
            saját tanulási célja. Metrika: elkészült szerződések száma.

      \item A modulok végén a portfóliók bővülnek, és azok tartalma a
            tantárgyakhoz
            kapcsolódik. Metrika: a portfólió elemeinek száma és kapcsolhatósága.
      \item A tanulási eredményekre vonatkozó megkötések
            (lásd a \aref{sec:kotelezo_tanulasi_eredmenyek}.~fejezet) időben teljesülnek.
            Metrika: elért kötelező tanulási eredmények száma gyerekekre
            bontva.

      \item A szülők biztonságban érzik gyereküket, és eleget tudnak arról,
            hogy
            mit
            tanulnak. Metrika: kérdőíves vizsgálat alapján.

      \item A tanárok hatékonynak tartják a munkájukat. Metrika: kérdőíves felmérés
            alapján.

      \item A gyerekek úgy érzik, folyamatosan tanulnak, támogatva vannak,
            vannak
            kihívásaik. Metrika: kérdőíves felmérés alapján.
\end{enumerate}


\section{Tanárok kiválasztása, tanulása, fejlődése és értékelése}

Mindenki egyetért abban, hogy az iskola legmeghatározobb összetevői a tanárok.
Ezért a Budapest School külön figyelmet fordít arra, hogy ki lehet tanár az
iskolákban, és hogyan segítjük az ő fejlődésüket.

Alapelveink
\begin{enumerate}
  \item Minden tanárnak tanulnia kell. Amit ma tudunk, az nem biztos, hogy elég
        arra, hogy a holnap iskoláját működtessük. És az is biztos, hogy még sokkal
        hatékonyabban lehetne segíteni a gyerekek tanulását, mint amilyenek a ma ismert
        módszereink.

  \item A tanároknak csapatban kell dolgozniuk, mert összetett
        (interdiszciplináris) tanulást csak vegyes összetételű (diverz) csapatok tudnak
        támogatni.

  \item Szakképesítés nem szükséges és nem elégséges feltétele annak, hogy a
        Budapest Schoolban jól teljesítő tanár legyen valaki.
\end{enumerate}

\paragraph{Bekerülés}
A Budapest School tanulásszervező tanár felvétele egy minimum háromlépcsős
folyamat, ahol vizsgálni kell a tanár egyéniségét (attitűdjét), felnőtt-felnőtt
kapcsolatokban a viselkedésmódját (társas kompetenciáit), és minden jelöltnek
próbafoglalkozást kell tartania, amit erre kijelölt Budapest School tanárok
megfigyelnek. A felvételi folyamatot a fenntartó felügyeli és irányítja.

\paragraph{Saját cél}

Minden tanárnak van saját, egyéni fejlődési célja: \emph{mitől tudok én jobb
  tanár lenni, jobban támogatni a gyerekek tanulását, segíteni a munkatársaimat
  és partnerként dolgozni a szülőkkel?}

\paragraph{Mentor}

Hasonlóan a gyerekekhez, minden tanárnak van egy mentora, aki segíti a saját
céljai kialakításában, és folyamatosan támogatja a célja elérésében.

\paragraph{Értékelés}

Minden tanárt évente legalább kétszer értékelnek a munkatársai. Ez a folyamat,
amit 360 fokos értékelésnek hívnak az üzleti szférában. A visszajelzések
feldolgozása után a saját célokat frissíteni kell.

Minden tanárt értékelnek a szülők is (kifejezetten a mentorált gyerekek szülei)
és a gyerekek is legalább évente kétszer.

\chapter{Jogszabályok által igényelt elvárások\\ rövid összefoglalója}
\chaptermark{Egyéb jogszabályi elvárások}
\label{sec:jogszabalyok}
\section{A nevelés-oktatás célja}

Minél több olyan dolgot tanuljanak a gyerekek, amit szeretnek, vagy amire szükségük van úgy, hogy a mindenkori Nemzeti Alaptanterv követelményeit teljesítsék, és saját céljaikat is el tudják érni.

\section{A tantárgyi rendszer}
\label{sec:jog_tantargy}
A kerettanterv a miniszter által kiadott kerettantervek tantárgystuktúráját átvéve tanulási eredmények halmazaként adja meg a tantárgyakat. A rendszert \aref{sec:tantargyak}.~fejezet részletesen ismerteti.  Budapest School-tantárgy a mindennapokban nem feltétlenül jelenik meg önálló tanóraként, ezek funkciója a fejlesztési célok elérése és az egyensúlytartás. A tudástartalom elsajátítása a tanulási modulokon keresztül történik. A tantárgy elvégzésének feltétele, hogy  a gyerekek egyéni portfóliója lefedje a kerettantervben tanulási eredményekként megadott követelményeket.

A tantárgyi struktúra nem tér el a miniszter által kiadott kerettantervek tantárgyi struktúrájától, mert ennek a kerettantervnek a tantárgyainak tartalmát kijelőlő tanulási eredmények kölcsönösen megfeleltethetőek a miniszter által kiadott kerettantervek eredményelvárásaival. Egyszerűen fogalmazva, ugyanazt a tananyagot definiálja ez a kerettanterv, ugyanabban a stuktúrában, mint a miniszter által kiadott kerettantervek. 

A NAT-ban meghatározott tananyagtartalmak tanévenként két félévre bontva jelennek meg a kerettantervekben, összhangban a tanulók félévenkénti értékelésével. A tananyag 
tartalmakat \aref{sec:tantargyi_tanulasi_eredmenyek}. fejezet definiálja.

\section[A Nat félévenkénti bontása]{A Nat-ban meghatározott tananyagtartalmak félévenkénti bontása}
A kerettanterv a tananyagtartalmakat tanulási eredmények halmazaként adja meg tantárgyanként és félévenkénti bontásban \aref{sec:tantargyi_tanulasi_eredmenyek}.~fejezetben. 

\Aref{sec:osztalyzatok}.~fejezetben leírt módon meghatározza a gyerekek félévenkénti értékelését, ami \aref{sec:feleves_bontas}~.fejezetben leírt módon kerül összhangba a trimeszter alapú tanév struktúrával.

\section{Tantárgyi struktúra eltérése} 
A kerettanterv tantárgyi struktúráját \aref{sec:tantargyak}.~fejezet tartalmazza. Ez pontosan ugyanazokat a tantárgyakat tartalmazza, mint a miniszter által kiadott kerettantervben foglalt tantárgyi struktúra. A tantárgyakhoz rendelt heti óraszámok a miniszter által kiadott kerettantervben megadott óraszámok 75\%-a.

\section{Kerettanterv eltérése}
A 51/2012. (XII. 21.) EMMI rendelet 7. § (cb) pontja \emph{,,szerint a kerettanterv részét képező tantárgyi kerettantervek tematikai egységeinek címei, az ismeretek és fejlesztési követelmények tantervi elemek tartalma és az azokhoz rendelt óraszámok vonatkozásában legalább húsz százalékban el kell térnie a miniszter által kiadott vagy jóváhagyott kerettantervtől.''}

A kerettanterv nem határoz meg tematikai egységeket, azok kidolgozását a tanárokra bízza a modulok összeállításakor. Ugyanakkor a kerettanterv szorgalmazza három kiemelt, interdiszciplináris fejlesztési irányelv (\ref{sec:kiemelt_fejlesztesi_iranyelvek}) követését, ami nagyban eltér a miniszter által kiadott kerettantervek tantárgy alapú fejlesztési követelményeitől. 

A tantárgyakhoz rendelt óraszámokban a kerettanterv  25\%-ban eltér a miniszter által kiadott kerettantervektől.

\section{NAT céljainak támogatása}
\label{sec:nat_celjai}
A Nemzeti Alaptantervben szereplő fejlesztési célok elérését és a
kulcskompetenciák fejlődését több minden támogatja:

Egyrészt a tantárgyak lefedik a NAT fejlesztési céljait, kulcskompetenciáit és
műveltségi területeit, mert a jelenleg érvényben lévő, a miniszter által az
\emph{51/2012.~(XII.~21.) számú EMMI-rendelet I--IV.~mellékletében} kiadott
kerettantervek \citep{ofi:kerettanterv} tanulási, tanítási eredményeiből
indultunk ki. Mivel a rendeletben szereplő kerettantervek teljesítik a NAT
feltételeit, így a Budapest School tantárgystruktúrája is teljesíti ezeket.

Másrészt az iskola életében, folyamatában való részvétel már önmagában
biztosítja a kulcskompetenciák fejlődését és a NAT fejlesztési céljainak
teljesülését sok esetben.

Az \ref{tbl:nat_fejlesztesi}.~táblázat bemutatja a NAT fejlesztési területeihez
való kapcsolódást, az
\ref{tbl:nat_kulcs}.~táblázat pedig az illeszkedési pontokat a NAT
kulcskompetenciáihoz.

\begin{table}
    \centering
  \begin{tabular}{p{5cm}|p{3cm}}

    \textbf{A NAT fejlesztési céljai}             & \textbf{Struktúra}           \\
    \hline
    Az erkölcsi nevelés                           & közösség                     \\ \hline
    Nemzeti öntudat, hazafias nevelés             & projektek                    \\ \hline
    Állampolgárságra, demokráciára nevelés        & közösség                     \\ \hline
    Az önismeret és a társas kultúra fejlesztése  & saját tanulási út, közösség  \\ \hline
    A családi életre nevelés                      &                              \\ \hline
    A testi és lelki egészségre nevelés           & közösség                     \\ \hline
    Felelősségvállalás másokért, önkéntesség      & közösség, pro\-jek\-tek      \\
    \hline
    Fenntarthatóság, környezettudatosság          & projektek                    \\ \hline
    Pályaorientáció                               & saját tanulási út            \\ \hline
    Gazdasági és pénzügyi nevelés                 & projektek                    \\ \hline
    Médiatudatosságra nevelés                     & projektek                    \\ \hline
    A tanulás tanítása                            & saját tanulási út, mentorság \\

  \end{tabular}
  \caption{A NAT fejlesztési céljainak elérését nemcsak a tantárgyak, hanem az
    iskola struktúrája is támogatja.}
  \label{tbl:nat_fejlesztesi}
\end{table}

A \emph{saját tanulási} út fogalma például önmagában segíti a tanulás
tanulását, hiszen az a gyerek, aki képes önmagának saját célt állítani (mentori
segítséggel), azt elérni, és a folyamatra való reflektálás során képességeit
javítani, az fejleszti a tanulási képességét.

Vagy másik példaként, a Budapest School iskoláiban a \emph{közösség} maga hozza
a működéséhez szükséges szabályokat, folyamatosan alakítja és fejleszti saját
működését a tagok aktív részvételével. Ez az aktív állampolgárságra, a
demokráciára való nevelés Nemzeti alaptantervben előírt céljait is támogatja.

\begin{table}
  \centering
  \begin{tabular}{p{5cm}|p{3cm}}

    \textbf{NAT kulcskompetenciái}                      &
    \textbf{Struktúra}                                                                  \\ \hline
    Anyanyelvi kommunikáció                             & tanulási szerződés, portfólió \\ \hline
    Idegen nyelvi kommunikáció                          & idegen nyelvű\hfill\break modulok          \\ \hline
    Matematikai kompetencia                             &                               \\ \hline
    Természettudományos és technikai kompetencia        & projektek                     \\ \hline
    Digitális kompetencia                               & digitális portfóliókezelés   \\ \hline
    Szociális és állampolgári kompetencia               & saját tanulási út,
    közösség                                                                            \\ \hline
    Kezdeményezőképesség és vállalkozói kompetencia     & saját tanulási út, közösség   \\ \hline
    Esztétikai-művészeti tudatosság és kifejezőkészség  &                               \\ \hline
    A hatékony, önálló tanulás                          & saját tanulási út,
    mentorság                                                                           \\

  \end{tabular}
  \caption{A NAT kulcskompetenciáinak fejlesztését támogatják a tantárgyak és
    az iskola felépítése is.}
  \label{tbl:nat_kulcs}
\end{table}



\section{A tantárgyközi tudás- és képességterületek fejlesztésének feladata}

A Budapest School iskola elvégzi a NAT kulcskompetenciáinak fejlesztését, támogatja a NAT által meghatározott fejlesztési területek céljait, és ellátja a műveltségi területekhez rendelt fejlesztési feladatokat. A modulok többsége nem tantárgyak alapján szerveződik, így nálunk a tantárgyközi tudás az alapértelmezett.

\section{A követelmények teljesítéséhez rendelkezésre álló kötelező, továbbá ajánlott időkeret}

\Apageref{tbl:oraszamok}. oldalon található \ref{tbl:oraszamok}.~táblázat
megadja a kiemelt tantárgyakkal töltendő minimális óraszámokat két hetes egységekre. A minimális óraszámokat mindenképpen biztosítani kell. 
Ezenfelül a gyerekek számára ajánlott minél többet foglalkozni azzal a tématerülettel, amit ők szeretnek vagy szükségük van rá.

\section{Kötött és kötetlen munkaidő szabályozása}

A Budapest School iskolákban a munkaviszonyban álló tanárok rendes vagy kötetlen munkaidőben dolgoznak.
A kerettanterv megközelítése, hogy a tanárok a kerettanterv, a saját és közösségi célok által kialakított eredmények elérésére vállalnak kötelezettséget. Ezért az Nkt. és kapcsolódó jogszabályok által előírt munkaidő-beosztási szabályoktól a jelen kerettanterv eltérést enged. Az eltérések mikroiskolánként különbözőek. Az eltérések a tanárok szerződéseiben, és az iskola megfelelő dokumentumaiban kerülnek rögzítésre.

A mikroiskola tanárcsapatának a feladata a szükséges óraszámok, beosztások kialakítása és a megfelelő modulvezetők megtalálása. Amennyiben egy tanár maga kevesebb modult tart, és több munkát tölt a tanulás megszervezésével, akár több mikroiskola tanáraként is dolgozhat egy időben.

\section{Elfogadott pedagógus végzettség és szakképzettség}

A Budapest School iskolában tanulásszervező tanárok, mentortanárok és modulvezető tanárok dolgoznak együtt, hogy a gyerekek számára megfelelő tanulási környezetet, kihívásokat, kereteket (stb.) biztosítsanak. A tanárok lehetnek alkalmazotti vagy megbízási jogviszonyban az iskolával.

Fontos alapelv, hogy a tanárok személyisége, tudása, képességei
és\linebreak
kompetenciái határozzák meg a gyerekek élményét.  Ezért az és csak az lehet tanár, aki képes segíteni a gyerekeket a tanulásban. A végzettség a tanárok hatékonyságának egyik indikátora.

\paragraph{A tanulásszervezőktől és mentoroktól elvárt képesítések} A
tanulásszervezők és a mentorok a mikroiskolák irányítói, a gyerekek tanulási környezetének alakítói és menedzserei. Elvárás, hogy a csoportos tanulási és fejlesztési helyzet vezetéséhez szükséges képességeik meglegyenek. Ehhez szükséges \aref{tbl:vegzettsegek} táblázatban felsorolt végzettségek közül valamelyik, vagy ezeket kiváltó releváns, igazolható szakmai tapasztalat.

Az érettségire felkészítő modulok esetén a tantárgyhoz tartozó szakos tanári diploma elengedhetetlen.

\begin{table}[ht]
   \centering
    \begin{tabular}{l | c}
        \textbf{elfogadott végzettség}     & \textbf{évfolyam}
        \\ \hline \hline
        drámapedagógus                     & 1--12             \\ \hline
        gyógypedagógus                     & 1--12             \\ \hline
        hittanár-nevelő tanár              & 6--12             \\ \hline
        játék- és szabadidő-szervező tanár & 6--12             \\ \hline
        kollégiumi nevelőtanár             & 6--12             \\ \hline
        konduktor                          & 1--12             \\ \hline
        mérnök tanár                       & 6--12             \\ \hline
        múzeumpedagógus                    & 1--12             \\ \hline
        nyelv- és beszédfejlesztő tanár    & 1--12             \\ \hline
        pedagógia szakos nevelő            & 1--12             \\ \hline
        pedagógia szakos tanár             & 6--12             \\ \hline
        pszichológus                       & 1--12             \\ \hline
        tantárgyi szakos tanár             & 6--12             \\ \hline
        szociálpedagógus                   & 1--12             \\ \hline
        tanulási és pályatanácsadó tanár   & 6--12             \\ \hline
        tanító                             & 1--10             \\ \hline
        tehetségfejlesztő tanár            & 6--12             \\ \hline
        Waldorf tanító                     & 1--12             \\ \hline
        óvodapedagógus                     & 1--4              \\ \hline
    \end{tabular}
    \caption{Tanulásszervező munkakörben elfogadott végzettségek.}
    \label{tbl:vegzettsegek}
\end{table}

\subsection{Modulvezetőktől elvárt végzettségek}
A modulvezetőktől azt várjuk el, hogy magas szinten ismerjék a modul tematikája által lefedett területet. Az érettségire felkészítő modulok esetén pedagógiai szakképesítést várunk el. Itt a tantárgyhoz tartozó tanári diploma elengedhetetlen.

A többi modul esetén a modul témájához kapcsolatos legalább 5 éves szakmai tapasztalat szükséges. Több tantárgy tartalmát lefedő modul esetén nem szükséges minden tantárgyhoz kapcsolatos végzettség: például egy összevont természettudományos kisérletezés modult ugyanúgy tud egy fizikus vagy egy biológus tartani.

\paragraph{Épületekre vonatkozó előírások}
\Aref{tbl:helyisegek}. táblázat meghatározza a Budapest School Általános Iskola
és Gimnázium munkájához kötelezően szükséges helyiségeket -- az Nkt. 9.§ (8) f pont felhatalmazása alapján -- a kerettantervben részletezett strukturális, szervezeti és tanulásszervezési elveket és folyamatokat, különösen a mikroiskola-hálózatos működési jellegét figyelembe véve.

\begin{spacing}{.99}
\begin{longtable}{@{}p{0.28\textwidth}|p{0.28\textwidth}|p{0.4\textwidth}@{}}

    \textbf{helyiség megnevezése} & \textbf{mennyiségi mutató}
                                  & \textbf{megjegyzés}
    \\ \hline
    tanterem                      & 16 gyerekenként egy                 &
    1,5 nm /
    gyerek előírás figyelembevételével
    \\ \hline
    tornaterem                    & iskolánként egy                     &
    kiváltható
    szerződéssel
    \\ \hline
    tornaszoba                    & székhelyenként, telephelyenként egy &
    ha a
    székhelynek v. telephelynek nincs saját tornaterme, kiváltható szerződéssel
    \\ \hline
    sportudvar                    & székhelyen, telephelyeken egy       &
    kiváltható
    szerződéssel, vagy helyettesíthető alkalmas szabad területtel
    \\ \hline
    sportszertár                  & iskolánként egy                     &
    tornateremhez kapcsolódóan (kiváltható szerződéssel)
    \\ \hline
    iskolatitkári iroda           & iskolánként egy                     &
    tantestületi szobával közös helyiségben  is kialakítható
    \\ \hline
    nevelőtestületi szoba         & iskolánként (telephelyenként) egy   &

    \\ \hline
    könyvtár                      & iskolánként egy                     &
    nyilvános
    könyvtár elláthatja a funkcót, megállapodás alapján
    \\ \hline
    orvosi szoba                  & iskolánként egy                     &
    amennyiben
    egészségügyi intézményben a gyerekek ellátása megoldható, nem kötelező
    \\
    \hline
    ebédlő                        & székhelyen, telephelyeken egy       &
    gyerek- és
    felnőttétkező közös helyiségben; tanteremmel közös helységben is kialakítható, de egyidőben csak egy funkció
    \\ \hline
    főzőkonyha                    & székhelyen, telephelyeken egy       &
    ha helyben
    főznek
    \\ \hline
    melegítőkonyha                & székhelyen, telephelyeken egy       &
    ha nem
    helyben főznek, de helyben étkeznek
    \\ \hline
    tálaló-mosogató               & székhelyen, telephelyeken egy       &
    ha nem
    helyben főznek, de helyben étkeznek; melegítőkonyhával közös helyiségben
    is
    kialakítható
    \\ \hline
    éléskamra                     & székhelyen, telephelyeken egy       &
    ha helyben
    főznek
    \\ \hline
    szárazáruraktár               & székhelyen, telephelyeken egy       &
    ha helyben
    főznek
    \\ \hline
    földesáru-raktár              & székhelyen, telephelyeken egy       &
    ha helyben
    főznek
    \\ \hline
    személyzeti WC                & székhelyen, telephelyeken egy       &

    \\ \hline
    gyerek WC                     & székhelyen, telephelyeken egy       &
    gyereklétszám figyelembevételével
    \\

    \caption{Kötelező helyiségek listája.}
    \label{tbl:helyisegek}

\end{longtable}
\end{spacing}
\newpage

\paragraph{Helyiségek bútorzata és egyéb berendezési tárgyai}

A szükséges berendezések és eszközök tekintetében a kerettanterv a	20/2012. (VIII. 31.) EMMI rendelet II.~mellékletét tekinti a fenntartó számára irányadónak, kivéve hogy
\begin{itemize}
    \item fenntartónak stabil szélessávú internethozzáférést kell biztosítania minden telephelyen;
    \item minden tanteremben minden órán elérhetőnek kell lennie legalább egy internethez kötött számítógépnek vagy tabletnek;
    \item nevelőtesületi szobában nem ,,\emph{pedagóguslétszám szerint 1 fiókos asztal és szék}'' szükséges, hanem csak amennyi a tanárok számára fontos;
    \item tantermekben nincs szükség \emph{1 nevelői asztalra és székre}, ha a tanárok tudnak a gyerekekkel együtt tevékenykedni.
\end{itemize}

Amennyiben a tanulásszervezőknek és a modulvezetőknek többletigénye merül fel, akkor a szervezeti és működési szabályzatban lefektett módon tudják a fenntartó segítségét kérni.

\chapter{Tantárgyi tanulási eredmények}
\label{sec:tantargyi_tanulasi_eredmenyek}
Az alábbi lista azokat a tantárgyi tanulási eredményeket tartalmazza, amikből a
gyerekeknek évfolyamonként legalább 50\%-ot kell teljesíteniük ahhoz, hogy a következő évfolyamszintre léhessenek. 
A tanulási eredmények a miniszter
által kiadott kerettantervek\cite{ofi:kerettanterv}  elvárt eredményeinek adatbázisba
helyezésével lettek egybegyűjtve.

A továbbiakban a \emph{gyerek} szó helyett a miniszter által kiadott
kerettantervek szóhasználata szerint a \emph{tanuló} szót alkalmazzuk.
\section{1-2. évfolyam}
\subsection{Harmónia tantárgy}
\paragraph{Erkölcstan tantárgy alapján}
\begin{itemize}
\item A tanuló életkorának megfelelő reális képpel rendelkezik saját külső tulajdonságairól, tisztában van legfontosabb személyi adataival.
\item Átlátja társas viszonyainak alapvető szerkezetét.
\item Képes érzelmileg kötődni környezetéhez és a körülötte élő emberekhez.
\item Képes kapcsolatba lépni és a partner személyét figyelembe véve kommunikálni környezete tagjaival, különféle beszédmódokat tud alkalmazni.
\item A beszélgetés során betartja az udvarias társalgás elemi szabályait.
\item Érzi a szeretetkapcsolatok fontosságát.
\item Érzi, hogy a hagyományok fontos szerepet játszanak a közösségek életében.
\item Törekszik rá, hogy érzelmeit a közösség számára elfogadható formában nyilvánítsa ki.
\item Képes átélni a természet szépségét, és érti, hogy felelősek vagyunk a körülöttünk lévő élővilágért.
\item Képes rá, hogy érzéseit, gondolatait és fantáziaképeit vizuális, mozgásos vagy szóbeli eszközökkel kifejezze.
\item Tisztában van vele és érzelmileg is képes felfogni azt a tényt, hogy más gyerekek sokszor egészen más körülmények között élnek, mint ő.
\item Meg tudja különböztetni egymástól a valóságos és a virtuális világot.
\end{itemize}
\paragraph{Testnevelés és sport tantárgy alapján}
\begin{itemize}
\item Az alapvető tartásos és mozgásos elemek felismerése, pontos végrehajtása.
\item A testrészek megnevezése.
\item A fizikai terhelés és a fáradás jeleinek felismerése.
\item A szív és az izomzat működésének elemi ismeretei.
\item Különbségtétel a jó és a rossz testtartás között álló és ülő helyzetben, a medence középhelyzete beállítása.
\item Az iskolatáska gerinckímélő hordása.
\item A gyakorláshoz szükséges térformák ismerete gyors és célszerű kialakításuk.
\item A testnevelésórák alapvető rendszabályai, a legfontosabb veszélyforrásai, balesetvédelmi szempontjai ismerete.
\item Fegyelmezett gyakorlás és odafigyelés a társakra, célszerű eszközhasználat.
\item A stressz és feszültségoldás alapgyakorlatainak használata.
\item Az alapvető hely- és helyzetváltoztató mozgások folyamatos végrehajtása egyszerű kombinációkban a tér- és energiabefektetés, valamint a mozgáskapcsolatok sokrétű felhasználásával.
\item A játékszabályok és a játékszerepek, illetve a játékfeladat alkalmazása.
\item Manipulatív természetes mozgásformák
\item A manipulatív természetes mozgásformák mozgásmintáinak végrehajtási módjainak, és a vezető műveletek tanulási szempontjainak ismerete.
\item Fejlődés az eszközök biztonságos és célszerű használatában.
\item A feladat-végrehajtások során pontosságra, célszerűségre, biztonságra törekvés.
\item A sporteszközök szabadidős használata igénnyé és örömforrássá válása.
\item Rendezett tartással, esztétikus végrehajtásra törekedve 2-4 mozgásformából álló egyszerű tornagyakorlat bemutatása.
\item Stabilitás a dinamikus és statikus egyensúlyi helyzetekben talajon és emelt eszközökön.
\item Próbálkozás a zenei ritmus követésére különféle ritmikus mozgásokban egyénileg, párban és csoportban.
\item A párhoz, társakhoz történő térbeli alkalmazkodásra törekvés tánc közben.
\item A tanult tánc(ok) és játékok térformáinak megvalósítása.
\item A futó-, ugró- és dobóiskolai alapgyakorlatok végrehajtása, azok vezető műveleteinek ismerete.
\item Különböző intenzitású és tartamú mozgások fenntartása változó körülmények között, illetve játékban.
\item Széleskörű mozgástapasztalat a Kölyökatlétika játékaiban vagy a Kölyökatlétika jellegű játékokban.
\item Az alapvető manipulatív mozgáskészségek elnevezéseinek ismerete.
\item Tapasztalat azok alkalmazásában, gyakorló és feladathelyzetben, sportjáték-előkészítő kisjátékokban.
\item A játékfeladat megoldásából és a játékfolyamatból adódó öröm, élmény és tanulási lehetőség felismerése.
\item A csapatérdek szerepének felismerése az egyéni érdekkel szemben, vagyis a közös cél fontosságának tudatosulása.
\item A sportszerű viselkedés néhány jellemzőjének ismerete.
\item Az alapvető eséstechnikák felismerése. A tompítás mozdulatának végrehajtása esés közben.
\item A mozgásának és akaratának gátlása, késleltetése.
\item Törekvés arra, hogy a támadó- és védőmozgások az ellenfél mozgásaihoz igazodjanak.
\item Az általános uszodai rendszabályok, baleset-megelőzési szempontok ismerete, és azok betartása.
\item Tudatos levegővétel.
\item Kellő vízbiztonság, hason és háton siklás és lebegés.
\item Bátor vízbeugrás.
\item Használható tudás természetben végzett testmozgások előnyeiről és problémáiról.
\item A természeti környezetben történő sportolás néhány egészségvédelmi és környezettudatos viselkedési szabályának ismerete.
\item A sporteszközök kreatív felhasználása a játéktevékenység során.
\item Az időjárási körülményeknek megfelelő öltözködés és az ahhoz kapcsolódó okok ismerete.
\end{itemize}
\paragraph{Technika, életvitel és gyakorlat tantárgy alapján}
\begin{itemize}
\item A természeti, a társadalmi és a technikai környezet megismert jellemzőinek felsorolása.
\item Tapasztalatok az ember természetátalakító (építő és romboló) munkájáról.
\item A család szerepének, időbeosztásának és egészséges munkamegosztásának megértése, káros sztereotípiák lebomlása.
\item A háztartási és közlekedési veszélyek tudatosulása, egészséges veszélyérzet.
\item Alapvető háztartási feladatok, eszközök, gépek és az ezekkel kapcsolatos veszélyforrások ismerete.
\item Példák ismerete az egészséges, korszerű táplálkozás és a célszerű öltözködés terén.
\item A hétköznapjainkban használatos anyagok felismerése, tulajdonságaik megállapítása érzékszervi megfigyelések és vizsgálatok alapján, a tapasztalatok megfogalmazása.
\item Életkori szintnek megfelelő problémafelismerés, problémamegoldás.
\item Anyagalakításhoz kapcsolódó foglalkozások megnevezése, az érintett szakmák, hivatások bemutatott jellemzőinek ismerete.
\item Célszerű takarékosság lehetőségeinek ismerete.
\item Képlékeny anyagok, papír, faanyagok, fémhuzal, szálas anyagok, textilek magabiztos alakítása.
\item Építés mintakövetéssel és önállóan.
\item Az elvégzett munkáknál alkalmazott eszközök biztonságos, balesetmentes használata.
\item A munka közbeni célszerű rend, tisztaság fenntartása.
\item Elemi higiéniai és munkaszokások szabályos gyakorlati alkalmazása.
\item Aktív részvétel, önállóság és együttműködés a tevékenységek során.
\item A közlekedési veszélyforrások tudatosulása. Az úttesten való átkelés szabályainak tudatos alkalmazása. A kulturált és balesetmentes járműhasználat (tömegközlekedési eszközökön és személygépkocsiban történő utazás) szabályainak gyakorlati alkalmazása.
\item Az alkalmakhoz illő kulturált viselkedés és öltözködés.
\end{itemize}
\subsection{STEM tantárgy}
\paragraph{Matematika tantárgy alapján}
\begin{itemize}
\item Halmazok összehasonlítása az elemek száma szerint. Halmazalkotás.
\item Állítások igazságtartalmának eldöntése. Állítások megfogalmazása.
\item Összehasonlítás, azonosítás, megkülönböztetés. Közös tulajdonság felismerése, megnevezése.
\item Több, kevesebb, ugyannyi fogalmának helyes használata.
\item Néhány elem sorba rendezése próbálgatással.
\item Számok írása, olvasása (100-as számkör). Helyi érték ismerete.
\item Római számok írása, olvasása (I, V, X).
\item Számok helye a számegyenesen. Számszomszédok értése. Természetes számok nagyság szerinti összehasonlítása.
\item Számok képzése, bontása helyi érték szerint.
\item Matematikai jelek: +, –, •, :, =, <, >, ( ) ismerete, használata.
\item Összeadás, kivonás, szorzás, osztás szóban és írásban.
\item Szorzótábla ismerete a százas számkörben.
\item A műveletek sorrendjének ismerete.
\item Szöveges feladat értelmezése, megjelenítése rajz segítségével, leírása számokkal.
\item Páros és páratlan számok megkülönböztetése.
\item Szimbólumok használata matematikai szöveg leírására, az ismeretlen szimbólum kiszámítása.
\item Növekvő és csökkenő számsorozatok szabályának felismerése, a sorozat folytatása.
\item Számpárok közötti kapcsolatok felismerése.
\item Vonalak (egyenes, görbe) ismerete.
\item A test és a síkidom megkülönböztetése.
\item Testek építése szabadon és megadott feltételek szerint.
\item Tájékozódási képesség, irányok ismerete.
\item A hosszúság, az űrtartalom, a tömeg és az idő mérése. A szabvány mértékegységek: cm, dm, m, cl, dl, l, dkg, kg, perc, óra, nap, hét, hónap, év. Átváltások szomszédos mértékegységek között. Mennyiségek közötti összefüggések felismerése. Mérőeszközök használata.
\item Adatokról megállapítások megfogalmazása.
\end{itemize}
\paragraph{Környezetismeret tantárgy alapján}
\begin{itemize}
\item Az emberi test nemre és korra jellemző arányainak leírása, a fő testrészek megnevezése. Az egészséges életmód alapvető elemeinek ismerete és alkalmazása.
\item Mesterséges és természetes életközösség összehasonlítása.
\item Az élővilág sokféleségének tisztelete, a természetvédelem fontosságának felismerése.
\item Alapvető tájékozódás az iskolában és környékén. Az évszakos és napszakos változások felismerése és kapcsolása életmódbeli szokásokhoz. Az időjárás elemeinek ismerete, az ezzel kapcsolatos piktogramok értelmezése; az időjáráshoz illő szokások.
\item Használati tárgyak és gyakori, a közvetlen környezetben előforduló anyagok csoportosítása tulajdonságaik szerint, kapcsolat felismerése az anyagi tulajdonságok és a felhasználás között. Mesterséges és természetes anyagok megkülönböztetése. A halmazállapotok felismerése.
\item Egyszerű megfigyelések végzése a természetben, egyszerű vizsgálatok és kísérletek kivitelezése. Az eredmények megfogalmazása, ábrázolása. Ok-okozati összefüggések keresésének igénye a tapasztalatok magyarázatára.
\end{itemize}
\subsection{KULT tantárgy}
\paragraph{Magyar nyelv és irodalom tantárgy alapján}
\begin{itemize}
\item A tanuló érthetően beszéljen, legyen tisztában a szóbeli kommunikáció alapvető szabályaival, alkalmazza őket.
\item Értse meg az egyszerű magyarázatokat, utasításokat és társai közléseit.
\item A kérdésekre értelmesen válaszoljon. Aktivizálja a szókincsét a szövegalkotó feladatokban. Használja a bemutatkozás, a felnőttek és a kortársak megszólításának és köszöntésének udvarias nyelvi formáit. Legyen képes összefüggő mondatok alkotására. Követhetően számoljon be élményeiről, olvasmányai tartalmáról. Szöveghűen mondja el a memoritereket.
\item Ismerje az írott és nyomtatott betűket, rendelkezzen megfelelő szókinccsel. Ismert és begyakorolt szöveget folyamatosságra, pontosságra törekvően olvasson fel. Tanítója segítségével emelje ki az olvasottak lényegét. Írása legyen rendezett, pontos.
\item Ismerje fel és nevezze meg a tanult nyelvtani fogalmakat, szükség szerint idézze fel és alkalmazza a helyesírási szabályokat a begyakorolt szókészlet szavaiban. Jelölje helyesen a j hangot 30–40 begyakorolt szóban. Helyesen válassza el az egyszerű szavakat.
\item Legyen tisztában a tanulás alapvető céljával, ítélőképessége, erkölcsi, esztétikai és történeti érzéke legyen az életkori sajátosságoknak megfelelően fejlett. Legyen nyitott, motivált az anyanyelvi képességek fejlesztése területén.
\end{itemize}
\paragraph{Ének-zene  b változat tantárgy alapján}
\begin{itemize}
\item A tanulók képesek 60 gyermekdalt és népdalt emlékezetből, a kapcsolódó játékokkal, c’–d” hangterjedelemben előadni.
\item Képesek kifejezően, egységes hangzással, tiszta intonációval énekelni, és új dalokat megfelelő előkészítést követően, hallás után megtanulni.
\item Kreatívan vesznek részt a generatív játékokban és feladatokban. Érzik az egyenletes lüktetést, tartják a tempót, érzékelik a tempóváltozást. A 2/4-es metrumot helyesen hangsúlyozzák.
\item Felismerik és lejegyzik a tanult zenei elemeket (ritmus, dallam).
\item Pontosan, folyamatosan szólaltatják meg a tanult ritmikai elemeket tartalmazó ritmusgyakorlatokat csoportosan és egyénileg is.
\item Szolmizálva éneklik a tanult dalok stílusában megszerkesztett rövid dallamfordulatokat kézjelről, betűkottáról és hangjegyről. Megfelelő előkészítés után hasonló dallamfordulatokat rögtönöznek.
\item Képesek társaikkal együtt figyelmesen zenét hallgatni. Megfigyelésekkel tapasztalatokat szereznek, melyek az egyszerű zenei elemzés alapjául szolgálnak. Különbséget tesznek az eltérő zenei karakterek között.
\item Fejlődik hangszínhallásuk. Megkülönböztetik a furulya, citera, zongora, hegedű, fuvola, fagott, gitár, dob, triangulum, réztányér, a testhangszerek és a gyermek-, női, férfihang hangszínét. Ismerik a hangszerek alapvető jellegzetességeit. Szívesen és örömmel hallgatják újra, és dramatizálják a meghallgatott zeneműveket.
\end{itemize}
\paragraph{Vizuális kultúra tantárgy alapján}
\begin{itemize}
\item Képzelőerő, belső képalkotás fejlődése.
\item Az életkornak megfelelő, felismerhető ábrázolás készítése.
\item Az alkotótevékenységnek megfelelő, rendeltetésszerű és biztonságos anyag- és eszközhasználat, a környezettudatosság szempontjainak egyre szélesebb körű figyelembevételével.
\item A felszerelés önálló rendben tartása.
\item A közvetlen környezet megfigyelése és értelmezése.
\item A képalkotó tevékenységek közül személyes kifejező alkotások létrehozása.
\item Téralkotó feladatok során a személyes térbeli szükségletek felismerése.
\item Alkotótevékenység és látványok, műalkotások szemlélése során néhány forma, szín, vonal, térbeli hely és irány, felismerése, használatára.
\item A szobor, festmény, tárgy, épület közötti különbségek felismerése.
\item Hagyományos kézműves technikával készült tárgyak megkülönböztetése.
\item Látványok, műalkotások néhány perces szemlélése.
\item Médiumok azonosítása, igény kialakítása a médiahasználat során a tudatosabb választásra, illetve reflektív médiahasználat.
\item Médiaélmények változásának és médiatapasztalattá alakíthatóságának felismerése.
\item A médiaszövegek néhány elemi kódjának a (kép, hang, cselekmény) azonosítása, illetve ezzel kapcsolatos egyszerű összefüggések felismerése (pl. médiaszövegek emberek által mesterségesen előállított tartalmak, kreatív kifejező eszközöket használnak, amelyek befolyásolják azok hatását).
\item A személyes kommunikáció és a közvetett kommunikáció közötti alapvető különbségek felismerése.
\item Az életkorhoz igazodó internetes tevékenységek gyakorlása és az abban rejlő veszélyek felismerése.
\item Az alkotó és befogadó tevékenység során a saját érzések felismerése, és azok kifejezése.
\end{itemize}
\section{3-4. évfolyam}
\subsection{Harmónia tantárgy}
\paragraph{Erkölcstan tantárgy alapján}
\begin{itemize}
\item A tanulónak életkorának megfelelő szinten reális képe van saját külső és belső tulajdonságairól, és késztetést érez arra, hogy fejlessze önmagát.
\item Oda tud figyelni másokra, szavakkal is ki tudja fejezni érzéseit és gondolatait, be tud kapcsolódni csoportos beszélgetésbe.
\item Képes másokkal tartós kapcsolatot kialakítani, törekszik e kapcsolatok ápolására, és ismer olyan eljárásokat, amelyek segítségével a kapcsolati konfliktusok konstruktív módon feloldhatók.
\item Érzelmileg kötődik a magyar kultúrához.
\item Érti és elfogadja, hogy az emberek sokfélék és sokfélék a szokásaik, a hagyományaik is; kész arra, hogy ezt a tényt tiszteletben tartsa, és kíváncsi a sajátjától eltérő kulturális jelenségekre.
\item Érti, hogy mi a jelentősége a szabályoknak a közösségek életében, kész a megértett szabályok betartására, részt tud venni szabályok kialakításában.
\item Érti, hogy a Föld mindannyiunk közös otthona, és közös kincsünk számos olyan érték, amit elődeink hoztak létre.
\item Képes a körülötte zajló eseményekre és a különféle helyzetekre a sajátjától eltérő nézőpontból is rátekinteni.
\item Képes jelenségeket, eseményeket és helyzeteket erkölcsi nézőpontból értékelni.
\end{itemize}
\paragraph{Testnevelés és sport tantárgy alapján}
\begin{itemize}
\item Egyszerű, általános bemelegítő gyakorlatok végrehajtása önállóan, zenére is. Önálló pulzusmérés.
\item A levezetés helyének és preventív szerepének megértése.
\item A nyújtó, erősítő, ernyesztő és légzőgyakorlatok pozitív hatásainak ismerete.
\item A gyakorláshoz szükséges egyszerűbb alakzatok, térformák önálló kialakítása.
\item Az alapvető tartásos és mozgásos elemek önálló bemutatása.
\item A testnevelésórán megfelelő cipő és öltözet, tisztálkodás igényének megszilárdulása.
\item A játékok, versenyek során erősödő személyes felelősség a magatartási szabályrendszer betartásában és a sportszerűen viselkedés terén.
\item Stressz- és feszültségoldó gyakorlatok megismerése.
\item Az alapvető hely- és helyzetváltoztató mozgások célszerű, folyamatos és magabiztos végrehajtása.
\item Az alapvető hely- és helyzetváltoztató mozgások kombinálása térben, szabályozott energiabefektetéssel, eszközzel, társakkal.
\item A bonyolultabb játékfeladatok, a játékszerepek és játékszabályok alkalmazása.
\item A természetes hely- és helyzetváltoztató mozgások megnevezése, valamint azok mozgástanulási szempontjainak (vezető műveletek) ismerete.
\item A manipulatív természetes mozgásformák célszerű, folyamatos és magabiztos végrehajtása.
\item A manipulatív természetes mozgásformák kombinálása térben és szabályozott energiabefektetéssel.
\item A rendszeres gyakorlás és siker mellett az önálló tanulás és fejlődés pozitív élményének megerősödése.
\item A pontosság, célszerűség és biztonság igénnyé válása.
\item A sporteszközök szabadidős használatának állandósulása.
\item Részben önállóan tervezett 3-6 torna- és/vagy táncelem összekötése zenére is.
\item A képességszintnek megfelelő magasságú eszközökre helyes technikával történő fel- és leugrás.
\item Nyújtott karral történő támasz a támaszugrások során.
\item Gurulások, átfordulások, fordulatok, dinamikus kar-, törzs- és lábgyakorlatok közben többnyire biztosan uralt egyensúlyi helyzet.
\item A tempóváltozások érzékelése és követése.
\item A tanult táncok, dalok, játékok eredeti közösségi funkciójának ismerete.
\item A futó-, ugró- és dobóiskolai gyakorlatok vezető műveleteinek ismerete, rendezettségre törekvő végrehajtása, változó körülmények között.
\item A 3 lépéses dobóritmus ismerete.
\item A különböző intenzitású és tartamú mozgások fenntartása játékos körülmények között, illetve játékban.
\item Tartós futás egyéni tempóban, akár járások közbeiktatásával is.
\item A Kölyökatlétika eszköz- és versenyrendszerének ismerete.
\item A Kölyökatlétikával kapcsolatos élmények kifejezése.
\item A tanult sportjátékok alapszabályainak ismerete.
\item Az alaptechnikai elemek ismerete és azok alkalmazása az előkészítő játékok során.
\item Az egyszerű védő és a támadó helyzetek felismerése.
\item Az emberfogásos és a területvédekezés megkülönböztetése.
\item Törekvés a legcélszerűbb játékhelyzet-megoldásra.
\item A csapatérdeknek megfelelő összjátékra törekvés.
\item A sportszerű viselkedés értékké válása.
\item Néhány önvédelmi fogás bemutatása párban.
\item Előre, oldalra és hátra esés, tompítással.
\item A tolások, húzások, emelések és hordások erőkifejtésének optimalizálódása.
\item A grundbirkózás alapszabályának ismerete és betartása.
\item A sportszerű küzdésre, az asszertív viselkedésre törekvés.
\item A saját agresszió kezelése.
\item Az önvédelmi feladatok céljának megértése.
\item Egy úszásnemben 25 méter leúszása biztonságosan.
\item Fejesugrással vízbe ugrás.
\item Az uszodai rendszabályok természetessé válása.
\item A tanult úszásnem fogalmi készletének ismerete.
\item Legalább négy szabadidős mozgásforma és alapszabályainak ismerete.
\item A tanult szabadidős mozgásformák sporteszközei biztonságos használatának, alaptechnikai és taktikai elemeinek ismerete, alkalmazása.
\item A szabadidős mozgásformák önszervező módon történő felhasználása szabad játéktevékenység során.
\item A szabadtéren, illetve speciális környezetben történő sportolással együtt járó veszélyforrások ismerete.
\end{itemize}
\paragraph{Technika, életvitel és gyakorlat tantárgy alapján}
\begin{itemize}
\item Mindennapokban nélkülözhetetlen praktikus ismeretek – háztartási praktikák – elsajátítása és begyakorlása.
\item Használati utasítások értő olvasása, betartása.
\item Sikerélmények (a felfedezés és önálló próbálkozás öröme, a motiváló hatás érvényesülése tárgyalkotáskor).
\item A hétköznapjainkban használatos anyagok felismerése, tulajdonságaik megállapítása érzékszervi megfigyelések és vizsgálatok alapján, a tapasztalatok megfogalmazása.
\item Egyszerű tárgyak elkészítése mintakövetéssel.
\item Munkaeszközök célszerű megválasztása és szakszerű, balesetmentes használata.
\item A gyalogosokra vonatkozó közlekedési szabályok tudatos készségszintű alkalmazása.
\item A kerékpár használatához szükséges ismeretek elsajátítása.
\item Aktív részvétel, önállóság és együttműködés a tevékenységek során.
\item Elemi higiéniai és munkaszokások szabályos gyakorlati alkalmazása.
\end{itemize}
\subsection{STEM tantárgy}
\paragraph{Matematika tantárgy alapján}
\begin{itemize}
\item Adott tulajdonságú elemek halmazba rendezése.
\item Halmazba tartozó elemek közös tulajdonságainak felismerése, megnevezése.
\item Annak eldöntése, hogy egy elem beletartozik-e egy adott halmazba.
\item A változás értelmezése egyszerű matematikai tartalmú szövegben.
\item Az összes eset megtalálása (próbálgatással).
\item Számok írása, olvasása (10 000-es számkör). Helyi érték, alaki érték, valódi érték fogalma 10 000-es számkörben.
\item Negatív számok a mindennapi életben (hőmérséklet, adósság).
\item Törtek a mindennapi életben: 2, 3, 4, 10, 100 nevezőjű törtek megnevezése, lejegyzése szöveggel, előállítása hajtogatással, nyírással, rajzzal, színezéssel.
\item Természetes számok nagyság szerinti összehasonlítása 10 000-es számkörben.
\item Mennyiségek közötti összefüggések észrevétele tevékenységekben.
\item A matematika különböző területein az ésszerű becslés és a kerekítés alkalmazása.
\item Fejben számolás százas számkörben.
\item A szorzótábla biztos ismerete 100-as számkörben.
\item Fejben számolás 10 000-ig nullákra végződő egyszerű esetekben.
\item Összeg, különbség, szorzat, hányados fogalmának ismerete. Műveletek tulajdonságainak, tagok, illetve tényezők felcserélhetőségének alkalmazása. Műveleti sorrend ismerete, alkalmazása.
\item Négyjegyű számok összeadása, kivonása, szorzás egy- és kétjegyű, osztás egyjegyű számmal írásban.
\item Műveletek ellenőrzése.
\item Szöveges feladat: a szöveg értelmezése, adatok kigyűjtése, megoldási terv, becslés, ellenőrzés, az eredmény realitásának vizsgálata.
\item Többszörös, osztó, maradék fogalmának ismerete.
\item Szabályfelismerés, szabálykövetés. Növekvő és csökkenő számsorozatok felismerése, készítése.
\item Összefüggések keresése az egyszerű sorozatok elemei között.
\item A szabály megfogalmazása egyszerű formában, a hiányzó elemek pótlása.
\item Egyenesek kölcsönös helyzetének felismerése: metsző és párhuzamos egyenesek.
\item A szabvány mértékegységek: mm, km, ml, cl, hl, g, t, másodperc. Átváltások szomszédos mértékegységek között.
\item Hosszúság, távolság és idő mérése (egyszerű gyakorlati példák).
\item Háromszög, négyzet, téglalap, sokszög létrehozása egyszerű módszerekkel, felismerésük, jellemzőik.
\item Kör fogalmának tapasztalati ismerete.
\item A test és a síkidom közötti különbség megértése.
\item Kocka, téglatest, felismerése, létrehozása, jellemzői.
\item Gömb felismerése.
\item Tükrös alakzatok és tengelyes szimmetria előállítása hajtogatással, nyírással, rajzzal, színezéssel.
\item Négyzet, téglalap kerülete.
\item Négyzet, téglalap területének mérése különféle egységekkel, területlefedéssel.
\item Tapasztalati adatok lejegyzése, táblázatba rendezése. Táblázat adatainak értelmezése.
\item Adatgyűjtés, adatok lejegyzése, diagram leolvasása.
\item Valószínűségi játékok, kísérletek értelmezése. Biztos, lehetetlen, lehet, de nem biztos tapasztalati ismerete.
\item Tanári segítséggel az életkorának megfelelő oktatási célú programok használata.
\item Egy rajzoló program ismerte; egyszerű ábrák elkészítése, színezése.
\item Együttműködés interaktív tábla használatánál.
\end{itemize}
\paragraph{Környezetismeret tantárgy alapján}
\begin{itemize}
\item Az egészséges életmód alapvető elemeinek alkalmazása az egészségmegőrzés és az egészséges fejlődés érdekében, a betegségek elkerülésére.
\item Az életkornak megfelelően a helyzethez illő felelős viselkedés segítségnyújtást igénylő helyzetekben.
\item A hosszúság és idő mérése, a mindennapi életben előforduló távolságok és időtartamok becslése.
\item Képesség adott szempontú megfigyelések végzésére a természetben, természeti jelenségek egyszerű kísérleti tanulmányozására.
\item A fenntartható életmód jelentőségének magyarázata konkrét példán keresztül, és a hagyományok szerepének értelmezése a természeti környezettel való harmonikus kapcsolat kialakításában, illetve felépítésében.
\item Az élőlények szerveződési szintjeinek és az életközösségek kapcsolatainak a bemutatása, az élőlények csoportosítása tetszőleges és adott szempontsor szerint.
\item Egy természetes életközösség bemutatása.
\item Egy konkrét gyártási folyamat kapcsán a technológiai folyamat értelmezése, az ezzel kapcsolatos felelős fogyasztói magatartás ismerete.
\item Magyarország elhelyezése a földrajzi térben, néhány fő kulturális és természeti értékének ismerete.
\item Informatikai és kommunikációs eszközök irányított használata az információkeresésben és a problémák megoldásában.
\end{itemize}
\subsection{KULT tantárgy}
\paragraph{Magyar nyelv és irodalom tantárgy alapján}
\begin{itemize}
\item A tanuló értelmesen és érthetően fejezze ki a gondolatait. Aktivizálja a szókincsét a szövegalkotó feladatokban. Használja a mindennapi érintkezésben az udvarias nyelvi fordulatokat. Beszédstílusát igazítsa beszélgető partneréhez.
\item Kapcsolódjon be a csoportos beszélgetésbe, vitába, történetalkotásba, improvizációba, közös élményekről, tevékenységekről való beszélgetésekbe, értékelésbe. A közös tevékenységeket együttműködő magatartással segítse.
\item Felkészülés után folyamatosan, érthetően olvasson fel ismert szöveget. Életkorának megfelelő szöveget értsen meg néma olvasás útján. Az olvasottakkal kapcsolatos véleményét értelmesen fogalmazza meg. Ismerjen, alkalmazzon néhány fontos olvasási stratégiát.
\item Tanulási tevékenységét fokozatosan növekvő időtartamban legyen képes tudatos figyelemmel irányítani. Feladatainak megoldásához szükség szerint vegye igénybe az iskola könyvtárát. A könyvekben, gyermekújságokban a tartalomjegyzék segítségével igazodjon el. Használja az ismert kézikönyveket.
\item A memoritereket szöveghűen mondja el.
\item Adott vagy választott témáról 8–10 mondatos fogalmazást készítsen a tanult fogalmazási ismeretek alkalmazásával.
\item Az alsó tagozaton tanult anyanyelvi ismeretei rendszerezettek legyenek. Biztonsággal ismerje fel a tanult szófajokat, és nevezze meg azokat szövegben is.
\item A begyakorolt szókészlet körében helyesen alkalmazza a tanult helyesírási szabályokat. Írásbeli munkái rendezettek, olvashatóak legyenek. Helyesírását önellenőrzéssel vizsgálja felül, szükség esetén javítsa.
\item Legyen tisztában a tanulás alapvető céljával, ítélőképessége, erkölcsi, esztétikai és történeti érzéke legyen az életkori sajátosságoknak megfelelően fejlett. Legyen nyitott, motivált az anyanyelvi képességek fejlesztése területén. Az anyanyelvi részképességeinek fejlettsége legyen összhangban, harmonikusan fejlődjön.
\end{itemize}
\paragraph{Idegen nyelv tantárgy alapján}
\begin{itemize}
\item A tanuló aktívan részt vesz a célnyelvi tevékenységekben, követi a célnyelvi óravezetést, az egyszerű tanári utasításokat, megérti az egyszerű, ismerős kérdéseket, válaszol ezekre, kiszűri egyszerű, rövid szövegek lényegét.
\item Elmond néhány verset, mondókát és néhány összefüggő mondatot önmagáról, minta alapján egyszerű párbeszédet folytat társaival.
\item Ismert szavakat, rövid szövegeket elolvas és megért jól ismert témában.
\item Tanult szavakat, ismerős mondatokat lemásol, minta alapján egyszerű, rövid szövegeket alkot.
\end{itemize}
\paragraph{Ének-zene
  b változat tantárgy alapján}
\begin{itemize}
\item Képesek a tanulók 60 népdalt, műzenei idézetet emlékezetből, a-e’ hangterjedelemben csoportosan, a népdalokat több versszakkal, csokorba rendezve is előadni.
\item Képesek kifejezően, egységes hangzással, tiszta intonációval énekelni, és új dalokat megfelelő előkészítést követően, hallás után és jelrendszerről megtanulni.
\item Többszólamú éneklési készségük fejlődik. Képesek csoportosan ismert dalokat ritmus osztinátóval énekelni, és egyszerű kétszólamú darabokat, kánonokat megszólaltatni.
\item Kreatívan vesznek részt a generatív játékokban és feladatokban. Érzik az egyenletes lüktetést, tartják a tempót, érzékelik a tempóváltozást. A 3/4-es és 4/4-es metrumot helyesen hangsúlyozzák.
\item Felismerik, lejegyzik és megszólaltatják a tanult zenei elemeket (metrum, dinamikai jelzések, ritmus, dallam).
\item Az ismert dalokat olvassák kézjelről, betűkottáról, hangjegyről és emlékezetből szolmizálják.
\item Szolmizálva éneklik a tanult dalok stílusában megszerkesztett rövid dallamfordulatokat kézjelről, betűkottáról és hangjegyről. Megfelelő előkészítés után hasonló dallamfordulatokat rögtönöznek.
\item Fejlődik zenei memóriájuk és belső hallásuk.
\item Fejlődik formaérzékük, a formai építkezés jelenségeit (azonosság, hasonlóság, különbözőség) felismerik és meg tudják fogalmazni.
\item Fejlődik hangszínhallásuk. Megkülönböztetik a tekerő, duda, oboa, klarinét, kürt, trombita, üstdob, a gyermekkar, nőikar, férfikar, vegyeskar hangszínét. Ismerik a hangszerek alapvető jellegzetességeit. Különbséget tesznek szóló és kórus, szólóhangszer és zenekar hangzása között.
\item Zenehallgatásuk tudatos, figyelmesen hallgatják a zeneműveket. A zenehallgatásra kiválasztott művek közül 4-5 alkotást/műrészletet ismernek.
\end{itemize}
\paragraph{Vizuális kultúra tantárgy alapján}
\begin{itemize}
\item Az alkotásra, megfigyelésre, elemzésre vonatkozó feladatok életkornak megfelelő értelmezése.
\item Élmény- és emlékkifejezés, illusztráció készítése; síkbáb és egyszerű jelmez készítése; jelek, ábrák készítése; egyszerű tárgyak alkotása.
\item Az újként megismert anyagok és eszközök, technikák az alkotótevékenységnek megfelelő, rendeltetésszerű és biztonságos anyag- és eszközhasználata.
\item A legismertebb formák, színek, vonalak, térbeli helyek és irányok, illetve komponálási módok használata, látványok, műalkotások olvasásába is beépítve.
\item Téralkotó feladatok során a személyes preferenciáknak és a funkciónak megfelelő térbeli szükségletek felismerése.
\item A szobrászati, festészeti, tárgyművészeti, építészeti területek közötti különbségek további differenciálása (pl. festészeten belül: arckép, csendélet, tájkép).
\item Látványok, műalkotások megfigyeléseinek során kialakult gondolatok, érzések elmondására a tantervben meghatározott legfontosabb fogalmak használatával, az életkornak megfelelően.
\item Különböző típusú médiaszövegek felismerése, a médiatartalmak közötti tudatos választás.
\item A médiaszövegekhez használt egyszerű kódok, kreatív kifejezőeszközök és azok érzelmi hatásának felismerése.
\item Kép- és hangrögzítő eszközök használata elemi technikáinak ismerete. Az elsajátított kifejezőeszközök segítségével saját gondolatok, érzések megfogalmazása, rövid, egyszerű történet megformálása.
\item A médiaszövegek előállításával, nyelvi jellemzőivel, használatával kapcsolatos alapfogalmak elsajátítása, helyes alkalmazása élőszóban.
\item A média alapvető funkcióinak (tájékoztatás, szórakoztatás, ismeretszerzés) megismerése.
\item A médiaszövegekben megjelenő információk valóságtartalmának felismerése.
\item Az életkorhoz igazodó biztonságos internet- és mobilhasználat szabályainak ismerete, alkalmazása. A hálózati kommunikációban való részvétel során fontos és szükséges viselkedési szabályok elsajátítása, alkalmazása. Életkorhoz igazodó fejlesztő, kreatív internetes tevékenységek megismerése.
\end{itemize}
\section{5-6. évfolyam}
\subsection{Harmónia tantárgy}
\paragraph{Erkölcstan tantárgy alapján}
\begin{itemize}
\item A tanuló tisztában van az egészség megőrzésének jelentőségével, és tudja, hogy maga is felelős ezért.
\item Tudatában van annak, hogy az emberek sokfélék,  elfogadja és értékeli a testi és lelki vonásokban megnyilvánuló sokszínűséget, valamint az etnikai és kulturális különbségeket.
\item Gondolkodik saját személyiségjegyein, törekszik a megalapozott véleményalkotásra, illetve vélekedéseinek és tetteinek utólagos értékelésére.
\item Gondolkodik rajta, hogy mit tekint értéknek; tudja, hogy ez befolyásolja a döntéseit, és hogy időnként választania kell még a számára fontos értékek között is.
\item Képes különféle szintű kapcsolatok kialakítására és ápolására; átlátja saját kapcsolati hálójának a szerkezetét; rendelkezik a konfliktusok kezelésének és az elkövetett hibák kijavításának néhány, a gyakorlatban jól használható technikájával.
\item Fontos számára a közösséghez való tartozás érzése; képes átlátni és elfogadni a közösségi normákat.
\item Nyitottan fogadja a sajátjától eltérő véleményeket, szokásokat és kulturális, illetve vallási hagyományokat.
\item Érzékeli, hogy a társadalom tagjai különféle körülmények között élnek, képes együttérzést mutatni az elesettek iránt, és lehetőségéhez mérten szerepet vállal a rászorulók segítésében. Megbecsüli a neki nyújtott segítséget.
\item Tisztában van azzal, hogy az emberi tevékenység hatással van a környezet állapotára, és törekszik rá, hogy életvitelével minél kevésbé károsítsa a természetet.
\item Ismeri a modern technika legfontosabb előnyeit és hátrányait, s felismeri magán a függőség kialakulásának esetleges előjeleit.
\item Tisztában van vele, hogy a reklámok a nézők befolyásolására törekszenek, és kritikusan viszonyul a különféle médiaüzenetekhez.
\item Érti, hogy a világ megismerésének többféle útja van (különböző világképek és világnézetek), s ezek mindegyike a maga sajátos eszközeivel közelít ugyanahhoz a valósághoz.
\end{itemize}
\paragraph{Testnevelés és sport tantárgy alapján}
\begin{itemize}
\item A gyakorláshoz szükséges alakzatok öntevékeny gyors kialakítása.
\item Öntevékeny részvétel a szervezési feladatok végrehajtásában.
\item A bemelegítésre, a sokoldalú előkészítésre, valamint a képességfejlesztésre alkalmas mozgásformák, gyakorlatok folyamatos, pontosságra törekvő és megfelelő intenzitású végrehajtása.
\item nyolc-tíz gyakorlattal részben önálló bemelegítés végrehajtása.
\item A tanévben alkalmazott gimnasztika alapvető szakkifejezéseinek ismerete.
\item A testtartásért felelős izmok tudatos, koncentratív fejlesztése.
\item A biomechanikailag helyes testtartás megőrzése.
\item Stressz- és feszültségoldó módszerek alkalmazása és a feszültségek önálló szabályozása.
\item A bemelegítés és a levezetés szempontjainak ismerete.
\item Választott úszásnemben készségszintű, egy másikban 25 méteren vízbiztos, folyamatos úszás.
\item Fejesugrás és folyamatos taposás a mélyvízben.
\item Az úszással összefüggő balesetvédelmi utasítások, valamint az uszoda, fürdő viselkedési szabályainak betartása.
\item Ismeretek az úszástudás, a vízbiztonság szerepéről az egészség megőrzésében és az életvédelemben.
\item Külső visszajelzés információinak elfogadása és hasznosítása a különböző úszásnemek gyakorlásánál.
\item A sportjátékok technikai és taktikai készletének elsajátítása, ezek alkalmazása testnevelési játékokban, játékos feladatokban és a sportjátékban.
\item Törekvés a játékelemek (technikai, taktikai elemek) pontos, eredményes végrehajtására és tudatos kontrollálására.
\item A sportjátékok játékszabályainak ismerete és alkalmazása.
\item Szabálykövető magatartás, önfegyelem, együttműködés kinyilvánítása a sportjátékokban.
\item Részvétel a kedvelt sportjátékban a tanórán kívüli sportfoglalkozásokon vagy egyéb szervezeti formában.
\item A tanult futó-, ugró-, dobógyakorlatok jártasság szintű elsajátítása.
\item A rajtok végrehajtása az indítási jeleknek megfelelően.
\item A vágta és a tartósfutás technikájának végrehajtása a mozgásmintának megfelelően.
\item Ugrásoknál a nekifutás távolságának és sebességének kialakítása tapasztalatok felhasználásával.
\item A kislabda-hajító technika képességeknek megfelelő elsajátítása.
\item A kar- és láblendítés szerepének ismerete az el- és felugrások eredményességében.
\item Az atlétikai versenyek alapvető szabályainak ismerete.
\item A testtömeg uralása nem szokványos támaszhelyzetekben és támaszgyakorlatokban – szükség esetén segítségadás mellett.
\item A tanult akadályleküzdési módok és feladatok biztonságos végrehajtása.
\item A dinamikus és statikus egyensúlygyakorlatok végrehajtása a képességnek megfelelő magasságon, szükség esetén segítségadás mellett.
\item Talaj-, illetve gerendagyakorlat részben önálló összeállítása.
\item Az aerobik alaplépések összekapcsolása egyszerű kartartásokkal és kargyakorlatokkal.
\item Az alaplépésekből 2-−4 ütemű gyakorlat végrehajtása helyben és haladással, zenére is.
\item A ritmikus sportgimnasztika egyszerű tartásos és mozgásos gyakorlatelemeinek bemutatása.
\item A járások, ritmizált lépések, futások és szökdelések technikailag megközelítően helyes végrehajtása.
\item A gyakorlatvégzések során előforduló hibák elismerése és a javítási megoldások elfogadása.
\item A balesetvédelmi utasítások betartása.
\item Alternatív környezetben űzhető sportok
\item A tanult alternatív környezetben űzhető sportágak alaptechnikai gyakorlatainak bemutatása.
\item A sportágak űzéséhez szükséges eszközök biztonságos használata.
\item A természeti és környezeti hatások és a szervezet alkalmazkodó képessége közötti összefüggés ismerete.
\item A természeti környezetben történő sportolás egészségvédelmi és környezettudatos viselkedési szabályainak elfogadása és betartása.
\item A mostoha időjárási feltételek mellett is aktív részvétel a foglalkozásokon.
\end{itemize}
\paragraph{Technika, életvitel és gyakorlat tantárgy alapján}
\begin{itemize}
\item Tapasztalatok megfogalmazása a környezet elemeiről, állapotáról, a környezetátalakító tevékenységgel járó felelősség belátása.
\item Tapasztalatok az ételkészítéssel, élelmiszerekkel összefüggő munkatevékenységekről.
\item Ételkészítés és tárgyalkotás során a technológiák helyes alkalmazása, eszközök szakszerű, biztonságos használata.
\item Elemi műszaki rajzi ismeretek alkalmazása a tervezés és a kivitelezés során.
\item Az elkészült produktumok (ételek, tárgyak, modellek) reális értékelése, a hibák felismerése, a javítás, fejlesztés lehetőségeinek meghatározása.
\item Az ember közvetlen tárgyi környezetének megőrzésére, alakítására vonatkozó szükségletek felismerése, a tevékenységek és beavatkozások következményeinek előzetes, helyes felismerése, az azzal járó felelősség belátása.
\item A tárgyi környezetben végzett tevékenységek biztonságossá, környezettudatossá, takarékossá és célszerűvé válása.
\item A gyalogos és kerékpáros közlekedés KRESZ szerinti szabályainak, valamint a tömegközlekedés szabályainak biztonságos alkalmazása.
\item A kerékpár karbantartásához szükséges ismeretek elsajátítása.
\item A vasúti közlekedésben való biztonságos és udvarias részvétel.
\item Tájékozódás közúti és vasúti menetrendekben, útvonaltérképeken. Útvonalterv olvasása, készítése.
\end{itemize}
\paragraph{Informatika tantárgy alapján}
\begin{itemize}
\item ismerje a számítógép részeinek és perifériáinak funkcióit, tudja azokat önállóan használni;
\item tudjon a könyvtárszerkezetben tájékozódni, mozogni, könyvtárat váltani, fájlt keresni;
\item tudjon segítséggel használni multimédiás oktatóprogramokat;
\item tudjon az iskolai hálózatba belépni, onnan kilépni, ismerje és tartsa be a hálózat használatának szabályait;
\item ismerje egy vírusellenőrző program kezelését.
\item ismerje a szövegszerkesztés alapfogalmait, legyen képes önállóan elvégezni a leggyakoribb karakter- és bekezdésformázásokat;
\item használja a szövegszerkesztő nyelvi segédeszközeit;
\item ismerje egy bemutatókészítő program egyszerű lehetőségeit, tudjon rövid bemutatót készíteni;
\item ismerje fel az összetartozó adatok közötti egyszerű összefüggéseket;
\item segítséggel tudjon használni tantárgyi, könyvtári, hálózati adatbázisokat, tudjon különféle adatbázisokban keresni;
\item tudjon különböző dokumentumokból származó részleteket saját munkájában elhelyezni.
\item legyen képes összegyűjteni a problémamegoldáshoz szükséges információt;
\item ismerje a problémamegoldás alapvető lépéseit;
\item képes legyen önállóan vagy segítséggel algoritmust készíteni;
\item tudjon egyszerű programot készíteni;
\item legyen képes egy fejlesztőrendszer alapszintű használatára;
\item a problémamegoldás során legyen képes együttműködni társaival.
\item legyen képes a böngészőprogram főbb funkcióinak használatára;
\item legyen képes tanári segítséggel megadott szempontok szerint információt keresni;
\item legyen képes a találatok értelmezésére;
\item legyen képes az elektronikus levelezőrendszer önálló kezelésére;
\item legyen képes elektronikus és internetes médiumok használatára;
\item legyen képes az interneten talált információk mentésére;
\item ismerje a netikett szabályait.
\item ismerje az informatikai biztonsággal kapcsolatos fogalmakat;
\item ismerje az adatvédelemmel kapcsolatos fogalmakat;
\item ismerje az adatvédelem érdekében alkalmazható lehetőségeket;
\item ismerje az informatikai eszközök etikus használatára vonatkozó szabályokat;
\item szerezzen gyakorlatot az információforrások saját dokumentumokban való feltüntetésében.
\item a különböző konkrét tantárgyi feladataihoz képes az iskolai könyvtárban a megadott forrásokat megtalálni, és további releváns forrásokat keresni;
\item konkrét nyomtatott és elektronikus forrásokban képes megkeresni a megoldáshoz szükséges információkat;
\item el tudja dönteni, mikor vegye igénybe az iskolai vagy a lakóhelyi könyvtár szolgáltatásait.
\end{itemize}
\subsection{STEM tantárgy}
\paragraph{Matematika tantárgy alapján}
\begin{itemize}
\item Halmazba rendezés adott tulajdonság alapján, részhalmaz felírása, felismerése.
\item Két véges halmaz közös részének, illetve uniójának felírása, ábrázolása.
\item Néhány elem kiválasztása adott szempont szerint.
\item Néhány elem sorba rendezése különféle módszerekkel.
\item Állítások igazságának eldöntése, igaz és hamis állítások megfogalmazása.
\item Összehasonlításhoz szükséges kifejezések helyes használata.
\item Néhány elem összes sorrendjének felírása.
\item Racionális számok írása, olvasása, összehasonlítása, ábrázolása számegyenesen.
\item Ellentett, abszolút érték, reciprok felírása.
\item Mérés, mértékegységek használata, átváltás egyszerű esetekben.
\item A mindennapi életben felmerülő egyszerű arányossági feladatok megoldása következtetéssel, az egyenes arányosság felismerése, használata.
\item Két-három műveletet tartalmazó műveletsor eredményének kiszámítása, a műveleti sorrendre vonatkozó szabályok ismerete, alkalmazása. Zárójelek alkalmazása.
\item Szöveges feladatok megoldása következtetéssel (az adatok közötti összefüggések felírása szimbólumokkal).
\item Becslés, ellenőrzés segítségével a kapott eredmények helyességének megítélése.
\item A százalék fogalmának ismerete, a százalékérték kiszámítása.
\item Számok osztóinak, többszöröseinek felírása. Közös osztók, közös többszörösök kiválasztása. Oszthatósági szabályok (2, 3, 5, 9, 10, 100) ismerete, alkalmazása.
\item A hosszúság, terület, térfogat, űrtartalom, idő, tömeg szabványmértékegységeinek ismerete. Mértékegységek egyszerűbb átváltásai gyakorlati feladatokban. Algebrai kifejezések gyakorlati használata a terület, kerület, felszín és térfogat számítása során.
\item Elsőfokú egyismeretlenes egyenletek, egyenlőtlenségek megoldása szabadon választott módszerrel.
\item Tájékozódás a koordinátarendszerben: pont ábrázolása, adott pont koordinátáinak a leolvasása.
\item Egyszerűbb grafikonok, elemzése.
\item Egyszerű sorozatok folytatása adott szabály szerint, szabályok felismerése, megfogalmazása néhány tagjával elkezdett sorozat esetén.
\item Térelemek, félegyenes, szakasz, szögtartomány, sík, fogalmának ismerete.
\item A geometriai ismeretek segítségével a feltételeknek megfelelő ábrák pontos szerkesztése. A körző, vonalzó célszerű használata.
\item Alapszerkesztések: pont és egyenes távolsága, két párhuzamos egyenes távolsága, szakaszfelező merőleges, szögfelező, szögmásolás, merőleges és párhuzamos egyenesek.
\item Alakzatok tengelyese tükörképének szerkesztése, tengelyes szimmetria felismerése.
\item A tanult síkbeli és térbeli alakzatok tulajdonságainak ismerete és alkalmazása feladatok megoldásában.
\item Téglalap és a deltoid kerületének és területének kiszámítása.
\item A téglatest felszínének és térfogatának kiszámítása.
\item A tanult testek térfogatszámítási módjának ismeretében mindennapjainkban található testek térfogatának, űrmértékének meghatározása.
\item Egyszerű diagramok készítése, értelmezése, táblázatok olvasása.
\item Néhány szám számtani közepének kiszámítása.
\item Valószínűségi játékok, kísérletek során adatok tervszerű gyűjtése, rendezése, ábrázolása.
\end{itemize}
\paragraph{Természetismeret tantárgy alapján}
\begin{itemize}
\item A tanuló tudjon anyagokat, kölcsönhatásokat, fizikai, kémiai változásokat felismerni, jellemezni. Értelmezze a jelenségeket az energiaváltozás szempontjából
\item Ismerje az emberi szervezet felépítését, működését, serdülőkori változásait és okait. Tudatosuljanak az egészséget veszélyeztető hatások, alapozódjon meg az egészséges életvitel szokásrendszere.
\item Formálódjon reális énképe, értse a családi és a társas kapcsolatok jelentőségét, élete irányításában kapjon döntő szerepet az erkölcsi értékrendnek való megfelelés. Legyen embertársaival empatikus és segítőkész.
\item Ismerje a Föld helyét a Világegyetemben, Magyarország helyét Európában.
\item Alakuljon ki átfogó kép hazai tájaink természetföldrajzi jellemzőiről, természeti-társadalmi erőforrásairól, gazdasági folyamatairól, környezeti állapotukról. Legyen képe a közöttük levő kölcsönhatásokról.
\item Ismerje hazánk legjellemzőbb életközösségeit, termesztett növényeit, a házban és ház körül élő állatait. Értse az élő és élettelen környezeti tényezők kölcsönhatását. Ismerje fel a környezet-szervezet-életmód, valamint a szervek felépítése és működése közötti összefüggéseket.
\item Tudjon tájékozódni a térképeken. Értelmezze helyesen a különböző tartalmú térképek jelrendszerét, használja fel az információszerzés folyamatában.
\item Erősödjön a természet és a haza iránti szeretete. Törekedjen a természeti és társadalmi értékek védelmére.
\item Ismerje fel szűkebb és tágabb környezetében az emberi tevékenység környezeti hatásait. Anyag- és energiatakarékos életvitelével, tudatos vásárlási szokásaival önmaga is járuljon hozzá a fenntartható fejlődéshez.
\item Legyen képes egyszerű kísérleteket, megfigyeléseket, méréseket önállóan, illetve. csoportban biztonságosan elvégezni, a tapasztalatokat rögzíteni, következtetéséket levonni.
\item Legyen nyitott, érdeklődő a világ megismerése iránt. Az internet és a könyvtár segítségével legyen képes tudása bővítésére. Legyenek saját ismeretszerzési, ismeretfeldolgozási módszerei.
\end{itemize}
\subsection{KULT tantárgy}
\paragraph{Magyar nyelv és irodalom
	b változat tantárgy alapján}
\begin{itemize}
\item A tanuló törekszik gondolatait érthetően, a helyzetnek megfelelően megfogalmazni, adekvátan alkalmazni a beszédet kísérő nem nyelvi jeleket. Képes rövidebb szóbeli üzenetek, rövidebb hallott történetek megértésére, összefoglalására, továbbadására.
\item Ismeri és alkalmazni tudja a legalapvetőbb anyaggyűjtési, vázlatkészítési módokat. Képes önállóan a tanult hagyományos és internetes műfajokban (elbeszélés, leírás, jellemzés, levél, SMS, e-mail stb.) szöveget alkotni. Törekszik az igényes, pontos és helyes fogalmazásra, írásra.
\item Képes az írott és elektronikus felületen megjelenő olvasott szövegek globális (átfogó) megértésére, a szövegből az információk visszakeresése mellett képes újabb és újabb szövegértési stratégiákat megismerni, azokat alkalmazni. Képes önálló feladatvégzésre az információgyűjtés és ismeretszerzés módszereinek alkalmazásával (kézikönyvek és korosztálynak szóló ismeretterjesztő források).
\item Felismeri a szövegértés folyamatát, annak megfigyelésével képes saját módszerét fejleszteni, a hibás olvasási szokásaira megfelelő javító stratégiát találni, és azt alkalmazni.
\item A tanuló ismeri a tanult alapszófajok (ige, főnév, melléknév, számnév, határozószó, igenevek, névmások), valamint az igekötők általános jellemzőit, alaki sajátosságait, a hozzájuk kapcsolódó főbb helyesírási szabályokat, amelyeket az írott munkáiban igyekszik alkalmazni is.
\item A megismert új szavakat, közmondásokat, szólásokat próbálja aktív szókincsében is alkalmazni.
\item A tanuló meg tud nevezni három mesetípust példákkal, és fel tud idézni címe vagy részlete említésével három népdalt. Különbséget tud tenni a népmese és a műmese között. Meg tudja fogalmazni, mi a különbség a mese és a monda között. El tudja különíteni a rímes, ritmikus szöveget a prózától. Meg tudja nevezni, melyik műnem mond el történetet, melyik jelenít meg konfliktust párbeszédes formában, és melyik fejez ki érzést, élményt. Felismeri a hexameteres szövegről, hogy az időmértékes, a felező tizenkettesről, hogy az ütemhangsúlyos. Fel tud sorolni három-négy művet Petőfitől és Aranytól, képes egyszerűbb összehasonlítást megfogalmazni János vitéz és Toldi Miklós alakjáról. Képes értelmezni A walesi bárdokban rejlő allegóriát, és meg tudja világítani 5–6 mondatban az Egri csillagok történelmi hátterét. El tudja különíteni az egyszerűbb versekben és prózai szövegekben a nagyobb szerkezeti egységeket. Össze tudja foglalni néhány hosszabb mű cselekményét (János vitéz, Toldi, A Pál utcai fiúk, Egri csillagok), meg tudja különböztetni, melyik közülük a regény és melyik az elbeszélő költemény. Értelmesen és pontosan, tisztán, tagoltan, megfelelő ritmusban tud felolvasni szövegeket. Részt tud venni számára ismert témájú vitában, és képes érveket alkotni. Ismert és könnyen érthető történetben párosítani tudja annak egyes szakaszait a konfliktus, bonyodalom, tetőpont fogalmával. Képes az általa jól ismert történetek szereplőit jellemezni, kapcsolatrendszerüket feltárni. Képes néhány példa közül kiválasztani az egyszerűbb metaforákat és metonímiákat. Képes egyszerűbb meghatározást megfogalmazni a következő fogalmakról: líra, epika, dráma, epizód, megszemélyesítés, ballada. Képes néhány egyszerűbb meghatározás közül kiválasztani azt, amely a következő fogalmak valamelyikéhez illik: dal, rím, ritmus, mítosz, motívum, konfliktus. Képes művek, műrészletek szöveghű felidézésére.
\item Az olvasott és megtárgyalt irodalmi művek nyomán képes azonosítani erkölcsi értékeket és álláspontokat, képes megfogalmazni saját erkölcsi ítéleteit.
\end{itemize}
\paragraph{Idegen nyelv tantárgy alapján}
\begin{itemize}
\item A1 szintű nyelvtudás:
\item A tanuló megérti a gazdagodó nyelvi eszközökkel megfogalmazott célnyelvi óravezetést, az ismert témákhoz kapcsolódó kérdéseket, rövid megnyilatkozásokat, szövegeket.
\item Egyszerű nyelvi eszközökkel, begyakorolt beszédfordulatokkal kommunikál.
\item Felkészülés után elmond rövid szövegeket.
\item Közös feldolgozás után megérti az egyszerű olvasott szövegek lényegét, tartalmát.
\item Ismert témáról rövid, egyszerű mondatokat ír, mintát követve önálló írott szövegeket alkot.
\end{itemize}
\paragraph{Történelem, társadalmi és állampolgári ismeretek tantárgy alapján}
\begin{itemize}
\item Az egyetemes emberi értékek elfogadása az ókori, középkori és kora újkori egyetemes és magyar kultúrkincs élményszerű megismerésével. A családhoz, a lakóhelyhez, a nemzethez való tartozás élményének személyes megélése.
\item A történelmet formáló, alapvető folyamatok, összefüggések felismerése (pl. a munka értékteremtő ereje) és egyszerű, átélhető erkölcsi tanulságok azonosítása.
\item Az előző korokban élt emberek, közösségek élet-, gondolkodás- és szokásmódjainak felidézése.
\item A tanuló ismerje fel, hogy a múltban való tájékozódást segítik a kulcsfogalmak és fogalmak, amelyek fejlesztik a történelmi tájékozódás és gondolkodás kialakulását.
\item Ismerje fel, hogy az utókor a nagy történelmi személyiségek, nemzeti hősök cselekedeteit a közösségek érdekében végzett tevékenységek szempontjából értékeli.
\item Tudja, hogy az egyes népeket főleg vallásuk és kultúrájuk, életmódjuk alapján tudjuk megkülönböztetni.
\item Ismerje fel, hogy a vallási előírások, valamint az államok által megfogalmazott szabályok döntő mértékben befolyásolhatják a társadalmi viszonyokat és a mindennapokat.
\item Tudja, hogy a történelmi jelenségeket, folyamatokat társadalmi, gazdasági tényezők együttesen befolyásolják.
\item Tudja, hogy az emberi munka nyomán elinduló termelés biztosítja az emberi közösségek létfenntartását.
\item Ismerje fel a társadalmi munkamegosztás jelentőségét, amely az árutermelés és pénzgazdálkodás, illetve a városiasodás kialakulásához vezet.
\item Tudja, hogy a társadalmakban eltérő jogokkal rendelkező és eltérő vagyoni helyzetű emberek alkotnak rétegeket, csoportokat.
\item Tudja, hogy az eredettörténetek, a közös szokások és mondák erősítik a közösség összetartozását, egyben a nemzeti öntudat kialakulásának alapjául szolgálnak.
\item Tudja, hogy a társadalmi, gazdasági, politikai és vallási küzdelmek számos esetben összekapcsolódnak.
\item Ismerje fel ezeket egy-egy történelmi probléma vagy korszak feldolgozása során.
\item Legyen képes különbséget tenni a történelem különböző típusú forrásai között, és legyen képes a korszakra jellemző képeket, tárgyakat, épületeket felismerni.
\item Legyen képes a történetek feldolgozásánál a tér- és időbeliség azonosítására, a szereplők csoportosítására fő- és mellékszereplőkre, illetve a valós és fiktív elemek megkülönböztetésére. Ismerje a híres történelmi személyiségek jellemzéséhez szükséges kulcsszavakat, cselekedeteket.
\item Legyen képes a tanuló történelmi ismeretet meríteni hallott és olvasott szövegekből, különböző médiumok anyagából.
\item Legyen képes emberi magatartásformák értelmezésre, információk rendezésére és értelmezésére, vizuális vázlatok készítésére.
\item Legyen képes információt gyűjteni adott témához könyvtárban és múzeumban; olvasmányairól készítsen lényeget kiemelő jegyzetet.
\item Képes legyen szóbeli beszámolót készíteni önálló gyűjtőmunkával szerzett ismereteiről, és kiselőadást tartani.
\item Ismerje az időszámítás alapelemeit (korszak, évszázad, évezred) tudjon ez alapján kronológiai számításokat végezni. Ismerje néhány kiemelkedő esemény időpontját.
\item Legyen képes egyszerű térképeket másolni, alaprajzot készíteni. Legyen képes helyeket megkeresni, megmutatni térképen, néhány kiemelt jelenség topográfiai helyét megjelölni vaktérképen, valamint távolságot becsülni és számítani történelmi térképen.
\item Legyen képes saját vélemény megfogalmazása mellett mások véleményének figyelembe vételére.
\end{itemize}
\paragraph{Hon- és népismeret tantárgy alapján}
\begin{itemize}
\item A tanulók megismerik lakóhelyük, szülőföldjük természeti adottságait, hagyományos gazdasági tevékenységeit, néprajzi jellemzőit, történetének nevezetesebb eseményeit, jeles személyeit. A tanulási folyamatban kialakul az egyéni, családi, közösségi, nemzeti azonosságtudatuk.
\item Általános képet kapnak a hagyományos gazdálkodó életmód fontosabb területeiről, a család felépítéséről, a családon belüli munkamegosztásról. A megszerzett ismeretek birtokában képesek lesznek értelmezni a más tantárgyakban felmerülő népismereti tartalmakat.
\item Felfedezik a jeles napok, ünnepi szokások, az emberi élet fordulóihoz kapcsolódó népszokások, valamint a társas munkák, közösségi alkalmak hagyományainak jelentőségét, közösségmegtartó szerepüket a paraszti élet rendjében. Élményszerűen, hagyományhű módon elsajátítják egy-egy jeles nap, ünnepkör köszöntő vagy színjátékszerű szokását, valamint a társas munkák, közösségi alkalmak népszokásait és a hozzájuk kapcsolódó tevékenységeket.
\item Megismerik a magyar nyelvterület földrajzi-néprajzi tájainak, tájegységeinek hon- és népismereti, néprajzi jellemzőit. Világossá válik a tanulók számára, hogyan függ össze egy táj természeti adottsága a gazdasági tevékenységekkel, a népi építészettel, hogyan élt harmonikus kapcsolatban az ember a természettel.
\end{itemize}
\paragraph{Ének-zene
  b változat tantárgy alapján}
\begin{itemize}
\item A tanulók képesek 20-22 népdalt, történeti éneket több versszakkal, valamint 8-10 műzenei idézetet emlékezetből, a–e” hangterjedelemben előadni.
\item Képesek kifejezően, egységes hangzással, tiszta intonációval énekelni, és új dalokat megfelelő előkészítést követően hallás után megtanulni.
\item Többszólamú éneklési készségük fejlődik. Képesek csoportosan egyszerű kánonokat megszólaltatni.
\item Kreatívan vesznek részt a generatív játékokban és feladatokban. Érzik az egyenletes lüktetést, tartják a tempót, érzékelik a tempóváltozást. A 6/8-os és 3/8-os metrumot helyesen hangsúlyozzák.
\item Felismerik és megszólaltatják a tanult zenei elemeket (metrum, dinamikai jelzések, ritmus, dallam).
\item Szolmizálva éneklik a tanult dalok stílusában megszerkesztett rövid dallamfordulatokat kézjelről, betűkottáról és hangjegyről. Megfelelő előkészítés után hasonló dallamfordulatokat rögtönöznek.
\item Fejlődik zenei memóriájuk és belső hallásuk.
\item Fejlődik formaérzékük, a formai építkezés jelenségeit felismerik és meg tudják fogalmazni.
\item Fejlődik hangszínhallásuk. Megkülönböztetik a tárogató, brácsa, cselló, nagybőgő, harsona, tuba, hárfa hangszínét. Ismerik a hangszerek alapvető jellegzetességeit.
\item Különbséget tesznek népi zenekar, vonósnégyes, szimfonikus zenekar hangzása között.
\item A két zenei korszakból zenehallgatásra kiválasztott művek közül 18-20 alkotást/műrészletet ismernek.
\end{itemize}
\paragraph{Vizuális kultúra tantárgy alapján}
\begin{itemize}
\item A vizuális nyelv és kifejezés eszközeinek megfelelő alkalmazása az alkotó tevékenység során a vizuális emlékezet segítségével és megfigyelés alapján.
\item Egyszerű kompozíciós alapelvek a kifejezésnek megfelelő használata a képalkotásban.
\item Térbeli és időbeli változások lehetséges vizuális megjelenéseinek értelmezése, és egyszerű mozgásélmények, időbeli változások megjelenítése.
\item A mindennapokban használt vizuális jelek értelmezése, ennek analógiájára saját jelzésrendszerek kialakítása.
\item Szöveg és kép együttes jelentésének értelmezése különböző helyzetekben és alkalmazása különböző alkotó jellegű tevékenység során.
\item Az épített és tárgyi környezet elemző megfigyelése alapján egyszerű következtetések megfogalmazása.
\item Néhány rajzi és tárgykészítési tecnika megfelelő használata az alkotótevékenység során.
\item Reflektálás társművészeti alkotásokra vizuális eszközökkel.
\item A legfontosabb művészettörténeti korok azonosítása.
\item Vizuális jelenségek, tárgyak, műalkotások elemzése során a vizuális megfigyelés pontos megfogalmazása.
\item Fontosabb szimbolikus és kultárális üzenetet közvetítő tárgyak felismerése.
\item A vizuális megfigyelés és elemzés során önálló kérdések megfogalmazása.
\item Önálló vélemény megfogalmazása saját és mások munkájáról.
\end{itemize}
\section{7-8. évfolyam}
\subsection{Harmónia tantárgy}
\paragraph{Erkölcstan tantárgy alapján}
\begin{itemize}
\item A tanuló megérti, hogy az ember egyszerre biológiai és tudatos lény, akit veleszületett képességei alkalmassá tesznek a tanulásra, mások megértésre és önmaga vizsgálatára.
\item Érti, hogy az emberek viselkedését, döntéseit tudásuk, gondolataik, érzelmeik, vágyaik, nézeteik és értékrendjük egyaránt befolyásolják.
\item Képes reflektálni saját maga és mások gondolataira, motívumaira és tetteire.
\item Életkorának megfelelő szinten ismeri önmagát, hosszabb távú elképzeléseinek kialakításakor képes reálisan felmérni a lehetőségeit.
\item Képes erkölcsi szempontok szerint mérlegelni különféle cselekedeteket, és el tudja viselni az értékek közötti választással együtt járó belső feszültséget.
\item Képes ellenállni a csoportnyomásnak, és saját értékrendje szerinti autonóm döntéseket hozni.
\item Tisztában van vele, hogy baráti- és párkapcsolataiban felelősséggel tartozik a társaiért.
\item Kialakultak benne az európai identitás csírái.
\item Nyitott más kultúrák értékeinek megismerésére és befogadására.
\item Érti a szabályok szerepét az emberi együttélésben, s e belátás alapján igyekszik alkalmazkodni hozzájuk; igényli azonban, hogy maga is alakítója lehessen a közösségi szabályoknak.
\item Van elképzelése saját jövőjéről, és tisztában van vele, hogy céljai eléréséért erőfeszítéseket kell tennie.
\item Életkorának megfelelő szinten tisztában van vele, hogy minden döntés szabadsága egyúttal felelősséggel is jár.
\item Fontosnak érzi a közösséghez tartozást, miközben törekszik személyes autonómiájának megőrzésére.
\item Képes megfogalmazni, hogy mi okoz neki örömet, illetve rossz érzést.
\item Tisztában van a függőséget okozó szokások súlyos következményeivel.
\item Tudja, hogy ugyanazt a dolgot különböző emberek eltérő módon ítélhetik meg, ami konfliktusok forrása lehet.
\end{itemize}
\paragraph{Testnevelés és sport tantárgy alapján}
\begin{itemize}
\item Gyakorlottság a célszerű óraszervezés megvalósításában.
\item Egyszerű stressz- és feszültségoldó technikákról tájékozottság.
\item Egyszerű gimnasztikai gyakorlatok önálló összefűzése és előadása zenére.
\item Az erősítés és nyújtás néhány ellenjavallt gyakorlatának ismerete.
\item Az összehangolt, rendezett testtartás kritériumainak való megfelelésre kísérletek.
\item A kamaszkori személyi higiénéről elemi tájékozottság.
\item A tanult két úszásnemben mennyiségi és minőségi teljesítményjavulás felmutatása.
\item Mellúszás az egyén adottságainak és képességeinek megfelelően.
\item A vízben mozgás prevenciós előnyeiről, a fizikai háttérről ismeretek.
\item A vízből mentés alapgyakorlatainak bemutatása.
\item Felelősségérzet, érdeklődés, segítőkészség kinyilvánítása a vizes feladatokban.
\item Gazdagabb sportjáték-technikai és -taktikai készlet.
\item Jártasság néhány taktikai formáció, helyzet megoldásában.
\item A játékszabályok kibővített körének megértése és alkalmazása.
\item A csapatjátékhoz szükséges együttműködés és kommunikáció fejlődése.
\item A sportjátékokhoz tartozó test-test elleni küzdelem megtapasztalása és elfogadása.
\item Konfliktusok, sportszerűtlenségek, deviáns magatartások esetén a gondolatok, vélemények szóban történő kulturált kifejezése.
\item Sporttörténeti alapvető tájékozottság a labdajátékokban.
\item Az atlétikai cselekvésminták sokoldalú és célszerű alkalmazása.
\item Futó-, ugró- és dobógyakorlatok képességeknek megfelelő végzése a tanult versenyszabályoknak megfelelően.
\item Mérhető fejlődés a képességekben és a sportági eredményekben.
\item Az atlétikai alapmozgásokban mozgásmintához közelítő bemutatás, a lendületszerzések és a befejező mozgások összekapcsolása.
\item A futás, a kocogás élettani jelentőségének ismerete.
\item A helyes testtartás, a koordinált mozgás és az erőközlés összhangjának jelenléte a torna jellegű mozgásokban.
\item Talajon és a választott tornaszeren növekvő önállóság jeleinek felmutatása a gyakorlásban, gyakorlat-összeállításban.
\item A szekrény- és a támaszugrások bátor végrehajtása, a képességnek megfelelő magasságon.
\item Látható fejlődés az aerobikgyakorlatok kivitelében és a zenével összhangban történő végrehajtása.
\item Önkontroll, együttműködés és segítségnyújtás a torna jellegű gyakorlatok végrehajtásában.
\item A helyben választott rekreációs célú sportágakban és népi hagyományokra épülő sportolási formákban bővülő gyakorlási tapasztalat és fellelhető erősebb belső motiváció némelyik területén.
\item Az egészséges életmóddal kapcsolatos ismeretek kinyilvánítása.
\item A természeti erők és a sport hasznos összekapcsolásának ismerete és az ezzel kapcsolatos előnyök, rutinok területén jártasság.
\item A környezettudatosság cselekedetekben való megjelenítése.
\item A verbális és nem verbális kommunikáció fejlődése a testkultúra hagyományos és újszerű mozgásanyagainak elsajátításában.
\item A szabadidőben végzett sportolás iránti pozitív beállítódás felmutatása.
\item A grundbirkózás alaptechnikájának, szabályainak gyakorlatban történő alkalmazása.
\item A különböző eséstechnikák, szabadulások, leszorítások és az önvédelmi gyakorlatainak kontrollált végrehajtása társsal.
\item A fenyegetettségi szituációkra, segítségkérésre, menekülésre vonatkozó ismeretek elsajátítása.
\item A sportszerű győzni akarás kinyilvánítása.
\item A fájdalomtűrésben és az önfegyelemben érzékelhető fejlődés.
\end{itemize}
\paragraph{Technika, életvitel és gyakorlat tantárgy alapján}
\begin{itemize}
\item Az egészséges, biztonságos, környezettudatos otthon működtetéséhez szükséges praktikus életvezetési ismeretek elsajátítása, készségek kialakulása.
\item A háztartás elektromos, víz-, szennyvíz-, gáz- és más tüzelőberendezéseinek biztonságos kezelése, takarékos és felelős használata, a használattal járó veszélyek és környezeti hatások tudatosulása, hibák, működészavarok felismerése. Egyszerű karbantartási, javítási munkák önálló elvégzése.
\item Környezettudatosság a háztartási hulladékok kezelése során.
\item A víz- és energiafogyasztással, hulladékokkal kapcsolatos mennyiségek és költségek érzékelésének, becslésének képessége.
\item Elköteleződés a takarékos életvitel és a környezetkímélő technológiák mellett.
\item A kerékpárosokra vonatkozó közlekedési szabályok tudatos készségszintű alkalmazása.
\item Tájékozottság a közlekedési környezetben.
\item Tudatos közlekedési magatartás.
\item A közlekedési morál alkalmazása.
\item Környezettudatos közlekedésszemlélet.
\item Alapvető tájékozottság a továbbtanulási lehetőségekről, elképzelés a saját felnőttkori életről, pályaválasztási lehetőségek mérlegelése.
\item Tapasztalatok, ismeretek, véleményalkotás a meglátogatott munkahelyekről, ezek összevetése a személyes tervekkel.
\item Az adottságok, képességek, igények, lehetőségek összhangjának keresése.
\item A munkatevékenységnek az önmegvalósítás részeként történő értékelése.
\item A munkába álláshoz szükséges alapkészségek és ismeretek elsajátítása.
\end{itemize}
\paragraph{Informatika tantárgy alapján}
\begin{itemize}
\item ismerje meg a különböző informatikai környezeteket;
\item tudja használni az operációs rendszer és a számítógépes hálózat alapszolgáltatásait;
\item segítséggel legyen képes az adott feladat megoldásához alkalmas hardver- és szoftvereszközök kiválasztására.
\item tudjon dokumentumokba különböző objektumokat beilleszteni;
\item tudjon szöveget, képet és táblázatot is tartalmazó dokumentumot minta vagy leírás alapján elkészíteni;
\item tudjon egyszerű táblázatot létrehozni;
\item ismerje a diagramok szerkesztésének, módosításának lépéseit;
\item tudjon bemutatót készíteni.
\item lássa át a problémamegoldás folyamatát;
\item ismerje és használja az algoritmusleíró eszközöket;
\item ismerje egy programozási nyelv alapszintű utasításait;
\item tudjon kódolni algoritmusokat;
\item tudjon egyszerű vezérlési feladatokat megoldani fejlesztői környezetben;
\item ismerjen és alkalmazzon tervezési eljárásokat;
\item legyen képes meghatározni az eredményt a bemenő adatok alapján;
\item legyen képes tantárgyi szimulációs programok használatára.
\item legyen képes megkeresni a kívánt információt;
\item legyen képes az információ értékelésére;
\item legyen képes előkészíteni az információt weben történő publikálásra;
\item tudja megkülönböztetni a publikussá tehető és védendő adatait;
\item használja a legújabb infokommunikációs technológiákat, szolgáltatásokat.
\item ismerje az informatikai biztonsággal és adatvédelemmel kapcsolatos fogalmakat;
\item ismerje az adatokkal való visszaélésekből származó veszélyeket és következményeket;
\item ismerjen megbízható információforrásokat;
\item legyen képes értékelni az információ hitelességét;
\item ismerje az informatikai eszközök etikus használatára vonatkozó szabályokat;
\item ismerje az információforrások etikus felhasználási lehetőségeit;
\item ismerje fel az informatikai eszközök használatának az emberi kapcsolatokra vonatkozó következményeit;
\item ismerjen néhány elektronikus szolgáltatást;
\item legyen képes a szolgáltatások igénybevételére, használatára, lemondására.
\item a könyvtár és az internet szolgáltatásait igénybe véve képes önállóan releváns forrásokat találni konkrét tantárgyi feladataihoz;
\item a választott forrásokat képes alkotóan és etikusan felhasználni a feladatmegoldásban;
\item képes alkalmazni a más tárgyakban tanultakat (pl. informatikai eszközök használata, szövegalkotás);
\item egyszerű témában képes az információs problémamegoldás folyamatát önállóan végrehajtani.
\end{itemize}
\subsection{STEM tantárgy}
\paragraph{Matematika tantárgy alapján}
\begin{itemize}
\item Elemek halmazba rendezése több szempont alapján.
\item Egyszerű állítások igaz vagy hamis voltának eldöntése, állítások tagadása.
\item Állítások, feltételezések, választások világos, érthető közlésének képessége, szövegek értelmezése egyszerűbb esetekben.
\item Kombinatorikai feladatok megoldása az összes eset szisztematikus összeszámlálásával.
\item Fagráfok használata feladatmegoldások során.
\item Biztos számolási ismeretek a racionális számkörben. A műveleti sorrendre, zárójelezésre vonatkozó szabályok ismerete, helyes alkalmazása. Az eredmény becslése, ellenőrzése., helyes és értelmes kerekítése.
\item Mérés, mértékegység használata, átváltás. Egyenes arányosság, fordított arányosság.
\item A százalékszámítás alapfogalmainak ismerete, a tanult összefüggések alkalmazása feladatmegoldás során.
\item A legnagyobb közös osztó kiválasztása az összes osztóból, a legkisebb pozitív közös többszörös kiválasztása a többszörösök közül.
\item Prímszám, összetett szám. Prímtényezős felbontás.
\item Egyszerű algebrai egész kifejezések helyettesítési értéke. Összevonás. Többtagú kifejezés szorzása egytagúval.
\item Négyzetre emelés, négyzetgyökvonás, hatványozás pozitív egész kitevők esetén.
\item Elsőfokú egyismeretlenes egyenletek és egyenlőtlenségek. A matematikából és a mindennapi életből vett egyszerű szöveges feladatok megoldása következtetéssel, egyenlettel. Ellenőrzés. A megoldás ábrázolása számegyenesen.
\item A betűkifejezések és az azokkal végzett műveletek alkalmazása matematikai, természettudományos és hétköznapi feladatok megoldásában.
\item Számológép ésszerű használata a számolás megkönnyítésére.
\item Megadott sorozatok folytatása adott szabály szerint.
\item Az egyenes arányosság grafikonjának felismerése, a lineáris kapcsolatokról tanultak alkalmazása természettudományos feladatokban is.
\item Grafikonok elemzései a tanult szempontok szerint, grafikonok készítése, grafikonokról adatokat leolvasása. Táblázatok adatainak kiolvasása, értelmezése, ábrázolása különböző típusú grafikonon.
\item A tanuló a geometriai ismeretek segítségével jó ábrák készítése, pontos szerkesztések végzése.
\item A tanult geometriai alakzatok tulajdonságainak ismerete (háromszögek, négyszögek belső és külső szögeinek összege, nevezetes négyszögek szimmetriatulajdonságai), ezek alkalmazása a feladatok megoldásában.
\item Tengelyes és középpontos tükörkép, eltolt alakzat képének szerkesztése. Kicsinyítés és nagyítás felismerése hétköznapi helyzetekben (szerkesztés nélkül).
\item A Pitagorasz-tételt kimondása és alkalmazása számítási feladatokban.
\item Háromszögek, speciális négyszögek és a kör kerületének, területének számítása feladatokban.
\item A tanult testek (háromszög és négyszög alapú egyenes hasáb, forgáshenger) térfogatképleteinek ismeretében a mindennapjainkban előforduló testek térfogatának, űrtartalmának kiszámítása.
\item Valószínűségi kísérletek eredményeinek értelmes lejegyzése, relatív gyakoriságok kiszámítása.
\item Konkrét feladatokban az esély, illetve valószínűség fogalmának értése, a biztos és a lehetetlen esemény felismerése.
\item Zsebszámológép célszerű használata statisztikai számításokban.
\item Néhány kiemelkedő magyar matematikus nevének ismerete, esetenként kutatási területének, eredményének megnevezés
\end{itemize}
\paragraph{Biológia-egészségtan
	b változat tantárgy alapján}
\begin{itemize}
\item A tanuló ismerje Magyarország legfontosabb nemzeti parkjait és a lakóhelyén vagy annak közelében található természeti értékeket (védett növények és védett természeti értékek).
\item Legyen tisztában a környezet-egészségvédelem alapjaival, a gyógy- és fűszernövényeknek a szervezetre gyakorolt hatásával.
\item Tudja, hogy milyen szerepe van a biológiai információnak az önfenntartásban és fajfenntartásban.
\item Értse a család szerepének biológiai és társadalmi jelentőségét.
\item Értse, hogy a párkapcsolatokból adódnak konfliktushelyzetek, és legyen kész azokat megfelelő módszerekkel kezelni.
\item Tudja a tanult nem sejtes és sejtes élőlényeket összekapcsolni az emberi szervezet működésével, értelmezze azokat az élőlények és környezetük egymásra hatásaként.
\item Legyen tisztában saját szervezete működésének alapjaival.
\item Értse és tudja bizonyítékokkal alátámasztani, hogy az élővilág különböző megjelenési formáit a különböző élőhelyekhez való alkalmazkodás alakította ki.
\item Legyen világos számára, hogy az ember a természet része, és ennek megfelelően cselekedjen.
\item Tudja, hogy az életmóddal nagymértékben befolyásolhatjuk szervezetünk egészséges működését. Tekintse az egészséget testi, lelki szociális jóllétnek.
\item Kerülje az egészséget veszélyeztető anyagok használatát, tevékenységeket.
\item Tudjon sérültet, beteget alapvető elsősegélynyújtásban részesíteni.
\item Empátiával viszonyuljon beteg és fogyatékkal élő társaihoz.
\item Tudjon egyszerű kísérleteket, vizsgálódásokat elvégezni, csoportmunkában és önállóan infokommunikációs eszközök segítségével beszámolókat készíteni, szemléltető anyagot összeállítani, adatokat elemezni és valós problémákra megoldásokat javasolni. Tanári irányítással tudjon projektmunkát végezni.
\end{itemize}
\paragraph{Fizika
	b változat tantárgy alapján}
\begin{itemize}
\item A tanuló használja a számítógépet adatrögzítésre, információgyűjtésre.
\item Eredményeiről tartson pontosabb, a szakszerű fogalmak tudatos alkalmazására törekvő, ábrákkal, irodalmi hivatkozásokkal stb. alátámasztott prezentációt.
\item Ismerje fel, hogy a természettudományos tények megismételhető megfigyelésekből, célszerűen tervezett kísérletekből nyert bizonyítékokon alapulnak.
\item Váljon igényévé az önálló ismeretszerzés.
\item Legalább egy tudományos elmélet esetén kövesse végig, hogy a társadalmi és történelmi háttér hogyan befolyásolta annak kialakulását és fejlődését.
\item Használja fel ismereteit saját egészségének védelmére.
\item Legyen képes a mások által kifejtett véleményeket megérteni, értékelni, azokkal szemben kulturáltan vitatkozni.
\item A kísérletek elemzése során alakuljon ki kritikus szemléletmódja, egészséges szkepticizmusa. Tudja, hogy ismeretei és használati készségei meglévő szintjén további tanulással túl tud lépni.
\item Ítélje meg, hogy különböző esetekben milyen módon alkalmazható a tudomány és a technika, értékelje azok előnyeit és hátrányait az egyén, a közösség és a környezet szempontjából. Törekedjék a természet- és környezetvédelmi problémák enyhítésére.
\item Legyen képes egyszerű megfigyelési, mérési folyamatok megtervezésére, tudományos ismeretek megszerzéséhez célzott kísérletek elvégzésére.
\item Legyen képes ábrák, adatsorok elemzéséből tanári irányítás alapján egyszerűbb összefüggések felismerésére. Megfigyelései során használjon modelleket.
\item Legyen képes egyszerű arányossági kapcsolatokat matematikai és grafikus formában is lejegyezni. Az eredmények elemzése után vonjon le konklúziókat.
\item Ismerje fel a fény szerepének elsőrendű fontosságát az emberi tudás gyarapításában, ismerje a fényjelenségeken alapuló kutatóeszközöket, a fény alapvető tulajdonságait.
\item Képes legyen a sebesség fogalmát különböző kontextusokban is alkalmazni.
\item Tudja, hogy a testek közötti kölcsönhatás során a sebességük és a tömegük egyaránt fontos, és ezt konkrét példákon el tudja mondani.
\item Értse meg, hogy a gravitációs erő egy adott testre hat és a Föld (vagy más égitest) vonzása okozza.
\item A tanuló magyarázataiban legyen képes az energiaátalakulások elemzésére, a hőmennyiséghez kapcsolódásuk megvilágítására. Tudja használni az energiafajták elnevezését. Ismerje fel a hőmennyiség cseréjének és a hőmérséklet kiegyenlítésének kapcsolatát.
\item Fel tudjon sorolni többféle energiaforrást, ismerje alkalmazásuk környezeti hatásait. Tanúsítson környezettudatos magatartást, takarékoskodjon az energiával.
\item A tanuló minél több energiaátalakítási lehetőséget ismerjen meg, és képes legyen azokat azonosítani. Tudja értelmezni a megújuló és a nem megújuló energiafajták közötti különbséget.
\item A tanuló képes legyen arra, hogy az egyes energiaátalakítási lehetőségek előnyeit, hátrányait és alkalmazásuk kockázatait elemezze, tényeket és adatokat gyűjtsön, vita során az érveket és az ellenérveket csoportosítsa és azokat a vita során felhasználja.
\item Képes legyen a nyomás fogalmának értelmezésére és kiszámítására egyszerű esetekben az erő és a felület hányadosaként.
\item Tudja, hogy nem csak a szilárd testek fejtenek ki nyomást.
\item Tudja magyarázni a gázok nyomását a részecskeképpel.
\item Tudja, hogy az áramlások oka a nyomáskülönbség.
\item Tudja, hogy a hang miként keletkezik, és hogy a részecskék sűrűségének változásával terjed a közegben.
\item Tudja, hogy a hang terjedési sebessége gázokban a legkisebb és szilárd anyagokban a legnagyobb.
\item Ismerje az áramkör részeit, képes legyen egyszerű áramkörök összeállítására, és azokban az áramerősség mérésére.
\item Tudja, hogy az áramforrások kvantitatív jellemzője a feszültség.
\item Tudja, hogy az elektromos fogyasztó elektromos energiát használ fel, alakít át.
\item A tanuló képes legyen az erőművek alapvető szerkezét bemutatni.
\item Tudja, hogy az elektromos energia bármilyen módon történő előállítása terheli a környezetet.
\end{itemize}
\paragraph{Kémia
		b változat tantárgy alapján}
\begin{itemize}
\item A tanuló ismerje a kémia egyszerűbb alapfogalmait (atom, kémiai és fizikai változás, elem, vegyület, keverék, halmazállapot, molekula, anyagmennyiség, tömegszázalék, kémiai egyenlet, égés, oxidáció, redukció, sav, lúg, kémhatás), alaptörvényeit, vizsgálati céljait, módszereit és kísérleti eszközeit, a mérgező anyagok jelzéseit.
\item Ismerje néhány, a hétköznapi élet szempontjából jelentős szervetlen és szerves vegyület tulajdonságait, egyszerűbb esetben ezen anyagok előállítását és a mindennapokban előforduló anyagok biztonságos felhasználásának módjait.
\item Tudja, hogy a kémia a társadalom és a gazdaság fejlődésében fontos szerepet játszik.
\item Értse a kémia sajátos jelrendszerét, a periódusos rendszer és a vegyértékelektron-szerkezet kapcsolatát, egyszerű vegyületek elektronszerkezeti képletét, a tanult modellek és a valóság kapcsolatát.
\item Értse és az elsajátított fogalmak, a tanult törvények segítségével tudja magyarázni a halmazállapotok jellemzőinek, illetve a tanult elemek és vegyületek viselkedésének alapvető különbségeit, az egyes kísérletek során tapasztalt jelenségeket.
\item Tudjon egy kémiával kapcsolatos témáról önállóan vagy csoportban dolgozva információt keresni, és tudja ennek eredményét másoknak változatos módszerekkel, az infokommunikációs technológia eszközeit is alkalmazva bemutatni.
\item Alkalmazza a megismert törvényszerűségeket egyszerűbb, a hétköznapi élethez is kapcsolódó problémák, kémiai számítási feladatok megoldása során, illetve gyakorlati szempontból jelentős kémiai reakciók egyenleteinek leírásában.
\item Használja a megismert egyszerű modelleket a mindennapi életben előforduló, a kémiával kapcsolatos jelenségek elemzéseskor.
\item Megszerzett tudását alkalmazva hozzon felelős döntéseket a saját életével, egészségével kapcsolatos kérdésekben, vállaljon szerepet személyes környezetének megóvásában.
\end{itemize}
\paragraph{Földrajz tantárgy alapján}
\begin{itemize}
\item A tanulók átfogó és reális képzettel rendelkezzenek a Föld egészéről és annak kisebb-nagyobb egységeiről (a földrészekről és a világtengerről, a kontinensek karakteres nagytájairól és tipikus tájairól, valamint a világgazdaságban kiemelkedő jelentőségű országcsoportjairól, országairól). Legyen átfogó ismeretük földrészünk, azon belül a meghatározó és a hazánkkal szomszédos országok természet- és társadalomföldrajzi sajátosságairól, lássak azok térbeli és történelmi összefüggéseit, érzékeljék a földrajzi tényezők életmódot meghatározó szerepét. Birtokoljanak reális ismereteket a Kárpát-medencében fekvő hazánk földrajzi jellemzőiről, erőforrásairól és az ország gazdasági lehetőségeiről az Európai Unió keretében. Legyenek tisztában az Európai Unió meghatározó szerepével, jelentőségével.
\item Ismerjék fel a földrajzi övezetesség kialakulásában megnyilvánuló összefüggéseket és törtvényszerűségeket. Legyenek képesek alapvető összefüggések, tendenciák felismerésére és megfogalmazására az egyes földrészekre vagy országcsoportokra, tájakra jellemző természeti jelenségekkel, társadalmi-gazdasági folyamatokkal kapcsolatban, ismerjék fel az egyes országok, országcsoportok helyét a világ társadalmi-gazdasági folyamataiban. Érzékeljék az egyes térségek, országok társadalmi-gazdasági adottságai jelentőségének időbeli változásait. Ismerjék fel a globalizáció érvényesülését regionális példákban. Ismerjék hazánk társadalmi-gazdasági fejlődésének jellemzőit összefüggésben a természeti erőforrásokkal. Értsék, hogy a hazai gazdasági, társadalmi és környezeti folyamatok világméretű vagy regionális folyamatokkal függenek össze.
\item Tudják példákkal bizonyítani a társadalmi-gazdasági folyamatok környezetkárosító hatását, a lokális problémák globális következmények elvének érvényesülését. Legyenek tisztában a Földet fenyegető veszélyekkel, értsék a fenntarthatóság lényegét példák alapján, ismerjék fel, hogy a Föld sorsa a saját magatartásukon is múlik.
\item Rendelkezzenek a tanulók valós képzetekkel a környezeti elemek méreteiről, a számszerűen kifejezhető adatok és az időbeli változások nagyságrendjéről. Tudjanak nagy vonalakban tájékozódni a földtörténeti időben. Legyenek képesek természet-, illetve társadalom- és gazdaságföldrajzi megfigyelések végzésére, a különböző nyomtatott és elektronikus információhordozókból földrajzi tartalmú információk gyűjtésére, összegzésére, a lényeges elemek kiemelésére. Ezek során alkalmazzák digitális ismereteiket. Legyenek képesek megadott szempontok alapján bemutatni földrajzi öveket, földrészeket, országokat és tipikus tájakat.
\item Legyenek képesek a tanulók a térképet információforrásként használni, szerezzék meg a logikai térképolvasás képességét. A topográfiai ismereteikhez tudjanak földrajzi-környezeti tartalmakat kapcsolni. Topográfiai tudásuk alapján a tanulók biztonsággal tájékozódjanak a köznapi életben a földrajzi térben, illetve a térképeken, és alkalmazzák topográfiai tudásukat más tantárgyak tanulása során is.
\item Legyenek képesek a társakkal való együttműködésre. Alakuljon ki bennük az igény arra, hogy későbbi életük folyamán önállóan tovább gyarapítsák földrajzi ismereteiket.
\end{itemize}
\subsection{KULT tantárgy}
\paragraph{Magyar nyelv és irodalom
	b változat tantárgy alapján}
\begin{itemize}
\item A tanuló képes a kulturált szociális érintkezésre, eligazodik és hatékonyan részt vesz a mindennapi páros és csoportos kommunikációs helyzetekben, vitákban. Figyeli és tudja értelmezni partnerei kommunikációs szándékát, nem nyelvi jeleit.
\item Képes érzelmeit kifejezni, álláspontját megfelelő érvek, bizonyítékok segítségével megvédeni, ugyanakkor empatikusan képes beleélni magát mások gondolatvilágába, érzelmeibe, megérti mások cselekvésének mozgatórugóit.
\item Képes a különböző megjelenésű és műfajú szövegek globális (átfogó) megértésére, a szöveg szó szerinti jelentésén túli üzenet értelmezésére, a szövegből információk visszakeresésére.
\item Össze tudja foglalni a szöveg tartalmát, tud önállóan jegyzetet és vázlatot készíteni. Képes az olvasott szöveg tartalmával kapcsolatos saját véleményét szóban és írásban megfogalmazni, állításait indokolni.
\item Ismeri és a törekszik a szövegalkotásban a különböző mondatfajták használatára. Alkalmazza az írásbeli szövegalkotásban a mondatvégi, a tagmondatok, illetve mondatrészek közötti írásjeleket. A helyesírási segédkönyvek segítségével jártas az összetett szavak és gyakoribb mozaikszók helyesírásában.
\item Ismeri a tömegkommunikáció fogalmát, legjellemzőbb területeit.
\item Képes a könnyebben besorolható művek műfaji azonosítására, 8-10 műfajt műnemekbe tud sorolni, és a műnemek lényegét meg tudja fogalmazni. A különböző regénytípusok műfaji jegyeit felismeri, a szereplőket jellemezni tudja, a konfliktusok mibenlétét fel tudja tárni. Felismeri az alapvető lírai műfajok sajátosságait különböző korok alkotóinak művei alapján (elsősorban 19-20. századi alkotások). Felismeri néhány lírai mű beszédhelyzetét, a megszólító-megszólított viszony néhány jellegzetes típusát, azonosítja a művek tematikáját, meghatározó motívumait. Felfedez műfaji és tematikus-motivikus kapcsolatokat, azonosítja a zenei és ritmikai eszközök típusait, felismeri funkciójukat, hangulati hatásukat. Azonosít képeket, alakzatokat, szókincsbeli és mondattani jellegzetességeket, a lexika jelentésteremtő szerepét megérti a lírai szövegekben, megismeri a kompozíció meghatározó elemeit (pl. tematikus szerkezet, tér- és időszerkezet, logikai szerkezet, beszédhelyzet és változása). Konkrét szövegpéldán meg tudja mutatni a mindentudó és a tárgyilagos elbeszélői szerep különbözőségét, továbbá a közvetett és a közvetlen elbeszélésmód eltérését. Képes a drámákban, filmekben megjelenő emberi kapcsolatok, cselekedetek, érzelmi viszonyulások, konfliktusok összetettségének értelmezésére és megvitatására. Az olvasott, megtárgyalt művek erkölcsi kérdésfeltevéseire véleményében, erkölcsi ítéleteiben, érveiben tud támaszkodni.
\item Képes egyszerűbb meghatározást megfogalmazni a következő fogalmakról: novella, rapszódia, lírai én, hexameter, pentameter, disztichon, szinesztézia, szimbólum, tragédia, komédia, dialógus, monológ. Képes néhány egyszerűbb meghatározás közül kiválasztani azt, amely a következő fogalmak valamelyikéhez illik: fordulat, retorika, paródia, helyzetkomikum, jellemkomikum. Az ismertebb műfajokról tudja az alapvető információkat.
\item Képes művek, műrészletek szöveghű felidézésére.
\item Képes beszámolót, kiselőadást, prezentációt készíteni és tartani különböző írott és elektronikus forrásokból, kézikönyvekből, atlaszokból/szakmunkákból, a témától függően statisztikai táblázatokból, grafikonokból, diagramokból.
\item Tisztában van a média alapvető kifejezőeszközeivel, az írott és az elektronikus sajtó műfajaival. Ismeri a média, kitüntetetten az audiovizuális média és az internet társadalmi szerepét, működési módjának legfőbb jellemzőit. Kialakul benne a médiatudatosság elemi szintje, az önálló, kritikus attitűd.
\end{itemize}
\paragraph{Idegen nyelv tantárgy alapján}
\begin{itemize}
\item A2 szintű nyelvtudás:
\item A tanuló egyszerű hangzó szövegekből kiszűri a lényeget és néhány konkrét információt.
\item Válaszol a hozzá intézett kérdésekre, sikeresen vesz részt rövid beszélgetésekben.
\item Egyre bővülő szókinccsel, egyszerű nyelvi eszközökkel megfogalmazva történetet mesél el, valamint leírást ad saját magáról és közvetlen környezetéről.
\item Megért ismerős témákról írt rövid szövegeket, különböző típusú, egyszerű írott szövegekben megtalálja a fontos információkat.
\item Összefüggő mondatokat, rövid szöveget ír hétköznapi, őt érintő témákról.
\end{itemize}
\paragraph{Történelem, társadalmi és állampolgári ismeretek tantárgy alapján}
\begin{itemize}
\item Az újkori és modern kori egyetemes és magyar történelmi jelenségek, események feldolgozásával a jelenben zajló folyamatok előzményeinek felismerése, a nemzeti azonosságtudat kialakulása.
\item Annak felismerése, hogy a modern nemzetállamokat különböző kultúrájú, vallású, szokású, életmódú népek, nemzetiségek együttesen alkotják
\item A múltat és a történelmet formáló alapvető folyamatok, összefüggések felismerése (pl. technikai fejlődésének hatásai a társadalomra és a gazdaságra) és egyszerű, átélhető erkölcsi tanulságok (pl. társadalmi kirekesztés) azonosítása, megítélése.
\item Az új- és modern korban élt emberek, közösségek élet-, gondolkodás- és szokásmódjainak azonosítása, a hasonlóságok és különbségek felismerése.
\item A tanuló ismerje fel, hogy a múltban való tájékozódást segítik a kulcsfogalmak és fogalmak, amelyek segítik a történelmi tájékozódás és gondolkodás kialakulását, fejlődését.
\item Ismerje fel, hogy az utókor a nagy történelmi személyiségek, nemzeti hősök cselekedeteit a közösségek érdekében végzett tevékenységek szempontjából értékeli, tudjon példát mondani ellentétes értékelésre.
\item Ismerje a XIX-XX. század nagy korszakainak megnevezését, illetve egy-egy korszak főbb jelenségeit, jellemzőit, szereplőit, összefüggéseit.
\item Ismerje a magyar történelem főbb csomópontjait egészen napjainkig. Legyen képes e hosszú történelmi folyamat meghatározó szereplőinek azonosítására és egy-egy korszak főbb kérdéseinek felismerésére.
\item Ismerje a korszak meghatározó egyetemes és magyar történelmének eseményeit, évszámait, történelmi helyszíneit. Legyen képes összefüggéseket találni a térben és időben eltérő fontosabb történelmi események között, különös tekintettel azokra, melyek a magyarságot közvetlenül vagy közvetetten érintik.
\item Tudja, hogy az egyes népek és államok a korszakban eltérő társadalmi, gazdasági és vallási körülmények között éltek, de a modern kor beköszöntével a köztük lévő kapcsolatok széleskörű rendszere épült ki.
\item Tudja, hogy Európához köthetők a modern demokratikus viszonyokat megalapozó szellemi mozgalmak és dokumentumok.
\item Tudja, hogy Magyarország a trianoni békediktátum következtében elvesztette területének és lakosságának kétharmadát, és közel öt millió magyar került kisebbségi sorba. Jelenleg közel hárommillió magyar nemzetiségű él a szomszédos államokban és a világ különböző részein, akik szintén a magyar nemzet tagjainak tekintendők.
\item Tudja, hogy a holokausztnak többszázezer magyar áldozata is volt, és legyen tisztában ennek hazai és nemzetközi történelmi, politikai előzményeivel, körülményeivel és erkölcsi vonatkozásaival.
\item Tudja, hogy a társadalmakban eltérő jogokkal rendelkező és vagyoni helyzetű emberek alkotnak rétegeket, csoportokat.
\item Tudjon különbséget tenni a demokrácia és a diktatúra között, s tudjon azokra példát mondani a feldolgozott történelmi korszakokból és napjainkból.
\item Ismerje fel a tanuló a hazánkat és a világot fenyegető globális veszélyeket, betegségeket, terrorizmust, munkanélküliséget.
\item Legyen képes különbséget tenni a történelem különböző típusú forrásai között, és legyen képes egy-egy korszakra jellemző képeket, tárgyakat, épületeket felismerni.
\item Tudja, hogy hol kell a fontos események forrásait kutatni, és legyen képes a megismert jelenségeket, eseményeket összehasonlítani.
\item Ismerje a híres történelmi személyiségek jellemzéséhez szükséges adatokat, eseményeket és kulcsszavakat.
\item Legyen képes a tanuló történelmi ismeretet meríteni és egyszerű következtetéseket megfogalmazni hallott és olvasott szövegekből, különböző médiumok anyagából.
\item Legyen képes információt gyűjteni adott témához könyvtárban és múzeumban; olvasmányairól készítsen lényeget kiemelő jegyzetet.
\item Legyen képes könyvtári munkával és az internet kritikus használatával forrásokat gyűjteni, kiselőadást tartani, illetve érvelni.
\item Ismerje fel a sajtó és média szerepét a nyilvánosságban, tudja azonosítani a reklám és médiapiac jellegzetességeit.
\item Legyen képes szóbeli beszámolót készíteni önálló gyűjtőmunkával szerzett ismereteiről, és kiselőadást tartani.
\item Legyen képes saját vélemény megfogalmazása mellett mások véleményének figyelembe vételére, és tudjon ezekre reflektálni.
\item Legyen képes példák segítségével értelmezni az alapvető emberi, gyermek- és diákjogokat, valamint a társadalmi szolidaritás különböző formáit.
\item Legyen képes példák segítségével bemutatni a legfontosabb állampolgári jogokat és kötelességeket, tudja értelmezni ezek egymáshoz való viszonyát.
\item Legyen képes a gazdasági és pénzügyi terület fontosabb szereplőit azonosítani, illetve egyszerű családi költségvetést készíteni, és mérlegelni a háztartáson belüli megtakarítási lehetőségeket.
\end{itemize}
\paragraph{Ének-zene
  b változat tantárgy alapján}
\begin{itemize}
\item A tanulók képesek 14-17 népdalt, balladát, históriás éneket több versszakkal, valamint 8-10 műzenei idézetet emlékezetből, g–f” hangterjedelemben előadni.
\item Képesek kifejezően, egységes hangzással, tiszta intonációval énekelni, és új dalokat megfelelő előkészítést követően hallás után megtanulni.
\item Többszólamú éneklési készségük fejlődik. Képesek csoportosan egyszerű orgánumokat megszólaltatni.
\item Kreatívan vesznek részt a generatív játékokban és feladatokban. Érzik az egyenletes lüktetést, tartják a tempót, érzékelik a tempóváltozást. A 5/8-os és 7/8-os és 8/8-os metrumot helyesen hangsúlyozzák.
\item Felismerik és megszólaltatják a tanult zenei elemeket (metrum, dinamikai jelzések, ritmus, dallam).
\item Fejlődik zenei memóriájuk és belső hallásuk.
\item Fejlődik formaérzékük, a formai építkezés jelenségeit felismerik és meg tudják fogalmazni.
\item Fejlődik hangszínhallásuk. Megkülönböztetik a cimbalom, lant, csembaló, orgona, szaxofon hangszínét. Ismerik a hangszerek alapvető jellegzetességeit.
\item A zenei korszakokból kiválasztott zeneművek közül 20-25 alkotást/műrészletet ismernek.
\end{itemize}
\paragraph{Vizuális kultúra tantárgy alapján}
\begin{itemize}
\item Célirányos vizuális megfigyelési szempontok önálló alkalmazása.
\item A vizuális nyelv és kifejezés eszközeinek tudatos és pontos alkalmazása az alkotótevékenység során adott célok kifejezése érdekében.
\item Bonyolultabb kompozíciós alapelvek használata kölönböző célok érdekében.
\item Térbeli és időbeli változások vizuális megjelenítésének kifejező vagy közlő szándéknak megjelelő értelmezése, és következtetések megfogalmazása.
\item Alapvetően közlő funkcióban lévő képi vagy képi és szöveges megjelenések egyszerű értelmezése.
\item Az épített és tárgyi környezet elemző megfigyelése alapján összetettebb következtetések megfogalmazása.
\item Több jól megkülönböztethető technika, médium (pl. állókép-mozgókép, síkbeli-térbeli) tudatos használata az alkotótevékenység során.
\item A médiatudatos gondolkodás megalapozása a vizuális kommunikációs eszközök és formák rendszerezőbb feldolgozása kapcsán.
\item A mozgóképi közlésmód, az írott sajtó és az online   kommunikáció szövegszervező alapeszközeinek felismerése.
\item Mozgóképi szövegek megkülönböztetése a valóság ábrázolásához való viszony, alkotói szándék és nézői elvárás karaktere szerint.
\item Társművészeti kapcsolatok lehetőségeinek értelmezése.
\item A legfontosabb kultúrák, művészettörténeti korok, stílusirányzatok megkülönböztetése és a meghatározó alkotók műveinek felismerése.
\item Vizuális jelenségek, tárgyak, műalkotások árnyaltabb elemzése, összehasonlítása.
\item A vizuális megfigyelés és elemzés során önálló kérdések megfogalmazása.
\item Önálló vélemény megfogalmazása saját és mások munkájáról.
\end{itemize}
\section{9-10. évfolyam}
\subsection{Harmónia tantárgy}
\paragraph{Informatika tantárgy alapján}
\begin{itemize}
\item tudjon digitális kamerával felvételt készíteni, legyen képes adatokat áttölteni kameráról a számítógép adathordozójára;
\item ismerje az adatvédelem hardveres és szoftveres módjait;
\item ismerje az ergonómia alapjait.
\item legyen képes táblázatkezelővel tantárgyi feladatokat megoldani, egyszerű számításokat elvégezni;
\item tudjon körlevelet készíteni;
\item tudja kezelni a rendelkezésére álló adatbázis-kezelő programot;
\item tudjon adattáblák között kapcsolatokat felépíteni, adatbázisokból lekérdezéssel információt nyerni. A nyert adatokat tudja esztétikus, használható formába rendezni.
\item tudjon algoritmusokat készíteni,
\item legyen képes a probléma megoldásához szükséges eszközöket kiválasztani;
\item legyen képes tantárgyi problémák megoldásának tervezésére és megvalósítására;
\item ismerjen és használjon tantárgyi szimulációs programokat;
\item legyen képes tantárgyi mérések eredményeinek kiértékelésére;
\item legyen képes egy csoportban tevékenykedni.
\item legyen képes információkat szerezni, azokat hagyományos, elektronikus vagy internetes eszközökkel publikálni;
\item legyen képes társaival kommunikálni az interneten, közös feladatokon dolgozni;
\item tudja használni az újabb informatikai eszközöket, információszerzési technológiákat.
\item ismerje az adatvédelemmel kapcsolatos fogalmakat;
\item legyen képes értékelni az információforrásokat;
\item ismerje az informatikai eszközök etikus használatára vonatkozó szabályokat;
\item ismerje a szerzői joggal kapcsolatos alapfogalmakat;
\item ismerje az infokommunikációs publikálási szabályokat;
\item ismerje fel az informatikai fejlesztések gazdasági, környezeti, kulturális hatásait;
\item ismerje fel az informatikai eszközök használatának személyiséget és az egészséget befolyásoló hatásait;
\item ismerje fel az elektronikus szolgáltatások szerepét,
\item legyen képes néhány elektronikus szolgáltatás kritikus használatára;
\item ismerje fel az elektronikus szolgáltatások jellemzőit, előnyeit, hátrányait;
\item ismerje fel a fogyasztói viselkedést befolyásoló módszereket a médiában;
\item ismerje fel a tudatos vásárló jellemzőit.
\item legyen képes bármely, a tanulmányaihoz kapcsolódó feladata során az információs problémamegoldás folyamatát önállóan, alkotóan végrehajtani;
\item legyen tisztában saját információkeresési stratégiáival, tudja azokat tudatosan alkalmazni, legyen képes azt értékelni, tudatosan fejleszteni.
\end{itemize}
\paragraph{Testnevelés és sport tantárgy alapján}
\begin{itemize}
\item Az adott iskolában a helyi tanterv szerinti technikai, taktikai és egyéb játékfeladatok ismerete és aktív, kooperatív gyakorlás.
\item Komplex szabályismeret, sportszerű alkalmazás és a játékok önálló továbbfejlesztése. Sportjátékok lényeges versenyszabályokkal.
\item A technikák és taktikai megoldások többnyire tudatos, a játékszerepnek megfelelő megválasztása.
\item A játékfolyamat, a taktikai megoldások szóbeli elemzése, a fair és a csapatelkötelezett játék melletti állásfoglalás.
\item Tapasztalat a játékvezetői gyakorlatban.
\item Játéktapasztalat a társas kapcsolatok ápolásában, a bármilyen képességű társakat elfogadó, bevonó játékok játszásában, megválasztásában.
\item A mozgáselemek mozgásbiztonságának és a gyakorlás mennyiségének, minőségének oksági viszonyainak megértése és érvényesítése a gyakorlatban.
\item A javító kritika elfogadása és a mozdulatok kivitelezésének javítása. Esztétikus és harmonikus előadásmód.
\item Önálló talaj és/vagy szergyakorlat, egyszerű aerobik elemkapcsolat, táncmotívumfüzér összeállítása.
\item Célszerű gyakorlási és gyakorlásszervezési formációk, versenyszituációk, versenyszabályok ismerete.
\item A tanult mozgások versenysportja területén, a magyar sportolók sikereiről elemi tájékozottság.
\item Egy kijelölt táv megtételéhez szükséges idő és sebesség helyes becslésére, illetve a becsült értékek alapján a feladat pontos végrehajtása. Évfolyamonként önmagához mérten javuló futó-, ugró-, dobóteljesítmény.
\item A tempóérzék és odafigyelési képesség fejlődése a váltófutás gyakorlásában.
\item A transzferhatás érvényesülése, más mozgásformák teljesítményének javulása az atlétikai képességek fejlődésének hatására.
\item Az adott sportmozgás technikájának elfogadható cselekvésbiztonságú végrehajtása.
\item Tapasztalat a sportolás során használt különféle anyagok, felületek tulajdonságairól és a baleseti kockázatokról.
\item Az adott alternatív sportmozgáshoz szükséges edzés és balesetvédelmi alapfogalmak ismerete, és azok alkalmazása a gyakorlatban.
\item Önvédelem és küzdősportok
\item Az önvédelmi és küzdőgyakorlatokban, harcokban a közös szabályok, biztonsági követelmények és a küzdésekkel kapcsolatos rituálé betartása.
\item A veszélyhelyzetek kerülése, az indulatok, agresszív magatartásformák feletti uralom.
\item Néhány támadási és védekezési megoldás, kombináció ismerete, eredményes önvédelem, és szabadulás a fogásból.
\item 1000 m-en a választott technikával, egyéni tempóban, szabályos fordulóval úszás.
\item Jelentős fejlődés az úszóerő és állóképesség területén.
\item Egy választott úszásnemhez tartozó öt szárazföldi képességfejlesztő gyakorlat bemutatása.
\item Az amatőr versenyekhez elegendő versenyszabályok ismerete.
\item Fejlődő saját teljesítmény a víz alatti úszásban.
\item Egyszerűbb feladatok, ugrások során másokkal szinkronban mozgás a vízbe és vízben.
\item Passzív társ vonszolása kisebb távon (4-5 méter) és a vízből mentés veszélyeinek, pontos menetének felsorolása.
\item Bemelegítés, fizikai felkészülés a sérülésmentes sporttevékenységre.
\item A biomechanikailag helyes testtartás jellemzőinek és néhány jellemző deformitás kockázatainak értelmezése, a megőrzés néhány gyakorlatának ismerete és felelős alkalmazása.
\item A gerinc sérüléseinek leggyakoribb fajtái, és a gerinc és az ízületek védelemének legfontosabb szempontjainak ismerete.
\item A preventív stressz- és feszültségoldó gyakorlatok tudatos alkalmazása. A fittségi paraméterek ismerete, mérésük tesztek segítségével, ezzel kapcsolatosan önfejlesztő célok megfogalmazása az egészség-edzettség érdekében.
\item A szükséges táplálkozási ismeretek alkalmazása a testsúly, testtömeg ismeretében.
\item A rendszeres testmozgás pozitív hatásainak ismerete a káros szenvedélyek leküzdésében, az érzelem- és a feszültségszabályozásban.
\end{itemize}
\subsection{STEM tantárgy}
\paragraph{Matematika tantárgy alapján}
\begin{itemize}
\item Halmazokkal kapcsolatos alapfogalmak ismerete, halmazok szemléltetése, halmazműveletek ismerete; számhalmazok ismerete.
\item Értsék és jól használják a matematika logikában megtanult szakkifejezéseket a hétköznapi életben.
\item Definíció, tétel felismerése, az állítás és a megfordításának felismerése; bizonyítás gondolatmenetének követése.
\item Egyszerű leszámlálási feladatok megoldása, a megoldás gondolatmenetének rögzítése szóban, írásban.
\item Gráffal kapcsolatos alapfogalmak ismerete. Alkalmazzák a gráfokról tanult ismereteiket gondolatmenet szemléltetésére, probléma megoldására.
\item Egyszerű algebrai kifejezések használata, műveletek algebrai kifejezésekkel; a tanultak alkalmazása a matematikai problémák megoldásában (pl. modellalkotás szöveg alapján, egyenletek megoldása, képletek értelmezése); egész kitevőjű hatványok, azonosságok.
\item Elsőfokú, másodfokú egyismeretlenes egyenlet megoldása; ilyen egyenletre vezető szöveges és gyakorlati feladatokhoz egyenletek felírása és azok megoldása, a megoldás önálló ellenőrzése.
\item Elsőfokú és másodfokú (egyszerű) kétismeretlenes egyenletrendszer megoldása; ilyen egyenletrendszerre vezető szöveges és gyakorlati feladatokhoz az egyenletrendszer megadása, megoldása, a megoldás önálló ellenőrzése.
\item Egyismeretlenes egyszerű másodfokú egyenlőtlenség megoldása.
\item Az időszak végére elvárható a valós számkör biztos ismerete, e számkörben megismert műveletek gyakorlati és elvontabb feladatokban való alkalmazása.
\item A tanulók képesek a matematikai szöveg értő olvasására, tankönyvek, keresőprogramok célirányos használatára, szövegekből a lényeg kiemelésére.
\item A függvény megadása, a szereplő halmazok ismerete (értelmezési tartomány, értékkészlet); valós függvény alaptulajdonságainak ismerete.
\item A tanult alapfüggvények ismerete (tulajdonságok, grafikon).
\item Egyszerű függvénytranszformációk végrehajtása.
\item Valós folyamatok elemzése a folyamathoz tartozó függvény grafikonja alapján.
\item Függvénymodell készítése lineáris kapcsolatokhoz; a meredekség.
\item A tanulók tudják az elemi függvényeket ábrázolni koordináta-rendszerben, és a legfontosabb függvénytulajdonságokat meghatározni, nemcsak a matematika, hanem a természettudományos tárgyak megértése miatt, és különböző gyakorlati helyzetek leírásának érdekében is.
\item Térelemek ismerete; távolság és szög fogalma, mérése.
\item Nevezetes ponthalmazok ismerete, szerkesztésük.
\item A tanult egybevágósági és hasonlósági transzformációk és ezek tulajdonságainak ismerete.
\item Egybevágó alakzatok, hasonló alakzatok; két egybevágó, illetve két hasonló alakzat több szempont szerinti összehasonlítása (pl. távolságok, szögek, kerület, terület, térfogat).
\item Szimmetria ismerete, használata.
\item Háromszögek tulajdonságainak ismerete (alaptulajdonságok, nevezetes vonalak, pontok, körök).
\item Derékszögű háromszögre visszavezethető (gyakorlati) számítások elvégzése Pitagorasz-tétellel és a hegyesszögek szögfüggvényeivel; magasságtétel és befogótétel ismerete.
\item Szimmetrikus négyszögek tulajdonságainak ismerete.
\item Vektor fogalmának ismerete; három új művelet ismerete: vektorok összeadása, kivonása, vektor szorzása valós számmal; vektor felbontása, vektorkoordináták meghatározása adott bázisrendszerben.
\item Kerület, terület, felszín és térfogat szemléletes fogalmának kialakulása, a jellemzők kiszámítása (képlet alapján); mértékegységek ismerete; valós síkbeli, illetve térbeli probléma geometriai modelljének megalkotása.
\item A geometriai ismeretek bővülésével, a megismert geometriai transzformációk rendszerezettebb tárgyalása után fejlődött a tanulók dinamikus geometriai szemlélete, diszkussziós képessége.
\item A háromszögekről tanult ismeretek bővülésével a tanulók képesek számítási feladatokat elvégezni, és ezeket gyakorlati problémák megoldásánál alkalmazni.
\item A szerkesztési feladatok során törekednek az igényes, pontos munkavégzésre.
\item Adathalmaz rendezése megadott szempontok szerint, adat gyakoriságának és relatív gyakoriságának kiszámítása.
\item Táblázat olvasása és készítése; diagramok olvasása és készítése.
\item Adathalmaz móduszának, mediánjának, átlagának értelmezése, meghatározása.
\item Véletlen esemény, biztos esemény, lehetetlen esemény, véletlen kísérlet, esély/valószínűség fogalmak ismerete, használata.
\item Nagyszámú véletlen kísérlet kiértékelése, az előzetesen „jósolt” esélyek és a relatív gyakoriságok összevetése.
\item A valószínűség-számítási, statisztikai feladatok megoldása során a diákok rendszerező képessége fejlődött. A tanulók képesek adatsokaságot jellemezni, ábrákról adatsokaság jellemzőit leolvasni. Szisztematikus esetszámlálással meg tudják határozni egy adott esemény bekövetkezésének esélyét.
\end{itemize}
\paragraph{Biológia-egészségtan
b változat tantárgy alapján}
\begin{itemize}
\item A tanuló tudja használni a fénymikroszkóp különböző fajtáit, ahhoz előkészíteni a vizsgálati anyagokat. Vizsgálatainak eredményeit rajzban/fényképekkel és írásban rögzíti.
\item Ismeri a vírusok, baktériumok biológiai egészségügyi jelentőségét, az általuk okozott emberi betegségek megelőzésének lehetőségeit, a védekezés formáit. Ismeri a féregfertőzéseket és azok megelőzési feltételeit, a kullancscsípés megelőzését, a csípés esetleges következményeit.
\item Képes a biológiai szerveződési szinteknek megfelelő sorrendben tanult nagyobb élőlénycsoportok (mikroba, növény,állat, gomba ) elhelyezésére a törzsfán. Képes ok-okozati összefüggések felismerésére az élőlények testfelépítése, életműködése, életmódja között. Ismeri az életmód és a környezet kölcsönhatásait.
\item Ismeri, illetve példákból felismeri az állatok különböző magatartásformáit.
\end{itemize}
\paragraph{Fizika tantárgy alapján}
\begin{itemize}
\item A kísérletezési, mérési kompetencia, a megfigyelő, rendszerező készség fejlődése.
\item A mozgástani alapfogalmak ismerete, grafikus feladatmegoldás. A newtoni mechanika szemléleti lényegének elsajátítása: az erő nem a mozgás fenntartásához, hanem a mozgásállapot megváltoztatásához szükséges.
\item Egyszerű kinematikai és dinamikai feladatok megoldása.
\item A kinematika és dinamika mindennapi alkalmazása.
\item Folyadékok és gázok sztatikájának és áramlásának alapjelenségei és ezek felismerése a gyakorlati életben.
\item Az elektrosztatika alapjelenségei és fogalmai, az elektromos és a mágneses mező fizikai objektumként való elfogadása.
\item Az áramokkal kapcsolatos alapismeretek és azok gyakorlati alkalmazásai, egyszerű feladatok megoldása.
\item A gázok makroszkopikus állapotjelzői és összefüggéseik, az ideális gáz golyómodellje, a nyomás és a hőmérséklet kinetikus értelmezése golyómodellel.
\item Hőtani alapfogalmak, a hőtan főtételei, hőerőgépek. Annak ismerete, hogy gépeink működtetése, az élő szervezetek működése csak energia befektetése árán valósítható meg, a befektetett energia jelentős része elvész, a működésben nem hasznosul, „örökmozgó” létezése elvileg kizárt.
  Mindennapi környezetünk hőtani vonatkozásainak ismerete.
\item Az energiatudatosság fejlődése.
\end{itemize}
\paragraph{Kémia tantárgy alapján}
\begin{itemize}
\item A tanuló ismerje az anyag tulajdonságainak anyagszerkezeti alapokon történő magyarázatához elengedhetetlenül fontos modelleket, fogalmakat, összefüggéseket és törvényszerűségeket, a legfontosabb szerves és szervetlen vegyületek szerkezetét, tulajdonságait, csoportosítását, előállítását, gyakorlati jelentőségét.
\item Értse az alkalmazott modellek és a valóság kapcsolatát, a szerves vegyületek esetében a funkciós csoportok tulajdonságokat meghatározó szerepét, a tudományos és az áltudományos megközelítés közötti különbségeket.
\item Ismerje és értse a fenntarthatóság fogalmát és jelentőségét.
\item Tudja magyarázni az anyagi halmazok jellemzőit összetevőik szerkezete és kölcsönhatásaik alapján.
\item Tudjon egy kémiával kapcsolatos témáról sokféle információforrás kritikus felhasználásával önállóan vagy csoportmunkában szóbeli és írásbeli összefoglalót, prezentációt készíteni, és azt érthető formában közönség előtt is bemutatni.
\item Tudja alkalmazni a megismert tényeket és törvényszerűségeket egyszerűbb problémák és számítási feladatok megoldása során, valamint a fenntarthatósághoz és az egészségmegőrzéshez kapcsolódó viták alkalmával.
\item Képes legyen egyszerű kémiai jelenségekben ok-okozati elemek meglátására, tudjon tervezni ezek hatását bemutató, vizsgáló egyszerű kísérletet, és ennek eredményei alapján tudja értékelni a kísérlet alapjául szolgáló hipotéziseket.
\item Képes legyen kémiai tárgyú ismeretterjesztő, vagy egyszerű tudományos, illetve áltudományos cikkekről koherens és kritikus érvelés alkalmazásával véleményt formálni, az abban szereplő állításokat a tanult ismereteivel összekapcsolni, mások érveivel ütköztetni.
\item Megszerzett tudása birtokában képes legyen a saját személyes sorsát, a családja életét és a társadalom fejlődési irányát befolyásoló felelős döntések meghozatalára.
\end{itemize}
\paragraph{Földrajz tantárgy alapján}
\begin{itemize}
\item A tanulók legyenek képesek a különböző szempontból elsajátított földrajzi (általános és leíró természet-, illetve társadalom-, valamint gazdaságföldrajzi) ismereteik szintetizálására. Rendelkezzenek valós képzetekkel a környezeti elemek méreteiről, a számszerűen kifejezhető adatok és az időbeli változások nagyságrendjéről.
\item Legyenek képesek a térkép információforrásként történő használatára, a leolvasott adatok értelmezésére. Ismerjék fel a Világegyetem és a Naprendszer felépítésében, a bolygók mozgásában megnyilvánuló törvényszerűségeket.
\item Tudjanak tájékozódni a földtörténeti időben, ismerjék a kontinenseket felépítő nagyszerkezeti egységek kialakulásának időbeli rendjét, földrajzi elhelyezkedését.
\item Legyenek képesek megadott szempontok alapján bemutatni az egyes geoszférák sajátosságait, jellemző folyamatait és azok összefüggéseit. Lássák be, hogy az egyes geoszférákat ért környezeti károk hatása más szférákra is kiterjedhet.
\item Legyenek képesek a földrajzi övezetesség kialakulásában megnyilvánuló összefüggések és törvényszerűségek értelmezésére.
\item Legyenek képesek alapvető összefüggések és törvényszerűségek felismerésére és megfogalmazására az egész Földre jellemző társadalmi-gazdasági folyamatokkal kapcsolatosan.
\item Tudják elhelyezni az egyes országokat, országcsoportokat és integrációkat a világ társadalmi-gazdasági folyamataiban, tudják értelmezni a világgazdaságban betöltött szerepüket.
\item Legyenek képesek összevetni és értékelni az egyes térségek, illetve országok eltérő társadalmi-gazdasági adottságait és az adottságok jelentőségének időbeli változásait.
\item Ismerjék a globalizáció gazdasági és társadalmi hatását, értelmezzék ellentmondásait.
\item Ismerjék a monetáris világ jellemző folyamatait, azok társadalmi-gazdasági hatásait.
\item Ismerjék hazánk társadalmi-gazdasági fejlődésének jellemzőit, a gazdasági fejlettség területi különbségeit és ennek okait.
\item Példákkal támasszák alá Európai Unió egészére kiterjedő, illetve a környezető országokkal kialakult regionális együttműködések szerepét
\item Tudják elhelyezni hazánkat a világgazdaság folyamataiban.
\item Tudják példákkal bizonyítani a társadalmi-gazdasági folyamatok környezetkárosító hatását, a lokális problémák globális következmények elvének érvényesülését. Ismerjék az egész Földünket érintő globális társadalmi és gazdasági problémákat.
\item Tudjanak érvelni a fenntarthatóságot szem előtt tartó gazdaság, illetve gazdálkodás fontossága mellett.
\item Ismerjék az egyén szerepét és lehetőségeit a környezeti problémák mérséklésben, nevezzék meg konkrét példáit.
\item Legyenek képesek természet-, illetve társadalom- és gazdaságföldrajzi megfigyelések elvégzésére, a tapasztalatok rögzítésére és összegzésére.
\item Legyenek képesek különböző nyomtatott és elektronikus információhordozókból földrajzi tartalmú információk gyűjtésére és feldolgozására, az információk összegzésére, a lényeges elemek kiemelésére. Ennek során alkalmazzák digitális ismereteiket.
\item Legyenek képesek véleményüket a földrajzi gondolkodásnak megfelelően megfogalmazni, logikusan érvelni.
\item Tudják alkalmazni ismereteiket földrajzi tartalmú problémák megoldása során a mindennapi életben.
\item Tudják földrajzi ismereteiket felhasználni különböző döntéshelyzetekben.
\item Legyenek képesek a társakkal való együttműködésre a földrajzi-környezeti tartalmú feladatok megoldásakor.
\item Alakuljon ki bennük az igény arra, hogy későbbi életük folyamán önállóan gyarapítsák tovább földrajzi ismereteiket.
\item Legyenek képesek topográfiai tudásuk alkalmazására más tantárgyak tanulása során, illetve a mindennapi életben.
\item Ismereteik alapján biztonsággal tájékozódjanak a földrajzi térben, illetve az azt megjelenítő különböző térképeken. Ismerjék a tananyagban meghatározott topográfiai fogalmakhoz kapcsolódó tartalmakat.
\end{itemize}
\subsection{KULT tantárgy}
\paragraph{Idegen nyelv tantárgy alapján}
\begin{itemize}
\item A tanuló képes főbb vonalaiban megérteni a köznyelvi beszédet, ha az számára rendszeresen előforduló, ismerős témákról folyik.
\item A mindennapi élet legtöbb helyzetében boldogul, gondolatokat cserél, véleményt mond, érzelmeit kifejezi és stílusában a kommunikációs helyzethez alkalmazkodik.
\item A tanuló képes begyakorolt szerkezetekkel érthetően, folyamatoshoz közelítően beszélni. Az átadott információ lényegét megközelítő tartalmi pontossággal fejti ki.
\item Megérti a hétköznapi nyelven írt, érdeklődési köréhez kapcsolódó, lényegre törő, autentikus vagy kismértékben szerkesztett szövegekben az általános vagy részinformációkat.
\item A tanuló több műfajban is képes egyszerű, rövid, összefüggő szövegeket fogalmazni ismert, hétköznapi témákról. Írásbeli megnyilatkozásaiban már kezdenek megjelenni műfaji sajátosságok és különböző stílusjegyek.
\end{itemize}
\paragraph{Történelem, társadalmi és állampolgári ismeretek tantárgy alapján}
\begin{itemize}
\item Az ókori, középkori és kora újkori egyetemes és magyar kultúrkincs rendszerező megismerésével az egyetemes emberi értékek tudatos vállalása, családhoz, lakóhelyhez, nemzethez, Európához való tartozás fontosságának felismerése, elfogadása.
\item A múltat és a történelmet formáló összetett folyamatok, látható és háttérben meghúzódó összefüggések felismerése, és ezek erkölcsi-etikai aspektusainak azonosítása.
\item A korábbi korokban élt emberek, közösségek élet-, gondolkodás- és szokásmódjainak azonosítása, a különböző államformák működési jellemzőinek felismerése.
\item Ismerje fel a tanuló a civilizációk történetének jellegzetes sémáját (kialakulás, virágzás, hanyatlás).
\item Ismerje és mind szélesebb körben alkalmazza a történelem értelmezését segítő kulcsfogalmakat és egyedi fogalmakat, az árnyalt történelmi tájékozódás és gondolkodás érdekében.
\item Ismerje fel, hogy az utókor a nagy történelmi személyiségek, nemzeti hősök cselekedeteit a közösségek érdekében végzett tevékenységek szempontjából értékeli, tudjon példákat mondani különböző korok eltérő értékítéleteiről egy-egy történelmi személyiség kapcsán.
\item Tudja az egyes népeket vallásuk és kultúrájuk, életmódjuk alapján azonosítani és megismerni. Ismerje fel, hogy a vallási előírások, valamint az államok által megfogalmazott szabályok döntő mértékben befolyásolhatják a társadalmi viszonyokat és a mindennapokat.
\item Tudja, hogy a történelmi jelenségeket, folyamatokat társadalmi, gazdasági, szellemi tényezők együttesen befolyásolják.
\item Ismerje a világ és az európai kontinens eltérő fejlődési irányait, ezek társadalmi, gazdasági és szellemi hátterét. Tudja azonosítani Európa különböző régióinak eltérő fejlődési útjait.
\item Ismerje fel a meghatározó vallási, társadalmi, gazdasági, szellemi összetevőket egy-egy történelmi jelenség, folyamat értelmezésénél.
\item Tudja értelmezni az eltérő uralkodási formák és társadalmi, gazdasági viszonyok közötti összefüggéseket.
\item Ismerje a keresztény Magyar Királyság létrejöttének, virágzásának és hanyatlásának főbb állomásait, a kora újkor békés építőmunkájának eredményeit, valamint a polgári Magyarország kiépülésének meghatározó gondolatait és kulcsszereplőit.
\item Legyen képes a tanuló ismereteket meríteni, beszámolót, kiselőadást készíteni és tartani különböző írott forrásokból, történelmi kézikönyvekből, atlaszokból/szakmunkákból, statisztikai táblázatokból, grafikonokból, diagramokból és internetről.
\item Legyen képes a szerzett információk rendezésére/értelmezésére, és tudja a rendelkezésre álló információforrásokat áttekinteni/értékelni is.
\item Tudjon kérdéseket megfogalmazni a forrás megbízhatóságára és a szerző esetleges elfogultságára vonatkozóan.
\item Legyen képes különböző magatartástípusok és élethelyzetek megfigyelésére, ezekből következtetések levonására.
\item Tudja írott és hallott szövegből a lényeget kiemelni tételmondatok meghatározásával, szövegek tömörítésével és átfogalmazásával egyaránt.
\item Legyen képes a többféleképpen értelmezhető szövegek jelentésrétegeinek feltárására.
\item Legyen képes feltevéseket megfogalmazni történelmi személyiségek cselekedeteinek, viselkedésének mozgatórugóiról.
\item Legyen képes történelmi helyzeteket elbeszélni, eljátszani a különböző szereplők nézőpontjából.
\item Legyen képes saját véleményét megfogalmazni, közben legyen képes vitában a tárgyilagos érvelés és a személyeskedés megkülönböztetésére.
\item Legyen képes folyamatábrát, diagramot, vizuális rendezőt (táblázatot, ábrát) készíteni, történelmi témákat vizuálisan ábrázolni.
\item Legyen képes az időmeghatározásra konkrét kronológiai adatokkal, valamint történelmi időszakokhoz kapcsolódóan egyaránt, és tudjon kronológiai adatokat rendszerezni.
\item Használja a történelmi korszakok és periódusok nevét.
\item Legyen képes összehasonlítani történelmi időszakokat, egybevetni eltérő korszakok emberi sorsait a változások szempontjából, és legyen képes a változások megkülönböztetésére is.
\item Legyen képes érzékelni és elemezni az egyetemes és a magyar történelem eltérő időbeli ritmusát, illetve ezek kölcsönhatásait. Tudja az egyes korszakokat komplex módon jellemezni és bemutatni.
\item Legyen képes különböző információforrásokból önálló térképvázlatok rajzolására, különböző időszakok történelmi térképeinek az összehasonlítására, a történelmi tér változásainak leolvasására, az adott témához leginkább megfelelő térkép kiválasztására.
\end{itemize}
\paragraph{Ének-zene tantárgy alapján}
\begin{itemize}
\item A tanulók képesek 8-10 népzenei, valamint 8-10 műzenei idézetet részben kottából, részben emlékezetből csoportosan előadni.
\item Képesek kifejezően, egységes hangzással, tiszta intonációval énekelni, és új dalokat megfelelő előkészítést követően hallás után megtanulni.
\item Képesek egyszerű két- és háromszólamú kórusműveket vagy azok részleteit, kánonokat megszólaltatni.
\item Fejlődik formaérzékük, a formai építkezés jelenségeit felismerik és meg tudják fogalmazni.
\item Ismerik a hangszerek alapvető jellegzetességeit.
\item A generatív tevékenységek eredményeként érzékelik, felismerik a zenei kifejezések, a formák, a műfajok, és a zenei eszközök közti összefüggéseket.
\item Megismerik és értelmezik a kottakép elemeit és az alapvető zenei kifejezéseket.
\item A zenei korszakokból kiválasztott zeneművek közül 20-25 alkotást/műrészletet ismernek és felismernek.
\item A zenei műalkotások megismerése révén helyesen tájékozódnak korunk kulturális sokszínűségében.
\end{itemize}
\paragraph{Dráma és tánc tantárgy alapján}
\begin{itemize}
\item A tanulók képessé válnak a pontos önkifejezésre, a mások előtti megnyilatkozásra és együttműködésre.
\item Növekvő intenzitással és mélységgel vesznek részt szerepjátékokban, csoportos improvizációkban.
\item Tudatosan és kreatívan alkalmazzák a megismert munkaformákat.
\item Képessé válnak a megismert dramaturgiai fogalomkészlet használatára.
\item Képesek színházi előadások drámás eszközökkel történő feldolgozására.
\end{itemize}
\paragraph{Vizuális kultúra tantárgy alapján}
\begin{itemize}
\item Célirányos vizuális megfigyelési szempontok önálló kiválasztása.
\item A vizuális nyelv és kifejezés eszközeinek önálló alkalmazása az alkotótevékenység és a vizuális jelenségek elemzése, értelmezése során.
\item Bonyolultabb kompozíciós alapelvek tudatos használata kölönböző célok érdekében.
\item Térbeli és időbeli változások vizuális megjelenítésének szándéknak megfelelő pontos értelmezése, és egyszerű mozgóképi közlések elkészítése.
\item Alapvetően közlő funkcióban lévő képi, vagy képi és szöveges  megjelenések árnyalt értelmezése.
\item Médiatudatos gondolkodás a tömegkommunikációs eszközök és formák rendszerező feldolgozása.
\item A tervezett, alakított környezet forma és funkció összefüggéseinek felismerése, ennek figyelembe vételével egyszerű tervezés és makettezés.
\item Tanult technikák célnak megfelelő, tudatos alkalmazása alkotótevékenységekben.
\item Társművészeti kapcsolatok árnyalt értelmezése.
\item Kultúrák, művészettörténeti korok, stílusirányzatok rendszerező ismerete és a meghatározó alkotók műveinek felismerése.
\item Az építészet alapvető elrendezési és szerkezeti alapelveinek, illetve stílust meghatározó vonásainak felismerése.
\item Vizuális jelenségek, tárgyak, műalkotások árnyaltabb elemzése összehasonlítása, műelemző módszerek alkalmazásával.
\item Adott vizuális problémakkal kapcsolatban önálló kérdések megfogalmazása.
\item A kreatív problémamegoldás lépéseinek alkalmazása
\item Önálló vélemény megfogalmazása saját és mások munkájáról.
\end{itemize}
\paragraph{Mozgóképkultúra és médiaismeret tantárgy alapján}
\begin{itemize}
\item A tanuló felismeri és megnevezi a mozgóképi közlésmód, az írott sajtó és az online kommunikáció szövegszervező alapeszközeit.
\item alkalmazza az ábrázolás megismert eszközeit egyszerű mozgóképi szövegek értelmezése során (cselekmény és történet megkülönböztetése, szemszög, nézőpont, képkivágat, kameramozgás jelentése az adott szövegben, montázsfunkciók felismerése).
\item a mozgóképi szövegeket meg tudja különböztetni a valóság ábrázolásához való viszonyuk és az alkotói szándék, és a nézői elvárás karaktere szerint (dokumentumfilm és játékfilm, műfaji és szerzői beszédmód).
\item az életkornak megfelelő szinten meg tudja különböztetni a fikció és a virtuális fogalmait egymástól.
\item tisztában van a média alapfunkcióival, a kommunikáció története alapfordulataival, meg tudja fogalmazni, milyen alapvető tényezőktől függ valamely kor és társadalom nyilvánossága.
\item tudja, melyek a kereskedelmi és közszolgálati médiaintézmények elsődleges céljai és eszközei a médiaipari versenyben.
\item megkülönbözteti azokat a fontosabb tényezőket, melyek alapján jellemezhetőek a médiaintézmények célközönségei.
\item meghatározza és alkalmas példákkal illusztrálja a sztereotípia és a reprezentáció fogalmát, ésszerűen indokolja az egyszerűbb reprezentációk különbözőségeit.
\item érvekkel támasztja alá álláspontját olyan vitában, amely a médiaszövegek (pl. reklám, hírműsor) valóságtartalmáról folynak.
\item jellemzi az internetes és mobilkommunikáció fontosabb sajátosságait, az internethasználat biztonságának problémáit.
\item az életkorának megfelelő szinten képes a különböző médiumokból és médiumokról szóló ismeretek összegyűjtésére, azok rendszerezésére, az önálló megfigyelésekre.
\item alkalmazza a mozgóképi szövegekkel, a média működésével kapcsolatos ismereteit a műsorválasztás során is.
\end{itemize}
\paragraph{Magyar nyelv és irodalom tantárgy alapján}
\begin{itemize}
\item A tanuló képes írott szövegek (pl. szépirodalmi, dokumentum- és ismeretterjesztő szöveg) globális (átfogó) megértésére, a szöveg szó szerinti jelentésén túli üzenet értelmezésére, a szövegből információk visszakeresésére.
\item Össze tudja foglalni a szöveg tartalmát, tud önállóan jegyzetet és vázlatot készíteni. Képes az olvasott szöveg tartalmával kapcsolatos saját véleményét szóban és írásban megfogalmazni, indokolni.
\item Képes szövegek kapcsolatának és különbségének felismerésére és értelmezésére, és e képességet alkalmazni tudja elemző szóbeli és írásbeli műfajokban.
\item Képes memoriterek szöveghű tolmácsolására a szövegfonetikai eszközök helyes alkalmazásával, tudatos szövegmondással.
\item Tudja alkalmazni a művek műfaji természetének megfelelő szöveg-feldolgozási eljárásokat, megközelítési módokat.
\item Felismeri a szépirodalmi és nem szépirodalmi szövegekben megjelenített értékeket, erkölcsi kérdéseket, motivációkat, magatartásformákat.
\item Tájékozott a feldolgozott lírai alkotások különböző műfajaiban és hangnemeiben.
\item Bizonyítja különféle szövegek megértését a szöveg felépítésére, grammatikai jellemzőire, témahálózatára, tagolására irányuló elemzéssel, képes szöveghű felolvasásra, kellő tempójú, olvasható, rendezett írásra.
\item Szóbeli és írásbeli kommunikációs helyzetekben megválasztja a megfelelő hangnemet, nyelvváltozatot, stílusréteget. Alkalmazza a művelt köznyelv (regionális köznyelv), illetve a nyelvváltozatok nyelvhelyességi normáit, képes felismerni és értelmezni az attól eltérő nyelvváltozatokat.
\item Képes definíció, magyarázat, egyszerűbb értekezés (kisértekezés) készítésére az olvasmányaival, a felvetett és tárgyalt problémákkal összefüggésben, maga is meg tud fogalmazni kérdéseket, problémákat.
\item Ismeri a hivatalos írásművek jellemzőit, képes önálló szövegalkotásra ezek gyakori műfajaiban.
\item Alkalmazza az idézés szabályait és etikai normáit.
\item Képes az órai eszmecserékben és az irodalmi művekben megjelenő álláspontok azonosítására, követésére, megvitatására, összehasonlítására, eltérő vélemények megértésére, újrafogalmazására.
\item Képes tudásanyagának megfogalmazására írásban a magyar és a világirodalom kiemelkedő alkotóiról.
\item Be tudja mutatni a tanult irodalomtörténeti korszakok és stílusirányzatok sajátosságait.
\item Igyekszik eligazodni a 2000 utáni évek kortárs irodalmában (is), illetve készség szintjén kezeli az online irodalmi felületeket.
\item Ismeri a digitális (szak)irodalmi szövegtárak világát, és kezd azokban otthonosan mozogni.
\item Ő maga is rendszeresen próbálkozik kreatív – papíralapú, illetve digitális – írással.
\item Örömmel vesz részt beszélgetésekben, vitákban, irodalomról folyó diskurzusokban és projektekben – ideértve a diákszínjátszást, -újságírást és -szerkesztést is.
\item Beszélgetéseiben, vitáiban tud idézni memoriterekből és egyéb irodalmi alkotásokból.
\end{itemize}
\section{11-12. évfolyam}
\subsection{Harmónia tantárgy}
\paragraph{Etika tantárgy alapján}
\begin{itemize}
\item A tanulók ismerik az erkölcsi hagyomány legfontosabb elemeit, és e tudás birtokában képesek a mindennapi életben felmerülő erkölcsi problémák felismerésére és kezelésére.
\item Értékítéleteiket ésszerű érvekkel tudják alátámasztani, képesek a felelős mérlegelésen alapuló döntésre. Rendelkeznek az etikai és közéleti vitákban való részvételhez, saját álláspontjuk megvédéséhez, illetve továbbfejlesztéséhez szükséges készségekkel és képességekkel.
\item Képesek elfogadni, megérteni és tisztelni a magukétól eltérő nézeteket.
\item Ismerik azokat az értékelveket, magatartásszabályokat és beállítódásokat, amelyeknek a közmegegyezés kitüntetett erkölcsi jelentőséget tulajdonít.
\end{itemize}
\paragraph{Technika, életvitel és gyakorlat tantárgy alapján}
\begin{itemize}
\item olyan gyakorlati tudás megszerzése, amelynek birtokában a tanulók könnyen eligazodhatnak a mindennapi élet számos területén.
\item Képesség kialakítása önmaguk megismerésére, a konfliktuskezelésre, a változásokhoz való rugalmas alkalmazkodásra, a pozitív életszemléletre, az egészséges életvitelre és a harmonikus családi életre.
\item Felelős gondolkodás, átgondolt döntések képességének kialakítása a pénzügyek kezelésében, a fogyasztási javak használatában, a szolgáltatások igénybevételével és a jövővel kapcsolatban.
\item Mindennapokban nélkülözhetetlen életvezetési és háztartási ismeretek, „háztartási praktikák” ismeretében a napi munka szakszerűbb, hatékonyabb, gazdaságosabb elvégzése.
\item Az élő és a tárgyi környezet kapcsolatából, kölcsönhatásainak megfigyeléséből származó tapasztalatok felhasználása a problémamegoldások során, a tevékenységek gyakorlásakor.
\item Használati utasítások értő olvasása, betartása.
\item Tudatos vásárlókká válás, a fogyasztóvédelem szerepének, a vásárlók jogainak ismerete.
\item A szociális érzékenység növekedése (a fogyatékkal élők és az idősek segítése). Karitatív tevékenységek végzése melletti elköteleződés.
\item Hivatalos ügyekben érdekek képviselete, kulturált stílusú ügyintézés szolgáltatóknál, ügyfélszolgálatoknál.
\item A korszerű pénzkezelés lehetőségeinek és eszközeinek megismerése.
\item A biztonságos, balesetmentes, udvarias közlekedés szabályainak betartása.
\item Magabiztos tájékozódás közvetlen és tágabb környezetben.
\item Közlekedési szabályok és a közlekedési etika alkalmazása.
\item A veszélyhelyzetek felismerése, elhárítása, az elsősegélynyújtás, valamint a balesetvédelem legalapvetőbb ismereteinek alkalmazása.
\item A személyes ambíciók, képességek, objektív lehetőségek komplex mérlegelése, a saját életpályára vonatkozó helyes döntések meghozatala.
\item A személyes kapcsolatok és a munkamagatartás, munkakultúra szerepének felismerése az álláskeresésben és a munkahely megtartásában.
\item A munka és az aktivitás iránti elkötelezettség.
\item Az egész életen át tartó tanulás, a szaktudás, a műveltség fontosságának elfogadása és érvényesítése.
\end{itemize}
\paragraph{Testnevelés és sport tantárgy alapján}
\begin{itemize}
\item A helyi tanterv szerint tanított két sportjátékra vonatkozóan:
\item Önállóság és önszervezés a bemelegítésben, a gyakorlásban, az edzésben és a játékban, játékvezetésben.
\item Az adott sportjáték főbb versenykörülményeinek ismerete.
\item Erős figyelemmel végrehajtott technikai elemek, taktikai megoldások, szimulálva a valódi játékszituációkat.
\item Ötletjáték és két-három tudatosan alkalmazott formáció, a csapaton belüli szerepnek való megfelelés.
\item A csapat taktikai tervének, teljesítményének szakszerű és objektív megfogalmazása.
\item A másik személy különféle szintű játéktudásának elfogadása.
\item Kreativitást, együttműködést, tartalmas, asszertív társas kapcsolatokat szolgáló mozgásos játéktípusok ismerete és célszerű használata.
\item A torna mozgásanyagában az optimális végrehajtására jellemző téri, időbeli és dinamikai sajátosságok megjelenítése.
\item Bonyolult gyakorlatelem sorok, folyamatok végrehajtása közben a mozgás koordinált irányítása.
\item Önállóan összeállított összefüggő gyakorlatok tervezése, gyakorolása, bemutatása.
\item Önálló zeneválasztás, a mozdulatok a zene időbeli rendjéhez illesztése.
\item Könnyed, plasztikus, esztétikus végrehajtás a táncos mozgásformákban.
\item A torna versenysport előnyei, veszélyei, a hozzá kapcsolódó testi képességek fejlesztésének lehetőségei ismerete.
\item Bemelegítő és képességfejlesztő gyakorlatok ismerete, a célnak megfelelő kiválasztása.
\item Optimális segítségadás, biztosítás, bíztatás.
\item Hibajavítás és annak asszertív kommunikációja.
\item Az izmok mozgáshatárát bővítő aktív és passzív eljárások ismerete.
\item A futások, ugrások és dobások képességfejlesztő hatásának felhasználása más mozgásrendszerekben.
\item Az atlétikai versenyszámok biomechanikai alapjainak ismerete.
\item Az állóképesség fejlesztésével, a lendületszerzés az izom-előfeszítések begyakorlásával a futó-, az ugró- és a dobóteljesítmények növelése.
\item Az alapvető atlétikai versenyszabályok ismerete.
\item Bemelegítés az atlétikai mozgásokhoz illeszkedően.
\item Uralom a test felett a sebesség, gyorsulás, tempóváltás, gurulás, csúszás, gördülés esetén.
\item Feladatok önálló tervezése és megoldása alternatív sporteszközökkel.
\item Az adott alternatív sportmozgáshoz szükséges edzés és balesetvédelmi alapfogalmak ismerete.
\item Az ismeretek alkalmazása az új sporttevékenységek során.
\item A szabályok és rituálék betartása.
\item Önfegyelem, az indulatok és agresszivitás kezelése.
\item Több támadási és védekezési megoldás, kombináció ismerete az álló és földharcban.
\item Magabiztos támadáselhárítás és viselkedés veszélyeztetettség esetén.
\item A bemelegítés szükségessége élettani okainak ismerete.
\item Az egészségük fenntartásához szükséges edzés, terhelés megtervezése. Tudatos védekezés a stresszes állapot ellen, feszültségek szabályozása.
\item A testtartásért felelős izmok erősítését és nyújtását szolgáló gyakorlatok ismerete, pontos gyakorlása, értő kontrollja.
\item A gerinckímélet alkalmazása a testnevelési és sportmozgásokban, kerti és házimunkákban, az esetleges sérüléses szituációk megfelelő kezelése.
\end{itemize}
\subsection{STEM tantárgy}
\paragraph{Matematika tantárgy alapján}
\begin{itemize}
\item A kombinatorikai problémához illő módszer önálló megválasztása.
\item A gráfok eszközjellegű használata problémamegoldásában.
\item Bizonyított és nem bizonyított állítás közötti különbség megértése.
\item Feltétel és következmény biztos felismerése a következtetésben.
\item A szövegben található információk önálló kiválasztása, értékelése, rendezése problémamegoldás céljából.
\item A szöveghez illő matematikai modell elkészítése.
\item A tanulók a rendszerezett összeszámlálás, a tanult ismeretek segítségével tudjanak kombinatorikai problémákat jól megoldani
\item A gráfok ne csak matematikai fogalomként szerepeljenek tudásukban, alkalmazzák ismereteiket a feladatmegoldásban is.
\item A kiterjesztett gyök- és hatványfogalom ismerete.
\item A logaritmus fogalmának ismerete.
\item A gyök, a hatvány és a logaritmus azonosságainak alkalmazása konkrét esetekben probléma megoldása céljából.
\item Egyszerű exponenciális és logaritmusos egyenletek felírása szöveg alapján, az egyenletek megoldása, önálló ellenőrzése.
\item A mindennapok gyakorlatában szereplő feladatok megoldása a valós számkörben tanult új műveletek felhasználásával.
\item Számológép értelmes használata a feladatmegoldásokban.
\item Trigonometrikus függvények értelmezése, alkalmazása.
\item Függvénytranszformációk végrehajtása.
\item Exponenciális függvény és logaritmusfüggvény ismerete.
\item Exponenciális folyamatok matematikai modelljének megértése.
\item A számtani és a mértani sorozat összefüggéseinek ismerete, gyakorlati alkalmazások.
\item Az új függvények ismerete és jellemzése kapcsán a tanulóknak legyen átfogó képük a függvénytulajdonságokról, azok felhasználhatóságáról.
\item Jártasság a háromszögek segítségével megoldható problémák önálló kezelésében.
\item A tanult tételek pontos ismerete, alkalmazásuk feladatmegoldásokban.
\item A valós problémákhoz geometriai modell alkotása.
\item Hosszúság, szög, kerület, terület, felszín és térfogat kiszámítása.
\item Két vektor skaláris szorzatának ismerete, alkalmazása.
\item Vektorok a koordináta-rendszerben, helyvektor, vektorkoordináták ismerete, alkalmazása.
\item A geometriai és algebrai ismeretek közötti összekapcsolódás elemeinek ismerete: távolság, szög számítása a koordináta-rendszerben, kör és egyenes egyenlete, geometriai feladatok algebrai megoldása.
\item Statisztikai mutatók használata adathalmaz elemzésében.
\item A valószínűség matematikai fogalma.
\item A valószínűség klasszikus kiszámítási módja.
\item Mintavétel és valószínűség.
\item A mindennapok gyakorlatában előforduló valószínűségi problémákat tudják értelmezni, kezelni.
\item Megfelelő kritikával fogadják a statisztikai vizsgálatok eredményeit, lássák a vizsgálatok korlátait, érvényességi körét.
\end{itemize}
\paragraph{Biológia-egészségtan
b változat tantárgy alapján}
\begin{itemize}
\item A tanulók megértik a környezet- és természetvédelem alapjait, elsajátítják az ökológiai szemléletet, és nyitottá válnak a környezetkímélő gazdasági- és társadalmi stratégiák befogadására.
\item Megszerzett ismereteiket a gyakorlatban, mindennapi életükben is alkalmazzák.
\item A tanulók felismerik a molekulák és a sejtalkotó részek kooperativitását, képesek a kémia, illetve a biológia tantárgyban tanult ismeretek összekapcsolására.
\item Megértik az anyag-, az energia- és az információforgalom összefüggéseit az élő rendszerekben.
\item Rendszerben látják a hormonális, az idegi és az immunológiai szabályozást, és képesek összekapcsolni a szervrendszerek működését, kémiai, fizikai, műszaki és sejtbiológiai ismeretekkel. Felismerik a biológiai, a technikai és a társadalmi szabályozás analógiáit.
\item Biológiai ismereteik alapján az ember egészségi állapotára jellemző következtetéseket képesek levonni.
\item Tudatosul bennük, hogy az ember szexuális életében alapvetőek a biológiai folyamatok, de a szerelemre épülő tartós párkapcsolat, az utódok tudatos vállalása, felelősségteljes felnevelése biztosít csak emberhez méltó életet.
\item Helyesen értelmezik az evolúciós modellt. A rendszerelvű gondolkodás alapján megértik az emberi és egyéb élő rendszerek minőségi és mennyiségi összefüggéseit.
\item Felismerik a biológia és a társadalmi gondolkodás közötti kapcsolatot.
\item Képessé és nyitottá válnak az interdiszciplináris gondolkodásra.
\item A saját életükben felismerik a biológiai eredetű problémákat, életmódjuk helyes megválasztásával, megbízható szakmai ismereteik alapján felelős egyéni és társadalmi döntéseket képesek hozni.
\end{itemize}
\paragraph{Fizika tantárgy alapján}
\begin{itemize}
\item A mechanikai fogalmak bővítése a rezgések és hullámok témakörével, valamint a forgómozgás és a síkmozgás gyakorlatban is fontos ismereteivel.
\item Az elektromágneses indukcióra épülő mindennapi alkalmazások fizikai alapjainak ismerete: elektromos energiahálózat, elektromágneses hullámok.
\item Az optikai jelenségek értelmezése hármas modellezéssel (geometriai optika, hullámoptika, fotonoptika). Hétköznapi optikai jelenségek értelmezése.
\item A modellalkotás jellemzőinek bemutatása az atommodellek fejlődésén.
\item Alapvető ismeretek a kondenzált anyagok szerkezeti és fizikai tulajdonságainak összefüggéseiről.
\item A magfizika elméleti ismeretei alapján a korszerű nukleáris technikai alkalmazások értelmezése. A kockázat ismerete és reális értékelése.
\item A csillagászati alapismeretek felhasználásával Földünk elhelyezése az Univerzumban, szemléletes kép az Univerzum térbeli, időbeli méreteiről.
\item A csillagászat és az űrkutatás fontosságának ismerete és megértése.
\item Képesség önálló ismeretszerzésre, forráskeresésre, azok szelektálására és feldolgozására.
\end{itemize}
\subsection{KULT tantárgy}
\paragraph{Idegen nyelv tantárgy alapján}
\begin{itemize}
\item A tanuló képes főbb vonalaiban és egyes részleteiben is megérteni a köznyelvi beszédet a számára ismerős témákról.
\item Képes önállóan boldogulni, véleményt mondani és érvelni a mindennapi élet legtöbb, akár váratlan helyzetében is. Stílusában és regiszterhasználatában alkalmazkodik a kommunikációs helyzethez.
\item Ki tudja magát fejezni a szintnek megfelelő szókincs és szerkezetek segítségével az ismerős témakörökben. Beszéde folyamatos, érthető, a főbb pontok tekintetében tartalmilag pontos, stílusa megfelelő.
\item Több műfajban képes részleteket is tartalmazó, összefüggő szövegeket fogalmazni ismert, hétköznapi és elvontabb témákról. Írásbeli megnyilatkozásaiban megjelennek műfaji sajátosságok és különböző stílusjegyek.
\item Képes megérteni a gondolatmenet lényegét és egyes részinformációkat a nagyrészt közérthető nyelven írt, érdeklődési köréhez kapcsolódó, lényegre törően megfogalmazott szövegekben.
\end{itemize}
\paragraph{Történelem, társadalmi és állampolgári ismeretek tantárgy alapján}
\begin{itemize}
\item Az újkori és modern kori egyetemes és magyar történelmi jelenségek, események rendszerező feldolgozásával a jelenben zajló folyamatok előzményeinek felismerése, a nemzeti öntudatra és aktív állampolgárságra nevelés.
\item A múltat és a történelmet formáló, alapvető folyamatok, ok-okozati összefüggések felismerése (pl. a globalizáció felerősödése és a lokális közösségek megerősödése) és egyszerű, átélhető erkölcsi tanulságok (pl. társadalmi kirekesztés) azonosítása, ezeknek jelenre vonatkoztatása, megítélése.
\item Az új- és modern korban élt emberek, közösségek sokoldalú élet-, gondolkodás- és szokásmódjainak azonosítása, a hasonlóságok és különbségek árnyalt felismerése, több szempontú értékelése.
\item A civilizációk története jellegzetes sémájának alkalmazása újkori és modern kori egyetemes történelemre.
\item A történelem értelmezését segítő kulcsfogalmak és egyéb egyedi fogalmak rendszeres és szakszerű alkalmazása révén, többoldalú történelmi tájékozódás és árnyalt gondolkodás.
\item Ismerje fel a tanuló, hogy az utókor, a történelmi emlékezet a nagy történelmi személyiségek tevékenységét többféle módon és szempont szerint értékeli, egyben legyen képes saját értékítélete megfogalmazásakor a közösség hosszú távú nézőpontját alkalmazni.
\item Ismerje a XIX-XX. század kisebb korszakainak megnevezését, illetve egy-egy korszak főbb jelenségeit, jellemzőit, szereplőit, összefüggéseit.
\item Ismerje a magyar történelem főbb csomópontjait az 1848-1849-es szabadságharc leverésétől az Európai Unióhoz való csatlakozásunkig.
\item Legyen képes e bonyolult történelmi folyamat meghatározó összefüggéseit, szereplőit beazonosítani, valamint legyen képes egy-egy korszak főbb kérdéseinek problémaközpontú bemutatására, elemzésére.
\item Ismerje az új- és modern korban meghatározó egyetemes és magyar történelem eseményeit, évszámait, történelmi helyszíneit.
\item Legyen képes összefüggéseket találni a térben és időben eltérő történelmi események között, különös tekintettel azokra, amelyek a magyarságot közvetlenül vagy közvetetten érintik.
\item Tudja, hogy a XIX–XX. században lényegesen átalakult Európa társadalma és gazdasága (polgárosodás, iparosodás), és ezzel párhuzamosan új eszmeáramlatok, politikai mozgalmak, pártok jelennek meg.
\item Ismerje fel, hogy az Egyesült Államok milyen körülmények között vált a mai világ vezető hatalmává, és mutasson rá az ebből fakadó ellentmondásokra.
\item Tudja a trianoni békediktátum máig tartó hatását, következményeit értékelni, és legyen képes a határon túli magyarság sorskérdéseit felismerni.
\item Tudja a demokratikus és diktatórikus államberendezkedések közötti különbségeket, legyen képes a demokratikus berendezkedés előnyeit és működési nehézségeit egyaránt felismerni és azokat elemezni.
\item Ismerje fel a tanuló a világot – és benne hazánkat is – fenyegető veszélyeket (pl. túlnépesedés, betegségek, elszegényesedés, munkanélküliség, élelmiszerválság, tömeges migráció).
\item Tudjon élni a globalizáció előnyeivel, benne az európai állampolgársággal.
\item Ismerje az alapvető emberi jogokat, valamint állampolgári jogokat és kötelezettségeket, Magyarország politikai rendszerének legfontosabb intézményeit, értse a választási rendszer működését.
\item Legyen képes ismereteket meríteni különböző ismeretforrásokból, történelmi, társadalomtudományi, filozófiai és etikai kézikönyvekből, atlaszokból, szaktudományi munkákból, legyen képes ezek segítségével történelmi oknyomozásra.
\item Jusson el kiselőadások, beszámolók önálló jegyzetelése szintjére.
\item Legyen képes az internet kritikus és tudatos használatára történelmi, filozófia- és etikatörténeti ismeretek megszerzése érdekében.
\item Legyen képes különböző történelmi elbeszéléseket (pl. emlékiratok) összehasonlítani a narráció módja alapján.
\item Legyen képes a különböző szövegek, hanganyagok, filmek stb. vizsgálatára és megítélésére a történelmi hitelesség szempontjából.
\item Legyen képes történelmi jeleneteket elbeszélni, adott esetben eljátszani különböző szempontokból.
\item Legyen képes erkölcsi kérdéseket felvető élethelyzeteket felismerni és bemutatni.
\item Fogalmazzon meg önálló véleményt társadalmi, történelmi eseményekről, szereplőkről, jelenségekről, filozófiai kérdésekről.
\item Legyen képes mások érvelésének összefoglalására, értékelésére és figyelembe vételére, a meghatározott álláspontok cáfolására, a véleménykülönbségek tisztázására, valamint a saját álláspont gazdagítására is.
\item Legyen képes történelmi-társadalmi adatokat, modelleket és elbeszéléseket elemezni a bizonyosság, a lehetőség és a valószínűség szempontjából.
\item Legyen képes összehasonlítani társadalmi-történelmi jelenségeket strukturális és funkcionális szempontok alapján. Legyen képes értékrendek összehasonlítására, saját értékek tisztázására.
\item Legyen képes történelmi-társadalmi témákat vizuálisan ábrázolni, esszét írni (filozófiai kérdésekről is), ennek kapcsán kérdéseket világosan megfogalmazni.
\item Legyen képes a történelmi időben történő sokoldalú tájékozódásra.
\item Legyen képes a különböző időszakot bemutató történelmi térképek összehasonlítása során a változások (területi változások, népsűrűség, vallási megosztottság stb.) hátterének feltárására.
\item Legyen képes a nemzet, a kisebbség, a nemzetiség fogalmának és a helyi társadalom fogalmának szakszerű használatára, tudjon érvelni a társadalmi felelősségvállalás, illetve a szolidaritás fontossága mellett.
\item Legyen képes átlátni a nemzetgazdaság, a bankrendszer, a vállalkozási formák működésének legfontosabb szabályait.
\item Legyen képes munkavállalással összefüggő, a munkaviszonyhoz kapcsolódó adózási, egészség- és társadalombiztosítási kötelezettségek, illetve szolgáltatások rendszerét átlátni.
\end{itemize}
\paragraph{Magyar nyelv és irodalom tantárgy alapján}
\begin{itemize}
\item A tanuló felismeri, és értő módon használja a tömegkommunikációs, illetve az audiovizuális, informatikai alapú szövegeket. Az értő, kritikus befogadáson kívül önálló szövegalkotás néhány publicisztikai, audiovizuális és informatikai hátterű műfajban, a képi elemek, lehetőségek és a szöveg összekapcsolásában rejlő közlési lehetőségek kihasználásával.
\item Szövegelemzési, szövegértelmezési jártassággal rendelkezik a tanult leíró nyelvtani, szövegtani, jelentéstani, pragmatikai ismeretek alkalmazásában és az elemzést kiterjeszti a szépirodalmi szövegek mellett a szakmai-tudományos, publicisztikai, közéleti (audiovizuális, informatikai alapú) szövegek feldolgozására, értelmezésére is.
\item A tanuló rendszeresen használja a könyvtárat, a különféle (pl. informatikai technológiákra épülő) információhordozókat, rendelkezik a képességgel, hogy kellő problémaérzékenységgel, kreativitással és önállósággal igazodjon el az információk világában; értelmesen és értékteremtően tudjon élni az önképzés lehetőségeivel.
\item Bizonyítja különféle szövegek megértését a szöveg felépítésére, grammatikai jellemzőire, témahálózatára, tagolására irányuló elemzéssel.
\item Képes olvasható, rendezett írásra.
\item Szóbeli és írásbeli kommunikációs helyzetekben megválasztja a megfelelő hangnemet, nyelvváltozatot, stílusréteget. Alkalmazza a művelt köznyelv (regionális köznyelv), illetve a nyelvváltozatok nyelvhelyességi normáit, képes felismerni és értelmezni az attól eltérő nyelvváltozatokat.
\item A hivatalos írásművek műfajaiban képes önálló szövegalkotásra (pl. önéletrajz, motivációs levél).
\item Alkalmazza az idézés szabályait és etikai normáit.
\item Bizonyítja a magyar nyelv rendszerének és történetének ismeretét, a grammatikai, szövegtani, jelentéstani, stilisztikai-retorikai, helyesírási jelenségek önálló fölismerését, a tanultak tudatos alkalmazását.
\item Átfogó ismerettel bír a nyelv és társadalom viszonyáról, illetve a nyelvi állandóság és változás folyamatáról. Anyanyelvi műveltségének fontos összetevője a tájékozottság a magyar nyelv eredetéről, rokonságáról, történetének főbb korszakairól; a magyar nyelv és a magyar művelődés kapcsolatának tudatosítása.
\item Képes memoriterek szöveghű tolmácsolására tudatos, kifejező szövegmondással.
\item Képes szövegek kapcsolatainak és különbségeinek felismerésére, értelmezésére (pl. tematikus, motivikus kapcsolatok, utalások, nem irodalmi és irodalmi szövegek, tények és vélemények összevetése), e képességek alkalmazására elemző szóbeli és írásbeli műfajokban.
\item Tudja alkalmazni a művek műfaji természetének, poétikai jellemzőinek megfelelő szövegfeldolgozási eljárásokat, megközelítési módokat.
\item Fel tudja ismerni a szépirodalmi és nem szépirodalmi szövegekben megjelenített értékeket, erkölcsi kérdéseket, álláspontokat, motivációkat, magatartásformákat, képes ezek értelmezésére, önálló értékelésére.
\item Képes erkölcsi kérdések, döntési helyzetek megnevezésére, példával történő bemutatására. Részt tud venni elemző beszélgetésekben, ennek tartalmához hozzájárul saját véleményével. Képes a felismert jelenségek értelmezésére, következtetések megfogalmazására.
\item Tájékozott az olvasott, feldolgozott lírai alkotások különböző műfajaiban, poétikai megoldásaiban, kompozíciós eljárásaiban.
\item Képes definíció, magyarázat, értekezés (kisértekezés) készítésére az olvasmányaival, a felvetett és tárgyalt problémákkal összefüggésben, maga is meg tud fogalmazni kérdéseket, problémákat.
\item Képes az irodalmi művekben megjelenő álláspontok azonosítására, követésére, megvitatására, összehasonlítására, eltérő vélemények megértésére, újrafogalmazására.
\item Képes tudásanyagának többféle szempontot követő megfogalmazására írásban a magyar és a világirodalom kiemelkedő alkotóiról.
\item Meggyőzően be tudja mutatni a tanult irodalomtörténeti korszakok és stílusirányzatok sajátosságait.
\item Képes a feldolgozott epikai, lírai és drámai művek jelentésének, erkölcsi tartalmának tárgyszerű ismertetésére.
\item Be tud mutatni műveket, alkotókat a 20. század magyar és világirodalmából, továbbá a kortárs irodalomból.
\item Írásban és szóban egyaránt bizonyítani tudja alkotói pályaképek ismeretét az alkotói életút jelentős tényeinek, a művek tematikai, formabeli változatosságának bemutatásával.
\item Felismeri különböző alkotók hatását az irodalmi hagyományban, ezzel összefüggésben képes művek közötti kapcsolatok, témák fölismerésére és értékelésére, az evokáció, az allúzió, a parafrázis, a palimpszeszt, vagyis az intertextualitás példáinak bemutatására.
\item Képes különböző korokban keletkezett alkotások tematikai, poétikai szempontú értelmezésére, összevetésére.
\item Olvasottsága kiterjed az online médiára és a populáris regiszter jegyében született internetes és nyomtatott sajtóra, alkotásokra is.
\item Tisztában van azzal, hogy az irodalmi művek a modernségben elsősorban nyelvi képződmények („fikciók”), amelyek grammatikai-poétikai összetettségük minőségétől függően fejtenek ki esztétikai hatást, hoznak létre örömérzést a lélekben.
\item Nem idegenkedik saját, kreatív vagy funkcionális szövegek létrehozásától, sőt, örömmel végez ilyen tevékenységet.
\item Szívesen jár színházba, moziba, kiállításra, hangversenyre, jó ízléssel választ a klasszikus és kortárs drámairodalom elérhető előadásaiból, illetve filmes adaptációiból, egyéb művészeti alkotásaiból.
\end{itemize}


\bibliography{references.bib}{}
\label{sec:bibliographyk}
\bibliographystyle{apalike}

\end{document}
