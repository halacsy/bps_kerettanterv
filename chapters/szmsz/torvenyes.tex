A működés az intézményben való benntartózkodásának rendje

A pedagógiai munka belső ellenőrzésének rendje

A belépés és benntartózkodás rendje azok részére, akik nem állnak jogviszonyban a nevelési-oktatási intézménnyel

Intézményegységgel való kapcsolattartás rendje


A vezetők és a szervezeti egységek közötti kapcsolattartás rendje, formája, továbbá a vezetők közötti feladatmegosztást, a kiadmányozás és a képviselet szabályait, a szervezeti egységek közötti kapcsolattartás rendje

Az intézményvezető vagy intézményvezető-helyettes akadályoztatása esetén a helyettesítés rendje

A vezetők és az iskolaszék, az intézményi tanács, az iskolai szülői szervezet, közösség közötti kapcsolattartás formája, rendje.

A nevelőtestület feladatkörébe tartozó ügyek átruházására, továbbá a feladatok ellátásával megbízott beszámolására vonatkozó rendelkezések.

A külső kapcsolatok rendszere, formája és módja, beleértve a pedagógiai szakszolgálatokkal, a pedagógiai szakmai szolgálatokkal, a gyermekjóléti szolgálattal, valamint az iskola-egészségügyi ellátást biztosító egészségügyi szolgáltatóval való kapcsolattartás

Az ünnepélyek, megemlékezések rendje, a hagyományok ápolásával kapcsolatos feladatok.

A szakmai munkaközösségek együttműködése, kapcsolattartásának rendje, részvétele a pedagógusok munkájának segítésében.

A rendszeres egészségügyi felügyelet és ellátás rendje.

Az intézményi védő, óvó előírásokat.

Bármely rendkívüli esemény esetén szükséges teendőket.

Hol, milyen időpontban lehet tájékoztatást kérni a pedagógiai programról?

Azon ügyek, amelyekben a szülői szervezetet, közösséget az SZMSZ véleményezési joggal ruházza fel.

A nevelési-oktatási intézményben a tanulóval szemben lefolytatásra kerülő fegyelmi eljárást megelőző egyeztető eljárás, valamint a tanulóval szemben lefolytatásra kerülő fegyelmi eljárás részletes szabályait.

Az elektronikus úton előállított papíralapú nyomtatványok hitelesítésének rendje.

Az elektronikus úton előállított, hitelesített és tárolt dokumentumok kezelése.

Az intézményvezető feladat- és hatásköréből leadott feladat- és hatásköröket, munkakörileírás-mintákat.


Az egyéb foglalkozások célja, szervezeti formái, időkeretei.

A felnőttoktatás formái

A diákönkormányzat, a diákképviselők, valamint az iskolai vezetők közötti kapcsolattartás formája és rendje, a diákönkormányzat működéséhez szükséges feltételek (helyiségek, berendezések használata, költségvetési támogatás biztosítása).

Az iskolai sportkör, valamint az iskola vezetése közötti kapcsolattartás formáka és rendje.



A gyermekek, tanulók egészségét veszélyeztető helyzetek kezelésére irányuló eljárásrend.

Az iskolai könyvtár SZMSZ-e

Azok a nevelési-oktatási intézmény biztonságos működését garantáló szabályok, amelyek megtartása kötelező az intézmény területén tartózkodó szülőknek, valamint az intézménnyel kapcsolatban nem álló más személyeknek.

Ha iskolaszék szék nem működik, az SZMSZ elfogadásakor az óvodai, iskolai, kollégiumi szülői szervezet, közösség véleményét kell beszerezni.

