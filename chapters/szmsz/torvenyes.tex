

\section{A működés és az intézményben való benntartózkodásának rendje}
A mikroiskolák telephelyeinek és a mikroiskolák által kizárólagosan használt tanulási pontok rendjét a tanulásszervező tanároknak kell kialakítani a gyerekekkel közösen. Ki kell térni arra, hogy mikor nyit az épület, mikor zár, meddig maradhat ott gyerek, mi történjen a késéssel.

Addig, amig más rendet nem alakított ki a mikroiskola, addig a következők érvényesek: 9 órától 4 óráig tartanak a struktúrált foglalkozások. Ha valaki késik, vagy aznap nem tud bejönni, akkor értesíte a tanárokat a beiratkozáskor megadott emailcímen.

\section{A pedagógiai munka belső ellenőrzésének rendje}
A pedagógia munkát a keretttantervben leírt monitorozó rendszer alapján ellenőrizzük: amig a gyerekek fejlődnek, a szülők bizonságban érzik magukat és a gyerekeket, és a tanárok hatékonyaknak érzik magukat, és mindeközben minden résztvevő boldog, akkor jól végezzük a munkánkat.

\section{A belépés és benntartózkodás rendje azok részére, akik nem állnak jogviszonyban a nevelési-oktatási intézménnyel.}
Egy mikroiskolába akkor menj be, ha a előtte a tanárokkal megbeszélted, hogy nem zavarod a csoportok működését. Ha nem tudtad megbeszélni, akkor kérdezd meg tőlük, mikor belépsz. Ha nem tudod megkérdezni, vagy éppen zavarsz, akkor várj.


\section{Intézményegységgel való kapcsolattartás rendje.}
Egyelőre nincs külön szabály a kapcsolattartásra. Beszéljünk egymással többet, mint amit a folyamatos működés csak úgy adna. Egymást tudjuk segíteni és támogatni.



\section{A vezetők és a szervezeti egységek közötti kapcsolattartás rendje, formája, továbbá a vezetők közötti feladatmegosztás, a kiadmányozás és a képviselet szabályai, a szervezeti egységek közötti kapcsolattartás rendje.}
Az iskolában nem szervezeti egységek tartanak kapcsolatot, hanem emberek. Ha két egység, csapat között kapcsolatra van szükség, akkor ki kell jelölni, azokat, akik képviselik a csapatokat egymásnál.

A vezetők, és a nem vezetők ugyanúgy osztanak meg feladatot. Az iskolában csapatoknak van feladata, ezeket bárki elvállalhatja. Felelősségeket a csapat szerepeken keresztül delegál. Szerepekre szereplőket a hozzájárulás alapú döntéshozási módszerrel kell választani. A delegálás, feladatvállalás akkor történik meg, ha mindkét fél ezt elfogadja.

Egy megbeszélés után érdemes átismételni, hogy miben állapodtak meg a felek.


\section{Az intézményvezető vagy intézményvezető-helyettes akadályoztatása esetén a helyettesítés rendje.}
Az intézményvezető, mint minden vezető akkor végzi jól a munkáját, ha nélküle is működik a szervezet, mert a szerepek, feladatok le vannak osztva.

Ha olyan sokáig akadályoztatva van az intézményvezető, akkor a fenntartó képviseli az iskolát addig, amíg új intézményvezetőt nem talál a szervezet.

\section{A vezetők és az iskolaszék, az intézményi tanács, az iskolai szülői szervezet, közösség közötti kapcsolattartás formája, rendje.}
Jelenleg nem működik iskolaszék, intézményi tanács az iskolában. A szülői közösségek mikroiskolánként maguk döntenek arról, hogyan tartják a kapcsolatot.


\section{A nevelőtestület feladatkörébe tartozó ügyek átruházására, továbbá a feladatok ellátásával megbízott beszámolására vonatkozó rendelkezések.}
A feladat meghatározásakor érdemes arról is beszélni, hogy a feladat elvégzése után kit, miről és hogyan érdemes értesíteni (azaz, kik az érintettek).

Felelősségek átruházása szerepek kialakításával történhet. 

\section{A külső kapcsolatok rendszere, formája és módja, beleértve a pedagógiai szakszolgálatokkal, a pedagógiai szakmai szolgálatokkal, a gyermekjóléti szolgálattal, valamint az iskola-egészségügyi ellátást biztosító egészségügyi szolgáltatóval való kapcsolattartás}
Minden mentor tanár képviselheti a mentorált gyerekeket és családokat, a mikroiskolát a külső szolgáltatók felé.

\section{Az ünnepélyek, megemlékezések rendje, a hagyományok ápolásával kapcsolatos feladatok.}
A mikroiskolák maguk döntenek arról, hogy a közösségük milyen ünnepeket, megemlékezéseket tart meg. A  Budapest School hagyományosan egy tanévnyitó ünneppel indítja el az évet, és egy hasonló ünnepel zárja. Erre a két rendezvényre az összes mikroiskolába járó gyerek és a szülők is hivatalosak. 

Év közben a nemzeti ünnepek beépülnek a tanulási napok és szünetek rendjébe, és az egyes mikroiskolák tanulásszervezőinek feladata, hogy az ehhez kapcsolódó megemlékezéseket saját belátásuk szerint megszervezzék. A közösség rítusai és hagyományai kiemelt szerepet kapnak, ezek megjelennek a mindennapokban.  

\section{A szakmai munkaközösségek együttműködése, kapcsolattartásának rendje, részvétele a pedagógusok munkájának segítésében.}
Az egyes mikroikolák tanulásszervezői korcsoporti, tematikus, vagy feladatorintált szempontok alapján csoportokat alkothatnak, hogy így segítsék egymás munkáját. A Budapest School kiemelten fontosnak tartja a tudás- és tapasztalatmegosztást, ezért ha egy tanulásszervező közösség mások számára is hasznos eredményt ér el, vagy tapasztalatot szerez valamely tanulási területen, felelőssége, hogy a közös kommmunikációs csatornákon más mikroiskolák tanulásszervezőivel is megossza ezeket. 


\section{A rendszeres egészségügyi felügyelet és ellátás rendje.}
Lásd pedagógia program.

\section{A balesetek megelőzését szolgáló védő, óvó előírások}
\begin{enumerate}

\item A tanulókkal a tanév első tanítási hetében ismertetni kell a tűz- és balesetvédelmi
szabályokat.
\item A tanítási idő alatt 13:30-ig a tanulók csak a mentor tanár engedélyével hagyhatják el
az épület területét. Utána is csak a szülő által engedélyezett módon és kísérővel. 
\item Az iskola épületét, felszerelését minden tanulónak óvni kell. Az okozott kárért a jogszabályokban és az iskolai működési rendben meghatározott anyagi felelősséggel
tartozik a kárt okozó diák. A kár rendezése a szülőt, illetve a gondviselőt terheli.
\item Mindenkinek kötelessége a másik testi- és lelki épségére vigyázni.
\item A modulvezető tanárok felelőssége a foglalkozások olyan megszervezése, vezetése, hogy a
baleset lehetősége a lehető legjobban kizárható legyen. Szintén feladatuk, hogy az
elvégzendő feladat jellegéből fakadó veszélyekre (pl. testnevelés órák, laborfoglalkozások) a diákok figyelmét felhívják, az elhárítás módszereit elsajátíttassák.
\item Az iskola egészének balesetmentes működéséért a tanulásszervezők a felelősek minden egyes mikroiskolában.
\item Amennyiben a tanítási időben, az iskolában egészségügyi problémája merül fel egy
tanulónak, vagy baleset történik, a tanár értesíti a fenntartót, a mentortanárt és a tanuló szüleit.

\end{enumerate}
\section{Bármely rendkívüli esemény esetén szükséges teendők}
Rendkívüli esemény bekövetkeztekor a mikroiskola tanulásszervezőinek azonnal ki kell jelölniük egy személyt, aki a krízis kommunikációjáért és egy másik személyt, aki a helyzet azonnali kezeléséért felel.

A krízis kommunikációjáért felelős mielőbb tájékoztatja  az érintetteket (gyerekek, tanárok, szülők).

A krízis kezeléséért felelős személy döntési helyzetbe kerül és az ő általa kidolgozott terv alapján kell a krízis helyzet megoldása felé haladni. Amennyiben úgy ítéli meg, hogy nem tudja a helyzetet a mikroiskolán belül kezelni, úgy a fenntartóhoz kell fordulnia segítségért. 
Amennyiben olyan rendkívüli esemény (pl. bombariadó, csőtörés, tűz, stb.) történik, amely veszélyeztetheti a gyerekeket, a veszély tudomásra jutását és vészjelzővel történt riasztást követően az épület teljes kiürítését azonnal el kell végezni, és értesíteni a megfelelő hatóságokat. 

A rendkívüli eseményről jegyzőkönyvet kell felvenni és post mortem retrospektív elemzést kell végezni. Ennek felelőse a kijelölt krízis kommunikátor. A tanév elején a tanulókkal meg kell ismertetni a rájuk vonatkozó menekülési tervet. Ennek útvonalát az iskolai dokumentumokban fel kell tüntetni. A tanév folyamán egy gyakorlatot kell tartani. 


\section{Hol, milyen időpontban lehet tájékoztatást kérni a pedagógiai programról?}
A pedagógiai program mindenki számára nyilvános a weboldalon.

\section{Azon ügyek, amelyekben a szülői szervezetet, közösséget az SZMSZ véleményezési joggal ruházza fel.}
A szülőknek minden a gyereküket érintő kérdésben lehet  véleménye és ezek meghallgatása és meghallása is fontos. A mentor tanár az elsődleges közvetítő, akivel a szülő a gyerekét érintő kérdésekben egyeztethet. Amennyiben olyan kérdésről van szó, amely több szülőt, vagy a teljes szülői közösséget érint, annak megvitatására a szülői körök alkalmasak, melyek minden mikroiskola trimeszterenként legalább egy alkalommal megszervez. 

\section{A nevelési-oktatási intézményben a tanulóval szemben lefolytatásra kerülő fegyelmi eljárást megelőző egyeztető eljárás, valamint a tanulóval szemben lefolytatásra kerülő fegyelmi eljárás részletes szabályait.}
Egyelőre nincs az iskolában fegyelmi eljárás rend. A pedagógiai program konfliktuskezelés fejezete ad segítséget arra az esetben, ha valaki viselkedése zavarja a többieket.

\section{Az elektronikus úton előállított papíralapú nyomtatványok hitelesítésének rendje.}
Az iskola az oktatási ágazat irányítási rendszerével a Közoktatási Információs Rendszeren (KIR) keresztül elektronikusan előállított, hitelesített és tárolt dokumentumrendszert alkalma a 229/2012. (VIII.28.) Kormányrendelet előírásainak megfelelően. A rendszerben alkalmazott fokozott biztonságú elektronikus aláírást általánosan az intézmény pedagógiai vezetője (igazgató) és igazgatóhelyettese (általánosan az ügyvezető), az ezzel a feladattal megbízott informatikus alkalmazhatja. A szakmai vizsgákkal kapcsolatban az igazgató (pedagógiai vezető), az igazgatóhelyettes (képzési vezető), a tagozatvezető és a gyakorlati oktatás vezetője jogosult a dokumentumok hitelesítésére. Az elektronikus rendszer használata mellett ki kell nyomtatni és az irattárban kell elhelyezni az alábbi dokumentumok papír alapú másolatát: - elektronikus napló - a szakmai vizsgák osztályozó íveit Az elektronikus úton előállított és kinyomtatott dokumentumokat az intézmény pecsétjével és az intézményvezető aláírásával hitelesített formában kell tárolni. A tárolás az iskolatitkár feladata, módja az iskolában szokásos iktatott irattári anyagok kezelésével egyezik meg. Az egyéb elektronikusan megküldött adatok írásbeli tárolása, hitelesítése nem szükséges. A dokumentumokat a KIR rendszerében, továbbá az iskola titkársági számítógépén egy külön e célra létrehozott mappában kell tárolni. A mappához való hozzáférésre az intézményvezető által felhatalmazott személyek jogosultak. 


\section{Az elektronikus úton előállított, hitelesített és tárolt dokumentumok kezelése.}
Jelenleg még nincs rá folyamat kialakítva.

\section{Az intézményvezető feladat- és hatásköréből leadott feladat- és hatásköröket, munkakörileírás-mintákat.}
Gyerekek felvételéről, a napirendek kialakításáról, a modulrendszerről, a mikroiskolák mindennapjait érintő kérdésekről a mikroiskola tanulásszervező tanárai döntenek.

\section{Az egyéb foglalkozások célja, szervezeti formái, időkeretei.}
Egyelőre nincsenek egyéb foglalkozások tervezve.

\section{A felnőttoktatás formái}
Az iskola jelenleg nem tervez felnőttoktatást.

\section{A diákönkormányzat, a diákképviselők, valamint az iskolai vezetők közötti kapcsolattartás formája és rendje, a diákönkormányzat működéséhez szükséges feltételek (helyiségek, berendezések használata, költségvetési támogatás biztosítása).}

Jelenleg nem működik még az iskolában diákönkormányzet.

\section{Az iskolai sportkör, valamint az iskola vezetése közötti kapcsolattartás formáka és rendje.}
Jelenleg nem működik az iskolában iskolai sportkör.

\section{A gyermekek, tanulók egészségét veszélyeztető helyzetek kezelésére irányuló eljárásrend.}
Lásd tyúklépések fejezet.

\section{Az iskolai könyvtár SZMSZ-e}
Az iskola a Szabó Ervin könyvtár szolgáltatásait veszi igénybe, saját könyvtára nincs. 

\section{Azok a nevelési-oktatási intézmény biztonságos működését garantáló szabályok, amelyek megtartása kötelező az intézmény területén tartózkodó szülőknek, valamint az intézménnyel kapcsolatban nem álló más személyeknek.}
Egyelőre nincsenek ilyen szabályok még.


\section{Ha iskolaszék szék nem működik, az SZMSZ elfogadásakor az óvodai, iskolai, kollégiumi szülői szervezet, közösség véleményét kell beszerezni.}
Jelenleg nincsenek az iskolában szülők, mert most indul az iskola. A következő SZMSZ módosításkor remélhetőleg már lesznek.