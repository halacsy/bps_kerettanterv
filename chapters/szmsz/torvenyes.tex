

\section{A működés az intézményben való benntartózkodásának rendje}
\section{A pedagógiai munka belső ellenőrzésének rendje}
A pedagógia munkát a keretttantervben leírt monitorozó rendszer alapján ellenőrizzük: amig a gyerekek fejlődnek, a szülők bizonságban érzik magukat és a gyerekeket, és a tanárok hatékonyaknak érzik magukat, és mindeközben minden résztvevő boldog, akkor jól végezzük a munkánkat.

\section{A belépés és benntartózkodás rendje azok részére, akik nem állnak jogviszonyban a nevelési-oktatási intézménnyel.}
Egy mikroiskolába akkor menj be, ha a előtte a tanárokkal megbeszélted, hogy nem zavarod a csoportok működését. Ha nem tudtad megbeszélni, akkor kérdezd meg tőlük, mikor belépsz. Ha nem tudod megkérdezni, vagy éppen zavarsz, akkor várj.


\section{Intézményegységgel való kapcsolattartás rendje.}
Egyelőre nincs külön szabály a kapcsolattartásra. Beszéljünk egymással többet, mint amit a folyamatos működés csak úgy adna. Egymást tudjuk segíteni és támogatni.



\section{A vezetők és a szervezeti egységek közötti kapcsolattartás rendje, formája, továbbá a vezetők közötti feladatmegosztás, a kiadmányozás és a képviselet szabályai, a szervezeti egységek közötti kapcsolattartás rendje.}
Az iskolában nem szervezeti egységek tartanak kapcsolatot, hanem emberek. Ha két egység, csapat között kapcsolatra van szükség, akkor ki kell jelölni, azokat, akik képviselik a csapatokat egymásnál.

A vezetők, és a nem vezetők ugyanúgy osztanak meg feladatot. Az iskolában csapatoknak van feladata, ezeket bárki elvállalhatja. Felelősségeket a csapat szerepeken keresztül delegál. Szerepekre szereplőket a hozzájárulás alapú döntéshozási módszerrel kell választani. A delegálás, feladatvállalás akkor történik meg, ha mindkét fél ezt elfogadja.

Egy megbeszélés után érdemes átismételni, hogy miben állapodtak meg a felek.


\section{Az intézményvezető vagy intézményvezető-helyettes akadályoztatása esetén a helyettesítés rendje.}
Az intézményvezető, mint minden vezető akkor végzi jól a munkáját, ha nélküle is működik a szervezet, mert a szerepek, feladatok le vannak osztva.

Ha olyan sokáig akadályoztatva van az intézményvezető, akkor a fenntartó képviseli az iskolát addig, amíg új intézményvezetőt nem talál a szervezet.

\section{A vezetők és az iskolaszék, az intézményi tanács, az iskolai szülői szervezet, közösség közötti kapcsolattartás formája, rendje.}
Jelenleg nem működik iskolaszék, intézményi tanács az iskolában. A szülői közösségek mikroiskolánként maguk döntenek arról, hogyan tartják a kapcsolatot.


\section{A nevelőtestület feladatkörébe tartozó ügyek átruházására, továbbá a feladatok ellátásával megbízott beszámolására vonatkozó rendelkezések.}
A feladat meghatározásakor érdemes arról is beszélni, hogy a feladat elvégzése után kit, miről és hogyan érdemes értesíteni (azaz, kik az érintettek).

Felelősségek átruházása szerepek kialakításával történhet. 

\section{A külső kapcsolatok rendszere, formája és módja, beleértve a pedagógiai szakszolgálatokkal, a pedagógiai szakmai szolgálatokkal, a gyermekjóléti szolgálattal, valamint az iskola-egészségügyi ellátást biztosító egészségügyi szolgáltatóval való kapcsolattartás}
\section{Az ünnepélyek, megemlékezések rendje, a hagyományok ápolásával kapcsolatos feladatok.}
\section{A szakmai munkaközösségek együttműködése, kapcsolattartásának rendje, részvétele a pedagógusok munkájának segítésében.}
\section{A rendszeres egészségügyi felügyelet és ellátás rendje.}
\section{Az intézményi védő, óvó előírásokat.}
\section{Bármely rendkívüli esemény esetén szükséges teendőket.}
\section{Hol, milyen időpontban lehet tájékoztatást kérni a pedagógiai programról?}
\section{Azon ügyek, amelyekben a szülői szervezetet, közösséget az SZMSZ véleményezési joggal ruházza fel.}
\section{A nevelési-oktatási intézményben a tanulóval szemben lefolytatásra kerülő fegyelmi eljárást megelőző egyeztető eljárás, valamint a tanulóval szemben lefolytatásra kerülő fegyelmi eljárás részletes szabályait.}
\section{Az elektronikus úton előállított papíralapú nyomtatványok hitelesítésének rendje.}
\section{Az elektronikus úton előállított, hitelesített és tárolt dokumentumok kezelése.}
\section{Az intézményvezető feladat- és hatásköréből leadott feladat- és hatásköröket, munkakörileírás-mintákat.}
\section{Az egyéb foglalkozások célja, szervezeti formái, időkeretei.}
\section{A felnőttoktatás formái}
\section{A diákönkormányzat, a diákképviselők, valamint az iskolai vezetők közötti kapcsolattartás formája és rendje, a diákönkormányzat működéséhez szükséges feltételek (helyiségek, berendezések használata, költségvetési támogatás biztosítása).}
\section{Az iskolai sportkör, valamint az iskola vezetése közötti kapcsolattartás formáka és rendje.}
\section{A gyermekek, tanulók egészségét veszélyeztető helyzetek kezelésére irányuló eljárásrend.}
\section{Az iskolai könyvtár SZMSZ-e}
\section{Azok a nevelési-oktatási intézmény biztonságos működését garantáló szabályok, amelyek megtartása kötelező az intézmény területén tartózkodó szülőknek, valamint az intézménnyel kapcsolatban nem álló más személyeknek.}
\section{Ha iskolaszék szék nem működik, az SZMSZ elfogadásakor az óvodai, iskolai, kollégiumi szülői szervezet, közösség véleményét kell beszerezni.}