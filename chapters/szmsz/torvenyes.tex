\section{Bevezető}
% forras: https://mindsetpszichologia.hu/2017/11/29/fonok-nelkul-szabadon-a-holacracy-nyomaban/

A régen alapított iskolák majdnem kivétel nélkül a hierarchikus modell szerint működnek: főnök-beosztott viszonyok, felülről jövő döntések és autoritás határozza meg a mindennapi folyamatokat. Az állandóság és a biztos siker érdekében ezek az iskolák a centralizált hatalmat alkalmazzák: a főnök dönt, a beosztottak végrehajtanak. Ez a modell hatásos volt, amikor azt kitalálták, de manapság világunk annyira dinamikus, hogy az iskoláknak meg kell tanulni alkalmazkodni és gyorsan reagálni a változásokra. Utóbbi pedig egy hierarchikus rendszerrel nem érhető el.

Jelenleg az iskolák gyakran újabb és újabb kihívásokkal találják szemben magukat: növekszik a komplexitás, előtérbe kerül a transzparencia, javulnak a kommunikációs lehetőségek, rövidülnek az idő korlátai, illetve gazdasági, környezetbeli, jogi instabilitás van jelen. Mindezek mellett nő a nyomás is, hogy önfenntartóak, környezettudatosak és etikusak legyenek a szervezetek. A fentről lefelé jövő döntések és a „bejósol és kontrollál” alapú folyamatok gyakran elbuknak, ha az alkalmazkodásról, az agilitásról vagy a gyors változásról van szó, valamint elnyomják a munkavállalók kreativitását és lelkesedését. 

\section{A Budapest School szervezeti modellje, szociokrácia}

A szociokrácia egy olyan új „szociális technológiát” biztosít a szervezetek irányítására és működtetésére, amelyet a hagyományos, hierarchikus szervezetektől eltérő alapvető szabályok határoznak meg.

Egy öngazdálkodó gyakorlat a célirányos és változásra fogékony szervezeteknek.

A megközelítés – az agilis és lean megoldásokhoz hasonlóan – „just in time”, vagyis épp a kellő pillanatban reagál a felmerülő változásokra és lehetőségekre. Mindezt pedig a vállalat minden szintjén önállóan teszi.

Az iskolai szervezet alapelemei a következők:
\begin{itemize}
    \item egy keretrendszer, ami lefekteti a „játékszabályokat” és újraosztja a hatalmat,
    \item egy módszer a szervezet újrastrukturálásához és az emberek szerepeinek és önállóságának meghatározásához,
    \item egy egyedülálló döntéshozatali folyamat a hagyományos döntési folyamatok felfrissítéséhez,
\end{itemize}

\section{Dinamikusan változó szerepek}

A klasszikus iskolában minden alkalmazott rendelkezik egy munkaköri leírással, mely felsorolja feladatait és hatáskörét. Ez általában a legkevésbé sincs összhangban a nap mint nap végzett feladatokkal. A Budapest School iskálban a dolgozók gyakran számos szereppel rendelkeznek, ezek pedig lehetnek akár teljesen különböző munkacsapatokon belül is. A szerepek folyamatos alakuláson mennek keresztül és frissülnek a csapat résztvevői által. Egy szerepkör tehát nem lesz szorosan egy emberhez kapcsolva: mindig az látja majd el, aki ért hozzá, van szabad kapacitása és elvállalja. Ezzel pedig sok személyes konfliktus kerülhető el. Ahogy például
\emph{a fociban is pontosan tudod, hogy a labdát a csatárnak kell passzolnod}.

Nem azért, mert jóban vagytok, hanem mert ő van a legjobb helyzetben, hogy gólt lőjön a csapatnak. Még ha nem is vagy vele jóban, nem kedveled vagy ki nem állhatod, akkor is neki fogsz passzolni, mert a játék stratégiája szerint így kell tenned. Ugyanígy az iskolánkban az egyes szerepek rendelkeznek hatalommal, nem pedig a személyek. Ez azt is jelenti, hogy a szerepek és a hatalom állandóan változhatnak anélkül, hogy a játékszabályokat megsértenénk. Nincs állandó főnök-beosztott viszony.

\section{Szétosztott hatalom}

A hagyományos iskolai struktúrában az igazgató, a menedzsment és a vezetőség tagjai nem delegálnak hatalmat. Minden döntés az ő kezeik között folyik át. A Budapest School iskolában a hatalom ténylegesen elosztásra kerül, így a hierarchia helyett egymással kapcsolatban lévő és kommunikáló, de önállóan döntő csapatok (úgynevezett körök) a döntéshozók. A körök közötti kapcsolat jelentősen meg tudja növelni az iskolai kapacitást az alkalmazkodáshoz.

\section{Állandó tyúklépések}

Budapest School iskolában a szervezeti struktúra minden hónapban felülvizsgálásra kerülhet: megvizsgálják, hogy az adott szerepek milyen feladatokkal és döntésekkel járnak. A változások tömérdek apró lépésben történnek, szünet nélkül, folyamatosan. Így a csapat kihasználhatja a lehetőségeket, hogy tanuljon az esetleges hibákból, fejlessze önmagát és tökéletesítse a folyamatokat. Ahogy Alexis Gonzales-Black, a Zappos munkatársa mondja (akik talán a legelsők között csatlakoztak a holacracy átültetéséhez), „a holacracy nem fogja megoldani helyetted a problémáidat; viszont egy jó eszköz arra, hogy te megoldhasd őket saját magadnak.”

Mindenki számára átlátható szerepek

„Mi mindig így csináljuk” – hangzik el számos vállalatnál a válasz, hogy ez vagy az miért éppen így történik. Gyakran senki sem tudja, hogy az adott szabály miért létezik, mi célt szolgál, ki döntött róla vagy ki tudná megváltoztatni. Ez pedig a hatalom szétosztását lehetetlenné teszi. Holokráciában

a hatalom nem a csoport élén álló vezetők kezében van,

hanem az expliciten definiált folyamatokhoz kötődik. Ezek a „játékszabályok” mindenki számára elérhetőek és ismertek, legyen szó akár régi motorosokról, akár az újoncakról.

Evan Williams, a Twitter társalapítója szerint „a holacracy egyik alapköve, hogy az implicitet explicitté kell tenni. A legnagyobb feladat tehát a transzparencia és az átláthatóság megalkotása – ki miért felelős, ki milyen döntést hozhat. Erre pedig van egy rendszer, ami meghatározza és meg is változtathatja ezeket.”

A holacracyval tehát nem káoszt és hatalmi játszmákat generálunk, hanem

eltöröljük a hagyományos hierarchiából fakadó lassú folyamatokat,

az átláthatatlan felelősségi köröket és szabad utat engedünk a kreativitásnak, az önállóságnak. Azáltal, hogy felhatalmazza az embereket arra, hogy értelmes döntéseket hozhassanak és részt vehessenek a változásban, a holakrácia felszabadítja a szervezet kihasználatlan erőforrásait.



\section{A működés az intézményben való benntartózkodásának rendje}
\section{A pedagógiai munka belső ellenőrzésének rendje}
A pedagógia munkát a keretttantervben leírt monitorozó rendszer alapján ellenőrizzük: amig a gyerekek fejlődnek, a szülők bizonságban érzik magukat és a gyerekeket, és a tanárok hatékonyaknak érzik magukat, és mindeközben minden résztvevő boldog, akkor jól végezzük a munkánkat.

\section{A belépés és benntartózkodás rendje azok részére, akik nem állnak jogviszonyban a nevelési-oktatási intézménnyel.}
Egy mikroiskolába akkor menj be, ha a előtte a tanárokkal megbeszélted, hogy nem zavarod a csoportok működését. Ha nem tudtad megbeszélni, akkor kérdezd meg tőlük, mikor belépsz. Ha nem tudod megkérdezni, vagy éppen zavarsz, akkor várj.


\section{Intézményegységgel való kapcsolattartás rendje.}
Egyelőre nincs külön szabály a kapcsolattartásra. Beszéljünk egymással többet, mint amit a folyamatos működés csak úgy adna. Egymást tudjuk segíteni és támogatni.



\section{A vezetők és a szervezeti egységek közötti kapcsolattartás rendje, formája, továbbá a vezetők közötti feladatmegosztás, a kiadmányozás és a képviselet szabályai, a szervezeti egységek közötti kapcsolattartás rendje.}
Az iskolában nem szervezeti egységek tartanak kapcsolatot, hanem emberek. Ha két egység, csapat között kapcsolatra van szükség, akkor ki kell jelölni, azokat, akik képviselik a csapatokat egymásnál.

A vezetők, és a nem vezetők ugyanúgy osztanak meg feladatot. Az iskolában csapatoknak van feladata, ezeket bárki elvállalhatja. Felelősségeket a csapat szerepeken keresztül delegál. Szerepekre szereplőket a hozzájárulás alapú döntéshozási módszerrel kell választani. A delegálás, feladatvállalás akkor történik meg, ha mindkét fél ezt elfogadja.

Egy megbeszélés után érdemes átismételni, hogy miben állapodtak meg a felek.


\section{Az intézményvezető vagy intézményvezető-helyettes akadályoztatása esetén a helyettesítés rendje.}
Az intézményvezető, mint minden vezető akkor végzi jól a munkáját, ha nélküle is működik a szervezet, mert a szerepek, feladatok le vannak osztva.

Ha olyan sokáig akadályoztatva van az intézményvezető, akkor a fenntartó képviseli az iskolát addig, amíg új intézményvezetőt nem talál a szervezet.

\section{A vezetők és az iskolaszék, az intézményi tanács, az iskolai szülői szervezet, közösség közötti kapcsolattartás formája, rendje.}
Jelenleg nem működik iskolaszék, intézményi tanács az iskolában. A szülői közösségek mikroiskolánként maguk döntenek arról, hogyan tartják a kapcsolatot.


\section{A nevelőtestület feladatkörébe tartozó ügyek átruházására, továbbá a feladatok ellátásával megbízott beszámolására vonatkozó rendelkezések.}
A feladat meghatározásakor érdemes arról is beszélni, hogy a feladat elvégzése után kit, miről és hogyan érdemes értesíteni (azaz, kik az érintettek).

Felelősségek átruházása szerepek kialakításával történhet. 

\section{A külső kapcsolatok rendszere, formája és módja, beleértve a pedagógiai szakszolgálatokkal, a pedagógiai szakmai szolgálatokkal, a gyermekjóléti szolgálattal, valamint az iskola-egészségügyi ellátást biztosító egészségügyi szolgáltatóval való kapcsolattartás}
\section{Az ünnepélyek, megemlékezések rendje, a hagyományok ápolásával kapcsolatos feladatok.}
\section{A szakmai munkaközösségek együttműködése, kapcsolattartásának rendje, részvétele a pedagógusok munkájának segítésében.}
\section{A rendszeres egészségügyi felügyelet és ellátás rendje.}
\section{Az intézményi védő, óvó előírásokat.}
\section{Bármely rendkívüli esemény esetén szükséges teendőket.}
\section{Hol, milyen időpontban lehet tájékoztatást kérni a pedagógiai programról?}
\section{Azon ügyek, amelyekben a szülői szervezetet, közösséget az SZMSZ véleményezési joggal ruházza fel.}
\section{A nevelési-oktatási intézményben a tanulóval szemben lefolytatásra kerülő fegyelmi eljárást megelőző egyeztető eljárás, valamint a tanulóval szemben lefolytatásra kerülő fegyelmi eljárás részletes szabályait.}
\section{Az elektronikus úton előállított papíralapú nyomtatványok hitelesítésének rendje.}
\section{Az elektronikus úton előállított, hitelesített és tárolt dokumentumok kezelése.}
\section{Az intézményvezető feladat- és hatásköréből leadott feladat- és hatásköröket, munkakörileírás-mintákat.}
\section{Az egyéb foglalkozások célja, szervezeti formái, időkeretei.}
\section{A felnőttoktatás formái}
\section{A diákönkormányzat, a diákképviselők, valamint az iskolai vezetők közötti kapcsolattartás formája és rendje, a diákönkormányzat működéséhez szükséges feltételek (helyiségek, berendezések használata, költségvetési támogatás biztosítása).}
\section{Az iskolai sportkör, valamint az iskola vezetése közötti kapcsolattartás formáka és rendje.}
\section{A gyermekek, tanulók egészségét veszélyeztető helyzetek kezelésére irányuló eljárásrend.}
\section{Az iskolai könyvtár SZMSZ-e}
\section{Azok a nevelési-oktatási intézmény biztonságos működését garantáló szabályok, amelyek megtartása kötelező az intézmény területén tartózkodó szülőknek, valamint az intézménnyel kapcsolatban nem álló más személyeknek.}
\section{Ha iskolaszék szék nem működik, az SZMSZ elfogadásakor az óvodai, iskolai, kollégiumi szülői szervezet, közösség véleményét kell beszerezni.}