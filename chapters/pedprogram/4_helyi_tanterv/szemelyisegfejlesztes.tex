\section{A személyiség és egészségfejlesztés}
\label{sec:szemilyesegfejlesztes}

A Budapest School gyerekek boldogak, egészségesek, hasznosak közösségüknek. Képesek önmaguknak célokat állítani, azokat elérni. Képesek  már kisgyerekkortól sajátjuknak megélni a tanulást és ahhoz kapcsolódóan célokat elérni, és fokozatosan tanulják meg azt, hogy egyénileg és csoportosan is tudnak nagyszabású projekteket véghezvinni. Tesznek a saját egészségükét, jövőjükért, társaikért, kapcsolódnak önmagukhoz és társaikhoz.

A gyerekek személyiségfejődését két szinten támogatja az iskola. Első szintet a mindennapi működés adja, mert az iskola működése önmagában személyiség és egészségfejlesztő hatással bír. Második szintet az összevont harmónia tantárgy biztosítja, ami holisztikus megközelítésével támogatja a gyerekek fizikai, lelki jóllétét és kapcsolódásukat a
környezethez.

\paragraph{Működésből adódó fejlesztések.}
Az iskola alap működése, hogy a gyerekek csoportban, közösségben élnek, tanulnak, dolgoznak, ezért ,,természetes'', hogy fejlődik az \emph{empátiájuk, kooperációs, kollaborációs képességük és érzelmi intelligenciájuk}. Alapelvünk: ha minél közelebb áll az iskola működése a jövő hétköznapjaihoz, a családhoz és a munkahelyhez, akkor a boldog családi életre, a sikeres munkahelyre való felkészülést már az iskolában való aktív részvélt önmagában támogatja. Hasonlóan, ahogy támogató, funkcionális, boldog családban felnőtt gyerekek nagyobb valószínűséggel lesznek maguk is egészségesebbek és boldogabbak.

A \emph{fejlődésfókuszú gondolkodásmód} kialakítását egyik kulcs tényezőnek gondoljuk a gyerekeink hosszútávú boldogulásához. Ezért a saját célok által irányított tanulási környezettől kezdve, a jutalmazás, értékelés, visszajelzés módjáig minden az iskolában azt a célt szolgálja, hogy a gyerekek képesek legyenek magukról pozítivan gondolkodni, ami az integráns és egészséges embernek talán egyik legfontosabb jellemzője.

A \emph{teljes körű iskolai egészségfejlesztés} az alábbi négy egészségfejlesztési feladat rendszeres végzése adja:

\begin{itemize}

    \item
          egészséges táplálkozás megvalósítása (elsősorban megfelelő, magas minőségű, lehetőleg helyi alapanyagokból; )
    \item
          mindennapi testmozgás minden gyereknek (változatos foglalkozásokkal, koncentráltan az egészség-javító elemekre, módszerekre, pl. tartásjavító torna, tánc, jóga)
    \item
          a gyerekek érett személyiséggé válásának elősegítése személyközpontú pedagógiai módszerekkel és a művészetek személyiségfejlesztő hatékonyságú alkalmazásával (ének, tánc, rajz, mesemondás, népi játékok, stb.)
    \item
          környezeti, médiatudatossági, fogyasztóvédelmi, balesetvédelmi egészségfejlesztési modulok, modulrészletek hatékony (azaz ``bensővé váló'') oktatása.
\end{itemize}

