
\input{chapters/kerettanterv/tanari_szerepek}

\section{A pedagógusok feladatai}\label{sec:tanarfeladatok}

\begin{verbatim}
a pedagógusok helyi intézményi feladatai, az osztályfőnöki munka tartalmát, az osztályfőnök feladatai
\end{verbatim}

A Budapest School kerettanterv háromféle tanárt, a tanulásszervező
tanárt, a mentorttanárt és a modulvezetőtanárt különböztet meg.

VAN BENNÜK Közös és mindent performanca review szerint adunk meg



\paragraph{Érzelmi intelligencia (EQ)
ninja}\label{uxe9rzelmi-intelligencia-eq-ninja}

\begin{itemize}

\item
  Gyerekek pszichológiai biztonságérzetét megerősíti.
\item
  Olyan visszajelzéseket ad, amelyek az erőfeszítésekre, a belefektetett
  energiára, munkára és a jövőbeni fejlődésre fókuszálnak (growth
  mindset), pozitív megerősítést alkalmaz, épít a gyerek erősségeire, és
  egyértelműen megfogalmazza, mit tehetne másként.
\item
  Értő figyelemmel van jelen, kedves, nyitott és figyel a gyerekekre.
\end{itemize}

\paragraph{SNI szakértő}\label{sni-szakuxe9rtux151}

\begin{itemize}

\item
  Jól tudja kezelni az egyéni bánásmódot, speciális nevelési igényű
  gyerekeket a csoportban.
\item
  Akinek külső segítségre van szüksége, azoknak a szüleivel ezt
  proaktívan leegyezteti és menedzseli a folyamatot.
\end{itemize}

\paragraph{Change agent}\label{change-agent}

\begin{itemize}

\item
  Nehéz helyzeteken is könnyen továbblendül, vannak módszerei arra
  hogyan töltse magát és ezeket használja is.
\item
  Azt keresi, mire lehet hatása, mit lehet eggyel jobban csinálni és
  ebben akciókat tesz.
\item
  Megünnepli az előrelépéseket, közös sikereket.
\end{itemize}

\subsection{Tanulásszervező}\label{tanuluxe1sszervezux151}

\paragraph{Jó csapattag}\label{juxf3-csapattag}

\begin{itemize}

\item
  Rendszeresen jelen van a tanári csapat megbeszélésein.
\item
  Elérhető telefonon vagy online a csapattal megállapodott kereteken
  belül.
\item
  Feladatokat vállal magára és azokat megbízhatóan, határidőre
  végrehajtja.
\item
  Kooperatív és támogató a közös munkák, ötletelések, megbeszélések
  alatt, képviseli a saját nézőpontját, gondolatait érzéseit, miközben a
  csapat és a többiek igényeire is figyel.
\item
  Kifejezi támogatását, ellenérveit és javaslatait a jobb megoldás
  érdekében.
\item
  A visszajelzést keresi, a kritikát jól fogadja, és megfontolja,
  átgondolja a lehetséges változtatásokat.
\item
  Kollégák fejlődését segíti rendszeres visszajelzésekkel.
\end{itemize}

\paragraph{BPS tag}\label{bps-tag}

\begin{itemize}

\item
  Rendszeresen jelen van heti hétfőkön, havi szombatokon.
\item
  Részt vesz a mikroiskolájának és a Budapest Schoolnak az építésében.
\item
  Proaktívan alakít ki rendszereket, folyamatokat, és a legjobb
  gyakorlatokat megosztja BPS szinten.
\item
  Közös témákban aktív, hozzászól, alakítja a véleményével és tudásával
  a BPS rendszerét.
\end{itemize}

\subsection{Mentor}\label{mentor}

\begin{itemize}
\item
  Az első trimeszter alatt a mentor megismeri mentoráltját,
  személyiségjegyeit, képességeit, érdeklődését, motivációit. Megismeri
  a családot.
\item
  Megállapodik a családdal a kapcsolattartás szabályaiban.
\item
  Bevezeti a családot a Budapest School rendszerébe.
\item
  Trimeszterenként a mentorált gyerekkel és szülőkkel egyetértésben
  kialakítja a mentorált gyerek tanulási céljait.
\item
  Képviseli mentoráltját a többi tanár felé.
\item
  Szülőkkel rendszeresen információt oszt meg, elérhető, asszertíven
  kommunikál, tiszta, mérhető megállapodásokat köt.
\item
  Szülőkkel erős partneri kapcsolatot épít.
\item
  Amikor a gyereknek külső fejlesztésre van szüksége, akkor a családot
  segíti a megfelelő fejlesztő felkeresésében, a külső fejlesztővel
  kapcsolatot tart és konzultál a mentoráltja haladásáról.
\item
  Maximum kéthetente talalkozik mentoráltjával. Követi, tudja, hogy a
  gyerek hogy van iskolában, családban, életben.
\item
  Mentorként nyomon követi, monitorozza mentoráltjai fejlődését és
  szükség esetén továbblendíti, inspirálja őket.
\item
  Mentor biztosítja, hogy a mentorát gyerek portfólió friss legyen.
\item
  Rendszeresen reflektál a mentorált gyerekkel együtt annak tanulási
  céljaira, és haladásáa.
\item
  Segíti a mentorált modulválasztását.
\item
  Visszajelzés gyűjt és ad a mentorált gyerek fejlődésére.
\end{itemize}

\subsection{Modulvezetők}\label{modulvezetux151k}

\begin{itemize}

\item
  Izgalmas, érdekes foglalkozásokat tart, amire felkészül és amiben a
  gyerekeket flowban tudja tartani.
\item
  Amikor a gyerekek vele vannak, akkor figyelnek, fókuszálnak,
  koncentrálnak, dolgoznak, tanulnak.
\item
  Kedvesen és határozottan vezeti a csoportot, figyel arra, hogy
  mindenkit bevonjon.
\end{itemize}

\subsection{A kiemelt figyelmet igénylő
gyerekek}\label{a-kiemelt-figyelmet-iguxe9nylux151-gyerekek}

\begin{verbatim}
A kiemelt figyelmet igénylő tanulókkal kapcsolatos pedagógiai tevékenység helyi rendje

> (ez a korábbi szabályozás terminológiájával élve a következő tematikai egységeket foglalja magába: a beilleszkedési, magatartási nehézségekkel összefüggő; a tehetség, képesség kibontakoztatását és a szociális hátrányok enyhítését segítő tevékenységet és a tanulási kudarcnak kitett tanulók felzárkóztatását segítő programokat, így tehát a sajátos nevelési igényű és a hátrányos helyzetű tanulók integrációjának sajátos programelemei is idetartoznak pl. Officina Bona, IPR);
\end{verbatim}

TODO: max 20\% TODO: mentor tudja, h mi a szitu
\todo[inline]{SNI, BTM megkulonboztetese, es irjuk le, hogy a mentor tudja, hogy mi a szitu.}
A tanulásszervezők, modulvezetők feladata:

\begin{itemize}
\item
  Megfelelő tanulásszervezési formákkal és módokkal biztosítani, hogy a
  tanórákon és a tanórán kívüli tevékenységben érvényesüljön a
  differenciált, az egyéniesített fejlesztés, eltérő képességekhez,
  viselkedéshez való alkalmazkodás.
\item
  Olyan tanulási környezetet, speciális módszerek, tapasztalatszerzési
  lehetőség biztosítása, amelyben sokoldalú szemléltetéssel,
  cselekvéssel, gazdag feladattárral, speciális eszközök alkalmazásával
  valósul meg készség- és képességfejlesztés.
\item
  A pedagógus a tanórai tevékenységek/foglalkozások tervezésébe építse
  be a pedagógiai diagnózisban szereplő javaslatokat.
\item
  A pedagógus a tananyag adaptálásánál, feldolgozásánál vegye figyelembe
  az egyes tanulók fejlettségi szintjét, a támogatás szükséges mértékét.
\item
  Az egyéni haladási ütem biztosítására egyéni fejlesztési és tanulási
  terv készítése, individuális módszerek, technikák alkalmazása.
\item
  A pedagógus működjön együtt a gyermek/tanuló fejlesztésében résztvevő
  szakemberekkel.
\end{itemize}
