\section{Évfolyamok, bizonyítványok, jegyek, vizsgák}
A Budapest School gyerekek saját tanulási célokat tűznek ki, modulokat
választanak, tanulnak, alkotnak, trimeszterenként frissíti a portfóliójukat,
értékelik haladásukat, újraterveznek. És újra indul a tanulás ciklus.

Eközben a gyerekek a tanulás és alkotás eredményeként évfolyamszinteken
lépkednek fel,
első szintről a tizenkettedik szintig tantárgyanként.
Azt, hogy ez hogyan és mikor történik, azaz az évfolyamszintek elismerését -- a
kerettantervvel összhangban -- az iskola egy transzparens folyamata
szabályozza.

Hivatalos, tantárgyankénti érdemjegyet a gyerekek akkor és csak akkor kapnak,
ha erre iskolaváltás, továbbtanulás, ösztöndíj	    (vagy más külső rendszer)
miatt szükségük van.
Tehát osztályzatok, érdemjegyek és vizsgák nélkül is van lehetőség
évfolyamszinteket lépni.

A vizsga így teljesen átértékelődik a Budapest Schoolban.
Az évfolyamszintek elismeréséhez és (szükség esetén) az érdemjegyek
megállapításához
nem elégséges a pillanatnyi tudást vagy képességet felmérő eseményt szervezni,
hanem
a teljes portfóliót kell értékelni és figyelembe venni.
A portfólia sokkal gazdagabban dokumentálja, hogy egy gyerek mit csinált, mire
volt képes, mint egy szóbeli vagy írásbeli feladatsor: előzetes tudás-,
képességpróbák
mellett tartalmazza az alkotások, projektek, visszajelzések, versenyek, stb.
dokumentációt is.

\subsection{Évfolyamszintek}
\label{sec:evfolyamok}

A Budapest Schoolban az évfolyamokra úgy tekintünk, mint egy játék nehézség
szintjeire: akkor léphet egy tanuló a következőbe, ha teljesítette az
évfolyamhoz köthető tantárgyi követelményeket. Ezeket a kerettanterv
\emph{Tantárgyi eredménycélok} c. fejezete határozza meg.

A Budapest School évfolyamszintjei eltérnek az iskolák többségében alkalmazott
évfolyamtól (osztályoktól).

\begin{itemize}
    \item Egy gyerek minden tantárgyból állhat más szinten.
    \item Nem biztos, hogy az egy korcsoportba tartozók vannak ugyanazon az
          évfolyamszinten.
    \item Nem mindig az egy évfolyamszinten lévők tanulnak együtt, sokszor
          előfordulhat az is, hogy a különböző szinten lévő tanulók tudnak
          együtt és akár egymástól is tanulni.
    \item Egy év alatt több évfolyamszintet is lehet lépni.
\end{itemize}

Bár egy gyerek, minden tantárgyból állhat más szinten, a hivatalos (azaz külső
hivatalok, rendszerek számára értelmezhető) bizonyítványába mindig csak annak
az évfolyamnak
az elvégzése kerül be, amelyből mind a három tantárgyhoz szükséges fejlesztési
célt elérte.
Formálisabban kifejezve a tantárgyankénti évfolyamszintek minimumát kell a
bizonyítványban rögzíteni.

\subsubsection{Évfolyamszint-lépés}
\label{sec:evfolyamszintlepes}
A tanulók portfóliója alapján megállapítható, hogy egy adott évfolyamhoz
köthető tantárgyi követelményeknek megfelel-e.
Ehhez a tanulók elvégzik a mentoruk segítségével a portfóliójuk (mit csináltak,
mit tanultak, mit tudnak) összehasonlítását \ifkerettanterv
    \aref{sec:tantargyi_celok}.
    fejezetben
\else
    a kerettanterv \emph{Tantárgyi eredménycélok} c. fejezetében
\fi
felsorolt tantárgyi
eredménycélokkal.
Ha szükséges, akkor a kapcsolódás biztosításához a portfóliójukat
kiegészíthetik tudáspróbák, tesztek, szabványos vizsgák teljesítésével, melynek
megszervezése az adott mikroiskola tanárközösségének a feladata.

Miután a gyerek (mentora és szülei) segítségével összeállította a portfólióját,
jelzi az iskolának az évfolyamszint lépési kérelmét.

Ezt az iskola által kijelölt bírálók mevizsgálják és elismerik az
évfolyamszinthez szükséges tantárgyi követelmények teljesítését.
Egy tantárgyból egy évfolyam teljesítettnek tekinthető, ha a tantárgyhoz
tartozó tanulási célok 50\%-ának elérése a portfólió alapján bizonyítható.

A kérelmet a gyerek digitálisan adja be.
Az elbírálás csak a portfólió alapján történhet, ami egy
online elérhető adatbázisként tartalmaz mindent, ami a döntéshez szükséges
lehet. A döntéshez így a bírálóknak és a gyereknek nem
kell egy időben és egy helyen lennie. Minden esetben szükséges a portfóliót és
a teljes folyamatot digitálisan rögzíteni.
Az iskolának két évig meg kell őriznie az osztályozó vizsgák anyagát.

Magántanuló vagy az órák látogatásáról valamilyen okkal
felmentett gyerekek ugyanígy, a portfóliójuk összeállításával és a szint lépés
kérelmezésével kérhetik az évfolyamszintek teljesítésének igazolását.

Amennyiben valamely diáknál egy adott évfolyam tantárgyi követelményei
elismerésre kerülnek, akkor az iskola igazolja, hogy a diák az adott tantárgy
vagy tantárgyak évfolyam szerinti követelményeit teljesítette.
Erről igazolást állít ki, és teljesíti a jelentési kötelezettségét az Oktatási
Hivatal felé.

\subsection{Osztályzatokra váltás}
\label{sec:osztalyzatok}
A Budapest School kerettanterve lehetővé teszi, hogy a gyerekek az érdemjegyek
és osztályzatok
helyett egy több szempontot figyelembe vevő szöveges vagy értékelőtáblázat
(rubric) alapú viszajelzést kapjanak.
A gazdag információtartalmú visszajelzések és portfólió osztályzatra való
átváltására mégis szükség lehet, például iskolaváltás vagy továbbtanulás
esetén.\footnote{Azt az NKT. 54.§ (4) pontja alapján.}

Az átváltás \aref{sec:evfolyamszintlepes}.
fejezetben leírtakhoz
hasonlóan is a portfólió értékelésén alapulnak.
A gyerek (mentora és szülei segítségével) összeállítja a portfóliót, és
bizonyítja, hogy a portfólió alapján megállapítható a kívánt osztályzat, az
adott tárgyhoz az adott évfolyamban.

\subsection{Kérelmek elbírálása}

A szülő és gyerek évfolyamszint-lépés vagy osztályzatra váltás kérelmét bírálók
értékelik. Minden esetben legalább három bíráló bírál egy kérelmet: a
bírálók közül egy a gyerek mentortanára, egy pedig mindenképp másik mikroiskola
tanulásszervezője. A bírálókat az iskola kinevezett felelőse\todo{ki nevezi ki
    a biralokinevezoket} választja ki.

Ha a bírálók közül akár egy is úgy véli, hogy az évfolyamszint-lépés vagy
osztályzat nem megalapozott, akkor
erről indoklásban értesítik a gyereket, szüleit és mentortanárát. Ilyenkor a
gyerek javíthata kérelmét és tetszőleges számú esetben ismételheti a
folyamatot. Tulajdonképpen az évfolyamszintekről és osztályzatokról mindig
olyan döntésnek kell születnie, ami a gyereknek, a szülőknek és a bírálóknak is
elfogadható. Ha nem tudnak megállapodni, akkor
\aref{sec:konfliktusok_kezelese}. fejezetben leírtak alapján kell keresniük a
mindenki számára elfogadható megoldást.

Az évfolyamszint lépésről és az osztályzatra váltásról vagy épp a kérelmek
elutasításáról az iskola minden tanára értesítést kap. Ha a
nevelőtestület\footnote{NKT szóhasználata.}
nem ért egyet a bírálók döntésével, akkor új bírálók kinevezését kérhetik.

A kérelmek elbírálását 20 tanítási nap alatt el kell végezni.\footnote{Ha a
    kérelem nyári szűnetben érkezett, akkor auguszutus 31-ig.}

Évfolyamszint-lépést az iskola az utolsó évfolyamszint-lépéstől számított két
tanéven belül automatikusan indít. Így nem fordulhat elő, hogy egy gyerek
évfolyamszintjei két éven keresztül nincsenek felülvizsgálva.

Magántanuló és az órák látogatásáról valamilyen okkal felmentett gyerekek
évfolyamszint-lépését és szükség esetén osztályzat kérelmét pontosan ugyanazon
folyamatokkal kell elvégezni, mint a nem magántanuló és nem felmentett
gyerekét.