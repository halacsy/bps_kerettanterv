\section{Emberkép}
\label{sec:gyerekkep}

A világ és benne az egyén állandóan változik, és a változás mégis számos állandóságot jelöl ki. Korábban a technológiai változás évszázadokban volt mérhető, az általunk használt eszközök száma és komplexitása jóval kevesebb volt, mára néhány év, vagy egy pár órás repülőút elegendő ahhoz, hogy olyan környezetbe, olyan emberek, eszközök közé kerüljünk, ahol másként kell alkalmzkodnunk, mint ahogy azt tanultuk. Ebben a világban a saját belső harmóniánk megtalálása, a saját közösségünkhöz, közösségeinkhez való kapcsolódás, a világ működésének megértése és a törekvés arra, hogy tegyünk a fenntartásáért különösen fontossá vált. Ahhoz, hogy ennek megfeleljünk a tanári, szülői és a tanulói szerepekben is folyamatos fejlődésre van szükségünk. A jövőbe nem látunk, de egy dolgot biztosan tudunk: bármilyenek is lesznek a jövő kihívásai, nekünk az a fontos, hogy ez a gyerekek számára ne félelmetes és szorongást keltő legyen, hanem lehetőséget, kihívást és örömet okozzon.

A célunk, hogy a fiatalok az iskolában és utána is könnyen megtalálják a saját útjukat. Képesek legyen önmaguknak célokat állítani, azokat elérni. Képesek legyenek már kisgyerekkortól sajátjuknak megélni a tanulást és ahhoz kapcsolódóan célokat elérni, és fokozatosan tanulják meg azt, hogy egyénileg és csoportosan is tudjanak nagyszabású projekteket véghezvinni. A tanulás három rétege, a tudás megszerzés, az annak felhasználását segítő őnálló gondolkodás és az aktív alkotás egyszerre jelenik meg a mindennapokban. Ezek egyensúlyban tartása épp oly fontos, mint az, hogy a Budapest School kerettantervének megfelelően az egyéniu célok legalább 50\%-ban a kerettantervi egyéni célokból és további 50\%-ban a saját célokból épüljenek fel. Saját célként a sajátnak megélt célokat értjük.

Ezért legfontosabb fejlődési dimenziónak az önálló, célorientált tanulási és stratégiai gondolkodást tekintjük, melynek ütemeit tanulási szakaszokra bontjuk. A társas kapcsolatok tanulási folyamatában a születéstől a felnőttkorig a teljes magára hagyatottság és énközpontúság csecsemőkori állapotából fejlődik az ember önállóan és kreatívan gondolkodó, önmagával és közösségével integránsan élő érett nagykorúig. Ezt az utat Erik Erikson pszichoszociális fejlődési modelljében rögzítette \citep{Erikson91}. A Budapest School ezen elmélet elemeit veszi sorra és integrálja tanulási struktúrájába, melynek eredményeképp az iskolakezdőtől az érettségizőig négy tanulási szakaszt különít el. Ezen tanulási szakaszok határait az önállósodás során tapasztalt egyéni mérföldkövek jelölik ki, melyek fokmérője a saját cél állításának időbeli, tartalmi előrelépései, valamint az egyén belső szabadságérzete és közösséghez való kapcsolódása. Az egyes tanulási szakaszok amellett, hogy a célállításban elkülönülnek egymástól, a szintugrás abban is mérhető, hogy mennyire képes egy gyerek az alatta lévő szinten lévő társait segíteni az előrelépésben. A folyamatos kortárs támogatás a szintekben való előrehaladás valós fokmérője a mentor és szülő visszajelzései mellett.

\paragraph{Első szint, 5-10 év }

Ebben a korban alakul ki a gyerekek logikus gondolkodása, melynek részeként feladatokat tudnak rendszerezni, sorrendeket képesek felállítani és azokat fogalmakhoz társítani. A gondolkodásuk elkezd a befelé fordulóból a társas kapcsolatok irányába nyílni, ezáltal megnyílik a közösségi problémamegoldás lehetősége. Igényük nő a belső és a külső rendezettségre, így elkezdhetnek önmaguk szabályokat alakítani a saját tanulási igényeik kapcsán. Az első szakaszban a gyerekek pszichoszociális fejlődésükkel összhangban a saját magunknak állított cél jelentőségének fogalmát tanulják rövid, eleinte néhány órás, majd a szakasz vége felé, néhány napos tervezéssel és erre adott önreflexiókkal és külső megerősítésekkel. Megtanulják a ma, a tegnap és a holnap fogalmát, tanulási szerződésük tartalmát eleinte főleg a mentor és a szülő segítségével állítják össze, hogy annak a szakasz végére már teljesen egyedi, önállóan kiválasztott elemei is lehessenek. Ebben a szakaszban a szabad játéknak még nagy szerepe van, a belső világ tágassága fokozatosan nyílik meg a külvilág felé, és kezd lényegessé válni az azokkal való kommunikáció kiérlelése. Ekkor tanulnak meg a gyerekek írni, olvasni, és a matematikai alapműveleteket, valamint a mindennapjaikban használatos  geometriai és kombinatorikai alapfogalmakat, mely tudás a mindennapjaikba beépülő önálló tanulási célokhoz kapcsolódóan folyamatosan egyre szükségesebbé válik. Önmagukhoz és környezetükhöz érzékenyen, odafigyeléssel és empátiával fordulnak, megtanulják tiszteletben tartani, hogy társaik másmilyenek, akiket más dolgok is érdekelhetnek, másfajta megoldásaik is lehetnek. A szakasz végére biztonsággal meg tudnak maguknak tervezni egy néhány napos projektet.

\paragraph{Második szint, 9-14 év}

A gyerek testi és pszichoszociális fejlődése szempontjából egyaránt kiemelten fontos időszak következik. A korai serdülőkorban alakulnak ki a másodlagos nemi jellegek, melyek ütemükben mind a fiúk és lányok, mind az egyének között jelentős eltéréseket mutathatnak. Az egyéni tanulási tervek összeállításának ezért ebben az időszakban megnő a szerepe. Az agyi struktúrák jelentős átalakulása is erre az időszakra tehető, a gondolkodás, a figyelem, az emlékezet területén is lehetnek ezen átalakulások miatt nehézségek, amit később, az átmeneti visszaesést követően az agy magasabb minőségű működése követi. Ekkor alapozhatja meg egy gyerek a tervezés, a saját tanulási célok komplexebb fogalmait. Aki ebbe a szakaszba lép már hetekre előre ki tudja jelölni a saját feladait. Három év alatt oda akar eljutni, hogy a szakasz végére önállóan képes legyen egy teljes trimeszternyi tanulási tervet összeállítani. Ehhez a hetek számát folyamatosan növelnie kell, miközben a problémák komplexitása is változik. A szülő még mindig megjelenik a célalkotásban, de egyre inkább átalakul a szerepe tanácsadóvá. A mentor az egyéni fejlődés mintázataira figyelve segíti a gyereket abban, hogy fejlettségi szintjének, aktuális testi, lelki változásainak megfelelő módon legyen terhelve.

A szakasz végén önállóan be tudja mutatni társainak egy pár hetes projekt eredményeit, és mind szövegértése, mind logikus gondolkodása és matematikai alapismeretei olyan szinten állnak, hogy elmélyült alkotó, kutató feladatokat vállalhat a következő szakaszban. Biztonságos nyelv használata sajátja mellett már egy másik nyelvben is elkezdődik, és értékrendjében önmaga és társai megismerésén túl a környezet és a világ fogalma is tágulni kezd.

\paragraph{Harmadik szint, 12-16 év}

A sajátként megélt tanulási célok eddigi alapozása ebben a szakaszban nyeri el mélyebb funkcióját. A tinédzserkor minden szempontból a gyerek kivirágzásának időpontja, melynek során a helyét kereső, befelé, saját változásaira fókuszáló gyerekből egy a jövő felé nyitott kamasz válhat. A gondolkodási struktúrák mellett megnő a családon túli társas kapcsolatok szerepe, és ekkor fontos, hogy a családias jelleg, a biztonságos tanulás, a szülővel való konzultáció mint a gyerek számára egyaránt fontos érték, és nem mint kényszer jelenjen meg. Az érvelés, a hipotézis alapú problémamegoldás, valamint a jelen fókuszált megélése ebben a korban alakul ki, ahogy annak a tudata is, hogy tetteinknek a jövőre nézve nagyobb mértékű következményei lehetnek. Kiemelten fontos ebben a szakaszban a különböző vélemények, információk, megoldási lehetőségek számba vétele, annak lehetősége, hogy a gyerek egyénileg ismerhesse fel az eltérő utak közötti különbségeket, és ha teheti kérdőjelezzen meg akár tudományos hipotéziseket, vagy írjon újra művészeti, kulturális tartalmakat. Ha a gyerek ezen határok megértésében szabadon fejlődik, akkor lehetősége lesz arra, hogy a megszerzett alapjait olyan kreatív irányokba fordíthassa, melyek saját élete és társadalmunk jövője formálásához egyaránt hasznosak lehetnek. Ebben az időszakban a saját tanulási célokat egy teljes trimeszterre vonatkozóan egyénileg határozza meg a gyerek, a szülő és a mentor ebben mint konzulens jelennek meg. Meghozza első komolyabb döntéseit arról, hogy milyen területekkel szeretne elmélyültebben foglalkozni és ehhez kapcsolódóan választott érettségi tantárgyait is kijelöli az szakasz végére. Előadásmódját, kutatási és alkotói munkáját felnőtt jegyek jellemzik.

\paragraph{Negyedik szint, 15-19 év} Ebben a szakaszban a gyerek fizikai
értelemben teljesen felnőtté válik. Érzelmi és szociális biztonságra azonban kiemelten szüksége van. A szerelem, a szexuális identitás erősödése, az önállósodásra való igény, az addikciók veszélye konfliktusokat szülhet, melyek kezelése különösen fontos ebben az időszakban. Az iskola elsődleges feladata ennek a biztonságos közegnek a megteremtése ebben a felnőttszerű időszakban. Ebben a mentor a szülővel együttműködve tudja segíteni a gyereket. A szakasz célja, hogy a gyerek képesssé váljon a saját életét	meghatározó egy-két éves távlatokban mérhető felelős döntések meghozatalára, melyek akár sorsfordító jelentőségűek is lehetnek. Ezek a döntések  vonatkozhatnak egy emelt szintű érettségire való felkészülésre, egy nemzetközi egyetemre való felkészülésre, egy nagyobb komplexitású kutatási vagy alkotói projektre. Fontos azonban a gyerek támogatása abban is, hogy nem kell örökérvényű döntéseket hozni. Ekkora már megtanult szakaszosan célokat állítani, és tudja, hogy bármikor lesz lehetősége az életben újratervezni. Önállóan készül az érettségire, tanárai segítik, hogy folyamatosan megtalálja a kihívást ebben. Saját céljai szűkebb környezetén túl könnyedén hatással lehetnek már a világra is.
