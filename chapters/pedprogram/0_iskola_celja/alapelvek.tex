\section{A tanulni tanulás hét pillére}
A tanulás egyénenként változó, és ha vegyes korosztályokban is, mégis egy közösségben valósul meg. \emph{A tanulni tanulás hét pillére} minden korosztályban állandóan befolyásolja és keretezi a Budapest School pedagógiai programját.

\subsection{Tanulni tanulnak -- fejlődésközpontúak vagyunk}
A Budapest School mikroiskoláiban mindig ott, akkor és annak kell történnie a gyerekekkel, ami őket a fejlődésükben a leginkább támogatja. Minden, ami az iskolában történik, újra és újra erre az alapkérdésre kell hogy visszatérjen. Az segíti a leginkább a gyerekek fejlődését, amit most csinálunk, vagy változtatnunk kell rajta? A Budapest School tehát rugalmas és integratív, a gyerekek fejlődéséhez igazodó.

A gyerekek tanulása \emph{fejlődésközpontú szemléletben} \citep{growthmindset} (growth mindset) történik.  Erőfeszítéseik segítségével képességeik fejleszthetők, megváltoztathatóak. Számukra inspiráló a kihívás, és a hibázás kevésbé töri le lelkesedésüket. A gyerekek és a tanárok számára ezáltal nagyobb fontossággal bír a tanulásba fektetett erőfeszítés és a fejlődés, mint az aktuális képességeik.

A hibázás inkább válik a gyakorlás és az új megismerésének jelzőjévé. Ennek feszültségmentes kezelése kulcsfontosságú abban, hogy a gyerekek merjék feszegetni a saját határaikat, hogy magabiztosan dolgozzanak azon, hogy képességeiket, ismereteiket vagy gyakorlataikat folyamatosan fejlesszék. A hibázásból való tanulás fő célja, hogy mindig új hibákat ismerjenek meg, és a korábbiakra minél jobb megoldásokat találjanak a gyerekek.

\subsection{Tanulni tanulnak -- saját célokat állítanak}
A gyerekek saját erősségeiket fejlesztve saját célokat állítanak, közben céljaikat folyamatosan igazítják a világ adta lehetőségekhez és szükségletekhez.

A tanulás egésze egy olyan folyamatként írható le, amely különböző állomásokra, rövid célokra bontható. A tanulási célok állításának folyamata, annak minősége az évek alatt folyamatosan változik, egyre tudatosabbá, pontosabbá, komplexebbé válik. Azonban már az első évektől el kell kezdeni annak gyakorlását, hogy később kialakulhasson a saját tanulásicél-állítás.

A kisgyerekkor jellemzője a kíváncsiság, az igény a felfedezésre, tapasztalásra. A gyerekek számára  \emph{a tanulás  egy önvezérelt aktív folyamat}, melynek megtartása és folyamatos fejlesztése a Budapest School tanárainak legfőbb feladata. Két út van: vagy megtanítjuk a gyerekeket azokra a képességekre, amelyekre ma szükségük van, ezzel kockáztatva azt, hogy a tanulásuk a világ változásával veszít korszerűségéből, vagy abban segítjük őket, hogy megtaníthassák magukat azokra a képességekre, amelyekre épp az adott élethelyzetükben szükségük van. A tanulás így élményszerűvé válik, ismeretszerző jellege csökken, és nő az önálló felfedezés lehetősége.

\subsection{Tanulni tanulnak -- mindig és mindenhol}
A tanulás a Budapest Schoolban egész nap történik, melynek pontos alakítása a gyerekek tanulási igényeitől, fejlettségi szintjétől és korosztályától is függ. A tanulási rend meghatározásáért a Budapest School mikroiskoláinak tanulásszervezői felelnek.

A tanulás trimeszterenként újraszervezett módon tanulási modulokban történik, melynek során jut idő egyéni és csoportos, gyakorló, ismeretszerző és gyakorlatias foglalkozásokra is. Az egész napos iskola lehetőséget biztosít arra, hogy a tanulási egységek között legyen idő fellélegezni, és felkészülni az újabb modulokra, valamint arra is, hogy ha egy tanulási egység nagyon magával ragadja a gyerekeket, akkor benne maradhassanak és annak megfelelően alakítsák újra az időrendjüket.

A tanulás az iskolában nem ér véget. A tanulás szeretetének kialakulásával folyamatossá válik az ezzel való foglalkozás, így a Budapest School tanulásnak veszi az otthon, szünetekben eltöltött időt is, ahol sokszor hatékonyabb módon tud egy gyerek gyakorolni, kutatni, alkotni, mint az iskolában, amikor társaival van egy közösségben és ezáltal számos más inger is éri.

Az iskolában történő tanulással egyenrangúnak tekintjük az otthon tanulást, az iskolától független iskola utáni programokat, a (nyári) táborokat, a családi utazásokat, a vállalatoknál töltött gyakornoki időt, az egyéni tanulást és projekteket. A tanulás bárhol és bármikor történhet, és célunk, hogy mindenhol és mindig tanuljanak.

\subsection{Tanulni tanulnak -- együtt, egymástól}

A tanulás egyénileg és csoportokban is történhet. A csoportok megszervezése mindig azon múlik, hogy az adott tanulási célt mi szolgálja a legjobban. Ennek megfelelően a gyerekek nem állandó, hanem a tanulási célokhoz, az érdeklődéshez, a képességi szintekhez alkalmazkodó rugalmas csoportokban tanulnak. A tanulás ezáltal kevert korcsoportokban is történhet, akár egy nagy családban. Együtt, egymástól tanulnak a gyerekek, egymást segítik a fejlődésben. Az egymásnak adott fejlesztő visszajelzések, pozitív megerősítések révén folyamatosan alakul ki a tanulás tisztelete és a képességek fejlesztésébe vetett hit.

A közösségben tanulás módja nagyban függhet attól, hogy egy gyerek mennyire zárkózott, mennyire tud és akar önállóan tanulni. A csoportos munkák során alapelv, hogy a zárkózott gyerekek is lehetőséget kapjanak, hogy csöndesen, vagy kisebb csoportban végezhessék a munkájukat, mondhassák el ötleteiket. Az egyéni tanulásban minden gyereknek lehetőséget kell adni arra és segíteni kell abban, hogy önállóan, fókuszáltan tudjon tanulni.\vfill\eject

\subsection{Tanulni tanulnak -- alkotnak és felfedeznek}
A tanulás három rétege, az ismeretszerzés, a gondolkodás fejlesztése és a gyakorlatias, aktív alkotás egyszerre jelenik meg a Budapest School mindennapjaiban. Az alkotó munka rugalmas időkereteket, változó csoportbontásokat, és a projektmódszerek sokszínű alkalmazását igényli. A tanulás ilyenkor sokszor csinálássá válik, az ismeret pedig termékké változik.

A gyerekek önmaguk és a világ számára releváns kérdésekkel foglalkoznak, amihez külső szakértőket is bevonnak, ha szükséges. A tanulás tehát célokhoz és nem tárgyakhoz kötött. A tanulás tartalmát igazítjuk a tanulás céljához, ezáltal az egyes tudományterületek, művészetek, vagy épp mesterségek gyakran keverednek egymással egy-egy modulon belül. Szintén a célhoz igazított tanuláshoz kötődik a kutató-felfedező attitűd, ami a már ismert újramegismerése mellett az ismeretlen felfedezésére, a megválaszolhatatlan megválaszolására irányul.

\subsection{Tanulni tanulnak -- elfogadóak és egymásra figyelnek}

A gyerekek tanulását családi hátterük változása, egyéni problémák, számos mindennapi esemény befolyásolhatja. Ezek figyelembevétele a mindennapokban, a \emph{mentor tanárral} való bizalmi viszonynak köszönhetően válik lehetségessé. Ennek a kapcsolatnak az alapjait ezért a partnerség, az értő figyelem adja.

A Budapest School emellett kiemelt figyelmet fordít arra, hogy a sajátos nevelési igényű tanulók is lehetőséget kapjanak a csoportban való munkára, amennyiben az a közösség számára is hasznos. Tanulásukat, ha szükséges, külső szakember segíti. A Budapest School a hátrányos helyzetű gyerekek számára is biztosítani kívánja az elfogadó, fejlesztő környezetet. Az egyenlő bánásmód megvalósulása érdekében olyan differenciált tanulási környezetet alakít ki, ami biztosítja a minél nagyobb mértékű inkluzivitást.

\subsection{Tanulni tanulnak -- tanulva tanítunk}
A Budapest School \emph{tanulásszervezői} partnerként, a tanulás folyamán segítő társként vannak jelen a gyerekek életében. A tanulás tanórák helyett pontos tanulási célokat tartalmazó tanulási modulokból épül fel, melyek során a tanár az adott cél eléréséhez szükséges eszközöket, tanulási segédleteket biztosítja.

A tanár akkor és annyira segíti a gyerekeket a saját céljuk elérésében, amennyire azt a gyerek igényli, és folyamatosan tekintettel van arra, hogy a gyerek saját fejlődési üteme megvalósulhasson. Ehhez tudatosan kell kezelnie nemcsak a gyerekeket érintő fejlesztési lehetőségeket, hanem azt is pontosan látnia kell, hogy egy adott tanulási cél elérésének milyen készségszintű vagy gyakorlatias alapfeltételei vannak. Ezért a gyerekek tanulási célját támogatandó segítenie kell abban, hogy a gyakorló idő, a gyakorlatias alkotó idő és az új ismeret megszerzésének ideje folyamatos egyensúlyban legyen. A tanár a gyerekekkel együtt fejlődik, saját tanulása a segítői szerepben folytonos.
