\section{Kapcsolat a hazai és nemzetközi oktatási reformokkal}
\label{sec:kapcsolat_reformokkal}

A Budapest School programja a hazai és nemzetközi oktatási reformok kontextusában és a pszichológia, a szociálpszichológia, valamint a szervezetfejlesztés terén elvégzett kortárs kutatások tükrében válik könnyebben értelmezhetővé.

Magyarországról több iskola története, működése is nagy hatással volt ránk. A 90-es évektől induló alternatív iskolák világát mi a \emph{Rogers Személyközpontú Általános Iskola}, a \emph{Lauder Javne Iskola}, a \emph{Kincskereső Iskola} és a \emph{Gyermekek Háza} alapján ismertük meg. A megújuló középiskolák modelljének mi az \emph{Alternatív Közgazdasági Gimnáziumot} és a \emph{Közgazdasági Politechnikumot} tartjuk. Ezek az iskolák a személyközpontúság, a gyerekközpontúság hangsúlyozása mellett elkezdték a gyakorlatban alkalmazni a partnerség alapú kommunikációt, a differenciálás, a kooperatív technikák módszerét és egyes projektmódszertanokat. A \emph{Zöld Kakas} iskola programjának egyszerűsége és rendszerszintű szemlélete az egyik legnagyobb inspirációt adta számunka.

Programunk kidolgozásában nagy szerepe volt annak, hogy ezek az iskolák olyan szemléletmódbéli alapokat fektettek le, amelyek mára alapelvárásként fogalmazódnak meg a szülők oldaláról az iskolákkal szemben.

Gyakorlati tapasztalatokat a világ más részein is gyűjtöttünk. A 21. században a Budapest Schoolhoz hasonló kezdeményezések sorra indulnak a világban. Ezek egyes jegyei a Budapest School modelljével összhangban vannak:

A \emph{Wildflower School}\footnote{https://wildflowerschools.org/} mikroiskolák hálózatát működteti kisebb üzlethelyiségekben. A Budapest Schoolhoz hasonlóan célja, hogy falakat romboljon a gyerekek és a világ között: a magántanulás és az intézményes tanulás, a tanár és a tudós szerepe, valamint az iskola és környezete közötti határok elmosása az egyik fő üzenete.

Hasonlóan az otthon tanulás és az  \emph{unschooling} strukturált formáját keresi az amerikai Texasban alapított \emph{Acton Academy}\footnote{https://www.actonacademy.org/}, amely a szokratikus módszereket (azaz, hogy megbeszéljük közösen), a valós projekten keresztüli tanulást, és a gyakornokoskodáshoz hasonló munka közbeni tanulást (,,learning on the job'') teszi a megközelítésének középpontjába.

A \emph{High Tech High}\footnote{https://www.hightechhigh.org/} iskoláiban a gyerekek elsősorban projektmódszertan alapján tanulnak. A tanulási jogokban való egyenlőség mellett az egyéni célokra szabott tanulás, a világ alakulásához kapcsolódó tartalmi elemek, valamint az együttműködés-alapú tanulás is megjelenik pedagógiájukban a Budapest School által is alkalmazott jegyekből.

A \emph{School21}\footnote{https://www.school21.org.uk} brit iskola 21.~századi képességek fejlesztését tűzte ki célul. Ezért a prezentációs, előadói skillek kiemelt jelentőségűek. Az iskola egyensúlyt akar teremteni a tudásbéli (akadémiai), a szívbéli (személyiség és jóllét) és a kézzel fogható (problémamegoldó, alkotó) között. A Budapest School iskoláinak hasonló módon célja, hogy a tanulás három rétegét, a tudást, a gondolkodást és az alkotást folyamatos harmóniában tartsa.

A \emph{Khan Lab School}\footnote{https://khanlabschool.org/} a Montessori-módszert keveri az online tanulással. Kevert korosztályú csoportokban, személyre szabott módszerekkel segítik a képességfejlesztést és a projektalapú munkát. 
\newpage
\thispagestyle{empty}
