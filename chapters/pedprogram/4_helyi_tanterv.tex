\section{A választott kerettanterv
  megnevezése}

Budapest School Általános Iskola és Gimnázium az oktatásért felelős
miniszter által jóváhagyott Budapest School Kerettanterv alapján
működik. Az iskola munkatársai a kerettantervet, a pedagógia programot,
sőt még a szervezeti és működési szabályokat is együtt, egy egységként
állították össze. Ez a Budapest School Módszer, aminek fő célja, hogy a
gyerekek úgy tudják azt és akkor tanulni, amit szeretnek, vagy amire
szükségük van, hogy közben a tanárok, szülők számára is kiszámítható,
tervezhető, biztonságos tanulási környezetet biztosít az iskola.

Az iskola személyre szabható tanulási környezetet biztosít. A
tanulásszervező tanárok, az iskola sok választási lehetőséget kínál a
gyerekek számára, akik a lehetőségekből összeválogatják a saját tanulási
céljaikhoz leginkább illeszkedő saját tanulási programjukat. A
választásban -- az önvezérelt tanulás menedzselésében -- a gyerekek
folyamatos egyéni támogatást kapnak a \emph{mentor} tanáruktól.

\section{A személyiségfejlesztés}
\label{sec:szemilyesegfejlesztes}
\begin{verbatim}
  Trv. elvarja: A személyiségfejlesztéssel kapcsolatos
  pedagógiai feladatok
  Itt irjuk le a 4x3 szintet
\end{verbatim}

A Budapest School gyerekek boldogak, egészségesek, hasznosak közösségüknek és
adaptívak, azaz képesek változni, tanulni és fejődni. Ennek megfeleően
foglalkoznunk kell az egészségfejlesztéssel (\aref{sec:egeszsegfejlesztes}.
fejezet)

\subsection{A teljeskörű
      egészségfejlesztés}\label{sec:egeszsegfejlesztes}

A teljes körű egészségfejlesztés célja, hogy a BPS-ben eltöltött időben
minden gyerek részesüljön a teljes testi-lelki jóllétét, egészségét
megőrző és hatékonyan fejlesztő, a BPS mindennapjaiban rendszerszerűen
működő egészségfejlesztő tevékenységekben.

\subsubsection{Négy alapfeladat}\label{nuxe9gy-alapfeladat}

A teljes körű iskolai egészségfejlesztés az alábbi négy
egészségfejlesztési alapfeladat rendszeres végzését jelenti - minden
gyerekkel, a tanárok és a szülők, valamint a BPS partneri kapcsolati
hálójában szereplők bevonásával:

\begin{itemize}

      \item
            egészséges táplálkozás megvalósítása (elsősorban megfelelő, magas
            minőségű, lehetőleg helyi alapanyagokat használó kiszállítóval való
            megállapodással; helyben főzés esetén az alapanyagok
            kiválasztásánál
            legyen elsődleges szempont)
      \item
            mindennapi testmozgás minden gyereknek (változatos
            foglalkozásokkal,
            koncentráltan az egészség-javító elemekre, módszerekre, pl.
            tartásjavító torna, tánc, jóga)
      \item
            a gyermekek érett személyiséggé válásának elősegítése
            személyközpontú
            pedagógiai módszerekkel és a művészetek személyiségfejlesztő
            hatékonyságú alkalmazásával (ének, tánc, rajz, mesemondás, népi
            játékok, stb.)
      \item
            környezeti, médiatudatossági, fogyasztóvédelmi, balesetvédelmi
            egészségfejlesztési modulok, modulrészletek hatékony (azaz
            ``bensővé
            váló'') oktatása.
\end{itemize}

\subsubsection{Az egészségfejlesztési ismeretek
      témakörei}

\begin{itemize}

      \item
            Az egészség fogalma
      \item
            Az egyén és az őt körülvevő közösség egészsége: felelősségünk
      \item
            A környezet egészsége
      \item
            Az egészséget befolyásoló tényezők
      \item
            A jó egészségi állapot megőrzése
      \item
            A betegség fogalma
      \item
            Megelőzés
      \item
            A táplálkozás és az egészség, betegség kapcsolata
      \item
            A testmozgás és az egészség, betegség kapcsolata
      \item
            Balesetek, baleset-megelőzés
      \item
            A lelki egészség.
      \item
            Önismeret, önértékelés, a másikat tiszteletben tartó kommunikáció
            módjai, ennek szerepe a másik önértékelésének segítésében
      \item
            A két agyfélteke harmonikus fejlődése
      \item
            Az érett, autonóm személyiség jellemzői
      \item
            A társas kapcsolatok
      \item
            A társadalom élete, a társadalmi együttélés normái
      \item
            A gyermek fejlődését elősegítő viszonyulás a gyermekhez -
            családban,
            iskolában
      \item
            A szenvedélybetegségek és megelőzésük (dohányzás, alkohol- és
            drogfogyasztás, játék-szenvedély, internet- és tv-függés)
      \item
            Művészeti és sporttevékenységek lelki egészséget, egészséges
            személyiségfejlődést és tanulási eredményességet elősegítő hatásai
      \item
            Médiatudatosság, a médiafogyasztás egészségvédő módja
      \item
            Az idő és az egészség, bioritmus, időbeosztás
      \item
            Tartós egészségkárosodással élő társakkal együttélés, a segítségre
            szorulók segítése
      \item
            Önmagunk és egészségi állapotunk ismerete
      \item
            A személyes krízishelyzetek felismerése és kezelési stratégiák
            ismerete
      \item
            Az idővel való gazdálkodás szerepe
      \item
            A rizikóvállalás és határai
      \item
            A tanulási környezet alakítása
      \item
            A természethez való viszony, az egészséges környezet jelentősége
\end{itemize}

\subsubsection{Indikátorok}\label{indikuxe1torok}

A teljes körű iskolai egészségfejlesztés az alábbi részterületeken
jelentkező hatások révén eredményezi a hatékonyság növekedését:

\begin{itemize}

      \item
            a tanulási eredményesség javítása
      \item
            a társadalmi befogadás és esélyegyenlőség elősegítése
      \item
            a társadalmi kapcsolatok javulása a kortársakkal, szülőkkel,
            tanárokkal
      \item
            az önismeret és önbizalom javulása
      \item
            az alkalmazkodókészség, a stresszkezelés, a problémamegoldás
            javulása
      \item
            érett, autonóm személyiség kialakulása
      \item
            a krónikus, nem fertőző megbetegedések (lelki betegségek,
            szív-érrendszeri, mozgásszervi és daganatos betegségek) elsődleges
            megelőzése
\end{itemize}

\subsubsection{A program végrehajtása - elsősorban a Harmónia (fizikai,
      lelki jóllét és kapcsolódás a környezethez) tantárgy
      keretében}
\todo{egeszsegfejlesztest nem harmonia targyban csinalunk, hanem modulokban
      jelenik meg.}
A Budapest School tanulási koncepciójának középpontjában az egyén, mint
a közösség jól funkcionáló, saját célokkal rendelkező tagja áll. Az
iskolában való fejlődése során elsősorban azt tanulja, hogy miként tud
specifikált saját célokat megfogalmazni, és hogyan tudja ezeket elérni.
Ebben a folyamatban egy mentor segíti a munkáját az iskola kezdetétől a
végéig. Ő figyel arra, hogy a gyerek fizikai és lelki biztonsága és
fejlődése folyamatos legyen, és segíti azokban a helyzetekben, amikor
biztonságérzete vagy stabilitása csökken.

A közösségben jól funkcionáló egyén belső harmóniájához ez a tantárgy a
következő fejlesztési területeket határozza meg:

\begin{itemize}

      \item
            Érzelmi és társas intelligencia
      \item
            Önismeret és önbizalom
      \item
            Konfliktuskezelés
      \item
            Rugalmasság (reziliencia)
      \item
            Kritikai gondolkodás
      \item
            Közösségi szabályok alkotásában való részvétel és azok alkalmazása
      \item
            Csapatmunka gyakorlati fejlesztése
      \item
            Oldott játék
      \item
            Egészséges testi fejlődés
      \item
            Saját igényekhez képest megfelelő táplálkozás
      \item
            A természettel való kapcsolódás
      \item
            Épített falusi és városi környezetben való eligazodás
      \item
            A technológia világában felhasználói szintű eligazodás és annak
            harmonikus alkalmazása
\end{itemize}

\paragraph{Közösségben,
      csapatban}

A Budapest School egy közösségi iskola, ahol a közösség tagjai egymással
és egymástól tanulnak. A közösségekhez való tartozáshoz, a csapatban
való gondolkodáshoz, és a családban való működéshez szükséges
képességeket leginkább úgy tudjuk fejleszteni, ha azt kezdetektől
megéljük. A közösség belső szabályainak megalkotása és az azokhoz való
kapcsolódás a tanulás folyamatosságának alapfeltétele.

\paragraph{Életképességek (life
      skills)}

Szeretnénk, ha gyerekeink általában alkalmazkodóan (adaptívan) és
pozitívan tudnának hozzáállni az élet kihívásaihoz, ha lelki és fizikai
erősségük és rugalmasságuk (rezilienciájuk) megmaradna és fejlődne. A
WHO a következőképpen definiálta (World Health Organization, 1999) az
életképességeket:

\begin{itemize}

      \item
            Döntéshozás, problémamegoldás
      \item
            Kreatív gondolkodás
      \item
            Kommunikáció és interperszonális képességek
      \item
            Önismeret, empátia
      \item
            Magabiztosság (asszertivitás) és higgadtság
      \item
            Terhelhetőség és érzelmek kezelése, stressztűrés
\end{itemize}

\paragraph{Érzelmi intelligencia}\label{uxe9rzelmi-intelligencia}

Sokszor kiemeljük az érzelmi intelligenciát, kihangsúlyozva, hogy
gyerekeinknek többet kell foglalkozniuk az érzelmek felismerésével,
kontrollálásával és kifejezésével, mint szüleinknek kellett.

\paragraph{Szabad mozgás és séta}\label{szabad-mozguxe1s-uxe9s-suxe9ta}

A különböző mozgásformák, sportok és a séta mindennapivá tétele
természetes módon, a gyerekek saját igényei szerint kell hogy történjen.

\paragraph{Gyakorlatias, mindennapi
      képességek}\label{gyakorlatias-mindennapi-kuxe9pessuxe9gek}

Ahhoz, hogy gyerekeink önállóan és hatékonyan tudják élni életüket, hogy
a társakhoz való kapcsolódás ne függőség legyen, egy csomó praktikus
mindennapi tudást el kell sajátítaniuk. A gyerekeknek folyamatosan
fejleszteniük kell az élethez szükséges minden- napi tudást a
levélszemét kezeléstől, a facebook profil tudatos használatán át,
egészen a személyi költségvetés készítéséig.

\paragraph{Egészséges
      táplálkozás}

Az egészséges táplálkozás tanulható viselkedésforma, melynek alapja nem
csupán a megfelelő élelmiszerek kiválasztása, hanem azok élettani
hatásainak megismerése, és az étkezési szokások alakítása is.

\subsection{Elsősegély-nyújtási alapismeretek}
\label{sec:elsosegely}
Ahogy azt a 20/2012 EMMI rendelet 7. § (1) ak) pont előírja, a pedagógia
progamunk ismerteti az elsősegély-nyújtási alapismeretek elsajátításával
kapcsolatos iskolai tervet.

A cél, hogy a gyerekek és tanárok megtanulják aktívan úgy alakítani
környezetüket és viselkedésüket, hogy a balesetek számát minimalizálják, hogy
felismerjék, amikor segítségre van szükség, hogy hatékonyan segítsenek és
tudjanak segítséget hívni. Ehhez gyakorlásra, a témával kapcsolatos
védett időre van szükség. Ezért a Harmónia tantárgy 2., 4., 6. 8., 10.
évfolyamszintja csak akkor teljesíthető,
ha a gyerek minden második évben elvégez egy minimum négy órás modult, aminek
célja,
\begin{itemize}
      \item hogy a gyerekek sajátítsák el a legalapvetőbb és legkorszerűbb
            elsősegély-nyújtási
            módokat, azaz tudjanak egymásnak segíteni baj esetén; (nem csak
            elméletben,
            hanem gyakorlatban)
      \item sajátítsák el, mikor és hogyan kell mentőt, segítséget hívni;
      \item foglalkozzanak azzal, hogyan tudják környezetüket, csoportjukat,
            mikroiskolájukat
            biztonságosabbá tenni és ezt dokumentálják is.
\end{itemize}

A modulszervezője próbáljon meg elsősegély-nyújtási bemutatót szervezni a
gyerekeknek az Országos Mentőszolgálat, a Magyar Ifjúsági Vöröskereszt vagy az
Ifjúsági Elsősegélynyújtók Országos Egyesületének vagy más, magyar, vagy
külföldi képesítést szerzett szakember bevonásával.

Az elsősegély-nyújtási alapismeretek elsajátításával kapcsolatos feladatok
megvalósításának elősegítése érdekében:
\begin{itemize}
      \item az iskola kapcsolatot épít ki az Országos Mentőszolgálattal, Magyar
            Ifjúsági
            Vöröskereszttel vagy az Ifjúsági Elsősegélynyújtók Országos
            Egyesületével
            tanulóink - választásuk szerint - bekapcsolódhatnak az
            elsősegély-nyújtással
            kapcsolatos iskolán kívüli vetélkedőkbe.
      \item  minden második évben legalább egyszer a tanároknak lehetőséget
            biztosít
            elsősegély tanfolyam látogatására.

\end{itemize}

\subsection{A
      közösségfejlesztés}

A közösségfejlesztéssel, az iskola szereplőinek együttműködésével
kapcsolatos feladatok

\todo[inline]{
      csak a nevelési-oktatási tartalmak relevanciája vonatkozásában, mert
      egyébként SZMSZ-kompetencia)
}

\subsection{Nemzetiségek
      megismerése}\label{sec:nemzetiseg}

Az iskola környezetében elő nemzetiségek kultúrájának megismerését fontosnak
tartja az iskola.
Ezért minden második tanévben a településen működő nemzetiségi
önkormányzatokkal
kapcsolatot vesz, és  velük együttműködve legalább 4 órás
modululakt alakít ki. A modulok célja, hogy a modul résztvevői megismerjék a
nemzetiségekről azt, amit az önkormányzatok fontosnak tartanak megmutatni a
nemzetiségekről.

Minden második évben legalább három önkörmányzattal három különböző modult
kínál fel az iskola választásra.\footnote{Amelyik településen ennél több
      nemzetiségi önkormányzat működik, ott a tanév
      megkezdése előtt, augusztus folyamán véletlenszerűen sorsoljuk ki az
      adott évi
      3 megismerendő nemzetiséget.}

\subsection{az egészségnevelési és környezeti nevelési elvek}

\begin{quote}
      az egészségnevelési és környezeti nevelési elvek
\end{quote}
\todo[inline]{hianyzik a az egészségnevelési és környezeti nevelési
      elvek}
\section{Esélyegyenlőség}\label{esuxe9lyegyenlux151suxe9g}

\begin{quote}
      a gyermekek, tanulók esélyegyenlőségét szolgáló intézkedések
\end{quote}
\todo{Kell valami az ''a gyermekek, tanulók esélyegyenlőségét szolgáló
      intézkedések''-rol. Valami már van a bevezetőben.}
\section{Tantárgyak}
\label{sec:tantargyak}
A Budapest School a ma gyerekeinek kínál olyan oktatást, ami segíti felkészíteni őket a jövő kihívásaira. Információs társadalmunk legnagyobb kihívása az adaptációs képességünk fejlesztése, ez az alapja annak, hogy képesek legyünk eligazodni a folyamatosan változó, komplex világunkban. A tanulásunk célja, hogy boldog, hasznos és egészséges tagjai legyünk a társadalomnak. Iskolánkban a tanulás három rétege, a tudásszerzés, a megtanultakat elmélyítő önálló gondolkodás és az aktív alkotás egyszerre jelennek meg.

A kerettanterv a célok eléréséhez a miniszter által kiadott kerettantervek tantárgyi struktúráját használja  a moduláris tanulás tartalmi keretezéséhez. A keretezésen azt értjük, hogy a tantárgyak tartalma határozza meg, hogy mivel kell mindenképp a Budapest School iskolákban foglalkozni, mit kell mindenképp megtanulni. 
Az egyes modulok ezen tantárgyak tanulási eredményeinek elérését támogatják.

\subsection{Tanulási eredmények -- a formális tanulás alapegységei}
\label{sec:tanulasi_eredmenyek}
A kerettanterv a tantárgyak témaköreit, tartalmát és követelményeit \emph{tanulási eredmények} halmazával adja meg, ezzel igazodva az Nkt.~5.~§ (5) pontjához. A tanulási eredmények (learning out\-comes) tudás, képesség, kompetencia, attitűd kontextusában meghatározott kijelentések arra vonatkozóan, hogy a tanulónak mit kell tudnia, mit kell értenie, és mire legyen képes, miután lezárt egy tanulási folyamatot, függetlenül attól, hogy hol, hogyan, mikor szerezte meg ezeket a kompetenciákat \citep{learning_outcomes}.  Tanulási eredmény a kerettanterv szellemében minden lehet, amit a gyerek egy tanulási folyamat során elsajátított és ezt demonstrálni tudja.

Az eredmény eléréséhez vezető út a modulokon keresztül történik, és a tanulás folyamata történhet az iskolában vagy azon kívül, lehet formális, non-formális vagy informális.
Az egyes modulok különféle tanulási eredmények elérését is támogathatják, ezzel több tantárgy részcéljait is teljesíthetik.

A tanulási eredmények több funkciót látnak el a kerettantervben.

\begin{itemize}

      \item A kerettanterv évfolyamonként meghatározza az adott tantárgy teljesítéséhez elérendő tanulási eredményeket. Egy gyerek akkor  léphet egy tantárgyból évfolyamszintet, ha a tantárgyhoz tartozó követelményeket teljesítette.

      \item A tanulási eredmények a modulok (és így a mindennapokban szervezett foglalkozások, órák stb.) építőelemei. Egy-egy modul célját a  tanulásszervezők az elérendő tanulási eredmények	összeválogatásával és saját célokkal, érdeklődéssel kiegészítve adják meg, figyelembe véve az életkori  sajátosságok, az egymásra épülés és az átjárhatóság  követelményeit.
      \item A tanulási eredmények alapján osztályzatok és évfolyamok egy átlátható és egyszerű számítás segítségével megállapíthatóak, ami biztosítja, hogy a Budapest  School tanulója más rendszerben működő iskolába is át tud menni, és a felvételikre is tud jelentkezni.
\end{itemize}

A kerettanterv tantárgyankénti és félévenkénti bontásban adja meg a továbbhaladáshoz elengedhetetlen tanulási eredmények listáját.

A tantárgyi definíciókhoz a miniszter által kiadott kerettanterv ,,elvárt eredmények a tanulási ciklus végén" fejezetek felsorolásait alakítottuk át egységes nyelvezetre, hogy azok valóban kompetenciákat írjanak.

A tantárgyi specifikációk nem térnek ki rész\-letesen a tematikákra. Ez szabadságot ad a tanároknak arra, hogy a tanmenet tekintetében akár jelentős eltérések legyenek addig, amig a miniszter által kiadott kerettanterv mérhető tanulási eredményei teljesülnek. A tanulási eredmény alapú szabályozás folyamatos visszacsatolást tud adni a tanulónak és a tanároknak, megmutatva, melyik tanulási eredményeket kell még elérni a következő szintre való lépéshez.

\subsection{Tantárgyak szerepe a mindennapokban}
A Budapest School iskoláiban a tantárgyak ugyanúgy kapnak szerepet, mint a NAT által definiált kulcskompetenciák, fejlesztési területek: tanár sose mondja azt a gyerekeknek, hogy „most kezdeményezőképességet és vállalkozói kompetenciát fejlesztünk'', hanem a különböző feladatok elvégzése eredményeképp történik a fejlesztés. A Budapest School iskolákban a tantárgyközi tevékenységek vannak előtérben. A tantárgyak a tanulás tartalmi elemeinek forrásai és keretei: a tanulandó dolgok listájaként működnek. Az, hogy milyen csoportosításban történik a tanulás, az a modulvezetőkre van bízva.

A tantárgyak ezért elsősorban a modulok kiírásakor és azok kimeneti értékelésekor jelennek meg, a mindennapok struktúráját, a napi- és hetirendet azonban a modulok adják. Egyes modulok több tantárgy fejlesztési céljainak is eleget tehetnek, több tantárgy tanulási eredményének elérését is célul tűzhetik ki, összhangban a NAT-tal. A tantárgyaknak ezzel együtt fontos célja, hogy segítse a tanulás tartalmi egyensúlyának fennmaradását. A tanulásszervezők, modulvezetők szakképesítése nem köthető a Budapest School tantárgyaihoz, felelősségük, hogy a saját moduljukban megfelelően tudják szervezni a tanulást, és legfőképp, hogy saját moduljuk megtartására alkalmasak legyenek.

\subsection{Modulok és tanulási eredmények}
\label{sec:modulok_es_tanulasi_eredmenyek}
A gyerekek egyik feladata az iskolában, hogy tanulási eredményeket érjenek el. Ezt megtehetik a modulok elvégzésével, vagy más tanulási helyzetekben. A tanulási eredményeket a portfólióban rögzítik. A mentor feladata, hogy folyamatosan kövesse, hogy megfelelő haladás történik-e a portfólióban a tanulási eredmények és a saját célok tekintetében. Az évfolyamszintlépés a portfólióban összegyűlt tanulási eredmények alapján történhet meg.

A modul kecsegtet a gyerekek haladásához releváns tanulási eredményekkel, a gyerekek által meghatározott saját célokkal és olyan kimenettel, amely a portfólióban rögzíthető, legyen az egy alkotás, az elért fizikai vagy szellemi eredmény dokumentációja, vagy egy értékelő visszajelzés. A modulok tehát tartalmaznak tanulási eredményeket, az önálló gondolkodás, szabad alkotás lehetőségét, és teret engednek az alkotásra, létrehozásra.

\paragraph{A modulok különféle tanulási eredmények elérését teszik elérhetővé}

Modulok tervezésekor és összeállításakor a tanulásszervezők a modulvezetővel közösen határozzák meg a modul céljait, de azok meghirdetéséért mindig a tanulásszervezők felelnek. A célok között fel kell sorolni, hogy milyen tanulási eredmények elérését várhatják el a gyerekek a modulon való részvételtől.

Például a 6--8 éves gyerekek számára megtervezett ,,\emph{3d nyomtató használata}'' modul során azon kívül, hogy megismerik a 3d nyomtatás folyamatát, a modul célja, hogy a gyerekek számára elérhetővé tegye a ,,\emph{Kocka, téglatest jellemzőit ismeri, képes őket létrehozni.}'' (Matematika tantárgy, 4. évfolyam 2. félév) tanulási eredményt is.

Lehetőség van egy modul esetében több tantárgyból való tanulási eredmény kiválasztására, ezzel biztosítva az interdiszciplinaritást, valamint a Budapest School tantárgyi fejlesztési céljaihoz való integrált kapcsolódást.

A tanulási eredmények egy időbeni egymásra épülést feltételeznek, melyben azonban van lehetőség előre- és hátrafele is lépni. Előre, amennyiben a modul meghirdetésekor az arra jelentkező gyerekcsoportnál a megfelelő előkészítés megtörtént, hátra, amennyiben ezt ismétlés/felzárkóztatás jelleggel szükségesnek ítéli a mentor vagy a modult szervező, vezető. Vagyis akkor foglalkozzon egy gyerek a 10~000-es számkörrel, ha a 100-as számkört már begyakorolta. Az egymásra épülésért a modult meghirdető tanulásszervező felel. A példát folytatva a 3d nyomtató használata modul lehetővé teszi, hogy a gyerek elérje a következő eredményeket is: \emph{,,Ismeri a számítógép
      részeinek és perifériáinak funkcióit, azokat önállóan használja.''}
(Harmónia, Informatika, 5. évfolyam 1. félév), és  \emph{,,Használati utasításokat
      értő módon olvas és tart be.''} (Harmónia, Életvitel, 4. évfolyam 2. félév)

\paragraph{Új tanulási eredmények}

A gyerekek olyan tanulási eredményt is elérhetnek, ami a modulok céljai között eredetileg nem volt megadva, mert

\begin{itemize}
      \item lehetőségük van egyénileg is tanulni;

      \item tanulási eredményekkel járnak a projektek, az iskolai lét, a közösségi élet és még számos informális és non-formális tanulási helyzet;

      \item egy modul során is alakulhatnak előre nem tervezett helyzetek, amik hozzásegíthetik a gyerekeket tanulási eredmények eléréséhez.
\end{itemize}

Az újonnan létrejövő tanulási eredmények is bekerülnek a portfólióba.

\paragraph{Tanulási eredmények dokumentációja}

Minden modul dokumentálásra kerül, hogy annak célja, elért eredményei nyilvánosak legyenek a Budapest School valamennyi mikroiskolája számára, és ha szükséges, újra meg lehessen hirdetni. A tanulási eredmények egy, a modulhoz kapcsolódó terv-tény összehasonlítás alapján kerülnek meghatározásra. Az elért eredmények újra elérhetőek, amennyiben a folyamatos fejlődés biztosítva van.

\paragraph{Egységes modulok egyedi alkalmazása}

Egy modul elvégzésével egy-egy gyerek más tanulási eredményt is elérhet.

\begin{itemize}
      \item
            Működhet a differenciálás, tehát nem minden gyerek ugyanazt és ugyanúgy csinálja a foglalkozásokon. Egy modulban tud együtt	tanulni az a gyerek, aki még ,,\emph{Ismeri az írott és nyomtatott  betűket''} eredményért dolgozik, és az, aki ,,\emph{Jelöli helyesen a j	hangot 30--40 begyakorolt szóban''.}
      \item
            A modulnak része lehet testre szabható sáv. Például egy tudományos kísérletező modulban néhány gyerek a rövid távú memória és a	fáradtság kapcsolatáról kutat, a másik csoport az esőzés és a	közlekedési dugók kialakulása közti kapcsolatot vizsgálja. Minden  gyerek elérheti a ,,\emph{valós folyamatokat képes elemezni a folyamathoz tartozó függvény grafikonja alapján}''  (forrás, Matematika) eredményt, de a ,,\emph{környezettudatos közlekedésszemlélet}'' (forrás, Harmónia)	eredményt is elérheti.
      \item
            Egy-egy gyerek saját tanulási célja érdekében extra lépéseket tehet, és olyan eredményeket is el tud érni, amit mások nem.	Például egy modul végén önálló prezentációt, saját kutatási  tervet vagy egy kész működő modellt alkothat.
\end{itemize}

\subsubsection{Kötelező tanulási eredmények}
\label{sec:kotelezo_tanulasi_eredmenyek}
A kerettanterv kötelező tanulási eredményként definiálja mindazokat az eredményeket, melyek a kötelező érettségi tárgyak teljesítéséhez szükségesek. Ezeket minden mikroiskola elérhetővé kell hogy tegye a gyerekek számára a modulok választékában.

Ezek az 1--4. évfolyamszinteken a miniszter által kiadott kerettantervek \emph{Magyar nyelv és irodalom}, \emph{Matematika}, \emph{Idegen nyelv}, \emph{Testnevelés és sport} tantárgyakból származó tantárgyak tanulási eredményei, és 5.~évfolyamszinttől a \emph{Történelem, társadalmi és állampolgári ismeretek} tantárgy eredményei. További kötelező tanulási eredményként jelennek meg 9.~évfolyamtól a választott érettségi tantárgyhoz kapcsolódó eredmények. Ezek a tanulási eredmények megtalálhatók a kerettanterv tantárgyainak elérhető eredményei között.

\paragraph{Kötelező modulok}
A kerettanterv és a pedagógiai program is előírhat kötelező modulokat a mikroiskolák számára. Ilyenek például a 11.~évfolyamszinten belépő érettségire felkészítő modulok, a minden mikroiskolára egységes pedagógiai program tetszőleges kötelező modult írhat elő. Így lehet biztosítani a kötelező tartalmi elemek és foglalkozás -- például elsősegélynyújtás vagy a nemzetiségekkel való ismerkedés -- elérhetőségét.

\subsubsection{Monitorozás}

Kötelező elérni az eredményeket? Nem tudunk hatalmi szóval tanulásra bírni gyereket, mert lehet, hogy annyira nem akarja, vagy nincs meg hozzá a képessége. A kerettanterv a tanároknak ad keretet. Azonban a fenntartó által üzemeltetett rendszerrel az iskola  monitorozza a haladást, és ha valaki a kötelező tanulási elemekkel nem halad, akkor az iskola erre felhívja a figyelmét. Mivel a többség haladni fog, ezért előre tudja az iskola jelezni, hogy le fog szakadni a többiektől, és túl nagy lesz az évfolyamszint-különbség közöttük. Ezekben az esetekben a mentortanárnak, a gyereknek és a szülőnek reagálnia kell a helyzetre. A fenntartó által működtetett monitorozó és minőségfejlesztő rendszerről \aref{sec:minosegbiztositas} fejezet ír részletesen.


\subsection{Tantárgyi struktúra és óraszámok}
\label{sec:tantargyi_struktura}
\paragraph{Heti óraszámok} 
A Budapest School közösségi tanulási élményeket és modulokat szervez a gyerekeknek, egyúttal lehetőséget ad arra, hogy a gyerekek a közösen kialakított szabályaik mentén, a tanulásszervezők felügyeletével a Budapest School székhelyén vagy egyes telephelyein, vagy más, erre alkalmas tanulási környezetben tartózkodjanak. A közösségben együtt töltött idő tanulásnak, fejlődésnek minősül akkor is, ha az nem egy modulhoz kapcsolódik, hanem az ebéd élvezetéhez, vagy épp a parkban a lehulló falevelek neszének megfigyeléséhez.

A gyerekek, a tanítási szüneteket leszámítva, általában naponta 8 órát tartózkodnak az iskolában.\footnote{Ez alól több kivétel lehet: külső foglalkozások, otthoni, egyéni tanulás. Ezekről külön megállapodást kell kötni a mentorral.} Ezekben az időkben vannak a tanítási órák, foglalkozások, szakkörök, műhelyek. Az egyes mikroiskolák ettől 20\%-ban bármelyik irányban eltérhetnek, ha ez segíti a tanulásszervezők munkáját és a gyerekek fejlődését. Így hetente minimum $5 \cdot 8 \cdot 0,8 = 32$ órát, maximum $48$ órát töltenek az iskolában.

Ennek az 1--4.~évfolyamszinten körülbelül a felét, azaz 18--22 órát, majd később 3--4.~évfolyamonszinten 20--26 órát; az 5--12.~évfolyamszinten pedig kétharmad részét, azaz 24--32 órát töltik a gyerekek előre eltervezett módon, azaz modulokkal. A többi időben a tanárok vezetése és felügyelete mellett szabadon alkotnak, játszanak, pihennek, közösségi életet élnek. Ugyanezek a számok a miniszter által kiadott kerettantervekben 1. évfolyamon 25 és a 10. évfolyamon 36 órát tesznek ki.

\paragraph{Modulok és a tantárgyi óraszámok}
A modulok során több tantárgyi tananyagot is érinthetnek a modul résztvevői. Egy modul így több tantárgyi órát is lefed, ráadásul ez óraszám megtakarítással is jár. 5 óra angolul tartott dráma foglalkozás egyszerre számíthat 5 óra \emph{magyar irodalom és nyelvnek} és 5 óra \emph{idegennyelv} órának. A tantárgyköziségből spórolt óraszámokra a kerettanterv óvatos előírást ad: egy egységnyi idő alatt átlagban legalább 1,25 egységnyi tantárgyi óraszámot kell teljesíteni (szemben az előző példában szereplő kettes szorzóval).

A modulok tantárgyi óraszámát a teljes modul hosszára kell számítani, és nem hetente. Egy összevont természettudományi modul, ami érinti a kémiát és a fizikát is, nem kell, hogy minden héten járjon kémia tanulási eredménnyel. Több tantárgyat lefedő modul esetén a tantárgyi óraszámokat úgy kell számolni, hogy először meg kell állapítani, hogy átlagban az idő hány százalékában foglalkozik a modul egy-egy tantárgy anyagával, majd a modul teljes hosszából becsülhető a tantárgyi óraszám. Például egy \emph{tudományos kísérletezés} modul során az idő 30\%-ban foglalkozunk kémiával, 40\%-ban fizikával és 30\%-ában szociálpszichológiával. A modul egy trimeszteren keresztül tart, kéthetente 4 órában. Ebből számolható a modul teljes hossza, ami itt $\frac{12}{2}x4 = 24$ óra. Ez hetente $24x0,3 = 7,2$ óra kémiának felel meg. 
A példa is mutatja, hogy a \emph{kerettanterv megengedi, hogy nem egészszámú óraszámokkal dolgozzon az iskola}. 

\subsection{Tantárgyi óraszámok trimeszterenként}
A Budapest School modulalapú tanulásszervezéséhez jobban illik trimeszterenként megadni az elvárt óraszámokat, mert ahogy \aref{sec:tanev_ritmusa}.~fejezet is mutatja, nem minden hét ugyanolyan az iskolában. A kerettanterv a miniszter által kiadott kerettantervek óraszámait veszi alapul, annak heti óraszámait szorozza fel kilenccel. 

\begin{landscape}
\begin{table}[]
  \begin{tabular}{l|l|l|l|l|l|l|l|l|l|l|l|l}
  
                                                        & \multicolumn{12}{l}{\textbf{Évfolyamszintek}}                                                                                                                                                           \\ \hline
    \textbf{Tantárgyak}                                          & 1                                     & 2           & 3           & 4           & 5           & 6           & 7           & 8           & 9           & 10          & 11          & 12          \\ \hline
    Biológia - egészségtan                              &     &     &     &     &     &     & 18  & 9   &     & 18  & 18  & 18  \\\hline
    Dráma és tánc/Mozgóképkultúra és médiaismeret       &     &     &     &     & 9   &     &     &     & 9   &     &     &     \\ \hline
    Életvitel és gyakorlat                              & 9   & 9   & 9   & 9   &     &     &     &     &     &     &     & 9   \\\hline
    Ének-zene                                           & 18  & 18  & 18  & 18  & 9   & 9   & 9   & 9   & 9   & 9   &     &     \\\hline
    Erkölcstan                                          & 9   & 9   & 9   & 9   & 9   & 9   & 9   & 9   &     &     &     &     \\\hline
    Etika                                               &     &     &     &     &     &     &     &     &     &     & 9   &     \\\hline
    Fizika                                              &     &     &     &     &     &     & 18  & 9   & 18  & 18  & 18  &     \\\hline
    Földrajz                                            &     &     &     &     &     &     & 9   & 18  & 18  & 18  &     &     \\\hline
    I. idegen nyelv                                     &     &     &     & 18  & 27  & 27  & 27  & 27  & 27  & 27  & 27  & 27  \\\hline
    II. idegen nyelv                                    &     &     &     &     &     &     &     &     & 27  & 27  & 27  & 27  \\\hline
    Informatika                                         &     &     &     &     &     & 9   & 9   & 9   & 9   & 9   &     &     \\\hline
    Kémia                                               &     &     &     &     &     &     & 9   & 18  & 18  & 18  &     &     \\\hline
    Környezetismeret                                    & 9   & 9   & 9   & 9   &     &     &     &     &     &     &     &     \\\hline
    Magyar nyelv és irodalom                            & 63  & 63  & 54  & 54  & 36  & 36  & 27  & 36  & 36  & 36  & 36  & 36  \\\hline
    Matematika                                          & 36  & 36  & 36  & 36  & 36  & 27  & 27  & 27  & 27  & 27  & 27  & 27  \\\hline
    Művészetek                                          &     &     &     &     &     &     &     &     &     &     & 18  & 18  \\\hline
    Technika, életvitel és gyakorlat                    &     &     &     &     & 9   & 9   & 9   &     &     &     &     &     \\\hline
    Természetismeret                                    &     &     &     &     & 18  & 18  &     &     &     &     &     &     \\\hline
    Testnevelés és sport                                & 45  & 45  & 45  & 45  & 45  & 45  & 45  & 45  & 45  & 45  & 45  & 45  \\\hline
    Történelem, társadalmi és állampolgársági ismeretek &     &     &     &     & 18  & 18  & 18  & 18  & 18  & 18  & 27  & 27  \\\hline
    Vizuális kultúra                                    & 18  & 18  & 18  & 18  & 9   & 9   & 9   & 9   & 9   & 9   &     &     \\\hline \hline
    \textbf{Összesen}                                   & 207 & 207 & 198 & 216 & 225 & 216 & 243 & 243 & 270 & 279 & 252 & 234
    

  \end{tabular}
  \caption{A minimális óraszámok trimeszterekre számolva. A táblázatban szereplő számok a miniszter által kiadott kerettanterv óraszámai alapján készültek.}  
  \label{tbl:oraszamok}
\end{table}

\end{landscape}

\paragraph{Heti óraszámok}

A Budapest School közösségi tanulási élményeket és modulokat szervez a
gyerekeknek, egyúttal lehetőséget ad arra, hogy a gyerekek a közösen kialakított
szabályaik mentén tanulásszervezők felügyeletével a Budapest School székhelyén
vagy egyes telephelyein, vagy más, erre alkalmas tanulási környezetben
tartózkodjanak. A közösségben együtt töltött idő tanulásnak, fejlődésnek
minősül akkor is, ha az nem egy modulhoz kapcsolódik, hanem az ebéd
élvezetéhez, vagy épp a parkban a lehulló falevelek neszének megfigyeléséhez.

A gyerekek, a tanítási szüneteket leszámítva, általában naponta 8 órát
tartózkodnak az
iskolában.\footnote{Ez alól több kivétel lehet: külső foglalkozások, otthoni,
  egyéni tanulás. Ezekről külön megállapodást kell kötni a mentorral.} Ezekben
az
időkben vannak a tanítási órák, foglalkozások, szakkörök,
műhelyek. Az egyes mikroiskolák ettől 20\%-ban bármelyik irányban eltérhetnek,
ha ez segíti a tanulásszervezők munkáját és a gyerekek fejlődését. Így hetente
minimum $5 \cdot 8 \cdot 0,8 = 32$ órát, maximum 48 órát töltenek az iskolában.

Ennek 1--4 évfolyamszinten körülbelül a felét, azaz 16--24 órát, a 5--12
évfolyamszinten a  kétharmad részét, azaz 21--32 órát
töltik a gyerekek előre eltervezett módon, azaz modulokkal. A többi időben a tanárok
vezetése és felügyelete mellett szabadon alkotnak, játszanak, pihennek,
közösségi életet élnek.

Mivel az elvárt kiegyensúlyozottság miatt mind a három tantárgyra körülbelül
ugyanannyi energiát kell fektetni, így az egyes tantárgyakra a teljes
rendelkezésre álló időkeret egyharmad részét kell számolni. Ettől az iskolák
$\pm$ 20\%-ban eltérhetnek, így kiszámolható, hogy minimum mennyi időt kell
egy-egy gyereknek egy héten egy tantárggyal foglalkoznia. Ezt összegzi
\aref{tbl:oraszamok}. táblázat.

\begin{table}

  \begin{tabular}{ l|l|l }

    \textbf{Tantárgy} & \textbf{1--4 évfolyam}                               & \textbf{5--12 évfolyam}
    \\ \hline
    Harmónia          & $\frac{5 \times 8 \times 0,8}{2} \times \frac{1}{3}
      \times 0.8 =
    4,27$ óra         &
    $\frac{5 \times 8 \times 0,8 \times 2}{3} \times \frac{1}{3} \times 0,8 =
      5,69$
    óra
    \\ \hline
    STEM              & 4,27 óra
                      & 5,69 óra                                                                     \\
    \hline
    KULT              & 4,27 óra
                      & 5,69 óra                                                                     \\
    \hline

  \end{tabular}
  \caption{Az elvárt kiegyensúlyozottság miatt a tantárgyakkal egyenlő
    minimális óraszámban kell foglalkozni.}
  \label{tbl:oraszamok}
\end{table}

Fontos, hogy \emph{egy-egy modul több tantárgy fejlesztési céljaihoz és
  tanulási eredményeihez is kapcsolódhat.}

\section{NAT céljainak támogatása}
\label{sec:nat_celjai}
A Nemzeti alaptantervben szereplő fejlesztési célok elérését és a
kulcskompetenciák fejlődését több minden támogatja:

Egyrészt a tantárgyak lefedik a NAT fejlesztési céljait, kulcskompetenciáit és
műveltségi területeit, mert a jelenleg érvényben lévő, a miniszter által az
\emph{51/2012. (XII. 21.) számú EMMI rendelet I-IV. mellékletében} kiadott
kerettantervek \citep{ofi:kerettanterv} tanulási, tanítási eredményeiből
indultunk ki. Mivel a rendeletben szereplő kerettantervek teljesítik a NAT
feltételeit, így a Budapest School tantárgystruktúrája is teljesíti ezeket.

Másrészt az iskola életében, folyamatában való részvétel már önmagában
biztosítja a kulcskompentenciák fejlődését és a NAT fejlesztési céljainak
teljesülését sok esetben.

A \ref{tbl:nat_fejlesztesi} táblázat bemutatja a NAT fejlesztési területeihez
való kapcsolódást, a
\ref{tbl:nat_kulcs} táblázat pedig az illeszkedési pontokat a NAT
kulcskompetenciáihoz.

\begin{table}

  \begin{tabular}{p{5cm}|>{\raggedright}p{3cm}|p{3cm}}

    \textbf{A NAT fejlesztési céljai}               & \textbf{Tantárgyak}  & \textbf{Struktúra}           \\
    \hline
    Az erkölcsi nevelés                          & KULT, harmónia       & közösség                     \\ \hline
    Nemzeti öntudat, hazafias nevelés            & KULT, harmónia       & projektek                    \\ \hline
    Állampolgárságra, demokráciára nevelés       & KULT, harmónia       & közösség                     \\ \hline
    Az önismeret és a társas kultúra fejlesztése & , harmónia, STEM & saját
    tanulási út, közösség                                                                              \\ \hline
    A családi életre nevelés                     & harmónia             &                              \\ \hline
    A testi és lelki egészségre nevelés          & harmónia             & közösség                     \\ \hline
    Felelősségvállalás másokért, önkéntesség     & harmónia             & közösség, pro\-jek\-tek      \\
    \hline
    Fenntarthatóság, környezettudatosság         & harmónia, STEM       & projektek                    \\ \hline
    Pályaorientáció                              & KULT, harmónia, STEM & saját tanulási út            \\ \hline
    Gazdasági és pénzügyi nevelés                & KULT, harmónia, STEM & projektek                    \\ \hline
    Médiatudatosságra nevelés                    & KULT                 & projektek                    \\ \hline
    A tanulás tanítása                           & KULT, harmónia, STEM & saját tanulási út, mentorság \\

  \end{tabular}
  \caption{A NAT fejlesztési céljainak elérését nemcsak a tantárgyak, hanem az
    iskola struktúrája is támogatja.}
  \label{tbl:nat_fejlesztesi}
\end{table}

A \emph{saját tanulási} út fogalma például önmagában segíti a tanulás
tanulását, hiszen az a gyerek, aki képes önmagának saját célt állítani (mentori
segítséggel), azt elérni, és a folyamatra való reflektálás során képességeit
javítani, az fejleszti a tanulási képességét.

Vagy másik példaként, a Budapest School iskoláiban a \emph{közösség} maga hozza
a működéséhez szükséges szabályokat, folyamatosan alakítja és fejleszti saját
működését a tagok aktív részvételével. Ez az aktív állampolgárságra, a
demokráciára való nevelés Nemzeti alaptantervben előírt céljait is támogatja.

\begin{table}
  \centering
  \begin{tabular}{p{5cm}|>{\raggedright}p{3cm}|p{3cm}}

    \textbf{NAT kulcskompetenciái}                     & \textbf{Tantárgyak}  &
    \textbf{Struktúra}                                                                                        \\ \hline
    Anyanyelvi kommunikáció                            &  KULT                & tanulási szerződés, portfólió \\ \hline
    Idegen nyelvi kommunikáció                         & KULT                 & idegennyelvű modulok          \\ \hline
    Matematikai kompetencia                            & STEM                 &                               \\ \hline
    Természettudományos és technikai kompetencia       & STEM                 & projektek                     \\ \hline
    Digitális kompetencia                              & harmónia, STEM       & digitális portfólió-kezelés   \\ \hline
    Szociális és állampolgári kompetencia              & harmónia             & saját tanulási út,
    közösség                                                                                                  \\ \hline
    Kezdeményezőképesség és vállalkozói kompetencia    & KULT, harmónia, STEM & saját
    tanulási út, közösség                                                                                     \\ \hline
    Esztétikai-művészeti tudatosság és kifejezőkészség & harmónia, KULT       &                               \\ \hline
    A hatékony, önálló tanulás                         & KULT, harmónia, STEM & saját tanulási út,
    mentorság                                                                                                 \\

  \end{tabular}
  \caption{A NAT kulcskompetenciáinak fejlesztését támogatják a tantárgyak és
    az iskola felépítése is.}
  \label{tbl:nat_kulcs}
\end{table}

\section{Tankönyvek kiválasztása}

\begin{quote}
      Az oktatásban alkalmazható tankönyvek, tanulmányi segédletek és
      taneszközök kiválasztásának elvei (figyelembe véve a tankönyv
      térítésmentes igénybevétele biztosításának kötelezettségét).
\end{quote}

Modulvezetők minden esetben maguk választják a modulhoz szükséges
tankönyvek, szoftverek, weboldalak és egyéb eszközöket úgy, hogy

\begin{itemize}

      \item
            az a megfelelő legyen annak a csoportnak, ahhoz a célhoz, amit el
            akar
            érni
      \item
            minden esetben legyen mindenki számára elérhető (esetek többségeben
            értsd ingyenes) megoldás
      \item
            modulvezetők bátorítva vannak arra, hogy új dolgokat próbáljanak
            ki,
            és tapasztalataikat az iskola többi tanárával megosszák.
\end{itemize}

Mivel a Budapest School kerettantervének értelmében az egyéni célok
legalább 50\%-át az állami kerettantervben meghatározott fejlesztési
célok közül kell választani, az ehhez szükséges ismeretek megszerzéséhez
a Budapest School az Oktatási Hivatal általi jegyzékben államilag
támogatott OFI által fejlesztett tankönyveket veszi alapul. A Budapest
School tanárcsapatának lehetősége van arra, hogy ettől eltérő, a
mindenkori tankönyvjegyzékben szereplő tankönyvvel segítse el a
kerettantervben meghatározott eredménycélok elérését. És arra is
lehetősége van, hogy egyátalán ne használjon tankönyvet, mert sokszor az
internet elegendő információt tartalmaz.

A Budapest School pedagógiai programjának alapja, hogy a gyerekek egyéni
céljaira szabott tanulási terveket készít. Ennek előfeltétele, hogy a
könyvek használata is ehhez kapcsolódó módon rugalmasan történjen,
minden esetben az adott tanulási modul igényeihez szabva. Ennek
érdekében a program pedagógusai folyamatosam állítják össze a gyerekek
eltérő céljaihoz és képességszintjeihez igazodó differenciált
tevékenységek és feladatsorok rendszerét.

\section{Érettségire készülés }

\begin{quote}
      középiskola esetén azon választható érettségi vizsgatárgyak megnevezése,
      amelyekből a középiskola tanulóinak közép- vagy emelt szintű érettségi
      vizsgára való felkészítését az iskola kötelezően vállalja, továbbá annak
      meghatározáse, hogy a tanulók milyen helyi tantervi követelmények
      teljesítése mellett melyik választható érettségi vizsgatárgyból tehetnek
      érettségi vizsgát,
\end{quote}

Az iskola a kötelező középszintű érettségi vizsgatárgyakra való
felkeszítést kötelezően vállalja érettségi felkészítő modulok
szervezésével. A
választható tantárgyak és az emeltszintű érettségi vizsgára csak akkor
szervez egy mikroiskola modult, ha arra legalább a közösség 10\%-a
igényt tart. Abban az esetben, ha minden választható tantárgyat csak kevesebb,
mint 10\% választ, és így a közösség nem tud válaszható érettségi tárgyat
választani, a fenntartó véletlenszerűen sorsol legalább egy választható tárgyat
a gyerekek által megjelöltből.

Érettségire felkészítő modulokat akkor kell meghírdetni, amikor a gyerekek
elérik a 11. évfolyamszintet minden tárgyból.

\subsubsection{középszintű érettségi vizsga témakörei}

\begin{quote}
      középiskola esetén az egyes érettségi vizsgatárgyakból a középszintű
      érettségi vizsga témakörei
\end{quote}

\todo{Balazs, HP szerint nincs szukseg ,,középszintű érettségi vizsga
      témakörei'' }

\section{Önkéntes közösségi
  szolgálat}\label{uxf6nkuxe9ntes-kuxf6zuxf6ssuxe9gi-szolguxe1lat}

Az Nkt 4. § (15) pontjában definiált közösségi szolgálat is modulként
kerül meghirdetésre, amit 12. évét betöltött gyerek választhat csak.
Közösségi szolgálatként elfogadható, ha a Budapest School iskola egy
másik mikroiskolájában segít a gyerek.
