\section{A választott kerettanterv
  megnevezése}

Budapest School Általános Iskola és Gimnázium az oktatásért felelős
miniszter által jóváhagyott Budapest School Kerettanterv alapján
működik. Az iskola munkatársai a kerettantervet, a pedagógia programot,
sőt még a szervezeti és működési szabályokat is együtt, egy egységként
állították össze. Ez a Budapest School Módszer, aminek fő célja, hogy a
gyerekek úgy tudják azt és akkor tanulni, amit szeretnek, vagy amire
szükségük van, hogy közben a tanárok, szülők számára is kiszámítható,
tervezhető, biztonságos tanulási környezetet biztosít az iskola.

Az iskola személyre szabható tanulási környezetet biztosít. A
tanulásszervező tanárok, az iskola sok választási lehetőséget kínál a
gyerekek számára, akik a lehetőségekből összeválogatják a saját tanulási
céljaikhoz leginkább illeszkedő saját tanulási programjukat. A
választásban -- az önvezérelt tanulás menedzselésében -- a gyerekek
folyamatos egyéni támogatást kapnak a \emph{mentor} tanáruktól.

\section{A személyiségfejlesztés}
\begin{verbatim}
  Trv. elvarja: A személyiségfejlesztéssel kapcsolatos
  pedagógiai feladatok
  Itt irjuk le a 4x3 szintet
\end{verbatim}

A Budapest School gyerekek boldogak, egészségesek, hasznosak közösségüknek és
adaptívak, azaz képesek változni, tanulni és fejődni. Ennek megfeleően
foglalkoznunk kell az egészségfejlesztéssel (\aref{sec:egeszsegfejlesztes}.
fejezet)

\subsection{A teljeskörű
      egészségfejlesztés}\label{sec:egeszsegfejlesztes}

Az egészségfejlesztés a WHO meghatározása szerint az a folyamat, ami
képessé teszi az embereket arra, hogy saját egészségüket felügyeljék és
javítsák. Az egészségnevelés pedig változatos kommunikációs formákat
használó, tudatosan létrehozott tanulási lehetőségek összessége, amely
az egészséggel kapcsolatos tudást, ismereteket és életkészségeket bővíti
az egyén és a környezetében élők egészségének előmozdítása érdekében.
Budapest School Általános Iskola és Gimnázium (a továbbiakban e
fejezetben BPS) ezen definíciót teszi magáévá a teljes körű
egészségfejlesztési program összeállításakor és alkalmazásával.

A teljes körű egészségfejlesztési program a BPS közösség életminőségének
javítását szolgáló, a közösséghez tartozók közös akaratát összegző
cselekvési program, melynek közvetlen és közvetett célja az életminőség,
ezen keresztül az egészségi állapot javítása, olyan közösségi
problémakezelési módszer, amely az érintettek aktív részvételére épít.

A teljes körű egészségfejlesztés célja, hogy a BPS-ben eltöltött időben
minden gyerek részesüljön a teljes testi-lelki jóllétét, egészségét
megőrző és hatékonyan fejlesztő, a BPS mindennapjaiban rendszerszerűen
működő egészségfejlesztő tevékenységekben.

\subsubsection{Négy alapfeladat}\label{nuxe9gy-alapfeladat}

A teljes körű iskolai egészségfejlesztés az alábbi négy
egészségfejlesztési alapfeladat rendszeres végzését jelenti - minden
gyerekkel, a tanárok és a szülők, valamint a BPS partneri kapcsolati
hálóban szereplők bevonásával:

\begin{itemize}

      \item
            egészséges táplálkozás megvalósítása (elsősorban megfelelő, magas
            minőségű, lehetőleg helyi alapanyagokat használó kiszállítóval való
            megállapodással; helyben főzés esetén az alapanyagok kiválasztásánál
            legyen elsődleges szempont)
      \item
            mindennapi testmozgás minden gyereknek (változatos foglalkozásokkal,
            koncentráltan az egészség-javító elemekre, módszerekre, pl.
            tartásjavító torna, tánc, jóga)
      \item
            a gyermekek érett személyiséggé válásának elősegítése személyközpontú
            pedagógiai módszerekkel és a művészetek személyiségfejlesztő
            hatékonyságú alkalmazásával (ének, tánc, rajz, mesemondás, népi
            játékok, stb.)
      \item
            környezeti, médiatudatossági, fogyasztóvédelmi, balesetvédelmi
            egészségfejlesztési modulok, modulrészletek hatékony (azaz ``bensővé
            váló'') oktatása.
\end{itemize}

\subsubsection{Az egészségfejlesztési ismeretek
      témakörei}\label{az-eguxe9szsuxe9gfejlesztuxe9si-ismeretek-tuxe9makuxf6rei}

\begin{itemize}

      \item
            Az egészség fogalma
      \item
            Az egyén és az őt körülvevő közösség egészsége: felelősségünk
      \item
            A környezet egészsége
      \item
            Az egészséget befolyásoló tényezők
      \item
            A jó egészségi állapot megőrzése
      \item
            A betegség fogalma
      \item
            Megelőzés
      \item
            A táplálkozás és az egészség, betegség kapcsolata
      \item
            A testmozgás és az egészség, betegség kapcsolata
      \item
            Balesetek, baleset-megelőzés
      \item
            A lelki egészség.
      \item
            Önismeret, önértékelés, a másikat tiszteletben tartó kommunikáció
            módjai, ennek szerepe a másik önértékelésének segítésében
      \item
            A két agyfélteke harmonikus fejlődése
      \item
            Az érett, autonóm személyiség jellemzői
      \item
            A társas kapcsolatok
      \item
            A társadalom élete, a társadalmi együttélés normái
      \item
            A gyermek fejlődését elősegítő viszonyulás a gyermekhez - családban,
            iskolában
      \item
            A szenvedélybetegségek és megelőzésük (dohányzás, alkohol- és
            drogfogyasztás, játék-szenvedély, internet- és tv-függés)
      \item
            Művészeti és sporttevékenységek lelki egészséget, egészséges
            személyiségfejlődést és tanulási eredményességet elősegítő hatásai
      \item
            Médiatudatosság, a médiafogyasztás egészségvédő módja
      \item
            Az idő és az egészség, bioritmus, időbeosztás
      \item
            Tartós egészségkárosodással élő társakkal együttélés, a segítségre
            szorulók segítése
      \item
            Önmagunk és egészségi állapotunk ismerete
      \item
            A személyes krízishelyzetek felismerése és kezelési stratégiák
            ismerete
      \item
            Az idővel való gazdálkodás szerepe
      \item
            A rizikóvállalás és határai
      \item
            A tanulási környezet alakítása
      \item
            A természethez való viszony, az egészséges környezet jelentősége
\end{itemize}

\subsubsection{Indikátorok}\label{indikuxe1torok}

A teljes körű iskolai egészségfejlesztés az alábbi részterületeken
jelentkező hatások révén eredményezi a hatékonyság növekedését:

\begin{itemize}

      \item
            a tanulási eredményesség javítása
      \item
            a társadalmi befogadás és esélyegyenlőség elősegítése
      \item
            a társadalmi kapcsolatok javulása a kortársakkal, szülőkkel,
            tanárokkal
      \item
            az önismeret és önbizalom javulása
      \item
            az alkalmazkodókészség, a stresszkezelés, a problémamegoldás javulása
      \item
            érett, autonóm személyiség kialakulása
      \item
            a krónikus, nem fertőző megbetegedések (lelki betegségek,
            szív-érrendszeri, mozgásszervi és daganatos betegségek) elsődleges
            megelőzése
\end{itemize}

\subsubsection{A program végrehajtása - elsősorban a Harmónia (fizikai,
      lelki jóllét és kapcsolódás a környezethez) tantárgy
      keretében}

A Budapest School tanulási koncepciójának középpontjában az egyén, mint
a közösség jól funkcionáló, saját célokkal rendelkező tagja áll. Az
iskolában való fejlődése során elsősorban azt tanulja, hogy miként tud
specifikált saját célokat megfogalmazni, és hogyan tudja ezeket elérni.
Ebben a folyamatban egy mentor segíti a munkáját az iskola kezdetétől a
végéig. Ő figyel arra, hogy a gyerek fizikai és lelki biztonsága és
fejlődése folyamatos legyen, és segíti azokban a helyzetekben, amikor
biztonságérzete vagy stabilitása csökken.

A közösségben jól funkcionáló egyén belső harmóniájához ez a tantárgy a
következő fejlesztési területeket határozza meg:

\begin{itemize}

      \item
            Érzelmi és társas intelligencia
      \item
            Önismeret és önbizalom
      \item
            Konfliktuskezelés
      \item
            Rugalmasság (reziliencia)
      \item
            Kritikai gondolkodás
      \item
            Közösségi szabályok alkotásában való részvétel és azok alkalmazása
      \item
            Csapatmunka gyakorlati fejlesztése
      \item
            Oldott játék
      \item
            Egészséges testi fejlődés
      \item
            Saját igényekhez képest megfelelő táplálkozás
      \item
            A természettel való kapcsolódás
      \item
            Épített falusi és városi környezetben való eligazodás
      \item
            A technológia világában felhasználói szintű eligazodás és annak
            harmonikus alkalmazása
\end{itemize}

\paragraph{Közösségben,
      csapatban}

A Budapest School egy közösségi iskola, ahol a közösség tagjai egymással
és egymástól tanulnak. A közösségekhez való tartozáshoz, a csapatban
való gondolkodáshoz, és a családban való működéshez szükséges
képességeket leginkább úgy tudjuk fejleszteni, ha azt kezdetektől
megéljük. A közösség belső szabályainak megalkotása és az azokhoz való
kapcsolódás a tanulás folyamatosságának alapfeltétele.

\paragraph{Életképességek (life
      skills)}

Szeretnénk, ha gyerekeink általában alkalmazkodóan (adaptívan) és
pozitívan tudnának hozzáállni az élet kihívásaihoz, ha lelki és fizikai
erősségük és rugalmasságuk (rezilienciájuk) megmaradna és fejlődne. A
WHO a következőképpen definiálta (World Health Organization, 1999) az
életképességeket:

\begin{itemize}

      \item
            Döntéshozás, problémamegoldás
      \item
            Kreatív gondolkodás
      \item
            Kommunikáció és interperszonális képességek
      \item
            Önismeret, empátia
      \item
            Magabiztosság (asszertivitás) és higgadtság
      \item
            Terhelhetőség és érzelmek kezelése, stressztűrés
\end{itemize}

\paragraph{Érzelmi intelligencia}\label{uxe9rzelmi-intelligencia}

Sokszor kiemeljük az érzelmi intelligenciát, kihangsúlyozva, hogy
gyerekeinknek többet kell foglalkozniuk az érzelmek felismerésével,
kontrollálásával és kifejezésével, mint szüleinknek kellett.

\paragraph{Szabad mozgás és séta}\label{szabad-mozguxe1s-uxe9s-suxe9ta}

A különböző mozgásformák, sportok és a séta mindennapivá tétele
természetes módon, a gyerekek saját igényei szerint kell hogy történjen.

\paragraph{Gyakorlatias, mindennapi
      képességek}\label{gyakorlatias-mindennapi-kuxe9pessuxe9gek}

Ahhoz, hogy gyerekeink önállóan és hatékonyan tudják élni életüket, hogy
a társakhoz való kapcsolódás ne függőség legyen, egy csomó praktikus
mindennapi tudást el kell sajátítaniuk. A gyerekeknek folyamatosan
fejleszteniük kell az élethez szükséges minden- napi tudást a
levélszemét kezeléstől, a facebook profil tudatos használatán át,
egészen a személyi költségvetés készítéséig.

\paragraph{Egészséges
      táplálkozás}

Az egészséges táplálkozás tanulható viselkedésforma, melynek alapja nem
csupán a megfelelő élelmiszerek kiválasztása, hanem azok élettani
hatásainak megismerése, és az étkezési szokások alakítása is.

\subsection{A
      közösségfejlesztés}

A közösségfejlesztéssel, az iskola szereplőinek együttműködésével
kapcsolatos feladatok

\begin{quote}
      csak a nevelési-oktatási tartalmak relevanciája vonatkozásában, mert
      egyébként SZMSZ-kompetencia)
\end{quote}

\subsection{Elsősegély-
      nyújtás}\label{elsux151seguxe9ly--nyuxfajtuxe1s}

az elsősegély- nyújtási alapismeretek elsajátításával kapcsolatos
iskolai terv

\subsection{Nemzetiségek
      megismerése}\label{nemzetisuxe9gek-megismeruxe9se}

\begin{quote}
      a nemzetiséghez nem tartozó tanulók részére a településen élő nemzetiség
      kultúrájának megismerését szolgáló tananyag,
\end{quote}

\subsection{az egészségnevelési és környezeti nevelési elvek}

\begin{quote}
      az egészségnevelési és környezeti nevelési elvek
\end{quote}
\todo[inline]{Nagyon hianyzik a az egészségnevelési és környezeti nevelési
      elvek}
\section{Esélyegyenlőség}\label{esuxe9lyegyenlux151suxe9g}

\begin{quote}
      a gyermekek, tanulók esélyegyenlőségét szolgáló intézkedések
\end{quote}

\section{Tantárgyak}
\label{sec:tantargyak}
A Budapest School a ma gyerekeinek kínál olyan oktatást, ami segíti felkészíteni őket a jövő kihívásaira. Információs társadalmunk legnagyobb kihívása az adaptációs képességünk fejlesztése, ez az alapja annak, hogy képesek legyünk eligazodni a folyamatosan változó, komplex világunkban. A tanulásunk célja, hogy boldog, hasznos és egészséges tagjai legyünk a társadalomnak. Iskolánkban a tanulás három rétege, a tudásszerzés, a megtanultakat elmélyítő önálló gondolkodás és az aktív alkotás egyszerre jelennek meg.

A kerettanterv a célok eléréséhez a miniszter által kiadott kerettantervek tantárgyi struktúráját használja  a moduláris tanulás tartalmi keretezéséhez. A keretezésen azt értjük, hogy a tantárgyak tartalma határozza meg, hogy mivel kell mindenképp a Budapest School iskolákban foglalkozni, mit kell mindenképp megtanulni. 
Az egyes modulok ezen tantárgyak tanulási eredményeinek elérését támogatják.


\subsection{Tantárgyi definiciók és a tanulási eredmények}
A tantárgyak tanulási eredmények felsorolásával adnak tartalmi szabályozást. A tanulási eredmények (learning outcomes) tudás, képesség, kompetencia kontextusában meghatározott kijelentések arra vonatkozóan, hogy a tanulónak mit kell tudnia, mit kell értenie, és mire legyen képes, miután lezárt egy tanulási folyamatot, függetlenül attól, hogy hol, hogyan, mikor szerezte meg ezeket a kompetenciákat \citep{learning_outcomes}.  Vagyis  az egyes modulok különféle tanulási eredmények elérését is támogathatják, ezzel több tantárgy részcéljait is teljesíthetik.

A kerettanterv tantárgyankénti és félévenkénti bontásban adja meg a továbbhaladáshoz elengedhetetlen tanulási eredmények listáját.

A tantárgyi definiciókhoz a miniszter által kiadott kerettanterv ,,elvárt eredmények a tanulási ciklus végén" fejezetek felsorolásait alakítottuk át egységes nyelvezetre, hogy azok valóban kompetenciákat írjanak.

A tantárgyi specifikációk nem térnek ki rész\-letesen a tematikákra. Ez szabadságot ad a tanároknak arra, hogy a tanmenet tekintetében akár jelentős eltérések legyenek addig, amig a miniszter által kiadott kerettanterv mérhető tanulási eredményei teljesülnek. A tanulási eredmény alapú szabályozás folyamatos visszacsatolást tud adni a tanulónak és a tanároknak, megmutatva, melyik tanulási eredményeket kell még elérni a következő szintre való lépéshez.

\paragraph{Tantárgyak szerepe a mindennapokban}

A Budapest School iskoláinak tantárgyi leírásai a miniszter által kiadott kerettanterv alapján készültek. Az egyes tanulási moduloknak a portfólióba való elhelyezését követően háromhavonta összevetjük az elért tanulási eredményeket és  a tantárgyi kötelezően választható tanulási eredményeket, hogy ezáltal folyamatosan monitorozni lehessen az iskolai követelmények és a gyerekek egyéni eredményei közötti egyensúlyt.

A Budapest School iskoláiban a tantárgyak ugyanúgy kapnak szerepet, mint a NAT által definiált kulcskompetenciák, fejlesztési területek: tanár sose mondja azt a gyerekeknek, hogy „most kezdeményezőképességet és vállalkozói kompetenciát fejlesztünk'', hanem a különböző feladatok elvégzése eredményeképp történik a fejlesztés. A Budapest School iskolákban a tantárgyközi tevékenységek vannak előtérben. A tantárgyak a tanulás tartalmi elemeinek forrása és keretei: a tanulandó dolgok listájaként működik. Az, hogy milyen csoportosításban történik a tanulás, az a modulvezetőkre van bízva.

A tantárgyak ezért elsősorban a modulok kiírásakor és azok kimeneti értékelésekor jelennek meg, a mindennapok struktúráját, a napi- és hetirendet azonban a modulok adják. Egyes modulok több tantárgy fejlesztési céljainak is eleget tehetnek, több tantárgy tanulási eredményének elérését is célul tűzhetik ki, összhangban a NAT-tal. A tantárgyaknak ezzel együtt fontos célja, hogy segítse a tanulás tartalmi egyensúlyának fennmaradását. A tanulásszervezők, modulvezetők szakképesítése nem köthető a Budapest School tantárgyaihoz, felelősségük, hogy a saját moduljukban megfelelően tudják szervezni a tanulást, és legfőképp, hogy saját moduljuk megtartására alkalmasak legyenek.

\paragraph{Heti óraszámok}

A Budapest School közösségi tanulási élményeket és modulokat szervez a
tanulóinak, egyúttal lehetőséget ad arra, hogy a gyerekek a közösen kialakított
szabályaik mentén tanulásszervezők felügyeletével a Budapest School székhelyén
vagy egyes telephelyein, vagy más, erre alkalmas tanulási környezetben
tartózkodjanak. A közösségben együtt töltött idő tanulásnak, fejlődésnek
minősül akkor is, ha az nem egy modulhoz kapcsolódik, hanem az ebéd
élvezetéhez, vagy épp a parkban a lehulló falevelek neszének megfigyeléséhez.

A gyerekek, a tanítási szüneteket leszámítva, naponta 8 órát tartózkodnak az
iskolában. Ezekben az időkben vannak a tanítási órák, foglalkozások, szakkörök,
műhelyek. Az egyes mikroiskolák ettől 20\%-ban bármelyik irányban eltérhetnek,
ha ez segíti a tanulásszervezők munkáját és a gyerekek fejlődését. Így hetente
minimum $5 \cdot 8 \cdot 0.8 = 32$ órát, maximum 48 órát töltenek az iskolában.

Ennek alsó tagozatban körülbelül a felét, a felső tagozatban kétharmad részét
töltik előre eltervezett módon, azaz modulokkal. A többi időben a tanárok
vezetése és felügyelete mellett szabadon alkotnak, játszanak, pihennek,
közösségi életet élnek. Azaz alsó tagozatban 16--24, míg felső tagozatban
21--32 órát töltenek modulokkal.

Mivel az elvárt kiegyensúlyozottság miatt mind a három tantárgyra körülbelül
ugyanannyi energiát kell fektetni, így az egyes tantárgyakra a teljes
rendelkezésre álló időkeret egyharmad részét kell számolni. Ettől az iskolák
$\pm$ 20\%-ban eltérhetnek, így kiszámolható, hogy minimum mennyi időt kell
egy-egy gyereknek egy héten egy tantárggyal foglalkoznia. Ezt összegzi
\aref{tbl:oraszamok}. táblázat.

\begin{table}

  \begin{tabular}{ l|l|l }

    \textbf{Tantárgy} & \textbf{Alsó
    tagozat}          & \textbf{Felső tagozat}                                                      \\ \hline
    Harmónia          & $\frac{5 \times 8 \times 0.8}{2} \times \frac{1}{3} \times 0.8 =
    4.27$ óra         &
    $\frac{5 \times 8 \times 0.8 \times 2}{3} \times \frac{1}{3} \times 0.8 = 5.69$
    óra                                                                                             \\ \hline
    STEM              & 4.27 óra                                                         & 5.69 óra \\ \hline
    KULT              & 4.27 óra                                                         & 5.69 óra \\ \hline

  \end{tabular}
  \caption{Az elvárt kiegyensúlyozottság miatt a tantárgyakkal egyenlő
    minimális óraszámban kell foglalkozni.}
  \label{tbl:oraszamok}
\end{table}

Fontos, hogy \emph{egy-egy modul több tantárgy fejlesztési céljaihoz és
  eredménycéljaihoz is kapcsolódhat.}

\section{NAT céljainak támogatása}
\label{sec:nat_celjai}
A Nemzeti Alaptantervben szereplő fejlesztési célok elérését és a
kulcskompetenciák fejlődését több minden támogatja:

Egyrészt a tantárgyak lefedik a NAT fejlesztési céljait, kulcskompetenciáit és
műveltségi területeit, mert a jelenleg érvényben lévő, a miniszter által a
\emph{51/2012. (XII. 21.) számú EMMI rendelet I-IV. mellékletében} kiadott
kerettantervek \citep{ofi:kerettanterv} tanulási, tanítási eredményeiből
indultunk ki. Mivel a rendeletben szereplő kerettantervek teljesítik a NAT
feltételeit, így a Budapest School tantárgystruktúrája is teljesíti ezeket.

Másrészt az iskola életében, folyamatában való részvétel, már önmagában
biztosítja a kulcskompentenciák fejlődését és a NAT fejlesztési céljainak
teljesülését sok esetben.

A \ref{tbl:nat_fejlesztesi} táblázat bemutatja a NAT fejlesztési területeihez
való kapcsolódást, a
\ref{tbl:nat_kulcs} táblázat pedig az illeszkedési pontokat a NAT
kulcskompetenciáihoz.

\begin{table}

  \begin{tabular}{p{5cm}|>{\raggedright}p{3cm}|p{3cm}}

    \textbf{NAT Fejlesztési célok}               & \textbf{Tantárgyak}  & \textbf{Struktúra}           \\
    \hline
    Az erkölcsi nevelés                          & kult, harmónia       & közösség                     \\ \hline
    Nemzeti öntudat, hazafias nevelés            & kult, harmónia       & projektek                    \\ \hline
    Állampolgárságra, demokráciára nevelés       & kult, harmónia       & közösség                     \\ \hline
    Az önismeret és a társas kultúra fejlesztése & kult, harmónia, stem & saját
    tanulási út, közösség                                                                              \\ \hline
    A családi életre nevelés                     & harmónia             &                              \\ \hline
    A testi és lelki egészségre nevelés          & harmónia             & közösség                     \\ \hline
    Felelősségvállalás másokért, önkéntesség     & harmónia             & közösség, pro\-jek\-tek      \\
    \hline
    Fenntarthatóság, környezettudatosság         & harmónia, stem       & projektek                    \\ \hline
    Pályaorientáció                              & kult, harmónia, stem & saját tanulási út            \\ \hline
    Gazdasági és pénzügyi nevelés                & kult, harmónia, stem & projektek                    \\ \hline
    Médiatudatosságra nevelés                    & kult                 & projektek                    \\ \hline
    A tanulás tanítása                           & kult, harmónia, stem & saját tanulási út, mentorság \\

  \end{tabular}
  \caption{A NAT fejlesztési céljainak elérését nemcsak a tantárgyak, hanem az
    iskola struktúrája is támogatja.}
  \label{tbl:nat_fejlesztesi}
\end{table}

A \emph{saját tanulási} út fogalma például önmagában segíti a tanulás
tanulását, hiszen az a gyerek, aki képes önmagának saját célt állítani (mentor
segítséggel), azt elérni, és a folyamatra való reflektálás során képességeit
javítani, az fejleszti a tanulási képességét.

Vagy másik példaként, a Budapest School iskoláiban a \emph{közösség} maga hozza
a működéséhez szükséges szabályokat, folyamatosan alakítja és fejleszti saját
működését a tagok aktív részvételével. Ez az aktív állampolgárságra, a
demokráciára való nevelés Nemzeti Alaptantervben előírt céljait is támogatja.

\begin{table}
  \centering
  \begin{tabular}{p{5cm}|>{\raggedright}p{3cm}|p{3cm}}

    \textbf{NAT kulcskompetenciái}                     & \textbf{Tantárgyak}  &
    \textbf{Struktúra}                                                                                        \\ \hline
    Anyanyelvi kommunikáció                            & kult                 & tanulási szerződés, portfólió \\ \hline
    Idegen nyelvi kommunikáció                         & kult                 & idegennyelvű modulok          \\ \hline
    Matematikai kompetencia                            & stem                 &                               \\ \hline
    Természettudományos és technikai kompetencia       & stem                 & projektek                     \\ \hline
    Digitális kompetencia                              & harmónia, stem       & digitális portfólió kezelés   \\ \hline
    Szociális és állampolgári kompetencia              & harmónia             & saját tanulási út,
    közösség                                                                                                  \\ \hline
    Kezdeményezőképesség és vállalkozói kompetencia    & kult, harmónia, stem & saját
    tanulási út, közösség                                                                                     \\ \hline
    Esztétikai-művészeti tudatosság és kifejezőkészség & harmónia, kult       &                               \\ \hline
    A hatékony, önálló tanulás                         & kult, harmónia, stem & saját tanulási út,
    mentorság                                                                                                 \\

  \end{tabular}
  \caption{NAT kulcskompetenciáinak fejlesztését támogatják a tantárgyak és
    az iskola felépítése is.}
  \label{tbl:nat_kulcs}
\end{table}

\section{Tankönyvek kiválasztása}

\begin{quote}
      Az oktatásban alkalmazható tankönyvek, tanulmányi segédletek és
      taneszközök kiválasztásának elvei (figyelembe véve a tankönyv
      térítésmentes igénybevétele biztosításának kötelezettségét).
\end{quote}

Modulvezetők minden esetben maguk választják a modulhoz szükséges
tankönyvek, szoftverek, weboldalak és egyéb eszközöket úgy, hogy

\begin{itemize}

      \item
            az a megfelelő legyen annak a csoportnak, ahhoz a célhoz, amit el akar
            érni
      \item
            minden esetben legyen mindenki számára elérhető (esetek többségeben
            értsd ingyenes) megoldás
      \item
            modulvezetők bátorítva vannak arra, hogy új dolgokat próbáljanak ki,
            és tapasztalataikat az iskola többi tanárával megosszák.
\end{itemize}

Mivel a Budapest School kerettantervének értelmében az egyéni célok
legalább 50\%-át az állami kerettantervben meghatározott fejlesztési
célok közül kell választani, az ehhez szükséges ismeretek megszerzéséhez
a Budapest School az Oktatási Hivatal általi jegyzékben államilag
támogatott OFI által fejlesztett tankönyveket veszi alapul. A Budapest
School tanárcsapatának lehetősége van arra, hogy ettől eltérő, a
mindenkori tankönyvjegyzékben szereplő tankönyvvel segítse el a
kerettantervben meghatározott eredménycélok elérését. És arra is
lehetősége van, hogy egyátalán ne használjon tankönyvet, mert sokszor az
internet elegendő információt tartalmaz.

A Budapest School pedagógiai programjának alapja, hogy a gyerekek egyéni
céljaira szabott tanulási terveket készít. Ennek előfeltétele, hogy a
könyvek használata is ehhez kapcsolódó módon rugalmasan történjen,
minden esetben az adott tanulási modul igényeihez szabva. Ennek
érdekében a program pedagógusai folyamatosam állítják össze a gyerekek
eltérő céljaihoz és képességszintjeihez igazodó differenciált
tevékenységek és feladatsorok rendszerét.

\section{Érettségire készülés }
Válaszható érettségi árgyak
\begin{quote}
      középiskola esetén azon választható érettségi vizsgatárgyak megnevezése,
      amelyekből a középiskola tanulóinak közép- vagy emelt szintű érettségi
      vizsgára való felkészítését az iskola kötelezően vállalja, továbbá annak
      meghatározáse, hogy a tanulók milyen helyi tantervi követelmények
      teljesítése mellett melyik választható érettségi vizsgatárgyból tehetnek
      érettségi vizsgát,
\end{quote}

Az iskola a kötelező középszintű érettségi vizsgatárgyakra való
felkeszítést kötelezően vállalja érettségi felkészítő modulok
szervezésével. (Más iskolákban ezt fakultációnak hívnák, de a Budapest
School iskola nem különbözteti meg a fakultációt a tanórától.) A
választható tantárgyak és az emeltszintű érettségi vizsgára csak akkor
szervez egy mikroiskola modult, ha arra legalább a közösség 10\%-a
igényt tart.

\subsubsection{középszintű érettségi vizsga témakörei}

\begin{quote}
      középiskola esetén az egyes érettségi vizsgatárgyakból a középszintű
      érettségi vizsga témakörei
\end{quote}

\texttt{EZ\ HP\ SZERINT\ LEHET\ HOGY\ NEM\ KELL}



\section{Önkéntes közösségi
  szolgálat}\label{uxf6nkuxe9ntes-kuxf6zuxf6ssuxe9gi-szolguxe1lat}

Az Nkt 4. § (15) pontjában definiált közösségi szolgálat is modulként
kerül meghírdetésre, amit 12. évét betöltött gyerek választhat csak.
Közösségi szolgálatként elfogadható, ha a Budapest School iskola egy
másik mikroiskolájában segít a gyerek.
