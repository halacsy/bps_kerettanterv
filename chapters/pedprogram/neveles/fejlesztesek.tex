\chapter{A gyerekek fejlesztése}
\section{A
személyiségfejlesztés}
\begin{verbatim}
  Trv. elvarja: A személyiségfejlesztéssel kapcsolatos
  pedagógiai feladatok
  Itt irjuk le a 4x3 szintet
\end{verbatim}


\subsection{A teljeskörű
egészségfejlesztés}\label{a-teljeskuxf6rux171-eguxe9szsuxe9gfejlesztuxe9s}

\subsubsection{Definíció, cél}\label{definuxedciuxf3-cuxe9l}

Az egészségfejlesztés a WHO meghatározása szerint az a folyamat, ami
képessé teszi az embereket arra, hogy saját egészségüket felügyeljék és
javítsák. Az egészségnevelés pedig változatos kommunikációs formákat
használó, tudatosan létrehozott tanulási lehetőségek összessége, amely
az egészséggel kapcsolatos tudást, ismereteket és életkészségeket bővíti
az egyén és a környezetében élők egészségének előmozdítása érdekében.
Budapest School Általános Iskola és Gimnázium (a továbbiakban e
fejezetben BPS) ezen definíciót teszi magáévá a teljes körű
egészségfejlesztési program összeállításakor és alkalmazásával.

A teljes körű egészségfejlesztési program a BPS közösség életminőségének
javítását szolgáló, a közösséghez tartozók közös akaratát összegző
cselekvési program, melynek közvetlen és közvetett célja az életminőség,
ezen keresztül az egészségi állapot javítása, olyan közösségi
problémakezelési módszer, amely az érintettek aktív részvételére épít.

A teljes körű egészségfejlesztés célja, hogy a BPS-ben eltöltött időben
minden gyerek részesüljön a teljes testi-lelki jóllétét, egészségét
megőrző és hatékonyan fejlesztő, a BPS mindennapjaiban rendszerszerűen
működő egészségfejlesztő tevékenységekben.

\subsubsection{Négy alapfeladat}\label{nuxe9gy-alapfeladat}

A teljes körű iskolai egészségfejlesztés az alábbi négy
egészségfejlesztési alapfeladat rendszeres végzését jelenti - minden
gyerekkel, a tanárok és a szülők, valamint a BPS partneri kapcsolati
hálóban szereplők bevonásával:

\begin{itemize}

\item
  egészséges táplálkozás megvalósítása (elsősorban megfelelő, magas
  minőségű, lehetőleg helyi alapanyagokat használó kiszállítóval való
  megállapodással; helyben főzés esetén az alapanyagok kiválasztásánál
  legyen elsődleges szempont)
\item
  mindennapi testmozgás minden gyereknek (változatos foglalkozásokkal,
  koncentráltan az egészség-javító elemekre, módszerekre, pl.
  tartásjavító torna, tánc, jóga)
\item
  a gyermekek érett személyiséggé válásának elősegítése személyközpontú
  pedagógiai módszerekkel és a művészetek személyiségfejlesztő
  hatékonyságú alkalmazásával (ének, tánc, rajz, mesemondás, népi
  játékok, stb.)
\item
  környezeti, médiatudatossági, fogyasztóvédelmi, balesetvédelmi
  egészségfejlesztési modulok, modulrészletek hatékony (azaz ``bensővé
  váló'') oktatása.
\end{itemize}

\subsubsection{Az egészségfejlesztési ismeretek
témakörei}\label{az-eguxe9szsuxe9gfejlesztuxe9si-ismeretek-tuxe9makuxf6rei}

\begin{itemize}

\item
  Az egészség fogalma
\item
  Az egyén és az őt körülvevő közösség egészsége: felelősségünk
\item
  A környezet egészsége
\item
  Az egészséget befolyásoló tényezők
\item
  A jó egészségi állapot megőrzése
\item
  A betegség fogalma
\item
  Megelőzés
\item
  A táplálkozás és az egészség, betegség kapcsolata
\item
  A testmozgás és az egészség, betegség kapcsolata
\item
  Balesetek, baleset-megelőzés
\item
  A lelki egészség.
\item
  Önismeret, önértékelés, a másikat tiszteletben tartó kommunikáció
  módjai, ennek szerepe a másik önértékelésének segítésében
\item
  A két agyfélteke harmonikus fejlődése
\item
  Az érett, autonóm személyiség jellemzői
\item
  A társas kapcsolatok
\item
  A társadalom élete, a társadalmi együttélés normái
\item
  A gyermek fejlődését elősegítő viszonyulás a gyermekhez - családban,
  iskolában
\item
  A szenvedélybetegségek és megelőzésük (dohányzás, alkohol- és
  drogfogyasztás, játék-szenvedély, internet- és tv-függés)
\item
  Művészeti és sporttevékenységek lelki egészséget, egészséges
  személyiségfejlődést és tanulási eredményességet elősegítő hatásai
\item
  Médiatudatosság, a médiafogyasztás egészségvédő módja
\item
  Az idő és az egészség, bioritmus, időbeosztás
\item
  Tartós egészségkárosodással élő társakkal együttélés, a segítségre
  szorulók segítése
\item
  Önmagunk és egészségi állapotunk ismerete
\item
  A személyes krízishelyzetek felismerése és kezelési stratégiák
  ismerete
\item
  Az idővel való gazdálkodás szerepe
\item
  A rizikóvállalás és határai
\item
  A tanulási környezet alakítása
\item
  A természethez való viszony, az egészséges környezet jelentősége
\end{itemize}

\subsubsection{Indikátorok}\label{indikuxe1torok}

A teljes körű iskolai egészségfejlesztés az alábbi részterületeken
jelentkező hatások révén eredményezi a hatékonyság növekedését:

\begin{itemize}

\item
  a tanulási eredményesség javítása
\item
  a társadalmi befogadás és esélyegyenlőség elősegítése
\item
  a társadalmi kapcsolatok javulása a kortársakkal, szülőkkel,
  tanárokkal
\item
  az önismeret és önbizalom javulása
\item
  az alkalmazkodókészség, a stresszkezelés, a problémamegoldás javulása
\item
  érett, autonóm személyiség kialakulása
\item
  a krónikus, nem fertőző megbetegedések (lelki betegségek,
  szív-érrendszeri, mozgásszervi és daganatos betegségek) elsődleges
  megelőzése
\end{itemize}

\subsubsection{A program végrehajtása - elsősorban a Harmónia (fizikai,
lelki jóllét és kapcsolódás a környezethez) tantárgy
keretében}\label{a-program-vuxe9grehajtuxe1sa---elsux151sorban-a-harmuxf3nia-fizikai-lelki-juxf3lluxe9t-uxe9s-kapcsoluxf3duxe1s-a-kuxf6rnyezethez-tantuxe1rgy-keretuxe9ben}

A Budapest School tanulási koncepciójának középpontjában az egyén, mint
a közösség jól funkcionáló, saját célokkal rendelkező tagja áll. Az
iskolában való fejlődése során elsősorban azt tanulja, hogy miként tud
specifikált saját célokat megfogalmazni, és hogyan tudja ezeket elérni.
Ebben a folyamatban egy mentor segíti a munkáját az iskola kezdetétől a
végéig. Ő figyel arra, hogy a gyerek fizikai és lelki biztonsága és
fejlődése folyamatos legyen, és segíti azokban a helyzetekben, amikor
biztonságérzete vagy stabilitása csökken.

A közösségben jól funkcionáló egyén belső harmóniájához ez a tantárgy a
következő fejlesztési területeket határozza meg:

\begin{itemize}

\item
  Érzelmi és társas intelligencia
\item
  Önismeret és önbizalom
\item
  Konfliktuskezelés
\item
  Rugalmasság (reziliencia)
\item
  Kritikai gondolkodás
\item
  Közösségi szabályok alkotásában való részvétel és azok alkalmazása
\item
  Csapatmunka gyakorlati fejlesztése
\item
  Oldott játék
\item
  Egészséges testi fejlődés
\item
  Saját igényekhez képest megfelelő táplálkozás
\item
  A természettel való kapcsolódás
\item
  Épített falusi és városi környezetben való eligazodás
\item
  A technológia világában felhasználói szintű eligazodás és annak
  harmonikus alkalmazása
\end{itemize}

\paragraph{Közösségben,
csapatban}\label{kuxf6zuxf6ssuxe9gben-csapatban}

A Budapest School egy közösségi iskola, ahol a közösség tagjai egymással
és egymástól tanulnak. A közösségekhez való tartozáshoz, a csapatban
való gondolkodáshoz, és a családban való működéshez szükséges
képességeket leginkább úgy tudjuk fejleszteni, ha azt kezdetektől
megéljük. A közösség belső szabályainak megalkotása és az azokhoz való
kapcsolódás a tanulás folyamatosságának alapfeltétele.

\paragraph{Életképességek (life
skills)}\label{uxe9letkuxe9pessuxe9gek-life-skills}

Szeretnénk, ha gyerekeink általában alkalmazkodóan (adaptívan) és
pozitívan tudnának hozzáállni az élet kihívásaihoz, ha lelki és fizikai
erősségük és rugalmasságuk (rezilienciájuk) megmaradna és fejlődne. A
WHO a következőképpen definiálta (World Health Organization, 1999) az
életképességeket:

\begin{itemize}

\item
  Döntéshozás, problémamegoldás
\item
  Kreatív gondolkodás
\item
  Kommunikáció és interperszonális képességek
\item
  Önismeret, empátia
\item
  Magabiztosság (asszertivitás) és higgadtság
\item
  Terhelhetőség és érzelmek kezelése, stressztűrés
\end{itemize}

\paragraph{Érzelmi intelligencia}\label{uxe9rzelmi-intelligencia}

Sokszor kiemeljük az érzelmi intelligenciát, kihangsúlyozva, hogy
gyerekeinknek többet kell foglalkozniuk az érzelmek felismerésével,
kontrollálásával és kifejezésével, mint szüleinknek kellett.

\paragraph{Szabad mozgás és séta}\label{szabad-mozguxe1s-uxe9s-suxe9ta}

A különböző mozgásformák, sportok és a séta mindennapivá tétele
természetes módon, a gyerekek saját igényei szerint kell hogy történjen.

\paragraph{Gyakorlatias, mindennapi
képességek}\label{gyakorlatias-mindennapi-kuxe9pessuxe9gek}

Ahhoz, hogy gyerekeink önállóan és hatékonyan tudják élni életüket, hogy
a társakhoz való kapcsolódás ne függőség legyen, egy csomó praktikus
mindennapi tudást el kell sajátítaniuk. A gyerekeknek folyamatosan
fejleszteniük kell az élethez szükséges minden- napi tudást a
levélszemét kezeléstől, a facebook profil tudatos használatán át,
egészen a személyi költségvetés készítéséig.

\paragraph{Egészséges
táplálkozás}\label{eguxe9szsuxe9ges-tuxe1pluxe1lkozuxe1s}

Az egészséges táplálkozás tanulható viselkedésforma, melynek alapja nem
csupán a megfelelő élelmiszerek kiválasztása, hanem azok élettani
hatásainak megismerése, és az étkezési szokások alakítása is.

\subsection{A
közösségfejlesztés}\label{a-kuxf6zuxf6ssuxe9gfejlesztuxe9s}

A közösségfejlesztéssel, az iskola szereplőinek együttműködésével
kapcsolatos feladatok

\begin{quote}
csak a nevelési-oktatási tartalmak relevanciája vonatkozásában, mert
egyébként SZMSZ-kompetencia)
\end{quote}

\subsection{Elsősegély-
nyújtás}\label{elsux151seguxe9ly--nyuxfajtuxe1s}

az elsősegély- nyújtási alapismeretek elsajátításával kapcsolatos
iskolai terv
