\section{Pedagógiai és pszichológiai háttér}

Az iskolák gyakorlatias tapasztalata mellett a Budapest School számos elméletet is kiemelten fontosnak tart. 

Ezek közül is oktatási programjának központjában Carol Dweck fejlődésközpontú szemlélete \citep{growthmindset} áll. Emellett nagy hangsúlyt fektetünk az alábbi elméletek gyakorlati alkalmazására is:

Reformpedagógiai irányzatok elméletei, különös tekintettel:

\begin{itemize}

      \item
            Montessori-pedagógia (Maria Montessori),
      \item
            kritikai pedagógia (Paulo Freire)
      \item
            élménypedagógia (John Dewey)
      \item
            felfedeztető tanulás (Jerome Bruner)
      \item
            projektmódszer (William Kilpatrick)
      \item
            kooperatív tanulás (Spencer Kagan)
\end{itemize}

Pszichológia és szociálpszichológiai kutatások eredményei:

\begin{itemize}

      \item
            kognitív interakcionista tanuláselmélet (Jean Piaget)
      \item
            személyközpontú pszichológia (Carl Rogers)
      \item
            kommunikáció és konfliktuskezelés (Thomas Gordon)
      \item
            erőszakmentes kommunikáció (Marshall Rosenberg)
      \item
            pozitív pszichológia eredményei, különös tekintettel: flow-elmélet, kreativitáskutatások (Csík\-szent\-mihályi Mihály)
      \item
            érzelmi és társas intelligencia (Peter Salovey, John D. Mayer, Daniel Goleman)
      \item motivációkutatások, amiket jól összefoglal Daniel H. Pink
        műve\linebreak
        \citep{pink2011drive}
      \item
            hősiesség pszichológiai alapjai (Phil Zimbardo)
      \item
            fejlődésfókuszú szemlélet (Carol Dweck)
\end{itemize}
