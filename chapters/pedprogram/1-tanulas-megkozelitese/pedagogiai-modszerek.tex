\section{Pedagógiai módszerek}
\label{sec:pedagogia_modszerek}

A Budapest School iskola tanárainak feladata, hogy mindig keressék azt a módszert, azt a környezetet, ami az adott gyerekekkel, adott környezetben, adott időben a leginkább működik. Nem tudjuk megmondani előre, hogy mikor milyen módszert érdemes választani, de azt tudjuk, hogy mi alapján keressük a megfelelő technikákat. Vannak olyan módszerek, amelyek a saját célok lehetőségeinek kitágítását és azok elérését nagyban támogatják. Ezek alkalmazása javasolt a csoportmunkák és az egyéni gyakorlások alatt.

Azt is tudjuk, hogy nem baj, ha nem elsőre találtuk meg a megfelelő módszert, mert a próbálkozások során rengeteg új információt nyerünk, amelyek segítségével már könnyebb megtalálni a valóban megfelelő megoldást. Az alábbiakban néhány a Budapest School számára meghatározó fontosságú módszert emelünk ki.

\subsection{Projektmódszer}
Projektmódszert alkalmazó modulok során fő célunk, hogy a gyerekek aktívak és kreatívak legyenek, és ezért a tevékenységek sokszínűségét helyezzük fókuszba. Projektmódszer esetén is bátorítva van minden tanár, hogy a legváltozatosabb módszertárral közelítse a gyerekeket, figyelve arra, hogy a tanár attitűdje, a csoport dinamikája és az aktuális tevékenységek mit kívánnak.

\subsubsection{A projektmunka folyamata}

\begin{description}
      \item[Téma, cél] Első lépés, hogy meghatározzuk a projekt témáját vagy
            célját. Ez jöhet a tanártól, a gyerekektől, vagy akár egy szülőtől is. Fontos, hogy a gyerekek a projekt témáját vagy célját már önmagában értelmesnek, relevánsnak tartsák.

      \item [Ötletroham]  Egy-egy téma feldolgozását csoportalakítással és
            ötletrohammal kezdjük. Ennek célja, hogy a résztvevők bevonódjanak, illetve megmutassák, hogy nekik milyen elképzeléseik vannak az adott témáról, továbbá milyen produktummal, eredménnyel szeretnék zárni a folyamatot. A létrehozott produktumoknak csak a képzelet szabhat határt. Lehetnek videók, prezentációk, fotók, rapdalok, telefonos applikációk, rajzok, tablók, tudományos cikkek stb. 

      \item [Kutatói kérdés] Ezek után úgynevezett kutatói kérdéseket teszünk
            fel, melyek meghatározzák a vizsgálat irányát. A kérdések feldolgozása a legváltozatosabb módon történhet. Az egyéni munkától kezdve a frontális instruáláson vagy a kooperatív csoportmunkán keresztül egészen a dráma- és zenefoglalkozásokig minden hasznosítható a tanár, a csoport és a téma igényeihez mérten.
      \item [Elmélyült csoportmunka] A projekt azon szakasza, amikor a tervek,
            kutatások alapján az implementáción dolgozik a csapat.
      \item [Prezentáció] A létrehozott produktumok bemutatására külön
            hangsúlyt kell fektetni. Ennek több módja is lehet: prezentációk, demonstrációk, plakátok, projektfesztiválok.
\end{description}

Az iskolai projektek célja egy fejlődésfókuszú tanár és gyerek számára mindig kettős: egyrészt cél a téma feldolgozása, a produktum létrehozása, másrészt az iskola fő célja, hogy a gyerekek, a csapatok mindig fejlesszék alkotó, együttműködő, problémamegoldó képességüket. Ezért a projekt folyamatára való reflektálás, visszajelzés ugyanolyan fontos, mint maga a cél elérése.

A munka során külön figyelmet kell fordítani arra, hogy mindent dokumentáljanak a résztvevők. Lehetőleg online felületen.

\paragraph{Értékelés} A projekt során több értékelési pontot érdemes beépíteni.
A foglalkozások végén a résztvevők visszajeleznek a folyamatra, értékelik a saját, a csoport és a tanár munkáját. A folyamat végén az egész projektfolyamatot értékelik, szintén kitérve a saját, a csoport és a tanár munkájára. A produktumok, az eredmény értékelése csoport- és egyéni szinten is megtörténik.

Az értékelés fókuszáljon a folyamatra, és ne (csak)  az eredményre, hogy fejlessze a fejlődésfókuszú gondolkodásmódot \citep{growthmindset}.

\subsection{Önszerveződő tanulási környezet}
A Sugata Mitra által kialakított módszertan (angolul Self Organizing Learning Environment) lényege, hogy a tanárok arra bátorítják a gyerekeket, hogy csoportban, az internet segítségével \emph{Nagy Kérdéseket} válaszoljanak meg. A jó kérdés az, amire nem egyszerű a válasz, sőt lehet, hogy nincs is rá egyfajta válasz. Cél, hogy a gyerekek maguk alakítsák a folyamatot, formálják a kérdést, és találjanak válaszokat.

\begin{itemize}
      \item A tanár kialakítja a teret: körülbelül négygyerekenként egy számítógép, amit körbe lehet ülni.
      \item A gyerekek maguk formálják a csoportjukat, sőt még csoportot is válthatnak a munka során. Mozoghatnak, kérdezhetnek, ,,leshetnek''

            más csoportoktól.
      \item Körülbelül 30--45 perc után a csoportok prezentálják a kutatásuk eredményét.
\end{itemize}

\paragraph{A jó kérdések}
A Nagy Kérdésekre nincs könnyű válasz. Ezek nyílt és nehéz kérdések, és előfordulhat, hogy senki sem tudja még rájuk a választ. A cél, hogy mély és hosszú beszélgetéseket generáljanak. Ezek azok a kérdések, amikre érdemes nagyobb elméleteket állítani, amiket jobb csoportban megvitatni, amikről érvelni lehet és kritikusan gondolkodni.

A jó kérdések több témát, területet (tantárgyat) kapcsolnak össze: a \emph{,,Mi a
      hangya''} kérdés például nem érint annyi különböző területet, mint a ,\emph{,Mi
      történne a Földdel, ha minden hangya eltűnne''}.

\paragraph{Fegyelmezés nélkül}
A tanár feladata a folyamat során meghatározni a Nagy Kérdést, és tartani a kereteket. A cél, hogy a gyerekek maguk szervezzék saját munkájukat, így minimális beavatkozás javasolt a tanár részéről. Kezdetben, gyakorló csoportoknál, a tanárnak sokszor kell emlékeztetnie magát, hogy idővel kialakul a rend. \emph{,,Bízz a
      folyamatban!''
} Amikor a tanár úgy látja, hogy nem megy a munka, akkor csak finoman emlékezteti a csoportokat, hogy lassan jön a prezentáció ideje. Amikor valaki a csoportjáról panaszkodik, akkor elmondhatja, hogy szabad csoportot váltani. Ha valaki zavarja a többieket, akkor megfigyelheti, hogy a gyerekek tudnak-e már konfliktust feloldani. Ha valaki nem vesz részt a munkában, akkor gondolkozhat olyan kérdésen, ami az éppen demotivált gyerekeket is bevonzza.

\subsection{Megfontolt gyakorlás}
Ahogy \citep{ericsson2016peak} is kimutatja, bárki tudja valamennyi készségét, képességét fejleszteni, ha megtervezetten, megfontoltan gyakorolja. A Budapest School a hagyományosan készségtárgyként számon tartott ének, rajz, testnevelés és technika témákon kívül nagyon sok mindent kezel készségként: írásbeli érettségi vizsgát tenni magyarból, geopolitikai elemzéseket végezni, hiperbolikus függvényekkel egyenletet megoldani épp annyira értelmezhetőek készségként, mint a domináns csoporttagokkal való együttműködés, vagy az, hogy egy stresszhelyzetben lenyugtassuk önmagunkat.

A készség- és képességfejlesztés legjobb eszköze a megtervezett gyakorlás: a fejlődés érdekében okosan gyakorlunk. A megfontolt gyakorlás jellemzője:

\begin{description}
      \item[Világos és specifikus cél] Fontos, hogy tudjuk, mit gyakorlunk, mit
            akarunk elérni.  Lehetőleg a cél legyen mérhető és mindenképp realisztikus, elérhető.
      \item[Fókusz] Gyakorlás során egy dologra érdemes figyelni
      \item[Komfortzónán kívül kell lenni] Az edzőnek, tanárnak, trénernek
            néha érdemes a tanulót kicsit ,,nyomni''. Emlékeztetni, hogy mindig lehet kicsit többet elérni.
      \item[Folyamatos visszajelzés] Nagyon gyakran kap a tanuló visszajelzést,
            mindig tudja, hogy mikor és miben fejlődött.
\end{description}
