\section{Jutalmazás és értékelés}
\label{jutalmazas}

Fontos alapelv, hogy minél inkább a folyamatot, cselekvést jutalmazzuk, értékeljünk. Tehát arra fókuszáljunk, hogy
 \emph{mit csinált} a gyerek, és ne arra, hogy \emph{milyen} a gyerek. Azokra
a viselkedésmintákra adjunk megerősítő visszajelzést, amit látni szeretnénk a gyerekben később. 
Lehetőleg kerüljük a statikus jellemzőkre, személyiségjegyekre vonatkozó értékelést. Így nem azt mondjuk, hogy
\emph{,,mindig jó vagy matekból''}, hanem, hogy \emph{,,kitartóan és odafigyelve oldottad meg a feladatot''}. A Budapest School tanárai
ezért nem szeretik, ha bármikor elhangzik, hogy \emph{,,tehetséges gyerek vagy''} vagy \emph{,,jaj, de cuki!''}. 

Álljon itt néhány példa cselekvésre vontakozó visszajelzésre. 

\begin{itemize}

      \item
            Fantasztikus, ma egy nagy kihívást választottál!
      \item
            Bátran vállaltad a rizikót!
      \item
            Nagyon jó! Tényleg sokat próbálgattad.
      \item
            Kitartóan csináltad, erre nagyon büszke vagyok!
      \item
            De jó, valami újat próbáltál ki ma!
      \item
            Köszönöm, hogy ma valakinek segítettél.
      \item
            Nagyon nagy öröm látni a haladásodat!
      \item
            Ne feledd, mindannyian tudunk a hibáinkból tanulni. Örüljünk annak,
            hogy ma valamit jobban tudunk, mint előtte.
      \item
            Wow, egy nehéz feladatot oldottál meg!
      \item
            Szép munka! Kipróbáltál egy másik módszert.
\end{itemize}

Az iskolában történő folyamatos visszajelzés a mindennapok része. Ezt írja le \aref{sec:ertekeles}. fejezet. 
