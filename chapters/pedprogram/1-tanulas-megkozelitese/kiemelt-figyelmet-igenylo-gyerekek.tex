\section{A kiemelt figyelmet igénylő gyerekek}
\label{sec:kiemelt_figyelem}

A kiemelt figyelmet igénylő tanulók támogatásához  elsődlegesen a tanulásszervezőknek, mentoroknak és a modulvezetőknek kell differenciáltan, sokszínűen, türelemmel és személyre szabottan megközelíteni a gyerekeket.

\begin{itemize}
      \item A mentoroknak ismerniük kell a gyerek sajátosságait. A szülőnek és a mentornak őszintén, egymást támogatva kell a gyerek egyéni támogatására felkészülni. Ehhez sokszor külső diagnózisra, szakember bevonására és a felnőttek közötti nehéz beszélgetésekre van szükség.
      \item A mentoroknak meg kell osztani a gyerekek sajátos igényeit a többi tanárral, a modulvezetőkkel, hogy ők is fel tudjanak készülni a gyerekek személyre szabott támogatására.
      \item A tanulásszervezésnek, a moduloknak, a foglalkozásoknak, szóval min\-dennek, ami az iskolában történik,  differenciáltnak, egyéniesítettnek kell lennie, hogy az eltérő képességekkel bíró gyerekek is tudjanak együtt tanulni. Ahogy egy nagycsaládban  is figyelünk a különböző gyerekek eltérő igényeire, képességszintjeire.
      \item A szemléltetésnek, tevékenységeknek sokoldalúaknak kell lenniük, sokféle feladatot, speciális eszközöket kell használni. A gyerekek élménye változatos kell hogy legyen.
      \item El kell kérni a szülőktől a diagnózist, beszélni kell róla. Külső szakemberekkel konzultálni kell, és a diagnózisban szereplő javaslatokat be kell építeni a mindennapok tervezésébe.
      \item Az egyéni haladási ütem biztosítására egyéni fejlesztési és tanulási tervet kell készíteni.
      \item A tanároknak együtt kell működniük a gyermek/tanuló fejlesztésében részt vevő szakemberekkel.
\end{itemize}

Ez sok feladat, ami a tanárokra, a csoportra nagy terhet tud róni. Ezért a mikroiskolákban a kiemelt figyelmet igénylő gyerekek arányát körülbelül 20\% alatt kell tartani, hogy a támogatásukra legyen egyéni figyelem.

A fenti listából a legfontosabb elem: a tanár, gyerek, szülő hármas mellé be kell általában hívni külső, a gyerek igényeit jól értő szakembert támogatónak. A Budapest School iskola csak akkor tud segíteni, ha minden fél tudatosan áll hozzá a kiemelt figyelem szükségletéhez.