\section{A kiemelt figyelmet igénylő
  gyerekek}\label{sec:kiemelt_figyelem}

\begin{verbatim}
A kiemelt figyelmet igénylő tanulókkal kapcsolatos pedagógiai tevékenység helyi rendje

> (ez a korábbi szabályozás terminológiájával élve a következő tematikai egységeket foglalja magába: a beilleszkedési, magatartási nehézségekkel összefüggő; a tehetség, képesség kibontakoztatását és a szociális hátrányok enyhítését segítő tevékenységet és a tanulási kudarcnak kitett tanulók felzárkóztatását segítő programokat, így tehát a sajátos nevelési igényű és a hátrányos helyzetű tanulók integrációjának sajátos programelemei is idetartoznak pl. Officina Bona, IPR);
\end{verbatim}

TODO: max 20\% TODO: mentor tudja, h mi a szitu
\todo[inline]{SNI, BTM megkulonboztetese, es irjuk le, hogy a mentor tudja,
  hogy mi a szitu.}
A tanulásszervezők, modulvezetők feladata:

\begin{itemize}
  \item
        Megfelelő tanulásszervezési formákkal és módokkal biztosítani, hogy a
        tanórákon és a tanórán kívüli tevékenységben érvényesüljön a
        differenciált, az egyéniesített fejlesztés, eltérő képességekhez,
        viselkedéshez való alkalmazkodás.
  \item
        Olyan tanulási környezetet, speciális módszerek, tapasztalatszerzési
        lehetőség biztosítása, amelyben sokoldalú szemléltetéssel,
        cselekvéssel, gazdag feladattárral, speciális eszközök alkalmazásával
        valósul meg készség- és képességfejlesztés.
  \item
        A pedagógus a tanórai tevékenységek/foglalkozások tervezésébe építse
        be a pedagógiai diagnózisban szereplő javaslatokat.
  \item
        A pedagógus a tananyag adaptálásánál, feldolgozásánál vegye figyelembe
        az egyes tanulók fejlettségi szintjét, a támogatás szükséges mértékét.
  \item
        Az egyéni haladási ütem biztosítására egyéni fejlesztési és tanulási
        terv készítése, individuális módszerek, technikák alkalmazása.
  \item
        A pedagógus működjön együtt a gyermek/tanuló fejlesztésében résztvevő
        szakemberekkel.
\end{itemize}