
Az oktatás tartalmának előzetes szabályozása helyett a Budapest School a
tanulás módjára helyezi a hangsúlyt. Az iskola alapelve,
hogy
integratív módon folyamatosan fejlessze azokat a pedagógiai, pszichológiai és
szervezetfejlesztési módszereket,
amelyek korszerű módon tudják segíteni a tanulás tanulását, az egyéni és
csoportos fejlődést, a konfliktusok feloldását.

A tanulás tartalmát tekintve a Budapest School saját alternatív
kerettantervére támaszkodik, amely a tanulás rendszerét, annak
folyamatát szabályozza. E dokumentum alapján az állami kerettanterv
tanulási eredménycéljain történő végighaladás mellett a Budapest School
nagy hangsúlyt fektet a gyerekek saját tanulási céljaira és a
célállítás módjára.

A gyerekek életkori, fejlettségi, valamint szociális és érzelmi
állapotának figyelemmel kísérésében, valamint a saját tanulási célok
kitűzésében és elérésében minden gyereket egy \emph{mentortanár} kísér végig
az iskolán.\todo{hivatkozni az AKG patronusra} A mentortanár személye változhat
az iskolában töltött évek
alatt, de arra is van lehetőség, hogy ugyanaz a mentor kísérjen végig
egy gyereket az általános iskola első osztályától az érettségiig.

\section{A Budapest School mikroiskola-hálózata}


A Budapest School mikroiskolák hálózataként működik. A mikroiskola a Budapest
School iskolahálózatának legkisebb egysége. Nem egy önálló intézmény, és nem a
telephely szinonimája.

Az egyes mikroiskolákban a gyerekek kevert korosztályú tanulóközösségként,
érdeklődésük és képességeik alapján, együtt és egymástól tanulnak és fejlődnek.
A tanulóközösség fontos célja, hogy biztonságot, támogatást nyújtson, és
\emph{így} segítse a közösség tagjainak a minőségi tanulását. Az egyes
mikroiskolák egymással is kapcsolatban vannak,
egymástól tanulnak, így a gyerekek tanulási helye változhat.

\paragraph{Épületek, tagintézménzek és telephelyek}
A Budapest School iskola egy székhellyel és több tagintézménnyel, illetve
telephellyel működhet. Fontos azonban, hogy egy telephelyen több mikroiskola is
működhet, és egy mikroiskola több telephely adottságait is kihasználhatja.

A 
\ifkerettanterv
kerettanterv
\else
program
\fi tudatos szándéka, hogy az épületet és a
tanulás szervezeti formáját ne kösse össze szorosan. Egy-egy mikroiskola az
épületére úgy gondol, mint egy átmeneti bérleményre vagy egy helyre, amit most
meglátogat.
Változhat, hogy egy-egy épületet éppen melyik mikroiskola használja.

A tanulás helyszínének változtathatósága lehetővé teszi, hogy a
múzeumpedagógiát, a tudományos kutatóközpontokkal való együttműködést, az erdei
iskolák világát, a sportegyesületek tevékenységeit, vagy más külső helyszínen
megvalósuló szakköröket a Budapest School gyerekek számára a mindennapok
integrált részvé tegyük. Fontos kiemelnünk, hogy mindeközben mindenkinek
szüksége van bázisra, biztonságot adó otthonra: ezért van minden Budapest
School gyereknek és tanárnak egy elsődleges helye.
\ifkerettanterv
  \section{A Budapest School fenntartója}
  A Budapest School fenntartójának a nemzeti köznevelésről szóló 2011. évi CXC.
  törvényben (a továbbiakban: Nkt)  szabályozottakon felül feladata
  \begin{itemize}
    \item a Budapest School hálózatának építése, működési struktúrájának
          fejlesztése, az adminisztratív és szabályozási rendszer kialakítása,
          valamint az egyes mikroiskolákban a gyerekek tanulását, fejlődését segítő
          folyamatok megalkotása;
    \item  az adminisztratív és jogi folyamatok kezelése;
    \item  a tanárok kiválasztási folyamatának kezelése és folyamatos tanulásuk
          szervezése;
    \item a minőségbiztosítási és fejlesztési rendszer kialakítása és
          működtetése.
  \end{itemize}

  Azaz a fenntartó nemcsak a fenntartásért, hanem a fenntartható fejlődésért is
  felel, és ebben támogatja az egyes mikroiskolákat.

  \section{A Budapest School mikroiskolái}
\fi
A tanulási folyamat működtetéséért a Budapest School egyes mikroiskolái
felelősek. Az iskolákat tanulásszervező-tanárok (a különböző pedagógus
szerepek kibontása a \ref{sec:tanarok}. fejezetben található) egy csoportja
alkotja és vezeti. Így a
mikroiskolák vezetéséért a tanulásszervező-tanárok felelősek.

Az egyes mikroiskolák különböznek egymástól abban, hogy az oda járó gyerekek
pontosan mit és mikor
tanulnak vagy alkotnak. Azonban a következő alapelvek az összes mikroiskolára
érvényesek.

\paragraph{Az iskoláknak \emph{saját fókuszuk van}.}

Van olyan mikroiskola, amely a fejlesztési célok eléréséhez és az egyéni
célok
mentén már 6 éves gyerekek tanulásánál a robotika eszközeit használja,
másutt drámafoglalkozásokkal fejlesztik 12 éves gyerekek a szövegértésüket
és
éntudatukat. A
mikroiskola-rendszerben rejlik annak a lehetősége, hogy egy adott tanulási
környezetben a hangsúlyok úgy váltakozhassanak a csoport és az egyén
érdeklődését követve, hogy közben a tanulási egyensúly fennmaradjon a
tantárgyak fejlesztési területei között. Az iskolák nemcsak abban térnek el
egymástól, hogy kevert korcsoportban, más korosztályú gyerekek, más
érdeklődések mentén, és ily módon más célokat követve tanulnak, hanem
területileg, regionálisan is eltérőek lehetnek.

\paragraph{Az iskolákban a tanulók nagymértékben befolyásolják, hogy mit és
  hogyan tanulnak és alkotnak.}

A tanárok választási lehetőségeket dolgoznak ki, amikből a gyerekek (a
mentoruk és szüleik segítségével) a saját céljaikat, érdeklődésüket
leginkább
támogató \emph{egyéni tanulási tervet} alkotnak. Az iskolákban (a tanárok
által
meghatározott kereteken belül) megfér egymással több, különböző egyéni
céllal
rendelkező gyerek.

Eltérhet, hogy egy-egy gyerek mit tanul, ezért az is, hogy mikor és hogyan
sajátítja el a szükséges ismereteket: egy közösségben megfér a központi
felvételire fókuszáló 11 éves gyerek, és az olyan is, aki ekkor inkább a
Mine\-craft programozásában akar elmélyülni, ezért más képességek
fejlesztésével
lassabban halad. A tanárok feladata és felelőssége, hogy olyan közösségeket
válogassanak össze, amelyek kellően diverzek, és mégis jól működnek, a
gyerekek
igényeit és a kerettanterv céljait egyaránt megfelelően kielégítik.

\paragraph{A mikroiskolák kevert korosztályos közösségek.}

A Budapest Schoolban a gyerekek nemcsak a saját korcsoportjukban, hanem
kevert korosztályok szerinti csoportokban is tanulnak egy-egy mikroiskolában,
hasonlóan a Montessori-féle kevert korosztályos csoportokhoz, vagy az
\emph{önállósági szintek} (independece level  \citep{indepence_level}) alapján
szervezett tanulócsoportokhoz. A csoportok létrehozásakor arra törekszünk, hogy
olyan gyerekek tanuljanak együtt, akik tudják egymást támogatni a tanulásban.

\paragraph{A mikroiskolák együtt fejlődő közösségek.}

Úgy fejlődnek, mintha első lépésben óvodapedagógusok által vezetett óvodai
csoportok
alakulnának át kéttanítós alsós osztályokká. Majd amikor a tanuláshoz újabb
tanárokra van szükségük, akkor bővül a tanárcsapat. Így jöhet létre akár 50
gyerek és 5-8
tanár közössége. Amikor a fejlődésükhöz újabb tanárra van szükségük a
gyerekeknek -- például egy speciális képesség erősítéséhez --, akkor vagy a
Budapest School másik mikroiskolájából jövő tanártól, vagy egy külsős
szakembertől
tanulhatnak. Amikor az érettségire készülve maguk alkotnak gyakorló
csoportot, akkor a tanulásukat akár már önmaguknak is megszervezhetik.

\paragraph{Különféle tanulási struktúrák jöhetnek létre a mikroiskolákon
  belül.}

A közösséget kisebb csoportokra bonthatjuk, ha a tanulásszervezés ezáltal
hatékonyabb. Egyes modulokban egy-egy projektre szerveződnek a gyerekek,
ilyenkor gyakran az eltérő képességű és életkorú gyerekek is megférnek
egymás
mellett. Más moduloknál a csoportokat általában képességszint alapján hozza
létre a tanár. Ilyen lehet a másodfokú egyenletek megoldóképletét megismerő
csoport, az írni tanulók csoportja, vagy egy angol nyelvű újság
szerkesztésére
és megírására alakult modul, ahol a nyelvismeretnek és a szövegalkotási
képességnek már egy olyan szintjén kell lenni, hogy a projektnek jól
mérhető
kimenete lehessen.

\paragraph{A mikroiskolák diverz, integratív közösségek.} A Budapest School
iskolák társadalmi,
kulturális és gazdasági értelemben is egyik fő céljuknak tartják az
integrációt addig,
amíg az a közösség céljait szolgálja.

\paragraph{A Budapest School mikroiskolái tanuló közösségek.} Mindig, minden
módszer,
folyamat fejleszthető, ezért a tanárok feladata, lehetősége, hogy az aktuális
helyzethez illő legalkalmasabb módszert válasszák a tanulás segítéséhez.

A Budapest School mikroiskolák célja, hogy jól átlátható, követhető és
folyamatosan fejlődő folyamattá váljon a tanulás mind a tanuló, mind a tanár,
mind
a szülő részéről. Kerettantervünk folyamatszabályozást nyújt, nem kimeneti
szabályozást. A kimenet a gyerekek és a közösség képességeitől, céljaitól és
érdeklődésétől,
valamint a társadalmi szabályozási környezettől függ.

\ifkerettanterv
  \ifkerettanterv
  \section{A Budapest School tanárai: a tanulásszervezők, a mentorok és a
    modulvezetők.}
\else
  \section{Különböző tanári szerepek: a tanulásszervező, a mentor és a
    modulvezető}

\fi

\label{sec:tanarok}
\begin{quote}

  Tanulni bárkitől lehet, aki tud olyasmit mutatni, ami felkelti a tanuló
  érdeklődését, és elő tudja segíteni a fejlődését. Tanítani az tud igazán,
  aki tanulni is tud.
\end{quote}
A Budapest Schoolban a gyerekek azokat a felnőtteket tekintik tanáruknak, akik
minőségi időt töltenek velük, és segítik, támogatják vagy vezetik őket a
tanulásukban. Több szerepre bontjuk a tanár fogalmát: a gyerek egy (és csak
egy) felnőtthöz különösen kapcsolódik, a \emph{mentortanárához}, aki rá
különösen figyel. Ezenkívül a gyerek tudja, hogy a mikroiskola mindennapjait
egy tanárcsapat, a \emph{tanulásszervezők} határozzák meg, azaz ők vezetik az
iskolát.  A foglalkozásokon megjelenhetnek más tanárok, a \emph{modulvezetők},
akik egy adott foglalkozást, szakkört, órát tartanak. Néha megjelennek más
felnőttek, akik párban vannak egy másik tanárral: ők az \emph{asszisztensek}, a
\emph{gyakornokok} vagy az \emph{önkéntes segítők}.

Szervezetileg minden mikroiskolának van egy állandó \emph{tanárcsapata}, a
tanulásszervezők. Állandó, mert legalább egy tanévre elköteleződnek, szemben a
modulvezetőkkel, akik lehet, hogy csak egy pár hetes projektre vesznek részt a
munkában.

A tanulásszervezők általában mentorok is, de nem minden esetben. Nem lehet
mentor az, aki a gyerek mikroiskolájában nem tanulásszervező, mert nem lenne
rálátása a mikroiskola történéseire. Egy tanulásszervező lehet több
mikroiskolában is ebben a szerepben, és így mentor is lehet több mikroiskolában.

\paragraph{Mentor}
Minden tanulónak van egy \emph{mentora}, aki az egyéni céljainak
megfogalmazásában és
a fejlődése követésében segíti. Minden mentorhoz több tanuló tartozik, de nem
több mint 12. A mentor együtt dolgozik a Budapest School tanárcsapatával, a
szülőkkel és az általa mentorált gyerekekkel. A mentor segít az általa
mentorált gyereknek, hogy a tantárgyi fejlesztési célok és
a
saját magának megfogalmazott egyéni célok között megtalálja  az egyensúlyt, és segít megalkotni a
gyerek \emph{saját
  tanulási tervét}.

A mentor a kapocs a Budapest School, a szülő és a gyerek között.

\begin{itemize}
  \item Képviseli a Budapest Schoolt, a mikroiskola közösségét.
        \begin{itemize}
          \item Ismeri a Budapest Schoolt, a lehetőségeket, a tanulásszervezés
                folyamatait.
          \item Együtt tanul más Budapest School mentorokkal, együtt dolgozik a
                tanártársaival.
        \end{itemize}

  \item Ismeri, segíti, képviseli a gyereket.
        \begin{itemize}
          \item  Tudja, hol és merre tart mentoráltja, ismeri a képességeit,
                körülményeit, szándékait, vágyait.
          \item    Segít az egyéni célok elérésében, felügyeli a haladást.
          \item    Megerősíti mentoráltjai pszichológiai biztonságérzetét.
          \item   Visszajelzéseket ad a mentoráltjainak.
          \item    Segít abban, hogy az elért célok a portfólióba kerüljenek.
          \item    Összeveti a portfólió tartalmát a tantárgyak fejlesztési
                céljaival.
        \end{itemize}

  \item Együtt dolgozik, gondolkozik a szülőkkel, képviseli igényüket a
        közösség felé.
        \begin{itemize}
          \item Erős partneri kapcsolatot épít ki a szülőkkel, információt oszt meg
                velük.
          \item Segít a gyerekekkel közös célokat állítani.
          \item Szülő számára a mentor az elsődleges kapcsolattartó a különféle
                iskolai ügyekkel kapcsolatban.
        \end{itemize}

\end{itemize}

A mentor egyszerre felelős a mentorált tanuló előrehaladásának segítéséért,
és
közös felelőssége van a mentortársakkal, hogy az iskolában a lehető legtöbbet
tanuljanak a gyerekek. A mentor folyamatosan figyelemmel követi az egyéni
tanulási tervben megfogalmazottakat, és ezzel kapcsolatos visszajelzést ad a
mentoráltnak és a szülőnek.

\paragraph{Tanulásszervező}
Csoportban dolgozó, iskolaszervező, strukturáló tanár. Egy mikroiskola
állandó tanári
csapatát 2-7 tanulásszervező alkotja, akik egyedileg meghatározott szerepek
mentén a mikroiskola mindennapjainak működtetéséért felelnek. Minden mentor
tanulásszervező is. A tanulásszervezők tarthatnak
modulokat, sőt, kívánatos is, hogy dolgozzanak a gyerekekkel, ne csak
szervezzék az életüket.
Ők rendelik meg a külső modulvezetőktől a munkát, ilyen értelemben a
tanulási utak projektmenedzserei.

\paragraph{Modulvezetők}

Bárki lehet modulvezető, aki képes akár egy egyetlen alkalommal történő, vagy
éppen
egy egész trimeszteren át tartó tanulási, alkotási folyamatot vezetni. Ők
általában
az adott tudományos, művészeti, nyelvi vagy bármilyen más terület szakértői.

Modulokat a tanulásszervezők is vezethetik, de külsős, egyedi megbízással
dolgozó szakemberek is megjelennek modulvezetőként. Modulvezető lehet bárki,
akiről az őt megbízó tanárcsapat tudja, hogy képes gyerekek folyamatos
fejlődését és egy tanulási cél felé való haladását segíteni. A moduláris
tanmenettel \aref{sec:modularis_tanmenet}. fejezet foglalkozik.
\fi


\section{Saját tanulási célok}
\label{sec:tanulasi_celok}

Minden gyerek megfogalmazza és háromhavonta újrafogalmazza a \emph{saját
      tanulási céljait}: eredményeket, amelyeket el akar érni, képességeket,
amelyeket fejleszteni akar, szokásokat, amelyeket ki akar alakítani. A saját
célok elfogadásakor a gyerek és a mentora a szülőkkel együtt \emph{tanulási
      szerződést} köt.

Csak olyan célok kerülhetnek a saját célok közé, amelyek
minden érintettnek biztonságosak, és amelyek összhangban vannak a tantárgyi
fejlesztési célokkal és tanulási eredményekkel. A szerződésben rögzíthetőek
tanulási eredményekre
vonatkozó megállapodások,
tantárgyi évfolyamszintekre vonatkozó elvárások (pl. ,,\emph{haladjon egy
      évfolyamszintet egy év alatt}'' vagy ,,\emph{készüljön fel emelt szintű
      érettségire}''), és a tantárgyi rendszeren kívüli célok és
feladatok.

Fontos megkötés, hogy a saját tanulási célok legalább a felének
\ifkerettanterv
      \aref{sec:tantargyi_tanulasi_eredmenyek}. fejezetben
\else
      a kerettanterv Tantárgyi tanulási eredmények fejezetében
\fi
felsorolt tanulási eredmények elérésére kell vonatkoznia. A másik
fele szabadon alakítható.

Háromhavonta a tanulásszervezők és a gyerekek megállnak, reflektálnak az elmúlt
időszakra, és a tapasztalatok, valamint az elért célok ismeretében és az új
célok figyelembevételével újratervezik, újraszervezik a foglalkozások rendjét,
tehát azt, hogy mikor és mit csinálnak majd a gyerekek az iskolában.
A mindennapi tevékenység során tapasztalt élmények, alkotások, elvégzett
feladatok, kitöltött vizsgák, tehát mindaz, ami a gyerekekkel történik, bekerül
a portfóliójukba. Még az is, amit nem terveztek meg előre.

A gyerekeket a mentoruk segíti a saját célok kitűzésében, a különböző
választásoknál, a portfólióépítésben, a reflektálásban. A tanulási célok
kitűzése az önirányított tanulás fokozatos fejlődésével és az életkor
előrehaladtával folyamatosan egyre önállóbb tevékenységgé válik. Tanulási
útján, céljai kitűzésében a mentor kíséri végig a gyerekeket.

A Budapest School személyre szabott tanulásszervezésének jellegzetessége, hogy
a gyerekek a saját céljuk irányába haladnak, az adott célhoz az adott
kontextusban leghatékonyabb úton. Tehát mindenki rendelkezik saját célokkal,
még akkor is, ha egy közösség tagjainak céljai a tantárgyi tanulási eredmények
azonossága, vagy a hasonló érdeklődés miatt akár  80\% átfedést mutatnak.

A NAT műveltségi területeiben megfogalmazott követelmények teljesítése is célja
a tanulásnak, a tanulás fő irányítója azonban más. Mi azt kérdezzük a
gyerekektől, hogy \emph{ezenfelül} mi az ő személyes céljuk.

\section{A tanulási szerződés}

A tanulási szerződés az előbbiekben említett gyerek-mentor-szülő közötti
megállapodás, ami rögzíti
\begin{enumerate}
      \item a gyerek, a mentor (iskola) és a szülő igényeit, elvárásait;

            ezek lehetnek: \emph{,,szeretném, ha a gyerekem naponta olvasna''}
            típusú
            folyamatra vonatkozó kérések, vagy erősebb \emph{,,változtatnod
                  kell a
                  viselkedéseden, ha a közösségben akarsz maradni''} igények,
            határok
            megfogalmazása;

      \item a gyerek céljait a következő trimeszterre, vagy a tanév végéig;

      \item a gyerek, mentorok (iskola) és szülő vállalásait, amivel támogatják
            a
            cél
            elérését és a felek igényének elérését.

\end{enumerate}

A tanulási szerződésre jellemző, hogy
\begin{itemize}
      \item A kitűzött célokat minél specifikusabban, mérhetőbben kell
            megfogalmazni.
            Javasolt az OKR  (Objectives and Key Results, azaz	Cél és Kulcs
            Eredmények)
            \citep{okr} vagy a SMART (Specific, Measurable, Achievable,
            Relevant,
            Time-bound, azaz Specifikus,  Mérhető, Elérhető, Releváns és Időhöz
            kötött)
            \citep{wiki:smart} technika alkalmazása, hogy minél specifikusabb,
            teljesíthetőbb, tervezhetőbb és könnyen mérhető célokat tűzzenek
            ki.

      \item A kitűzött célokban való megállapodást követően, megállapodást
            kell
            kötni arról is, hogy ki és mit tesz azért, hogy a gyerek a célokat
            elérje.

      \item A mentor a teljes mikroiskolát (a többi tanárt, a közösséget)
            képviseli
            a
            megállapodás során.
\end{itemize}

A tanulási szerződést néha hívjuk \emph{megállapodásnak} is. A megállapodás és
szerződés szavakat ez a kerettanterv szinonimának tekinti. A \emph{learn\-ing
      con\-tract} az önirányított tanulást hangsúlyozó felnőttképzéssel
foglakozó
irodalomban
bevett szakkifejezés már a 80-as évektől \citep{Malcolm77}. Ennek a magyar
nyelvben inkább a szerződés felel meg. Egy másik szakterületen, a
pszichoterápiás munkában a terápiás szerződések megkötésekor a közös munka
kereteinek kialakítását és fenntarthatóságát hangsúlyozzák
\citep{pszichoterapia}. Erre is utalunk a tanulási szerződés elnevezéssel. Van,
amikor a \emph{hármas szerződés} kifejezést használjuk, hangsúlyozva, hogy mind
a három szereplőnek elfogadhatónak kell tartania a szerződés tartalmát.

\section{Visszajelzés, értékelés}
\label{sec:ertekeles}
Ahhoz, hogy hatékony legyen a tanulás, fejlődés, fontos, hogy a gyerekek,
tanárok és szülők is tudják, hogy
\begin{enumerate}
      \item hol tart most egy gyerek, mit tud most,
      \item hova akar vagy kell eljutni, azaz, mi a célja,
      \item mi kell ahhoz, hogy elérje a célját.
\end{enumerate}
Ezek mellett mindenkinek hinnie kell abban, hogy odafigyeléssel, gyakorlással a
gyerek meg tud tanulni egy konkrét dolgot. Fontos, hogy magas legyen a gyerekek
énhatékonysága,  erős legyen az önbizalmuk, és nem szabad félniük a hibázástól,
a nem-tudástól,
mert a tanulás első lépése, hogy elfogadjuk, hogy valamit nem tudunk. Azaz
fontos, hogy fejlődésfókuszú gondolkodásuk (growth mindset)
\citep{growthmindset} legyen, azaz
\begin{enumerate}
      \setcounter{enumi}{3}
      \item hinniük kell, hogy el tudják érni a céljukat.
\end{enumerate}

Egy visszajelzés, értékelés akkor jó és hasznos, azaz hatékony, ha ebben a négy
dologban segít. Mai tudásunk szerint ehhez:
\begin{itemize}
      \item Rendszeresen visszajelzést kell kapnunk és adnunk.
      \item A tanulási céloknak és visszajelzéseknek minél specifikusabbaknak
            kell
            lenniük (azaz például ne a 8. oszályos \emph{matematikatudást}
            értékeljük,
            hanem hogy mennyire képes valaki \emph{fagráfokat
                  használni
                  feladatmegoldások során}\footnote{Ez a konkrét példa a STEM
                  tantárgy
                  egyik
                  tanulási eredménye.}).
      \item A \emph{,,hol tartok most''} diagnózisnak mindig cselekvésre,
            viselkedésre, aktív tevékenységre kell vonatkoznia. Ne az legyen a
            visszajelzés, hogy \emph{,,ügyes vagy egyenletekből''}, hanem
            \emph{,,gyorsan és
                  pontosan oldottad meg a 4 egyenletet''}. A legjobb, amikor a
            visszajelzés
            konkrét megfigyelésen alapul, és tudni, hogy mikor, hol történt az
            eset:
            \emph{,,amikor társaiddal Minecraftban házat építettél, akkor
                  pontosan
                  kiszámoltad a ház területét''.}
      \item Ha a cél nem a mások legyőzése, akkor a visszajelzés se
            tartalmazzon
            olyan állítást, ami másokhoz hasonlít (így kerüljük a
            \emph{tehetség}
            szót is,
            aminek bevett definíciója szerint az átlagnál jobb képesség). A
            másokhoz való
            szint felmérése akkor (és csak akkor) fontos, amikor a cél egy
            versenyszituációban jó eredményt elérni.

      \item A gyerek legyen részese a visszajelzésnek. Értse, tudja, hogy miért
            kapta
            azt a visszajelzést, a legjobb, ha -- amikor ezt a képességei
            engedik
            -- önmaga
            képes elvégezni a visszajelzést, vagy annak egy részét.
      \item A visszajelzésnek transzparensen hatással kell lennie a
            tanulásszervezésre. Legyen része a folyamatnak, és a gyerek, tanár
            és a
            szülő
            is értse, hogy a visszajelzés alapján mit és hogyan csinálunk
            másképp.
\end{itemize}

\paragraph{Többszintű visszajelzés} A Budapest School iskolákban a gyerekek
többféle visszajelzést kapnak. \begin{enumerate}
      \item Minden modul elvégzése után a modul céljai, témája, fókusza alapján
            a
            modulvezetők visszajelzést adnak a tanulásról, eredményekről,
            viselkedésről.
      \item Trimeszterenként a mentorok visszajelzést adnak arról, hogy a
            gyerek
            általában hogyan haladt a tanulási célok felé.
      \item Ennek része, hogy a tantárgyi tanulási eredmények alapján hogyan
            haladt a
            gyerek a tantárgyak évfolyamszinthez tartozó követelmények
            teljesítésében. Az
            évfolyamok, mint elérhető szintek Budapest School értelmezését
            \aref{sec:evfolyamok}. fejezet tárgyalja.
      \item A mentorok irányításával a gyerekek visszajelzést kapnak arról,
            hogyan
            működnek a közösségben.
\end{enumerate}

\paragraph{Érdemjegyek, osztályzatok helyett értékelő táblázatok} A Budapest
School visszajelzéseinek sokkal részletesebbeknek kell lenniük, mint azt a
tantárgyi érdemjegyek és osztályzatok lehetővé teszik, ezért azok helyett a
kerettanterv
értékelő táblázatokat (angolul rubric) alkalmaz. Az értékelő táblázatban
szerepelnek az értékelés szempontjai és szempontonkénti szintek, rövid
leírásokkal.
Ezek alapján a gyerekek maguk is láthatják, hogy hol tartanak, hogyan
javíthatnak még a munkájukon. A táblázatok formája minden visszajelzés esetén
(értsd modulonként, célonként)
változtatható.

\section{Portfólió}
\label{sec:portfolio}
A modulok eredményeiből, a produktumokból és visszajelzésekből a gyerek és a
mentor portfóliót
állít össze, hogy a tanulás mintázatait észlelhesse, és a
tanárok tudatosabban tudják
a gyereket segíteni a céljai kitalálásában és elérésében. A portfólió a gyerek
céljainak nyomon követését szolgálja, és egyúttal a szülők felé történő
visszajelzés eszköze
is. Minden gyerek portfóliója folyamatosan épül: az tartalmazza az általa
elvégzett feladatokat, projekteket vagy azok dokumentációját, alkotásait,
eredményeit, az esetleges vizsgák eredményeit és a társaitól, tanáraitól kapott
visszajelzéseket. A \emph{portfólió célja}, hogy minden információ meglegyen
ahhoz,
hogy

\begin{itemize}
      \item a gyerek és mentora fel tudja mérni, hogy sikerült-e a kitűzött
            célokat
            elérni, illetve mire van szüksége még a gyereknek új célok
            eléréséhez;

      \item a szülő folyamatosan rálásson a gyereke tanulási útjára;

      \item megítélhető legyen, hogy a tantárgyi követelményekhez
            képest
            hol
            tart a gyerek;

      \item a gyerek a portfólió megtekintésével visszaemlékezhessen a
            tanultakra,
            ismételhessen, tudása elmélyülhessen;

      \item eredményei alapján bizonyítványt lehessen kiállítani.

\end{itemize}

A portfólió folyamatosan frissül, a mindennapi, formális, non-formális és
informális tanulási helyzetek
bármikor adhatnak okot a portfólió frissítésére. Az iskola életében kiemelt
szerepe van a következő eseményeknek.

\begin{enumerate}
      \item Minden \emph{modul végeztével} a portfólióba kerül:

            \begin{enumerate}

                  \item  A képesség elsajátításának, tanulási eredmény
                        elérésének a ténye.
                        Nincs
                        félig elsajátított képesség, tehát már értékelni nem
                        kell. Ha a modul során a gyerek megtanult százas
                        számkörben alapműveleteket
                        végezni, 
                        akkor
                        annyi kerül be a portfólióba, hogy ,,\emph{Szóban és
                              írásban
                              összead, kivon, szoroz és oszt a százas
                              számkörben.}''. Amennyiben a
                        készséget a
                        gyerek és a
                        tanár megítélése alapján nem sikerült megfelelően
                        elsajátítani,
                        úgy a gyakorlás
                        ténye kerül be a portfólióba.
                  \item Az alkotás vagy a projektmunka eredménye, ha a modul
                        célja egy
                        alkotás
                        létrehozása volt.
                  \item A részvétel ténye, ha a jelenlét volt a modul célja
                        (például
                        kirándulás
                        az Országos Kéktúra útvonalán).

            \end{enumerate}
      \item Az elvégzett vizsgák, tudáspróbák, képességfelmérők, diagnózisok
            eredményeit érdemes rögzíteni.

      \item A \emph{kipakolás} célja, hogy a gyerekek a tanároknak, szülőknek
            és
            más érintetteknek bemutassák elvégzett
            munkájukat, azaz
            a portfólióváltozásukat. A kipakolásra való felkészülés
            tulajdonképpen
            a
            portfólió összeállítása, prezentálásra való felkészítése, a
            \emph{portfólió
                  frissítése}.

      \item Társas visszajelzés eredményeként minden gyerek kap visszajelzést a
            társaitól. Ilyenkor összegyűjtik, mit tett a gyerek, ami a többiek
            elismerését
            és háláját kivívta. Ez is releváns adatokkal szolgálhat a
            portfólióhoz.

      \item A gyerek saját értékelése, reflexiója arról, hogyan értékeli, amit
            elért, fontos eleme a portfóliónak.

      \item A tanárok adhatnak kompetenciatanúsítványokat. Ezek
            rövid,
            specifikus visszajelzések, amelyek mutatják, ha valamit a gyerek
            megcsinált,
            valamiben fejlődött.
\end{enumerate}

A mentorok segítenek a gyerekeknek a tanulás módját, folyamatát és eredményeit
bemutatni
portfólióban.

\paragraph{Formai követelmények}
A portfóliónak rendezettnek, hozzáférhetőnek,
elérhetőnek, visszakereshetőnek és könnyen bővíthetőnek kell lennie. Olyan
(technológiai)
megoldást kell a mikroiskoláknak választaniuk, ami alapján
a gyerek, tanár és a szülő \emph{naponta} tudja a portfóliót bővíteni, és akár
\emph{heti rendszerességgel} át tudják tekinteni időrendben, modulonként vagy
tantárgyanként a portfólió bővülését.

A portfólió formátumára nincs egységes megkötés. Minden mikroiskola maga
alakítja ki a gyerekek, tanárok és szülők számára legjobban működő rendszert.
Évfolyamszintlépéshez és osztályzatokra váltáshoz az iskola csak digitális
formában tárolt és a kijelölt tanárok számára online elérhetővé tett portfóliót
fogad el.
\section{A csoportbontások}
\label{sec:csoportbontasok}

A Budapest School iskoláit mikroiskolák közösségeiből hozzuk létre. Így
egy gyerek elsődleges csoportja a mikroiskolájának közzössége, ami lehet
12 - 50 gyerek. Ezen belül modulonként eltér, hogy milyen
csoportbontásban dolgoznak. Több szinje van a csoportmunkának.
\begin{enumerate}

  \item A mikroiskola közössége heti rendszerességgel tarthat iskolagyűlést,
        fórumot, plenárist, iskolakonferenciát. Ilyenkor a mikroiskola közössége
        dolgozik együtt.
  \item  Modulokra kisebb csoportok jelentkezhetnek. Egy modul
        csoportjának rendezőelve lehet, hogy
        \begin{enumerate}
          \item egy képességszinten lévő gyerekek tanuljanak együtt;
          \item a közös érdeklődés hozza össze a
                csoporttagokat;
          \item an, hogy direkt a véletlenszerűségben van az
                érdekesség, mert keveredni akarunk;
          \item kölcsönös szimpátia és vonzalom
                lehet a modultagok között: most azért vannak egy csoportban, mert egy
                csoportban akartak lenni.
        \end{enumerate}
  \item  Egy-egy foglalkozáson belül is sokszor
        csoportot alkotunk, az előző elvek alapján.

\end{enumerate}

Arra is lehetőség van, hogy egy csoport tagjai több mikroiskolából álljanak
össze, ha az támogatja a tanulást és az utazás biztonságosan megoldható.
% \ketoldaltkep{pics/6A_TANTARGYI_SOMA.JPG}{pics/6b_TANTARGYI_FLORA.jpg}
\section{Évfolyamok és osztályzatok}
\label{sec:evfolyamok_osztalyzatok}
A Budapest Schoolban tanuló gyerekek saját tanulási célokat tűznek ki, modulokat választanak, tanulnak, alkotnak, trimeszterenként frissítik a portfóliójukat, mentorukkal és a modulvezetőkkel értékelik haladásukat, és ha kell, újraterveznek.

Eközben a gyerekek a tanulás és alkotás eredményeként évfolyamszinteken lépkednek fel, első szintről a tizenkettedik szintig tantárgyanként. Azt, hogy ez hogyan és mikor történik, azaz az évfolyamszintek elismerését -- a kerettantervvel összhangban -- az iskola transzparens folyamata szabályozza.

Hivatalos, tantárgyankénti osztályzatokat a gyerekek akkor és csak akkor kapnak, ha erre iskolaváltás, továbbtanulás, ösztöndíj (vagy más külső rendszer) miatt szükségük van vagy ezt a törvény az iskolának kötelezővé tesz. Tehát osztályzatok, érdemjegyek és vizsgák nélkül is van lehetőség évfolyamszinteket lépni.\footnote{Egy tantárgyból csak azon az évfolyamszinten lehet osztályzatot kapni, amelyik évfolyamon az a tantárgy kötelezőként szerepel a tantárgyi struktúrában \aref{tbl:oraszamok}~.táblázat alapján.}

A vizsga így teljesen átértékelődik a Budapest Schoolban. Az évfolyamszintek elismeréséhez és (szükség esetén) az érdemjegyek megállapításához nem elégséges a pillanatnyi tudást vagy képességet felmérő eseményt szervezni, hanem a teljes portfóliót kell értékelni és figyelembe venni. A portfólió sokkal gazdagabban dokumentálja, hogy egy gyerek mit csinált, mire volt képes, mint egy szóbeli vagy írásbeli feladatsor: előzetes tudás- és képességpróbák mellett tartalmazza az alkotások, projektek, visszajelzések, versenyek, stb. dokumentációit is.

\subsection{Évfolyamszintek}
\label{sec:evfolyamok}

A Budapest Schoolban az évfolyamokra úgy tekintünk, mint egy szerepjáték nehézségi szintjeire \citep{wiki:game_levels}: akkor léphet egy gyerek a következőbe, ha az évfolyamhoz köthető tantárgyi tanulási eredményekből eleget összegyűjtött.

A Budapest School évfolyamszintjei eltérnek az iskolák többségében alkalmazott évfolyamtól. A különbség kihangsúlyozása végett a kerettanterv az évfolyamszint kifejezést használja. A különbségek:

\begin{itemize}
      \item Egy gyerek tantárgyanként más-más szinten állhat.
      \item Nem biztos, hogy az egy korcsoportba tartozó gyerekek
        vannak\linebreak
        ugyanazon az évfolyamszinten.

      \item Nem mindig az egy évfolyamszinten lévők tanulnak együtt, előfordulhat, hogy a különböző szinten lévő gyerekek tudnak együtt és akár egymástól is tanulni.

      \item Egy év alatt több évfolyamszintet is lehet lépni.
      \item Évfolyamszintlépéshez szükséges eredményeket tanév közben is el lehet érni, még akkor is, ha a hivatalos bizonyítványok a szintlépést csak tanév végén mutatják meg.
\end{itemize}

Bár egy gyerek tantárgyanként eltérő szinten állhat, a hivatalos (azaz külső hivatalok, rendszerek számára értelmezhető) bizonyítványába mindig csak annak az évfolyamnak az elvégzése kerül be, amelyből minden tantárgy által definiált szükséges tanulási eredményt elérte. Formálisabban kifejezve a bizonyítványban a tantárgyankénti évfolyamszintek minimumát kell rögzíteni.

\subsubsection{Évfolyamszintlépés}
\label{sec:evfolyamszintlepes}
A gyerekek portfóliója alapján megállapítható, hogy egy adott évfolyamhoz köthető tantárgyi követelményeknek megfelel-e. Ehhez a gyerekek elvégzik a mentoruk segítségével a portfóliójuk (mit csináltak, mit tanultak, mit tudnak) összehasonlítását 
      \aref{sec:tantargyi_tanulasi_eredmenyek}.~fejezetben
 felsorolt tantárgyankénti bontásban megadott elvárt tanulási eredményekkel.

Ha a kapcsolódás biztosításához szükséges, a gyerekek a portfóliójukat kiegészíthetik kihívások, tudáspróbák, tesztek, szabványos vizsgák teljesítésével, melynek megszervezése az adott mikroiskola tanárközösségének a feladata.

Miután a gyerek (mentora és szülei segítségével) összeállította a port\-fó\-lió\-ját, jelzi az iskolának az évfolyamszintlépési kérelmét.

Ezt az iskola által kijelölt tanulásszervezők\footnote{A kérelmeket
  elbíráló tanulásszervezők kijelölését a pedagógiai programnak vagy a
  szervezeti és működési szabályzatnak kell meghatároznia.}
megvizsgálják, és elismerik az évfolyamszinthez szükséges tantárgyi
követelmények teljesítését.\linebreak
Egy tantárgyból egy évfolyam teljesítettnek csak akkor tekinthető, ha
a tantárgyhoz tartozó tanulási eredmények 50\%-ának elérése a
portfólió\linebreak
alapján bizonyítható.

A kérelmet a gyerek digitálisan adja be. Az elbírálás csak a portfólió alapján történhet, ami egy online elérhető adatbázisként tartalmaz mindent, ami a döntéshez szükséges lehet. A döntéshez így a tanulásszervezőknek és a gyereknek nem kell egy időben és egy helyen lennie. Minden esetben szükséges a portfóliót és a teljes folyamatot digitálisan rögzíteni. Az iskolának tizenkét évig meg kell őriznie a portfóliót, a kérelmet és a döntéshez használt minden dokumentációt.

Amennyiben valamely gyereknél egy adott évfolyamszint
tantárgyi\linebreak
követelményei elismerésre kerülnek, akkor az iskola igazolja, hogy a gyerek az adott tantárgy vagy tantárgyak évfolyam szerinti követelményeit teljesítette. Erről igazolást állít ki, és teljesíti a jelentési kötelezettségét az Oktatási Hivatal felé.

Egyéni munkarendben tanuló vagy az órák látogatásáról valamilyen okkal felmentett gyerekek ugyanígy, a portfóliójuk összeállításával és a szintlépés kérelmezésével kérhetik az évfolyamszintek teljesítésének igazolását.

\subsection{Osztályzatokra váltás}
\label{sec:osztalyzatok}
A kerettanterv lehetővé teszi, hogy a gyerekek az érdemjegyek és osztályzatok helyett egy több szempontot figyelembe vevő szöveges vagy ér\-té\-ke\-lő\-táb\-lá\-zat- (rubric-) alapú visszajelzést kapjanak. A gazdag információtartalmú visszajelzések és portfólió osztályzatra való átváltására mégis szükség lehet, például iskolaváltás vagy továbbtanulás esetén.\footnote{Az Nkt. 54.§ (4) pontja alapján.}

Az átváltás \aref{sec:evfolyamszintlepes}.~fejezetben leírtakhoz hasonlóan is a portfólió értékelésén alapul. A gyerek (mentora és szülei segítségével) összeállítja a portfóliót, és bizonyítja, hogy a portfólió alapján megállapítható a kívánt osztályzat, az adott tárgyhoz az adott évfolyamszinten.

% atmasoltam a kerettantervbol, mert elkezdtem atirni

\section{Évfolyamok, bizonyítványok, jegyek, vizsgák}
A Budapest School gyerekek saját tanulási célokat tűznek ki, modulokat
választanak, tanulnak, alkotnak, trimeszterenként frissíti a portfóliójukat,
értékelik haladásukat, újraterveznek. És újra indul a tanulás ciklus.

Eközben a gyerekek a tanulás és alkotás eredményeként évfolyamszinteken
lépkednek fel,
első szintről a tizenkettedik szintig tantárgyanként.
Azt, hogy ez hogyan és mikor történik, azaz az évfolyamszintek elismerését -- a
kerettantervvel összhangban -- az iskola egy transzparens folyamata
szabályozza.

Hivatalos, tantárgyankénti érdemjegyet a gyerekek akkor és csak akkor kapnak,
ha erre iskolaváltás, továbbtanulás, ösztöndíj	    (vagy más külső rendszer)
miatt szükségük van.
Tehát osztályzatok, érdemjegyek és vizsgák nélkül is van lehetőség
évfolyamszinteket lépni.

A vizsga így teljesen átértékelődik a Budapest Schoolban.
Az évfolyamszintek elismeréséhez és (szükség esetén) az érdemjegyek
megállapításához
nem elégséges a pillanatnyi tudást vagy képességet felmérő eseményt szervezni,
hanem
a teljes portfóliót kell értékelni és figyelembe venni.
A portfólia sokkal gazdagabban dokumentálja, hogy egy gyerek mit csinált, mire
volt képes, mint egy szóbeli vagy írásbeli feladatsor: előzetes tudás-,
képességpróbák
mellett tartalmazza az alkotások, projektek, visszajelzések, versenyek, stb.
dokumentációt is.

\subsection{Évfolyamszintek}
\label{sec:evfolyamok}

A Budapest Schoolban az évfolyamokra úgy tekintünk, mint egy játék nehézség
szintjeire: akkor léphet egy tanuló a következőbe, ha teljesítette az
évfolyamhoz köthető tantárgyi követelményeket. Ezeket a kerettanterv
\emph{Tantárgyi eredménycélok} c. fejezete határozza meg.

A Budapest School évfolyamszintjei eltérnek az iskolák többségében alkalmazott
évfolyamtól (osztályoktól).

\begin{itemize}
    \item Egy gyerek minden tantárgyból állhat más szinten.
    \item Nem biztos, hogy az egy korcsoportba tartozók vannak ugyanazon az
          évfolyamszinten.
    \item Nem mindig az egy évfolyamszinten lévők tanulnak együtt, sokszor
          előfordulhat az is, hogy a különböző szinten lévő tanulók tudnak
          együtt és akár egymástól is tanulni.
    \item Egy év alatt több évfolyamszintet is lehet lépni.
\end{itemize}

Bár egy gyerek, minden tantárgyból állhat más szinten, a hivatalos (azaz külső
hivatalok, rendszerek számára értelmezhető) bizonyítványába mindig csak annak
az évfolyamnak
az elvégzése kerül be, amelyből mind a három tantárgyhoz szükséges fejlesztési
célt elérte.
Formálisabban kifejezve a tantárgyankénti évfolyamszintek minimumát kell a
bizonyítványban rögzíteni.

\subsubsection{Évfolyamszint-lépés}
\label{sec:evfolyamszintlepes}
A tanulók portfóliója alapján megállapítható, hogy egy adott évfolyamhoz
köthető tantárgyi követelményeknek megfelel-e.
Ehhez a tanulók elvégzik a mentoruk segítségével a portfóliójuk (mit csináltak,
mit tanultak, mit tudnak) összehasonlítását \ifkerettanterv
    \aref{sec:tantargyi_celok}.
    fejezetben
\else
    a kerettanterv \emph{Tantárgyi eredménycélok} c. fejezetében
\fi
felsorolt tantárgyi
eredménycélokkal.
Ha szükséges, akkor a kapcsolódás biztosításához a portfóliójukat
kiegészíthetik tudáspróbák, tesztek, szabványos vizsgák teljesítésével, melynek
megszervezése az adott mikroiskola tanárközösségének a feladata.

Miután a gyerek (mentora és szülei) segítségével összeállította a portfólióját,
jelzi az iskolának az évfolyamszint lépési kérelmét.

Ezt az iskola által kijelölt bírálók mevizsgálják és elismerik az
évfolyamszinthez szükséges tantárgyi követelmények teljesítését.
Egy tantárgyból egy évfolyam teljesítettnek tekinthető, ha a tantárgyhoz
tartozó tanulási célok 50\%-ának elérése a portfólió alapján bizonyítható.

A kérelmet a gyerek digitálisan adja be.
Az elbírálás csak a portfólió alapján történhet, ami egy
online elérhető adatbázisként tartalmaz mindent, ami a döntéshez szükséges
lehet. A döntéshez így a bírálóknak és a gyereknek nem
kell egy időben és egy helyen lennie. Minden esetben szükséges a portfóliót és
a teljes folyamatot digitálisan rögzíteni.
Az iskolának két évig meg kell őriznie az osztályozó vizsgák anyagát.

Magántanuló vagy az órák látogatásáról valamilyen okkal
felmentett gyerekek ugyanígy, a portfóliójuk összeállításával és a szint lépés
kérelmezésével kérhetik az évfolyamszintek teljesítésének igazolását.

Amennyiben valamely diáknál egy adott évfolyam tantárgyi követelményei
elismerésre kerülnek, akkor az iskola igazolja, hogy a diák az adott tantárgy
vagy tantárgyak évfolyam szerinti követelményeit teljesítette.
Erről igazolást állít ki, és teljesíti a jelentési kötelezettségét az Oktatási
Hivatal felé.

\subsection{Osztályzatokra váltás}
\label{sec:osztalyzatok}
A Budapest School kerettanterve lehetővé teszi, hogy a gyerekek az érdemjegyek
és osztályzatok
helyett egy több szempontot figyelembe vevő szöveges vagy értékelőtáblázat
(rubric) alapú viszajelzést kapjanak.
A gazdag információtartalmú visszajelzések és portfólió osztályzatra való
átváltására mégis szükség lehet, például iskolaváltás vagy továbbtanulás
esetén.\footnote{Azt az NKT. 54.§ (4) pontja alapján}

Az átváltás \aref{sec:evfolyamszintlepes}.
fejezetben leírtakhoz
hasonlóan is a portfólió értékelésén alapulnak.
A gyerek (mentora és szülei segítségével) összeállítja a portfóliót, és
bizonyítja, hogy a portfólió alapján megállapítható a kívánt osztályzat, az
adott tárgyhoz az adott évfolyamban.

\subsection{Kérelmek elbírálása}

A szülő és gyerek évfolyamszint-lépés vagy osztályzatra váltás kérelmét bírálók
értékelik. Minden esetben legalább három bíráló bírál egy kérelmet: a
bírálók közül egy a gyerek mentortanára, egy pedig mindenképp másik mikroiskola
tanulásszervezője. A bírálókat az iskola kinevezett felelőse választja ki.

Ha a bírálók közül akár egy is úgy véli, hogy az évfolyamszint-lépés vagy
osztályzat nem megalapozott, akkor
erről indoklásban értesítik a gyereket, szüleit és mentortanárát. Ilyenkor a
gyerek javíthata kérelmét és tetszőleges számú esetben ismételheti a
folyamatot.

Az évfolyamszint lépésről és az osztályzatra váltásról vagy épp a kérelmek
elutasításáról az iskola minden tanára értesítést kap. Ha a
nevelőtestület\footnote{NKT szóhasználat.}
nem ért egyet a bírálók döntésével, akkor új bírálók kinevezését kérhetik.

A kérelmek elbírálását 20 tanítási nap alatt el kell végezni.\footnote{Ha a
    kérelem nyári szűnetben érkezett, akkor auguszutus 31-ig.}

Évfolyamszint-lépést az iskola az utolsó évfolyamszint-lépéstől számított két
tanéven belül automatikusan indít. Így nem fordulhat elő, hogy egy gyerek
évfolyamszintjei két éven keresztül nincsenek felülvizsgálva.

Magántanuló és az órák látogatásáról valamilyen okkal felmentett gyerekek
évfolyamszint-lépését és szükség esetén osztályzat kérelmét pontosan ugyanazon
folyamatokkal kell elvégezni, mint a nem magántanuló és nem felmentett
gyerekét.

\section{A felvétel és az
  átvétel}

\todo[inline]{Mit kell irni a felvetelrol es atvetelrol?}

a felvétel és az átvétel - Nkt. keretei közötti - helyi szabályai

\section{A közösségi lét szabályai}
\label{sec:kozossegilet}
A Budapest School egy közösségi iskola: a gyerekek együtt tanulnak és alkotnak,
a tanárok
csapatokban dolgoznak és a szülők is egy elfogadó közösség részének érezhetiek
magukat. A tagok -- a tanárok, a gyerekek, a szülők,
és az adminisztrátorok -- azért csatlakoznak
a közösséghez, mert itt szeretnének lenni és itt érzik magukat boldognak,
egészségesnek és hasznosnak. A közösség azért fogad be új tagokat, hogy
nagyobb, erősebb közösség tudjunk lenni.

\paragraph{Alap csoportok} A Budapest School iskola kisebb mikroiskolák
hálózataként működik (lásd \aref{sec:mikroiskola} fejezetet).
Ez a gyerekek és tanárok elsődleges közössége: gyerekként ez az a közösség,
akikkel együtt járok iskolába, együtt alakítom az iskolámat,
tanárként ez az a csapat, akikkel együtt dolgozom, hogy létrehozzuk, tarsuk és
fejlesszük a mikroiskola gyerekeinek tanulási környezetét.

A tanulás egysége a modul, amiről \aref{sec:modulok} fejezet részletesen ír.
Egy modul csoportjában a közös érdeklődésű és célú gyerekek tanulnak együtt,
mert együtt boldogabban, hatékonyabban el tudják érni a céljukat.

\paragraph{Saját szabályok}

A mikroiskola és akár egy-egy modul kereteit,
szabályait a résztvevők alakítják ki. Pontosabban a tanárok
\footnote{Mikroiskola esetén a tanulásszervező tanárok, modulok
    esetén a modulvezető tanárok.} felelőssége és
feladata, hogy mindenki számára elofgadható, betartható, releváns és értelmes
szabályok
legyenek. Azaz minden mikroiskolának saját házirendje, szabályai
lehetnek, és modulonként alakulhat, hogy mit, mikor, hogyan és kivel csinálnak
a résztvevők.

Szabályokat úgy kell alakítani, hogy
hogy a gyerekek fejlődése, tanulása, alkotása és a közösség működése egyre jobb
legyen.

\subsection{Konfliktus, feszültségek kezelése}
Tudjuk, hogy a Budapest School szereplői, a gyerekek, a tanárok, a szülők,
a pedagógia program, a kerettanterv, a fenntartó, a szomszédok, az állami
hivatalok  között feszültségek és konfliktusok alakulhatnak ki, mert
különbözőek vagyunk, különbözőek az igényeink. A feszültségekre és a
konfliktusokra a Budapest School olyan lehetőségként tekint, amelyek
együttműködésen alapuló megoldása építi a kapcsolatot, és segíti a fejlődést.

Minden vágy, ötlet, szándék, cél, viselkedés közötti különbség, ha az
valamelyik
félben negatív érzéseket kelt, feszültséget és konfliktus okozhat. Ebbe
beletartozik az is, ha valaki nem
azt és úgy csinálja, ahogy nekünk erre szükségünk van, vagy ha bármilyen
okból nem érezzük magunkat biztonságban, vagy más univerzális emberei
szükségletünk \citep{rosenberg2003nonviolent} nem elégül ki.

Konfliktus alakulhat ki a gyerekek, szülők és tanárok között bármilyen
relációban és adódhatnak egyéb, belső konfliktusok, nehézségek is akár a
család, akár a Budapest School életében, amelyek kihathatnak a közösségi
kapcsolatainkra.

\subsubsection{A BPS konfliktusok feloldása}

Az iskola a partnerségen alapuló szervezetében nem autoritások, főnökök,
hivatalok, bírók oldják
meg a konfliktusokat, hanem
egyenrangú társak. Az iskola feltételezi, hogy a felek tudnak gondolkodni,
következtetni,
felelősséget vállalni a döntéseikért és cselekedetikért.\footnote{Kisgyerekeket
    mentoraik és szüleik reprezentálnak.}

\paragraph{Alapértékek}
Ahhoz, hogy tényleg partneri kapcsolatban, egyenrangú felekként tudjunk
konfliktusokat, problémákat megoldani, érdemes közös értékeket elfogadni.
\begin{itemize}
    \item Először a saját változásunkon dolgozunk, mert nem nagyon tudunk mást
          embert megváltoztatni, csak magunk változásáért
          lehetünk
          felelősek.
    \item Felelősséget vállalunk gondolatainkért, hiedelmeinkért, szavainkért
          és
          viselkedésünkért.
    \item Nem pletykálunk, szóbeszédet nem terjesztjük.
    \item Nem beszélünk ki embereket a hátuk mögött.
    \item Félreértéseket tisztázunk, konfliktusokat felszínre hozunk.
    \item Személyesen, 1-on-1 beszélünk meg problémákat, másokat nem húzunk be
          a
          problémába.
    \item Nem hibáztatunk másokat a problémákért. Amikor mégis, akkor az egy jó
          alkalom arról gondolkozni, hogy
          miként vagyunk mi is része a problémának, és kell a megoldás részévé
          vállnunk.
    \item Erősségekre több figyelmet fordítunk, mint a gyengeségekre, és a
          lehetőségekről, megoldásokról többet beszélünk, mint a problémákról.
\end{itemize}

\paragraph{Feszültségre felszínre hozása}
Olyan módszereket, folyamatokat, szabályokat, szokásokat kell kialakítani
minden közösségben, hogy legyen tere és ideje a feszültségeket előhozni.
\begin{itemize}
    \item Iskolai csoportok rendszeresen kezdik a napjukat egy bejelentkező
          körrel, ahol van lehetőség a feszültségeket is felhozni.
    \item A csoportok rendszeresen tartanak retrospektív gyűlést, ahol
          értékelik, mi volt jó és nem annyira jó egy vizsgált időszakban.
    \item Évente legalább kétszer a tanárok egymásnak, a szülők a tanároknak, a
          gyerekek a tanároknak, a tanárok a gyerekeknek visszajelzést adnak
          szervezett
          formában.
    \item A gyerekek a mentorukkal rendszeresen találkoznak, ahol teret kapnak
          a felmerülő feszültségek.
\end{itemize}

\paragraph{Megbeszélés}

A Budapest School közösségének valamennyi tagja, a tanárok, gyerekek,
szülők, adminisztrátorok, iskolát képviselő fenntartó vállalja:

\begin{itemize}

    \item
          A közösség mindennapjaival kapcsolatos konfliktusok esetén elsőként
          az
          abban érintett személynek jelez közvetlenül.
    \item
          Személyes kritikát mindig privát csatornán fogalmazza meg először, ha
          kell, akkor segítő bevonásával.
    \item
          Bármelyik fél jelzése esetén lehetőséget biztosít
          arra, hogy a vitás kérdést megbeszéljék közvetlenül, a folyamatban
          részt vesz.
    \item
          Szakítás, kilépés, lezárás előtt legalább három alkalommal megpróbál
          egyeztetni.
    \item
          Az egyeztetésre elegendő időt hagy, amely legalább 30 nap vagy --
          amennyiben több időre van szükség -- a másik féllel megállapodott
          idő.
    \item
          Teljes figyelemmel, nyitottsággal, a probléma megoldására fókuszálva
          igyekszik feloldani a konfliktust, és közösen megoldást találni a
          problémára.
\end{itemize}

Összefoglalva: ha problémánk van egymással, akkor azt megbeszéljük. Nem
okozunk egymásnak meglepetést, mert vállaljuk, hogy rögtön elmondjuk
egymásnak konfliktusainkat.

\paragraph{Közvetítő bevonása}

Ha úgy érezzük, hogy a személyes egyeztetés nem vezetett megoldásra, a
tárgyalást külső segítség bevonásával folytatjuk. Ez lehet egy másik
csoporttag, egy tanár, vagy egy teljesen külsős meditátor.

Amikor bármely fél közvetítőt kér, akkor a másik ezt elfogadja. Nem mondhatjuk
azt, hogy  \emph{,,de hát mi magunk is meg tudjuk oldani a konfliktust''}.

\paragraph{Eszkalálás}

Ha két fél nem tudja megoldani a konfliktust, akkor kérhetnek segítséget a
\texttt{segitseg@budapestschool.org}
címen, amire 48 órán belül kell válaszolnia. Az email kezeléséért az iskola
igazgató felel.

\paragraph{Megállapodások}

Ha az egyeztetés, tárgyalás és közvetítő bevonása során sikerül
valamilyen megoldást vagy a megoldáshoz vezető folyamatot egyeztetni, a
felek \emph{megállapodnak} abban, hogy ki mit tesz, vagy milyen szabályokat
alakítanak
ki, illetve, hogy mennyi időt adnak egymásnak, hogy kipróbálják, sikerült-e
feloldani a konfliktust.
Segíteni szokott a kérdés, hogy \emph{most megállapodtunk vagy csak beszéltünk
    róla?}, hogy minden fél számára világos legyen a megállapodás.

Nagyobb konfliktusok esetén jó gyakorlat, és bármely fél kérheti, hogy
írásban is rögzítsék a megállapodást. Ez lehet egy papirfecni is, vagy egy
email. Lényege nem a formátum, hanem  hogy minden fél emlékezzen a
megállapodásra.

\paragraph{Egyeztetés sikertelen}

Ha a felek között az egyeztetés sikertelen volt, vagy a megoldási
javaslat nem működött, felek ezt írásban meg kell, hogy állapítsák. Erre
azért van szükség, hogy egyetértés legyen abban közöttük, hogy értik, a
másik fél sikertelennek érzi az egyeztetést.

\paragraph{Lezárás}

Az egyeztetés sikertelensége esetén elengedjük egymást. De ez a végső
megoldás.

\subsection{Kiemelt konfliktusok}

\subsubsection{Gyerek-gyerek
    konfliktus}

A mindennapokban a gyerekek akarva akaratlanul belecsöppennek olyan
helyzetekbe, amikor a közösségben vagy interperszonális kapcsolataik során
megborul a mindennapi egyensúly. Ezekben a konfliktushelyzetekben leginkább a
felborult egyensúly helyreállítására törekszünk \emph{resztoratív
    konfliktusfeloldási
    technikával}.

A resztoratív konfliktusfeloldás alaptézise az, hogy minden ilyen megborult
egyensúlyi állapot egy lehetőség valami megújítására, újragondolására. A
résztvevők egy külső személy segítségével (általában a jelenlévő tanár) együtt
alakítják
a megoldást, egészen addig, amíg az eredmény mindenki számára a konfliktus
feloldását jelenti, azaz a megborult egyensúly helyreállítását.

A folyamat során a konfliktusban résztvevő összes személy elmondja az érzéseit,
meglátását a felmerülő helyzettel kapcsolatban, valamint az én-közléseken túl a
szükségleteikről is beszélnek. Ezen szükségletek képezik a megoldás alapját,
azaz ezeket egy tető alá hozva feloldhatjuk a fennálló konfliktust. Ilyenkor
mindig először azokat a pontokat keressük meg, amelyekben egyetértenek a
résztvevők, hiszen ez egy közös alapot szolgáltat arra, hogy a valódi feloldást
megtalálhassuk.

Fontos, hogy a beszélgetésben az összes fél hallassa a hangját és meg is legyen
hallgatva. Az értő figyelem kompetenciája is fejlődik ezen módszer alkalmazása
során, például, ha az érintett felek elmondják, hogy mit hallottak meg abból,
amit a másik elmondott.

\subsubsection{Gyerek-iskola, tanár-szülő
    konfliktus}\label{gyerek-iskola-tanuxe1r-szuxfclux151-konfliktus}

Minden gyereknek van egy mentora. A szülők számára a mentor az
elsődleges kapocs az iskola felé. Ezért, ha a szülőben jelenik meg egy
feszültség, akkor elsődlegeses a mentornak jelez. Ugyanígy, a mentor
közvetíti a család felé a gyerekkel kapcsolat feszültségeket.

Ha egy gyerek, vagy szülő úgy érzi, hogy egy gyereknek nem jó az iskolai
élménye, például nem tanul eleget, vagy kiközösítik, vagy csak nem
szeret bemenni, akkor feladata, hogy rögtön beszéljen a mentor tanárral.

Ha nem sikerül a mentorral megbeszélni a konfliktust és megoldást találni,
akkor a szülőnek is lehetősége van segítőt behívni, aki lehet egy másik tanár,
másik szülő, az iskolaigazgató vagy bárki, akinek képességeiben bízik.

Előfordulhat, hogy a tanárok vagy az iskola úgy érzik, hogy egy
gyereknek nem tesz jót a Budapest School közössége, vagy a hozzáállása
súlyosan zavarja vagy sérti a Budapest School közösséget vagy azok
tagjait. Az is lehet, hogy a tanárok vagy a Budapest School a szülővel
való kapcsolatot érzik konfliktus vagy feszültség forrásának. Ilyen
esetben ugyanígy le kell folytatni a konfliktuskezelés folyamatát és
megpróbálni feloldani a feszültséget. Ennek sikertelensége esetén az
iskola jelzi a családnak írásban, hogy el fog válni.

\subsubsection{Pedagógia program, kerettanterv be nem tartásával
    kapcsolatos konfliktusok}

Amikor egy tanár, egy gyerek vagy egy mikroiskola nem tartja be a
kerettantervet, a pedagógia programot, a házirendet, vagy egyéb közösen
megalkotott szabályokat, akkor konfliktus alakul ki közte és az iskola
között. Ilyenkor szintén a konfliktuskezelés folyamatát kell
lefolytatni.

\subsubsection{Évfolyamszint-lépéssel, osztályzatra való átváltás konfliktusai}
Szülő-iskola konfliktusainak nagy része sok iskolában az osztályzatokkal és a
bizonyítvánnyal kapcsolatos.
A BUdapest School iskolában a gyerekek maguk kérelmezek az évfolyamszint-lépés
és szükség esetén az osztályzatra való átváltást. Maguk tesznek javaslatot
arra, hogy mikor és hány évfolyamot lépjenek és hogy milyen osztályzat kerüljön
a bionyítványba. A bírálók ezt efogadják vagy elitasítják. Amikor a
gyerek/szülő kérelmét a bírálók elutasították, akkor konfliktus alakul ki.
Ilyenkor is az előbbeikben leírt elveket, folyamatokat kell alkalmazni.

\section{Néhány bevált tanulásszervezési technika}
\todo[inline]{ezt a fejezetet csak azert csinaltam, mert kivagtam a reggeli
  kort az elejerol, mert oda nem illet.
  ha van meg mas, amit ide berakhatunk, akkor szoljatok. Esetleg a
  kezikonyvbol?}
\paragraph{Reggeli körök}
A reggeli
beszélgetőkör minden iskola napinditásának alapeleme. Ez a tere annak,
hogy a mikroiskolák tagjai megbeszélhessék közös ügyeiket a gyerekek és tanárok
beszámolhassanak a velük történt eseményekről, és az őket éppen foglalkoztató
témákról.
