\chapter{Az iskola helyi tanterve}
\label{az-iskola-helyi-tanterve}

\section{A választott kerettanterv
megnevezése}
\label{a-vuxe1lasztott-kerettanterv-megnevezuxe9se}

Budapest School Általános Iskola és Gimnázium az oktatásért felelős
miniszter által jóváhagyott Budapest School Kerettanterv alapján
működik. Az iskola munkatársai a kerettantervet, a pedagógia programot,
sőt még a szervezeti és működési szabályokat is együtt, egy egységként
állították össze. Ez a Budapest School Módszer, aminek fő célja, hogy a
gyerekek úgy tudják azt és akkor tanulni, amit szeretnek, vagy amire
szükségük van, hogy közben a tanárok, szülők számára is kiszámítható,
tervezhető, biztonságos tanulási környezetet biztosít az iskola.

Az iskola személyre szabható tanulási környezetet biztosít. A
tanulásszervező tanárok, az iskola sok választási lehetőséget kínál a
gyerekek számára, akik a lehetőségekből összeválogatják a saját tanulási
céljaikhoz leginkább illeszkedő saját tanulási programjukat. A
választásban -- az önvezérelt tanulás menedzselésében -- a gyerekek
folyamatos egyéni támogatást kapnak a \emph{mentor} tanáruktól.

A tanulási célokat, a választásokat, az igényeket és az elvárásokat
trimeszterenként a gyerek, a tanár és a szülő egy tanulási szerződésben
rögzíti. Erre mintát ad XX melléklet.

\begin{quote}
Példa: egy 7 éves gyerek, Elemér szeretne önállóbban olvasni, és ennek a
szülők és tanárok is örülnek. Megállapodnak, hogy a cél, hogy a
trimeszter végére a gyermek képes legyen egy minimum 20 oldalas könyvet
önállóan elolvasni. Ezért a mentor vállalja, hogy társaival együtt
olvasóklubot szervez, amikor hetente minimum 30 percet egyénileg
olvasnak a gyerekek. A szülők vállalják, hogy megvásárolják vagy
könyvtárból kikölcsönöznek Elemérnek 3 könyvet a trimeszter során.
Elemér pedig vállalja, hogy hetente 30 percet önállóan olvas.
\end{quote}

\section{Modulok és a tantárgyak}
\marginpar{ez kell ide extran?}
A Budapest School Kerettanterv a három tantárgyat határoz meg: STEM,
KULT és Harmónia tantárgyakkal fedi le a Nemzeti Alaptanterv által
meghatározott tartalmi követelményeket. A gyerekek órarendjében azonban
nem ezek a tantárgyak jelennek meg, hanem specifikusabb, rövidebb,
célorientáltabb egységek: a modulok. A Budapest School modulok
különböznek a megszokott tantárgyi tanóráktól. Abban azonban
hasonlítanak, hogy a moduloknak is mindig van előre meghatározott
fejlesztési célja, azaz, hogy a modul során mit tanul, miben fejlődik
várhatóan a résztvevő gyerek.

\paragraph{A modulok rövidebbek.}

A tantárgyak céljait, folyamatai általában minimum egy tanévre
határozzák meg. A miniszter által kiadott kerettantervekben két éves
egységekben gondolkoznak a szerzők. A Budapest School moduljai lehetnek
egy hetesek is, vagy legfeljebb egy trimeszter hosszúak. A lezárását
követően egy tanulási folyamat egy új modulban tovább folytatódhat.

\paragraph{Tervezettség}\label{tervezettsuxe9g}

A tantárgyi rendszerek a kimeneti követelmények alapján előre
megtervezettek. A Budapest School a gyerekek céljai és érdeklődése
alapján, akár az igény megjelenése utáni héten is indulhatnak.
Természetesen nem kell mindig kitalálni a spanyol viaszt. Ha egy modul
bevált, akkor lehet egy másik gyerekcsoporttal is kipróbálni. De mindig
meg kell tartanunk annak a lehetőségét, hogy más gyerekekkel, más
tanárral, más helyen ugyanaz a modul is másképp sikerül.

\paragraph{Tantárgyközi tudás}\label{tantuxe1rgykuxf6zi-tuduxe1s}

A modulok alapértelmezett tulajdonsága, hogy több tantárgy tartalmából
építkeznek, hisz \emph{minden, mindennel} összefügg. Így minden modul
lefedhet egy, kettő vagy akár három tantárgy tartalmát. A
kerettantervben meghatározott kiegyensúlyozottság elve alapján a
tanároknak úgy kell modult hirdetniük és gyerekeknek választaniuk, hogy
minden tantárgy tartalmával közel egyenlő mértékben találkozzanak.

\paragraph{Modulindítás}\label{modulinduxedtuxe1s}

Modulokként jelennek meg a foglalkozások, a tanórák, a projektek, a
fakultációk, szóval minden olyan tevékenységek, amit a gyerekek az
iskolában csinálhatnak. Az a tanulásszervező tanárok feladata, hogy a
gyerekek minden trimeszter előtt legalább két héttel megismerhessék a
következő trimeszter választható moduljait.

Modul indulásakor mindig elkészül egy modulkiírás, ami tartalmazza a
modul célját, azaz, hogy \emph{miért} csináljuk, hogy \emph{mit} fognak
a gyerekek csinálni a modul során, a tervezett tantárgyi
eredménycélokat, \emph{mikor} és mennyi ideig tart a modul, ha érdekes,
akkor a módszert, azaz \emph{hogyan szervezzük a tanulást}, és a modul
során tapasztalható fejlődés érteklési szempontjait. Példa

\begin{itemize}

\item
  Néptánc Sárival

  \begin{itemize}
  \item
    Cél: közösen táncoljunk, hogy többet mozogjunk és zenei érzéket
    fejlesszük
  \item
    Mit: 8 alkalom, hetente 1 órában 2 népdalra tanuljuk be a
    koreográfiát
  \item
    Tantárgyi célok: Harmnónia, BLA BLA
  \item
    Értékelési szempontok: együtt mozgott társaival, energikusan
    mozgott, követte a zenét, megfelelő ruhában érkezett
  \end{itemize}
\end{itemize}

\section{Óraszámok}\label{uxf3raszuxe1mok}

\begin{quote}
A választott kerettanterv által meghatározott óraszám feletti kötelező
tanórai foglalkozások, továbbá a kerettantervben meghatározottakon felül
a nem kötelező tanórai foglalkozások megtanítandó és elsajátítandó
tananyaga, az ehhez szükséges kötelező, kötelezően választandó vagy
szabadon választható tanórai foglalkozások megnevezése, óraszáma
\end{quote}

Minden gyereknek úgy kell tanulási célokat és tanulási utat, azaz
modulokat választani, hogy három egymásutáni trimeszter óraszámai
alapján mind a három tantárggyal közel egyenlő mértékben foglalkozzon.

A gyerekek trimeszterenként a mentortanáruk segítségével és a szüleikkel
egyetértésben választanak modulokat, így nem tudjuk előre meghatározni,
hogy mit, mekkora óraszámban tanul. Azt azonban tudjuk figyelni, hogy a
kerettanterv kiegyensúlyozottság elve teljesül-e. Modulonként
kiszámolható, hogy egy-egy gyerek egy tantárggyal hány órát
foglalkozott. Ezeket összegezve ki tudjuk számolni, hogy egy gyerek hány
órát fektetett be egy-egy tantárgy tartalmának megismerésébe. Ezek
alapján megállapítható, hogy egyensúlyban vannak-e a gyerek tantárgyi
élményei.

\section{Tankönyvek
kiválasztása}\label{tankuxf6nyvek-kivuxe1lasztuxe1sa}

\begin{quote}
Az oktatásban alkalmazható tankönyvek, tanulmányi segédletek és
taneszközök kiválasztásának elvei (figyelembe véve a tankönyv
térítésmentes igénybevétele biztosításának kötelezettségét).
\end{quote}

Modulvezetők minden esetben maguk választják a modulhoz szükséges
tankönyvek, szoftverek, weboldalak és egyéb eszközöket úgy, hogy

\begin{itemize}

\item
  az a megfelelő legyen annak a csoportnak, ahhoz a célhoz, amit el akar
  érni
\item
  minden esetben legyen mindenki számára elérhető (esetek többségeben
  értsd ingyenes) megoldás
\item
  modulvezetők bátorítva vannak arra, hogy új dolgokat próbáljanak ki,
  és tapasztalataikat az iskola többi tanárával megosszák.
\end{itemize}

Mivel a Budapest School kerettantervének értelmében az egyéni célok
legalább 50\%-át az állami kerettantervben meghatározott fejlesztési
célok közül kell választani, az ehhez szükséges ismeretek megszerzéséhez
a Budapest School az Oktatási Hivatal általi jegyzékben államilag
támogatott OFI által fejlesztett tankönyveket veszi alapul. A Budapest
School tanárcsapatának lehetősége van arra, hogy ettől eltérő, a
mindenkori tankönyvjegyzékben szereplő tankönyvvel segítse el a
kerettantervben meghatározott eredménycélok elérését. És arra is
lehetősége van, hogy egyátalán ne használjon tankönyvet, mert sokszor az
internet elegendő információt tartalmaz.

A Budapest School pedagógiai programjának alapja, hogy a gyerekek egyéni
céljaira szabott tanulási terveket készít. Ennek előfeltétele, hogy a
könyvek használata is ehhez kapcsolódó módon rugalmasan történjen,
minden esetben az adott tanulási modul igényeihez szabva. Ennek
érdekében a program pedagógusai folyamatosam állítják össze a gyerekek
eltérő céljaihoz és képességszintjeihez igazodó differenciált
tevékenységek és feladatsorok rendszerét.

\section{NAT pedagógiai
feladatok}\label{nat-pedaguxf3giai-feladatok}

\begin{quote}
a Nemzeti alaptantervben meghatározott pedagógiai feladatok helyi
megvalósításának részletes szabályai
\end{quote}

\texttt{EZT\ KI\ KELL\ KERESNI}
\texttt{ide\ csak\ azt\ kell\ irni,\ h\ a\ tantargyak\ ezt\ elintezik}

\section{Mindennapos testnevelés}\label{mindennapos-testneveluxe9s}

\begin{quote}
a mindennapos testnevelés, testmozgás megvalósításának módja, ha azt nem
az Nkt. 27. § (11) bekezdésében* meghatározottak szerint szervezik meg,
Az iskola az Nkt. 27. § (11) bekezdésben meghatározottak szerint
szervezi meg, hogy mindennap alkalma legyen a gyerekeknek a
testnevelésre, testmozgásra. Mivel a Budapest School kerettanterv nem
határoz meg külön testnevelés tantárgyat, így minden modul annak
tekintendő, aminek célja a mozgás, az egészséges életmód és a Harmónia
tantárgy témáját (is) feldolgozza.
\end{quote}

\section{Válaszható érettségi
tárgyak}\label{vuxe1laszhatuxf3-uxe9rettsuxe9gi-tuxe1rgyak}

\begin{quote}
középiskola esetén azon választható érettségi vizsgatárgyak megnevezése,
amelyekből a középiskola tanulóinak közép- vagy emelt szintű érettségi
vizsgára való felkészítését az iskola kötelezően vállalja, továbbá annak
meghatározáse, hogy a tanulók milyen helyi tantervi követelmények
teljesítése mellett melyik választható érettségi vizsgatárgyból tehetnek
érettségi vizsgát,
\end{quote}

Az iskola a kötelező középszintű érettségi vizsgatárgyakra való
felkeszítést kötelezően vállalja érettségi felkészítő modulok
szervezésével. (Más iskolákban ezt fakultációnak hívnák, de a Budapest
School iskola nem különbözteti meg a fakultációt a tanórától.) A
választható tantárgyak és az emeltszintű érettségi vizsgára csak akkor
szervez egy mikroiskola modult, ha arra legalább a közösség 10\%-a
igényt tart.

\subsection{középszintű érettségi vizsga
témakörei}\label{kuxf6zuxe9pszintux171-uxe9rettsuxe9gi-vizsga-tuxe9makuxf6rei}

\begin{quote}
középiskola esetén az egyes érettségi vizsgatárgyakból a középszintű
érettségi vizsga témakörei
\end{quote}

\texttt{EZ\ HP\ SZERINT\ LEHET\ HOGY\ NEM\ KELL}

\section{Értékelés,
minősítés}\label{uxe9rtuxe9keluxe9s-minux151suxedtuxe9s}

\begin{quote}
a tanuló tanulmányi munkájának írásban, szóban vagy gyakorlatban történő
ellenőrzési és értékelési módját, diagnosztikus, szummatív, fejlesztő
formáit, valamint a magatartás és szorgalom minősítésének elvei
\end{quote}

A Budapest School iskolában a felnőttek a gyerekek munkáját,
erőfeszítését, részvételét a Budapest School kerettantervének 3.4
fejezetében leírt módon értékeli. Hasonlóan a gyerekek is folyamatosan
értékelik a tanárok munkáját.

Minden modul zárásakor a részvételt, a haladást, a fejlődést és az elért
eredményeket is értekelik a gyerekek és tanárok. Trimeszter zárásakor a
mentorok és mentorált gyerekek értekelik a trimeszter tanulási
eredményeit, és hogy hogy van a gyerek az iskolában, a céljaival, a
közösséggel.

\subsection{Portfólió}\label{portfuxf3liuxf3}

Minden projekt eredménye, minden visszajelzés, minden tudáspróba, szóval
a gyerekek iskolai munkásságának az eredménye a \emph{portfólióba}
kerül. A potfólió alapján megállapítható, hogy a gyerek mit csinált,
amiből következtethetünk arra, hogy mit tud, milyen képességei,
készségei, kompetenciái vannak.

Portfólió fejlesztése, rendbentartása elsősorban a gyerekek feladata. A
tanárok és szülők pedig elsődlegesen arra figyelnek, hogy a gyerekek
megtanulják: nem elég valamit megcsinálni, megtanulni, fontos, hogy
megtanuljuk megmutatni is magukat.

IDE EGY MONDAT: a mentorok meg azert felelosek, hogy a gyerekekek
megcsinaljak.

\subsection{Évfolyamok elismerése, a
bizonyítvány}\label{uxe9vfolyamok-elismeruxe9se-a-bizonyuxedtvuxe1ny}

Ahogy a Budapest School Kerettanterv 4.5 fejezete is meghatározza, a
gyerekek mind a három tantárgyból más évfolyamszinten állhatnak.
Trimeszterenként a gyerekek, szülők kérhetik, hogy a portfóliójuk
alapján évfolyamszintet léphessenek egy vagy több tantárgyból.

Ennek első lépése, hogy a portfólióba bekerüljenek olyan elemek, amik
bizonyítják, hogy a tantárgyi követelményeknek legalább a felét
teljesítették. Ezután a gyerek build a case to level up. Fontos, hogy a
teljes dokumentációnak digitálisan, online elérhetőnek kell lennie úgy,
hogy távolról megítélhető legyen XXX

\subsection{Osztályzatok, vizsga}\label{osztuxe1lyzatok-vizsga}

Ha iskolaváltás vagy továbbtanulás miatt a szülő vagy gyerek kéri, hogy
gyereket félévkor és/vagy év végén osztályzattal minősítsük, és a
portfólia értékelésünk alapján megajánlott osztályzatot nem fogadja el,
a tanuló osztályozó vizsgát tehet. Ebben az esetben a szülő osztályozó
vizsga iránti kérelmet nyújthat be az mentortanárnál írásban is
megerősítve. A tanév alatt beadott írásbeli kérelem beadását követően az
iskolának 20 tanítási nap áll a rendelkezésére a vizsga megszervezésére.
(Ez alatt a tanév elején a tanév rendjében kiírt tanítási napok
értendőek. A tanév elején tanítási szünetnek kiírt napok nem számítanak
tanítási napnak!) Amennyiben a szülő a nyári szünet ideje alatt
kérelmezi az osztályozó vizsga megtartását, úgy azt legkésőbb június
30-ig jeleznie kell (írásban is). Az ilyen esetekben az osztályozó
vizsga augusztus utolsó hetében kerül megtartásra.

\subsubsection{Vizsgák tartalma}
A VIZSGA nem tudáspróba, hanem a portfólió alapján védés.
TODO, FONTOS

\todo[inline]{itt azt kell leírni, hogy a vizsga nem más, mint hogy hozza a portfólióját, és az alapján elmondja, hogy miért kell abban az évfolyamban azt a jegyet kapnia. Az portfolionak legyen lehetoleg digitalis valtozata.}
az egesz vizsa tortenhet online es aszinkron.



\section{A csoportbontások}\label{a-csoportbontuxe1sok}

\begin{quote}
a csoportbontások és az egyéb foglalkozások szervezésének elvei
\end{quote}

A Budapest School iskoláit mikroiskolák közösségeiből hozzuk létre. Így
egy gyerek elsődleges csoportja a mikroiskolájának közzéssége, ami lehet
12 - 50 gyerek. Ezen belül modulonként eltér, hogy milyen
csoportbontásban dolgoznak. Több szinje van a csoportmunkának. 1. A
mikroiskola közössége heti rendszerességgel tarthat iskolagyűlést,
fórumot, HOGY HIVJAK A BREZNOBAN?. Ilyenkor a mikro iskola közössége
együtt van. 2. Modulokra kisebb csoportok jelentkezhetnek. Egy modul
csoportjának rendező elve lehet, hogy 1. egy képességszinten lévő
gyerekek tanuljanak együtt; 2. vagy közös érdeklődés hozza össze a
csoporttagokat; 3. van, hogy direkt a véletlenszerűségben van az
érdekesség, mert keveredni akarunk; 4. kölcsönös szimpátia és vonzalom
lehet a modultagok között: most azért vannak egy csoportban, mert egy
csoportban akartak lenni. 3. Egy-egy foglalkozáson belül is sokszor
csoportot alkotunk, az előző elvek alapján.

\section{Nemzetiségek
megismerése}\label{nemzetisuxe9gek-megismeruxe9se}

\begin{quote}
a nemzetiséghez nem tartozó tanulók részére a településen élő nemzetiség
kultúrájának megismerését szolgáló tananyag,
\end{quote}

\section{az egészségnevelési és környezeti nevelési
elvek}\label{az-eguxe9szsuxe9gneveluxe9si-uxe9s-kuxf6rnyezeti-neveluxe9si-elvek}

\begin{quote}
az egészségnevelési és környezeti nevelési elvek
\end{quote}

\section{Esélyegyenlőség}\label{esuxe9lyegyenlux151suxe9g}

\begin{quote}
a gyermekek, tanulók esélyegyenlőségét szolgáló intézkedések
\end{quote}

\section{Jutalmazás és
értékelés}\label{jutalmazuxe1s-uxe9s-uxe9rtuxe9keluxe9s}

\begin{quote}
a tanuló jutalmazásával összefüggő, a tanuló magatartásának,
szorgalmának értékeléséhez, minősítéséhez kapcsolódó elvek
\end{quote}

\todo[inline]{ legyen growth mindsetet, praise the effort, ugy mint Ha adunk,
akkor adjunk pozitív visszajelzést, fókuszáljunk arra, amit csinált a
gyerek és nem arra, amit elért. jutalmazas bevezeto}

\begin{itemize}

\item
  Fantasztikus, ma egy nagy kihívást választottál!
\item
  Bátran rizikót vállaltál!
\item
  Nagyon jó! Tényleg sokat próbálgattad.
\item
  Kitartóan csináltad, erre nagyon büszke vagyok!
\item
  De jó, valami újat próbáltál ki ma!
\item
  Köszönöm, hogy ma valakinek segítettél.
\item
  Nagyon nagy öröm látni a haladásodat!
\item
  Ne feledd, mindannyian tudunk a hibáinkból tanulni. Örüljünk annak,
  hogy ma valamit jobban tudunk, mint előtte.
\item
  Wow, egy nehéz feladatot oldottál meg!
\item
  Szép munka! Kipróbáltál egy másik módszert.
\end{itemize}

\section{Önkéntes közösségi
szolgálat}\label{uxf6nkuxe9ntes-kuxf6zuxf6ssuxe9gi-szolguxe1lat}

Az Nkt 4. § (15) pontjában definiált közösségi szolgálat is modulként
kerül meghírdetésre, amit 12. évét betöltött gyerek választhat csak.
Közösségi szolgálatként elfogadható, ha a Budapest School iskola egy
másik mikroiskolájában segít a gyerek.
