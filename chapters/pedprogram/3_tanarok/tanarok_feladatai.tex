\section{A pedagógusok feladatai}\label{sec:tanarfeladatok}

A Budapest School szervezetben a felelősségek és feladatok körét egy-egy \emph{szerep} formájában határozza meg a csapat. A szerep által kijelölt területen (domain) belül önállósága van a szerepet betöltő (role keeper) csapattagnak. Egy-egy munkatárs több szerepben is lehet, és a szerepeket és a szerepkiosztást úgy kell változtatni, hogy azok a szervezet céljait, jelen esetben a gyerekek tanulását jobban szolgálják.

Például a mikroiskola teendőinek koordinálása, az iskolai adminisztráció is egy szerep, nem egy önálló munkakör.  Mentortanárnak lenni, az egy másik szerep, és az is keverhető más szereppel, így egy iskolai adminisztrátor is lehet men\-tor, vagy másképp fogalmazva, egy mentortanár kaphat adminisztrációs szerepet is.



\subsection{Tanulásszervező}\label{sec:tanulasszervezo}
\begin{itemize}
    \item Kiszámítható, átlátható rendszert épít, ahol a szülők és a gyerekek is biztonságban, informálva, bevonva érzik magukat.
    \item A tanulási célokkal rezonáló modulokat hirdet meg, szervez le.
    \item A szülők és a gyerekek is érzik, értik, hogy ,,történik'' a tanulás.
    \item A modulvezetőknek megad minden szükséges információt, kontextust, így ők hatékonyan tudják végezni a munkájukat.
    \item Megadja a résztvevőknek az informált választás lehetőségét.
\end{itemize}

\subsection{Mentor}\label{mentor}

\begin{itemize}
    \item
          Az első trimeszter alatt a mentor megismeri mentoráltját, személyiségjegyeit, képességeit, érdeklődését, motivációit. Megismeri a családot.
    \item
          Megállapodik a családdal a kapcsolattartás szabályaiban.
    \item
          Bevezeti a családot a Budapest School rendszerébe.
    \item
          Trimeszterenként a mentorált gyerekkel és szülőkkel egyetértésben kialakítja, majd folyamatosan monitorozza a mentorált gyerek tanulási céljait.
    \item
          Képviseli mentoráltját a többi tanár felé.
    \item
          A szülőkkel rendszeresen információt oszt meg, elérhető, asszertíven kommunikál, tiszta, mérhető megállapodásokat köt.
    \item
          A szülőkkel erős partneri kapcsolatot épít.
    \item
          Amikor a gyereknek külső fejlesztésre, mentorra, tanárra, trénerre van szüksége, akkor a családot segíti a megfelelő segítő felkeresésében, a külsőssel kapcsolatot tart, és konzultál a mentoráltja haladásáról.
    \item
          Minimum kéthetente találkozik mentoráltjával. Követi, tudja, hogy a gyerek hogy van az iskolában, a családban, az életben.
    \item
          Mentorként nyomon követi, monitorozza a mentoráltjai fejlődését, és szükség esetén továbblendíti, inspirálja őket.
    \item
          A mentor biztosítja, hogy a mentorált gyerek portfóliója friss legyen.
    \item
          Rendszeresen reflektál a mentorált gyerekkel együtt annak tanulási céljaira és haladására.
    \item
          Segíti a mentorált modulválasztását.
    \item
          Visszajelzést gyűjt és ad a mentorált gyerek fejlődésére.
\end{itemize}

\subsection{Modulvezetők}\label{modulvezetux151k}

\begin{itemize}

    \item
          Izgalmas, érdekes foglalkozásokat tart, amire felkészül, és amiben a gyerekeket flow-ban tudja tartani.
    \item
          Amikor a gyerekek vele vannak, akkor figyelnek, fókuszálnak, koncentrálnak, dolgoznak, tanulnak.
    \item
          Kedvesen és határozottan vezeti a csoportot, figyel arra, hogy mindenkit bevonjon.
    \item Változatos, gazdag módszertani eszköztárából mindig a foglalkozáshoz megfelelő módszert tudja elővenni.
    \item  A gyerekek kritikai gondolkozását foglalkozásai fejlesztik.
    \item A gyerekek sokat dolgoznak csoportban foglalkozásain.
    
\end{itemize}

\subsection{A közös szerepek}

Minden BPS-tanulásszervező, -modulvezető és -mentor egyszerre  \emph{Csapattag}, \emph{EQ ninja}, \emph{Change Agent} és \emph{SNI-szakértő}. 

\paragraph{Csapattag}

\begin{itemize}

    \item
          Rendszeresen jelen van a tanári csapat megbeszélésein.
    \item
          El lehet érni telefonon vagy online, a csapattal megállapodott kereteken belül.
    \item
          Feladatokat vállal magára, és azokat megbízhatóan, határidőre végrehajtja.
    \item
          Kooperatív és támogató a közös munkák, ötletelések, megbeszélések alatt, képviseli a saját nézőpontját, gondolatait, érzéseit, miközben a csapat és a többiek igényeire is figyel.
    \item
          Kifejezi támogatását, ellenérveit és javaslatait a jobb megoldás érde\-kében.
    \item
          A visszajelzést keresi, a kritikát jól fogadja, és megfontolja, átgondolja a lehetséges változtatásokat.
    \item
          A kollégák fejlődését segíti rendszeres visszajelzésekkel.
\end{itemize}

\paragraph{BPS-tag}

\begin{itemize}

    \item
          Rendszeresen jelen van heti hétfőkön, havi szombatokon.
    \item
          Részt vesz a mikroiskolájának és a Budapest Schoolnak az építésében.
    \item
          Proaktívan alakít ki rendszereket, folyamatokat, és a legjobb gyakorlatokat megosztja BPS-szinten.
    \item
          Közös témákban aktív, hozzászól, alakítja a véleményével és tudásával a BPS rendszerét.
\end{itemize}

\paragraph{Érzelmi intelligencia (EQ) ninja}

\begin{itemize}

    \item
          Gyerekek pszichológiai biztonságérzetét megerősíti.
    \item
          Olyan visszajelzéseket ad, amelyek az erőfeszítésekre, a belefektetett energiára, munkára és a jövőbeni fejlődésre fókuszálnak (growth mindset), pozitív megerősítést alkalmaz, épít a gyerek erősségeire, és egyértelműen megfogalmazza, mit tehetne másként.
    \item
          Értő figyelemmel van jelen, kedves, nyitott, és figyel a gyerekekre.
    \item Munkája során gondoskodik a gyermek személyiségének, képességeinek, kompetenciájának fejlődéséről.
    \item Figyelembe veszi a gyerekek egyéni képességeit, adottságait, fejlődésének ütemét, szociokulturális helyzetét.
    \item Előmozdítja a gyerek erkölcsi fejlődését.
    \item Segíti a gyerekeket jobban működni a csoportban.

\end{itemize}

\paragraph{SNI-szakértő}

\begin{itemize}

    \item
          Jól tudja kezelni az egyéni, különleges bánásmódot, speciális nevelést igénylő gyerekeket a csoportban.
    \item A különleges bánásmódot igénylő gyermekek családjával, a nevelést, oktatást segítő más szakemberekkel együttműködik, hogy a gyerek támogatása minden esetben megtörténjen.
    \item
          Akinek külső segítségre van szüksége, azoknak a szüleivel ezt proaktívan egyezteti, és menedzseli a folyamatot.
\end{itemize}

\paragraph{Change agent}\label{change-agent}

\begin{itemize}

    \item
          Nehéz helyzeteken is könnyen továbblendül, vannak módszerei arra, hogyan töltse magát, és ezeket használja is.
    \item
          Azt keresi, mire lehet hatása, mit lehet eggyel jobban csinálni, és ebben akciókat tesz.
    \item
          Megünnepli az előrelépéseket, közös sikereket.
\end{itemize}


