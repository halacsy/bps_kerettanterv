\section{A közösségfejlesztés}
\label{sec:kozossegfejlesztes}
A Budapest School közösségét a tanulásszervezők és a családok alkotják közösen, akiknek közös célja, hogy közösségük a gyerekek fejlődését együtt támogassa. A tanulás holisztikus szemlélete révén a gyerekeknek egyszerre kell jó kapcsolatot ápolniuk saját társaikkal, partnerségben lenni tanáraikkal és meghallani a szülők feléjük irányuló kéréseiket.  A tanulás célját a gyerekek maguk határozzák meg, azok az értékek, rituálék viszont melyek a mindennapjaikat meghatározzák túlmutatnak az egyénen és a közösség céljait szolgálják oly módon, hogy az az egyes családok életvitelével is összhangban legyen.

Minden mikroiskola közösségnek feladata ezért, hogy az együttműködésük, a közösségi szabályaik a házirendjük meghatározásakor a családok véleményét, javaslatait is kikérjék, és úgy hozzanak döntést, hogy az mindenki számára kellően biztonságos és elég jó döntés legyen.

Ez vonatkozhat a mindennapi eszközök használatára, a közlekedésre, az együtt, tanulással eltöltött egyéni és csoportos idő arányaira, a közösen megtartott ünnepekre, jeles napokra,  mindazokra a kérdésekre, melyek közösségi döntések és hatással lehetnek az egyén fejlődésére.
