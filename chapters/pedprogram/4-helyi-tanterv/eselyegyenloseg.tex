\section{Út az esélyegyenlőség felé}
\label{sec:ut_az_eselyegyenloseg_fele}
A Budapest School lehetőséget kíván biztosítani arra, hogy a
mikroiskoláiba a társadalom minél szélesebb rétegeiből kerülhessenek be
gyerekek, hogy a családok társadalmi, gazdasági státuszától
függetlenül, kizárólag saját fejlődési útjuk kiteljesedése jegyében
válhassanak a közösség részévé.

Tudjuk, hogy az esélyek sosem egyenlőek, tenni viszont lehet azért, hogy
minél kiegyenlítettebbek legyenek. A következő területeken a családok
kiválasztásakor és az iskolába járó gyerekekkel való partneri
kapcsolatban azon dolgozunk, hogy senki se érezhesse magát hátrányosan
megkülönböztetve testi, szellemi, kulturális, szociális, nemi vagy
hitbéli egyediségei miatt.

\paragraph{A társadalmi státuszban}

  A Budapest School mikroiskoláiba járó családok legalább 30\%-a
  kevesebb hozzájárulást fizet, mint amennyi az iskola működésének
  teljes bekerülési költsége. Az ő számuk, az általuk fizetendő
  hozzájárulás mértéke a támogató családoktól függ. Minél érzékenyebb,
  minél nagyvonalúbb egy közösség azok irányába, akik nem tehetik meg,
  hogy a gyerekük tanulásáért kifizessék az ahhoz szükséges költségeket,
  annál nagyobb mértékű a társadalmi státuszbéli esélyegyenlőség egy
  adott Budapest School mikroiskolában.


\paragraph{A nemek arányában}


  A Budapest School mikroiskoláiban törekszünk a nemek arányának
  kiegyenlítésére. Folyamatosan monitorozzuk a fiú-lány arányt az egyes
  közösségekben, és amennyiben elcsúszik valamely irányba, akkor a
  felvételi során a kiegyenlítés irányába hozunk döntéseket.


\paragraph{A kulturális és vallási egyediségekben}


  A Budapest School mikroiskoláiban meghatározó a családok értékrendje. Ezen
  családok pedig jöhetnek azonos, de jöhetnek diverz kulturális és
  vallási környezetből is. Az ő szempontjaik tiszteletben tartása
  mindaddig kiemelten fontos, amíg az nem veszélyezteti a közösségben
  együtt tanuló gyerekek fejlődését.


\paragraph{A fejlődés sebességében}

  A Budapest School kerettanterve lehetőséget biztosít arra, hogy egy
  gyerek a saját tempójában, a saját maga által kijelölt és komfortos
  tanulási úton haladjon addig, amíg ebben a mentortanárával és a
  szüleivel is közös megállapodást kötnek. A fejlődés sebessége azonban
  nem akadályozhatja a közösségben tanulást. A tanulásszervezők
  felelőssége annak meghatározása, hogy egy, a közösségben lassabban
  fejlődő gyerek mennyiben veszélyezteti, vagy mennyiben segíti a
  tanulásszervezést a közösség egésze szempontjából.


\paragraph{A tanulás tartalmában}


  A Budapest School mikroiskoláiban minden gyereknek van saját célja.
  Ennek hossza, komplexitása minden esetben egyedi, függően a gyerek
  érdeklődésétől, érettségétől, családi helyzetétől, mentális és fizikai
  állapotától. A tanulás tartalmában mindenkinek van lehetősége arra,
  hogy a maga útját járja, ha figyelembe veszi, hogy ezt a közösség
  részeként kell tennie.


\paragraph{A testi és szellemi egyediségben}

  A Budapest School mikroiskoláiban a tanulásszervezők felelőssége annak
  eldöntése, hogy egy adott közösség milyen mértékben tud befogadni
  sérült, vagy saját nevelési igényű gyerekeket. Az ő befogadásukkor és
  a velük való kiemelt foglalkozáskor mindig arra a kérdésre kell
  válaszolni, hogy tudjuk-e garantálni a gyerek fejlődését, elég
  biztonsá\-gos-e a közeg a számára, és az iskolában való szerepe miként
  segíti a többi gyerek fejlődését.


\paragraph{A döntéshozásban}

A Budapest School-döntéshozatal nem többségi és\linebreak
nem konszenzusos
  megállapodások alapján történik. A döntések esélyt adnak arra, hogy
  minden kellően biztonságos, elég jó javaslatot ki lehessen próbálni
  akkor, ha az nem áll szemben a közösen elfogadott célokkal, és nem
  sérti  bármely egyén érdekeit olyan mértékben, ami sérti a
  biztonságérzetét.

