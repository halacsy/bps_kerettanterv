\section{Elsősegély-nyújtási alapismeretek}
\label{sec:elsosegely}
Ahogy azt a 20/2012 EMMI-rendelet 7.~§ (1) ak) pont előírja, a pedagógiai programunk ismerteti az elsősegély-nyújtási alapismeretek elsajátításával kapcsolatos iskolai tervet.

A cél, hogy a gyerekek és tanárok megtanulják aktívan úgy alakítani környezetüket és viselkedésüket, hogy a balesetek számát minimalizálják, hogy felismerjék, amikor segítségre van szükség, hogy hatékonyan segítsenek és tudjanak segítséget hívni. Ehhez gyakorlásra, a témával kapcsolatos védett időre van szükség. Ezért a Harmónia tantárgy 2., 4., 6., 8., 10.~évfolyamszintje csak akkor teljesíthető, ha a gyerek minden második évben elvégez egy minimum négyórás modult, aminek célja,
\begin{itemize}
    \item hogy a gyerekek sajátítsák el a legalapvetőbb és legkorszerűbb elsőse\-gély-nyújtási módokat, azaz tudjanak egymásnak segíteni baj esetén (nemcsak elméletben, hanem gyakorlatban is);
    \item sajátítsák el, mikor és hogyan kell mentőt, segítséget hívni;
    \item foglalkozzanak azzal, hogyan tudják környezetüket, csoportjukat, mikroiskolájukat biztonságosabbá tenni, és ezt dokumentálják is.
\end{itemize}

A modul szervezője próbáljon meg elsősegély-nyújtási bemutatót szervezni a gyerekeknek az Országos Mentőszolgálat, a Magyar Ifjúsági Vöröskereszt vagy az Ifjúsági Elsősegélynyújtók Országos Egyesületének vagy más, magyar vagy külföldi képesítést szerzett szakembernek a bevonásával.

Az elsősegély-nyújtási alapismeretek elsajátításával kapcsolatos feladatok megvalósításának elősegítése érdekében az iskola
\begin{itemize}
    \item kapcsolatot épít ki az Országos Mentőszolgálattal, a Magyar Ifjúsági Vöröskereszttel vagy az Ifjúsági Elsősegélynyújtók Országos Egyesületével. Tanulóink -- választásuk szerint -- bekapcsolódhatnak az elsősegélynyújtással kapcsolatos iskolán kívüli vetélkedőkbe;
    \item  minden második évben legalább egyszer a tanároknak lehetőséget biztosít elsősegély-tanfolyam látogatására.

\end{itemize}
