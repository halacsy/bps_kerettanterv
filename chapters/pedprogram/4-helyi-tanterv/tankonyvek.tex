\section{Tankönyvek kiválasztása}
\label{sec:tankonyvek}

Modulvezetők minden esetben maguk választják a modulhoz szükséges tankönyvek, szoftverek, weboldalak és egyéb eszközöket úgy, hogy

\begin{itemize}

      \item
            az a megfelelő legyen annak a csoportnak, ahhoz a célhoz, amit el akar érni
      \item
            minden esetben legyen mindenki számára elérhető (esetek többségeben értsd ingyenes) megoldás
      \item
            modulvezetők bátorítva vannak arra, hogy új dolgokat próbáljanak ki, és tapasztalataikat az iskola többi tanárával megosszák.
\end{itemize}

Mivel a Budapest School kerettantervének értelmében a saját célok legalább 50\%-át az állami kerettantervben meghatározott tanulási eredmények közül kell választani, az ehhez szükséges ismeretek megszerzéséhez a Budapest School az Oktatási Hivatal általi jegyzékben államilag támogatott OFI által fejlesztett tankönyveket veszi alapul. A Budapest School tanárcsapatának lehetősége van arra, hogy ettől eltérő, a mindenkori tankönyvjegyzékben szereplő tankönyvvel segítse el a kerettantervben meghatározott tanulási eredmények elérését. És arra is lehetősége van, hogy egyátalán ne használjon tankönyvet, mert sokszor az internet elegendő információt tartalmaz.

A Budapest School pedagógiai programjának alapja, hogy a gyerekek egyéni céljaira szabott tanulási terveket készít. Ennek előfeltétele, hogy a könyvek használata is ehhez kapcsolódó módon rugalmasan történjen, minden esetben az adott tanulási modul igényeihez szabva. Ennek érdekében a program pedagógusai folyamatosam állítják össze a gyerekek eltérő céljaihoz és képességszintjeihez igazodó differenciált tevékenységek és feladatsorok rendszerét.

A Budapest School iskolában, egy (modul)csoport csak akkor választhat egy tankönyvet, ha az minden család számára elérhető. Ha valamelyik család nem tudja a könyvet magának megvásárolni, akkor a csoport többi tagja megvásárolja neki.