\section{Környezeti nevelés}
\label{sec:kornyezeti-neveles}
Ahogy az UNESCO 1977-ben definiálta: \emph{,,a környezeti nevelés olyan folyamat,
    melynek célja, hogy a világ népessége környezettudatosan gondolkodjék, figyeljen oda a környezetre és minden azzal kapcsolatos problémára. Rendelkezzen az ehhez szükséges tudással, beállítódással, képességekkel, motivációval, valamint mind egyéni, mind közösségi téren eltökélten törekedjék a jelenlegi problémák megoldására és az újabbak megelőzésére”}.

A környezeti nevelés céljainak eléréséhez a tanárok környezettudatosságára, rendszerszemléletére és aktív részvételére van szükség. Ezért az iskola vállalja, hogy a tanárainak tanévente legalább 4 órás workshopot szervez a témában, ahol a tanárok arról  is tudnak beszélni, hogyan vitték be a környezeti nevelést a gyerekek mindennapjaiba.
