\section{A személyiség és egészségfejlesztés}
\label{sec:szemilyesegfejlesztes}

A Budapest School-gyerekek boldogak, egészségesek, hasznosak közösségüknek. Képesek önmaguknak célokat állítani, azokat elérni. Képesek  már kisgyerekkortól sajátjuknak megélni a tanulást és ahhoz kapcsolódóan célokat elérni, és fokozatosan tanulják meg azt, hogy egyénileg és csoportosan is tudnak nagyszabású projekteket véghezvinni. Tesznek a saját egészségükért, jövőjükért, társaikért, kapcsolódnak önmagukhoz és társaikhoz.

A gyerekek személyiségfejődését két szinten támogatja az iskola. Első szintet a mindennapi működés adja, mert az iskola működése önmagában személyiség- és egészségfejlesztő hatással bír. Második szintet a harmónia kiemelt fejlesztési irányelv biztosítja, ami holisztikus megközelítésével támogatja a gyerekek fizikai, lelki jóllétét és kapcsolódásukat a
környezethez.

\paragraph{Működésből adódó fejlesztések.}
Az iskola alapműködése, hogy a gyerekek csoportban, közösségben élnek, tanulnak, dolgoznak, ezért ,,természetes'', hogy fejlődik az \emph{empátiájuk, kooperációs, kollaborációs képességük és érzelmi intelligenciájuk}. Alapelvünk: ha minél közelebb áll az iskola működése a jövő hétköznapjaihoz, a családhoz és a munkahelyhez, akkor a boldog családi életre, a sikeres munkahelyre való felkészülést már az iskolában való aktív részvétel önmagában támogatja. Hasonlóan, ahogy támogató, funkcionális, boldog családban felnőtt gyerekek nagyobb valószínűséggel lesznek maguk is egészségesebbek és boldogabbak.

A \emph{fejlődésfókuszú gondolkodásmód} kialakítását  kulcstényezőnek gondoljuk a gyerekeink hosszú távú boldogulásához. Ezért a saját célok által irányított tanulási környezettől kezdve, a jutalmazás, értékelés, visszajelzés módjáig minden az iskolában azt a célt szolgálja, hogy a gyerekek képesek legyenek magukról pozitívan gondolkodni, ami az integráns és egészséges embernek talán egyik legfontosabb jellemzője.

A \emph{teljes körű iskolai egészségfejlesztést} az alábbi négy egészségfejlesztési feladat rendszeres végzése adja:

\begin{itemize}

    \item
          egészséges táplálkozás megvalósítása (elsősorban megfelelő, magas minőségű, lehetőleg helyi alapanyagokból)
    \item
          mindennapi testmozgás minden gyereknek (változatos foglalkozásokkal, koncentráltan az egészségjavító elemekre, módszerekre, pl. tartásjavító torna, tánc, jóga)
    \item
          a gyerekek érett személyiséggé válásának elősegítése személyközpontú pedagógiai módszerekkel és a művészetek személyiségfejlesztő hatékonyságú alkalmazásával (ének, tánc, rajz, mesemondás, népi játékok, stb.)
    \item
          környezeti, médiatudatossági, fogyasztóvédelmi,
          balesetvédelmi\linebreak
          egészségfejlesztési modulok, modulrészletek hatékony (azaz ``bensővé váló'') oktatása
\end{itemize}

\subsection{Egészségügyi felmérés szervezése és hatása a gyerekek életére}
A Budapest School gyerekek megelőző jelleggel rendszeresen iskolaorvosi, védőnői és fogorvosi felülvizsgálaton vesznek részt.  Az orvosi, védőnői és fogorvosi vizsgálatot a fenntartó a mindenkori jogszabályokban meghatározott rendszerességgel szervezi meg. Jelenleg ezeket külső helyszínen, megbízott orvossal, fogorvossal és védőnővel szervezi meg a fenntartó.
 
Minden mikroiskola tanulásszervező csapata a tanév megkezdése előtt kijelöli az egészségnapokat: amikor a védőnői, orvosi, fogorvosi és egyéb fizikai és mentális felülvizsgálatokat megszervezi. 20 főig egy, afölött két napot kell megjelölni, és a fenntartóval egyeztetni. Egyeztetni azért kell, hogy a különböző mikroiskolák között ne legyen időpontütközés. Ha a gyerek az egészségnapon hiányzik az iskolából, akkor a szülő feladata a felülvizsgálatot megszerveznie.
 
Ezeken a napokon a gyerekek és a tanárok, iskolaidőben elutaznak a rendelőkbe, felkeresik az orvost, fogorvost, védőnőt. Mivel a gyerekek kivizsgálása feltehetőleg egyesével történik, ezért a ,,többi” gyereknek sokat kell várnia. Ezért a tanárok erre az időre egészség témában mikromodulokat terveznek, amikor a gyerekek olyan tanulási eredményeket érhetnek el, mint a \emph{,,Tisztában van az egészség megőrzésének jelentőségével, és tudja, hogy maga is felelős ezért.''} (5. Évfolyam 1. félév).
 
A felülvizsgálatok eredményeit a gyerekekkel a mentorokkal megbeszélik, és ha szükséges, akkor az eredmények alapján fejlődési célokat fogalmaznak meg. A vizsgálatok eredményeit a gyerekek a portfóliójukban ugyanúgy megőrzik, mint egy tudáspróba eredményét.
 
