\section{A kerettanterv és a pedagógiai program\hfill\break  kapcsolata}
\label{sec:tanterv-program}
Budapest School Általános Iskola és Gimnázium az oktatásért felelős miniszter által jóváhagyott Budapest School Kerettanterv alapján működik. Az iskola munkatársai a kerettantervet, a pedagógiai programot, sőt még a szervezeti és működési szabályokat is együtt, egy egységként kezelik. Ez a Budapest School Model, aminek fő célja, hogy a gyerekek úgy tudják azt és akkor tanulni, amit szeretnek, vagy amire szükségük van, hogy közben a tanárok, szülők és a teljes társadalmunkat képviselő hivatalok számára is kiszámítható, tervezhető, biztonságos tanulási környezetet biztosít az iskola.

A minimális tantervet, azaz, hogy mit tanulnak mindenkép a gyerekek, a kerettanterv határozza meg kötelezően elérendő tanulási eredmények félévenkénti tantárgyi bontásában. Ettől az iskola nem tér el. Óraszámokban is követi a kerettanterv által megadott minimális óraszámokat. Az iskolánk a gyerekeknek azt a kérdést teszi fel, hogy ezen felül mit akarnak tanulni. Az iskolánk akkor tud igazán segíteni, ha valaki a minimális elváráson túl, valamivel mélyebben, alaposabban akar foglalkozni.
