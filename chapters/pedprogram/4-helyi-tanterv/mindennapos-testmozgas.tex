\section{Mindennapos testmozgás}
\label{sec:mindennapos-testmozgas}

A Budapest School kerettantervben bemutatott rendszer szerint a "mikroiskoláknak saját fókuszuk, helyszínük, stílusuk alakulhat ki." Ennek részeként a gyerekek mindennapos testmozgását is a mikroiskolák saját identitásuk és lehetőségeik szerint alakítják ki. Az iskola csak annyit ír elő, hogy minden nap legyen minimum 35 perc, aminek elsődleges célja a testmozgás és testnevelés.

Nincs olyan testmozgás forma, amelyet ez a program preferálna vagy amely kizárt lenne. A mindennapos testmozgás fizikai szükségleteinek megszervezése is változatos módon történhet:
\begin{enumerate}
    \item Egyes mikroiskolák (telephelyek) saját tornatermében, tornaszobájában vagy udvarán.
    \item Szerződés, megállapodás alapján elérhető közelben lévő más oktatási vagy sport létesítményben (úszás, korcsolya, torna, foci, falmászás, sípálya),
    \item A szabadban, amikor épített helyszínhez nem kötődő mozgásszervezés (pl. kirándulás, séta, barangolás, felfedező játékok).
    \item. A feladatellátási helyeken, magyarul a tantermkeben, speciálisan sportra alkalmas helyiséget nem igénylő mozgások esetén (pl. jóga).
\end{enumerate}

A testmozgás minden esetben a mozgást ismerő, abban képesítéssel, végzettséggel, vagy megfelelő gyakorlattal rendelkező személy vezetésével vagy felügyelete mellett zajlik. Az iskola feladata monitorozoni a mindennapos testnevelés megvalósulását: ki, mikor, hol milyen testmozgást vezetett a gyerekeknek.
