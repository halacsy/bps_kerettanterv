\ketoldaltkep{pics/6A_TANTARGYI_SOMA.JPG}{pics/6b_TANTARGYI_FLORA.jpg}
\section{Évfolyamok és osztályzatok}
\label{sec:evfolyamok_osztalyzatok}
A Budapest Schoolban tanuló gyerekek saját tanulási célokat tűznek ki, modulokat választanak, tanulnak, alkotnak, trimeszterenként frissítik a portfóliójukat, mentorukkal és a modulvezetőkkel értékelik haladásukat, és ha kell, újraterveznek.

Eközben a gyerekek a tanulás és alkotás eredményeként évfolyamszinteken lépkednek fel, első szintről a tizenkettedik szintig tantárgyanként. Azt, hogy ez hogyan és mikor történik, azaz az évfolyamszintek elismerését -- a kerettantervvel összhangban -- az iskola transzparens folyamata szabályozza.

Hivatalos, tantárgyankénti osztályzatokat a gyerekek akkor és csak akkor kapnak, ha erre iskolaváltás, továbbtanulás, ösztöndíj (vagy más külső rendszer) miatt szükségük van vagy ezt a törvény az iskolának kötelezővé tesz. Tehát osztályzatok, érdemjegyek és vizsgák nélkül is van lehetőség évfolyamszinteket lépni.\footnote{Egy tantárgyból csak azon az évfolyamszinten lehet osztályzatot kapni, amelyik évfolyamon az a tantárgy kötelezőként szerepel a tantárgyi struktúrában \aref{tbl:oraszamok}~.táblázat alapján.}

A vizsga így teljesen átértékelődik a Budapest Schoolban. Az évfolyamszintek elismeréséhez és (szükség esetén) az érdemjegyek megállapításához nem elégséges a pillanatnyi tudást vagy képességet felmérő eseményt szervezni, hanem a teljes portfóliót kell értékelni és figyelembe venni. A portfólió sokkal gazdagabban dokumentálja, hogy egy gyerek mit csinált, mire volt képes, mint egy szóbeli vagy írásbeli feladatsor: előzetes tudás- és képességpróbák mellett tartalmazza az alkotások, projektek, visszajelzések, versenyek, stb. dokumentációit is.

\subsection{Évfolyamszintek}
\label{sec:evfolyamok}

A Budapest Schoolban az évfolyamokra úgy tekintünk, mint egy szerepjáték nehézségi szintjeire \citep{wiki:game_levels}: akkor léphet egy gyerek a következőbe, ha az évfolyamhoz köthető tantárgyi tanulási eredményekből eleget összegyűjtött.

A Budapest School évfolyamszintjei eltérnek az iskolák többségében alkalmazott évfolyamtól. A különbség kihangsúlyozása végett a kerettanterv az évfolyamszint kifejezést használja. A különbségek:

\begin{itemize}
      \item Egy gyerek tantárgyanként más-más szinten állhat.
      \item Nem biztos, hogy az egy korcsoportba tartozó gyerekek
        vannak\linebreak
        ugyanazon az évfolyamszinten.

      \item Nem mindig az egy évfolyamszinten lévők tanulnak együtt, előfordulhat, hogy a különböző szinten lévő gyerekek tudnak együtt és akár egymástól is tanulni.

      \item Egy év alatt több évfolyamszintet is lehet lépni.
      \item Évfolyamszintlépéshez szükséges eredményeket tanév közben is el lehet érni, még akkor is, ha a hivatalos bizonyítványok a szintlépést csak tanév végén mutatják meg.
\end{itemize}

Bár egy gyerek tantárgyanként eltérő szinten állhat, a hivatalos (azaz külső hivatalok, rendszerek számára értelmezhető) bizonyítványába mindig csak annak az évfolyamnak az elvégzése kerül be, amelyből minden tantárgy által definiált szükséges tanulási eredményt elérte. Formálisabban kifejezve a bizonyítványban a tantárgyankénti évfolyamszintek minimumát kell rögzíteni.

\subsubsection{Évfolyamszintlépés}
\label{sec:evfolyamszintlepes}
A gyerekek portfóliója alapján megállapítható, hogy egy adott évfolyamhoz köthető tantárgyi követelményeknek megfelel-e. Ehhez a gyerekek elvégzik a mentoruk segítségével a portfóliójuk (mit csináltak, mit tanultak, mit tudnak) összehasonlítását 
      \aref{sec:tantargyi_tanulasi_eredmenyek}.~fejezetben
 felsorolt tantárgyankénti bontásban megadott elvárt tanulási eredményekkel.

Ha a kapcsolódás biztosításához szükséges, a gyerekek a portfóliójukat kiegészíthetik kihívások, tudáspróbák, tesztek, szabványos vizsgák teljesítésével, melynek megszervezése az adott mikroiskola tanárközösségének a feladata.

Miután a gyerek (mentora és szülei segítségével) összeállította a port\-fó\-lió\-ját, jelzi az iskolának az évfolyamszintlépési kérelmét.

Ezt az iskola által kijelölt tanulásszervezők\footnote{A kérelmeket
  elbíráló tanulásszervezők kijelölését a pedagógiai programnak vagy a
  szervezeti és működési szabályzatnak kell meghatároznia.}
megvizsgálják, és elismerik az évfolyamszinthez szükséges tantárgyi
követelmények teljesítését.\linebreak
Egy tantárgyból egy évfolyam teljesítettnek csak akkor tekinthető, ha
a tantárgyhoz tartozó tanulási eredmények 50\%-ának elérése a
portfólió\linebreak
alapján bizonyítható.

A kérelmet a gyerek digitálisan adja be. Az elbírálás csak a portfólió alapján történhet, ami egy online elérhető adatbázisként tartalmaz mindent, ami a döntéshez szükséges lehet. A döntéshez így a tanulásszervezőknek és a gyereknek nem kell egy időben és egy helyen lennie. Minden esetben szükséges a portfóliót és a teljes folyamatot digitálisan rögzíteni. Az iskolának tizenkét évig meg kell őriznie a portfóliót, a kérelmet és a döntéshez használt minden dokumentációt.

Amennyiben valamely gyereknél egy adott évfolyamszint
tantárgyi\linebreak
követelményei elismerésre kerülnek, akkor az iskola igazolja, hogy a gyerek az adott tantárgy vagy tantárgyak évfolyam szerinti követelményeit teljesítette. Erről igazolást állít ki, és teljesíti a jelentési kötelezettségét az Oktatási Hivatal felé.

Egyéni munkarendben tanuló vagy az órák látogatásáról valamilyen okkal felmentett gyerekek ugyanígy, a portfóliójuk összeállításával és a szintlépés kérelmezésével kérhetik az évfolyamszintek teljesítésének igazolását.

\subsection{Osztályzatokra váltás}
\label{sec:osztalyzatok}
A kerettanterv lehetővé teszi, hogy a gyerekek az érdemjegyek és osztályzatok helyett egy több szempontot figyelembe vevő szöveges vagy ér\-té\-ke\-lő\-táb\-lá\-zat- (rubric-) alapú visszajelzést kapjanak. A gazdag információtartalmú visszajelzések és portfólió osztályzatra való átváltására mégis szükség lehet, például iskolaváltás vagy továbbtanulás esetén.\footnote{Az Nkt. 54.§ (4) pontja alapján.}

Az átváltás \aref{sec:evfolyamszintlepes}.~fejezetben leírtakhoz hasonlóan is a portfólió értékelésén alapul. A gyerek (mentora és szülei segítségével) összeállítja a portfóliót, és bizonyítja, hogy a portfólió alapján megállapítható a kívánt osztályzat, az adott tárgyhoz az adott évfolyamszinten.

\subsection{Kérelmek elbírálása}

A szülő és gyerek évfolyamszintlépés vagy osztályzatra váltás kérelmét bírálók értékelik. Minden esetben legalább három bíráló bírál egy kérelmet: a bírálók közül egy a gyerek mentortanára, egy pedig mindenképp másik mikroiskola tanulásszervezője. A bírálókat a fenntartó választja ki és kéri fel. Ha a mentortanár valamilyen okból nem tudja feladatát elvégezni, akkor helyette az a tanulásszervező lesz a bíráló csapat tagja, aki legtöbb időt töltött az elmúlt két trimeszterben a gyerekkel.

Ha a bírálók közül akár egy is úgy véli, hogy az évfolyamszintlépés vagy osztályzat nem megalapozott, akkor erről indoklással értesítik a gyereket, szüleit és mentortanárát. Ilyenkor a gyerek javíthatja kérelmét és tetszőleges számú esetben ismételheti a folyamatot. Tulajdonképpen az évfolyamszintekről és osztályzatokról mindig olyan döntésnek kell születnie, ami a gyereknek, a szülőknek és a bírálóknak is elfogadható. Ha nem tudnak megállapodni, akkor \aref{sec:konfliktusok_kezelese}. fejezetben leírtak alapján kell keresniük a mindenki számára elfogadható megoldást.

Az évfolyamszintlépésről és az osztályzatra váltásról vagy épp a kérelmek elutasításáról az iskola minden tanára értesítést kap. Ha a nevelőtestület\footnote{Nkt. szóhasználata.} nem ért egyet a bírálók döntésével, akkor új bírálók kinevezését kérhetik.

A kérelmek elbírálását 20 tanítási nap alatt el kell végezni.\footnote{Ha a kérelem nyári szünetben érkezett, akkor augusztus 31-ig.}

Évfolyamszintlépést az iskola az utolsó évfolyamszintlépéstől számított két tanéven belül automatikusan indít. Így nem fordulhat elő, hogy egy gyerek évfolyamszintjei két éven keresztül nincsenek felülvizsgálva.

Magántanuló és az órák látogatásáról valamilyen okkal felmentett gyerekek évfolyamszintlépését és szükség esetén osztályzatkérelmét pontosan ugyanazon folyamatokkal kell elvégezni, mint a nem magántanuló és nem felmentett gyerekét.

\subsection{Osztályozóvizsga}
% forras: http://www.petroczigabor.hu/cikkek/tanugyigazgatas/Osztalyozo_vizsga.html
A \emph{Osztályzatokra váltás} és \emph{Évfolyamszintlépés} folyamat a gyerek évközi alkotásai, munkája, teljesítménye eredményeként készült portfólió alapján történik, ezért a folyamat megfelel a 20/2012. (VIII.31.) EMMI-rendelet 64.~§ (1) azon elvárásának, hogy a tanuló osztályzatait \emph{,,évközi teljesítménye és érdemjegyei''} alapján kell megállapítani azzal a kikötéssel, hogy a Budapest School-kerettanterv elfogadja az érdemjegyeknél gazdagabb szöveges és értékelőtáblázatok-alapú visszajelzést.

Ha a portfólió alapján nem állapítható meg az évfolyamszintlépéshez és az osztályzathoz szükséges tanulási eredmények megléte, akkor a gyereknek ki kell egészítenie a portfólióját. Akár egy tudáspróbával, vagy online teszttel. Vagy el kell fogadnia, hogy \emph{még} nem érte el a következő évfolyamszintre lépéshez szükséges tanulási eredményeket.

Ugyancsak kiegészítendő a portfólió, ha igazolt vagy igazolatlan mulasztások miatt az iskolában végzett munka nem volt elengendő az elégséges portfólió elemek összegyűjtésére. Ez azt is jelenti, hogy az iskolának nem kell az \emph{Osztályzatokra váltás} és \emph{Évfolyamszintlépés} folyamatoktól eltérnie, csak a portfóliót kell kibővíteni, vagy megint csak a gyereknek el kell fogadnia, hogy \emph{még} nem érte el a következő évfolyamszintre lépéshez szükséges tanulási eredményeket.

Ha a gyereket bármely okból felmentették a tanórai foglalkozásokon való részvétele alól, akkor a 20/2012. (VIII.31.) EMMI-rendelet 64.~§ (2)  alapján osztályozóvizsgát kell tennie. A Budapest School iskolában a fent részletezett \emph{Osztályzatokra váltás} és \emph{Évfolyamszintlépés} folyamatok  felelnek meg az osztályozóvizsgának.
