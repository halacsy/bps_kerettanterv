\section{Tanulási helyek a Budapest\hfill\break Schoolban}

A Budapest School tanulási formájának esszenciája, hogy a gyerekek a tantárgyi eredményeket és az általuk megfogalmazott saját célokat változó hosz\-szúságú, de legfeljebb egy trimeszterig tartó modulokon keresztül érik el. A modulokat a tanulásszervezők hirdetik meg a trimesztereket megelőzően, és minden gyereknek a saját mentora segít abban, hogy a fejlődéséhez és az eredményekben való továbblépéshez szükséges modulok meghirdetésre kerüljenek, és mentoráltjuk válasszon belőlük. A moduláris tanmenet egyik jellemzője, hogy a tanulás alkotó, projektjelleggel, külön e célra alkalmas környezetben valósul meg. A tanulás ezáltal számos alkalommal nem csupán az iskola székhelyén vagy telephelyein, hanem az iskolával állandó szerződésben lévő partnereknél, úgynevezett tanulási pontokon, vagy alkalmi jelleggel, egyedi megállapodások alapján más intézményekben, így például kulturális-, sport-, tudományos központokban, vagy céges környezetben valósul meg. A tanulás helyét tehát a tanulás célja határozza meg, ahhoz igazítja a tanulásszervező és nem fordítva.  A Budapest School kiemelten fontosnak tartja, hogy a Nemzeti Alaptanterv szabályozásának \footnote{Nemzeti alaptantervről szóló 100/2012.~Kormányrendelet (Nat) 7.~§  (1)} megfelelően tanórai foglalkozásait a hagyományos, tantermi szervezési formáktól eltérő módon is megszervezze, így különös hangsúlyt fektet a projekt oktatásra, az erdei iskolára, vagy másként nevezve a természetjárásra, a múzeumi foglalkozásra és tágabb értelemben az élő művészet megismerésére, a könyvtári foglalkozásra, vagyis az olvasás, kutatás megszerettetésére olyan környezetben, ahol olvasni és kutatni jó.

\subsection{Élménynapok}
Minden mikroiskola választ egy napot, amikor a gyerekek előre tervezetten és rendszeresen a tanárokkal együtt \emph{kimennek az iskolából}. Ezt a napot nevezzük \emph{élménynapnak}. Az, hogy a hét melyik napján és hogy hány hetente szerveznek a tanulásszervezők élménynapot, az az ő döntésük.\footnote{Javasolt a pénteket választani, de telephelyenként eltérhet, hogy mikor szervezik az élménynapot.}  Ezeken a napokon nagyon változatos programok szervezhetők, a külső moduloktól, a nem formális (tanórán és iskolán kívül szervezett) és az informális (nem szervezett, spontán tevékenység során megvalósuló) tanulási formákig minden. Egyetlen egy megkötés, hogy legalább egy héttel előtte értesíteni kell a szülőket és a fenntartót a tervezett programról.\footnote{A terv lehet az, hogy nincs több terv, csak kint vagyunk a Duna parton és ott játszunk együtt.} Hiszen azt tudnunk és dokumentálnunk kell, hogy hol vannak a gyerekek.

Átszervezhető-e egy élménynap másik napra? (Mert például egy múzeum csak egy másik nap van nyitva?) Igen, ha erről legalább egy héttel előtte minden érintett tudomást szerez. Lehet-e két nap egy héten a külső tanulás? Igen, ha ez a gyerekek tanulását és fejlődését segíti, és a tanulásszervezők biztonságban meg tudják szervezni ezeket a napokat úgy, hogy a meghirdetett és folyamatban lévő modulok vezetői hozzájárulásukat adják.

\subsection{Állandó külső tanulási pontok}

A Budapest School iskola több telephellyel rendelkezik, amelyek otthont adhatnak egy vagy több mikroiskolának (összevont osztálynak). Mikroiskolák kicsi közösségként jönnek létre. Kezdetben akár csak 10 gyerek ,,köré'' szerveződik egy mikroiskola, és onnan nő fel 30, 40, és végül 50 főre. Kezdetben érdemes lehet olyan helyszínen megszervezni a mindennapokat, amik az engedélyezési eljárás során még nem tudtak telephellyé változni. Ezeket a helyszíneket hívjuk \emph{tanulási pontoknak}.  Másképp fogalmazva, ha egy mikroiskola egy trimeszternél hosszabb ideig egy külső helyen tartja a moduljait, akkor ott tanulási pontot kell kialakítani.

A tanulási pontok a fenntartó által kialakított, és ellenőrzött biztonságos terek, amit egy mikroiskola sajátjaként tud megélni. A tanulási pontok, bár hivatalosan nem telephelyek, felszerelését és működtetését a fenntartó és az iskola a telephelyekkel megegyező módon végzi. A tanulási pontokban található helyiségeket a Budapest School Általános Iskola és Gimnázium a 20/201. (VIII. 31.) EMMI rendelet 2.~számú mellékletében meghatározott felhatalmazás alapján alakítja ki: \emph{Az eltérő pedagógiai elveket tartalmazó nevelési program az eszköz- és felszerelési jegyzéktől eltérően határozhatja meg a nevelőmunka eszköz és felszerelési feltételeit.}

\paragraph{A tanulási pontok fizikailag biztonságosak}
A tanulási pontok kialakításakor ugyanazokat az előírásokat kell betartani, mint amit a 40/2018. (XII. 4.) 16. EMMI rendelet a Tanodáknak előír, azaz:

A tanulási pontnak vagy a tanulási pontnak helyet biztosító, más szolgáltatást is befogadó integrált térnek meg kell felelnie a létesítési, használati és üzemeltetési tűzvédelmi előírásoknak, amelyet -- a fenntartó kérésére -- egy független ellenőr állapít meg és évente ellenőriz. A tanulási pontban biztosítani kell
\begin{enumerate}
    \item legalább egy WC-t és folyóvíz vételére is alkalmas kézmosót;
    \item legalább egy teakonyhát és
    \item   legalább egy, a következő bekezdésben foglaltak szerint kialakított közösségi teret, amely
    \begin{enumerate}
        \item legfeljebb két helyiség egybenyitásával alakítható ki, legalább 30 négyzetméter alapterületen,
        \item jól szellőztethető, fűthető és természetes fénnyel megvilágított helyiségben, helyiségekben alakítható ki,
        \item a tanoda szolgáltatás nyújtásával azonos időben más célra nem használható, és
        \item nem alakítható ki iskola működő feladatellátási helyén;
    \end{enumerate}
\end{enumerate}


A WC, kézmosó és teakonyha más szolgáltatással közösen használt helyiségek is lehetnek.

\paragraph{A tanulási pontok egy telephely alá tartoznak.}

A tanulási pontok nem alkalmasak minden iskolai feladat és szolgáltatás biztosítására, ezért a telephely és a fenntartó biztosítja, hogy a kerettanterv által a gyerekek érdekében előírt feltételek a tanulási pontokon lévő gyerekek számára is elérhető legyen vagy a telephelyen, vagy más szerződött partnernél. A tanulási pontokon akár állandó jelleggel tartózkodó gyerekek is iskolába járással teljesítik a tankötelezettségüket Budapest School Általános Iskola és Gimnáziumban (Nkt. 45. § (4) első fordulata).

Az élménynapokon a tanulási pontokon lévő gyerekek rendszeresen meglátogatják a telephelyeket. Ott el fognak férni, mert a telephelyen állandó jelleggel tanuló gyerekek épp az iskolán kívül vannak. 

\paragraph{A tanulási pontok ellenőrzött helyszínek.}

A tanulási pontokat a fenntartó évente ellenőrzi -- megfelelő végzettséggel és jogosítvánnyal bíró hatósági vagyy nem hatósági ellenőrök bevonásával --, hogy azok megfeleljenek a gyerekek rendszeres ott tartózkodása miatt elvárható legszükségesebb közegészségügyi előírásoknak, valamint a tűzvédelmi, egészségvédelmi  munkavédelmi követelményeknek.

A fenntartó minden mikroiskolát, és így a tanulási pontokon működő mikroiskolákat is trimeszterenként legalább egy tanítási napon megfigyeli, és ez alapján értékelő visszajelzést ad mind a pedagógiai munkával, mind a fenti elvárásokkal kapcsolatban.

\paragraph{A tanulási pontokon megfelelő tanárok vannak.}

Egy tanulási pont egy telephely teljes mikroiskájának ad helyet, így a tanulási pontokon a mikroiskola tanulásszervezői és modulvezetői elérhetőek, rendszeresen jelen vannak. A tanulási pontokon dolgozó tanárok a telephely szervezet alá tartoznak a munkaszerződésük alapján, munkavégzésük helye módosul, kiegészül a tanulási ponttal.

\paragraph{Gyerekek nyomonkövethetőek.}

Az iskola napra pontosan nyilvántartja,\linebreak
hogy egy gyerek melyik helyszínen tanul: telephelyen, tanulási ponton,
külső szerződött partnernél vagy eseti külső helyszínen tanul-e
aznap,\linebreak
vagy éppen tervezett otthon tanulás vagy nem tervezett hiányzás van.

\subsection{A Budapest School tanulási helyeinek meghatározására vonatkozó szabályok}
A gyerekek állandó bázisa egy telephelyen (székhelyen) van. Ha egy mikroiskolának tanulásszervezői javasolják, és ez a gyerekek tanulását támogatja, akkor a mikroiskola kiteheti működését egy tanulási pontra a fenntartó hozzájárulása esetén.

Egy mikroiskola dönthet úgy, hogy egy szerződött partner helyszínén, például egy múzeumban tart egy modult. Egy modulvezető -- ha ez a gyerekek tanulását támogatja --  külső helyszíneken tarthat foglalkozásokat eseti jelleggel, amennyiben a mikroiskola tanulásszervezői és a szülők ezt támogatják.

\begin{table}[h]
    \begin{center}
        \scalebox{0.7}{%
            \begin{tabular}{p{0.25\textwidth}|l|p{0.25\textwidth}|p{0.25\textwidth}}
                \textbf{helyszín}     & \textbf{felelős} & \textbf{feladat}                      & \textbf{hozzájárulás}                                      \\ \hline
                eseti külső           & modulvezető      & egy-egy foglalkozás megtartása        & a tanulásszervező hozzájárul és a szülők    nem ellenzik \\
                \hline
                szerződött partner    & mikroiskola      & egy-egy modul megtartása              & a tanulásszervező hozzájárul és a szülők    nem ellenzik  \\
                \hline
                tanulási pont         & iskola           & több modul megtartása, közösségi élet & fenntartó, tanulásszervező és a szülők                                                   \\
                \hline

                telephely v. székhely & iskola           & az iskola minden feladata             & fenntartó
            \end{tabular}}
    \end{center}
    \label{tbl:tanulasi-helyek}
    \caption{Az iskolai tanulás különböző helyszínei.}
\end{table}

Bármely telephelyen meghirdetett modul megnyitható más telephelyen  vagy tanulási ponton tanuló gyereknek, de a tanulási pontok moduljai más gyereket nem fogadhatnak. Azaz csak a telephelyeken lehet a különböző mikroiskolák gyerekeinek keveredésével modult szervezni.

\paragraph{Utazás biztosítása} Az iskola és nem a szülő feladata, hogy megoldja a gyerekek utazását a külöböző helyszínek között. A tanulási pontok kivételek: ott a gyerek kezdi és befejezi a napját, és amíg önállóan nem közlekedik, addig a szülőnek kell arról gondoskodnia, hogy odajusson. Éppen ezért szükséges a tanulási pontok kialakításához minden fél hozzájárulása.


\subsubsection{Kizárólag a székhelyen és telephelyeken végezhető feladatok}
A székhely és telephelyek az iskola törvényileg elismert feladatellátási helyei. Csak itt történhet
\begin{enumerate}
    \item az iskolába való be- és kiiratkozás, a bizonyítvány kiállítása;
    \item a trimeszterenkénti egy \emph{nagy tudáspróba és portfólióbemutatás}, amikor a gyerekek más gyerekeknek és más tanároknak mutatják be, hogy mit tanultak, alkottak az elmúlt időszakban;
    \item  más mikroiskolák számára elérhető \emph{közös modulok} foglalkozásai.
\end{enumerate}

\subsubsection{Kizárólag szerződött partnernél, tanulási pontokon végezhető feladatok}
\begin{itemize}
    \item A trimeszterek teljes hosszán végigfutó modulokat olyan szerződött partnereknél lehet végezni, melyeknek legalább a trimeszter teljes hosszára van az iskolával vagy a fenntartóval bérleti vagy szolgáltatói szerződése.
    \item A speciális feltételeket, eszközöket, vagy termeket megkívánó, de nem egy teljes trimeszterig, hanem akár eseti jelleggel meghirdetett modulokat (például úszás, digitális alkotó labor, falmászás, csillagvizsgálat) kizárólag akkor lehet meghirdetni, ha erre van legalább a modul megtartását megelőző egy héttel előbb megállapodása az adott gyerekek tanulásáért felelő tanulásszervezőknek egy a feladat elvégzését hitelesen bizonyító partnerrel, és erről ugyaneddig a határidőig a szülőket is tájékoztatja.
\end{itemize}

\subsubsection{Bárhol végezhető feladatok}

Eseti jelleggel a tanulásszervezők koordinátorának beleegyezésével minden modulvezető dönthet úgy, hogy valamely modulját külső helyszínen, például parkban, erdőben, egy cégnél vagy civil szervezetnél vendégeskedve tölti el.

Abban az esetben, ha egy modul lényeges tartalmi eleme, hogy külső helyszíneken történik a tanulás, akkor erről előre kell a szülőket tájékoztatni. Például egy \emph{gyárlátogatás} modul esetén egyben a modul elején megadják a szülők a hozzájárulásukat a külső helyszínekhez, nem kell minden héten erről újra megállapodni.

\subsubsection{Otthon végezhető feladatok}

A Budapest School egész napos iskolát biztosít, mégis lehetnek olyan esetek, amikor egy modul elvégzésének épp az a feltétele, hogy a gyerek otthon maradhasson. Vagy azért, mert egyéni munkában akar egy projektje végére jutni, vagy azért, mert szülői támogatásra van szüksége ahhoz, hogy tanulmányi eredményeiben előreléphessen. Erre akkor kerülhet sor, ha az adott modulban történő előrehaladás a gyerek mentorának beleegyezésével történik, és nem veszélyezteti a trimeszterben felvett további modulok elvégzését.

