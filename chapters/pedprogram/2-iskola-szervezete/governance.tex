\section{Az iskola kormányzása}
Az iskola szervezetét a tagjai együtt alakítják. A szervezet élő, folyamatosan változik, abban az irányba, hogy
\begin{itemize}
    \item agilis, hatékony, gyorsan tanuló és reagáló, önszerveződő szervezetként tudja az iskola egyre jobban támogatni a gyerekek tanulását;
    \item a gyerekekről legtöbbet tudók, a hozzájuk legközelebb állók, a tanárok (gyerekekkel és szülőkkel erős kapcsolatban), minél több döntést tudjanak maguk meghozni;
    \item az együttműködés, a partneri kapcsolat, a kölcsönös felhatalmazás a tanári szobában is megjelenjen, ne csak a tanár-gyerek kapcsolatban.
\end{itemize}

Vágyunk, hogy gyerekeink kreatív, környezetüket aktívan alakító, mély partneri kapcsolatokban élő, problémamegoldó, csapatjátékos, folyamatosan tanuló felnőtekké válljanak. Ehhez olyan iskolát kell építenünk, ahol a tanárok ma kreatívak, környezetüket aktívan alakítják, mély partneri kapcsolatban élnek, problémákat oldanak meg, csapatban dolgoznak és folyamatosan tanulnak.

Az iskolát, mint szervezetet 
a \emph{szociokrácia} (sociocracy) szervezeti modell \citep{sociocracy} alapján működtetjük, mert ma így tudjuk leginkább elérni, hogy egyszerre legyen örömteli és a hatékony az együttműködésünk.

\subsection{Döntéshozás}

Az iskola minden csapata a \emph{hozzájáruláson alapúló döntési mechanizmust} (consent based decision making) használja, mert fontos, hogy a szervezet gyorsan tudjon döntést hozni \emph{és} minden tagja tudja a hangját hallatni. 

A csapat minden tagja hozhat javaslatokat ha működési hatékonytalanságot, feszültséget, problémát talál és van rá megoldása. 
A javaslat értelmezése után az érintettek mindegyikét meg kell hallgatni, hogy \emph{elfogadhatónak} tartja-e a javaslatot, azaz hozzájárul a változáshoz, mert \emph{,,elég jónak és biztonságosnak találja, hogy kipróbáljuk az új működést''} (\emph{"is this good enough for now and safe enough to try?"}). Fontos, hogy mindenki egyenként hallassa hangját.

Mindenki kifejezheti a \emph{fenntartásait} (concern), és a csapat feladata ezeket meghallanni, azokra reagálni. A fenntartás azonban még nem jelenti a javaslat elutasítását, csak fontos információt ad hozzá a döntés végrehajtásához.

Ezzel szemben ha valakinek \emph{ellenvetése} (objection) van, akkor ez azt jelenti, hogy ha a csapat ezt a döntést meghozza, akkor ,,annak negatív hatása lesz az elköteleződésemre, ezért nem tudom támogatni a döntést." Ilyenkor a csapat feladata egy olyan javaslatot találni, ami az ellenvetést feloldja és az eredeti javaslat céljai felé viszi a szervezetet.

\paragraph{Hozzájárulás nem konszenzus.} A konszenzus alapú döntések esetén mindenkinek egyet kell értenie abban, hogy a döntés a legjobb, leghelyesebb, leghelyénvalóbb. A Budapest School iskolában azt a kérdést tesszük fel inkább, hogy van-e valakinek ellenvetése. Nem azt a kérdést tesszük fel, hogy mindenki ezt a döntést hozta volna-e és mindenki egyetért-e döntéssel hanem azt, hogy mindenki tudja-e támogatni a csapat egy másik tagját, senki sem ismer olyan rizikót, amit nem tud vállalni.

\paragraph{Hozzájárulás nem szavazás.} A Budapest School iskolában nem a többség dönt, és nem az számít, hogy hányan akarnak egy döntés mellé állni. Mindenki hozhat döntést, amit elfogad a csapat minden tagja, azaz egyetlen egy ellenvetés sincs.

\paragraph{Az ellenvetés nem vétó.} A vetó jog gyakorlatban a döntés megakadályozását jelenti. Amikor valaki megvétóz egy döntést, akkor azzal a folyamat általában megakad. Az iskola működésében használt ellenvetés egy beszélgetés megindítását jelenti: ,,ezt én így nem tudom támogatni, inkább legyen így''.

\paragraph{A ,,van egy jobb ötletem'', nem ellenvetés.} A szervezetnek nem az a feladata, hogy a legjobb döntéseket hozza, hanem, hogy amikor szükséges, akkor javítson a működésén. Ezért minden döntéskor mindenkinek azt kell mérlegelnie először, hogy elfogadható-e neki, hogy azt a bizonyos javaslatot kipróbálja a csapat. Attól, hogy valaki jobb, más javaslatot is tud, attól még először az eredeti javaslatot érdemes kipróbálni és tesztelni.

\paragraph{Minden javaslat csak egy hipotézis.} Amikor valamit változtatunk a szervezet a működésén, akkor egy kísérletbe vágunk bele: kipróbáljuk, hogy az új működés tényleg jobb-e, megoldja-e  problémát, feszültsét, kielégíti-e az igényeket. A döntés támogatásakor ezt a kísérletet támogatjuk. 

\paragraph{Nem döntünk mindenről együtt.} A Budapest School iskolában csak az iskola és a mikroiskola működését megváltoztató kormányzási kérdésekről döntünk együtt. A Budapest School minden tagja szerepeiből kifolyólag fel van hatalmazva arra, hogy a mindennapi döntéseit maga meghozhassa, ezért nem kell mindent megbeszélnünk. A cél, hogy olyan szerepeket és rendszereket alakítsunk ki, hogy a mindennapi döntéseket mindenki maga meg tudja hozni.

\subsection{Csapatok - az iskola szervezeti egységei}
A Budapest School csapatok (szociokrácia terminológiában a \emph{körök}) önálló csoportok egy jól meghatározott céllal, felruházott felelősséggel, döntési körrel. A csapatok maguk határozzák meg a saját működésüket (policy making) és végzik el a saját feladatukat. 
Az iskolában azok döntenek együtt, akik együtt dolgoznak, egy csapatban (“those	who	associate	together	
govern	 together”). És fontos, hogy akik együtt dolgoznak, azok jól legyenek egymással.

Azt is tudjuk, hogy akik egy munkát elvégeznek, azok a munka szekértői, így ők tudnak arról a legjobban dönteni, hogy hogyan érdemes a munkájukat szervezni, alakítani. Nincs főnök, külső szakértő, aki megmondja egy csapatnak, mit és hogyan csináljanak addig, amíg a rájuk felhatalmazott kereteken belül maradnak. Az természetes, hogy segítséget, támogatást, információt megkapnak, amire szükségük van. De a kormányzás a kezükben van.


\paragraph{Mikroiskola tanulásszervező csapata.} A Budapest School szervezet állandó csapatai az egy-egy mikroiskolát vezető tanulásszervezők csapata, ami egy \emph{szocikratikus kör}. A mikroiskola gyerekeinek (családjainak) és tanárainak életét meghatározó döntéseket maguk hozzák meg. Így például a napirend, a csoportbontások, a szülői értekezletek tematikája a saját döntéseik alapján alakul ki. Fontos, hogy a csapat tagok maguk tudják meghatározni kivel tudnak és akarnak együtt dolgozni, mikor és mit akarnak csinálni.

\paragraph{Csapatok kapcsolódása}
Egy-egy ember több csapatnak is tagja lehet. Egyrészt munkacsoportok alakulhatnak egy-egy feladat elvégzésére és a Budapest Schoolban egy ember több részfeladatot is ellát. Másrészt a csapatokat kifejezetten úgy alakítja a közösség, hogy legyenek közte kapcsolódások, olyan tagok, akik összekötik a csapatokat.

Vannak olyan csapatok, aminek elsődleges célja, összekötni a kisebb csapatokat. Például minden mikroiskolai csapat delegál egy képviselőt a közös naptárat lérehozó munkacsoportba.


\paragraph{Csapatok vezetői}
Minden csapatnak van egy \emph{vezetője} (szociokrácia terminológiában a \emph{circle leader}). A vezető feladata,  hogy mindenki ismerje a csapat célját, a \emph{,,miért létezünk?"} kérdésre a választ és hogy a csapat működjön: tiszták legyenek a szerepek és megtörténjen, az amiben megállapodott a csapat (vagy beszéljenek arról, mi akadályozta a csapatot, és hogyan hárítják el az akadályokat). 

A csapatvezető a Budapest School rendszerében nem az, aki megmondja, ki mit csináljon, nem osztja, ellenőrzi vagy felügyeli a feladatokat, nem rúg ki, és nem vesz fel embereket, hanem szolgálja a csapatot (servant leadership) azzal, hogy segíti a megállapodásokat betartani: facilitál, moderált, szintetizál, kísér, kérdez. A csapatvezető megválasztásához, mint minden szerep megválasztásához a csapat minden tagjának hozzájárulása szükséges.

\subsection{Mi van, amikor egy csapat nem tud döntéshozni, együttműködni?}
Amikor a csapattagok úgy érzik, hogy nem tudnak mindenki számára elfogadható döntéseket hozni, nem haladnak, vagy megjelentek a játszmák és ezért már nem tudják a csapatcélját szolgálni, akkor konfliktus, feszültség alakul ki, aminek feloldásához segítséget hívhatnak be a szervezet többi részéről (lásd \aref{sec:konfliktusok_kezelese}.~fejezet).

A csapat folyamatos harmóniájáért folyamatosan dolgozni kell, ahogy az egészségünk megőrzése és a problémák megelőzése is napi rutinná kell hogy válljon. Ezért a Budapest School csapatainak erősen ajánlott a rendszeres visszajelzés, visszatekintés (retrospektív) és a team coaching.