\section{Az iskola kormányzása}
\label{sec:az_iskola_kormanyzasa}
Az iskola szervezetét a tagjai együtt alakítják. A szervezet élő,
folyamatosan változik, a következő alapelvek mentén:

\begin{itemize}

\item
  Gyorsan és folyamatosan tanuló, agilis szervezetként az iskola minden
  nap jobban támogatja a gyerekek tanulását, mint tegnap.
\item
  Minden résztvevő stabilitás, biztonság, kiszámíthatóság iránti igénye
  pont annyira fontos, mint a változás, a javulás, a fejlődés igénye.
\item
  A gyerekeket leginkább ismerő, hozzájuk legközelebb álló tanárok
  (gyerekekkel és szülőkkel erős kapcsolatban), minél több helyzetben
  hoznak döntéseket.
\item
  Az együttműködés, a partneri kapcsolat, a kölcsönös felhatalmazás a
  tanárok, adminisztrátorok között is folytonos, nem csak a tanár-gyerek
  kapcsolatban.
\end{itemize}

Azt szeretnénk, hogy gyerekeink kreatív, környezetüket aktívan alakító,
mély partneri kapcsolatokban élő, problémamegoldó, csapatjátékos,
folyamatosan tanuló, a viágot változtatni tudni felnőttekké váljanak.
Ehhez olyan iskolát építünk, ahol a tanárok ma kreatívak, környezetüket
aktívan alakítják, mély partneri kapcsolatban élnek, problémákat oldanak
meg, csapatban dolgoznak és folyamatosan tanulnak.

Az iskolát, mint szervezetet a \emph{szociokrácia} (sociocracy)
szervezeti modell alapján működtetjük, mert ma így tudjuk leginkább
elérni, hogy egyszerre legyen örömteli és a hatékony az
együttműködésünk. Ez egy olyan, az üzleti világban is kipróbált
dinamikus döntéshozatali rendszer, amely egyszerre segíti a harmonikus
és örömteli közösségi együttműködést és jelent garanciát arra, hogy az
együttműködés hatékony lesz az egyes csapatokon belül.


\subsection{Döntéshozás}
\label{sec:consent_based}

Az iskola minden csapata (egy mikroiskolát működtető tanulásszervezők,
gyerekek egy csapata stb.) a \emph{hozzájáruláson alapúló döntési
mechanizmust} (consent based decision making) használja ahhoz, hogy a
szervezet gyorsan tudjon döntést hozni \emph{és} minden tagja
hallathassa a hangját.

A csapat valamennyi tagja hozhat javaslatokat, ha működési
hatékonytalanságot, feszültséget, problémát talál és van rá megoldása. A
javaslat értelmezése után az érintettek mindegyikét meg kell hallgatni,
hogy \emph{elfogadhatónak} tartja-e a javaslatot, azaz hozzájárul a
változáshoz, mert \emph{,,elég jónak és biztonságosnak találja, hogy
kipróbáljuk az új működést''} (\emph{``is this good enough for now and
safe enough to try?''}). Fontos, hogy mindenki egyenként hallassa
hangját. A javaslat elfogadásra kerül, ha és amikor minden érintett
hozzájárult.

Mindenki kifejezheti a \emph{fenntartásait} (concern), és a csapat
feladata ezeket meghallanni, azokra reagálni. A fenntartás azonban még
nem jelenti a javaslat elutasítását, csak fontos információt ad hozzá a
döntés végrehajtásához.

A javaslatot a csapat nem fogadja el, ha valamelyik csapattag
\emph{ellenzi} (objection) azt. Az ellenzés egy én-üzenet, valami
olyasmi: ,,ha a csapat ezt a döntést meghozná, akkor mélyen sérülne a
csapathoz való elköteleződésem, mert az én igényemet, ami \ldots{}, nem
elégíti ki a javaslat. Nekem szükségem van \ldots{}, ezért inkább
javaslom, hogy \ldots{}'' Fontos, hogy az ellenvetést megfogalmazó
mondja el a saját igényeit, szükségleteit és tegyen új javaslatot, vagy
kérjen segítséget, hogy milyen új javaslatot tehetne. Ellenvetés esetén
a csapat együtt dolgozik azon, hogy új javaslatot találjon, ami az
ellenvetést feloldja, és az eredeti javaslat célja felé viszi a
szervezetet.


\paragraph{Hozzájárulás nem
konszenzus.}

A konszenzus alapú döntések esetén mindenkinek egyet kell értenie abban,
hogy a döntés a legjobb, leghelyesebb, leghelyénvalóbb. A Budapest
School iskolában azt a kérdést tesszük fel inkább, hogy van-e valakinek
ellenvetése és a javaslat kellően biztonságos-e ahhoz, hogy kipróbáljuk.
Nem azt a kérdést tesszük fel, hogy mindenki ezt a döntést hozta volna-e
és mindenki egyetért-e a döntéssel hanem azt, hogy mindenki tudja-e
támogatni a csapat egy másik tagját, és nincs-e olyan ismert kockázat,
ami az egyén vagy a szervezet szempontjából nem vállalható fel.


\paragraph{Hozzájárulás nem
szavazás.}

A Budapest School iskolában nem a többség dönt, és nem az számít, hogy
hányan akarnak egy döntés mellé állni. Mindenki hozhat döntést, amit
elfogad a csapat minden tagja, azaz egyetlen egy ellenvetés sincs.


\paragraph{Az ellenvetés nem
vétó.}

A vetó jog gyakorlatban a döntés megakadályozását jelenti. Amikor valaki
megvétóz egy döntést, akkor azzal a folyamat általában megakad. Az
iskola működésében használt ellenvetés egy beszélgetés megindítását
jelenti: ,,ezt én így nem tudom támogatni, helyette ezt javaslom
inkább''.


\paragraph{A ,,van egy jobb ötletem'', nem
ellenvetés.}

A szervezetnek nem az a feladata, hogy a legjobb döntéseket hozza,
hanem, hogy amikor szükséges, akkor javítson a működésén. Ezért minden
döntéskor mindenkinek azt kell mérlegelnie először, hogy elfogadható-e
neki, hogy azt a bizonyos javaslatot kipróbálja a csapat. Attól, hogy
valaki jobb, más javaslatot is tud, attól még először az eredeti
javaslatot érdemes kipróbálni és tesztelni.


\paragraph{Minden javaslat csak egy
hipotézis.}

Amikor valamit változtatunk a szervezet működésén, akkor egy kísérletbe
vágunk bele: kipróbáljuk, hogy az új működés tényleg jobb-e, megoldja-e
a problémát, feszültséget, kielégíti-e az igényeket. A döntés
támogatásakor ezt a próbálkozást támogatjuk.


\paragraph{Nem döntünk mindenről
együtt.}

A Budapest School iskolában csak az iskola és a mikroiskola működését
megváltoztató kormányzási kérdésekről döntünk együtt. A Budapest School
minden tagja szerepeiből kifolyólag fel van hatalmazva arra, hogy a
mindennapi döntéseit maga meghozhassa, ezért nem kell mindent
megbeszélnünk. A cél, hogy olyan szerepeket és rendszereket alakítsunk
ki, hogy a mindennapi döntéseket mindenki maga meg tudja hozni.


\subsection{Csapatok - az iskola szervezeti
egységei}

A Budapest School csapatai (a szociokrácia terminológiájában a
\emph{körök}) önálló csoportok egy jól meghatározott céllal, felruházott
felelősséggel, döntési körrel. A csapatok maguk határozzák meg a saját
működésüket (policy making) és végzik el a saját feladatukat. Az
iskolában azok döntenek együtt, akik együtt dolgoznak, egy csapatban
(``those who associate together govern together''). És fontos, hogy akik
együtt dolgoznak, azok jól legyenek egymással.

Azt is tudjuk, hogy akik egy munkát elvégeznek, azok a munka szekértői,
így ők tudnak arról a legjobban dönteni, hogy hogyan érdemes a
munkájukat szervezni, alakítani. Nincs főnök, külső szakértő, aki
megmondja egy csapatnak, mit és hogyan csináljanak addig, amíg a rájuk
felhatalmazott kereteken belül maradnak. Az természetes, hogy minden
segítséget, támogatást, információt megkapnak, amire szükségük van. De a
kormányzás az ő kezükben van.


\paragraph{Mikroiskola tanulásszervező
csapata.}

A Budapest School szervezet állandó csapatai az egy-egy mikroiskolát
vezető tanulásszervezők csapata, ami egy \emph{szocikratikus kör}. A
mikroiskola gyerekeinek (családjainak) és tanárainak életét meghatározó
döntéseket maguk hozzák meg. Így például a napirend, a csoportbontások,
a szülői értekezletek tematikája a saját döntéseik alapján alakul ki.
Fontos, hogy a csapat tagok maguk tudják meghatározni kivel tudnak és
akarnak együtt dolgozni, mikor és mit akarnak csinálni.


\paragraph{Csapatok kapcsolódása}

Egy-egy ember több csapatnak is tagja lehet. Egyrészt munkacsoportok
alakulhatnak egy-egy feladat elvégzésére és a Budapest Schoolban egy
ember több részfeladatot is ellát. Másrészt a csapatokat kifejezetten
úgy alakítja a közösség, hogy legyenek közte kapcsolódások, olyan tagok,
akik összekötik a csapatokat.

Vannak olyan csapatok, melyek elsődleges célja, összekötni a kisebb
csapatokat. Például minden mikroiskolai csapat delegál egy képviselőt az
iskola közös naptárát létrehozó munkacsoportba.


\paragraph{Csapatok vezetői}

Minden csapatnak van egy \emph{vezetője} (szociokrácia terminológiában a
\emph{circle leader}). A vezető feladata, hogy mindenki ismerje a csapat
célját, a \emph{,,miért létezünk?"} kérdésre a választ és hogy a csapat
működjön: tiszták legyenek a szerepek és megtörténjen az, amiben
megállapodott a csapat és működjenek és fejlődjenek a folyamatok.

A csapatvezető a Budapest School rendszerében nem az, aki megmondja, ki
mit csináljon, nem osztja, ellenőrzi vagy felügyeli a feladatokat, nem
rúg ki, és nem vesz fel embereket, hanem szolgálja a csapatot (servant
leadership) azzal, hogy segíti a megállapodásokat betartani: facilitál,
moderál, szintetizál, kísér, kérdez. A csapatvezető megválasztásához,
mint minden szerep megválasztásához a csapat minden tagjának
hozzájárulása szükséges.

% Strukturál, összeszervez, koordinál, levezet, emlékeztet.


%Learning architect
%ne legyen bystander
%Ő sem autoritas,  hanem javasol, tematizál, szempontot hoz.

%az, aki arról beszél, hogy a gyerekek igyényeinek milyen struktúrák, keretek, %csoportbontások, modulok, tevékenyslg felelnek meg leginkább 



\subsection{Mi van, amikor egy csapat nem tud döntést hozni,
együttműködni?}

Amikor a csapattagok úgy érzik, hogy nem tudnak mindenki számára
elfogadható döntéseket hozni, nem haladnak, vagy megjelentek a játszmák
és ezért már nem tudják a csapatcélját szolgálni, akkor konfliktus,
feszültség alakul ki, aminek feloldásához segítséget hívhatnak be a
szervezet többi részétől (lásd .~fejezet).

A csapat folyamatos harmóniájáért folyamatosan dolgozni kell, ahogy az
egészségünk megőrzése és a problémák megelőzése is napi rutinná kell
hogy váljon. Ezért a Budapest School csapatainak erősen ajánlott a
rendszeres visszajelzés, visszatekintés (retrospektív) és a team
coaching.