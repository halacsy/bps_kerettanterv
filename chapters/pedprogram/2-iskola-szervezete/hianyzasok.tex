\section{Hiányzások, mulasztások, igazolások, késések}

A Budapest School feladata, hogy olyan környezetet biztosítson a gyerekeknek, amiben boldogak, felszabadultak, magabiztosak és hatékonyak tudnak lenni. Budapest School családok maguk és önszántukból választják ezt az iskolát, áldoznak sok időt és energiát arra, hogy az iskolában tudjanak tanulni. \emph{Ezért az iskola feltételezi, hogy a gyerekek az iskolában akarnak tanulni önszántukból}.

Sok oka lehet annak, hogy egy gyerek még sincs az iskolában. Például
\begin{itemize}
    \item betegnek, fáradtnak érezheti magát, fizikailag vagy lelkileg kimerült, vagy lehet valamilyen fertőző betegsége;
    \item családjával tölt értékes, minőségi időt, mert fejlődését ez szolgálja a leginkább;
    \item előre nem tervezett esemény miatt nem tud az iskolába menni;
    \item utazik, felfedez, külső helyszínre szervezett tanulási programokon vesz részt;
    \item egy projektjébe úgy belemerül, hogy érdemesnek találja nem az iskolában, fóku\-száltan végezni a munkát.
 \end{itemize}

A fenti példák is két jól elkülöníthető kategóriába sorolhatóak. A \emph{nem tervezett hiányzástól} jól elkülöníthetőek azok az esetek, amikor a gyerek, bár nem az iskolában tartozkodik, mégis szervezett, struktúrált módon biztosított a fejlődése, tanulása. Ezt az esetet a \emph{,,távmunka''} mintájára \emph{,,távtanulásnak''} hívja az iskola, mert ezt az időt is tanulásra szánjuk.

Alapelv, hogy \emph{a mentornak, gyereknek és szülőnek meg kell állapodni a távtanulásról}. Minden félnek tudnia kell róla, meg kell előre tervezni és nem lehet esetleges. Mindenkép meg kell különböztetni a \emph{nem tervezett hiányzástól}.

\paragraph{Tervezett távtanulás} tehát az, amikor előre eltervezett módon, valamilyen program miatt nincs a gyerek az iskolában. Ilyenkor a mentor és a gyerek megtervezi a tanulás célját, várható eredményeit. A terv létrejöttéért a gyerek és a szülő felelős, és minden félnek el kell fogadnia a tervet. Tehát a mentortanárnak hozzá kell járulnia. Ha a mentortanár nem járul hozzá, akkor addig nem kezdhető meg a távtanulás, amig \aref{sec:konfliktusok_kezelese}.~fejezet alapján megállapodás nem születik.\footnote{Ha mégis, akkor azt nem tervezett hiányzásnak kell tekinteni.}

Ha a tervezett távtanulás elérte a 20 napot, vagy 160 órát, akkor a mentortanár mellett egy másik mentortanár szerepben dolgozó tanulásszervezőnek is meg kell ismernie és el kell fogadnia a tervet. 40 nap felett három mentortanárnak kell együtt elfogadnia a tervet, melyek egyike egy másik mikroiskola mentortanára. Eföllötti távtanulás már egyéni tanrendnek vagy egyéni munkarendnek számít, amire más szabályok vonatkoznak.


\paragraph{Nem tervezett hiányzás} A gyerek és a mentortanár nem tud előre felkészülni az iskolán kívüli tanulásra, mert a hiányzás előző nap vagy aznap derül ki, vagy más okból a megállapodás nem jön létre. A szülő feladata, hogy még ebben az esetben is erről reggel 9 óra előtt értesítse a mentortanárt. Mikroiskolánként eltérhet a preferált kommunikációs eszköz, ezért a tanulásszervezők feladata meghatározni, hogyan kérik az értesítés formáját.

Egy tanévben 15 munkanap, vagy 120 óra, de alkalmanként csak 5 munkanap, nem megtervezett hiányzást igazolhat a szülő, 
(rögzített és dokumentált módon). Orvos által igazolt betegség, hatósági intézkedés és egyéb alapos indok esetén a 20/2012.~(VIII.~31.) EMMI-rendelet 51.~§~(2) értelmében igazoltnak kell tekinteni a hiányzást.

\paragraph{Igazolatlan hiányzásnak} azt az esetet kell tekinteni, amikor a szülő vagy a mentortanár nem tudott a hiányzásról, nem volt előre megtervezve, vagy a 15 napos, 120 órás keret kimerült. Ebben az esetben az iskola szigorúbban jár el, mint általában más iskolák. Ilyenkor a 20/2012. (VIII.~31.) EMMI-rendelet 51.~§~(3) pontja értelmében minden esetben az iskola értesíti a szülőt, és 10 igazolatlan óra után figyelmezteti, hogy a következő igazolatlan után ,,az iskola a gyermekjóléti szolgálat közreműködését igénybe véve megkeresi a tanuló szülőjét'' (idézés az EMMI-rendeletből). Az iskola megközelítése egyszerű: mivel partneri viszonyban van a tanár, gyerek és szülő, ezért az alapértelmezett az, hogy vagy előre meg lehetett volna beszélni a hiányzásokat, amely esetben tervezett távtanulásról beszélnénk, vagy betegség miatt kellett túllépni a 15 napot. Elég tág keretet enged az iskola. Abban az esetben azonban, amikor a gyerek vagy a szülő nem tartja be a kereteket, nem él a partneri viszonnyal, akkor ott valami baj van. Gyorsan kell reagálni.

\subsubsection{Hogyan biztosítja a rendszer a visszaélések kiküszöbölését?}
A Budapest Schoolba járó gyerek szülei és tanárai egyetlen igazolatlan óra hiányzás után értesítést kapnak arról, hogy a gyerek nem jelent meg az iskolában. Tehát egy gyerek nem tud szülei tudta nélkül távolmaradni.

Egy szülő 15 napon keresztül ,,igazolhat'' nem tervezett hiányzást, hogy ne kelljen minden náthánál a körzeti orvosi rendszert terhelni, ahogy azt a Házi Gyermekorvosok Egyesülete javasolja. Miért feltételezzük, hogy nem él ezzel vissza a szülő és a gyerek?  Mert az iskolában maradás feltétele a tanulás és a folyamatosan újra felállított tanulási célok követése. Az ezzel való visszaélés az iskola céljaival ellentétes és legkésőbb a soron következő trimeszter tanulási szerződésekor a felszínre kerül.

A távtanulást pedig nem tekinti az iskola hiányzásnak, mert a tanulás folytatólagos, dokumentált, megtervezett. A portfóliók bővülését pedig folyamatosan monitorozza az iskola. A mentortanár és a gyerek tévesen mérheti fel a helyzetet, és előfordulhat, hogy tanulásnak, fejlődésnek látnak valamit, ami nem az. Ezért került a rendszerbe a ,,külső megfigyelő'' kitétel, hogy 20 nap után új tanárt kell bevonni a döntésbe. Ha a gyerek tanulási veszélybe kerül, akkor az a tanulási eredmények elmaradásából fél éven belül felfedezhető.

\subsection{Késések kezelése}
A Budapest School mikroiskolák maguk állítják fel a napirenddel kapcsolatos kereteket: mikor kezdenek, meddig tartanak a strukturált foglalkozások, mikor vannak a szünetek és hogyan kezdődik újra a nap folyamán a fóku\-szált munka. A kereteket a tanulásszervező tanárok feladata kialakítani és trimeszterenként megkezdése előtt kihirdetni.

Fontos, megbeszélendő részlet, hogy hogyan kezeli a közösség a késéseket: mikortól lehet érkezni, mikor kezd a közösség annyira dolgozni, hogy zavaró, amikor valaki belép és megzavarja a folyamatot. Megállapodást köt a közösség, hogy hogyan kívánja kezelni a késéseket, mi segíti a csapatot leginkább a céljai elérésében.

Az iskola nem regisztrálja a késéseket, mert az iskola nem tudhatja, hogy egy-egy késés elfogadható-e a közösségnek vagy nem. Egy színdarab főpróbájáról 5 percet késni mást jelent a közösség számára, mint arról az óráról, ahol mindenki egyedül füllhallgatóval böngészi egy online tananyag számára legrelevánsabb fejezetét.

Ha egy csoportot megzavar valakinek az ismételt késése, akkor konfliktus alakul ki a csoport és a késő vagy a tanár és a késő között. Ezt a típusú konfliktust (is) \aref{sec:konfliktusok_kezelese}.~fejezet szerint kell feloldani.