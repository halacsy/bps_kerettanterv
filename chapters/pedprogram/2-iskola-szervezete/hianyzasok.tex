\section{Hiányzások, mulasztások, igazolások}
A Budapest School feladata, hogy olyan környezetet biztosítson a gyerekeknek, amiben boldogak, felszabadultak, magabiztosak és hatékonyak tudnak lenni. Budapest School családok ezért választják ezt az iskolát, áldoznak sok pénzt és időt arra, hogy az iskolában tudjanak tanulni. Ezért az iskola feltételezi, hogy a gyerekek az iskolában akarnak lenni önszántukból.

Sok oka lehet annak, hogy egy gyerek még sincs az iskolában. Például
\begin{itemize}
    \item betegnek, fáradtnak érezheti magát, fizikailag vagy lelkileg kimerült, vagy lehet valamilyen fertőző betegsége;
    \item családjával tölt értékes, minőségi időt, mert fejlődését ez szolgálja a leginkább;
    \item utazik, felfedez, külső helyszínre szervezett tanulási programokon vesz részt;
    \item egy projektjébe úgy belemerül, hogy érdemesnek találja otthon, fóku\-száltan végezni a munkát (home office);
    \item előre nem tervezett esemény miatt nem tud az iskolába menni.
\end{itemize}

A nem iskolában töltött időt a ,,home office'' mintájára ,,otthon tanulásnak'' hívja az iskola, mert feltételezi, hogy a nem iskolában töltött idő is tanulással jár.

Alapelv, hogy \emph{a mentornak, gyereknek és szülőnek meg kell állapodni az otthon tanulásról}. Minden félnek tudnia kell róla, és meg kell különböztetni a tervezett otthon tanulást a nem tervezett hiányzástól.

\paragraph{Tervezett otthon tanulás} Előre eltervezett módon, valamilyen program miatt nincs a gyerek az iskolában. Ilyenkor a mentor és a gyerek megtervezi az otthon tanulás célját, várható eredményeit. A terv létrejöttéért a gyerek és a szülő felelős, és minden félnek el kell fogadnia a tervet.

\paragraph{Nem tervezett hiányzás} A gyerek és a mentortanár nem tud előre felkészülni az iskolán kívüli tanulásra, mert a hiányzás előző nap vagy aznap derül ki, vagy más okból a megállapodás nem jön létre. A szülő feladata, hogy még ebben az esetben is erről reggel 9 óra előtt értesítse a mentortanárt. Mikroiskolánként eltérhet a preferált kommunikációs eszköz, ezért a tanulásszervezők feladata meghatározni, hogyan kérik az értesítés formáját.
Ha a tervezett otthon tanulás elérte a 20 napot, vagy 160 órát, akkor a mentortanár mellett egy másik mentortanár szerepben dolgozó tanulásszervezőnek is meg kell ismernie és el kell fogadnia a tervet. 40 nap felett három mentortanárnak kell együtt elfogadnia a tervet, melyek egyike egy másik mikroiskola mentortanára.

Egy tanévben 15 nap, vagy 120 óra, nem megtervezett hiányzást igazolhat a szülő (rögzített és dokumentált módon). Orvos által igazolt betegség, hatósági intézkedés és egyéb alapos indok esetén a 20/2012.~(VIII.~31.) EMMI-rendelet 51.~§~(2) értelmében igazoltnak kell tekinteni a hiányzást.

Igazolatlan hiányzásnak azt az esetet kell tekinteni, amikor a szülő vagy a mentortanár nem tudott a hiányzásról, nem volt előre megtervezve, vagy a 15 napos, 120 órás keret kimerült. Ilyenkor a 20/2012. (VIII.~31.) EMMI-rendelet 51.~§~(3) pontja értelmében minden esetben az iskola értesíti a szülőt, és 10 igazolatlan óra után figyelmezteti, hogy a következő igazolatlan után ,,az iskola a gyermekjóléti szolgálat közreműködését igénybe véve megkeresi a tanuló szülőjét'' (idézés az EMMI-rendeletből). Az iskola megközelítése egyszerű: mivel partneri viszonyban van a tanár, gyerek, szülő, ezért meg tudják beszélni a hiányzásokat. Elég tág keretet enged az iskola. Abban az esetben azonban, amikor a gyerek vagy a szülő nem tartja be a kereteket, nem él a partneri viszonnyal, akkor ott valami baj van. Gyorsan kell reagálni.

\subsection{Késések kezelése}
A Budapest School mikroiskolák maguk állítják fel a napirenddel kapcsolatos kereteket: mikor kezdenek, meddig tartanak a strukturált foglalkozások, mikor vannak a szünetek és hogyan kezdődik újra a nap folyamán a fóku\-szált munka. A kereteket a tanulásszervező tanárok feladata kialakítani és trimeszterenként kihirdetni.

Fontos, megbeszélendő részlet, hogy hogyan kezeli a közösség a késéseket: mikortól lehet érkezni, mikor kezd a közösség annyira dolgozni, hogy zavaró, amikor valaki belép és megzavarja a folyamatot. Megállapodást köt a közösség, hogy hogyan kívánja kezelni a késéseket, mi segíti a csapatot leginkább a céljai elérésében.

Az iskola nem regisztrálja a késéseket, mert az iskola nem tudhatja, hogy egy-egy késés elfogadható-e a közösségnek vagy nem. Egy színdarab főpróbájáról 5 percet késni mást jelent a közösség számára, mint arról az óráról, ahol mindenki egyedül füllhallgatóval böngészi egy online tananyag számára legrelevánsabb fejezetét.

Ha egy csoportot megzavar valakinek az ismételt késése, akkor konfliktus alakul ki a csoport és a késő vagy a tanár és a késő között. Ezt a típusú konfliktust (is) \aref{sec:konfliktusok_kezelese}.~fejezet szerint kell feloldani.