\section{Hiányzások, mulasztások, igazolások}
A Budapest School feladata, hogy olyan környezetet biztosítson a gyerekeknek, amiben boldogak, felszabadultak, magabiztosak és hatékonyak tudnak lenni. Budapest School családok ezért választják ezt az iskolát, áldoznak sok pénzt és időt arra, hogy az iskolában tudjanak tanulni. Ezért az iskola feltételezi, hogy a gyerekek az iskolában akarnak lenni önszántukból.

Sok oka lehet annak, hogy egy gyerek nem az iskolában van. Például
\begin{itemize}
    \item érezheti betegnek, fáradtnak magát, fizikailag nem elég erősnek vagy épp fertőzőnek;
    \item családjával tölthet értékes, minőségi időt;
    \item utazhat, felfedezhet, külső tanulási programokon vehet részt;
    \item egy projektjébe úgy belemerült, hogy érdemesnek találja otthon, fókuszáltan végezni a munkát (home office terjed) \item nem tervezett módon valami közbejöhet.
\end{itemize}

A nem iskolában töltött időt a ,,Home office'' mintájára ,,otthon tanulásnak'' hívja az iskola, mert feltételezi, hogy a nem iskolában töltött idő is tanulással jár.

Alapelv, hogy a mentornak, gyereknek és szülőnek meg kell állapodni az otthon tanulásról. Minden félnek tudnia kell róla, és meg kell különböztetni a tervezett otthontanulást és a nem tervezett hiányzástól.

\paragraph{Tervezett otthontanulás} Előre eltervezett módon, valamilyen program miatt nincs a gyerek az iskolában. Ilyenkor a mentor és a gyerek megtervezi az otthontanulás célját, várható eredményeit. A terv létrejöttéért a gyerek és a szülő felelős és minden félnek el kell fogadnia a tervett.

\paragraph{Nem tervezett hiányzás} A gyerek és a mentortanár nem tud előre felkészülni az iskolán kívüli tanulásra, mert a hiányzás előző nap vagy aznap derül ki vagy más okból a megállapodás nem jön létre. A szülő feladata, hogy még ebben az esetben is erről reggel 9 óra előtt értesítse a mentortanárt. Mikroiskolánként eltérhet a preferált kommunikációs eszköz, ezért a tanulásszervezők feladata meghatározni, hogyan kérik az értesítés formáját.

Ha a tervezett otthontanulás elérte a 20 napot, vagy 160 órát, akkor a mentortanár mellett egy másik mentortanár szerepben dolgozó tanulásszervezőnek is meg kell ismernie és el kell fogadnia a tervet. 40 nap felett három mentortanárnak kell együtt elfogadnia a tervet és így tovább.

Egy tanévben 15 napot naponta, vagy 120 órát, a szülő igazolhat nem tervezett hiányzást. Orvos által igazolt betegség, hatósági intézkedés és egyéb alapos indok esetén 20/2012. (VIII.~31.) EMMI rendelet 51.~§~(2) értelmében igazoltnak kell tekinteni a hiányzást.

Igazolatlan hiányzásnak az esetet kell tekinteni, amikor a szülő vagy a mentor tanár nem tudott a hiányzásról vagy a 15 napos keret kimerült. Ilyenkor a 20/2012. (VIII.~31.) EMMI rendelet 51.~§~(3) pontja értelmében minden esetben az iskola értesíti a szülőt, és 10 igazolatlan óra után figyelmezteti, hogy a következő igazolatlan után ,,az iskola a gyermekjóléti szolgálat közreműködését igénybe véve megkeresi a tanuló szülőjét".
