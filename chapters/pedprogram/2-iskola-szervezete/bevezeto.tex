\todo{Lehet, hogy ez a bevezeto nem kell, vagyis biztos, hogy ossze kell fesulni a kerettanterv bevezetojevel}
Az oktatás tartalmának előzetes szabályozása helyett a Budapest School a
tanulás módjára helyezi a hangsúlyt. Az iskola alapelve,
hogy
integratív módon folyamatosan fejlessze azokat a pedagógiai, pszichológiai és
szervezetfejlesztési módszereket,
amelyek korszerű módon tudják segíteni a tanulás tanulását, az egyéni és
csoportos fejlődést, a konfliktusok feloldását.

A tanulás tartalmát tekintve a Budapest School saját alternatív
kerettantervére támaszkodik, amely a tanulás rendszerét, annak
folyamatát szabályozza. E dokumentum alapján az állami kerettanterv
tanulási eredményein történő végighaladás mellett a Budapest School
nagy hangsúlyt fektet a gyerekek saját tanulási céljaira és a
célállítás módjára.

A gyerekek életkori, fejlettségi, valamint szociális és érzelmi
állapotának figyelemmel kísérésében, valamint a saját tanulási célok
kitűzésében és elérésében minden gyereket egy \emph{mentortanár} kísér végig
az iskolán. A mentortanár személye változhat
az iskolában töltött évek
alatt, de arra is van lehetőség, hogy ugyanaz a mentor kísérjen végig
egy gyereket az általános iskola első évétől az érettségiig.