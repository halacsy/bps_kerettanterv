\paragraph{A tanulás rendszerszemléletű megközelítése}
Az oktatás tartalmának
előzetes szabályozása helyett a Budapest School a tanulás módjára helyezi a hangsúlyt. Az iskola alapelve, hogy integratív módon folyamatosan keresse és fejlessze a pedagógiai, pszichológiai és szervezetfejlesztési módszereket, amelyek korszerű módon tudják segíteni a tanulás tanulását, az egyéni és csoportos fejlődést, a konfliktusok feloldását.

A tanulás tartalmát tekintve a Budapest School a miniszter által kiadott kerettantervekre támaszkodik. A Budapest School model pedig a tanulás rendszerét, annak folyamatát szabályozza. E dokumentum alapján az állami kerettanterv tanulási eredményein történő végighaladás mellett a Budapest School nagy hangsúlyt fektet a gyerekek saját tanulási céljaira és a célállítás módjára.
