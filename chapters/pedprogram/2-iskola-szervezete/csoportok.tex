\section{A csoportbontások}
\label{sec:csoportok}

A Budapest School iskoláit mikroiskolák közösségeiből hozzuk létre. Így egy gyerek elsődleges csoportja a mikroiskolájának közössége, ami lehet 6--60  gyerek. Ezen belül modulonként eltér, hogy milyen csoportbontásban dolgoznak. Több szintje van a csoportmunkának.
\begin{enumerate}
      \item A mikroiskola közössége heti rendszerességgel tarthat iskolagyűlést, fórumot, plenárist, iskolakonferenciát. Ilyenkor a mikroiskola közössége dolgozik együtt.
      \item  Modulokra kisebb csoportok jelentkezhetnek. Egy modul csoportjának rendezőelve lehet:
            \begin{enumerate}
                  \item egy képességszinten lévő gyerekek tanulnak együtt;
                  \item a közös érdeklődés hozza össze a csoporttagokat;
                  \item  direkt a véletlenszerűségben van az érdekesség, mert keveredni akarnak;
                  \item kölcsönös szimpátia és vonzalom a modultagok között: most azért vannak egy csoportban, mert egy csoportban akartak lenni.
            \end{enumerate}
      \item  Egy-egy foglalkozáson belül is sokszor csoportot alkotunk, az előző elvek alapján.

\end{enumerate}

Arra is lehetőség van, hogy egy csoport tagjai több mikroiskolából álljanak össze, ha az támogatja a tanulást és az utazás biztonságosan megoldható.