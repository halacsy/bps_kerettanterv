\section{A felvétel és az átvétel}
\label{sec:felvetel-atvetel}
Egy mikroiskola közösségéhez bármikor lehet csatlakozni, ha és amikor a
csatlakozó
család ezt szeretné, és ha ettől a közösség minden tagjának valamiért  jobb
lesz vagy nem változik (de rosszabb nem lehet).
Az iskola nem azért fogad be valakit, mert
ezt kell, hanem mert a mikroiskola közössége ezt szeretné.
A családok nem azért csatlakoznak, mert valamit kell találni a gyereknek, hanem
mert
szeretnének a Budapest School egyik mikroiskola közösségéhez tartozni.

A 12 évfolyamos egységes iskolába bármikor lehet csatlakozni, az iskola
normálisnak tartja, hogy a közösség tagjai változnak.
Ezért az iskolában egy iskolát most kezdő 6 éves felvétele, egy 8 éves, az előző
iskoláját nem kedvelő felvételi kérelme, egy 9 éves
külföldről hazaköltöző év közbeni csatlakozása, egy 12 éves ,,gimnáziumba''
jelentkezése és egy 16 éves más városból érkező  között az iskola
számára
a \emph{felvételi és átvételi folyamatot} tekintve nincs különbség.

Kizárólag egy mikroiskola tanulásszervezőinek és a fenntartónak
a\linebreak
hozzájárulása
szükséges ahhoz, hogy
egy család csatlakozzon egy mikroiskolához.

Egy család jelentkezése után legalább három dolognak kell történnie.
\begin{itemize}
      \item A családnak meg kell ismernie a Budapest School alapelveit,
            működését, jellegzetességeit. Az iskolának meg kell mutatnia
            önmagát. A
            családnak meg kell értenie,  és meg kell fogalmaznia, hogy miért
            akarnak
            csatlakozni a közösséghez.
      \item A tanulásszervezőknek meg kell ismerniük a családot, megnézni, hogy
            ,,működik-e a kémia'', tudják-e vállalni a gyerek tanulásának
            támogatását.
      \item A gyereknek időt kell eltöltenie a mikroiskolában, a mindennapokhoz
            minél inkább hasonló körülmények között, hogy mindenki meg tudja
            tapasztalni,
            érezni, hogy milyen lenne együtt és egymástól tanulni.
\end{itemize}

\paragraph{Szempontok a döntéshez}
A mikroiskolák közössége legyen minél inkább diverz és kiegyensúlyozott: kevert
korosztályú, kevert nemi, kevert szociális státuszú, kevert
érdeklődésű, kevert személyiségjegyű csoportok úgy, hogy legyen egy erős,
mindenkit megtartó szociális
hálózat. A mikroiskolák közösségét egyenként kell kiegyensúlyozni.%\eject

Így az is előfordulhat, hogy egy gyerek egy mikroiskola közösségében nem talál
helyet magának, de az iskola egy másik mikroiskolájában igen. Mert a
közösségek különbözőek.\footnote{Legegyszerűbb példa: van, ahol több lányt
      szeretnénk, mint ma, és van, ahol több fiút, és van, ahol ez most nem
      szempont.}

\paragraph{Nincs felvételi vizsga.}
Az iskola nem követel meg se írásbeli (központi), se szóbeli
felvételit, és nem is az előző iskolák osztályzatai alapján dönt. Egyetlen
szempont, hogy jobban tud-e egy mikroközösség működni egy gyerek/család
csatlakozásával. A döntést a mikroiskola tanárai, a fenntartó (vagy
delegáltja) és a család hozzák meg.

%A mikroiskolák folyamatosan veszik fel a gyerekeket. Az iskolában nincs egy nap, amikor a felvéte

\paragraph{Szakítás, távozás, elengedés}
Működésünk része, hogy konfliktusok, kényelmetlenségek, változó körülmények
között, aki egyszer csatlakozott, az egyszer távozhat is a közösségből.
\begin{itemize}
      \item Az iskola, a tanárok és a család alapelvei, értékei közötti
            különbségek okozhatnak annyi és olyan konfliktust, amit már nem
            tudnak a felek
            feloldani.
      \item Van, hogy egy gyerek nem találja meg a helyét, vagy épp valamiért
            elkezd a közösségben ,,nem boldog'' pozícióba kerülni. Vagy épp a
            közösség
            többi tagjának lesz kényelmetlen az együttlét.
      \item Családok élete, vágyai, motivációjuk, körülményei
        változhatnak\linebreak
        úgy,
            hogy épp más közösségben jobb helyet találnának.
\end{itemize}

Bármi legyen is az ok, a távozás, szakítás feszültséggel teli szituáció. Ezért
is fontos, hogy minden fél betartsa  \aref{sec:konfliktusok_kezelese}.
fejezetben leírt konfliktuskezelési szokásokat.
