\section{Külső modulvezetők aránya}
A pedagógia program egy fontos megkötést ad a modulok vezetésére: a mikroiskola tanulásszervezőinek kell vezetnie a gyerekek moduljainak nagy részét, másképp fogalmazva korlátozva van a külsős, nem tanulásszervezők által tartott modulok óraszáma, ahogy ezt \aref{tbl:belso_modulok}.~táblázat mutatja.  Ennek a megkötésnek az az oka, hogy
\begin{itemize}
    \item Kisebb korban szeretnénk, ha kevesebb, állandóbb felnőttekkel találkoznának a gyerekek (két tanítós rendszer mintájára).
    \item Biztosan találkozzak elég időt a gyerekekkel az iskolát vezető, irányító tanulásszervezők.
    \item Mindenképp legyen szoros kapcsolatuk a gyerekeknek pedagógusvégzettséggel bíró gyerekekkel.
    \item Érettségihez közeledve legyen lehetőség minél több külsős, akár speciális szaktudással bíró embertől tanulni.
\end{itemize}

\begin{table}[ht]
    \begin{center}
    \begin{tabular}{l|c|c|c|c}
        évfolyamszint             & 1--4 & 5--8 & 9--11 & 12   \\ \hline
        ,,belsős'' modulok aránya & 60\% & 50\% & 50\%  & 40\%
    \end{tabular}
\end{center}
    \caption{A mikroiskola tanulásszervezői által vezetett modulok aránya évfolyamszintenként.}
    \label{tbl:belso_modulok}
\end{table}

