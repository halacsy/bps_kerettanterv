\section{Külső modulvezetők aránya}
A pedagógiai program egy fontos megkötést ad a modulok vezetésére: a mikroiskola tanulásszervezőinek kell vezetnie a gyerekek moduljainak nagy részét, másképp fogalmazva korlátozva van a külsős, nem tanulásszervezők által tartott modulok óraszáma, ahogy ezt \aref{tbl:belso_modulok}.~táblázat mutatja.  Ennek a megkötésnek az az oka, hogy
\begin{itemize}
    \item Kisebb korban szeretnénk, ha kevesebb, állandóbb felnőttekkel találkoznának a gyerekek (két tanítós rendszer mintájára).
    \item Biztosan találkozzanak elég időt a gyerekekkel az iskolát vezető, irányító tanulásszervezők.
    \item Mindenképp legyen szoros kapcsolatuk a gyerekeknek pedagógus végzettséggel bíró tanárokkal.
    \item Érettségihez közeledve legyen lehetőség minél több külsős, akár speciális szaktudással bíró embertől tanulni.
\end{itemize}

\begin{table}[ht]
    \begin{center}
        \begin{tabular}{l|c|c|c|c|c}
            évfolyamszint                  & 1--2 & 3-4 & 5--8 & 9--11 & 12   \\ \hline
            ,,belsős'' modulok min. aránya & 70\% & 60  & 55\% & 50\%  & 40\%
        \end{tabular}
    \end{center}
    \caption{A mikroiskola tanulásszervezői által vezetett modulok aránya évfolyamszintenként.}
    \label{tbl:belso_modulok}
\end{table}

\section{Modulok nyomonkövetése}
Minden trimeszter megkezdésekor a tanulásszervező tanárok meghirdetik a kötelező, a kötelezően válaszható és a választható modulokat, azaz rögzítik, hogy
\begin{itemize}
    \item ki a modul vezetője és melyik pedagógus munkakörben alkalmazott tanulásszervező felelős (elszámoltatható a RACI\footnote{https://www.pmsz.hu/hirek-aktualitasok/havi-mustra/havi-mustra-a-felelosseg-hozzarendelesi-matrixrol} menedzsment rendszer értelmezésében) a modulért;
    \item mi a modul célja, keretei és várható eredményei;
    \item hol, mikor és milyen rendszerességű foglalkozások lesznek;
    \item és mi a részvétel feltétele (előzetes tudás, nivó szint, kor, maximum létszám), azon belül, hogy teljes folyamatra kell-e elköteleződni, vagy esetileg is lehet a modult látogatni.
\end{itemize}
Ezután eldöl, hogy ki mikor  melyik modulon vesz részt. Ennek rendszerét a tanárok alakítják ki: beoszthatják a gyerekeket, ahogy ők ezt jónak látják, vagy épp hagyatkozhatnak a gyerekek választására. A lényeg, hogy alakuljon ki a rendszer a trimeszter megkezdése előtt.

\paragraph{Nyomonkövethetőség}
Az őszi első trimeszterben  -- a kerettanterv szerint -- a mikroiskola ismerkedéssel kezd. Ebben a trimeszterben október 1. a modulrendszer felállításának határideje. A második trimeszter esetén január 1. és tavasszal április 1. a határidő. Ezektől a határidőktől kezdve kialakuló rendszerben minden nap lehet tudni, hogy ki, hol, kivel, melyik modulok keretében tanul. Azaz kialakul a gyerekek órarendje, ami nagyon hasonló a megszokott órarendekhez: \emph{mikor melyik foglalkozáson és hol vagyok}.

A Budapest School iskolában annyi a különbség, hogy a személyreszabhatóság miatt az egy mikroiskolába járó gyerekek órarendje akár nagy mértékben is eltérhet egymástól. Tehát itt nem egy osztálynak és mikroiskolának van órarendje, hanem a gyerekeknek van saját órarendje. A fenntartó felelőssége kialakítani azt a számítógépes rendszert, ami a gyerekek órarendjét a gyerekek, szülők, tanárok, mikroiskolák és a teljes Budapest School iskola szintjén átláthatóvá és nyomonkövethető teszi.

\subsection{Inkrementális fejlesztés}
A moduláris rendszer nagy szabadságot enged a tanároknak abban, hogy a mindennapokat olyan tevékenység köré szervezzék, ami szerintük a gyerekek tanulását a legjobban szolgálja. Ez a szabadság bizonytalanságot is adhat: ha bármilyen modult szervezhetünk, akkor milyen modult szervezzünk? A pedagógia program a tanároknak azt javasolja, hogy induljanak ki egy számukra ismert, stabil rendszerből, és azt fejlesszék lépésről lépésre.

Sokaknak a legbiztosabb alap a jól ismert rendszer: a miniszter által kiadott kerettantervek tantárgyaiból létrehozott modulok, amik követik a kiadott kerettanterv tematikus egységeit. A moduláris rendszer megengedi, hogy ugyanolyan tanórákat szervezzünk, mint a miniszter által kiadott kerettantervek alapján működő iskolák. Van, amikor innen indulva, lassan, trimeszterenként változtatva érhetjük el a legjobb eredményt: ezután össze lehet vonni tantárgyakat és egy modulba szervezni például a magyar nyelv és irodalmat és a történelmet egy trimeszterre, vagy az összes természettudományi tantárgyat egy kísérletezős modulba. Lehet tömbösítve vagy epochálisan szervezni a mindennapi tanulást.

A modulrendszer le tudja fedni a szakkörök, iskola utáni foglalkozások, nyári táborok rendszerét is. Egy mikroiskola a szokásos tantárgyi modulok mellé meghirdethet más iskolákban szakkörnek nevezett modulokat: robotika, néptánc, foci edzés. A modul rendszer sajátja, hogy ezeket a szakköröket ugyanúgy tudja kezelni, mint a történelem érettségi felkészítő fakultációt.

A modul rendszer lehetővé teszi a projektpedagógiával való szabad kísérletezést is. Lehet olyan modulrendszert alkotni, ahol minden páros héten projekteken dolgoznak a gyerekek, a páratlan héten pedig klasszikus tantárgyi struktúrák mentén szervezett modulokban haladnak az akadémia tudás elsajátításával.

\paragraph{Biztonságos felfedezés}
A tanárok bátorítva vannak egy olyan saját, rugalmas struktúra kialakításában, amely jól működik és biztonságot nyújt mind számukra, mind pedig a gyerekek és a szülők számára. A Budapest School tanulásmonitoring rendszer miatt mindig tudjuk, hogy egy gyerek egy-egy tantárgy tanulási eredményeivel hogyan haladt. És ez biztonsági hálót ad a tanároknak: mindig tudjuk, hogy a gyerekek milyen irányban haladnak, lemaradtak-e valamiből, előre szaladtak-e valami másból. Egyben folyamatos visszajelzést ad a modulstruktúráról. Ezért mondhatjuk, hogy nyugodtan kereshetjük a jobb struktúrát, a tökéleteset sose tudjuk elérni (''better, never the best'').