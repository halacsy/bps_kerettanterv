\begin{quote}
      Az iskolában folyó nevelő-oktató munka pedagógiai alapelvei, értékei,
      céljai,
      feladatai, eszközei, eljárásai.

      Ide kene valami frappans idezet. Talan Sugata Mitratol. Mondjuk ilyen We
      need
      to look at learning as the product of educational self-organization. It’s
      not
      about making learning happen; it’s about letting it happen.
      !!! If a child can't learn the way we teach, maybe we should teach the
      way
      they learn!!!

      !!! We are what we repeatedly do. Excellence therefore is not an act, but
      a
      habit. - Aristotle!!!

\end{quote}
\section{Az iskola célja}
\label{sec:iskola_celja}

A Budapest School abban támogatja a gyerekeket, hogy azok az
attitűdök, képességek és szokások alakuljanak ki bennük, amelyek segítségével
boldog, egészséges és a társadalom számára hasznos felnőttekké válhatnak. A
cél, hogy a gyerekek a mai világ szükségleteihez és lehetőségeihez a saját
erősségeik felhasználásával kapcsolódhassanak.	Hogy tanulási útjukat
sajátjukként éljék meg, és felelősséget érezzenek az alakításáért, az újabb és
újabb kihívások megtalálásáért.

Olyan tanulási környezetet kell kialakítani ehhez, ahol a szülők, tanárok és
gyerekek tesznek magukért és egymásért, ahol gyerekeink képesek nehéz
helyzetekben is életük, kapcsolataik és környezetük aktív alakítói lenni,
cselekedeteikkel, tetteikkel elérni a kitűzött céljaikat.

Mindenki kíváncsinak születik, és meg tudja tanulni azt, amit
igazán szeretne. Nincs is másra szükség, csak izgalmas kihívásokra, kérdésekre,
biztonságra, támogatásra és lehetőségekre.

Ennek szellemében az iskolánk elsődleges feladata, hogy a gyerekek közösségét
segítsék abban, hogy sokat és hatékonyan tanuljanak és alkossanak.
\emph{Tanulják azt, amit szeretnének, és azt, amire szükségük van.}


\section{A fejlődésközpontú módszer}
A Budapest School Általános Iskola és Gimnázium mikroiskoláinak legfőbb
pedagógiai alapelve, hogy mindig ott, akkor és az történjen a
gyerekekkel, ami őket a fejlődésükben a leginkább támogatja. Minden, ami
az iskolában történik újra és újra erre az alapkérdésre kell, hogy
visszatérjen. Az segíti a leginkább a gyerekek fejlődését, amit most
csinálunk, vagy változtatnunk kell rajta? A Budapest School
tehát rugalmas és integratív, a gyerekek fejlődéséhez igazodó. Abban
támogatja a gyerekeket, hogy örömmel, harmóniában, mégis folyamatos
kihívásokat keresve fejlődjenek, és így egészséges, boldog és a
társadalom számára hasznos felnőttekké váljanak. A Budapest School
célja ennek megfelelően, hogy a mikroiskoláiba járó gyerekek
megtanuljanak saját erősségeiket fejlesztve egyéni célokat állítani és
azokat később a  világ adta lehetőségekhez és szükségletekhez
igazítani.

\section{Tanulásszervezési elvek}
\label{sec:tanulasszervezesi_elvek}
\todo[inline]{A tanulásszervezi elvek fejezetnek kéne valami struktúrát adni. ?}

A tanulás módja egyénenként változó. Az alábbi elvek azonban minden
korosztályban állandóan befolyásolják és keretezik a Budapest School
pedagógiai programját.


\subsection{Tanulni tanuljunk}


\paragraph{A fejlődésközpontú szemlélet}

A gyerekek tanulása fejlődésközpontú szemléletben (growth mindset)
történik. A diákok és a tanárok számára ezáltal nagyobb fontossággal
bír a tanulásba fektetett erőfeszítésük, mint a képességeik.
Erőfeszítéseik segítségével képességeik fejleszthetők,
megváltoztathatóak. Számukra inspiráló a kihívás, és a hibázás kevésbé
töri le lelkesedésüket.

\paragraph{A hibázás tisztelete}

A hibázás a gyakorlás és az új megismerésének jele. A hibázás
feszültségmentes kezelése kulcsfontosságú abban, hogy a gyerekek merjék
feszegetni a saját határaikat, hogy magabiztosan dolgozzanak azon, hogy
képességeiket, ismereteiket, vagy gyakorlataikat folyamatosan
fejlesszék. A hibázásból való tanulás fő célja, hogy mindig új hibákat
ismerjünk meg, és a korábbiakra minél jobb megoldásokat találjanak a
gyerekek.


\paragraph{Önirányított
      tanulás}

A kisgyerekkor élettani jellemzője a kíváncsiság, az igény a
felfedezésre, tapasztalásra. A tanulás számukra egy önvezérelt aktív
folyamat, melynek megtartása és folyamatos fejlesztése a Budapest School
tanárainak legfőbb feladata, hogy ez a későbbiek során rögzült
viselkedésformává válhasson. Két út van, megtanítani a gyerekeket azokra
a képességekre, amelyekre szükségük van, ezzel kockáztatva azt, hogy a
tanulásuk a világ változásával veszít korszerűségükből, vagy abban
segíteni a gyerekeket, hogy megtaníthassák magukat azokra a
képességekre, amelyekre épp az adott élethelyzetükben szükségük van. A
tanulás így élményszerűvé válik, ismeretszerző jellege csökken, és nő az
önálló felfedezés lehetősége.

\subsection{Rugalmas struktúrában}
\paragraph{A tanulás folyamata}

A tanulás egy olyan folyamat, amely különböző állomásokra, rövid célokra
bontható. A tanulási célok állításának folyamata, annak minősége évek
alatt folyamatosan változik, egyre tudatosabbá, pontosabbá, komplexebbé
válhat. Azonban már az első évektől el kell kezdeni annak gyakorlását,
hogy később kialakulhasson az önálló tanulási cél állítás.



\paragraph{Egyéni és csoportos
      tanulás}

A tanulás egyénileg és csoportokban is történhet. A csoportok
megszervezése mindig azon múlik, hogy az adott tanulási célt mi
szolgálja a legjobban. Ennek megfelelően a gyerekek nem állandó, hanem a
tanulási célokhoz, az érdeklődéshez, a képességi szintekhez alkalmazkodó
rugalmas csoportokban tanulnak.


\paragraph{Együtt, diverzen}

A tanulás kevert korcsoportokban is történhet, akár egy nagy családban. Együtt,
egymástól tanulnak a gyerekek, egymást segítik a fejlődésben. Az egymásnak
adott fejlesztő visszejelzések, pozitív megerősítések révén folyamatosan alakul
ki a tanulás tisztelete és a képességek fejlesztésébe vetett hit.


\paragraph{A zárkózottság és fókuszált
      tanulás}

A tanulás módja nagyban függhet attól, hogy egy gyerek mennyire
zárkózott, mennyire tud és akar önállóan tanulni. A csoportos munkák
során alapelv, hogy a zárkózott gyerekek is lehetőséget kapjanak, hogy
írásban, csöndesen, vagy kisebb csoportban végezhessék a munkájukat,
mondhassák el ötleteiket. Az egyéni tanulásban minden gyereknek
lehetőséget kell adni arra és segíteni kell abban, hogy önállóan,
fókuszáltan tudjon tanulni.

\paragraph{Projektek és gyakorlatias tanulás}

A tanulás három rétege, az ismeretszerzés, a gondolkodás fejlesztése és
a gyakorlatias, aktív alkotás egyszerre jelenik meg a Budapest School
mindennapjaiban. Az alkotó munka rugalmas időkereteket, változó
csoportbontásokat, és a projektmódszerek sokszínű alkalmazását igényli.
A tanulás ilyenkor sokszor csinálássá válik, az ismeret pedig termékké
változik.


\paragraph{Interdiszciplináris tanulás}

A tanulás célokhoz és nem táryakhoz kötött. Alapértékünk, hogy a tanulás
céljához igazítsuk annak tartalmát, ezáltal az egyes tudományterületek,
művészetek, vagy épp mesterségek gyakran keverednek egymással egy-egy modulon
belül. Szintén a célhoz igazított tanuláshoz kötődik a kutató-felfedező
attitűd, ami a már ismert újrammegismerése mellett, az ismeretlen
felfedezésére,
a megválaszolhatatlan megválaszolására irányul.


\paragraph{Tanulni egész nap és egész évben}

A tanulás a Budapest Schoolban délelőtti és délutáni idősávokra tagolódik,
melynek pontos
alakítása a gyerekek tanulási igényeitől, fejlettségi szintjétől és
korosztályától is függ. A tanulási rend meghatározásáért a Budapest
School mikroiskoláinak tanulásszervzezői felelnek. A reggeli
beszélgetőkör minden iskola napinditásának alapeleme. Ez a tere annak,
hogy a mikroiskolák tagjai megbeszélhessék közös ügyeiket. A reggeli kör
után után a tanulás trimeszterenként újraszervezett módon tanulási
modulokban történik, melynek során jut idő egyéni és csoportos,
gyakorló, ismeretszerző és gyakorlatias foglalkozásokra is. Az egész
napos iskola lehetőséget biztosít arra, hogy a tanulási egységek között
legyen idő fellélegezni, és felkészülni az újabb modulokra, valamint
arra is, hogy ha egy tanulási egység nagyobb magával ragadja a
gyerekeket, akkor benne maradhassanak és annak megfelelően alakítsák
újra az időrendjüket. A tanulás az iskolában nem ér véget. A tanulás
szeretetének kialakulásával folyamatossá válik az ezzel való foglalkozás, így a
Budapest School tanulásnak veszi az otthon, szünetekben eltöltött időt is, ahol
sokszor hatékonyabb módon tud egy gyerek gyakorolni, mint az iskolában, amikor
társaival van egy közösségben és ezáltal számos más inger is éri.


\subsection{Közösséget építünk}
\label{sec:kozosseget_epitunk}
\paragraph{Partnerségi kapcsolatra törekszünk}
\todo[inline]{parntersegrol, mint elerheto celrol irni kell.}

\paragraph{A mindennapok
      nehézségei}

A gyerekek tanulását családi hátterük változása, egyéni problémák,
számos mindennapi esemény befolyásolhatja. Ezek figyelembe vétele a
mindennapokban, a mentor tanárral való bizalmi viszonynak köszönhetően
válik lehetségessé. Ennek a kapcsolatnak az alapjait ezért a partnerség,
az értő figyelem adja.

\paragraph{Esélyegyenlőség a
      tanulásban}

A Budapest School kiemelt figyelmet fordít arra, hogy a sajátos nevelési
igényű tanulók is lehetőséget kapjanak a csoportban való munkára.
Tanulásukat amennyiben szükséges, külső szakember bevonásával segíti. A
Budapest School a hátrányos helyzetű tanuló számára is biztosítani
kívánja az elfogadó, fejlesztő környezetet. Az egyenlő bánásmód
megvalósulása érdekében olyan differenciált tanulási környezetet alakít
ki, ami biztosítja a minél nagyobb mértékű inkluzivitást. A tanulási
esélyegyenlőség feltételeinek megvalósulását az egész napos iskola is
nagyban segíti.




\paragraph{A tanár szerepe a tanulási
      folyamatban}

A Budapest School tanulásszervezői partnerként, a tanulás folyamán
segítő társként vannak jelen a gyerekek életében. A tanulás tanórák
helyett pontos tanulási célokat tartalmazó tanulási modulokból épül fel,
melyek során a tanár az adott cél eléréséhez szükséges eszközöket,
tanulási segédleteket biztosítja. A tanár akkor és annyira segíti a
gyerekeket a saját céljuk elérésében, amennyire azt a gyerek igényli, és
folyamatosan tekintettel van arra, hogy a gyerek saját fejlődési üteme
megvalósulhasson. Ehhez tudatosan kell kezelnie nem csak a gyerekeket
érintő fejlesztési lehetőségeket, hanem azt is pontosan látnia kell,
hogy egy adott tanulási cél elérésének milyen készségszintű, vagy
gyakorlatias alapfeltételei vannak. Ezért a gyerekek tanulási célját
támogatandó segítenie kell abban, hogy a gyakorló idő, a gyakorlatias
alkotó idő és az új ismeret megszerzésének ideje folyamatos egyensúlyban
legyen.

\section{Emberkép}
\label{sec:gyerekkep}

A világ és benne az egyén állandóan változik, és a változás mégis számos állandóságot jelöl ki. Korábban a technológiai változás évszázadokban volt mérhető, az általunk használt eszközök száma és komplexitása jóval kevesebb volt, mára
néhány év, vagy egy pár órás repülőút elegendő ahhoz, hogy olyan környezetbe, olyan emberek, eszközök közé kerüljünk, ahol másként kell alkalmzkodnunk, mint ahogy azt tanultuk. Ebben a világban a saját belső harmóniánk megtalálása, a saját közösségünkhöz, közösségeinkhez való kapcsolódás, a világ működésének megértése és a törekvés arra, hogy tegyünk a fenntartásáért különösen fontossá vált. Ahhoz, hogy ennek megfeleljünk a tanári, szülői és a tanulói szerepekben is folyamatos fejlődésre van szükségünk. 
A jövőbe nem látunk, de egy dolgot biztosan tudunk: bármilyenek is lesznek a jövő
kihívásai, nekünk az a fontos, hogy ez a gyerekek számára ne félelmetes és
szorongást keltő legyen, hanem lehetőséget, kihívást és örömet okozzon.

A célunk, hogy a fiatalok az iskolában és utána is könnyen megtalálják a saját útjukat. Képesek legyen önmaguknak célokat állítani, azokat elérni. Képesek legyenek már kisgyerekkortól sajátjuknak megélni a tanulást és ahhoz kapcsolódóan célokat elérni, és fokozatosan tanulják meg azt, hogy egyénileg és csoportosan is tudjanak nagyszabású projekteket véghezvinni.
A tanulás három rétege, a tudás megszerzés, az annak felhasználását segítő őnálló gondolkodás és az aktív alkotás egyszerre jelenik meg a mindennapokban. Ezek egyensúlyban tartása épp oly fontos, mint az, hogy a Budapest School kerettantervének megfelelően az egyéniu célok legalább 50\%-ban a kerettantervi egyéni célokból és további 50\%-ban a saját célokból épüljenek fel. Saját célként a sajátnak megélt célokat értjük.

Ezért legfontosabb fejlődési dimenziónak az önnáló, célorintált tanulási és stratégiai gondolkodást tekintjük, melynek ütemeit tanulási szakaszokra bontjuk. A társas kapcsolatok tanulási folyamatában a születéstől a felnőttkorig a teljes magára hagyatottság és énközpontúság csecsemőkori állapotából fejlődik az ember önállóan és kreatívan gondolkodó, önmagával és közösségével integránsan élő érett nagykorúig. Ezt az utat Erik Erikson pszichoszociális fejlődési modelljében rögzítette \citep{Erikson91}. A Budapest School ezen elmélet elemeit veszi sorra és integrálja tanulási struktúrájába, melynek eredményeképp az iskolakezdőtől az érettségizőig négy tanulási szakaszt különít el. Ezen tanulási szakaszok határait az önállósodás során tapasztalt egyéni mérföldkövek jelölik ki, melyek fokmérője a saját cél állításának időbeli, tartalmi előrelépései, valamint az egyén belső szabadságérzete és közösséghez való kapcsolódása. 
Az egyes tanulási szakaszok amellett, hogy a célállításban elkülönülnek egymástól, a szintugrás abban is mérhető, hogy mennyire képes egy gyerek az alatta lévő szinten lévő társait segíteni az előrelépésben. A folyamatos kortárs támogatás a szintekben való előrehaladás valós fokmérője a mentor és szülő visszajelzései mellett. 

\paragraph{Első szint, 5-10 év }

Ebben a korban alakul ki a gyerekek logikus gondolkodása, melynek részeként feladatokat tudnak rendszerezni, sorrendeket képesek felállítani és azokat fogalmakhoz társítani. A gondolkodásuk elkezd a befelé fordulóból a társas kapcsolatok irányába nyílni, ezáltal megnyílik a közösségi problémamegoldás lehetősége. Igényük nő a belső és a külső rendezettségre, így elkezdhetnek önmaguk szabályokat alakítani a saját tanulási igényeik kapcsán. 
Az első szakaszban a gyerekek pszichoszociális fejlődésükkel összhangban a saját magunknak állított cél jelentőségének fogalmát tanulják rövid, eleinte néhány órás, majd a szakasz vége felé, néhány napos tervezéssel és erre adott önreflexiókkal és külső megerősítésekkel. Megtanulják a ma, a tegnap és a holnap fogalmát, tanulási szerződésük tartalmát eleinte főleg a mentor és a szülő segítségével állítják össze, hogy annak a szakasz végére már teljesen egyedi, önállóan kiválasztott elemei is lehessenek.
Ebben a szakaszban a szabad játéknak még nagy szerepe van, a belső világ tágassága fokozatosan nyílik meg a külvilág felé, és kezd lényegessé válni az azokkal való kommunikáció kiérlelése.
Ekkor tanulnak meg a gyerekek írni, olvasni, és a matematikai alapműveleteket, valamint a mindennapjaikban használatos  geometriai és kombinatorikai alapfogalmakat, mely tudás a mindennapjaikba beépülő önálló tanulási célokhoz kapcsolódóan folyamatosan egyre szükségesebbé válik. Önmagukhoz és környezetükhöz érzékenyen, odafigyeléssel és empátiával fordulnak, megtanulják tiszteletben tartani, hogy társaik másmilyenek, akiket más dolgok is érdekelhetnek, másfajta megoldásaink is lehetnek.
A szakasz végére biztonsággal meg tudnak maguknak tervezni egy néhány napos  projektet.

\paragraph{Második szint, 9-14 év}

      A gyerek testi és pszichoszociális fejlődése szempontjából egyaránt kiemelten fontos időszak következik. A korai serdülőkorban alakulnak ki a másodlagos nemi jellegek, melyek ütemükben mind a fiúk és lányok, mind az egyének között jelentős eltéréseket mutathatnak. Az egyéni tanulási tervek összeállításának ezért ebben az időszakban megnő a szerepe. Az agyi struktúrák jelentős átalakulása is erre az időszakra tehető, a gondolkodás, a figyelem, az emlékezet területén is lehetnek ezen átalakulások miatt nehézségek, amit később, az átmeneti visszaesést követően az agy magasabb minőségű működése követi. Ekkor alapozhatja meg egy gyerek a tervezés, a saját tanulási célok komplexebb fogalmait. Aki ebbe a szakaszba lép már hetekre előre ki tudja jelölni a saját feladait. Három év alatt oda akar eljutni, hogy a szakasz végére önállóan képes legyen egy teljes trimeszternyi tanulási tervet összeállítani. Ehhez a hetek számát folyamatosan növelnie kell, miközben a problémák komplexitása is változik. A szülő még mindig megjelenik a célalkotásban, de egyre inkább átalakul a szerepe tanácsadóvá. A mentor az egyéni fejlődés mintázataira figyelve segíti a gyereket abban, hogy fejlettségi szintjének, aktuális testi, lelki változásainak megfelelő módon legyen terhelve. 
      A szakasz végén önállóan be tudja mutatni társainak egy pár hetes projekt eredményeit, és mind szövegértése, mind logikus gondolkodása és matematikai alapismeretei olyan szinten állnak, hogy elmélyült alkotó, kutató feladatokat vállalhat a következő szakaszban. Biztonságos nyelv használata sajátja mellett már egy másik nyelvben is elkezdődik és értékrendjében önmaga és társai megismerésén túl a környezet és a világ fogalma is tágulni kezd.
      
      \paragraph{Harmadik szint, 12-16 év}

  
A sajátként megélt tanulási célok eddigi alapozása ebben a szakaszban nyeri el mélyebb funkcióját. A tinédzserkor minden szempontból a gyerek kivirágzásának időpontja, melynek során a helyét kereső, befelé, saját változásaira fókuszáló gyerekből egy a jövő felé nyitott kamasz válhat. A gondolkodási struktúrák mellett megnő a családon túli társas kapcsolatok szerepe, és ekkor fontos, hogy a családias jelleg, a biztonságos tanulás, a szülővel való konzultáció mint a gyerek számára egyaránt fontos érték, és nem mint kényszer jelenjen meg. 
Az érvelés, a hipotézis alapú problémamegoldás, valamint a jelen fókuszált megélése ebben a korban alakul ki, ahogy annak a tudata is, hogy tetteinknek a jövőre nézve nagyobb mértékű következményei lehetnek. Kiemelten fontos ebben a szakaszban a különböző vélemények, információk, megoldási lehetőségek számba vétele, annak lehetősége, hogy egyénileg ismerhesse fel a gyerek az eltérő utak közötti különbségeket, és ha teheti kérdőjelezzen meg akár tudományos hipotéziseket, vagy írjon újra művészeti, kulturális tartalmakat.
Ha a gyerek ezen határok megértésében szabadon fejlődik, akkor lehetősége lesz arra, hogy a megszerzett alapjait olyan kreatív irányokba fordíthassa, melyek saját élete és társadalmunk jövője formálásához egyaránt hasznosak lehetnek.
Ebben az időszakban a saját tanulási célokat egy teljes trimeszterre vonatkozóan egyénileg határozza meg a gyerek, a szülő és a mentor ebben mint konzulens jelennek meg. Meghozza első komolyabb döntéseit arról, hogy milyen területekkel szeretne elmélyültebben foglalkozni és ehhez kapcsolódóan választott érettségi tantárgyait is kijelöli az szakasz végére. 
Előadásmódját, kutatási és alkotói munkáját felnőtt jegyek jellemzik. 
      
      \paragraph{Negyedik szint, 15-19 év} Ebben a szakaszban a gyerek fizikai értelemben teljesen felnőtté válik. Érzelmi és szociális biztonságra azonban kiemelten szüksége van. A szerelem, a szexuális identitás erősödése, az önállósodásra való igény, az addikciók veszélye konflktusokat szülhet, melyek kezelése különösen fontos ebben az időszakban.
Ennek a biztonságos közegnek a megteremtése az elsődleges feladata az iskolának ebben a felnőttszerű időszakban. Ebben a mentor a szülővel együttműködve tudja segíteni a gyereket. A szakasz célja, hogy a gyerek képesssé váljon a saját életét  meghatározó egy-két éves távlatokban mérhető felelős döntések meghozatalára, melyek akár sorsfordító jelentőségűek is lehetnek. Ezek a döntések  vonatkozhatnak egy emelt szintű érettségire való felkészülésre, egy nemzetközi egyetemre való felkészülésre, egy nagyobb komplexitású kutatási vagy alkotói projektre. Fontos azonban a gyerek támogatása abban is, hogy nem kell örökérvényű döntéseket hozni. Ekkora már megtanult szakaszosan célokat állítani, és tudja, hogy bármikor lesz lehetősége az eletben újratervezni. Önállóan készül az érettségire, tanárai abban segítik, hogy folyamatosan megtalálja magának a kihívást benne. Saját céljai szűkebb környezetén túl könnyedén hatással lehetnek már a világra is.  

\section{Kapcsolat a hazai és nemzetközi oktatási
  reformokkal}\label{sec:kapcsolat_reformokkal}

A Budapest School programja a hazai és nemzetközi oktatási reformok
kontextusában és a pszichológia, a szociálpszichológia, valamint a
szervezetfejlesztés terén elvégzett kortárs kutatások tükrében válik
könnyebben értelmezhetővé.

Magyarországról több iskola története, működése is nagy hatással volt ránk. Az
alternatív iskolák hagyományát többek között a 90-es évektől a Rogers
Személyközpontú Általános Iskola, a Lauder Javne Iskola, a Kincskereső
\todo{ZK-t megemliteni, a koszonet reszben.}
Iskola és a Gyermekek Háza iskolák teremtették meg. A megújuló
középiskolák úttörője az Alternatív Közgazdasági Gimnázium és a
Közgadasági Politechnikum voltak. Ezek az iskolák a személyközpontúság,
a gyerekközpontúság hangsúlyozása mellett elkezdték a gyakorlatban
alkalmazni a differenciálás, a kooperetív technikák alkalmazását és
egyes projektmódszertanokat. Programunk kidolgozásában nagy szerepe volt
annak, hogy ezek az iskolák olyan szemléletmódbéli alapokat fektettek
le, amelyek mára alapelvárásként fogalmazdónak meg a szülők oldaláról az
iskolákkal szemben.

Gyakorlati tapasztalatokat a világ más részein is gyűjöttünk. A 21.
Században a Budapest Schoolhoz hasonló kezdeményezések sorra indulnak a
világban. Ezek egyes jegyei a Budapest School pedagógiai programjával
összhangban vannak:

A Wildflower School mikroiskolák hálózatát működteti kisebb
üzlethelyiségekben. A Budapest Schoolhoz hasonlóan célja, hogy falakat
romboljon a gyerekek és a világ között: a magántanulás és az intézményes
tanulás, a tanár és a tudós szerepe, valamint az iskola és környezete
közötti határok elmosása az egyik fő üzenete.
https://wildflowerschools.org/

Hasonlóan az otthontanulás és az  unschooling struktúrált formáját keresi az
Amerikai Texasban alapított Acton Academy, ami a szokratikus módszereket (azaz,
hogy megbeszéljük közösen), valós projekten keresztüli tanulást, és a
gyakornokoskodáshoz hasonló munka közbeni tanulást (,,learning on the job")
teszi a megközelítésének középpontjába.
https://www.actonacademy.org/

A High Tech High iskoláiban a gyerekek elsősorban projektmódszertan
alapján tanulnak. A tanulási jogokban való egyenlőség mellett az egyéni
célokra szabott tanulás, a világ alakulásához kapcsolódó tartalmi
elemek, valamint az együttműködés alapú tanulás is megjelenik
pedagógiáukban a Budapest School által is alkalmazott jegyekből.
https://www.hightechhigh.org/

A School21 brit iskola a 21. századi képességek fejlészétését tűzte ki
célul. Ennek jegyében a prezentációs, előadói skillek kiemelt
jelentőségüek. Iskolájuk egyensúlyt akar teremteni a tudásbéli
(akadémiai), a szívbéli (személyiség és jóllét) és a kézzel fogható
(problémamegoldó, alkotó) között. A Budapest School iskoláinak hasonló
módon célja, hogy a tanulás három rétegét, a tudást, a gondolkodást és
az alkotást folyamatos harmóniában tartsa. https://www.school21.org.uk

A Khan Lab School Monterossi módszert keveri az online tanulással. Kevert
korosztályú csoportokban, személyreszabott módszerekkel segítik a
képességfejlesztést és a projekt alapú munkát.
https://khanlabschool.org/
