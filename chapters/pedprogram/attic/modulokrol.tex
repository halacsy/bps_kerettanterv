\marginpar{ez kell ide extran?}
A Budapest School Kerettanterv a három tantárgyat határoz meg: STEM,
KULT és Harmónia tantárgyakkal fedi le a Nemzeti Alaptanterv által
meghatározott tartalmi követelményeket. A gyerekek órarendjében azonban
nem ezek a tantárgyak jelennek meg, hanem specifikusabb, rövidebb,
célorientáltabb egységek: a modulok. A Budapest School modulok
különböznek a megszokott tantárgyi tanóráktól. Abban azonban
hasonlítanak, hogy a moduloknak is mindig van előre meghatározott
fejlesztési célja, azaz, hogy a modul során mit tanul, miben fejlődik
várhatóan a résztvevő gyerek.

\paragraph{A modulok rövidebbek.}

A tantárgyak céljait, folyamatai általában minimum egy tanévre
határozzák meg. A miniszter által kiadott kerettantervekben két éves
egységekben gondolkoznak a szerzők. A Budapest School moduljai lehetnek
egy hetesek is, vagy legfeljebb egy trimeszter hosszúak. A lezárását
követően egy tanulási folyamat egy új modulban tovább folytatódhat.

\paragraph{Tervezettség}\label{tervezettsuxe9g}

A tantárgyi rendszerek a kimeneti követelmények alapján előre
megtervezettek. A Budapest School a gyerekek céljai és érdeklődése
alapján, akár az igény megjelenése utáni héten is indulhatnak.
Természetesen nem kell mindig kitalálni a spanyol viaszt. Ha egy modul
bevált, akkor lehet egy másik gyerekcsoporttal is kipróbálni. De mindig
meg kell tartanunk annak a lehetőségét, hogy más gyerekekkel, más
tanárral, más helyen ugyanaz a modul is másképp sikerül.

\paragraph{Tantárgyközi tudás}\label{tantuxe1rgykuxf6zi-tuduxe1s}

A modulok alapértelmezett tulajdonsága, hogy több tantárgy tartalmából
építkeznek, hisz \emph{minden, mindennel} összefügg. Így minden modul
lefedhet egy, kettő vagy akár három tantárgy tartalmát. A
kerettantervben meghatározott kiegyensúlyozottság elve alapján a
tanároknak úgy kell modult hirdetniük és gyerekeknek választaniuk, hogy
minden tantárgy tartalmával közel egyenlő mértékben találkozzanak.

\paragraph{Modulindítás}\label{modulinduxedtuxe1s}

Modulokként jelennek meg a foglalkozások, a tanórák, a projektek, a
fakultációk, szóval minden olyan tevékenységek, amit a gyerekek az
iskolában csinálhatnak. Az a tanulásszervező tanárok feladata, hogy a
gyerekek minden trimeszter előtt legalább két héttel megismerhessék a
következő trimeszter választható moduljait.

Modul indulásakor mindig elkészül egy modulkiírás, ami tartalmazza a
modul célját, azaz, hogy \emph{miért} csináljuk, hogy \emph{mit} fognak
a gyerekek csinálni a modul során, a tervezett tantárgyi
eredménycélokat, \emph{mikor} és mennyi ideig tart a modul, ha érdekes,
akkor a módszert, azaz \emph{hogyan szervezzük a tanulást}, és a modul
során tapasztalható fejlődés érteklési szempontjait. Példa

\todo{Ez legyen egy akademikus modul}
\begin{itemize}

  \item
        Néptánc Sárival

        \begin{itemize}
          \item
                Cél: közösen táncoljunk, hogy többet mozogjunk és zenei érzéket
                fejlesszük
          \item
                Mit: 8 alkalom, hetente 1 órában 2 népdalra tanuljuk be a
                koreográfiát
          \item
                Tantárgyi célok: Harmnónia, BLA BLA
          \item
                Értékelési szempontok: együtt mozgott társaival, energikusan
                mozgott, követte a zenét, megfelelő ruhában érkezett
        \end{itemize}
\end{itemize}