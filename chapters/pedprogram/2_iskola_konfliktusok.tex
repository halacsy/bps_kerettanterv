\section{A közösségi lét szabályai}
\label{sec:kozossegi_elet}
A Budapest School egy közösségi iskola: a gyerekek együtt tanulnak és alkotnak,
a tanárok
csapatokban dolgoznak és a szülők is egy elfogadó közösség részének érezhetiek
magukat. A tagok -- a tanárok, a gyerekek, a szülők,
és az adminisztrátorok -- azért csatlakoznak
a közösséghez, mert itt szeretnének lenni és itt érzik magukat boldognak,
egészségesnek és hasznosnak. A közösség azért fogad be új tagokat, hogy
nagyobb, erősebb közösség tudjunk lenni.

\paragraph{Alap csoportok} A Budapest School iskola kisebb mikroiskolák
hálózataként működik (lásd \aref{sec:mikroiskola} fejezetet).
Ez a gyerekek és tanárok elsődleges közössége: gyerekként ez az a közösség,
akikkel együtt járok iskolába, együtt alakítom az iskolámat,
tanárként ez az a csapat, akikkel együtt dolgozom, hogy létrehozzuk, tarsuk és
fejlesszük a mikroiskola gyerekeinek tanulási környezetét.

A tanulás egysége a modul, amiről \aref{sec:modulok} fejezet részletesen ír.
Egy modul csoportjában a közös érdeklődésű és célú gyerekek tanulnak együtt,
mert együtt boldogabban, hatékonyabban el tudják érni a céljukat.

\paragraph{Saját szabályok}
\label{sec:sajat_szabalyok}
A mikroiskola és akár egy-egy modul kereteit,
szabályait a résztvevők alakítják ki. Pontosabban a tanárok
\footnote{Mikroiskola esetén a tanulásszervező tanárok, modulok
      esetén a modulvezető tanárok és a tanulásszervező tanárok közösen.}
felelőssége és
feladata, hogy mindenk tanár és gyerek számára elofgadható, betartható,
releváns és értelmes
szabályok
legyenek. Mind a gyerekek, mind a tanárok javasolhatnak új szabályokat, vagy
szabályváltoztatást. Minden egyes gyereknek és tanárnak hozzájárulását kell
adnia minden javaslathoz, mondván: ,,it's good enough for now and safe enough
to try", azaz, ,,elég jónak és biztonságosnak tartom, hogy kiprobáljuk'' a
szabályokat.

Ebből következik, hogy minden mikroiskolának saját házirendje, szabályai
lehetnek, és modulonként alakulhat, hogy mit, mikor, hogyan és kivel csinálnak
a résztvevők.

Szabályok alakításának elsődleges szándéka mindig az legyen, hogy
hogy a gyerekek fejlődése, tanulása, alkotása és a közösség működése egyre jobb
legyen.

\subsection{Konfliktus, feszültségek kezelése}
\label{sec:konfliktusok_kezelese}
Tudjuk, hogy a Budapest School szereplői, a gyerekek, a tanárok, a szülők,
a pedagógia program, a kerettanterv, a fenntartó, a szomszédok, az állami
hivatalok  között feszültségek és konfliktusok alakulhatnak ki, mert
különbözőek vagyunk, különbözőek az igényeink. A feszültségekre és a
konfliktusokra a Budapest School olyan lehetőségként tekint, amelyek
együttműködésen alapuló megoldása építi a kapcsolatot, és segíti a fejlődést.

Minden vágy, ötlet, szándék, cél, viselkedés közötti különbség, ha az
valamelyik
félben negatív érzéseket kelt, feszültséget és konfliktus okozhat. Ebbe
beletartozik az is, ha valaki nem
azt és úgy csinálja, ahogy nekünk erre szükségünk van, vagy ha bármilyen
okból nem érezzük magunkat biztonságban, vagy más univerzális emberei
szükségletünk \citep{rosenberg2003nonviolent} nem elégül ki.

Konfliktus alakulhat ki a gyerekek, szülők és tanárok között bármilyen
relációban és adódhatnak egyéb, belső konfliktusok, nehézségek is akár a
család, akár a Budapest School életében, amelyek kihathatnak a közösségi
kapcsolatainkra.

\subsubsection{A BPS konfliktusok feloldása}

Az iskola a partnerségen alapuló szervezetében nem autoritások, főnökök,
hivatalok, bírók oldják
meg a konfliktusokat, hanem
egyenrangú társak. Az iskola feltételezi, hogy a felek tudnak gondolkodni,
következtetni,
felelősséget vállalni a döntéseikért és cselekedetikért.\footnote{Kisgyerekeket
      mentoraik és szüleik reprezentálnak.}

\paragraph{Alapértékek}
Ahhoz, hogy tényleg partneri kapcsolatban, egyenrangú felekként tudjunk
konfliktusokat, problémákat megoldani, érdemes közös értékeket elfogadni.
\begin{itemize}
      \item Először a saját változásunkon dolgozunk, mert nem nagyon tudunk
            mást
            embert megváltoztatni, csak magunk változásáért
            lehetünk
            felelősek.
      \item Felelősséget vállalunk gondolatainkért, hiedelmeinkért, szavainkért
            és
            viselkedésünkért.
      \item Nem pletykálunk, szóbeszédet nem terjesztjük.
      \item Nem beszélünk ki embereket a hátuk mögött.
      \item Félreértéseket tisztázunk, konfliktusokat felszínre hozunk.
      \item Személyesen, 1-on-1 beszélünk meg problémákat, másokat nem húzunk
            be
            a
            problémába.
      \item Nem hibáztatunk másokat a problémákért. Amikor mégis, akkor az egy
            jó
            alkalom arról gondolkozni, hogy
            miként vagyunk mi is része a problémának, és kell a megoldás
            részévé
            vállnunk.
      \item Erősségekre több figyelmet fordítunk, mint a gyengeségekre, és a
            lehetőségekről, megoldásokról többet beszélünk, mint a
            problémákról.
\end{itemize}

\paragraph{Feszültségre felszínre hozása}
Olyan módszereket, folyamatokat, szabályokat, szokásokat kell kialakítani
minden közösségben, hogy legyen tere és ideje a feszültségeket előhozni.
\begin{itemize}
      \item Iskolai csoportok rendszeresen kezdik a napjukat egy bejelentkező
            körrel, ahol van lehetőség a feszültségeket is felhozni.
      \item A csoportok rendszeresen tartanak retrospektív gyűlést, ahol
            értékelik, mi volt jó és nem annyira jó egy vizsgált időszakban.
      \item Évente legalább kétszer a tanárok egymásnak, a szülők a tanároknak,
            a
            gyerekek a tanároknak, a tanárok a gyerekeknek visszajelzést adnak
            szervezett
            formában.
      \item A gyerekek a mentorukkal rendszeresen találkoznak, ahol teret
            kapnak
            a felmerülő feszültségek.
\end{itemize}

\paragraph{Megbeszélés}

A Budapest School közösségének valamennyi tagja (a tanárok, gyerekek,
szülők, adminisztrátorok, iskolát képviselő fenntartó) vállalja:

\begin{itemize}

      \item A közösség mindennapjaival kapcsolatos konfliktusok esetén elsőként
            az
            abban érintett személynek jelez közvetlenül.
      \item Személyes kritikát mindig privát csatornán fogalmazza meg először,
            ha
            kell, akkor segítő bevonásával.
      \item   Bármelyik fél jelzése esetén lehetőséget biztosít
            arra, hogy a vitás kérdést megbeszéljék közvetlenül, a folyamatban
            részt vesz.
      \item
            Szakítás, kilépés, lezárás előtt legalább három alkalommal
            megpróbál
            egyeztetni.
      \item
            Az egyeztetésre elegendő időt hagy, amely legalább 30 nap vagy --
            amennyiben több időre van szükség -- a másik féllel megállapodott
            idő.
      \item
            Teljes figyelemmel, nyitottsággal, a probléma megoldására
            fókuszálva
            igyekszik feloldani a konfliktust, és közösen megoldást találni a
            problémára.
\end{itemize}

Összefoglalva: ha problémánk van egymással, akkor azt megbeszéljük. Nem
okozunk egymásnak meglepetést, mert vállaljuk, hogy rögtön elmondjuk
egymásnak konfliktusainkat.

\paragraph{Közvetítő bevonása}

Ha úgy érezzük, hogy a személyes egyeztetés nem vezetett megoldásra, a
tárgyalást külső segítség bevonásával folytatjuk. Ez lehet egy másik
csoporttag, egy tanár, vagy egy teljesen külsős meditátor.

Amikor bármely fél közvetítőt kér, akkor a másik ezt elfogadja. Nem mondhatjuk
azt, hogy  \emph{,,de hát mi magunk is meg tudjuk oldani a konfliktust''}.

\paragraph{Eszkalálás}

Ha két fél nem tudja megoldani a konfliktust, akkor kérhetnek segítséget a
\texttt{segitseg@budapestschool.org}
címen, amire 48 órán belül kell válaszolnia. Az email kezeléséért az iskola
igazgató felel.

\paragraph{Megállapodások}

Ha az egyeztetés, tárgyalás és közvetítő bevonása során sikerül
valamilyen megoldást vagy a megoldáshoz vezető folyamatot egyeztetni, a
felek \emph{megállapodnak} abban, hogy ki mit tesz, vagy milyen szabályokat
alakítanak
ki, illetve, hogy mennyi időt adnak egymásnak, hogy kipróbálják, sikerült-e
feloldani a konfliktust.
Segíteni szokott a kérdés, hogy \emph{most megállapodtunk vagy csak beszéltünk
      róla?}, hogy minden fél számára világos legyen a megállapodás.

Nagyobb konfliktusok esetén jó gyakorlat, és bármely fél kérheti, hogy
írásban is rögzítsék a megállapodást. Ez lehet egy papirfecni is, vagy egy
email. Lényege nem a formátum, hanem  hogy minden fél emlékezzen a
megállapodásra.

\paragraph{Egyeztetés sikertelen}

Ha a felek között az egyeztetés sikertelen volt, vagy a megoldási
javaslat nem működött, a felek ezt írásban meg kell, hogy állapítsák. Erre
azért van szükség, hogy egyetértés legyen abban közöttük, hogy értik, a
másik fél sikertelennek érzi az egyeztetést.

\paragraph{Lezárás}

Az egyeztetés sikertelensége esetén elengedjük egymást. De ez a végső
megoldás.

\subsection{Kiemelt konfliktusok}

\subsubsection{Gyerek-gyerek
      konfliktus}

A mindennapokban a gyerekek akarva akaratlanul belecsöppennek olyan
helyzetekbe, amikor a közösségben vagy interperszonális kapcsolataik során
megborul a mindennapi egyensúly. Ezekben a konfliktushelyzetekben leginkább a
felborult egyensúly helyreállítására törekszünk \emph{resztoratív
      konfliktusfeloldási
      technikával}.

A resztoratív konfliktusfeloldás alaptézise az, hogy minden ilyen megborult
egyensúlyi állapot egy lehetőség valami megújítására, újragondolására. A
résztvevők egy külső személy segítségével (általában a jelenlévő tanár) együtt
alakítják
a megoldást, egészen addig, amíg az eredmény mindenki számára a konfliktus
feloldását jelenti, azaz a megborult egyensúly helyreállítását.

A folyamat során a konfliktusban résztvevő összes személy elmondja az érzéseit,
meglátását a felmerülő helyzettel kapcsolatban, valamint az én-közléseken túl a
szükségleteikről is beszélnek. Ezen szükségletek képezik a megoldás alapját,
azaz ezeket egy tető alá hozva feloldhatjuk a fennálló konfliktust. Ilyenkor
mindig először azokat a pontokat keressük meg, amelyekben egyetértenek a
résztvevők, hiszen ez egy közös alapot szolgáltat arra, hogy a valódi feloldást
megtalálhassuk.

Fontos, hogy a beszélgetésben az összes fél hallassa a hangját, és meg is
legyen
hallgatva. Az értő figyelem kompetenciája is fejlődik ezen módszer alkalmazása
során, például, ha az érintett felek elmondják, hogy mit hallottak meg abból,
amit a másik elmondott.

Előfordulhat, hogy ez a folyamat nem egyből a konfliktus után indul el: ha a
résztvevők beleegyeznek, akkor a beszélgetés elhalasztható, de lehetőleg még
aznap történjen meg.

\subsubsection{Gyerek-iskola, tanár-szülő
      konfliktus}

Minden gyereknek van egy mentora. A szülők számára a mentor az
elsődleges kapocs az iskola felé. Ezért, ha a szülőben jelenik meg egy
feszültség, akkor elsődlegesen a mentornak jelez. Ugyanígy, a mentor
közvetíti a család felé a gyerekkel kapcsolatos feszültségeket.

Ha egy gyerek, vagy szülő úgy érzi, hogy egy gyereknek nem jó az iskolai
élménye, például nem tanul eleget, vagy kiközösítik, vagy csak nem
szeret bemenni, akkor feladata, hogy rögtön beszéljen a mentor tanárral.

Ha nem sikerül a mentorral megbeszélni a konfliktust és megoldást találni,
akkor a szülőnek is lehetősége van segítőt behívni, aki lehet egy másik tanár,
másik szülő, az iskolaigazgató vagy bárki, akinek képességeiben bízik.

Előfordulhat, hogy a tanárok vagy az iskola úgy érzik, hogy egy
gyereknek nem tesz jót a Budapest School közössége, vagy a hozzáállása
súlyosan zavarja vagy sérti a Budapest School közösséget vagy azok
tagjait. Az is lehet, hogy a tanárok vagy a Budapest School a szülővel
való kapcsolatot érzik konfliktus vagy feszültség forrásának. Ilyen
esetben ugyanígy le kell folytatni a konfliktuskezelés folyamatát és
megpróbálni feloldani a feszültséget. Ennek sikertelensége esetén az
iskola jelzi a családnak írásban, hogy el fog válni.

\subsubsection{Pedagógia program, kerettanterv be nem tartásával
      kapcsolatos konfliktusok}

Amikor egy tanár, egy gyerek vagy egy mikroiskola nem tartja be a
kerettantervet, a pedagógia programot, a házirendet, vagy egyéb közösen
megalkotott szabályokat, akkor konfliktus alakul ki közte és az iskola
között. Ilyenkor szintén a konfliktuskezelés folyamatát kell
lefolytatni.

\subsubsection{Évfolyamszint-lépéssel, osztályzatra való átváltás konfliktusai}
Szülő-iskola konfliktusainak nagy része sok iskolában az osztályzatokkal és a
bizonyítvánnyal kapcsolatos.
A BUdapest School iskolában a gyerekek maguk kérelmezek az évfolyamszint-lépés
és szükség esetén az osztályzatra való átváltást. Maguk tesznek javaslatot
arra, hogy mikor és hány évfolyamot lépjenek és hogy milyen osztályzat kerüljön
a bionyítványba. A bírálók ezt efogadják vagy elitasítják. Amikor a
gyerek/szülő kérelmét a bírálók elutasították, akkor konfliktus alakul ki.
Ilyenkor is az előbbeikben leírt elveket, folyamatokat kell alkalmazni.