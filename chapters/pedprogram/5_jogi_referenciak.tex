\newcommand{\pref}[1] {
    \aref{#1}. fejezet \apageref{#1}. oldalon}

\newcommand{\tlaw}[3]
{
    #1 & #2 & #3 \\ \hline
}

\begin{longtable}{p{7cm} | p{5cm} | l}

    \textbf{feladat} & megjegyzés              & \textbf{referencia} \\
    \hline
    az iskolában folyó nevelő-oktató munka pedagógiai alapelvei,
    értékei, céljai, feladatai, eszközei, eljárásai
                     &
                     & \pref{sec:alapelvek}
    \\

    \hline
    a személyiségfejlesztéssel kapcsolatos pedagógiai feladatok
                     &
                     & \pref{sec:gyerekkep} és
    \pref{sec:szemilyesegfejlesztes}
    \\
    \hline

    \tlaw{a közösségfejlesztéssel, az iskola szereplőinek együttműködésével
        kapcsolatos feladatok}
    {}{\pref{sec:kozossegfejlesztes}}

    \tlaw{a pedagógusok helyi intézményi feladatai, az osztályfőnöki munka
        tartalmát, az osztályfőnök feladatai}
    {osztályfőnök szerepet a Budapest Schoolban a tanulásszervezők és
        mentorok veszik át.}{\pref{sec:tanarfeladatok}}
    \tlaw{Az iskola által alkalmazott jelölés, értékelés érdemjegyre,
        osztályzatra való átváltásának szabályai. NKT 54. § (4)}
    {}{\pref{sec:osztalyzatok}}
    \tlaw{ a kiemelt figyelmet igénylő tanulókkal kapcsolatos pedagógiai
        tevékenység helyi rendje
    }{}{\pref{sec:kiemelt_figyelem}}
    \tlaw{ a tanulóknak az intézményi döntési folyamatban való részvételi
        jogai
        gyakorlásának rendje
    }{}{}
    \tlaw{ a szülő, a tanuló, a pedagógus és az intézmény partnerei
        kapcsolattartásának formái
    }{mind az alapelvekben, mind a döntéshozásban és konfliktusok
        kezelésében erre törekszik az iskola}
    {\pref{sec:kozosseget_epitunk} és \pref{sec:kozossegilet}}

    \tlaw{ a tanulmányok alatti vizsgák és az alkalmassági vizsga
        szabályai; az
        iskolai írásbeli, szóbeli, gyakorlati beszámoltatások, az ismeretek
        számonkérésének rendje
    }{}{}
    \tlaw{ a felvétel és az átvétel - Nkt. keretei közötti - helyi
        szabályai
    }{}{}
    \tlaw{ az elsősegély- nyújtási alapismeretek elsajátításával
        kapcsolatos
        iskolai
        terv
    }{}{\pref{sec:elsosegely}}
    \tlaw{ az iskolában alkalmazott sajátos pedagógiai módszerek,
        beleértve a
        projektoktatást

    }{a pedagógia program a tanárok kísérletezését támogatja

    }{\pref{sec:pedagogia_modszerek}}
    \tlaw{ az iskola helyi tanterve
    }{}{}
    \tlaw{ a választott kerettanterv megnevezése
    }{}{}
    \tlaw{ a választott kerettanterv által meghatározott óraszám feletti
        kötelező
        tanórai foglalkozások, továbbá a kerettantervben meghatározottakon
        felül a nem
        kötelező tanórai foglalkozások megtanítandó és elsajátítandó
        tananyaga, az
        ehhez szükséges kötelező, kötelezően választandó vagy szabadon
        választható
        tanórai foglalkozások megnevezése, óraszáma
    }{}{}
    \tlaw{ az oktatásban alkalmazható tankönyvek, tanulmányi segédletek és
        taneszközök kiválasztásának elvei (figyelembe véve a tankönyv
        térítésmentes
        igénybevétele biztosításának kötelezettségét)
    }{}{}
    \tlaw{ a Nemzeti alaptantervben meghatározott pedagógiai feladatok
        helyi megvalósításának részletes szabályai
    }{}{\pref{sec:nat_celjai}}
    \tlaw{ a mindennapos testnevelés, testmozgás megvalósításának módja,
        ha azt
        nem
        az Nkt. 27. § (11) bekezdésében meghatározottak szerint szervezik
        meg,
    }{Nkt. 27. szerint valósítja meg az iskola}{}
    \tlaw{ a választható tantárgyak, foglalkozások, továbbá ezek esetében
        a
        pedagógusválasztás szabályai
    }{az iskolában a modulokválasztásával a modulvezető tanárt is
        választ}{}
    \tlaw{ középiskola esetén azon választható érettségi vizsgatárgyak
        megnevezése,
        amelyekből a középiskola tanulóinak közép- vagy emelt szintű
        érettségi vizsgára
        való felkészítését az iskola kötelezően vállalja, továbbá annak
        meghatározáse,
        hogy a tanulók milyen helyi tantervi követelmények teljesítése
        mellett melyik
        választható érettségi vizsgatárgyból tehetnek érettségi vizsgát,
    }{}{}
    \tlaw{ középiskola esetén az egyes érettségi vizsgatárgyakból a
        középszintű
        érettségi vizsga témakörei
    }{}{}
    \tlaw{ a tanuló tanulmányi munkájának írásban, szóban vagy
        gyakorlatban
        történő
        ellenőrzési és értékelési módját, diagnosztikus, szummatív,
        fejlesztő
        formáit,
        valamint a magatartás és szorgalom minősítésének elvei
    }{}{}
    \tlaw{ a csoportbontások és az egyéb foglalkozások szervezésének elvei
    }{}{}
    \tlaw{ a nemzetiséghez nem tartozó tanulók részére a településen élő
        nemzetiség
        kultúrájának megismerését szolgáló tananyag,
    }{}{\pref{sec:nemzetiseg}}
    \tlaw{ az egészségnevelési és környezeti nevelési elvek
    }{}{}
    \tlaw{ a gyermekek, tanulók esélyegyenlőségét szolgáló intézkedések
    }{}{}
    \tlaw{ a tanuló jutalmazásával összefüggő, a tanuló magatartásának,
        szorgalmának
        értékeléséhez, minősítéséhez kapcsolódó elvek
    }{}{}
    \tlaw{intézmény partnerei kapcsolattartásának
        formái}{}{\pref{sec:kozossegilet}}

\end{longtable}
