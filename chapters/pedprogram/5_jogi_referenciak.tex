\begin{itemize}
   

\item \emph{az iskolában folyó nevelő-oktató munka pedagógiai alapelvei, értékei, céljai, feladatai, eszközei, eljárásai: }    
       lásd        \ref{sec:alapelvek}.~fejezet \apageref{sec:alapelvek}.~oldalon
              
\item \emph{a személyiségfejlesztéssel kapcsolatos pedagógiai feladatok: }    
       lásd        \ref{sec:gyerekkep}.~fejezet \apageref{sec:gyerekkep}.~oldalon és 
              \ref{sec:szemilyesegfejlesztes}.~fejezet \apageref{sec:szemilyesegfejlesztes}.~oldalon
              
\item \emph{a közösségfejlesztéssel, az iskola szereplőinek együttműködésével kapcsolatos feladatok: }    
       lásd        \ref{sec:kozossegfejlesztes}.~fejezet \apageref{sec:kozossegfejlesztes}.~oldalon
              
\item \emph{a pedagógusok helyi intézményi feladatai, az osztályfőnöki munka tartalmát, az osztályfőnök feladatai: }   a Budapest Schoolban osztályfőnök szerep külön nincs nevesítve, a tanulászervezők és mentorok feladatai állnak ehhez a legközelebb 
       lásd        \ref{sec:tanarfeladatok}.~fejezet \apageref{sec:tanarfeladatok}.~oldalon
              
\item \emph{Az iskola által alkalmazott jelölés, értékelés érdemjegyre, osztályzatra való átváltásának szabályai. NKT 54. § (4): }    
       lásd        \ref{sec:osztalyzatok}.~fejezet \apageref{sec:osztalyzatok}.~oldalon
              
\item \emph{a kiemelt figyelmet igénylő tanulókkal kapcsolatos pedagógiai tevékenység helyi rendje: }    
       lásd        \ref{sec:kiemelt_figyelem}.~fejezet \apageref{sec:kiemelt_figyelem}.~oldalon
              
\item \emph{a tanulóknak az intézményi döntési folyamatban való részvételi jogai gyakorlásának rendje: }    
       lásd        \ref{sec:sajat_szabalyok}.~fejezet \apageref{sec:sajat_szabalyok}.~oldalon
              
\item \emph{a szülő, a tanuló, a pedagógus és az intézmény partnerei kapcsolattartásának formái: }   mind az alapelvekben, mind a döntéshozásban és konfliktusok kezelésében erre törekszik az iskola 
       lásd        \ref{sec:kozossegi_elet}.~fejezet \apageref{sec:kozossegi_elet}.~oldalon és 
              \ref{sec:sajat_szabalyok}.~fejezet \apageref{sec:sajat_szabalyok}.~oldalon és 
              \ref{sec:konfliktusok_kezelese}.~fejezet \apageref{sec:konfliktusok_kezelese}.~oldalon
              
\item \emph{a tanulmányok alatti vizsgák és az alkalmassági vizsga szabályai; az iskolai írásbeli, szóbeli, gyakorlati beszámoltatások, az ismeretek számonkérésének rendje: }   a Budapest School portfólióalapú értékelést alkalmaz a kerettantervet követve 
       lásd        \ref{sec:evfolyamok_osztalyzatok}.~fejezet \apageref{sec:evfolyamok_osztalyzatok}.~oldalon
              
\item \emph{a felvétel és az átvétel -- Nkt. keretei közötti -- helyi szabályai: }    
       lásd        \ref{sec:felvetel-atvetel}.~fejezet \apageref{sec:felvetel-atvetel}.~oldalon
              
\item \emph{az elsősegély-nyújtási alapismeretek elsajátításával kapcsolatos iskolai terv: }    
       lásd        \ref{sec:elsosegely}.~fejezet \apageref{sec:elsosegely}.~oldalon
              
\item \emph{az iskolában alkalmazott sajátos pedagógiai módszerek, beleértve a projektoktatást: }   a pedagógia program (az elfogadott kerettantervnek megfelelően) a tanárok kísérletezését támogatja, ezért minden módszert előre nem tudunk meghatározni 
       lásd        \ref{sec:pedagogia_modszerek}.~fejezet \apageref{sec:pedagogia_modszerek}.~oldalon
              
\item \emph{a választott kerettanterv megnevezése: }   Budapest School Kerettantenterv 
       lásd        \ref{sec:tanterv-program}.~fejezet \apageref{sec:tanterv-program}.~oldalon
              
\item \emph{a választott kerettanterv által meghatározott óraszám feletti kötelező tanórai foglalkozások, továbbá a kerettantervben meghatározottakon felül a nem kötelező tanórai foglalkozások megtanítandó és elsajátítandó tananyaga, az ehhez szükséges kötelező, kötelezően választandó vagy szabadon választható tanórai foglalkozások megnevezése, óraszáma: }   a Budapest School elfogadott kerettanterve szerint a foglalkozások rendszere (ebből következően az óraszám, tananyag, stb.) speciális, a pedagógiai programra vonatkozó jelen előírást így e kerettanterv rendszerén belül kerül kifejtésre.
 
       lásd        \ref{sec:elsosegely}.~fejezet \apageref{sec:elsosegely}.~oldalon és 
              \ref{sec:nemzetiseg}.~fejezet \apageref{sec:nemzetiseg}.~oldalon és 
              \ref{sec:mindennapos-testmozgas}.~fejezet \apageref{sec:mindennapos-testmozgas}.~oldalon
              
\item \emph{az oktatásban alkalmazható tankönyvek, tanulmányi segédletek és taneszközök kiválasztásának elvei (figyelembe véve a tankönyv térítésmentes igénybevétele biztosításának kötelezettségét): }    
       lásd        \ref{sec:tankonyvek}.~fejezet \apageref{sec:tankonyvek}.~oldalon
              
\item \emph{a Nemzeti alaptantervben meghatározott pedagógiai feladatok helyi megvalósításának részletes szabályai: }    
       lásd        \ref{sec:nat_celjai}.~fejezet \apageref{sec:nat_celjai}.~oldalon
              
\item \emph{a mindennapos testnevelés, testmozgás megvalósításának módja, ha azt nem az Nkt. 27. § (11) bekezdésében meghatározottak szerint szervezik meg: }    
       lásd        \ref{sec:mindennapos-testmozgas}.~fejezet \apageref{sec:mindennapos-testmozgas}.~oldalon
              
\item \emph{a választható tantárgyak, foglalkozások, továbbá ezek esetében a pedagógusválasztás szabályai: }   A Budapest School alapvetése a saját tanulási út és a saját tanulási cél. 
       lásd        \ref{sec:modulok}.~fejezet \apageref{sec:modulok}.~oldalon és 
              \ref{sec:modulok_meghirdetese}.~fejezet \apageref{sec:modulok_meghirdetese}.~oldalon
              
\item \emph{középiskola esetén azon választható érettségi vizsgatárgyak megnevezése, amelyekből a középiskola tanulóinak közép- vagy emelt szintű érettségi vizsgára való felkészítését az iskola kötelezően vállalja, továbbá annak meghatározáse, hogy a tanulók milyen helyi tantervi követelmények teljesítése mellett melyik választható érettségi vizsgatárgyból tehetnek érettségi vizsgát: }    
       lásd        \ref{sec:sec:erettsegi_kesobb}.~fejezet \apageref{sec:sec:erettsegi_kesobb}.~oldalon
              
\item \emph{a tanuló tanulmányi munkájának írásban, szóban vagy gyakorlatban történő ellenőrzési és értékelési módját, diagnosztikus, szummatív, fejlesztő formáit, valamint a magatartás és szorgalom minősítésének elvei: }    
       lásd        \ref{sec:ertekeles}.~fejezet \apageref{sec:ertekeles}.~oldalon és 
              \ref{sec:jutalmazas_es_ertekeles}.~fejezet \apageref{sec:jutalmazas_es_ertekeles}.~oldalon
              
\item \emph{a csoportbontások és az egyéb foglalkozások szervezésének elvei: }    
       lásd        \ref{sec:csoportbontasok}.~fejezet \apageref{sec:csoportbontasok}.~oldalon és 
              \ref{sec:csoportok}.~fejezet \apageref{sec:csoportok}.~oldalon
              
\item \emph{a nemzetiséghez nem tartozó tanulók részére a településen élő nemzetiség kultúrájának megismerését szolgáló tananyag: }    
       lásd        \ref{sec:nemzetiseg}.~fejezet \apageref{sec:nemzetiseg}.~oldalon
              
\item \emph{az egészségnevelési és környezeti nevelési elvek: }    
       lásd        \ref{sec:kornyezeti-neveles}.~fejezet \apageref{sec:kornyezeti-neveles}.~oldalon és 
              \ref{sec:mindennapos-testmozgas}.~fejezet \apageref{sec:mindennapos-testmozgas}.~oldalon
              
\item \emph{a gyermekek, tanulók esélyegyenlőségét szolgáló intézkedések: }    
       lásd        \ref{sec:csoportbontasok}.~fejezet \apageref{sec:csoportbontasok}.~oldalon
              
\item \emph{a tanuló jutalmazásával összefüggő, a tanuló magatartásának, szorgalmának értékeléséhez, minősítéséhez kapcsolódó elvek: }   magatartás és szorgalom minősítés nem történik a Budapest Schoolban 
       lásd        \ref{sec:jutalmazas_es_ertekeles}.~fejezet \apageref{sec:jutalmazas_es_ertekeles}.~oldalon
              
\item \emph{intézmény partnerei kapcsolattartásának formái: }    
       lásd        \ref{sec:az_iskola_kormanyzasa}.~fejezet \apageref{sec:az_iskola_kormanyzasa}.~oldalon
              
\item \emph{a teljeskörű egészségfejlesztéssel összefüggő feladatok: }    
       lásd        \ref{sec:szemilyesegfejlesztes}.~fejezet \apageref{sec:szemilyesegfejlesztes}.~oldalon
              
\end{itemize}

\begin{longtable}{p{0.4\textwidth} | p{0.3\textwidth} |p{0.2\textwidth}}

    \textbf{feladat} & megjegyzés              & \textbf{referencia} \\
    \hline

    az iskolában folyó nevelő-oktató munka pedagógiai alapelvei, értékei, céljai, feladatai, eszközei, eljárásai  &   & 
              \ref{sec:alapelvek}.~fejezet \apageref{sec:alapelvek}.~oldalon
              \\ \hline

    a személyiségfejlesztéssel kapcsolatos pedagógiai feladatok  &   & 
              \ref{sec:gyerekkep}.~fejezet \apageref{sec:gyerekkep}.~oldalon és 
              \ref{sec:szemilyesegfejlesztes}.~fejezet \apageref{sec:szemilyesegfejlesztes}.~oldalon
              \\ \hline

    a közösségfejlesztéssel, az iskola szereplőinek együttműködésével kapcsolatos feladatok  &   & 
              \ref{sec:kozossegfejlesztes}.~fejezet \apageref{sec:kozossegfejlesztes}.~oldalon
              \\ \hline

    a pedagógusok helyi intézményi feladatai, az osztályfőnöki munka tartalmát, az osztályfőnök feladatai  &  a Budapest Schoolban osztályfőnök szerep külön nincs nevesítve, a tanulászervezők és mentorok feladatai állnak ehhez a legközelebb & 
              \ref{sec:tanarfeladatok}.~fejezet \apageref{sec:tanarfeladatok}.~oldalon
              \\ \hline

    Az iskola által alkalmazott jelölés, értékelés érdemjegyre, osztályzatra való átváltásának szabályai. NKT 54. § (4)  &   & 
              \ref{sec:osztalyzatok}.~fejezet \apageref{sec:osztalyzatok}.~oldalon
              \\ \hline

    a kiemelt figyelmet igénylő tanulókkal kapcsolatos pedagógiai tevékenység helyi rendje  &   & 
              \ref{sec:kiemelt_figyelem}.~fejezet \apageref{sec:kiemelt_figyelem}.~oldalon
              \\ \hline

    a tanulóknak az intézményi döntési folyamatban való részvételi jogai gyakorlásának rendje  &   & 
              \ref{sec:sajat_szabalyok}.~fejezet \apageref{sec:sajat_szabalyok}.~oldalon
              \\ \hline

    a szülő, a tanuló, a pedagógus és az intézmény partnerei kapcsolattartásának formái  &  mind az alapelvekben, mind a döntéshozásban és konfliktusok kezelésében erre törekszik az iskola & 
              \ref{sec:kozossegi_elet}.~fejezet \apageref{sec:kozossegi_elet}.~oldalon és 
              \ref{sec:sajat_szabalyok}.~fejezet \apageref{sec:sajat_szabalyok}.~oldalon és 
              \ref{sec:konfliktusok_kezelese}.~fejezet \apageref{sec:konfliktusok_kezelese}.~oldalon
              \\ \hline

    a tanulmányok alatti vizsgák és az alkalmassági vizsga szabályai; az iskolai írásbeli, szóbeli, gyakorlati beszámoltatások, az ismeretek számonkérésének rendje  &  a Budapest School portfólióalapú értékelést alkalmaz a kerettantervet követve & 
              \ref{sec:evfolyamok_osztalyzatok}.~fejezet \apageref{sec:evfolyamok_osztalyzatok}.~oldalon
              \\ \hline

    a felvétel és az átvétel -- Nkt. keretei közötti -- helyi szabályai  &   & 
              \ref{sec:felvetel-atvetel}.~fejezet \apageref{sec:felvetel-atvetel}.~oldalon
              \\ \hline

    az elsősegély-nyújtási alapismeretek elsajátításával kapcsolatos iskolai terv  &   & 
              \ref{sec:elsosegely}.~fejezet \apageref{sec:elsosegely}.~oldalon
              \\ \hline

    az iskolában alkalmazott sajátos pedagógiai módszerek, beleértve a projektoktatást  &  a pedagógia program (az elfogadott kerettantervnek megfelelően) a tanárok kísérletezését támogatja, ezért minden módszert előre nem tudunk meghatározni & 
              \ref{sec:pedagogia_modszerek}.~fejezet \apageref{sec:pedagogia_modszerek}.~oldalon
              \\ \hline

    a választott kerettanterv megnevezése  &  Budapest School Kerettantenterv & 
              \ref{sec:tanterv-program}.~fejezet \apageref{sec:tanterv-program}.~oldalon
              \\ \hline

    a választott kerettanterv által meghatározott óraszám feletti kötelező tanórai foglalkozások, továbbá a kerettantervben meghatározottakon felül a nem kötelező tanórai foglalkozások megtanítandó és elsajátítandó tananyaga, az ehhez szükséges kötelező, kötelezően választandó vagy szabadon választható tanórai foglalkozások megnevezése, óraszáma  &  a Budapest School elfogadott kerettanterve szerint a foglalkozások rendszere (ebből következően az óraszám, tananyag, stb.) speciális, a pedagógiai programra vonatkozó jelen előírást így e kerettanterv rendszerén belül kerül kifejtésre.
 & 
              \ref{sec:elsosegely}.~fejezet \apageref{sec:elsosegely}.~oldalon és 
              \ref{sec:nemzetiseg}.~fejezet \apageref{sec:nemzetiseg}.~oldalon és 
              \ref{sec:mindennapos-testmozgas}.~fejezet \apageref{sec:mindennapos-testmozgas}.~oldalon
              \\ \hline

    az oktatásban alkalmazható tankönyvek, tanulmányi segédletek és taneszközök kiválasztásának elvei (figyelembe véve a tankönyv térítésmentes igénybevétele biztosításának kötelezettségét)  &   & 
              \ref{sec:tankonyvek}.~fejezet \apageref{sec:tankonyvek}.~oldalon
              \\ \hline

    a Nemzeti alaptantervben meghatározott pedagógiai feladatok helyi megvalósításának részletes szabályai  &   & 
              \ref{sec:nat_celjai}.~fejezet \apageref{sec:nat_celjai}.~oldalon
              \\ \hline

    a mindennapos testnevelés, testmozgás megvalósításának módja, ha azt nem az Nkt. 27. § (11) bekezdésében meghatározottak szerint szervezik meg  &   & 
              \ref{sec:mindennapos-testmozgas}.~fejezet \apageref{sec:mindennapos-testmozgas}.~oldalon
              \\ \hline

    a választható tantárgyak, foglalkozások, továbbá ezek esetében a pedagógusválasztás szabályai  &  A Budapest School alapvetése a saját tanulási út és a saját tanulási cél. & 
              \ref{sec:modulok}.~fejezet \apageref{sec:modulok}.~oldalon és 
              \ref{sec:modulok_meghirdetese}.~fejezet \apageref{sec:modulok_meghirdetese}.~oldalon
              \\ \hline

    középiskola esetén azon választható érettségi vizsgatárgyak megnevezése, amelyekből a középiskola tanulóinak közép- vagy emelt szintű érettségi vizsgára való felkészítését az iskola kötelezően vállalja, továbbá annak meghatározáse, hogy a tanulók milyen helyi tantervi követelmények teljesítése mellett melyik választható érettségi vizsgatárgyból tehetnek érettségi vizsgát  &   & 
              \ref{sec:erettsegi}.~fejezet \apageref{sec:erettsegi}.~oldalon
              \\ \hline

    a tanuló tanulmányi munkájának írásban, szóban vagy gyakorlatban történő ellenőrzési és értékelési módját, diagnosztikus, szummatív, fejlesztő formáit, valamint a magatartás és szorgalom minősítésének elvei  &   & 
              \ref{sec:ertekeles}.~fejezet \apageref{sec:ertekeles}.~oldalon és 
              \ref{sec:jutalmazas_es_ertekeles}.~fejezet \apageref{sec:jutalmazas_es_ertekeles}.~oldalon
              \\ \hline

    a csoportbontások és az egyéb foglalkozások szervezésének elvei  &   & 
              \ref{sec:csoportbontasok}.~fejezet \apageref{sec:csoportbontasok}.~oldalon és 
              \ref{sec:csoportok}.~fejezet \apageref{sec:csoportok}.~oldalon
              \\ \hline

    a nemzetiséghez nem tartozó tanulók részére a településen élő nemzetiség kultúrájának megismerését szolgáló tananyag  &   & 
              \ref{sec:nemzetiseg}.~fejezet \apageref{sec:nemzetiseg}.~oldalon
              \\ \hline

    az egészségnevelési és környezeti nevelési elvek  &   & 
              \ref{sec:kornyezeti-neveles}.~fejezet \apageref{sec:kornyezeti-neveles}.~oldalon és 
              \ref{sec:mindennapos-testmozgas}.~fejezet \apageref{sec:mindennapos-testmozgas}.~oldalon
              \\ \hline

    a gyermekek, tanulók esélyegyenlőségét szolgáló intézkedések  &   & 
              \ref{sec:csoportbontasok}.~fejezet \apageref{sec:csoportbontasok}.~oldalon
              \\ \hline

    a tanuló jutalmazásával összefüggő, a tanuló magatartásának, szorgalmának értékeléséhez, minősítéséhez kapcsolódó elvek  &  magatartás és szorgalom minősítés nem történik a Budapest Schoolban & 
              \ref{sec:jutalmazas_es_ertekeles}.~fejezet \apageref{sec:jutalmazas_es_ertekeles}.~oldalon
              \\ \hline

    intézmény partnerei kapcsolattartásának formái  &   & 
              \ref{sec:az_iskola_kormanyzasa}.~fejezet \apageref{sec:az_iskola_kormanyzasa}.~oldalon
              \\ \hline

    a teljeskörű egészségfejlesztéssel összefüggő feladatok  &   & 
              \ref{sec:szemilyesegfejlesztes}.~fejezet \apageref{sec:szemilyesegfejlesztes}.~oldalon
              \\ \hline




\end{longtable}

