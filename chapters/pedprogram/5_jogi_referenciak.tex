

\begin{longtable}{p{7cm} | p{5cm} | l}

    \textbf{feladat} & megjegyzés              & \textbf{referencia} \\
    \hline

    az iskolában folyó nevelő-oktató munka pedagógiai alapelvei, értékei, céljai, feladatai, eszközei, eljárásai  &   & 
              \ref{sec:alapelvek}. fejezet \apageref{sec:alapelvek}. oldalon
              \\ \hline

    a személyiségfejlesztéssel kapcsolatos pedagógiai feladatok  &   & 
              \ref{sec:gyerekkep}. fejezet \apageref{sec:gyerekkep}. oldalon és 
              \ref{sec:szemilyesegfejlesztes}. fejezet \apageref{sec:szemilyesegfejlesztes}. oldalon
              \\ \hline

    a közösségfejlesztéssel, az iskola szereplőinek együttműködésével kapcsolatos feladatok  &   & 
              \ref{sec:kozossegfejlesztes}. fejezet \apageref{sec:kozossegfejlesztes}. oldalon
              \\ \hline

    a pedagógusok helyi intézményi feladatai, az osztályfőnöki munka tartalmát, az osztályfőnök feladatai  &  osztályfőnök szerepet a Budapest Schoolban a tanulásszervezők és mentorok veszik át. & 
              \ref{sec:tanarfeladatok}. fejezet \apageref{sec:tanarfeladatok}. oldalon
              \\ \hline

    Az iskola által alkalmazott jelölés, értékelés érdemjegyre, osztályzatra való átváltásának szabályai. NKT 54. § (4)  &   & 
              \ref{sec:osztalyzatok}. fejezet \apageref{sec:osztalyzatok}. oldalon
              \\ \hline

    a kiemelt figyelmet igénylő tanulókkal kapcsolatos pedagógiai tevékenység helyi rendje  &   & 
              \ref{sec:kiemelt_figyelem}. fejezet \apageref{sec:kiemelt_figyelem}. oldalon
              \\ \hline

    a tanulóknak az intézményi döntési folyamatban való részvételi jogai gyakorlásának rendje  &   & 
              \ref{sec:sajat_szabalyok}. fejezet \apageref{sec:sajat_szabalyok}. oldalon
              \\ \hline

    a szülő, a tanuló, a pedagógus és az intézmény partnerei kapcsolattartásának formái  &  mind az alapelvekben, mind a döntéshozásban és konfliktusok kezelésében erre törekszik az iskola & 
              \ref{sec:kozosseget_epitunk}. fejezet \apageref{sec:kozosseget_epitunk}. oldalon és 
              \ref{sec:kozossegilet}. fejezet \apageref{sec:kozossegilet}. oldalon
              \\ \hline

    a tanulmányok alatti vizsgák és az alkalmassági vizsga szabályai; az iskolai írásbeli, szóbeli, gyakorlati beszámoltatások, az ismeretek számonkérésének rendje  &   & 
              \\ \hline

    a felvétel és az átvétel - Nkt. keretei közötti - helyi szabályai  &   & 
              \\ \hline

    az elsősegély- nyújtási alapismeretek elsajátításával kapcsolatos iskolai terv  &   & 
              \ref{sec:elsosegely}. fejezet \apageref{sec:elsosegely}. oldalon
              \\ \hline

    az iskolában alkalmazott sajátos pedagógiai módszerek, beleértve a projektoktatást  &  a pedagógia program a tanárok kísérletezését támogatja & 
              \ref{sec:pedagogia_modszerek}. fejezet \apageref{sec:pedagogia_modszerek}. oldalon
              \\ \hline

    az iskola helyi tanterve  &   & 
              \\ \hline

    a választott kerettanterv megnevezése  &  Budapest School Kerettantenterv & 
              \\ \hline

    a választott kerettanterv által meghatározott óraszám feletti kötelező tanórai foglalkozások, továbbá a kerettantervben meghatározottakon felül a nem kötelező tanórai foglalkozások megtanítandó és elsajátítandó tananyaga, az ehhez szükséges kötelező, kötelezően választandó vagy szabadon választható tanórai foglalkozások megnevezése, óraszáma  &   & 
              \\ \hline

    az oktatásban alkalmazható tankönyvek, tanulmányi segédletek és taneszközök kiválasztásának elvei (figyelembe véve a tankönyv térítésmentes igénybevétele biztosításának kötelezettségét)  &   & 
              \\ \hline

    a Nemzeti alaptantervben meghatározott pedagógiai feladatok helyi megvalósításának részletes szabályai  &   & 
              \ref{sec:nat_celjai}. fejezet \apageref{sec:nat_celjai}. oldalon
              \\ \hline

    a mindennapos testnevelés, testmozgás megvalósításának módja, ha azt nem az Nkt. 27. § (11) bekezdésében meghatározottak szerint szervezik meg,  &  Nkt. 27. szerint valósítja meg az iskola & 
              \\ \hline

    a választható tantárgyak, foglalkozások, továbbá ezek esetében a pedagógusválasztás szabályai  &  Nkt. 27. szerint valósítja meg az iskola & 
              \\ \hline

    középiskola esetén azon választható érettségi vizsgatárgyak megnevezése, amelyekből a középiskola tanulóinak közép- vagy emelt szintű érettségi vizsgára való felkészítését az iskola kötelezően vállalja, továbbá annak meghatározáse, hogy a tanulók milyen helyi tantervi követelmények teljesítése mellett melyik választható érettségi vizsgatárgyból tehetnek érettségi vizsgát,  &   & 
              \\ \hline

    középiskola esetén az egyes érettségi vizsgatárgyakból a középszintű érettségi vizsga témakörei  &   & 
              \\ \hline

    a tanuló tanulmányi munkájának írásban, szóban vagy gyakorlatban történő ellenőrzési és értékelési módját, diagnosztikus, szummatív, fejlesztő formáit, valamint a magatartás és szorgalom minősítésének elvei  &   & 
              \\ \hline

    a csoportbontások és az egyéb foglalkozások szervezésének elvei  &   & 
              \\ \hline

    a nemzetiséghez nem tartozó tanulók részére a településen élő nemzetiség kultúrájának megismerését szolgáló tananyag,  &   & 
              \ref{sec:nemzetiseg}. fejezet \apageref{sec:nemzetiseg}. oldalon
              \\ \hline

    az egészségnevelési és környezeti nevelési elvek  &   & 
              \\ \hline

    a gyermekek, tanulók esélyegyenlőségét szolgáló intézkedések  &   & 
              \\ \hline

    a tanuló jutalmazásával összefüggő, a tanuló magatartásának, szorgalmának értékeléséhez, minősítéséhez kapcsolódó elvek  &   & 
              \ref{sec:kozossegilet}. fejezet \apageref{sec:kozossegilet}. oldalon
              \\ \hline

    intézmény partnerei kapcsolattartásának formái  &   & 
              \\ \hline




\end{longtable}

