

\begin{longtable}{p{6cm} | p{4cm} |p{4cm}}

    \textbf{feladat} & megjegyzés              & \textbf{referencia} \\
    \hline

    az iskolában folyó nevelő-oktató munka pedagógiai alapelvei, értékei, céljai, feladatai, eszközei, eljárásai  &   & 
              \ref{sec:alapelvek}.~fejezet \apageref{sec:alapelvek}.~oldalon
              \\ \hline

    a személyiségfejlesztéssel kapcsolatos pedagógiai feladatok  &   & 
              \ref{sec:gyerekkep}.~fejezet \apageref{sec:gyerekkep}.~oldalon és 
              \ref{sec:szemilyesegfejlesztes}.~fejezet \apageref{sec:szemilyesegfejlesztes}.~oldalon
              \\ \hline

    a közösségfejlesztéssel, az iskola szereplőinek együttműködésével kapcsolatos feladatok  &   & 
              \ref{sec:kozossegfejlesztes}.~fejezet \apageref{sec:kozossegfejlesztes}.~oldalon
              \\ \hline

    a pedagógusok helyi intézményi feladatai, az osztályfőnöki munka tartalma, az osztályfőnök feladatai  &  osztályfőnök szerepet a Budapest Schoolban a tanulásszervezők és mentorok veszik át & 
              \ref{sec:tanarfeladatok}.~fejezet \apageref{sec:tanarfeladatok}.~oldalon
              \\ \hline

    Az iskola által alkalmazott jelölés, értékelés érdemjegyre, osztályzatra való átváltásának szabályai. NKT 54. § (4)  &   & 
              \ref{sec:osztalyzatok}.~fejezet \apageref{sec:osztalyzatok}.~oldalon
              \\ \hline

    a kiemelt figyelmet igénylő tanulókkal kapcsolatos pedagógiai tevékenység helyi rendje  &   & 
              \ref{sec:kiemelt_figyelem}.~fejezet \apageref{sec:kiemelt_figyelem}.~oldalon
              \\ \hline

    a tanulóknak az intézményi döntési folyamatban való részvételi jogai gyakorlásának rendje  &   & 
              \ref{sec:sajat_szabalyok}.~fejezet \apageref{sec:sajat_szabalyok}.~oldalon
              \\ \hline

    a szülő, a tanuló, a pedagógus és az intézmény partnerei kapcsolattartásának formái  &  mind az alapelvekben, mind a döntéshozásban és konfliktusok kezelésében erre törekszik az iskola & 
              \ref{sec:kozossegi_elet}.~fejezet \apageref{sec:kozossegi_elet}.~oldalon és 
              \ref{sec:sajat_szabalyok}.~fejezet \apageref{sec:sajat_szabalyok}.~oldalon és 
              \ref{sec:konfliktusok_kezelese}.~fejezet \apageref{sec:konfliktusok_kezelese}.~oldalon
              \\ \hline

    a tanulmányok alatti vizsgák és az alkalmassági vizsga szabályai; az iskolai írásbeli, szóbeli, gyakorlati beszámoltatások, az ismeretek számonkérésének rendje  &  portfólióalapú értékelés a kerettanterv alapján & 
              \ref{sec:evfolyamok_osztalyzatok}.~fejezet \apageref{sec:evfolyamok_osztalyzatok}.~oldalon
              \\ \hline

    a felvétel és az átvétel -- Nkt. keretei közötti -- helyi szabályai  &   & 
              \ref{sec:felvetel-atvetel}.~fejezet \apageref{sec:felvetel-atvetel}.~oldalon
              \\ \hline

    az elsősegély-nyújtási alapismeretek elsajátításával kapcsolatos iskolai terv  &   & 
              \ref{sec:elsosegely}.~fejezet \apageref{sec:elsosegely}.~oldalon
              \\ \hline

    az iskolában alkalmazott sajátos pedagógiai módszerek, beleértve a projektoktatást  &  a pedagógiai program a tanárok kísérletezését támogatja & 
              \ref{sec:pedagogia_modszerek}.~fejezet \apageref{sec:pedagogia_modszerek}.~oldalon
              \\ \hline

    az iskola helyi tanterve  &   & 
              \ref{sec:tanterv-program}.~fejezet \apageref{sec:tanterv-program}.~oldalon
              \\ \hline

    a választott kerettanterv megnevezése  &  Budapest School Kerettantanterv & 
              \ref{sec:tanterv-program}.~fejezet \apageref{sec:tanterv-program}.~oldalon
              \\ \hline

    a választott kerettanterv által meghatározott óraszám feletti kötelező tanórai foglalkozások, továbbá a kerettantervben meghatározottakon felül a nem kötelező tanórai foglalkozások megtanítandó és elsajátítandó tananyaga, az ehhez szükséges kötelező, kötelezően választandó vagy szabadon választható tanórai foglalkozások megnevezése, óraszáma  &   & 
              \ref{sec:elsosegely}.~fejezet \apageref{sec:elsosegely}.~oldalon és 
              \ref{sec:nemzetiseg}.~fejezet \apageref{sec:nemzetiseg}.~oldalon és 
              \ref{sec:mindennapos-testmozgas}.~fejezet \apageref{sec:mindennapos-testmozgas}.~oldalon
              \\ \hline

    az oktatásban alkalmazható tankönyvek, tanulmányi segédletek és taneszközök kiválasztásának elvei (figyelembe véve a tankönyv térítésmentes igénybevétele biztosításának kötelezettségét)  &   & 
              \ref{sec:tankonyvek}.~fejezet \apageref{sec:tankonyvek}.~oldalon
              \\ \hline

    a Nemzeti alaptantervben meghatározott pedagógiai feladatok helyi megvalósításának részletes szabályai  &   & 
              \ref{sec:nat_celjai}.~fejezet \apageref{sec:nat_celjai}.~oldalon
              \\ \hline

    a mindennapos testnevelés, testmozgás megvalósításának módja, ha azt nem az Nkt. 27. § (11) bekezdésében meghatározottak szerint szervezik meg  &   & 
              \ref{sec:mindennapos-testmozgas}.~fejezet \apageref{sec:mindennapos-testmozgas}.~oldalon
              \\ \hline

    a választható tantárgyak, foglalkozások, továbbá ezek esetében a pedagógusválasztás szabályai  &   & 
              \ref{sec:modulok}.~fejezet \apageref{sec:modulok}.~oldalon és 
              \ref{sec:modulok_meghirdetese}.~fejezet \apageref{sec:modulok_meghirdetese}.~oldalon
              \\ \hline

    középiskola esetén azon választható érettségi vizsgatárgyak megnevezése, amelyekből a középiskola tanulóinak közép- vagy emelt szintű érettségi vizsgára való felkészítését az iskola kötelezően vállalja, továbbá annak meghatározása, hogy a tanulók milyen helyi tantervi követelmények teljesítése mellett melyik választható érettségi vizsgatárgyból tehetnek érettségi vizsgát  &   & 
              \ref{sec:erettsegi}.~fejezet \apageref{sec:erettsegi}.~oldalon
              \\ \hline

    a tanuló tanulmányi munkájának írásban, szóban vagy gyakorlatban történő ellenőrzési és értékelési módja, diagnosztikus, szummatív, fejlesztő formái, valamint a magatartás és szorgalom minősítésének elvei  &   & 
              \ref{sec:ertekeles}.~fejezet \apageref{sec:ertekeles}.~oldalon és 
              \ref{sec:jutalmazas_es_ertekeles}.~fejezet \apageref{sec:jutalmazas_es_ertekeles}.~oldalon
              \\ \hline

    a csoportbontások és az egyéb foglalkozások szervezésének elvei  &   & 
              \ref{sec:csoportbontasok}.~fejezet \apageref{sec:csoportbontasok}.~oldalon és 
              \ref{sec:csoportok}.~fejezet \apageref{sec:csoportok}.~oldalon
              \\ \hline

    a nemzetiséghez nem tartozó tanulók részére a településen élő nemzetiség kultúrájának megismerését szolgáló tananyag  &   & 
              \ref{sec:nemzetiseg}.~fejezet \apageref{sec:nemzetiseg}.~oldalon
              \\ \hline

    az egészségnevelési és környezeti nevelési elvek  &   & 
              \ref{sec:kornyezeti-neveles}.~fejezet \apageref{sec:kornyezeti-neveles}.~oldalon és 
              \ref{sec:mindennapos-testmozgas}.~fejezet \apageref{sec:mindennapos-testmozgas}.~oldalon
              \\ \hline

    a gyermekek, tanulók esélyegyenlőségét szolgáló intézkedések  &   & 
              \ref{sec:csoportbontasok}.~fejezet \apageref{sec:csoportbontasok}.~oldalon
              \\ \hline

    a tanuló jutalmazásával összefüggő, a tanuló magatartásának, szorgalmának értékeléséhez, minősítéséhez kapcsolódó elvek  &   & 
              \ref{sec:jutalmazas_es_ertekeles}.~fejezet \apageref{sec:jutalmazas_es_ertekeles}.~oldalon
              \\ \hline

    intézmény partnerei kapcsolattartásának formái  &   & 
              \ref{sec:az_iskola_kormanyzasa}.~fejezet \apageref{sec:az_iskola_kormanyzasa}.~oldalon
              \\ \hline




\end{longtable}

