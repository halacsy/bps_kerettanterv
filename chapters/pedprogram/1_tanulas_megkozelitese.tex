\section{Pedagógiai és pszichológiai háttér}

Az iskolák gyakrolatias tapasztalata mellett a Budapest School számos
elméletet is kiemelten fontosnak tart.

Ezek közül is oktatási programjának központjában Carol Dweck
fejlődésközpontű szemlélete áll. E mellett nagy hangsúlyt fektetünk az
alábbi elméletek gyakorlati alkalmazására is:

Reformpedagógiai irányzatok elméletei, különös tekintettel:

\begin{itemize}

      \item
            Motessori-pedagógia (Maria Montessori),
      \item
            kritikai pedagógia (Paulo Freire)
      \item
            élménypedagógia (John Dewey)
      \item
            felfedeztető tanulás (Jerome Bruner)
      \item
            projektmódszer (William Kilpatrick)
      \item
            kooperatív tanulás (Spencer Kagan)
\end{itemize}

Pszichológia és szociálpszichológiai kutatások eredményei:

\begin{itemize}

      \item
            kognitív interakcionista tanuláselmélet (Jean Piaget)
      \item
            személyközpontú pszichológia (Carl Rogers)
      \item
            kommunikáció és konfliktuskezelés (Thomas Gordon)
      \item
            erőszakmentes kommunikáció (Marshall Rosenberg)
      \item
            pozitív pszichológia kutatási eredményei, különös tekintettel:
            flow-elmélet, kreativitás kutatások (Csíkszentmihályi Mihály)
      \item
            érzelmi és társas intelligencia (Peter Salovey, John D. Mayer,
            Daniel
            Goleman)
      \item motiváció kutatások, amiket jól összefoglal \citep{pink2011drive}
      \item
            hősiesség pszichológiai alapjai (Phil Zimbardo)
      \item
            fejlődésfókuszú szemlélet (Carol Dweck)
\end{itemize}

\section{Pedagógia
  módszerek}
\label{sec:pedagogia_modszerek}

A Budapest School iskola tanárainak feladata, hogy mindig keressék azt a
módszert, azt a környezetet, ami az adott gyerekekkel, adott mikroiskolában a
leginkább működik. A módszer kiválaszatásakor mindig azt kell szem előtt
tartaniuk, hogy az a gyerekek egyéni céljaihoz, a közösség tanulási
lehetőségeihez a leginkább alkalmazkodjon. Nem tudjuk megmondani előre, hogy mikor milyen módszert érdemes választani, de
azt tudjuk, hogy mi alapján keressük a megfelelő technikákat. Vannak olyan módszerek, amelyek az
saját célok lehetőségeinek kitágítását és azok elérését nagyban támogatják.
Ezek alkalmazása javasolt a csoportmunkák és az egyéni gyakorlások alatt.

Azt is tudjuk,
hogy nem baj, ha nem elsőre találtuk meg a megfelelő módszert, hiszen a
próbálkozások során rengeteg új információt nyerünk, amelyek segítségével már
könnyebb megtalálni a valóban megfelelő megoldást. Alábbiakban néhány a Budapest School számára meghatározó fontosságú módszert
emelünk ki.

\subsection{Projektmódszer}
Projektmódszert alkalmazó modulok során fő célünk, hogy a gyerekek aktivak és
kreatívak legyenek, és ezért a tevékenységek sokszinűségét helyezzük fókuszba.
Projektmódszer esetén is bátorítva van minden tanár, hogy a legváltozatosabb
módszertárral közelítse a gyerekeket, figyelve arra, hogy a tanár attitűdje, a
csoport dinamikája és az aktuális tevékenységek mit kívánnak.

Projektmunka folyamata.

\begin{description}
      \item[Téma, cél] Első lépés, hogy meghatározzuk a projekt témáját vagy
            célját. Ez jöhet a tanártól, a gyerekektől, vagy akár egy szülőtől
            is.
            Fontos,
            hogy a gyerekek a projekt témáját már önmagában értelmesnek,
            relevánsnak
            tartsák.

      \item [Ötletroham]  Egy-egy téma feldolgozását csoportalakítással és
            ötletrohammal kezdjük. Ennek célja, hogy a résztvevők bevonódjanak,
            illetve
            megmutassák, hogy nekik milyen elképzeléseik vannak az adott
            témáról,
            továbbá
            milyen produktummal, eredménnyel szeretnék zárni a folyamatot. A
            létrehozott
            produktumoknak csak a képzelet szabhat határt. Lehetnek videók,
            prezentációk,
            fotók, rapdalok, telefonos applikációk, rajzok, tablók, tudományos
            cikkek stb.

      \item [Kutatói kérdés] Ezek után úgynevezett kutatói kérdéseket teszünk
            fel,
            melyek meghatározzák a vizsgálat irányát. A kérdések feldolgozása a
            legváltozatosabb módon történhet. Az egyéni munkától kezdve, a
            forntális
            instruáláson vagy a kooperatív csoportmunkán keresztűl egészen a
            dráma-
            és
            zenefoglalkozásokig minden hasznosítható a tanár, a csoport és a
            téma
            igényeihez mérten.
      \item [Elmélyült csoportmunka] A projekt azon szakasza, amikor a tervek,
            kutatások alapján az implementáción dolgozik a csapat.
      \item [Prezentáció] A létrehozott produktumokat bemutatására külön
            hangsúlyt
            kell fektetni. Ennek több módja is lehet: prezentációk,
            demonstrációk,
            plakátok, projekt fesztiválok.
\end{description}

Az iskolai projektek célja egy fejlődésfókuszú tanár és gyerek számára mindig
kettős: egyrészt cél a téma feldolgozása, a produktum létrehozása, másrészt az
iskola fő célja, hogy a gyerekek, a csapatok mindig fejleszzék alkotó,
együttműködő, problémamegoldó képességüket. Ezért a projekt folyamatára való
reflektálás, visszajelzés ugyanolyan fontos, mint maga a cél elérése.

A munka során külön figyelmet kell fordítani arra, hogy mindent dokumentáljanak
a résztvevők. Lehetőleg online felületen.
\paragraph{Értékelés} A projekt során több értékelési pontot érdemes beépíteni.
A foglalkozások végén a résztvevők visszajeleznek a folyamatra, értékelik a
saját, a csoport és tanár munkáját. A folyamat végén az egész projektfolyamatot
értékelik, szintén kitérve a saját, a csoport és a tanár munkájára. A
produktumok, az eredmény értékelése csoport és egyéni szinten is megtörténik.

Értékelésnél fontos a fejlődésfókuszú gondolkodásmódot fejlesztő visszajelzést
adni: inkább a folyamatra magára és ne a végeredményre fókuszáljon a
visszajelzés.

\subsection{Önszerveződő tanulási környezet}
A Sugata Mitra által kialakított módszertan (angolul Self Organizing Learning
Environment) lényege, hogy a tanárok arra
bátorítják a gyerekeket, hogy csoportban, az internet segítségével Nagy
Kérdéseket válaszoljanak meg. A jó kérdés az, amire nem egyszerű a válasz, sőt
lehet, hogy nincs is rá egyfajta választ. Cél, hogy a gyerekek maguk alakítsák
a folyamatot, formálják a kérdést és alakítsák a válaszokat.

\begin{itemize}
      \item Tanár kialakítja a teret: körülbelül négy gyerekenként egy
            számítógép,
            amit körbe lehet ülni.
      \item Gyerekek maguk formálják a csoportjukat, sőt, még csoportot is
            válthatnak a munka során. Mozoghatnak, kérdezhetnek, ,,leshetnek''
            a
            másik
            csoportoktól.
      \item Körülbelül 30-45 perc után a csoportok prezentálják a kutatásuk
            eredményét.
\end{itemize}

\paragraph{A jó kérdések}
Nagy kérdésekre nincs könnyű válasz. Ezek nyílt és nehéz kérdések, és
előfordulhat, hogy senki sem tudja még rá a választ. A cél, hogy mély és hosszú
beszélgetéseket generáljon a kérdés. Ezek azok a kérdések, amelyekre érdemes
nagyobb elméleteket állítani, amiket jobb csoportban megvitatni, amikről
érvelni lehet és kritikusan gondolkodni.

A jó kérdések több témát, területet (tantárgyat) kapcsolnak össze: ,,Mi a
hangya'' kérdés például nem érint annyi különböző területet, mint a ,,Mi
történne a Földdel, ha minden hangya eltünne''.

\paragraph{Fegyelmezés nélkül}
Tanár feladata a folyamat során meghatározni a nagy kérdést, és tartani a
kereteket. A cél, hogy a gyerekek maguk szervezzék saját munkájukat, így
minimális beavatkozás javasolt. Kezdetben, gyakorló csoportoknál a tanárnak
sokszor kell emlékeztetni magát, hogy idővel kialakul a rend. ,,Bízz a
folyamatban!''

Amikor a tanár úgy látja, hogy nem megy a munka, akkor csak finoman emlékezteti
a csoportokat, hogy lassan jön a prezentáció ideje. Amikor valaki a
csoportjáról panaszkodik, akkor elmondhatja, hogy szabad csoportot váltani. Ha
valaki zavarja a többieket, akkor megfigyelheti, hogy a gyerekek tudnak-e már
konfliktust feloldani. Ha valaki nem vesz rész a munkában, akkor gondolkozhat
olyan kérdésen, ami a demotivált gyerekeket is bevonzza.

\subsection{Megfontolt gyakorlás}
Ahogy \citep{ericsson2016peak} is kimutatja bárki tudja valamennyi készségét,
képességét fejleszteni, ha megtervezetten, megfontoltan gyakorolja. A Budapest
School a hagyományosan készségtárgyként számon tartott ének, rajz, testnevelés
és technika témákon kívül nagyon sok mindent kezel készségként: írásbeli
érettségi vizsgát tenni magyarból, geopolitikai elemzéseket végezni,
hiperbolikus függvényekkel egyenletet megoldani épp annyira értelmezhetőek
készségként, mint a domináns csoporttagokkal való együttműködés, vagy az hogy
egy stresszhelyzetben lenyugtassuk önmagunkat.

A készség és képesség fejlesztés legjobb eszköze a megtervezett gyakorlás:
fejlődés érdekében okosan gyakorlunk. A megfontolt gyakorlás jellemzője, hogy

\begin{description}
      \item[Világos és specifikus cél] Fontos, hogy tudjuk, mit gyakorlunk, mit
            akarunk elérni.  Lehetőleg a cél legyen mérhető és mindenképp
            realisztikus,
            elérhető.
      \item[Fókusz] Gyakorlás során egy dologra érdemes figyelni
      \item[Konfortzónán kivül kell lenni] Az edzőnek, tanárnak, trainernek
            néha
            érdemes a tanulót kicsit ,,nyomni''. Emlékeztetni, hogy mindig
            lehet
            kicsit
            többet elérni.
      \item[Folyamatos visszajelzés] Nagyon gyakran kap a tanuló visszajelzést,
            mindig tudja, hogy mikor és miben fejlődött.
\end{description}

\section{Jutalmazás és értékelés}
\label{jutalmazas}

\todo[inline]{ legyen growth mindsetet, praise the effort, ugy mint Ha adunk,
      akkor adjunk pozitív visszajelzést, fókuszáljunk arra, amit csinált a
      gyerek és nem arra, amit elért. jutalmazas bevezeto}

\begin{itemize}

      \item
            Fantasztikus, ma egy nagy kihívást választottál!
      \item
            Bátran rizikót vállaltál!
      \item
            Nagyon jó! Tényleg sokat próbálgattad.
      \item
            Kitartóan csináltad, erre nagyon büszke vagyok!
      \item
            De jó, valami újat próbáltál ki ma!
      \item
            Köszönöm, hogy ma valakinek segítettél.
      \item
            Nagyon nagy öröm látni a haladásodat!
      \item
            Ne feledd, mindannyian tudunk a hibáinkból tanulni. Örüljünk annak,
            hogy ma valamit jobban tudunk, mint előtte.
      \item
            Wow, egy nehéz feladatot oldottál meg!
      \item
            Szép munka! Kipróbáltál egy másik módszert.
\end{itemize}