Európa fölött repültünk épp egy fapados járattal, amikor először fogalmazódott meg bennünk, hogy szeretnénk azt az élményt megadni a gyerekeinknek, amit mi is kaptunk az élettől. A szabadon alkotás és tanulás semmivel sem összehasonlítható örömét. Ebben az állapotban a képzeletünk segítségével olyan célokat tűzünk ki magunk elé, amelyek kellő mértékű kihívást jelentenek, és közben lázba hoz minket a megvalósítás reménye. Az elképzeléseinket ezt követően átbeszéljük másokkal, hogy ötleteinket kritikusan újragondolhassuk, majd belevágunk, és amikor úgy érezzük, hogy eljutottunk valahova, akkor rátekintünk és megpróbáljuk kiértékelni az elért eredményeinket. Így írunk, mi emberek könyveket, így építünk hidakat, így kutatunk, és így hozunk létre újabb és újabb szervezeteket, melyekben közösségek működnek együtt, hogy céljaikat elérhessék.

Budapest felé repülve arról álmodoztunk, hogy létrehozhatunk egy olyan iskolát, ami semmi másra nem figyel, pusztán a gyerekeknek arra a képességére, ahogy tanulni, fejlődni tudnak a bennük rejlő kíváncsiság és érdeklődés mentén. És közben ott lebegett előttünk az ismeretlen megismerésének lehetősége. Hogy belevághatunk valamibe együtt, amiről akkor még semmit sem tudtunk. Egy új iskola megalapításába. Így jött létre a Budapest School 2015 nyarán. Barátainkkal való beszélgetéseinkben hamar eljutottunk a mikroiskolák hálózatának gondolatáig, melyek egy tanulási környezetbe szerveződve egymást segítik, egymástól tanulnak. Az első két óvodánk alakult meg ekkor, és bár hatalmas elánnal vágtunk bele, tudtuk, hogy ez még csak a bemelegítés. Mire egy évvel később már az első iskoláskorú gyerekeket is fogadtuk új helyszíneinken, úgy éreztük, hogy soha ilyen intenzíven nem tanultunk, mint a Budapest School megalapításának első évében. És ez az érzésünk azóta is elkísér.

Tanuló szervezetként tekintünk a Budapest Schoolra több okból is. Az iskoláról vallott elképzeléseink, ahogy az iskola szerepe globálisan is átalakulóban van. Csak egy iskolát működtetve tanulhatunk arról, hogy ezt az átalakulást, miként érdemes lekövetni. Az egyén fejlődéséről, a tanulás céljairól vallott elképzeléseink szintén sokat változnak. Egyre nagyobb szükségünk van arra, hogy a világ technológiai, kulturális és kommunikációs változásaihoz folyamatosan igazodjunk. És miközben a világ az egyén szabadsága felé tolja az embert, egyre nő a közösségben fejlődés, az egymástól tanulás, az együtt alkotás jelentősége. A tanuló szervezet számunkra egy olyan hely, ahol örömmel tanulhatunk másoktól és egymástól, ahol a tanár, a szülő és a gyerek együtt tesz a fejlődésért. Tanulunk magunkról, a világról és a kapcsolatainkról.

Ebben a könyvben azt mutatjuk be, amit az elmúlt években a tanulásról megtanultunk. Annak az útnak a tapasztalatait összegezzük, amely során az álomból az első csoportjaink létrejöttek, majd újabbak és újabbak alakultak. Ahogy nőtt a Budapest School, úgy kezdtük egyre jobban megérteni, mit is csinálunk és miként kell ezt a tanárok, a családok, a gyerekek, valamint a társadalom és a jogalkotó elvárásaihoz igazítanunk. Az elmúlt másfél évben azon dolgoztunk, hogy magántanulók közösségéből államilag elfogadott iskolává válhasson a Budapest School. Arra a kérdésre kerestük a választ, hogy létrehozhatunk-e  egy személyre szabott, önvezérelt tanulási modellt úgy, hogy az mindenben megfeleljen a törvényi előírásoknak. Ezért írtunk egy \emph{kerettantervet}, ami arra a kérdésre válaszol, hogy mit tanulnak a gyerekek. Amikor ezzel elkészültünk, leírtuk azt is, hogyan tanulnak, és mire a kérdés minden részlete kibomlott előttünk, elkészült a \emph{pedagógiai programunk} is. Menet közben ráébredtünk, hogy a mit és a hogyan kérdése közötti határok jóval elmósódottabbak, mint azt elsőre gondolnánk. A Budapest Schoolba járó gyerekek ugyanis éppannyira dönthetnek a tartalomról, mint amennyire formálói lehetnek a mindennapok működésének is. Ahogy a jogalkotó is megváltoztatja a törvényeket, hogy szándéka szerint jobbá tegye az állam működését, úgy egy tanuló szervezet is folyamatosan újraírja a saját szabályait igazodva a körülményekhez. Ebben a könyvben a hogyan és mit tanulnak a gyerekek kérdésekre adott válaszainkat gyúrtuk egybe.

Az elvégzett munkánk eredményeképp ma sokkal pontosabban látjuk, mennyi feladat, mennyi lehetőség áll még előttünk, hogy valóban létrehozhassuk álmaink iskoláját. Vagy ahogy mostanában gondoljuk, hogy létrehozzunk egy modern tanulásirányítási rendszert. Olykor elbizonytalanodunk, hogy valóban jól mértük-e fel, mibe is vágtunk bele, amikor elhatároztuk, hogy egy iskolahálózatot hozunk létre, amely egy egész generációra lesz hatással. Néha még abban is elbizonytalanodunk, hogy valóban jót teszünk-e a gyerekeinkkel. Aztán megkérdezzük őket, és megnyugtatnak, hogy imádják az iskolájukat. Azokban az áldott pillanatokban pedig amikor a külső elvárások nyomása nélkül a képzeletünk világába kerülhetünk újra, vagy épp időtlenül beszélgethetünk a terveinkről, újra ott érezzük magunkat a felhők között Budapest felé az égben, és hálásak vagyunk azért, hogy van egy iskolánk nekünk is, ahol örömmel tanulhatunk. Ez a Budapest School modell. Tanulni tanulunk, miközben alkotunk és felfedezünk. 

\bigskip
{
\raggedright Halácsy Péter és Halmos Ádám\\
\raggedright 2019. augusztus
}