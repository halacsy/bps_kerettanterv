\chapter{Bevezető}
A Budapest School alternatív kerettanterv a többcélú, egységes  Budapest School Általános Iskola és Gimnázium számára készült, ami 12 évfolyammal működő nevelési-oktatási intézményként ellátja az általános iskola és a gimnázium feladait. 

A kerettanterv a köznevelésről szóló 2011. évi CXC. törvény 9§ (8)-(9) bekezdésének felhatalmazása alapján a motiváció \cite{pink2011drive}\footnote{A hivatkozások jegyzékét lásd a \pageref{sec:bibliographyk}. oldalon}, a fejlődési szemlélet \cite{growthmindset} és az elmélyült gyakorlás \cite{ericsson2016peak}  pszichológiai kutatási eredményeire építve, 
a modern tanulásszervezési paradigmák, mint az önvezérelt \cite{mitra2012beyond} és személyre szabott \cite{khan2012one} tanulás alapján,
az  OECD, azaz a Gazdasági Együttműködési és Fejlesztési Szervezet \emph{Az iskolázás a jövőben}  (Schooling for tomorrow) programjával összhangban \cite{2006schooling}
határozza meg a Budapest School iskolákban folyó sajátos köznevelési tevékenységet.


A kerettanterv a köznevelési törvény által megengedett körben él a speciális szabályozás lehetőségével: meghatározza a modern tudományos eredményekre épülő tanulásszervezés folyamatát, értékelő/visszajelző rendszerét és az iskola szervezet-működési sajátosságokat, így különösen bemutatja, a pedagógus munkakör, az elfogadott pedagógus végzettségek és szakképzettségek tekintetében elvárt követelményeket, a Budapest School iskolára és telephelyeire irányadó építésügyi előírásokat, helyiség- és felszerelésigényt, a mikroiskola-hálózat modelljét, a Budapest School elveknek megfelelő minőségfejlesztés alapelveit, a mikroiskolák saját vezetési modelljét és szervezetét.

A kerettanterv elsősorban a tanulás folyamatát szabályozza, azaz elsősorban nem a \emph{mit tanulunk?} kérdésre, hanem a \emph{hogyan szervezzük meg, a tanulás?} kérdésre ad választ. Tantárgyi specifikációja a miniszter által kiadott kerettantervekre\cite{ofi:kerettanterv} épül, azokat strukturálja újra és rövidíti le. Kerettantervünk elsődleges célja, hogy bemutassuk hol, milyen módon, kitől és mit tanulhat a Budapest School egy mikroiskolájába járó gyerek. A 12. évfolyam végén a tanulóknak lehetőségük van arra, hogy leérettségizzenek. Tanulmányuk fő fukciója mégis mindvégig a tanulás aktív tanulása, saját fejlődésük kereteinek megtalálása, és a folyamatosan újragondolt egyéni célok állítása és az abba az irányba történő haladás marad.

A tanulás során a gyerekek a kerettantervben részletezett módon az érdeklődésüknek megfelelően specifikus tanulási egységeket, vagyis modulokat végeznek el, melynek eredményeit a saját portfóliójukban gyűjtik össze. Így a gyerekek tanulási útja a portfólió fejlődésével nyomonkövethető, és a portfólió tartalma alapján megállapítható a gyerek tudása, képességei.

A Budapest School kerettanterve a gyerekek, tanárok és a szülők közös döntésére bízza, hogy mit és hogyan tanulnak a Budapest School iskolákban. A kerettanterv a specifikus célkitűzés-tervezés, a tanulás, és az arra történő reflektálás módját írja le, vagyis a tanulás folyamatát rögzíti, míg annak pontos tartalmában szabadságot enged.

Ennek a szabadságnak a \emph{kereteit} adja meg a kerettanterv által definiált három tantárgy, amik tanulási célok listájaként kerültek megfogalmazásra. Ezek lefedik a NAT műveltségi területeit, fejlesztési céljait és kulcskompetenciáit. A gyerekeknek időről időre a tantárgyak által előre definiált tanulási célokból is kell választania, így biztosak lehetünk abban, hogy a gyerekek a NAT által előírt célokat teljesítik. 

A kerettantervünk semleges a pedagógiai módszerekkel kapcsolatban, a tanár feladatának tekinti, hogy mindig az optimálisnak tűnő tanulási, tanítási, gyakorlási módszert válassza. Ezért ez a kerettanterv nem beszél arról, hogy a modulokat (foglalkozásokat, tanórákat) milyen pedagógiai módszer alapján szervezi a tanár. A mindenkori hatályos törvényeket és a Nemzeti Alaptantervben foglaltakat a Budapest School minden tagja\footnote{Budapest School tagjai a gyerekek, tanárok, szülők és az adminisztrátorok.} kötelezőnek tekinti.

A kerettantervünk az iskolába járókat gyerekeknek hívja, nem tanulóknak és nem diákoknak. Ennek fő oka, hogy a rendszerünkben a tanárok és a szülők is tanulók, sőt az egész iskola egy tanuló szervezet, így nem akartuk  kizárólag az iskola egyik szereplőjére alkalmazni ezt a szót. Másodsorban hangsúlyozni szeretnénk, hogy a \emph{család} fontos szerepet kap a Budapest School rendszerében: itt a szülők, a gyerekek és a tanárok együttműködésben dolgoznak a fejlődésért. Azok a gyerekek, akik tanulmányaik vége felé felnőtté érnek a Budapest Schoolban tanulók is maradnak, és talán egy kicsit gyerekek is, ezért a szóhasználaton miattuk sem változtatunk. Az ő esetükben a gyerek az iskolába járó tanulót jelenti.

\section{Az iskola célja}

A Budapest School iskolái abban támogatják a gyerekeket, hogy kialakuljanak bennük azok az attitűdök, képességek és szokások, amelyek segítségével boldog, a társadalom számára hasznos és egészséges felnőttekké válhatnak. Az a célunk, hogy a gyerekek saját erősségeik kihasználásával a mai világ szükségleteihez és lehetőségeihez kapcsolódhassanak.

Olyan tanulási környezetet alakítunk ki ehhez, ahol a szülők, tanárok és gyerekek tesznek magukért és egymásért, ahol gyerekeink képesek nehéz helyzetekben is életük, kapcsolataik és környezetük aktív alakítói lenni, cselekedeteikkel, tetteikkel elérni a kitűzött céljaikat.

Hisszük, hogy mindenki kíváncsinak születik, és meg tudja tanulni azt, amit igazán szeretne. Nincs is másra szükség, csak izgalmas kihívásokra, kérdésekre, biztonságra, támogatásra és lehetőségekre.

Ennek szellemében az iskoláink elsődleges feladata, hogy a tanulók közösségét segítsék abban, hogy sokat és hatékonyan tanuljanak, alkossanak. \emph{Tanulják azt, amit szeretnének, és azt, amire szükségük van.} A tanulás három rétege, a tudásszerzés, a megtanult tudást elmélyítő és továbblendítő önálló gondolkodás és az aktív alkotás egyszerre jelenik meg a mindennapokban.
