\chapter{Jogszabályok által igényelt elvárások rövid összefoglalója}
\chaptermark{Egyéb jogszabályi elvárások}
\paragraph{A nevelés-oktatás célja}

Minél több olyan dolgot tanuljanak az iskola tanulói, amit szeretnek, vagy
amire szükségük van, úgy hogy a mindenkori Nemzeti Alaptanterv követelményeit
teljesítsék, és saját céljaikat is el tudják érni.

\paragraph{A tantárgyi rendszer}

A miniszter által kiadott kerettanterv tantárgyainak összevonásából létrejött
három Budapest School tantárgy a mindennapokban nem jelenik meg önálló
tanóraként, ezek funkciója a fejlesztési célok elérése és az egyensúlytartás. A
tudástartalom elsajátítása a tanulási modulokon keresztül történik. A tantárgy
elvégzésének feltétele, hogy a NAT pedagógiai szakaszainak befejeztével a
gyerekek egyéni portfóliója és a kerettantervben tanulási eredményekként
megadott követelmények kapcsolódjanak
egymáshoz úgy, hogy a tanulási eredmények legalább 50\%-ban teljesíti a gyerek.

\paragraph{A tantárgyközi tudás- és képességterületek fejlesztésének feladata}

A Budapest School iskola elvégzi a NAT kulcskompetenciáinak
fejlesztését, támogatja a NAT által meghatározott fejlesztési területek céljait
és ellátja a műveltségi területekhez rendelt fejlesztési feladatokat ellátja.
A modulok többsége nem tantárgyak alapján szerveződik, így nálunk a
tantárgyközi tudás az alapértelmezett.

\paragraph{A követelmények teljesítéséhez rendelkezésre álló kötelező, továbbá
    ajánlott időkeret}

A \pageref{tbl:oraszamok}. oldalon található \ref{tbl:oraszamok}. táblázat
alapján a három tantárgy által leírt területekkel az idő egyharmad részében
kell foglalkozni, ami minimum heti 4.27 óra 1-4 évfolyamszinten és 5.69 óra
5-12 évfolyamszinten
tagozaton.

\paragraph{Kötött és kötetlen munkaidő szabályozása}

A Budapest School iskolákban a munkaviszonyban álló tanárok rendes vagy kötetlen
munkaidőben dolgoznak.

A kerettanterv megközelítése, hogy a tanárok a kerettanterv, a saját és közösségi
célok által kialakított eredmények elérésére vállalnak kötelezettséget.

Ezért az Nkt. és kapcsolódó jogszabályok által előírt munkaidő-beosztási szabályoktól a jelen kerettanterv eltérést enged. Az eltérések mikroiskolánként különbözőek. Az eltérések a tanárok szerződéseiben, és az iskola megfelelő dokumentumaiban kerülnek rögzítésre.

A mikroiskola tanárcsapatának a
feladata a szükséges óraszámok, beosztások kialakítása és a megfelelő
modulvezetők megtalálása. Amennyiben egy tanár maga kevesebb modult tart, és
több munkát tölt a tanulás megszervezésével, akár több mikroiskola tanáraként
is dolgozhat egy időben.

\paragraph{Elfogadott pedagógus végzettség és szakképzettség}

A Budapest School iskolában tanulásszervező tanárok, mentortanárok és
modulvezető tanárok dolgoznak együtt, hogy a gyerekek számára megfelelő
tanulási környezetet, kihívásokat, kereteket (stb.) biztosítsanak. A tanárok
lehetnek alkalmazotti vagy megbízási jogviszonyban az iskolával.

Fontos alapelv, hogy a tanárok személyisége, tudása, képességei és
kompetenciái határozzák meg a gyerekek élményét.  Ezért az és csak az lehet
tanár, aki képes segíteni a tanulókat a
tanulásban. A végzettség a tanárok hatékonyságának egyik indikátora.

\paragraph{A tanulásszervezők és mentoroktól elvárt képesítések} A
tanulásszervezők és a mentorok a mikroiskolák irányítói, a gyerekek tanulási
környezetének alakítói és menedzserei. Elvárás, hogy csoportos tanulási és
fejlesztési helyzet vezetéséhez szükséges képességeik meglegyenek. Ehhez
szükséges \aref{tbl:vegzettsegek} táblázatban felsorolt végzettségek közül
valamelyik, vagy ezeket kiváltó
releváns, igazolható szakmai tapasztalat.

Az érettségire felkészítő modulok esetén a tantárgyhoz tartozó szakos tanári
diploma elengedhetetlen.

\begin{table}[h]
    \begin{tabular}{l | c}
        \textbf{elfogadott végzettség}     & \textbf{évfolyam}
        \\ \hline \hline
        drámapedagógus                     & 1-12              \\ \hline
        gyógypedagógus                     & 1-12              \\ \hline
        hittanár-nevelő tanár              & 6-12              \\ \hline
        játék- és szabadidő-szervező tanár & 6-12              \\ \hline
        kollégiumi nevelőtanár             & 6-12              \\ \hline
        konduktor                          & 1-12              \\ \hline
        mérnöktanár                        & 6-12              \\ \hline
        múzeumpedagógus                    & 1-12              \\ \hline
        nyelv- és beszédfejlesztő tanár    & 1-12              \\ \hline
        pedagógia szakos nevelő            & 1-12              \\ \hline
        pedagógia szakos tanár             & 6-12              \\ \hline
        pszichológus                       & 1-12              \\ \hline
        szakos tanár                       & 6-12              \\ \hline
        szociálpedagógus                   & 1-12              \\ \hline
        tanulási és pályatanácsadó tanár   & 6-12              \\ \hline
        tanító                             & 1-10              \\ \hline
        tehetségfejlesztő tanár            & 6-12              \\ \hline
        waldorf tanító                     & 1-12              \\ \hline
        óvodapedagógus                     & 1-4               \\ \hline
    \end{tabular}
    \caption{Tanulásszervező munkakörben elfogadott végzettségek.}
    \label{tbl:vegzettsegek}
\end{table}

\paragraph{Modulvezetőktől elvárt végzettségek}
A modulvezetőktől azt várjuk el, hogy magas szinten ismerjék a modul tematikája
által lefedett területet. Az érettségire
felkészítő modulok esetén pedagógiai szakképesítést várunk el. Itt a
tantárgyhoz tartozó tanári diploma
elengedhetetlen.

A többi modul esetén a modul témájához kapcsolatos legalább 5 éves szakmai
tapasztalat szükséges.

\paragraph{Épületekre vonatkozó előírások}
\Aref{tbl:helyisegek}. táblázat meghatározza Budapest School Általános Iskola
és Gimnázium munkájához kötelezően szükséges helyiségeket -- az Nkt. 9.§ (8) f
pont felhatalmazása alapján -- a kerettantervben részletezett strukturális,
szervezeti és tanulásszervezési elveket és folyamatokat, különösen a
mikroiskola-hálózatos működési jellegét figyelembe véve.

\begin{longtable}{@{}p{4cm}|p{4cm}|p{6cm}@{}}

    \textbf{helyiség megnevezése} & \textbf{mennyiségi mutató}
                                  & \textbf{megjegyzés}
    \\ \hline
    tanterem                      & 16 gyerekenként egy                 &
    1,5 nm /
    gyerek előírás figyelembevételével
    \\ \hline
    tornaterem                    & iskolánként egy                     &
    kiváltható
    szerződéssel
    \\ \hline
    tornaszoba                    & székhelyenként, telephelyenként egy &
    ha a
    székhelynek v. telephelynek nincs saját tornaterme, kiváltható szerződéssel
    \\ \hline
    sportudvar                    & székhelyen, telephelyeken egy       &
    kiváltható
    szerződéssel, vagy helyettesíthető alkalmas szabad területtel
    \\ \hline
    sportszertár                  & iskolánként egy                     &
    tornateremhez kapcsolódóan (kiváltható szerződéssel)
    \\ \hline
    iskolatitkári iroda           & iskolánként egy                     &
    tantestületi szobával közös helyiségben  is kialakítható
    \\ \hline
    nevelőtestületi szoba         & iskolánként (telephelyenként) egy   &

    \\ \hline
    könyvtár                      & iskolánként egy                     &
    nyilvános
    könyvtár elláthatja a funkcót, megállapodás alapján
    \\ \hline
    orvosi szoba                  & iskolánként egy                     &
    amennyiben
    egészségügyi intézményben a gyerekek ellátása megoldható, nem kötelező
    \\
    \hline
    ebédlő                        & székhelyen, telephelyeken egy       &
    gyerek és
    felnőtt étkező közös helyiségben
    \\ \hline
    főzőkonyha                    & székhelyen, telephelyeken egy       &
    ha helyben
    főznek
    \\ \hline
    melegítőkonyha                & székhelyen, telephelyeken egy       &
    ha nem
    helyben főznek, de helyben étkeznek
    \\ \hline
    tálaló-mosogató               & székhelyen, telephelyeken egy       &
    ha nem
    helyben főznek, de helyben étkeznek; melegítő konyhával közös helyiségben
    is
    kialakítható
    \\ \hline
    éléskamra                     & székhelyen, telephelyeken egy       &
    ha helyben
    főznek
    \\ \hline
    szárazáru raktár              & székhelyen, telephelyeken egy       &
    ha helyben
    főznek
    \\ \hline
    földesáru raktár              & székhelyen, telephelyeken egy       &
    ha helyben
    főznek
    \\ \hline
    személyzeti wc                & székhelyen, telephelyeken egy       &

    \\ \hline
    gyerek wc                     & székhelyen, telephelyeken egy       &
    tanulói
    létszám figyelembevételével
    \\

    \caption{Kötelező helyiségek listája.}
    \label{tbl:helyisegek}

\end{longtable}

\paragraph{Helyiségek bútorzata és egyéb berendezési tárgyai}

A szükséges
berendezések és eszközök tekintetében a kerettanterv a	20/2012. (VIII. 31.)
EMMI rendelet
II.~mellékletét tekinti a fenntartó számára irányadónak, kivéve hogy
\begin{itemize}
    \item fenntartónak stabil szélessávú internethozzáférést kell
          biztosítania minden telephelyen;
    \item minden tanteremben minden órán elérhetőnek kell lennie legalább egy
          internethez kötött számítógépnek vagy tabletnek;
    \item nevelőstesületi szobában nem ,,\emph{pedagóguslétszám szerint 1
              fiókos
              asztal és szék}'' szükséges, hanem csak amennyire a tanárok
          számára fontos;
    \item tantermekben nincs szükség \emph{1 nevelői asztalra és székre}, ha a
          tanárok tudnak a gyerekekkel együtt tevékenykedni.
\end{itemize}

Amennyiben a tanulásszervezőknek és a modulvezetőknek többletigénye merül fel,
akkor a szervezeti és működési szabályzatban lefektett módon tudják a fenntartó segítségét kérni.