\chapter{Jogszabályok által igényelt elvárások rövid összefoglalója}
\chaptermark{Egyéb jogszabályi elvárások}
\label{sec:jogszabalyok}
\section{A nevelés-oktatás célja}

Minél több olyan dolgot tanuljanak a gyerekek, amit szeretnek, vagy amire szükségük van úgy, hogy a mindenkori Nemzeti alaptanterv követelményeit teljesítsék, és saját céljaikat is el tudják érni.

\section{A tantárgyi rendszer}

A kerettanterv a miniszter által kiadott kerettantervek tantárgystuktúráját átvéve tanulási eredmények halmazaként adja meg a tantárgyakat. A rendszert \aref{sec:tantargyak}.~fejezet részletesen ismerteti.  Budapest School-tantárgy a mindennapokban nem feltétlenül jelenik önálló tanóraként, ezek funkciója a fejlesztési célok elérése és az egyensúlytartás. A tudástartalom elsajátítása a tanulási modulokon keresztül történik. A tantárgy elvégzésének feltétele, hogy  a gyerekek egyéni portfóliója lefedje a kerettantervben tanulási eredményekként megadott követelményeket.

A tantárgyi struktúra nem tér el a miniszter által kiadott kerettantervek tantárgyi struktúrájától, mert ennek a keretanttervnek a tantárgyainak tartalmát kijelőlő tanulási eredmények kölcsönösen megfeleltethetőek a miniszter által kiadott kerettantervek eredményelvárásaival. Egyszerűen fogalmazva, ugyanazt a tananyagot definiálja ez a kerettanterv, ugyanabban a stuktúrában, mint a miniszter által kiadott kerettantervek. 

A Nat-ban meghatározott tananyag tartalmakat tanévenként két félévre bontva jellenek meg a kerettantervekben, összhangban a tanulók félévenkénti értékelésével  \aref{sec:tantargyi_tanulasi_eredmenyek}. fejezetben.

\section{A Nat-ban meghatározott tananyag tartalmak félévenkénti bontása}
A kerettanterv a tananyag tartalmakat tanulási eredmények halmazaként adja meg tantárgyanként és félévenkénti bontásban \aref{sec:tantargyi_tanulasi_eredmenyek}.~fejezetben. 

\Aref{sec:osztalyzatok}.~fejezetben leírt módon meghatározza a gyerekek félévenkénti értékelését, ami \aref{sec:feleves_bontas}~.fejezetben leírt módon kerül összhangba a trimeszter alapú tanév struktúrával.

\section{Tantárgyi struktúra eltérése} 
A kerettanterv tantárgyi struktúráját \aref{sec:tantargyak}.~fejezet tartalmazza. Ez pontosan ugyanazokat a tantárgyakat tartamazza, mint mint a miniszter által kiadott kerettantervben foglalt tantárgyi struktúrának. A tantárgyakhoz rendelt heti óraszámok a miniszter által kiadott kerettantervben megadott óraszámok 75\%-a.

\section{Kerettanterv eltérése}
A 51/2012. (XII. 21.) EMMI rendelet 7. § (cb) pontja szerint a kerettanterv részét képező tantárgyi kerettantervek tematikai egységeinek címei, az ismeretek és fejlesztési követelmények tantervi elemek tartalma és az azokhoz rendelt óraszámok vonatkozásában legalább húsz százalékban el kell térnia a miniszter által kiadott vagy jóváhagyott kerettantervtől.

A kerettanterv nem határoz meg tematikai egységeket, azok kidolgozását a tanárokra bízza a modulok összeállításakor. Ugyanakkor a kerettanterv szorgalmazza három kiemelt, interdiszciplináris fejlesztési irányelv (\ref{sec:kiemelt_fejlesztesi_iranyelvek}) követését, ami nagyban eltére a miniszter által kiadott kerettantervek tantárgy alapú fejlesztési követelményeitől. 

A tantárgyakhoz rendelt óraszámokban a kerettanterv  25\%-ban eltér a miniszter által kiadott kerettantervektől.

\section{NAT céljainak támogatása}
\label{sec:nat_celjai}
A Nemzeti alaptantervben szereplő fejlesztési célok elérését és a
kulcskompetenciák fejlődését több minden támogatja:

Egyrészt a tantárgyak lefedik a NAT fejlesztési céljait, kulcskompetenciáit és
műveltségi területeit, mert a jelenleg érvényben lévő, a miniszter által az
\emph{51/2012.~(XII.~21.) számú EMMI-rendelet I--IV.~mellékletében} kiadott
kerettantervek \citep{ofi:kerettanterv} tanulási, tanítási eredményeiből
indultunk ki. Mivel a rendeletben szereplő kerettantervek teljesítik a NAT
feltételeit, így a Budapest School tantárgystruktúrája is teljesíti ezeket.

Másrészt az iskola életében, folyamatában való részvétel már önmagában
biztosítja a kulcskompetenciák fejlődését és a NAT fejlesztési céljainak
teljesülését sok esetben.

Az \ref{tbl:nat_fejlesztesi}.~táblázat bemutatja a NAT fejlesztési területeihez
való kapcsolódást, az
\ref{tbl:nat_kulcs}.~táblázat pedig az illeszkedési pontokat a NAT
kulcskompetenciáihoz.

\begin{table}
    \centering
  \begin{tabular}{p{5cm}|p{3cm}}

    \textbf{A NAT fejlesztési céljai}             & \textbf{Struktúra}           \\
    \hline
    Az erkölcsi nevelés                           & közösség                     \\ \hline
    Nemzeti öntudat, hazafias nevelés             & projektek                    \\ \hline
    Állampolgárságra, demokráciára nevelés        & közösség                     \\ \hline
    Az önismeret és a társas kultúra fejlesztése  & saját tanulási út, közösség  \\ \hline
    A családi életre nevelés                      &                              \\ \hline
    A testi és lelki egészségre nevelés           & közösség                     \\ \hline
    Felelősségvállalás másokért, önkéntesség      & közösség, pro\-jek\-tek      \\
    \hline
    Fenntarthatóság, környezettudatosság          & projektek                    \\ \hline
    Pályaorientáció                               & saját tanulási út            \\ \hline
    Gazdasági és pénzügyi nevelés                 & projektek                    \\ \hline
    Médiatudatosságra nevelés                     & projektek                    \\ \hline
    A tanulás tanítása                            & saját tanulási út, mentorság \\

  \end{tabular}
  \caption{A NAT fejlesztési céljainak elérését nemcsak a tantárgyak, hanem az
    iskola struktúrája is támogatja.}
  \label{tbl:nat_fejlesztesi}
\end{table}

A \emph{saját tanulási} út fogalma például önmagában segíti a tanulás
tanulását, hiszen az a gyerek, aki képes önmagának saját célt állítani (mentori
segítséggel), azt elérni, és a folyamatra való reflektálás során képességeit
javítani, az fejleszti a tanulási képességét.

Vagy másik példaként, a Budapest School iskoláiban a \emph{közösség} maga hozza
a működéséhez szükséges szabályokat, folyamatosan alakítja és fejleszti saját
működését a tagok aktív részvételével. Ez az aktív állampolgárságra, a
demokráciára való nevelés Nemzeti alaptantervben előírt céljait is támogatja.

\begin{table}
  \centering
  \begin{tabular}{p{5cm}|p{3cm}}

    \textbf{NAT kulcskompetenciái}                      &
    \textbf{Struktúra}                                                                  \\ \hline
    Anyanyelvi kommunikáció                             & tanulási szerződés, portfólió \\ \hline
    Idegen nyelvi kommunikáció                          & idegen nyelvű\hfill\break modulok          \\ \hline
    Matematikai kompetencia                             &                               \\ \hline
    Természettudományos és technikai kompetencia        & projektek                     \\ \hline
    Digitális kompetencia                               & digitális portfóliókezelés   \\ \hline
    Szociális és állampolgári kompetencia               & saját tanulási út,
    közösség                                                                            \\ \hline
    Kezdeményezőképesség és vállalkozói kompetencia     & saját tanulási út, közösség   \\ \hline
    Esztétikai-művészeti tudatosság és kifejezőkészség  &                               \\ \hline
    A hatékony, önálló tanulás                          & saját tanulási út,
    mentorság                                                                           \\

  \end{tabular}
  \caption{A NAT kulcskompetenciáinak fejlesztését támogatják a tantárgyak és
    az iskola felépítése is.}
  \label{tbl:nat_kulcs}
\end{table}



\section{A tantárgyközi tudás- és képességterületek fejlesztésének feladata}

A Budapest School iskola elvégzi a NAT kulcskompetenciáinak fejlesztését, támogatja a NAT által meghatározott fejlesztési területek céljait, és ellátja a műveltségi területekhez rendelt fejlesztési feladatokat. A modulok többsége nem tantárgyak alapján szerveződik, így nálunk a tantárgyközi tudás az alapértelmezett.

\section{A követelmények teljesítéséhez rendelkezésre álló kötelező, továbbá ajánlott időkeret}

\Apageref{tbl:oraszamok}. oldalon található \ref{tbl:oraszamok}.~táblázat
megadja a kiemelt tantárgyakkal töltendő minimális óraszámokat két hetes egységekre. A minimális óraszámokat mindenképpen biztosítani kell. 
Ezenfelül a gyerekek számára ajánlott minél többet foglalkozni azzal a tématerülettel, amit ők szeretnek vagy szükségük van rá.

\section{Kötött és kötetlen munkaidő szabályozása}

A Budapest School iskolákban a munkaviszonyban álló tanárok rendes vagy kötetlen munkaidőben dolgoznak.
A kerettanterv megközelítése, hogy a tanárok a kerettanterv, a saját és közösségi célok által kialakított eredmények elérésére vállalnak kötelezettséget. Ezért az Nkt. és kapcsolódó jogszabályok által előírt munkaidő-beosztási szabályoktól a jelen kerettanterv eltérést enged. Az eltérések mikroiskolánként különbözőek. Az eltérések a tanárok szerződéseiben, és az iskola megfelelő dokumentumaiban kerülnek rögzítésre.

A mikroiskola tanárcsapatának a feladata a szükséges óraszámok, beosztások kialakítása és a megfelelő modulvezetők megtalálása. Amennyiben egy tanár maga kevesebb modult tart, és több munkát tölt a tanulás megszervezésével, akár több mikroiskola tanáraként is dolgozhat egy időben.

\section{Elfogadott pedagógus végzettség és szakképzettség}

A Budapest\break School iskolában tanulásszervező tanárok, mentortanárok és modulvezető tanárok dolgoznak együtt, hogy a gyerekek számára megfelelő tanulási környezetet, kihívásokat, kereteket (stb.) biztosítsanak. A tanárok lehetnek alkalmazotti vagy megbízási jogviszonyban az iskolával.

Fontos alapelv, hogy a tanárok személyisége, tudása, képességei és kompetenciái határozzák meg a gyerekek élményét.  Ezért az és csak az lehet tanár, aki képes segíteni a gyerekeket a tanulásban. A végzettség a tanárok hatékonyságának egyik indikátora.

\paragraph{A tanulásszervezőktől és mentoroktól elvárt képesítések} A
tanulásszervezők és a mentorok a mikroiskolák irányítói, a gyerekek tanulási környezetének alakítói és menedzserei. Elvárás, hogy a csoportos tanulási és fejlesztési helyzet vezetéséhez szükséges képességeik meglegyenek. Ehhez szükséges \aref{tbl:vegzettsegek} táblázatban felsorolt végzettségek közül valamelyik, vagy ezeket kiváltó releváns, igazolható szakmai tapasztalat.

Az érettségire felkészítő modulok esetén a tantárgyhoz tartozó szakos tanári diploma elengedhetetlen.

\begin{table}[ht]
    \begin{tabular}{l | c}
        \textbf{elfogadott végzettség}     & \textbf{évfolyam}
        \\ \hline \hline
        drámapedagógus                     & 1--12             \\ \hline
        gyógypedagógus                     & 1--12             \\ \hline
        hittanár-nevelő tanár              & 6--12             \\ \hline
        játék- és szabadidő-szervező tanár & 6--12             \\ \hline
        kollégiumi nevelőtanár             & 6--12             \\ \hline
        konduktor                          & 1--12             \\ \hline
        mérnök tanár                       & 6--12             \\ \hline
        múzeumpedagógus                    & 1--12             \\ \hline
        nyelv- és beszédfejlesztő tanár    & 1--12             \\ \hline
        pedagógia szakos nevelő            & 1--12             \\ \hline
        pedagógia szakos tanár             & 6--12             \\ \hline
        pszichológus                       & 1--12             \\ \hline
        tantárgyi szakos tanár             & 6--12             \\ \hline
        szociálpedagógus                   & 1--12             \\ \hline
        tanulási és pályatanácsadó tanár   & 6--12             \\ \hline
        tanító                             & 1--10             \\ \hline
        tehetségfejlesztő tanár            & 6--12             \\ \hline
        Waldorf tanító                     & 1--12             \\ \hline
        óvodapedagógus                     & 1--4              \\ \hline
    \end{tabular}
    \caption{Tanulásszervező munkakörben elfogadott végzettségek.}
    \label{tbl:vegzettsegek}
\end{table}

\subsection{Modulvezetőktől elvárt végzettségek}
A modulvezetőktől azt várjuk el, hogy magas szinten ismerjék a modul tematikája által lefedett területet. Az érettségire felkészítő modulok esetén pedagógiai szakképesítést várunk el. Itt a tantárgyhoz tartozó tanári diploma elengedhetetlen.

A többi modul esetén a modul témájához kapcsolatos legalább 5 éves szakmai tapasztalat szükséges. Több tantárgy tartalmát lefedő modul esetén nem szükséges minden tantárgyhoz kapcsolatos végzettség: például egy összevont természettudományos kisérletezés modult ugyanúgy tud egy fizikus vagy egy biológus tartani.

\paragraph{Épületekre vonatkozó előírások}
\Aref{tbl:helyisegek}. táblázat meghatározza a Budapest School Általános Iskola
és Gimnázium munkájához kötelezően szükséges helyiségeket -- az Nkt. 9.§ (8) f pont felhatalmazása alapján -- a kerettantervben részletezett strukturális, szervezeti és tanulásszervezési elveket és folyamatokat, különösen a mikroiskola-hálózatos működési jellegét figyelembe véve.

\begin{longtable}{@{}p{4cm}|p{4cm}|p{6cm}@{}}

    \textbf{helyiség megnevezése} & \textbf{mennyiségi mutató}
                                  & \textbf{megjegyzés}
    \\ \hline
    tanterem                      & 16 gyerekenként egy                 &
    1,5 nm /
    gyerek előírás figyelembevételével
    \\ \hline
    tornaterem                    & iskolánként egy                     &
    kiváltható
    szerződéssel
    \\ \hline
    tornaszoba                    & székhelyenként, telephelyenként egy &
    ha a
    székhelynek v. telephelynek nincs saját tornaterme, kiváltható szerződéssel
    \\ \hline
    sportudvar                    & székhelyen, telephelyeken egy       &
    kiváltható
    szerződéssel, vagy helyettesíthető alkalmas szabad területtel
    \\ \hline
    sportszertár                  & iskolánként egy                     &
    tornateremhez kapcsolódóan (kiváltható szerződéssel)
    \\ \hline
    iskolatitkári iroda           & iskolánként egy                     &
    tantestületi szobával közös helyiségben  is kialakítható
    \\ \hline
    nevelőtestületi szoba         & iskolánként (telephelyenként) egy   &

    \\ \hline
    könyvtár                      & iskolánként egy                     &
    nyilvános
    könyvtár elláthatja a funkcót, megállapodás alapján
    \\ \hline
    orvosi szoba                  & iskolánként egy                     &
    amennyiben
    egészségügyi intézményben a gyerekek ellátása megoldható, nem kötelező
    \\
    \hline
    ebédlő                        & székhelyen, telephelyeken egy       &
    gyerek- és
    felnőttétkező közös helyiségben; tanteremmel közös helységben is kialakítható, de egyidőben csak egy funkció
    \\ \hline
    főzőkonyha                    & székhelyen, telephelyeken egy       &
    ha helyben
    főznek
    \\ \hline
    melegítőkonyha                & székhelyen, telephelyeken egy       &
    ha nem
    helyben főznek, de helyben étkeznek
    \\ \hline
    tálaló-mosogató               & székhelyen, telephelyeken egy       &
    ha nem
    helyben főznek, de helyben étkeznek; melegítőkonyhával közös helyiségben
    is
    kialakítható
    \\ \hline
    éléskamra                     & székhelyen, telephelyeken egy       &
    ha helyben
    főznek
    \\ \hline
    szárazáruraktár               & székhelyen, telephelyeken egy       &
    ha helyben
    főznek
    \\ \hline
    földesáru-raktár              & székhelyen, telephelyeken egy       &
    ha helyben
    főznek
    \\ \hline
    személyzeti WC                & székhelyen, telephelyeken egy       &

    \\ \hline
    gyerek WC                     & székhelyen, telephelyeken egy       &
    gyereklétszám figyelembevételével
    \\

    \caption{Kötelező helyiségek listája.}
    \label{tbl:helyisegek}

\end{longtable}

\vfill\eject

\paragraph{Helyiségek bútorzata és egyéb berendezési tárgyai}

A szükséges berendezések és eszközök tekintetében a kerettanterv a	20/2012. (VIII. 31.) EMMI rendelet II.~mellékletét tekinti a fenntartó számára irányadónak, kivéve hogy
\begin{itemize}
    \item fenntartónak stabil szélessávú internethozzáférést kell biztosítania minden telephelyen;
    \item minden tanteremben minden órán elérhetőnek kell lennie legalább egy internethez kötött számítógépnek vagy tabletnek;
    \item nevelőstesületi szobában nem ,,\emph{pedagóguslétszám szerint 1 fiókos asztal és szék}'' szükséges, hanem csak amennyire a tanárok számára fontos;
    \item tantermekben nincs szükség \emph{1 nevelői asztalra és székre}, ha a tanárok tudnak a gyerekekkel együtt tevékenykedni.
\end{itemize}

Amennyiben a tanulásszervezőknek és a modulvezetőknek többletigénye merül fel, akkor a szervezeti és működési szabályzatban lefektett módon tudják a fenntartó segítségét kérni.