\section{Saját tanulási célok}
\label{sec:tanulasi_celok}

Minden gyerek megfogalmazza és háromhavonta újrafogalmazza a \emph{saját
      tanulási céljait}: eredményeket, amelyeket el akar érni, képességeket,
amelyeket fejleszteni akar, szokásokat, amelyeket ki akar alakítani. A saját
célok elfogadásakor a gyerek és a mentora a szülőkkel együtt \emph{tanulási
      szerződést} köt.

Csak olyan célok kerülhetnek a saját célok közé, amelyek
minden érintettnek biztonságosak, és amelyek összhangban vannak a tantárgyi
fejlesztési célokkal és tanulási eredményekkel. A szerződésben rögzíthetőek
tanulási eredményekre
vonatkozó megállapodások,
tantárgyi évfolyamszintekre vonatkozó elvárások (pl. ,,\emph{haladjon egy
      évfolyamszintet egy év alatt}'' vagy ,,\emph{készüljön fel emelt szintű
      érettségire}''), és a tantárgyi rendszeren kívüli célok és
feladatok.

Fontos megkötés, hogy a saját tanulási célok legalább a felének
\ifkerettanterv
      \aref{sec:tantargyi_tanulasi_eredmenyek}. fejezetben
\else
      a kerettanterv Tantárgyi tanulási eredmények fejezetében
\fi
felsorolt tanulási eredmények elérésére kell vonatkoznia. A másik
fele szabadon alakítható.

Háromhavonta a tanulásszervezők és a gyerekek megállnak, reflektálnak az elmúlt
időszakra, és a tapasztalatok, valamint az elért célok ismeretében és az új
célok figyelembevételével újratervezik, újraszervezik a foglalkozások rendjét,
tehát azt, hogy mikor és mit csinálnak majd a gyerekek az iskolában.
A mindennapi tevékenység során tapasztalt élmények, alkotások, elvégzett
feladatok, kitöltött vizsgák, tehát mindaz, ami a gyerekekkel történik, bekerül
a portfóliójukba. Még az is, amit nem terveztek meg előre.

A gyerekeket a mentoruk segíti a saját célok kitűzésében, a különböző
választásoknál, a portfólióépítésben, a reflektálásban. A tanulási célok
kitűzése az önirányított tanulás fokozatos fejlődésével és az életkor
előrehaladtával folyamatosan egyre önállóbb tevékenységgé válik. Tanulási
útján, céljai kitűzésében a mentor kíséri végig a gyerekeket.

A Budapest School személyre szabott tanulásszervezésének jellegzetessége, hogy
a gyerekek a saját céljuk irányába haladnak, az adott célhoz az adott
kontextusban leghatékonyabb úton. Tehát mindenki rendelkezik saját célokkal,
még akkor is, ha egy közösség tagjainak céljai a tantárgyi tanulási eredmények
azonossága, vagy a hasonló érdeklődés miatt akár  80\% átfedést mutatnak.

A NAT műveltségi területeiben megfogalmazott követelmények teljesítése is célja
a tanulásnak, a tanulás fő irányítója azonban más. Mi azt kérdezzük a
gyerekektől, hogy \emph{ezenfelül} mi az ő személyes céljuk.

\section{A tanulási szerződés}

A tanulási szerződés az előbbiekben említett gyerek-mentor-szülő közötti
megállapodás, ami rögzíti
\begin{enumerate}
      \item a gyerek, a mentor (iskola) és a szülő igényeit, elvárásait;

            ezek lehetnek: \emph{,,szeretném, ha a gyerekem naponta olvasna''}
            típusú
            folyamatra vonatkozó kérések, vagy erősebb \emph{,,változtatnod
                  kell a
                  viselkedéseden, ha a közösségben akarsz maradni''} igények,
            határok
            megfogalmazása;

      \item a gyerek céljait a következő trimeszterre, vagy a tanév végéig;

      \item a gyerek, mentorok (iskola) és szülő vállalásait, amivel támogatják
            a
            cél
            elérését és a felek igényének elérését.

\end{enumerate}

A tanulási szerződésre jellemző, hogy
\begin{itemize}
      \item A kitűzött célokat minél specifikusabban, mérhetőbben kell
            megfogalmazni.
            Javasolt az OKR  (Objectives and Key Results, azaz	Cél és Kulcs
            Eredmények)
            \citep{okr} vagy a SMART (Specific, Measurable, Achievable,
            Relevant,
            Time-bound, azaz Specifikus,  Mérhető, Elérhető, Releváns és Időhöz
            kötött)
            \citep{wiki:smart} technika alkalmazása, hogy minél specifikusabb,
            teljesíthetőbb, tervezhetőbb és könnyen mérhető célokat tűzzenek
            ki.

      \item A kitűzött célokban való megállapodást követően, megállapodást
            kell
            kötni arról is, hogy ki és mit tesz azért, hogy a gyerek a célokat
            elérje.

      \item A mentor a teljes mikroiskolát (a többi tanárt, a közösséget)
            képviseli
            a
            megállapodás során.
\end{itemize}

A tanulási szerződést néha hívjuk \emph{megállapodásnak} is. A megállapodás és
szerződés szavakat ez a kerettanterv szinonimának tekinti. A \emph{learn\-ing
      con\-tract} az önirányított tanulást hangsúlyozó felnőttképzéssel
foglakozó
irodalomban
bevett szakkifejezés már a 80-as évektől \citep{Malcolm77}. Ennek a magyar
nyelvben inkább a szerződés felel meg. Egy másik szakterületen, a
pszichoterápiás munkában a terápiás szerződések megkötésekor a közös munka
kereteinek kialakítását és fenntarthatóságát hangsúlyozzák
\citep{pszichoterapia}. Erre is utalunk a tanulási szerződés elnevezéssel. Van,
amikor a \emph{hármas szerződés} kifejezést használjuk, hangsúlyozva, hogy mind
a három szereplőnek elfogadhatónak kell tartania a szerződés tartalmát.

\section{Visszajelzés, értékelés}
\label{sec:ertekeles}
Ahhoz, hogy hatékony legyen a tanulás, fejlődés, fontos, hogy a gyerekek,
tanárok és szülők is tudják, hogy
\begin{enumerate}
      \item hol tart most egy gyerek, mit tud most,
      \item hova akar vagy kell eljutni, azaz, mi a célja,
      \item mi kell ahhoz, hogy elérje a célját.
\end{enumerate}
Ezek mellett mindenkinek hinnie kell abban, hogy odafigyeléssel, gyakorlással a
gyerek meg tud tanulni egy konkrét dolgot. Fontos, hogy magas legyen a gyerekek
énhatékonysága,  erős legyen az önbizalmuk, és nem szabad félniük a hibázástól,
a nem-tudástól,
mert a tanulás első lépése, hogy elfogadjuk, hogy valamit nem tudunk. Azaz
fontos, hogy fejlődésfókuszú gondolkodásuk (growth mindset)
\citep{growthmindset} legyen, azaz
\begin{enumerate}
      \setcounter{enumi}{3}
      \item hinniük kell, hogy el tudják érni a céljukat.
\end{enumerate}

Egy visszajelzés, értékelés akkor jó és hasznos, azaz hatékony, ha ebben a négy
dologban segít. Mai tudásunk szerint ehhez:
\begin{itemize}
      \item Rendszeresen visszajelzést kell kapnunk és adnunk.
      \item A tanulási céloknak és visszajelzéseknek minél specifikusabbaknak
            kell
            lenniük (azaz például ne a 8. oszályos \emph{matematikatudást}
            értékeljük,
            hanem hogy mennyire képes valaki \emph{fagráfokat
                  használni
                  feladatmegoldások során}\footnote{Ez a konkrét példa a STEM
                  tantárgy
                  egyik
                  tanulási eredménye.}).
      \item A \emph{,,hol tartok most''} diagnózisnak mindig cselekvésre,
            viselkedésre, aktív tevékenységre kell vonatkoznia. Ne az legyen a
            visszajelzés, hogy \emph{,,ügyes vagy egyenletekből''}, hanem
            \emph{,,gyorsan és
                  pontosan oldottad meg a 4 egyenletet''}. A legjobb, amikor a
            visszajelzés
            konkrét megfigyelésen alapul, és tudni, hogy mikor, hol történt az
            eset:
            \emph{,,amikor társaiddal Minecraftban házat építettél, akkor
                  pontosan
                  kiszámoltad a ház területét''.}
      \item Ha a cél nem a mások legyőzése, akkor a visszajelzés se
            tartalmazzon
            olyan állítást, ami másokhoz hasonlít (így kerüljük a
            \emph{tehetség}
            szót is,
            aminek bevett definíciója szerint az átlagnál jobb képesség). A
            másokhoz való
            szint felmérése akkor (és csak akkor) fontos, amikor a cél egy
            versenyszituációban jó eredményt elérni.

      \item A gyerek legyen részese a visszajelzésnek. Értse, tudja, hogy miért
            kapta
            azt a visszajelzést, a legjobb, ha -- amikor ezt a képességei
            engedik
            -- önmaga
            képes elvégezni a visszajelzést, vagy annak egy részét.
      \item A visszajelzésnek transzparensen hatással kell lennie a
            tanulásszervezésre. Legyen része a folyamatnak, és a gyerek, tanár
            és a
            szülő
            is értse, hogy a visszajelzés alapján mit és hogyan csinálunk
            másképp.
\end{itemize}

\paragraph{Többszintű visszajelzés} A Budapest School iskolákban a gyerekek
többféle visszajelzést kapnak. \begin{enumerate}
      \item Minden modul elvégzése után a modul céljai, témája, fókusza alapján
            a
            modulvezetők visszajelzést adnak a tanulásról, eredményekről,
            viselkedésről.
      \item Trimeszterenként a mentorok visszajelzést adnak arról, hogy a
            gyerek
            általában hogyan haladt a tanulási célok felé.
      \item Ennek része, hogy a tantárgyi tanulási eredmények alapján hogyan
            haladt a
            gyerek a tantárgyak évfolyamszinthez tartozó követelmények
            teljesítésében. Az
            évfolyamok, mint elérhető szintek Budapest School értelmezését
            \aref{sec:evfolyamok}. fejezet tárgyalja.
      \item A mentorok irányításával a gyerekek visszajelzést kapnak arról,
            hogyan
            működnek a közösségben.
\end{enumerate}

\paragraph{Érdemjegyek, osztályzatok helyett értékelő táblázatok} A Budapest
School visszajelzéseinek sokkal részletesebbeknek kell lenniük, mint azt a
tantárgyi érdemjegyek és osztályzatok lehetővé teszik, ezért azok helyett a
kerettanterv
értékelő táblázatokat (angolul rubric) alkalmaz. Az értékelő táblázatban
szerepelnek az értékelés szempontjai és szempontonkénti szintek, rövid
leírásokkal.
Ezek alapján a gyerekek maguk is láthatják, hogy hol tartanak, hogyan
javíthatnak még a munkájukon. A táblázatok formája minden visszajelzés esetén
(értsd modulonként, célonként)
változtatható.

\section{Portfólió}
\label{sec:portfolio}
A modulok eredményeiből, a produktumokból és visszajelzésekből a gyerek és a
mentor portfóliót
állít össze, hogy a tanulás mintázatait észlelhesse, és a
tanárok tudatosabban tudják
a gyereket segíteni a céljai kitalálásában és elérésében. A portfólió a gyerek
céljainak nyomon követését szolgálja, és egyúttal a szülők felé történő
visszajelzés eszköze
is. Minden gyerek portfóliója folyamatosan épül: az tartalmazza az általa
elvégzett feladatokat, projekteket vagy azok dokumentációját, alkotásait,
eredményeit, az esetleges vizsgák eredményeit és a társaitól, tanáraitól kapott
visszajelzéseket. A \emph{portfólió célja}, hogy minden információ meglegyen
ahhoz,
hogy

\begin{itemize}
      \item a gyerek és mentora fel tudja mérni, hogy sikerült-e a kitűzött
            célokat
            elérni, illetve mire van szüksége még a gyereknek új célok
            eléréséhez;

      \item a szülő folyamatosan rálásson a gyereke tanulási útjára;

      \item megítélhető legyen, hogy a tantárgyi követelményekhez
            képest
            hol
            tart a gyerek;

      \item a gyerek a portfólió megtekintésével visszaemlékezhessen a
            tanultakra,
            ismételhessen, tudása elmélyülhessen;

      \item eredményei alapján bizonyítványt lehessen kiállítani.

\end{itemize}

A portfólió folyamatosan frissül, a mindennapi, formális, non-formális és
informális tanulási helyzetek
bármikor adhatnak okot a portfólió frissítésére. Az iskola életében kiemelt
szerepe van a következő eseményeknek.

\begin{enumerate}
      \item Minden \emph{modul végeztével} a portfólióba kerül:

            \begin{enumerate}

                  \item  A képesség elsajátításának, tanulási eredmény
                        elérésének a ténye.
                        Nincs
                        félig elsajátított képesség, tehát már értékelni nem
                        kell. Ha a modul során a gyerek megtanult százas
                        számkörben alapműveleteket
                        végezni, 
                        akkor
                        annyi kerül be a portfólióba, hogy ,,\emph{Szóban és
                              írásban
                              összead, kivon, szoroz és oszt a százas
                              számkörben.}''. Amennyiben a
                        készséget a
                        gyerek és a
                        tanár megítélése alapján nem sikerült megfelelően
                        elsajátítani,
                        úgy a gyakorlás
                        ténye kerül be a portfólióba.
                  \item Az alkotás vagy a projektmunka eredménye, ha a modul
                        célja egy
                        alkotás
                        létrehozása volt.
                  \item A részvétel ténye, ha a jelenlét volt a modul célja
                        (például
                        kirándulás
                        az Országos Kéktúra útvonalán).

            \end{enumerate}
      \item Az elvégzett vizsgák, tudáspróbák, képességfelmérők, diagnózisok
            eredményeit érdemes rögzíteni.

      \item A \emph{kipakolás} célja, hogy a gyerekek a tanároknak, szülőknek
            és
            más érintetteknek bemutassák elvégzett
            munkájukat, azaz
            a portfólióváltozásukat. A kipakolásra való felkészülés
            tulajdonképpen
            a
            portfólió összeállítása, prezentálásra való felkészítése, a
            \emph{portfólió
                  frissítése}.

      \item Társas visszajelzés eredményeként minden gyerek kap visszajelzést a
            társaitól. Ilyenkor összegyűjtik, mit tett a gyerek, ami a többiek
            elismerését
            és háláját kivívta. Ez is releváns adatokkal szolgálhat a
            portfólióhoz.

      \item A gyerek saját értékelése, reflexiója arról, hogyan értékeli, amit
            elért, fontos eleme a portfóliónak.

      \item A tanárok adhatnak kompetenciatanúsítványokat. Ezek
            rövid,
            specifikus visszajelzések, amelyek mutatják, ha valamit a gyerek
            megcsinált,
            valamiben fejlődött.
\end{enumerate}

A mentorok segítenek a gyerekeknek a tanulás módját, folyamatát és eredményeit
bemutatni
portfólióban.

\paragraph{Formai követelmények}
A portfóliónak rendezettnek, hozzáférhetőnek,
elérhetőnek, visszakereshetőnek és könnyen bővíthetőnek kell lennie. Olyan
(technológiai)
megoldást kell a mikroiskoláknak választaniuk, ami alapján
a gyerek, tanár és a szülő \emph{naponta} tudja a portfóliót bővíteni, és akár
\emph{heti rendszerességgel} át tudják tekinteni időrendben, modulonként vagy
tantárgyanként a portfólió bővülését.

A portfólió formátumára nincs egységes megkötés. Minden mikroiskola maga
alakítja ki a gyerekek, tanárok és szülők számára legjobban működő rendszert.
Évfolyamszintlépéshez és osztályzatokra váltáshoz az iskola csak digitális
formában tárolt és a kijelölt tanárok számára online elérhetővé tett portfóliót
fogad el.