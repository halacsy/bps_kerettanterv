\section{Saját tanulási célok}

Minden gyerek megfogalmazza és háromhavonta újrafogalmazza a \emph{saját
  tanulási céljait}: eredményeket, amelyeket el akar érni, képességeket,
amelyeket fejleszteni akar, szokásokat, amelyeket ki akar alakítani. A saját
célok elfogadásakor a tanuló és a mentora a szülőkkel együtt \emph{tanulási
  szerződést} köt. Csak olyan célok kerülhetnek a saját célok közé, amelyek
minden érintettnek biztonságosak, és amelyek összhangban vannak a tantárgyi
fejlesztési célokkal és eredményelvárásokkal.

Fontos megkötés, hogy a saját tanulási célok felét a 
\ifkerettanterv
\ref{sec:tantargyi_celok}. fejezetben 
\else
a kerettanterv Tantárgyi eredménycélok fejezetében
\fi

felsorolt tantárgyi tanulási célokból kell összeválogatni. A másik
fele szabadon alakítható. Az iskola tanárai a közösségek igényei, céljai
alapján alakítják ki, hogy pontosan mi történik az iskolákban: mikor és milyen
modulok vannak, hogyan szervezik a mindennapokat.

Háromhavonta a tanárok és a tanulók megállnak, reflektálnak az elmúlt
időszakra, és a tapasztalatok, valamint az elért célok ismeretében és az új
célok figyelembevételével újratervezik, újraszervezik a foglalkozások rendjét,
tehát azt, hogy mikor és mit csinálnak majd a gyerekek az iskolában.
A mindennapi tevékenység során tapasztalt élmények, alkotások, elvégzett
feladatok, kitöltött vizsgák, tehát mindaz, ami a gyerekekkel történik, bekerül
a portfóliójukba. Még az is, amit nem terveztünk előre.

A gyerekeket a mentoruk segíti a saját célok kitűzésében, a különböző
választásoknál, a portfólióépítésben, a reflektálásban. A tanulási célok
kitűzése az önirányított tanulás fokozatos fejlődésével és az életkor
előrehaladtával folyamatosan egyre önállóbb tevékenységgé válik. Tanulási
útján, céljai kitűzésében a mentor kíséri végig a gyerekeket.

A Budapest School személyre szabott tanulásszervezésének jellegzetessége, hogy
a tanulók a saját egyéni céljuk irányába haladnak, az adott célhoz az adott
kontextusban leghatékonyabb úton. Tehát mindenki rendelkezik egyéni célokkal,
még akkor is, ha egy közösség tagjainak céljai a tantárgyi eredménycélok
azonossága, vagy a hasonló érdeklődés miatt akár  80\% átfedést mutatnak.

A NAT műveltségi területeiben megfogalmazott követelmények teljesítése is célja
a tanulásnak, a tanulás fő irányítója azonban más. Mi azt kérdezzük a
tanulóktól, hogy \emph{ezen felül} mi az ő személyes céljuk.

\section{A tanulási szerződés}

A tanulási szerződés az előbbiekben említett gyerek-mentor-szülő közötti
megállapodás, ami rögzíti
\begin{enumerate}
  \item a gyerek, a mentor (iskola) és a szülő igényeit, elvárásait;

        ezek lehetnek: \emph{,,szeretném, ha a gyerekem naponta olvasna''} típusú
        folyamatra vonatkozó kérések, vagy erősebb \emph{,,változtatnod kell a
          viselkedéseden, ha a közösségben akarsz maradni''} igények, határok
        megfogalmazása;

  \item a gyerek céljait a következő trimeszterre, vagy a tanév végéig;

  \item a gyerek, mentorok (iskola) és szülő vállalásait, amivel támogatják a cél
        elérését és a felek igényének elérését.

\end{enumerate}

A tanulási szerződésre jellemző, hogy
\begin{itemize}
  \item A kitűzött célokat minél specifikusabban, mérhetőbben kell megfogalmazni.
        Javasolt az OKR,  (Objectives and Key Results, azaz  Cél és Kulcs Eredmények)
        \citep{okr} vagy a SMART (Specific, Measurable, Achievable, Relevant,
        Time-bound, azaz Specifikus,  Mérhető, Elérhető, Releváns és Időhöz kötött)
        \citep{wiki:smart} technika alkalmazása, hogy minél specifikusabb,
        teljesíthetőbb, tervezhetőbb és könnyen mérhető célokat tűzzenek ki.

  \item A kitűzött célokban való megállapodást követően, megállapodást  kell
        kötni arról is, hogy ki és mit tesz azért, hogy a tanuló a célokat elérje.

  \item A mentor a teljes mikroiskolát (a többi tanárt, a közösséget) képviseli a
        megállapodás során.
\end{itemize}

A tanulási szerződést néha hívjuk \emph{megállapodásnak} is. A megállapodás és
szerződés szavakat ez a kerettanterv szinonímának tekinti. A \emph{learn\-ing
  con\-tract} az önirányított tanulást hangsúlyozó felnőttképzés irodalomban
bevett szakkifejezés már a 80-as évektől \citep{Malcolm77}. Ennek a magyar
nyelvben inkább a szerződés felel meg. Egy másik szakterületen, a
pszichoterápiás munkában a terápiás szerződések megkötésekor a közös munka
kereteinek kialakítását és fenntarthatóságát hangsúlyozzák.
\citep{pszichoterapia} Erre is utalunk a tanulási szerződés elnevezéssel. Van,
amikor a \emph{hármas szerződés} kifejezést használjuk, hangsúlyozva, hogy mind
a három szereplőnek elfogadhatónak kell tartania a szerződés tartalmát.

Tekintettel arra, hogy a kiskorú mellett minden esetben a szülő (törvényes
képviselő) is aláírója a szerződésnek, a Ptk. 2:14 .§ (1) és (3) bekezdése
szerint a jognyilatkozat érvényesen megtehető, a szerződés érvényesen létre
jöhet.

\section{Moduláris tanmenet}
\label{sec:modularis_tanmenet}

Az iskola fő célja, hogy képes legyen alkalmazkodni a menet közben felmerülő
tanulási igényekhez, és a tanulók saját céljai felé vezető egyéni tanulási
útjait minél rugalmasabban meg tudja szervezni. Ennek támogatására a tanulást
modulokra, rövid tanulási egységekre bontjuk.

A modulok a tanulásszervezés egységei: olyan foglalkozások megtervezett
sorozata, amelyek során egy meghatározott időn belül a gyerekek valamely
képességüket fejlesztik, valamilyen ismeretet elsajátítanak, vagy valamilyen
produktumot létrehoznak. A modulok célja sokféle lehet, de a legfontosabb, hogy
a résztvevők a portfóliójukba bejegyzésre érdemes eredményt hozzanak létre.
Modulokba szerveződve az iskola tanulói tudnak például:

\begin{itemize}
  \item Produktum létrehozására szerveződő projektben részt venni
  \item Felfedezni, feltalálni, kutatni, vizsgálni, azaz kérdésekre választ
        keresni
  \item Egy jelenséget több nézőpontból megismerni
  \item Valamely képességüket, készségüket fejleszteni
  \item Adott vizsgára gyakorló feladatokkal felkészülni
  \item Közösségi programokban részt venni
  \item Az önismeretükkel, a tudatosságukkal, a testi-lelki jóllétükkel
        foglalkozni.
\end{itemize}

A modulokból álló rendszert azért kínáljuk a gyerekeknek, hogy egyszerre adjunk
átlátható struktúrát a tanulásnak, és kellő rugalmasságot is biztosítsunk, hogy
az egyéni, a közösségi és a társadalmi célok harmóniába kerülhessenek.

\paragraph{A modulokat a tanulásszervezők állítják össze.}

A modulok kiválasztása, felkínálása az iskola tanulásszervező tanárainak
felelőssége, hiszen ők figyelnek és reagálnak a tanulók, szülők céljaira és
igényeire. Kidolgozáshoz és a modulok tartásához külsős embereket hívhatnak
meg, de ebben az esetben is a tanulásszervező tanárok felelőssége, hogy a
modulok és a modulvezetők mit és hogyan csinálnak.

A tanulásszervezők az egyes modulok tematikáját, azok hosszát és feladatát a
gyerekek tanulási céljainak megismerését követően és a kerettantervben
meghatárzott tantárgyi eredménycélokat figyelembe véve határozzák meg. A
meghirdetett modulokból áll össze a tanulás trimeszterenkénti tanulási rendje.
A modulokba való csatlakozásról a mentor, a szülő és a gyerek közösen dönt,
mindig szem előtt tartva, hogy folyamatos előrelépés legyen a már elért egyéni
és tantárgyi eredménycélokban is. Egy modul megkezdésének lehet feltétele egy
korábbi modul elvégzése, a gyerek képességszintje, a jelentkezők száma, és
lehet egyedüli feltétele a gyerekek érdeklődése. Egy modulvezető különféle
tematikájú modulokat tarthat függően attól, hogy az egyéni célok, a tantárgyi
eredménycélok mit kívánnak, és a tanulásszervezők, valamint a modulvezetők
kapacitása mit enged. Egy modul kapcsolódhat

\begin{itemize}

  \item valamilyen tantárgyi eredménycél eléréséhez;
  \item egy tantárgyi céloktól független önálló tanulási célhoz;
  \item vagy tantárgyi céloktól független csoportos tanulási célhoz.
\end{itemize}

A Budapest School tanulói a felkínált modulokból maguk választják napi- és
hetirendjük tartalmát, a mentoraik és szüleik segítségével. Az a
tanulásszervező tanárok felelőssége, hogy amikor egy tanuló moduljai
befejeződnek, és újat tud felvenni, akkor az érdeklődési körének, tanulási
céljainak megfelelő modulok közül választhasson.

\paragraph{A modulok formátuma változatos.}
Egy-egy modul hossza és foglalkozásainak gyakorisága változó lehet: az egyszeri
alkalomtól a teljes trimeszteren át tartó, heti 3-5 foglalkozást magában
foglaló modul is lehetséges. Egy trimeszternél azonban nem lehet hosszabb, és a
lezárását az értékelés és az elért eredmények portfólióba emelése követi. Egy
modul folytatásaként a következő trimeszterben új modult lehet meghirdetni.

A modulok nemcsak témájukban, céljaikban, időtartamukban, hanem
módszertanukban, folyamataikban is különbözhetnek: bizonyos modulokban a
felfedeztető (inquiry based) módszer, másokban az ismétlő (repetitív) gyakorlás
a célravezető. Így mindig a modul céljához, a tanárok és a tanulók
képességeihez és igényeihez választható a legjobb módszer. Modulonként
változhat, hogy a folyamatot a tanulók vagy a tanárok befolyásolják-e, és
milyen mértékben. Két szélsőség:
\begin{enumerate}
  \item Egy digitális kézműves modul célja, hogy építsünk valamit, ami
        programozható. Annak kitalálása, hogy mit és hogyan építünk, a tanulók
        feladata. Itt a modul vezetője csak támogatja a tanulás folyamatát, azaz
        \emph{facilitál}.

  \item Egy „A vizuális kommunikáció fejlődése a XX. század második felében"
        modul esetén a tanár előre eldönti, hogy mely alkotók, alkotások tartoznak
        szerinte a mindenképpen említendők sorába (a kánonba), és ezeket a tanulókkal
        sorban végigveszi.
\end{enumerate}

\paragraph {Modulok helyszíne}

A tanulás az egyes mikroiskolák helyszínén, vagy más Budapest School
mikroiskolákban, vagy a tanár által kiválasztott külső helyszínen, esetenként
pedig online, virtuális térben történik. A tanulásra, mint az élethez szorosan
kapcsolódó holisztikus fejlődési igényre tekintünk, melynek jegyében az
elsődleges szocializációs formától, a szülői, családi környezettől sem akarjuk
a tanulást leválasztani. Az élethosszig tartó tanulás jegyében a tanulás tere
az iskolai időszak után és az iskola terein kívül is folytatódik.

A tanulók több ok miatt is tanulnak az iskolán kívül:
\begin{enumerate}
  \item Modulok szervezhetők külső helyszínekre, úgymint múzeumokba, erdei
        iskolákba, parkokba, vállalatokhoz, vagy tölthetik az idejüket „kint a
        társadalomban".

  \item Az önirányított tanulás okán a gyerekek otthon vagy külső helyszíneken is
        elvégezhetnek egy modult, amennyiben ez egyéni céljuk elérését nem
        veszélyezteti, és mentoruknak tudomása van arról, hogy a folyamatos fejlődés
        biztosított.
\end{enumerate}

Minden modul végeztével a tanulók és modulvezetők visszajelzést adnak
egymásnak, aminek része, hogy megosztják saját élményeiket, reflektálnak a
közös időre, összegyűjtik és értékelik az elért eredményt, és kitérnek az
esetleges fejlődési lehetőségekre.

\section{Visszajelzés, értékelés}
\label{sec:ertekeles}
Ahhoz, hogy hatékony legyen a tanulás, fejlődés, érdemes a gyerekeknek,
tanároknak és szülőknek tudnia, hogy
\begin{enumerate}
  \item hol tart most egy gyerek, mit tud most,
  \item hova akar vagy kell eljutni, azaz, mi a célja,
  \item mi kell ahhoz, hogy elérje a célját.
\end{enumerate}
Ezek mellett mindenkinek hinnie kell abban, hogy odafigyeléssel, gyakorlással a
gyerek meg tud tanulni egy konkrét dolgot. Fontos, hogy magas legyen a gyerekek
énhatékonysága, önbizalmuk, és nem szabad félniük a hibázástól, a nem-tudástól,
mert a tanulás első lépése, hogy elfogadjuk, hogy valamit nem tudunk. Azaz
fontos, hogy fejlődésfókuszú gondolkodásuk (growth mindset)
\citep{growthmindset} legyen, azaz
\begin{enumerate}
  \setcounter{enumi}{3}
  \item hinniük kell, hogy el tudják érni a céljukat.
\end{enumerate}

Egy visszajelzés, értékelés akkor jó és hasznos, azaz hatékony, ha ebben a négy
dologban segít. Mai tudásunk szerint ehhez:
\begin{itemize}
  \item Rendszeresen visszajelzést kell kapnunk és adnunk.
  \item A tanulási céloknak és visszajelzéseknek minél specifikusabbaknak kell
        lenniük (azaz például ne a 8. oszályos \emph{matematikatudást} értékeljük,
        hanem például, hogy mennyire képes valaki \emph{fagráfok használatára
          feladatmegoldások során}\footnote{Ez a konkrét példa a STEM tantárgy egyik
          eredménycélja}).
  \item A \emph{,,hol tartok most''} diagnózisnak mindig cselekvésre,
        viselkedésre, aktív tevékenységre kell vonatkoznia. Ne az legyen a
        visszajelzés, hogy \emph{,,ügyes vagy egyenletekből"}, hanem \emph{,,gyorsan és
          pontosan oldottad meg a 4 egyenletet"}. A legjobb, amikor a visszajelzés
        konkrét megfigyelésen alapul, és tudni, hogy mikor, hol történt az eset:
        \emph{,,amikor társaiddal Minecraftban házat építettél, akkor pontosan
          kiszámoltad a ház területét.}
  \item Ha a cél nem a mások legyőzése, akkor a visszajelzés se tartalmazzon
        olyan állítást, ami másokhoz hasonlít (így kerüljük a \emph{tehetség} szót is,
        aminek bevett definíciója szerint az átlagnál jobb képességű). A másokhoz való
        szint felmérése akkor (és csak akkor) fontos, amikor a cél egy
        versenyszituációban jó eredményt elérni.

  \item A tanuló legyen részese a visszajelzésnek. Értse, tudja, hogy miért kapta
        azt a visszajelzést, a legjobb, ha -- amikor ezt a képességei engedik -- önmaga
        képes elvégezni a visszajelzést, vagy annak egy részét.
  \item A visszajelzésnek transzparensen hatással kell lennie a
        tanulásszervezésre. Legyen része a folyamatnak, és a gyerek, tanár és a szülő
        is értse, hogy a visszajelzés alapján mit és hogyan csinálunk másképp.
\end{itemize}

\paragraph{Többszintű visszajelzés} A Budapest School iskolákban a gyerekek
többféle visszajelzést kapnak. \begin{enumerate}
  \item Minden modul elvégzése után a modul céljai, témája, fókusza alapján a
        modulvezetők visszajelzést adnak a tanulásról, eredményekről, viselkedésről.
  \item Trimeszterenként a mentorok visszajelzést adnak arról, hogy a tanuló
        általában hogyan haladt a tanulási célok felé.
  \item Ennek része, hogy a tantárgyi eredménycélok alapján hogyan haladt a
        tanuló a tantárgyak évfolyam-szinthez tartozó követelmények teljesítésében. Az
        évfolyamok, mint elérhető szintek Budapest School értelmezését
        \aref{sec:evfolyamok}. fejezet tárgyalja.
  \item A mentorok irányításával a tanulók visszajelzést kapnak arról, hogyan
        működnek a közösségben.
\end{enumerate}

\paragraph{Érdemjegyek, osztályzatok helyett értékelőtáblázatok} A Budapest
School visszajelzéseinek sokkal részletesebbeknek kell lenniük, mint azt a
tantárgyi érdemjegyek és osztályzatok lehetővé teszik, ezért azok helyett mi
értékelőtáblázatokat (angolul rubric) alkalmazunk. Az értékelőtáblázatban
szerepelnek az értékelés szempontjai és azok szintjei is, rövid leírásokkal.
Ezek alapján a gyerekek maguk is láthatják, hogy hol tartanak, hogyan
javíthatnak még a munkájukon. A táblázatok formája minden visszajelzés esetén
változtatható (értsd modulonként, célonként).

\section{Portfólió}

A modulok eredményeiből, a produktumokból és visszajelzésekből portfóliót
állítunk össze, hogy a tanulás mintázatait észlelhessük, és tudatosabban tudjuk
a gyereket segíteni a céljai kitalálásában és elérésében. A portfólió a gyerek
céljainak nyomonkövetését szolgálja, és egyúttal a szülői visszajelzés eszköze
is. Minden tanulónak folyamatosan épül a portfóliója: ez tartalmazza az általa
elvégzett feladatokat, projekteket vagy azok dokumentációját, alkotásait,
eredményeit, az esetleges vizsgák eredményeit és a visszajelzéseket társaitól,
tanáraitól. A \emph{portfólió célja}, hogy minden információ meglegyen ahhoz,
hogy

\begin{itemize}
  \item a tanuló és mentora fel tudja mérni, hogy sikerült-e a kitűzött célokat
        elérni, illetve mire van szüksége még a tanulónak új célok eléréséhez;

  \item a szülő folyamatosan rálásson a gyereke tanulási útjára;

  \item megítélhető legyen, hogy a tantárgyi fejlesztési területekhez képest hol
        tart a tanuló;

  \item a gyerek a portfólió megtekintésével visszaemlékezhessen a tanultakra,
        ismételhessen, tudása elmélyülhessen;

  \item eredményei alapján bizonyítványt lehessen kiállítani.

\end{itemize}
A portfólió alakításának eseményei
\begin{enumerate}
  \item Minden \emph{modul végeztével} a portfólióba kerül:

        \begin{enumerate}

          \item  A készség elsajátításának ténye (mastery based learning, nálunk nincs
                félig elsajátított képesség), ha a modul célja például a 20-as számkörben való
                biztonságos összeadás és kivonás volt. Amennyiben a készséget a gyerek és a
                tanár megítélése alapján nem sikerült megfelelően elsajátítani, úgy a gyakorlás
                ténye kerül be a portfólióba.
          \item Az alkotás vagy a projektmunka eredménye, ha a modul célja egy alkotás
                létrehozása volt.
          \item A részvétel ténye, ha a jelenlét volt a modul célja (például kirándulás
                az Országos Kéktúra útvonalán).

        \end{enumerate}
  \item Az elvégzett vizsgák, tudáspróbák, képességfelmérők, diagnózisok
        eredményei.

  \item A kipakolás, amelynek célja, hogy a tanulók a tanároknak, szülőknek és
        más érintetteknek bemutassák a két kipakolás között elvégzett munkájukat, azaz
        a portfólió változásukat. A kipakolásra való felkészülés tulajdonképpen a
        portfólió összeállítása, prezentálásra való felkészítése, a \emph{portfólió
          frissítése}.

  \item Társas visszajelzés eredményeként minden tanuló kap visszajelzést a
        társaitól. Ilyenkor összegyűjtik, mit tett a tanuló, ami a többiek elismerését
        és háláját kivívta.

  \item A tanuló saját értékelése, reflexiója arról, hogyan értékeli, amit elért.

  \item A tanárok folyamatosan adnak kompetencia tanúsítványokat. Ezek rövid,
        specifikus visszajelzések, amelyek mutatják, ha valamit a tanuló megcsinált,
        valamiben fejlődött.
\end{enumerate}

A mentorok segítenek a tanulás módját, folyamatát és eredményeit bemutatni
portfólióban.

\paragraph{Formai követelmények}
A portfólió formátumára nincs egységes megkötés. Minden mikroiskola maga
alakítja ki a gyerekek, tanárok és szülők számára legjobban működő rendszert.
Így arról is ők döntenek, hogy digitális vagy analóg formátumban tárolják a
portfóliót. Azonban fontos, hogy a portfóliónak rendezettnek, hozzáférhetőnek,
elérhetőnek, visszakereshetőnek és könnyen bővíthetőnek kell lennie. Olyan
megoldást kell a mikroiskoláknak választaniuk, ami alapján
a gyerek, tanár és a szülő \emph{naponta} tudja a portfóliót bővíteni, és akár
\emph{heti rendszerességgel} át tudják tekinteni időrendben, modulonként vagy
tantárgyanként a portfólió bővülését.
