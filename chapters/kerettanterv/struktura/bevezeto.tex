A Budapest School Általános Iskola és Gimnázium egy több helyszínen működő tanulási hálózat, amelynek célja, hogy otthonos környezetben, rugalmas, mégis jól szabályozott keretek között integrálja a NAT műveltségi területeit, fejlesztési céljait és kulcskompetenciáit a gyerekek saját tanulási céljaival. A különböző helyszínek, a Budapest School \emph{mikroiskolái}, legfeljebb hat évfolyamot átölelő összevont osztályokként 6-60 fős tanulási közösségekként működnek (ld. \ref{sec:mikroiskola}. fejezet), ahol a gyerekek többfajta csoportbontásban tanulnak attól függően, hogy a Budapest School kerettantervében megfogalmazottaknak megfelelően milyen területeken kell, és milyen területeken akarnak fejlődni.

Az egyes mikroiskolákat \emph{tanulásszervező} tanárok (tanulásszevezők) csapata vezeti. Minden gyereknek van egy kitüntetett tanára, a \emph{mentora}, aki egyéni figyelmével a fejlődésben segíti (ld. \ref{sec:tanarok}. fejezet). Minden gyerek a mentortanára segítségével és a szülők aktív részvételével trimeszterenként meghatározza a \emph{saját tanulási céljait}
(ld.
\ref{sec:tanulasi_celok}).

A tanulásszervezők \emph{modulokat} hirdetnek ezen célokból és a kerettanterv tantárgyainak tartalmából. A modulok reflektálnak a mai világ alapvető kérdéseire, integrálják	a tudományterületeket és művészeti ágakat (tantárgyakat), és egyenlő lehetőséget adnak a tudásszerzésre, az önálló gondolkodásre és az alkotásra a gyerekek mindennapjaiban (ld. \ref{sec:modulok}. fejezet).

A kerettanterv három tantárgyat határoz meg, azok témaköreit, tartalmát és követelményeit \emph{tanulási eredmények} listájaként adja meg (ld. \ref{sec:tanulasi_eredmenyek}). A gyerekek feladata az iskolában, hogy tanulási eredményeket érjenek el. Modulok elvégzésével tanulási eredményeket lehet elérni.

A modulok végeztével a gyerekek eredményei bekerülnek saját portfóliójukba (ld. \ref{sec:portfolio}. fejezet), melyek tartalmazhatnak önálló vagy csoportos alkotásokat, tudáspróbákat, vizsgafeladatokat, egymás felé történő visszajelzéseket, a fejlődést jól mérő dokumentációkat vagy bármit amire a gyerek és tanárai büszkék vagy fontosnak tartanak. Erre a portfólióra épül a Budapest School visszajelző és értékelő rendszere (ld. \ref{sec:ertekeles}). A portfólió alapján ismeri el az iskola az évfolyamok teljesítését (ld. \ref{sec:evfolyamszintlepes}. fejezet). Szükség esetén a portfólió alapján kaphatnak a gyerekek osztályzatokat is (ld. \ref{sec:osztalyzatok}. fejezet).

Az egyes tanulási modulok során elért eredmények a Budapest School három tantárgyában való fejlődést segítik úgy, hogy egyszerre jelennek meg benne a miniszter által kiadott kerettantervek tantárgyi elvárásai, a gyerekek saját céljai és a mai világra való integrált reflexió. Sajátként megélt tanulási útjuk során a gyerekek így egyaránt fejlődnek a világ tudományos megértésében, leírásában és újraírásában (STEM tantárgy), az emberiség művészeti, történeti és szociokulturális megismerésében, és az azt segítő kommunikáció\-ban (KULT tantárgy), valamint az önmagukhoz és környezetükhöz való kapcsolódásban (Harmónia tantárgy) (ld. \ref{sec:tantargyak}).