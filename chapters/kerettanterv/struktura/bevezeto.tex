\section{A tanulás rendszere és folyamata}

A Budapest School Általános Iskola és Gimnázium több helyszínen működő iskola, amelynek célja, hogy otthonos környezetben, rugalmas, mégis jól szabályozott keretek között integrálja a NAT műveltségi területeit, fejlesztési céljait, kulcskompetenciáit és a miniszter által kiadott kerettantervek tantárgyait a gyerekek saját tanulási céljaival. A különböző helyszínek, a Budapest School \emph{mikroiskolái}, legfeljebb hat évfolyamot átölelő összevont osztályokként, 6--60 fős tanulási közösségekként működnek, ahol a gyerekek többfajta csoportbontásban tanulnak.

Az egyes mikroiskolákat \emph{tanulásszervező} tanárok (tanulásszervezők) csapata vezeti. Minden gyereknek van egy kitüntetett tanára, a \emph{mentora}, aki egyéni figyelmével a fejlődésben segíti. Minden gyerek a mentortanára segítségével és a szülők aktív részvételével trimeszterenként meghatározza a \emph{saját tanulási céljait}.

A tanulásszervezők \emph{modulokat} hirdetnek ezen célokból és a kerettanterv tantárgyainak tartalmából. A modulok reflektálnak a mai világ alapvető kérdéseire, integrálják	a tudományterületeket és mű\-vé\-sze\-ti á\-ga\-kat, azaz a tantárgyakat, és egyenlő lehetőséget adnak a tudásszerzésre, az önálló gondolkodásra és az alkotásra a gyerekek mindennapjaiban.

A modulok végeztével a gyerekek eredményei bekerülnek saját portfóliójukba, melyek tartalmazhatnak önálló vagy csoportos alkotásokat, tudáspróbákat, vizsgafeladatokat, egymás felé történő visszajelzéseket, a fejlődést jól mérő dokumentációkat vagy bármit, amire a gyerek és tanárai büszkék vagy amit fontosnak tartanak. Erre a portfólióra épül a Budapest School visszajelző és értékelő rendszere. A portfólió alapján ismeri el az iskola az évfolyamok teljesítését. 

A gyerekek mindennapjait meghatározó modulok több műveltségi területet, többféle kompetenciát, több tantárgy anyagát is lefedhetik, és egy tantárgy anyagát több modul is érintheti.
Ezért is mondhatjuk, hogy a Budapest School iskolákban a tantárgyközi tevékenységek vannak előtérben. 
A kerettanterv szándéka, hogy a gyerekek folyamatosan fejlődjenek a világ tudományos megismerésében (STEM), a saját és mások kulturális közegéhez való kapcsolódásban (KULT), valamint a testi-lelki egyensúlyuk fenntartásában (Harmónia), vagyis a \emph{kiemelt tantárgyközi fejlesztési irányelvekben}. 

A modulok fejlesztési irányelvei útmutatóul szolgálnak a tanároknak arra, hogy az elérendő eredményekhez milyen elvek mentén szervezzenek modulokat.

A tanárok döntése, hogy a gyerekek kémia órán kísérleteznek, vagy kísérletezés órán foglalkoznak kémiával. A a miniszter által kiadott kerettanterv annyit határoz meg, hogy  a 7--10. évfolyamszinten kémia tantárgyhoz kapcsolódóan 17 különböző tanulási eredményt kell elérni, és kísérletezéssel kapcsolatban pedig 15 különböző tanulási eredményt több különböző tantárgyból (ezekből csak 3 kapcsolódik a kémia tantárgyhoz).

Tehát a tantárgyak a tanulás tartalmi elemeinek forrása és keretei: a tanulandó dolgok halmazaként működik. Az, hogy milyen csoportosításban történik a tanulás, az a modulvezetőkre van bízva. A gyerekek lehet, hogy csak félévente, az elszámolás időszakában találkoznak a tantárgyak taxonómiájával. Ebben az időszakban veti össze minden gyerek és mentor, hogy amit tanultak, alkottak és amiben fejlődtek, az hogyan viszonyul a társadalom és a törvények elvárásaival, a Nemzeti Alaptantervvel és a kerettantervvel.

A Budapest School modell a tantárgyak témaköreit, tartalmát és követelményeit \emph{tanulási eredmények} halmazaként kigyűjtötte a gyerekek, a tanárok és szülők számára a miniszter által kiadott kerettantervekből, hogy azzel is segítse munkájukat. A tanulási eredmények tisztán kommunikálják minden résztvevő felé a kerettantervek elvárásait, átláthatóbbá teszik az értékelést. 

A gyerekek feladata az iskolában, hogy tanulási eredményeket érjenek el és így sajátítsák el a tantárgyak által szabott követelményeket. Tanulási eredményeket modulok elvégzésével (is) lehet elérni, tehát a modulok elsődleges feladata, hogy a tanulási eredményekhez vezető utat mutassák.

Az iskolában egyszerre jelennek meg a miniszter által kiadott kerettantervek tantárgyi elvárásai, a legújabb NKT módosítás szándékai, a gyerekek saját céljai és a mai világra való integrált reflexió. 

\subsection{A Budapest School modell nyelvezete}
Az iskolában történteket elég egyszerűen le lehet fordítani sokak számára ismertebb nyelvezetre. A tanulásszervezők a pedagógusok, sokszor olyan feladat látnak el, mint az osztályfőnökök. A modulok megfelelnek a közoktatás tanáóráinak. A modulvezetők az óraadó tanárok. A portfólió a naplónak egy gyerekre vonatkozó adata. Az összes gyerek portfóliója kiadja az osztálynaplót. 