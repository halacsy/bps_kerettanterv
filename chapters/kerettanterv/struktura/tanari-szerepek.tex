\section[A tanulásszervező, a mentor és a
modulvezető]{Különböző tanári szerepek: a tanulásszervező, a mentor és a
    modulvezető}
\label{sec:tanarok}

A Budapest Schoolban a gyerekek azokat a felnőtteket tekintik tanáruknak, akik
minőségi időt töltenek velük, és segítik, támogatják vagy vezetik őket a
tanulásukban. Több szerepre bontjuk a tanár fogalmát: a gyerek egy (és csak
egy) felnőtthöz különösen kapcsolódik, a \emph{mentortanárához}, aki rá
különösen figyel. Ezenkívül a gyerek tudja, hogy a mikroiskola mindennapjait
egy tanárcsapat, a \emph{tanulásszervezők} határozzák meg, ők vezetik az
iskolát.  A foglalkozásokon megjelenhetnek további tanárok, a \emph{modulvezetők},
akik egy adott foglalkozást, szakkört, órát tartanak.

Szervezetileg minden mikroiskolának van egy állandó \emph{tanárcsapata}, a
tanulásszervezők. A tanulásszervezők legalább egy tanévre elköteleződnek, szemben a
modulvezetőkkel, akik lehet, hogy csak egy pár hetes projekt keretében vesznek részt a
munkában.
A tanulásszervezők általában mentorok is, de nem minden esetben. Nem lehet
mentor az, aki a gyerek mikroiskolájában nem tanulásszervező, mert nem lenne
rálátása a mikroiskola történéseire. Egy tanulásszervező lehet több
mikroiskolában is ebben a szerepben, és így mentor is lehet több mikroiskolában.

\paragraph{Mentor}
Minden gyereknek van egy \emph{mentora}, aki a saját céljainak
megfogalmazásában és
a fejlődése követésében segíti. Minden mentorhoz több gyerek tartozik, de nem
több, mint 12. A mentor együtt dolgozik a mikroiskola többi tanulásszervezőjével, a
szülőkkel és az általa mentorált gyerekekkel. A mentor segít az általa
mentorált gyereknek, hogy a tantárgyi fejlesztési célok és
a
saját magának megfogalmazott saját célok között megtalálja  az egyensúlyt, és segít megalkotni a
gyerek \emph{saját
  tanulási tervét}.

A mentor a kapocs a Budapest School, a szülő és a gyerek között.

\begin{itemize}
  \item Képviseli a Budapest Schoolt, a mikroiskola közösségét.
        \begin{itemize}
          \item Ismeri a Budapest Schoolt, a lehetőségeket, a tanulásszervezés
                folyamatait.
          \item Együtt tanul más Budapest School-mentorokkal, együtt dolgozik a
                tanártársaival.
        \end{itemize}

  \item Ismeri, segíti, képviseli a gyereket.
        \begin{itemize}
          \item  Tudja, hol és merre tart a mentoráltja, ismeri a képességeit,
                körülményeit, szándékait, vágyait.
          \item    Segít a saját célok elérésében, felügyeli a haladást.
          \item    Megerősíti a mentoráltjai pszichológiai biztonságérzetét.
          \item   Visszajelzéseket ad a mentoráltjainak.
          \item    Segít abban, hogy az elért célok a portfólióba kerüljenek.
          \item    Összeveti a portfólió tartalmát a tantárgyak fejlesztési
                céljaival.
        \end{itemize}

  \item Együtt dolgozik, gondolkozik a szülőkkel, képviseli igényüket a
        közösség felé.
        \begin{itemize}
          \item Erős partneri kapcsolatot épít ki a szülőkkel, információt oszt meg
                velük.
          \item Segít a gyerekekkel közös célokat állítani.
          \item A szülő számára a mentor az elsődleges kapcsolattartó a különféle
                iskolai ügyekkel kapcsolatban.
        \end{itemize}

\end{itemize}

A mentor egyszerre felelős a mentorált gyerek előrehaladásának segítéséért,
és
közös felelőssége van a mentortársakkal, hogy az iskolában a lehető legtöbbet
tanuljanak a gyerekek. A mentor folyamatosan figyelemmel követi az egyéni
tanulási tervben megfogalmazottakat, és ezzel kapcsolatos visszajelzést ad a
mentoráltnak és a szülőnek.

\paragraph{Tanulásszervező}
Csoportban dolgozó, iskolaszervező, strukturáló tanár. Egy mikroiskola
állandó tanári
csapatát 2--7 tanulásszervező alkotja, akik egyedileg meghatározott szerepek
mentén a mikroiskola mindennapjainak működtetéséért felelnek. Minden mentor
tanulásszervező is. A tanulásszervezők tarthatnak
modulokat, sőt kívánatos is, hogy dolgozzanak a gyerekekkel, ne csak
szervezzék az életüket.
Ők rendelik meg a külső modulvezetőktől a munkát, ilyen értelemben a
tanulási utak projektmenedzserei.

\paragraph{Modulvezetők}

Bárki lehet modulvezető, aki képes akár egy egyetlen alkalommal történő, vagy
éppen
egy egész trimeszteren át tartó tanulási, alkotási folyamatot vezetni. Ők
általában
az adott tudományos, művészeti, nyelvi vagy bármilyen más terület szakértői.

A modulokat a tanulásszervezők is vezethetik, de külsős, egyedi megbízással
dolgozó szakemberek is megjelennek modulvezetőként. Modulvezető lehet bárki,
akiről az őt megbízó tanárcsapat tudja, hogy képes gyerekek folyamatos
fejlődését és egy tanulási cél felé való haladását segíteni. A moduláris
tanmenettel \aref{sec:modulok}. fejezet foglalkozik.