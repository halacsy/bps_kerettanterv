A Budapest School Általános Iskola és Gimnázium egy több helyszínen működő
tanulási hálózat, amelynek célja, hogy otthonos környezetben, rugalmas, mégis
jól szabályozott keretek között integrálja a NAT
műveltségi
területeit, fejlesztési céljait és kulcskompetenciáit a gyerekek saját tanulási céljaival. A különböző helyszínek, a Budapest
School \emph{mikroiskolái}, legfeljebb hat évfolyamot átölelő összevont
osztályokként
6-60 fős tanulási közösségekként működnek (ld. \ref{sec:mikroiskola}. fejezet),
ahol a gyerekek többfajta
csoportbontásban tanulnak attól függően, hogy a Budapest School
kerettantervében megfogalmazottaknak megfelelően milyen területeken kell, és
milyen területeken akarnak fejlődni.

Az egyes mikroiskolákat \emph{tanulásszervező} tanárok (tanulásszevezők)
csapata
vezeti.
Minden gyereknek van egy kitűntetett tanára, a \emph{mentora},
aki egyéni figyelmével a fejlődésben segíti.
(ld. \ref{sec:tanarok}. fejezet)
Minden gyerek mentortanára segítségével és a szülőkkel való aktív
részvétellel trimeszterenként meghatározza a \emph{saját tanulási céljait} (ld.
\ref{sec:tanulasi_celok}).

A tanulásszervezők \emph{modulokat} hirdetnek ezen célokból és a kerettanterv
tantárgyainak tartalmából. A tanárok által meghirdetett modulok reflektálnak a
mai világ alapvető kérdéseire, tudományterületeken és művészeti ágakon átívelő
módon, interdiszciplináris keretek között kerülnek megszervezésre úgy, hogy a
tudásszerzés, az önálló gondolkodás és az alkotás megfelelő arányban jelenik
meg a gyerekek mindennapjaiban. (ld. \ref{sec:modulok}. fejezet)

A kerettanterv három tantárgyat határoz meg, azok
témaköreit, tartalmát és követelményeit \emph{tanulási eredmények}
listájaként adja meg (ld. \ref{sec:tanulasi_eredmenyek}). A gyerekek feladata
az iskolában, hogy tanulási eredményeket
érjenek el. Modulok elvégzésével tanulási eredményeket lehet
elérni.

A modulok végeztével a gyerekek eredményei bekerülnek saját portfóliójukba (ld.
\ref{sec:portfolio}. fejezet),
melyek tartalmazhatnak önálló, vagy csoportos alkotásokat, vizsgafeladatokat,
egymás felé történő visszajelzéseket, a fejlődést jól mérhető dokumentációkat.
Erre a portfólióra épül a Budapest School visszajelző és értékelő rendszere (ld. \ref{sec:ertekeles}), a
portfólió alapján ismeri el az iskola az évfolyamok teljesítését. Szükség
esetén a portfólió alapján kaphatnak a gyerekek osztályzatokat is.

Az egyes
tanulási modulok során elért eredmények a Budapest School három tantárgyában
való fejlődést segítik úgy, hogy egyszerre jelennek meg benne a miniszter által
kiadott kerettantervek tantárgyi elvárásai, a gyerekek saját céljai és a mai
világra való integrált reflexió. Sajátként megélt tanulási útjuk során a
gyerekek így egyaránt fejlődnek a világ tudományos megértésében, leírásában és
újraírásában (STEM tantárgy), az emberiség művészeti, történeti és
szociokulturális
megismerésében, és az azt segítő kommunikációban (Kult tantárgy), valamint az
önmagukhoz
és környezetükhöz való kapcsolódásban (Harmónia tantárgy). (ld.
\ref{sec:tantargyak})

\section{A mikroiskolák, a Budapest School összevont osztályai}
\label{sec:mikroiskola}

A Budapest School iskolában összevont osztályokban, azaz kevert korosztályú és
maximum 6 évfolyamszintű közösségben tanulnak a gyerekek. Az összevont
osztályokat a Budapest School kerettanterv \emph{mikroiskolának} hívja, ezzel
is
hangsúlyossá téve ezek egyedi jellemzőit:
\begin{itemize}
      \item A mikroiskolákban  tanulásszervező  \emph{tanárcsapatok} vezetik a
            közösségeket. 
      \item A mikroiskolák saját szabályaik, normarendszerük, szokásaik,
            kultúrájuk alakul ki.
      \item A mikroiskolák maguk alakítják saját órarendjüket, modulkínálatukat
            és ebben nem kell más mikroiskolákhoz igazodniuk.
\end{itemize}


\paragraph{A mikroiskolákat tanulásszervezők vezetik.}

A tanulóközösség fontos célja, hogy biztonságot, támogatást nyújtson, és
\emph{így} segítse a közösség tagjainak a minőségi tanulását.

A mikroiskolát az igazgató által kinevezett tanulásszervezők irányítják.
Ők felelnek a tanulás tartalmáért, a modulok meghirdetéséért és a
tanulási eredmények nyomonkövetéséért. Ők döntenek a jelentkező gyerekek
kiválasztásáról és a mikroiskola mint közösség összetételéről.

A különböző tanárszerepeket, a tanulásszervezők, mentorok és modulvezetők kapcsolódását \aref{sec:tanarok}. fejezet mutatja be.

\paragraph{Mikroiskola egy nagy csoport.}

Egy mikroiskola minimális létszáma 6 maximális létszáma 60 fő. Minden
mikroiskolának megfelelő számú olyan tanulásszervezővel kell
rendelkeznie, aki mentortanárként is végzi munkáját.

\paragraph{A mikroiskolák korkülönbségei állandók, a gyerekek együtt nőnek.}
Egy mikroiskolában fő szabály szerint legfeljebb 6 (egymást követő)
évfolyamszintnek megfelelő korosztály tanul együtt. A mikroiskola
korosztályait, és az induláskor meglévő évfolyamokszinteket a jelentkező ill. a
felvett gyerekek életkora alapján határozza meg az alapító.

A mikroiskolák korhatára, mint az összevont osztályok korhatárai, a gyerekekkel
változnak. Ettől eltérően a korhatárokat tágítani és szűkíteni évente egyszer
lehet, és
erről az iskola értesíti a szülőket minden tanévkezdést megelőző február 15-ig.
(pl. Amennyiben ezzel nem sérül a maximum 6 évfolyam elve, a korhatárok
tágíthatóak, vagy ha a legalsó vagy legfelső korosztályba tartozó gyerekek
elmentek, a mikroiskola dönthet a korhatárok szűkítéséről.)
A Budapest School mikroiskolái a 12. évfolyamig tartanak, kivéve ha a
mikroiskola valamilyen okból megszűnik.

\paragraph{A mikroiskola állandó, a gyerekek és tanárok jöhetnek és mehetnek.}
A Budapest School mikroiskolái úgy működnek, mint egy összevont osztály. A
tanulásszervező tanárok vagy gyerekek kilépése a mikroiskola fennállását nem
érinti, helyettük a mikroiskola új tanulásszervező tanárt és
gyereket vehet fel.  A mikroiskola létrehozásakor arra kell törekedni, hogy
olyan
gyerekek tanuljanak együtt, akik támogatni tudják egymást a tanulásban. A
gyerekek a mikroiskola tagjai addig, amíg ott jól tudnak tanulni, és a közösség
és a gyerek kapcsolat gyümölcsöző.

\paragraph{A mikroiskoláknak saját fókuszuk, helyszínük, stílusuk alakulhat
      ki.}
A mikroiskolák nemcsak abban térnek el egymástól, hogy kevert korcsoportban,
más korosztályú gyerekek, más érdeklődések mentén, és ily módon más célokat
követve tanulnak, hanem területileg, regionálisan is eltérőek lehetnek.

A mikroiskola-rendszerben lehetőség van arra, hogy adott tanulási környezetben
úgy váltakozhassanak a hangsúlyok a csoport és az egyén érdeklődését követve,
hogy közben
fennmaradjon a tanulási egyensúly a tantárgyak között.

Van olyan mikroiskola, amely a fejlesztési célok eléréséhez és az egyéni célok
mentén már 6 éves gyerekek tanulásánál a robotika eszközeit használja, másutt
drámafoglalkozásokkal fejlesztik 12 éves gyerekek a szövegértésüket és
éntudatukat.

\paragraph{A mikroiskolákban a tanulók nagymértékben befolyásolják, hogy mit és
      hogyan
      tanulnak és alkotnak.}
A mikroiskolákban (a tanulásszervezők által meghatározott kereteken belül)
megfér egymással több, különböző egyéni céllal rendelkező gyerek addig amíg a
tanulásszervezők minden gyerek számára biztosítani tudják a kerettantervben
megfogalmazott tanulási eredmények eléréset.

A tanulásszervezők feladata és felelőssége, hogy olyan közösségeket
válogassannak össze és építsenek, amelyek kellően diverzek, és mégis jól
működnek. A közösségnek a gyerekek igényeit és a kerettanterv céljait egyaránt
ki kell elégíteni.

A tanulásszervezők választási lehetőségeket kínálnak, (azaz modulokat dolgoznak
ki), amikből a gyerekek (a mentoruk és szüleik segítségével) a saját céljaikat,
érdeklődésüket leginkább támogató egyéni tanulási tervet és utat alkotnak.

Eltérhet, hogy egy-egy gyerek mit tanul, ezért az is, hogy mikor és hogyan
sajátítja el a szükséges ismereteket: egy közösségben megfér a központi
felvételire fókuszáló 11 éves gyerek, és az is, aki ekkor inkább a Minecraft
programozásában akar elmélyedni, ezért más képességek fejlesztésével lassabban
halad.

\paragraph{Kisebb csoportokban tanulhatnak a gyerekek.}
\label{sec:csoportbontasok}

A mikroiskolákban a közösséget kisebb csoportokra bonthatjuk, ha a
tanulásszervezés ezáltal hatékonyabb. Egyes moduloknál a gyerekek egy-egy
projektre szerveződnek, ilyenkor általában az eltérő képességű és életkorú
gyerekek is kitűnően tudnak együtt dolgozni. Más modulok esetén a csoportokat a
tanár képességszint alapján hozza létre. Ilyen csoportok lehetnek a másodfokú
egyenletek megoldóképletét megismerő csoport, az írni tanulók csoportja, vagy
egy angol nyelvű újság szerkesztésére és megírására alakult modul, ahol a
nyelvismeretnek és a szövegalkotási képességnek már egy olyan szintjén kell
állni, hogy a projektnek jól mérhető kimenete lehessen.

\paragraph{A mikroiskolák diverz, integratív közösségek.}
A Budapest School mikroiskolák társadalmi, kulturális és gazdasági értelemben
is diverzek és egyik fő céljuknak tekintik az integrációt addig, mindaddig amíg
az a közösség céljait szolgálja.

\paragraph{A mikroiskolák tanuló közösségek.}
A Budapest School célja, hogy a mikroiskolákban történő tanulás mind a gyerek,
mind a tanár, mind a szülő számára jól átlátható, követhető legyen, és gyerek
és a közösség folyamatosan fejlődjön.
A Budapest School kiemelt elve, hogy ,mindig, minden módszer, folyamat
fejleszthető, ezért a tanárok feladata, lehetősége, hogy az aktuális helyzethez
illő legmegfelelőbb módszert válasszák meg a gyerekek tanulásának segítéséhez.

\paragraph{Mikroiskolát a fenntartó indít és addig él, amíg szolgálja a
      gyerekek tanulását.}
A mikroiskolát a fenntartó indítja. Meghatározza, hogy milyen telephelyen,
tagintézményben, milyen korhatárokkal és létszámokkal
induljon a mikroiskola,. A fenntartó feladata a szükséges épületet, eszközöket
biztosítania. A
mikroiskola alapításához legalább 6 gyerek és egy tanulásszervező (aki
értelemszerűen mentortanár)
szükséges.

A mikroiskola a tanárokon és a gyerekeken is túlmutató közösség, amely akkor is
tovább működik, ha egy tanár vagy gyerek távozik. Tanulásszervező illetve
gyerek távozása esetén a mikroiskola  új tanulásszervező tanárt és gyereket
vesz fel, mindaddig, amíg a mikroiskola számára meghatározott maximális
létszámot el nem érik.

A mikroiskola abban az esetekben megszűnik meg, ha összeolvad egy másik
mikroiskolával vagy
ha a mikroiskola létszáma 6 gyerek és egy mentortanár alá csökken.

\paragraph{Gyerek csatlakozása a mikroiskolához.}
Arról, hogy egy gyerek csatlakozhat-e egy mikroiskolá közösségéhez, a
tanulásszervező
tanárok döntenek, figyelembe véve a gyerek életkorát, a közösségben való
eligazodását, érdeklődését, egyéni fejlődési igényét. Elv egyszerű: minden
gyereket fogadjon be a közösség, amitől a közösség jobban tudja támogatni az
egyének tanulási céljait.\footnote{A mikroiskolához gyerek akkor csatlakozhat,
      ha ő már másik mikroiskola, így az iskola tanulója, vagy az iskola
      igazgatója a
      gyerek felvétele mellett döntött.}

\footnote{Átjárás mikroiskolák között.}
Egy gyerek akkor válthat a Budapest School egyes mikroiskolái között,
amennyiben a fogadó mikroiskola őt elfogadja. Ilyenkor új mentortanárt kell
számára kijelölni.

\section{Moduláris tanmenet és a tanulási eredmények}

\subsection{Modulok -- a tanulásszervezés alapegységei}
\label{sec:modulok}

A \emph{modulok} a tanulásszervezés \emph{alapegységei}: olyan foglalkozások megtervezett sorozata, amelyek során egy meghatározott időn belül a gyerekek valamely képességüket fejlesztik, valamilyen ismeretet elsajátítanak, vagy valamilyen produktumot létrehoznak. A modulok célja sokféle lehet, de kötelező elvárás, hogy a résztvevők a portfóliójukba bejegyzésre érdemes eredményt hozzanak létre, vagyis hogy legyen egyértelmű célja.

A mindennapi tanulás a modulok elvégzésén keresztül történik, ezzel biztosítva, hogy rugalmas keretek között, pontosan megfogalmazott célok mentén, a gyerekek számára érthető, átlátható és sajátnak megélt tartalommal történjen a tanulás.

A tanulási modulokat, vagyis a tanulás tartalmának és formájának alapegységét a tanulásszervezők három kötelező összetevőből állítják össze:

\begin{enumerate}
      \item
            a kerettanterv tantárgyainak tartalmából,
      \item
            a gyerekek, tanárok érdeklődéséből, aktuális tudásából,
      \item
            és a környezetük és a világ aktuális kihívásaiból.
\end{enumerate}

A három komponensből a legelső a legstatikusabb, hiszen a kerettanterv -- összhangban a NAT-al -- meghatározza a tantárgyakat és azok tartalmát, valamint azt, hogy milyen lehetséges eredmények elérését várjuk az ezekben való fejlődéstől. Az egyes modulokban ezek személyre, illetve a csoport igényeire szabhatóak, hiszen az elérhető eredményeket különféle gyakorlati és elméleti tanulási módszerekkel el lehet érni.

A gyerekek és tanárok érdeklődése -- ami a sajátként megélt cél és a minél nagyobb fokú bevonódás alapfeltétele -- alakítja ki a modulok témáját, a projekteket, és a gyerekek egyéni tanulási idejét is meghatározhatja.

Mindemellett a kerettanterv szándéka, hogy a tanárok, gyerekek reagáljanak a környezetükre, a világ aktuális kihívásaira, kérdéseire. A kerettanterv meghatározza például, hogy a gyerek ,,\emph{táblázatkezelővel feladatot old meg}''. Az azonban, hogy a gyerekek milyen táblázatokat szerkesztenek szívesen, csak a modulok összeállításakor és a modulok elvégzése során derül ki. Nagyon hasonló táblázatkezelési képességeket lehet fejleszteni, ha valaki az önvezető autóktól várt csökkenő baleseti halálozási arányról, vagy ha a vegánok számának és a GDP-növekedés alakulásának arányáról készít táblázatot.

A moduláris rendszer fő célja, hogy egyszerre képes legyen alkalmazkodni a menet közben felmerülő tanulási igényekhez, adjon átlátható struktúrát a tanulásnak, és hogy a mikroiskola minél rugalmasabban tudja támogatni\break a tanulást, úgy, hogy a saját, a közösségi és a társadalmi célok harmóniába kerülhessenek.

Ez is mutatja, hogy bár közösek a kereteink, végtelen az elképzelhető modulok (a tanulási utak építőkövei, és így a különböző tanulási utak) száma. Ezért tartja fontosabbnak a kerettanterv annak meghatározását, hogy hogyan kell a modulokat létrehozni, mint azt, hogy a modulokat tételesen felsorolja.

Modulok során a gyerekek tudnak

\begin{itemize}
      \item produktum létrehozására szerveződő projektben részt venni;

      \item felfedezni, feltalálni, kutatni, vizsgálni, azaz kérdésekre választ keresni;

      \item egy jelenséget több nézőpontból megismerni;

      \item valamely képességüket, készségüket fejleszteni;

      \item adott vizsgára gyakorló feladatokkal felkészülni;

      \item közösségi programokban részt venni;

      \item az önismeretükkel, a tudatosságukkal, a testi-lelki jóllétükkel foglalkozni.
\end{itemize}

\subsubsection{A modulok meghirdetése}
\label{sec:modulok_meghirdetese}
A modulok kiválasztása, felkínálása a tanulásszervezők feladata, hiszen ők figyelnek és reagálnak a gyerekek, szülők céljaira és igényeire. A meghirdetett modulokból áll össze a tanulás trimeszterenkénti tanulási rendje.

A tanulásszervezők az egyes modulok tematikáját, azok hosszát és feladatát a gyerekek tanulási céljainak megismerését követően és a kerettantervben meghatározott tantárgyi tanulási eredményeket figyelembe véve határozzák meg.

A nem kötelező modulokba való csatlakozásról a mentor, a szülő és a gyerek közösen dönt, mindig szem előtt tartva, hogy folyamatos előrelépés legyen a már elért egyéni és tantárgyi eredményekben is. Egy modul megkezdésének lehet feltétele egy korábbi modul elvégzése, a gyerek képességszintje, a jelentkezők száma, és lehet egyedüli feltétele a gyerekek érdeklődése.

Egy modulvezető különféle tematikájú modulokat tarthat függően attól, hogy a saját célok, a tantárgyi eredmények mit kívánnak, és a tanulásszervezők, valamint a modulvezetők kapacitása mit enged.

Amikor egy gyerek moduljai befejeződnek, és újat vesz fel, a tanulásszervező feladata a gyereket segíteni abban, hogy az érdeklődési körének, tanulási céljainak, és a soron következő, még el nem ért tantárgyi eredményekben való fejlődéshez megfelelő modulok közül választhasson.

A tanulásszervezők feladata a tantárgyi eredményelvárások nyomon követése is. A modulok kidolgozáshoz és azok megtartásához külsős szakembereket is meghívhatnak, azonban ilyenkor is ők felelnek azért, hogy a modulokkal elérni kívánt tanulási célok teljesüljenek.

\subsubsection{A modulok formátuma}

Egy-egy modul hossza és a modulhoz kapcsolódó foglalkozások száma és gyakorisága változó: egy alkalomtól legfeljebb egy teljesen trimeszteren keresztül tarthat. A modul végén azt a tanulásszervező és a gyerek(ek) lezárják, értékelik és az elért eredményeket rögzítik a (tanulási) portfólióban. Egy modul folytatásaként a következő trimeszterben új modult lehet meghirdetni.

A modulok nemcsak témájukban, céljaikban, időtartamukban, hanem módszertanukban, folyamataikban is különbözhetnek: bizonyos modulokban a felfedeztető (inquiry based) módszer, másokban az ismétlő (repetitív) gyakorlás a célravezető. Így mindig a modul céljához, a tanárok és a gyerekek képességeihez és igényeihez választható a legjobb módszer. Modulonként változhat, hogy a folyamatot a gyerekek vagy a tanárok befolyásolják-e, és milyen mértékben. Két példa az eltérésre:

\begin{enumerate}
      \item Egy digitális kézműves modul célja, hogy építsünk valamit, ami programozható. Annak kitalálása, hogy mit és hogyan építünk, a gyerekek feladata. Itt a modul vezetője csak támogatja a tanulás folyamatát, azaz \emph{facilitál}.

      \item Egy „\emph{A vizuális kommunikáció fejlődése a XX. század második felében}'' modul esetén a tanár előre felépíti a tanmenetet, pl. hogy mely alkotók munkásságát, alkotásokat fogja bemutni, és ezeket a gyerekekkel sorban végigveszi. Ilyenkor is bővülhet azonban a tematika a gyerekek érdeklődése, felvetései mentén.

\end{enumerate}

\subsubsection{A modulok helyszíne}

A tanulás az egyes mikroiskolák helyszínén, egy másik Budapest School mikroiskolában, a tanár által kiválasztott külső helyszínen, vagy akár online, virtuális térben történik. A tanulásra úgy tekintünk, mint az élethez szorosan kapcsolódó holisztikus fejlődési igényre, melynek jegyében az elsődleges szocializációs tértől és formától, a szülői, családi környezettől sem akarjuk a tanulást leválasztani. Az élethosszig tartó tanulás jegyében a tanulás tere az iskolai időszak után és az iskola terein kívül is folytatódik.

A gyerekek több ok miatt is tanulnak az iskolán kívül:

\begin{enumerate}
      \item Modulok vagy modulok foglalkozásai szervezhetők külső helyszínekre, úgymint múzeumokba, erdei iskolákba, parkokba, vállalatokhoz, vagy tölthetik az idejüket „kint a társadalomban''.

      \item Amennyiben ez saját céljuk elérését nem veszélyezteti, és a folyamatos fejlődés biztosított, a mentoruk tudomásával a gyerekek az önirányított tanulás elvére figyelemmel a mikroiskolán kívüli egyéb helyszínen is elvégezhetnek egy modult.
\end{enumerate}

A modul lezárásaként a gyerekek és modulvezetők visszajelzést adnak egymásnak, aminek része, hogy megosztják saját élményeiket, reflektálnak a közös időre, összegyűjtik és értékelik az elért eredményt, és kitérnek az esetleges fejlődési lehetőségekre.

\subsection{Tanulási eredmények -- a formális tanulás alapegységei}
\label{sec:tanulasi_eredmenyek}
A kerettanterv három interdiszciplináris tantárgyat jelöl meg, azok témaköreit, tartalmát és követelményeit \emph{tanulási eredmények} listájaként adja meg, ezzel igazodva az Nkt.~5.~§ (5) pontjához. Tanulási eredmény (learning outcome) lehet a kerettanterv szellemében minden olyan tudás, képesség, kompetencia, attitűd, amit a gyerek egy tanulási folyamat során elsajátított és/vagy ezt demonstrálni tudja. Az eredmény eléréséhez vezető út a modulokon keresztül történik, és a tanulás folyamata történhet az iskolában vagy azon kívül, lehet formális, non-formális vagy informális.

A tanulási eredmények több funkciót látnak el a kerettantervben.

\begin{itemize}

      \item A kerettanterv évfolyamonként meghatározza az adott tantárgy teljesítéséhez elérendő tanulási eredményeket. Egy gyerek akkor  léphet egy tantárgyból évfolyamszintet, ha a tantárgyhoz tartozó követelményeket teljesítette.

      \item A tanulási eredmények a modulok (és így a mindennapokban szervezett foglalkozások, órák stb.) építőelemei. Egy-egy modul célját a  tanulásszervezők az elérendő tanulási eredmények	összeválogatásával és saját célokkal, érdeklődéssel való	kiegészítésével adják meg, figyelembe véve az életkori  sajátosságok, az egymásra épülés és az átjárhatóság  követelményeit.
      \item A tanulási eredmények megfeleltethetőek a miniszter által kiadott kerettantervek tantárgyai (és így a kötelező érettségi tárgyai )  és a NAT műveltségi területeivel, ami biztosítja, hogy a Budapest  School tanulója más rendszerben működő iskolába is illeszkedik.  Vagyis a tanulási eredmények a Budapest School saját interdiszciplináris tantárgyi elvárásain túl, az azokon belüli halmazt képző diszciplináris bontásban is követhetőek, így az	elért eredmények alapján mind a Budapest School tantárgyi  struktúrájával, mind (pl. egy esetleges iskolaváltás esetén) a	miniszter által kiadott kerettantervek tantárgystruktúrájával megfeleltethetőek.
\end{itemize}

\subsection{Modulok és tanulási eredmények}
\label{sec:modulok_es_tanulasi_eredmenyek}
A gyerekek egyik feladata az iskolában, hogy tanulási eredményeket érjenek el. Ezt megtehetik a modulok elvégzésével, vagy más tanulási helyzetekben. A tanulási eredményeket a portfólióban rögzítik. A mentor feladata, hogy folyamatosan kövesse, hogy megfelelő haladás történik-e a portfólióban a tanulási eredmények és a saját célok tekintetében. Az évfolyamszintlépés a portfólióban összegyűlt tanulási eredmények alapján történhet meg.

A modul kecsegtet a gyerekek haladásához releváns tanulási eredményekkel, a gyerekek által meghatározott saját célokkal és olyan kimenettel, amely a portfólióban rögzíthető, legyen az egy alkotás, az elért fizikai vagy szellemi eredmény dokumentációja, vagy egy értékelő visszajelzés. A modulok tehát tartalmaznak tanulási eredményeket, az önálló gondolkodás, szabad alkotás lehetőségét és teret engednek az alkotásra, létrehozásra.

\paragraph{A modulok különféle tanulási eredmények elérését teszik elérhetővé}

Modulok tervezésekor és összeállításakor a tanulásszervezők a modulvezetővel közösen határozzák meg a modul céljait, de azok meghirdetéséért mindig a tanulásszervezők felelnek. A célok között fel kell sorolni, hogy milyen tanulási eredmények elérését várhatják el a gyerekek a modulon való részvételtől.

Például a 6-8 éves gyerekek számára megtervezett ,,\emph{3d nyomtató használata}'' modul során azon kívül, hogy megismerik a 3d nyomtatás folyamatát, a modul célja, hogy a gyerekek számára elérhetővé tegye a ,,\emph{A kockát, téglatestet, gömböt felismeri, és képes létrehozni egyszerű módszerekkel. Ismeri ezeknek a testeknek a jellemzőit.}'' (STEM tantárgy, Matematika tématerület, 3--4 évfolyam) tanulási eredményt is.

Lehetőség van egy modul esetében több tantárgyból való tanulási eredmény kiválasztására, ezzel biztosítva az interdiszciplinaritást, valamint a Budapest School tantárgyi fejlesztési céljaihoz való integrált kapcsolódást.

A tanulási eredmények egy időbeni egymásra épülést feltételeznek, melyben azonban van lehetőség előre- és hátrafele is lépni. Előre, amennyiben a modul meghirdetésekor az arra jelentkező gyerekcsoportnál a megfelelő előkészítés megtörtént, hátra, amennyiben ezt ismétlés/felzárkóztatás jelleggel szükségesnek ítéli a mentor vagy a modult szervező, vezető. Vagyis akkor foglalkozzon egy gyerek a 10~000-es számkörrel, ha a 100-as számkört már begyakorolta. Az egymásra épülésért a modult meghirdető tanulásszervező felel. A példát folytatva a 3d nyomtató használata modul lehetővé teszi, hogy a gyerek elérje a következő eredményeket is: \emph{,,Ismeri a számítógép
      részeinek és perifériáinak funkcióit, azokat önállóan használja.''}
(Harmónia, Informatika, 5--6 évfolyam), és  \emph{,,Használati utasításokat
      értő módon olvas és tart be.''} (Harmónia, Életvitel, 3--4)

\paragraph{Új tanulási eredmények}

A gyerekek olyan tanulási eredményt is elérhetnek, ami a modulok céljai között eredetileg nem volt megadva, mert

\begin{itemize}
      \item lehetőségük van egyénileg is tanulni;

      \item tanulási eredményekkel járnak a projektek, az iskolai lét, a közösségi élet és még számos informális és non-formális tanulási helyzet;

      \item egy modul során is alakulhatnak előre nem tervezett helyzetek, amik hozzásegíthetik a gyerekeket tanulási eredmények eléréséhez.
\end{itemize}

Az újonnan létrejövő tanulási eredmények is bekerülnek a portfólióba.

\paragraph{Tanulási eredmények dokumentációja}

Minden modul dokumentálásra kerül, hogy annak célja, elért eredményei nyilvánosak legyenek a Budapest School valamennyi mikroiskolája számára, és ha szükséges, újra meg lehessen hirdetni. A tanulási eredmények egy, a modulhoz kapcsolódó terv-tény összehasonlítás alapján kerülnek meghatározásra. Az elért eredmények újra elérhetőek, amennyiben a folyamatos fejlődés biztosítva van.

\paragraph{Egységes modulok egyedi alkalmazása}

Egy modul elvégzésével egy-egy gyerek más tanulási eredményt is elérhet.

\begin{itemize}
      \item
            Működhet a differenciálás, tehát nem minden gyerek ugyanazt és ugyanúgy csinálja a foglalkozásokon. Egy modulban tud együtt	tanulni az a gyerek, aki még ,,\emph{Ismeri az írott és nyomtatott  betűket''} eredményért dolgozik, és az, aki ,,\emph{Jelöli helyesen a j	hangot 30--40 begyakorolt szóban''.}
      \item
            A modulnak része lehet testreszabható sáv. Például egy tudományos kísérletező modulban néhány gyerek a rövid távú memória és a	fáradtság kapcsolatáról kutat, a másik csoport az esőzés és a	közlekedési dugók kialakulása közti kapcsolatot vizsgálja. Minden  gyerek elérheti a ,,\emph{Valós folyamatokat képes elemezni a folyamathoz tartozó függvény grafikonja alapján.}'' (forrás, STEM), de a ,,\emph{Környezettudatos közlekedésszemlélet.}'' (forrás, Harmónia)	eredményt is elérheti.
      \item
            Egy-egy gyerek saját tanulási célja érdekében extra lépéseket tehet, és olyan eredményeket is el tud érni, amit mások nem.	Például egy modul végén önálló prezentációt, saját kutatási  tervet, vagy egy kész működő modellt alkothat.
\end{itemize}

\subsubsection{Kötelező tanulási eredmények}
\label{sec:kotelezo_tanulasi_eredmenyek}
A kerettanterv kötelező tanulási eredményként definiálja mindazokat az eredményeket, melyek a kötelező érettségi tárgyak teljesítéséhez szükségesek. Ezeket minden mikroiskola elérhetővé kell, hogy tegye a gyerekek számára a modulok választékában.

Ezek az 1--4 évfolyamszinteken a miniszter által kiadott kerettantervek \emph{Magyar nyelv és irodalom}, \emph{Matematika}, \emph{Idegen nyelv} tantárgyakból származó tanulási eredmények, és 5.~évfolyamszinttől kiegészülnek a \emph{történelem, társadalmi és állampolgári ismeretek} tantárgyak alapján létrehozott tanulási eredményekkel. További kötelező tanulási eredményként jelennek meg 9.~évfolyamtól a választott érettségi tantárgyhoz kapcsolódó eredmények. Ezek a tanulási eredmények megtalálhatók a kerettanterv három tantárgyának elérhető eredményei között.

A mikroiskolában meghirdetett moduloknak a kötelező tanulási eredmények 80\%-át le kell fednie.

\paragraph{Tanulási eredmények kiegyenlítettsége}

Szintén fontos kötöttség, hogy a modulok meghirdetésénél a kerettanterv három tantárgyából egyenlő súllyal (plusz/minusz 20\%) legyenek elérhetőek tanulási eredmények. Emellett a kerettanterv minden tématerületéről (vagyis a tantárgyak eredményeit alkotó diszciplinákból) legalább 20\% tanulási eredményt kell választani, így biztosítva, hogy a NAT minden műveltségterülete megjelenjen a tanárok által lefedett témák között.

\paragraph{Kötelező modulok}
A kerettanterv és a pedagógia program is előírhat kötelező modulokat a mikroiskolák számára. Ilyen például a 11. évfolyamszinten belépő érettségire felkészítő modulok (ld. \ref{sec:erettsegi}. fejezet), a minden mikroiskolára egységes pedagógia program tetszőleges kötelező modult írhat elő. Így lehet biztosítani kötelező tartalmi elemek és foglalkozás -- úgymint testnevelés, elsősegélynyújtás -- elérhetőségét.

\subsubsection{Monitorozás}

Kötelező elérni az eredményeket? Nem tudunk hatalmi szóval tanulásra bírni gyereket, mert lehet, hogy annyira nem akarja, vagy nincs meg hozzá a képessége. A kerettanterv a tanároknak ad keretet. Azonban a fenntartó által üzemeltetett rendszerrel az iskola  monitorozza a haladást, és ha valaki a kötelező tanulási elemekkel nem halad, akkor az iskola erre felhívja a figyelmét. Mivel a többség haladni fog, ezért előre tudja az iskola jelezni, hogy le fog szakadni a többiektől, és túl nagy lesz az évfolyamszint-különbség közöttük. Ezekben az esetekben a mentortanárnak, a gyereknek és a szülőnek reagálnia kell a helyzetre. A fenntartó által működtetett monitorozó és minőségfejlesztő rendszerről \aref{sec:minosegbiztositas} fejezet ír részletesen.

\ifkerettanterv
  \section{Különböző tanári szerepek: a tanulásszervező, a mentor és a
    modulvezető}
\label{sec:tanarok}

A Budapest Schoolban a gyerekek azokat a felnőtteket tekintik tanáruknak, akik
minőségi időt töltenek velük, és segítik, támogatják vagy vezetik őket a
tanulásukban. Több szerepre bontjuk a tanár fogalmát: a gyerek egy (és csak
egy) felnőtthöz különösen kapcsolódik, a \emph{mentortanárához}, aki rá
különösen figyel. Ezenkívül a gyerek tudja, hogy a mikroiskola mindennapjait
egy tanárcsapat, a \emph{tanulásszervezők} határozzák meg, ők vezetik az
iskolát.  A foglalkozásokon megjelenhetnek további tanárok, a \emph{modulvezetők},
akik egy adott foglalkozást, szakkört, órát tartanak.

Szervezetileg minden mikroiskolának van egy állandó \emph{tanárcsapata}, a
tanulásszervezők. A tanulásszervezők legalább egy tanévre elköteleződnek, szemben a
modulvezetőkkel, akik lehet, hogy csak egy pár hetes projekt keretében vesznek részt a
munkában.

A tanulásszervezők általában mentorok is, de nem minden esetben. Nem lehet
mentor az, aki a gyerek mikroiskolájában nem tanulásszervező, mert nem lenne
rálátása a mikroiskola történéseire. Egy tanulásszervező lehet több
mikroiskolában is ebben a szerepben, és így mentor is lehet több mikroiskolában.

\paragraph{Mentor}
Minden gyereknek van egy \emph{mentora}, aki a saját céljainak
megfogalmazásában és
a fejlődése követésében segíti. Minden mentorhoz több gyerek tartozik, de nem
több, mint 12. A mentor együtt dolgozik a mikroiskola többi tanulásszervezőjével, a
szülőkkel és az általa mentorált gyerekekkel. A mentor segít az általa
mentorált gyereknek, hogy a tantárgyi fejlesztési célok és
a
saját magának megfogalmazott saját célok között megtalálja  az egyensúlyt, és segít megalkotni a
gyerek \emph{saját
  tanulási tervét}.

A mentor a kapocs a Budapest School, a szülő és a gyerek között.

\begin{itemize}
  \item Képviseli a Budapest Schoolt, a mikroiskola közösségét.
        \begin{itemize}
          \item Ismeri a Budapest Schoolt, a lehetőségeket, a tanulásszervezés
                folyamatait.
          \item Együtt tanul más Budapest School-mentorokkal, együtt dolgozik a
                tanártársaival.
        \end{itemize}

  \item Ismeri, segíti, képviseli a gyereket.
        \begin{itemize}
          \item  Tudja, hol és merre tart mentoráltja, ismeri a képességeit,
                körülményeit, szándékait, vágyait.
          \item    Segít a saját célok elérésében, felügyeli a haladást.
          \item    Megerősíti mentoráltjai pszichológiai biztonságérzetét.
          \item   Visszajelzéseket ad a mentoráltjainak.
          \item    Segít abban, hogy az elért célok a portfólióba kerüljenek.
          \item    Összeveti a portfólió tartalmát a tantárgyak fejlesztési
                céljaival.
        \end{itemize}

  \item Együtt dolgozik, gondolkozik a szülőkkel, képviseli igényüket a
        közösség felé.
        \begin{itemize}
          \item Erős partneri kapcsolatot épít ki a szülőkkel, információt oszt meg
                velük.
          \item Segít a gyerekekkel közös célokat állítani.
          \item Szülő számára a mentor az elsődleges kapcsolattartó a különféle
                iskolai ügyekkel kapcsolatban.
        \end{itemize}

\end{itemize}

A mentor egyszerre felelős a mentorált gyerek előrehaladásának segítéséért,
és
közös felelőssége van a mentortársakkal, hogy az iskolában a lehető legtöbbet
tanuljanak a gyerekek. A mentor folyamatosan figyelemmel követi az egyéni
tanulási tervben megfogalmazottakat, és ezzel kapcsolatos visszajelzést ad a
mentoráltnak és a szülőnek.

\paragraph{Tanulásszervező}
Csoportban dolgozó, iskolaszervező, strukturáló tanár. Egy mikroiskola
állandó tanári
csapatát 2-7 tanulásszervező alkotja, akik egyedileg meghatározott szerepek
mentén a mikroiskola mindennapjainak működtetéséért felelnek. Minden mentor
tanulásszervező is. A tanulásszervezők tarthatnak
modulokat, sőt, kívánatos is, hogy dolgozzanak a gyerekekkel, ne csak
szervezzék az életüket.
Ők rendelik meg a külső modulvezetőktől a munkát, ilyen értelemben a
tanulási utak projektmenedzserei.

\paragraph{Modulvezetők}

Bárki lehet modulvezető, aki képes akár egy egyetlen alkalommal történő, vagy
éppen
egy egész trimeszteren át tartó tanulási, alkotási folyamatot vezetni. Ők
általában
az adott tudományos, művészeti, nyelvi vagy bármilyen más terület szakértői.

A modulokat a tanulásszervezők is vezethetik, de külsős, egyedi megbízással
dolgozó szakemberek is megjelennek modulvezetőként. Modulvezető lehet bárki,
akiről az őt megbízó tanárcsapat tudja, hogy képes gyerekek folyamatos
fejlődését és egy tanulási cél felé való haladását segíteni. A moduláris
tanmenettel \aref{sec:modulok}. fejezet foglalkozik.
\fi
