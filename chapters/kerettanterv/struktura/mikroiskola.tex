\section{A mikroiskolák, a Budapest School összevont osztályai}
\label{sec:mikroiskola}

A Budapest School iskolában összevont osztályokban, azaz kevert korosztályú és maximum 6 évfolyamszintű közösségben tanulnak a gyerekek. Az összevont osztályokat a Budapest School kerettanterv \emph{mikroiskoláknak} hívja, ezzel is hangsúlyossá téve ezek egyedi jellemzőit:
\begin{itemize}
      \item A mikroiskolákban  tanulásszervező	\emph{tanárcsapatok} vezetik a közösségeket.
      \item A mikroiskolák maguk alakítják ki saját szabályaikat, normarendszerüket, szokásaikat és kultúrájukat.
      \item A mikroiskolák maguk alakítják saját órarendjüket, modulkínálatukat, és ebben nem kell más mikroiskolákhoz igazodniuk.
\end{itemize}

\paragraph{A mikroiskolákat tanulásszervezők vezetik.}

A tanulóközösség fontos célja, hogy biztonságot, támogatást nyújtson, és \emph{így} segítse a közösség tagjainak a minőségi tanulását.

A mikroiskolát az igazgató által kinevezett tanulásszervezők irányítják. Ők felelnek a tanulás tartalmáért, a modulok meghirdetéséért és a tanulási eredmények nyomon követéséért. Ők döntenek a jelentkező gyerekek kiválasztásáról és a mikroiskola mint közösség összetételéről.

A különböző tanárszerepeket, a tanulásszervezők, mentorok és modulvezetők kapcsolódását \aref{sec:tanarok}.~fejezet mutatja be.

\paragraph{Mikroiskola egy nagy csoport.}

Egy mikroiskola minimális létszáma 6, maximális létszáma 60 fő. Minden mikroiskolának megfelelő számú olyan tanulásszervezővel kell rendelkeznie, aki mentortanárként is végzi munkáját.

\paragraph{A mikroiskolák korkülönbségei állandók, a gyerekek együtt nőnek.}
Egy mikroiskolában fő szabály szerint legfeljebb 6 (egymást követő) évfolyamszintnek megfelelő korosztály tanul együtt. A mikroiskola korosztályait, és az induláskor meglévő évfolyamszinteket a jelentkező, illetve a felvett gyerekek életkora alapján határozza meg az alapító.

A mikroiskolák korhatára, mint az összevont osztályok korhatárai, a gyerekekkel változnak. Ettől eltérően a korhatárokat tágítani és szűkíteni évente egyszer lehet, és erről az iskola értesíti a szülőket minden tanévkezdést megelőző február 15-ig (például amennyiben ezzel nem sérül a maximum 6 évfolyam elve, a korhatárok tágíthatóak, vagy ha a legalsó vagy legfelső korosztályba tartozó gyerekek elmentek, a mikroiskola dönthet a korhatárok szűkítéséről). A Budapest School mikroiskolái a 12.~évfolyamig tartanak, kivéve, ha a mikroiskola valamilyen okból korábban megszűnik.

\paragraph{A mikroiskola állandó, a gyerekek és tanárok jöhetnek és mehetnek.}
A Budapest School mikroiskolái úgy működnek, mint egy összevont osztály. A tanulásszervező tanárok vagy gyerekek kilépése a mikroiskola fennállását nem érinti, helyettük a mikroiskola új tanulásszervező tanárt és gyereket vehet fel.  A mikroiskola létrehozásakor arra kell törekedni, hogy olyan gyerekek tanuljanak együtt, akik támogatni tudják egymást a tanulásban. A gyerekek a mikroiskola tagjai addig, amíg ott jól tudnak tanulni, és a közösség és a gyerek kapcsolata gyümölcsöző.

\paragraph{A mikroiskoláknak saját fókuszuk, helyszínük, stílusuk alakulhat ki.}
A mikroiskolák nemcsak abban térnek el egymástól, hogy kevert korcsoportban, más korosztályú gyerekek, más érdeklődések mentén, és ily módon más célokat követve tanulnak, hanem területileg, regionálisan is eltérőek lehetnek.

A mikroiskola-rendszerben lehetőség van arra, hogy adott tanulási környezetben úgy váltakozhassanak a hangsúlyok a csoport és az egyén érdeklődését követve, hogy közben fennmaradjon a tanulási egyensúly a tantárgyak között.

Van olyan mikroiskola, amely a fejlesztési célok eléréséhez és a saját célok mentén már 6 éves gyerekek tanulásánál a robotika eszközeit használja, másutt drámafoglalkozásokkal fejlesztik 12 éves gyerekek a szövegértésüket és éntudatukat.

\paragraph{A mikroiskolákban a gyerekek nagymértékben befolyásolják, hogy mit és hogyan tanulnak és alkotnak.}
A mikroiskolákban (a tanulásszervezők által meghatározott kereteken belül) megfér egymással több, különböző saját céllal rendelkező gyerek addig, amíg a tanulásszervezők minden gyerek számára biztosítani tudják a kerettantervben megfogalmazott tanulási eredmények elérését.

A tanulásszervezők feladata és felelőssége, hogy olyan közösségeket építsenek, amelyek kellően diverzek, és mégis jól működnek. A közösségnek a gyerekek igényeit és a kerettanterv céljait egyaránt ki kell elégíteni.

A tanulásszervezők választási lehetőségeket kínálnak (azaz modulokat dolgoznak ki), amikből a gyerekek (a mentoruk és szüleik segítségével) a saját céljaikat, érdeklődésüket leginkább támogató saját tanulási tervet és utat alkotnak.

Eltérhet, hogy egy-egy gyerek mit tanul, ezért az is, hogy mikor és hogyan sajátítja el a szükséges ismereteket: egy közösségben megfér a központi felvételire fókuszáló 11 éves gyerek, és az is, aki ekkor inkább a Minecraft programozásában akar elmélyedni, ezért más képességek fejlesztésével lassabban halad.

\paragraph{Kisebb csoportokban tanulhatnak a gyerekek.}
\label{sec:csoportbontasok}

A mikroiskolákban a közösséget kisebb csoportokra bonthatjuk, ha a tanulásszervezés ezáltal hatékonyabb. Egyes moduloknál a gyerekek egy-egy projektre szerveződnek, ilyenkor általában az eltérő képességű és életkorú gyerekek is kitűnően tudnak együtt dolgozni. Más modulok esetén a csoportokat a tanár képességszint alapján hozza létre. Ilyen csoportok lehetnek a másodfokú egyenletek megoldóképletét megismerő csoport, az írni tanulók csoportja, vagy egy angol nyelvű újság szerkesztésére és megírására alakult modul, ahol a nyelvismeretnek és a szövegalkotási képességnek már egy olyan szintjén kell állni, hogy a projektnek jól mérhető kimenete lehessen.

\paragraph{A mikroiskolák diverz, integratív közösségek.}
A Budapest School mikroiskolák társadalmi, kulturális és gazdasági értelemben is diverzek, és egyik fő céljuknak tekintik az integrációt mindaddig, amíg az a közösség céljait szolgálja.

\paragraph{A mikroiskolák tanuló közösségek.}
A Budapest School célja, hogy a mikroiskolákban történő tanulás mind a gyerek, mind a tanár, mind a szülő számára jól átlátható, követhető legyen, és a gyerek és a közösség folyamatosan fejlődjön. A Budapest School kiemelt elve, hogy mindig, minden módszer, folyamat fejleszthető, ezért a tanárok feladata, lehetősége, hogy az aktuális helyzethez illő legmegfelelőbb módszert válasszák meg a gyerekek tanulásának segítéséhez.

\paragraph{Mikroiskolát a fenntartó indít, és addig él, amíg szolgálja a gyerekek tanulását.}
A mikroiskolát a \emph{fenntartó} indítja el. Meghatározza, hogy a mikroiskola milyen telephelyen, tagintézményben, milyen korhatárokkal és létszámokkal induljon. A fenntartó feladata a szükséges épületet és eszközöket biztosítani. A mikroiskola alapításához legalább 6 gyerek és egy tanulásszervező (aki értelemszerűen mentortanár) szükséges.

A mikroiskola a tanárokon és a gyerekeken is túlmutató közösség, amely akkor is tovább működik, ha egy tanár vagy gyerek távozik. Tanulásszervező vagy gyerek távozása esetén a mikroiskola  új tanulásszervező tanárt, illetve gyereket vesz fel mindaddig, amíg a mikroiskola a számára meghatározott maximális létszámot el nem éri.

A mikroiskola abban az esetben szűnik meg, ha összeolvad egy másik mikroiskolával. Ha a mikroiskola létszáma 6 gyerek vagy egy mentortanár alá csökken és a létszám a következő tanév elejére sem éri el a minimiális szintet, akkor a mikroiskolát másik mikroiskolával kell összeolvasztani.

\paragraph{Csatlakozás a mikroiskolához.}
Arról, hogy egy gyerek csatlakozhat-e egy mikroiskola közösségéhez, a tanulásszervezők döntenek, figyelembe véve a gyerek életkorát, a közösségben való eligazodását, érdeklődését, saját fejlődési igényét. A kiválasztás fő elve,  hogy a közösség fejlődjön minden gyerek csatlakozásával.

Egy gyerek akkor válthat a Budapest School egyes mikroiskolái között, amennyiben a fogadó mikroiskola őt elfogadja. Ilyenkor új mentortanárt kell számára kijelölni.\footnote{Átjárás mikroiskolák között.}
