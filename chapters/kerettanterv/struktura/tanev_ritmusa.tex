\section{A tanév ritmusa}
\label{sec:tanev_ritmusa}
A tanév három trimeszter ismétlődésével írható le: a tanulási célok tervezése után következik a tanulás, és a ciklust a visszajelzés és értékelés zárja.	Amint egy ciklus véget ér, elkezdődik egy új.


Az egyes trimeszterek átlagosan 12 hétig tartanak úgy, hogy a tanítással eltöltött napok  és a tanítási szünetek mindig az EMMI által kiadott tanév rendjéhez igazodva kerülnek meghatározásra. A trimeszterek első hete mindig a tervezéssel, utolsó hete mindig az értékeléssel telik. Trimeszterenként átlagosan további egy hét a közösség építésével, önálló tanulással  zajlik.

A ciklusok állandósága adja a tanulás irányításához szükséges kereteket. Ezek megtartásáért az egyes mikroiskolák tanulásszervezői felelnek, melynek működését a fenntartó monitorozza. A tanév ritmusát \aref{tbl:tanevritmus}. táblázat mutatja.


\begin{table}
  \centering
  \begin{tabular}{ l|l }
    \textbf{időszak} & \textbf{tevékenység}                \\
    \hline
    Szeptember       &
    közösségépítés                                         \\
                     & saját célok meghatározása           \\
                     & modulok kialakítása és meghirdetése
    \\ \hline

    Október          &
    tanulás, alkotás
    \\ \hline

    November         &
    tanulás, alkotás
    \\ \hline

    December         &
    portfólió frissítése                                   \\
                     & reflexiók                           \\
                     & visszajelzések                      \\
                     & célok felülvizsgálata               \\
                     & modulok változtatása igény esetén
    \\ \hline

    Január           &
    tanulás, alkotás                                       \\
    féléves értékelés kiadása
    \\ \hline

    Február          &
    tanulás, alkotás
    \\ \hline

    Március          &
    portfólió frissítése                                   \\
                     & reflexiók                           \\
                     & visszajelzések                      \\
                     & célok felülvizsgálata               \\
                     & modulok változtatása igény esetén
    \\ \hline

    Április          &
    tanulás, alkotás
    \\ \hline

    Május            &
    tanulás, alkotás
    \\ \hline

    Fél június       &
    évzárás, értékelés, bizonyítványok
  \end{tabular}
  \caption{Egy tanévben háromszor ismételjük a célállítás, tanulás, reflektálás ciklust.}
  \label{tbl:tanevritmus}
\end{table}

A tanév három periódusból áll: ez a felosztás követi az üzleti világ negyedéves tervezését, néhány egyetem trimeszterekre bontását, de leginkább az évszakokat. Minden periódus után értékeljük az elmúlt három hónapot, ünnepeljük az eredményeket, és megtervezzük a következő időszakot.  A trimesztereken belül az egyes mikroiskolák között lehetnek néhány hetes eltérések, melyek a közösség sajátosságait követik.

\paragraph{Féléves és évvégi elszámolás}
\label{sec:feleves_bontas}
A külső rendszerekkel és a törvényeknek való megfelelés miatt a kerettanterv két félévre bontva jeleníti meg a tananyag tartalmakat, és ezzel összhangban az iskola félévenkénti értékelést állít ki automatikusan a portfólió alapján, ami nem más, mint a portfólióba bekerült értékelések összegyűjtése.

A félév vége január vége\footnote{A félév végét miniszteri rendelet évente szabályozza.}, ami a második trimeszterbe esik, ezért a féléves értékelést az első trimeszter végén rögzített állapot szerint adja ki az iskola.  Az 5--12. évfolyamszinteken lévő gyereknek januárban módjuk és lehetőségük van a portfóliójuk frissítésére, ha a féléves értékelés és az osztályzatokra váltás eredménye számukra iskolaváltás, továbbtanulás vagy egyéb okból fontos. Ilyenkor januárban alkalmuk van felkészülni \aref{sec:evfolyamok_osztalyzatok}.~fejezetben részletezett osztályzatokra váltás folyamatra.

Év végén, június hónap fele marad az évvégi zárások és igény szerint az osztályzatokra váltásra való felkészülésre, ami a portfólió frissítését, bővítését, kiegészítését jelenti.