\section{Moduláris tanmenet és a tanulási eredmények}

\subsection{Modulok -- a tanulásszervezés alapegységei}
\label{sec:modulok}

A Budapest Schoolban a mindennapi tanulás specifikus modulok elvégzésén
keresztül történik ezzel biztosítva, hogy rugalmas keretek között,
pontosan megfogalmazott célok mentén, a gyerekek számára érthető,
átlátható és sajátnak megélt tartalommal történjen a tanulás.

A tanulási modulokat, vagyis a tanulás tartalmának és formájának
alapegységét a tanulásszervezők három kötelező összetevőből állítják
össze:

\begin{enumerate}
  \item
        a kerettanterv tantárgyainak tartalmából,
  \item
        a gyerekek, tanárok érdeklődéséből, aktuális tudásából,
  \item
        és a környezetük és a világ aktuális kihívásaiból.
\end{enumerate}

A három komponensből a legelső a legstatikusabb, hiszen a kerettanterv
meghatározza a tantárgyakat és azok tartalmát, valamint azt, hogy milyen
lehetséges eredmények elérését várjuk az ezekben való fejlődéstől. Az
egyes modulokban ezek személyre, illetve a csoport igényeire szabhatóak,
hiszen az elérhető eredményeket különféle gyakorlati és elméleti
tanulási módszerekkel el lehet érni.

A gyerekek és tanárok érdeklődése - ami a sajátként megélt cél és a
minél nagyobb fokú bevonódás alapfeltétele - irányíthatja a modulok
témáját, a projekteket, és a gyerekeknek sok egyéni tanulási idejét is
meghatározhatja.

Mindemellett a kerettanterv szándéka, hogy a tanárok, gyerekek
reagáljanak a környezetükre, a világ aktuális kihívásaira, kérdéseire. A
kerettanterv például meghatározza, hogy a gyereknek ,,\emph{táblázatkezelővel
  feladatot oldanak meg}'', az azonban a
modulok összeállításakor és a modulok elvégzése során derül ki, hogy a
gyerekek milyen táblázatokat szerkesztenek, hogy mi a tanulásuk belső
célja, motivációja. Nagyon hasonló táblázatkezelési képességeket lehet
fejleszteni, ha valaki az önvezető autók által csökkenő halálozási
arányról, vagy ha a vegánok számának és a GDP-növekedés alakulásának
arányáról készít táblázatot.

Ez is mutatja, hogy bár közösek a kereteink, végtelen az elképzelhető
modulok (a tanulási utak építőkövei, és így a különböző tanulási utak)
száma. Ezért tartja fontosabbnak a kerettanterv azt meghatározni, hogy
hogyan kell a modulokat létrehozni, mint felsorolni lehetséges
modulokat.

Az iskola fő célja, hogy képes legyen alkalmazkodni a menet közben
felmerülő tanulási igényekhez, és a tanulók saját céljait minél
rugalmasabban meg tudja szervezni és azokat az elérni kívánt tanulási
eredményekkel összhangba hozza.

A modulok tehát a tanulásszervezés alapegységei: olyan foglalkozások
megtervezett sorozata, amelyek során egy meghatározott időn belül a
gyerekek valamely képességüket fejlesztik, valamilyen ismeretet
elsajátítanak, vagy valamilyen produktumot létrehoznak. A modulok célja
sokféle lehet, de kötelező elvárás, hogy a résztvevők a portfóliójukba
bejegyzésre érdemes eredményt hozzanak létre, vagyis hogy legyen
egyértelmű célja.

Modulokba szerveződve az iskola tanulói tudnak például:

\begin{itemize}
  \item Produktum létrehozására szerveződő projektben részt venni

  \item Felfedezni, feltalálni, kutatni, vizsgálni, azaz kérdésekre választ
        keresni

  \item Egy jelenséget több nézőpontból megismerni

  \item Valamely képességüket, készségüket fejleszteni

  \item Adott vizsgára gyakorló feladatokkal felkészülni

  \item Közösségi programokban részt venni

  \item Az önismeretükkel, a tudatosságukkal, a testi-lelki jóllétükkel
        foglalkozni.
\end{itemize}

A modulokból álló rendszert azért kínáljuk a gyerekeknek, hogy egyszerre
adjunk átlátható struktúrát a tanulásnak, és kellő rugalmasságot is
biztosítsunk, hogy az egyéni, a közösségi és a társadalmi célok
harmóniába kerülhessenek.

\subsubsection{A modulok meghirdetése}

A modulok kiválasztása, felkínálása a mikroiskola tanulásszervezőinek
felelőssége, hiszen ők figyelnek és reagálnak a tanulók, szülők céljaira
és igényeire és a tantárgyi eredményelvárások nyomonkövetése is az ő
felelősségük. A modulok kidolgozáshoz és azok megtartásához külsős
embereket hívhatnak meg, de ebben az esetben is a tanulásszervező
tanárok felelőssége, hogy a modulok során milyen célok milyen módon
valósulnak meg.

A tanulásszervezők az egyes modulok tematikáját, azok hosszát és
feladatát a gyerekek tanulási céljainak megismerését követően és a
kerettantervben meghatározott tantárgyi tanulási eredményeket figyelembe
véve határozzák meg. A meghirdetett modulokból áll össze a tanulás
trimeszterenkénti tanulási rendje. A modulokba való csatlakozásról a
mentor, a szülő és a gyerek közösen dönt, mindig szem előtt tartva, hogy
folyamatos előrelépés legyen a már elért egyéni és tantárgyi
eredményekben is. Egy modul megkezdésének lehet feltétele egy korábbi
modul elvégzése, a gyerek képességszintje, a jelentkezők száma, és lehet
egyedüli feltétele a gyerekek érdeklődése. Egy modulvezető különféle
tematikájú modulokat tarthat függően attól, hogy a saját célok, a
tantárgyi eredmények mit kívánnak, és a tanulásszervezők, valamint a
modulvezetők kapacitása mit enged.

A gyerekek a felkínált modulokból maguk választják napi-
és hetirendjük tartalmát, a mentoraik és szüleik segítségével. Az a
tanulásszervezők felelőssége, hogy amikor egy tanuló moduljai
befejeződnek, és újat vesz fel, akkor az érdeklődési körének, tanulási
céljainak, és a soron következő, még el nem ért tantárgyi eredményekben
való fejlődéshez megfelelő modulok közül választhasson.

\subsubsection{A modulok formátuma}

Egy-egy modul hossza és foglalkozásainak gyakorisága változó lehet: az
egyszeri alkalomtól a teljes trimeszteren át tartó, heti 3-5
foglalkozást magában foglaló modul is lehetséges. Egy trimeszternél
azonban nem lehet hosszabb, és a lezárását az értékelés és az elért
eredmények portfólióba emelése követi. Egy modul folytatásaként a
következő trimeszterben új modult lehet meghirdetni.

A modulok nemcsak témájukban, céljaikban, időtartamukban, hanem
módszertanukban, folyamataikban is különbözhetnek: bizonyos modulokban a
felfedeztető (inquiry based) módszer, másokban az ismétlő (repetitív)
gyakorlás a célravezető. Így mindig a modul céljához, a tanárok és a
tanulók képességeihez és igényeihez választható a legjobb módszer.
Modulonként változhat, hogy a folyamatot a tanulók vagy a tanárok
befolyásolják-e, és milyen mértékben. Két szélsőség:

\begin{enumerate}
  \item Egy digitális kézműves modul célja, hogy építsünk valamit, ami
        programozható. Annak kitalálása, hogy mit és hogyan építünk, a tanulók
        feladata. Itt a modul vezetője csak támogatja a tanulás folyamatát, azaz
        \emph{facilitál}.

  \item Egy „A vizuális kommunikáció fejlődése a XX. század második felében"
        modul esetén a tanár előre eldönti, hogy mely alkotók, alkotások
        tartoznak szerinte a mindenképpen említendők sorába (a kánonba), és
        ezeket a gyerekekkel sorban végigveszi.
\end{enumerate}

\subsubsection{A modulok helyszíne}

A tanulás az egyes mikroiskolák helyszínén, vagy más Budapest School
mikroiskolákban, vagy a tanár által kiválasztott külső helyszínen,
esetenként pedig online, virtuális térben történik. A tanulásra, mint az
élethez szorosan kapcsolódó holisztikus fejlődési igényre tekintünk,
melynek jegyében az elsődleges szocializációs formától, a szülői,
családi környezettől sem akarjuk a tanulást leválasztani. Az élethosszig
tartó tanulás jegyében a tanulás tere az iskolai időszak után és az
iskola terein kívül is folytatódik.

A tanulók több ok miatt is tanulnak az iskolán kívül:

\begin{enumerate}
  \item Modulok szervezhetők külső helyszínekre, úgymint múzeumokba, erdei
        iskolákba, parkokba, vállalatokhoz, vagy tölthetik az idejüket „kint a
        társadalomban".

  \item Az önirányított tanulás okán a gyerekek otthon vagy külső
        helyszíneken is elvégezhetnek egy modult, amennyiben ez egyéni céljuk
        elérését nem veszélyezteti, és mentoruknak tudomása van arról, hogy a
        folyamatos fejlődés biztosított.
\end{enumerate}

Minden modul végeztével a tanulók és modulvezetők visszajelzést adnak
egymásnak, aminek része, hogy megosztják saját élményeiket, reflektálnak
a közös időre, összegyűjtik és értékelik az elért eredményt, és kitérnek
az esetleges fejlődési lehetőségekre.

\subsection{Tanulási eredmények -- a formális tanulás alapegységei}
\label{sec:tanulasi_eredmenyek}
A kerettanterv három interdiszcilináris tantárgyat jelöl meg, azok
témaköreit, tartalmát és követelményeit \emph{tanulási eredmények}
listájaként adja meg ezzel igazodva a Nkt. 5. § (5) pontjához. Tanulási
eredmény (learning outcome) lehet a kerettanterv szellemében minden
olyan tudás, képesség, kompetencia, attitűd amit a gyerek egy tanulási
folyamat során elsajátított és/vagy ezt demonstrálni tudja. Az eredmény
eléréséhez vezető út a modulokon keresztül történik, és a tanulás
folyamata történhet az iskolában vagy azon kívül, lehet formális,
non-formális vagy informális.

A tanulási eredmények több funkciót látnak el a kerettantervben.

\begin{itemize}

  \item Egy gyerek akkor halad egy tantárggyal, és ennek következtében akkor
        léphet évfolyamszintet, ha a tantárgyhoz tartozó
        tanulási  eredményeknek a kerettanterv jelen fejezetének
        végén meghatározott  részét elérte. Így a tanulási
        eredmények definiálják a tantárgyak  által meghatározott
        követelményeket.
  \item A tanulási eredmények a modulok (és így a mindennapokban szervezett
        foglalkozások, órák stb.) építő elemei. Egy-egy modul
        célját a  tanulásszervezők az elérendő tanulási
        eredmények  összeválogatásával és saját célokkal,
        érdeklődéssel való  kiegészítésével adják meg,
        figyelembe véve az életkori  sajátosságok, az egymásra
        épülés és az átjárhatóság  követelményeit.
  \item A tanulási eredmények megfeleltethetőek a miniszter által kiadott
        kerettantervek tantárgyai (és így a kötelező érettségi
        tárgyai )  és a NAT műveltségi területeivel, ami
        biztosítja, hogy a Budapest  School tanulója más
        rendszerben működő iskolába is illeszkedik.  Vagyis a
        tanulási eredmények a Budapest School saját
        interdiszciplináris tantárgyi elvárásain túl, az azokon belüli
        halmazt képző diszciplináris bontásban is követhetőek,
        így az  elért eredmények alapján mind a Budapest School
        tantárgyi  struktúrájával, mind (pl. egy esetleges
        iskolaváltás esetén) a  miniszter által kiadott
        kerettantervek tantárgystruktúrájával is
        megfeleltethetőek.
\end{itemize}

\subsection{Modulok és tanulási eredmények}

A gyerekek egyik feladata az iskolában, hogy tanulási eredményeket
érjenek el. Ennek színtere a moduláris tanmenet. A modulokat a tanárok
hozzák létre, és a modulok elvégzésével tanulási eredményeket lehet
elérni.

A gyerekek a modulok elvégzésével, vagy más tanulási helyzetekben
összegyűjtött tanulási eredményeik tényét a portfólióban rögzítik. A
mentor feladata, hogy folyamatosan kövesse, hogy megfelelő haladás
történik-e a portfólióban a tanulási eredmények és a saját célok
tekintetében. Az évfolyamszintlépés a portfólióban összegyűlt tanulási
eredmények alapján történhet meg.

A modul kecsegtet a gyerekek haladásához releváns tanulási
eredményekkel, a gyerekek által meghatározott saját célokkal és olyan
kimenettel, amely a portfólióban rögzíthető, legyen az egy alkotás, az
elért fizikai vagy szellemi eredmény dokumentációja, vagy egy értékelő
visszajelzés. A modulok tehát tartalmaznak tanulási eredményeket, az
önálló gondolkodás, szabad alkotás lehetőségét és teret engednek az
alkotásra, létrehozásra.

\paragraph{A modulok különféle tanulási eredmények elérését teszik
  elérhetővé}

Modulok tervezésekor és összeállításakor a tanulásszervezők a
modulvezetővel közösen határozzák meg a modul céljait, de azok
meghirdetéséért mindig a tanulásszervezők felelnek. A célok között fel
kell sorolni, hogy milyen tanulási eredmények elérését várhatják el a
gyerekek a modulon való részvételtől.

Például a 6-8 éves gyerekek számára megtervezett ,,\emph{3d nyomtató
  használata}'' modul során azon kívül, hogy megismerik a 3d nyomtatás
folyamatát, a modul célja, hogy a gyerekek számára elérhetővé tegye a
,,\emph{A kockát, téglatestet, gömböt felismeri, és képes létrehozni
  egyszerű módszerekkel. Ismeri ezeknek a testeknek a jellemzőit.}" (STEM
tantárgy, Matematika tématerület- 3-4 évfolyam) tanulási eredményt is.

Lehetőség van egy modul esetében több tantárgyból való tanulási eredmény
kiválasztására, ezzel biztosítva az interdiszciplinaritást, valamint a
Budapest School tantárgyi fejlesztési céljaihoz való integrált
kapcsolódást.

A tanulási eredmények egy időbeni egymásra épülést feltételeznek,
melyben azonban van lehetőség előre és hátrafele is lépkedni, előre,
amennyiben a modul meghirdetésekor az arra jelentkező gyerekcsoportnál a
megfelelő előkészítés megtörtént, hátra, amennyiben ezt
ismétlés/felzárkóztatás jelleggel szükségesnek ítéljük. Vagyis akkor
foglalkozzunk a 10 000-es számkörrel, ha a 100-as számkört már
begyakoroltuk. Az egymásra épülésért a modult meghirdető tanulásszervező
felel.

A példát folytatva a 3d nyomtató használata modul lehetővé teszi, hogy a
gyerek elérje a következő eredményeket is: \emph{,,Ismeri a számítógép
  részeinek és perifériáinak funkcióit, azokat önállóan használja.''}
(Harmónia, Informatika, 5-6 évfolyam), és a \emph{,,Használati utasításokat
  értő módon olvas és tart be."} (Harmónia, Életvitel, 3-4).

\paragraph{Új tanulási eredmények}

A gyerekek olyan tanulási eredményt is elérhetnek, ami a modulok céljai
között eredetileg nem volt megadva, mert

\begin{itemize}
  \item lehetőségük van egyénileg is tanulni;

  \item tanulási eredményekkel járnak a projektek, az iskolai lét, a közösségi
        élet és még számos informális és non formális tanulási helyzet;

  \item egy modul során is alakulhatnak előre nem tervezett helyzetek, amik
        hozzásegíthetik a gyerekeket tanulási eredmények eléréséhez.
\end{itemize}

Az újonnan létrejövő tanulási eredmények is bekerülnek a portfólióba.

\paragraph{Tanulási eredmények dokumentációja}

Minden modul dokumentálásra kerül, hogy annak célja, elért eredményei
nyilvánosak legyenek a Budapest School valamennyi mikroiskolája számára,
és ha szükséges, újra meg lehessen hirdetni. A tanulási eredmények egy,
a modulhoz kapcsolódó terv-tény összehasonlítás alapján kerülnek
meghatározásra. Az elért eredmények újra elérhetőek, amennyiben a
folyamatos fejlődés biztosítva van.

\paragraph{Egységes modulok egyedi alkalmazása}

Egy modul elvégzésével egy-egy gyerek más tanulási eredményt is elérhet.

\begin{itemize}
  \item
        Működhet a differenciálás, tehát nem minden gyerek ugyanazt és
        ugyanúgy csinálja a foglalkozásokon. Egy modulban tud
        együtt  tanulni az a gyerek, aki még ,,\emph{Ismeri az
          írott és nyomtatott  betűket,''} eredményért dolgozik,
        és az, aki ,,\emph{Jelöli helyesen a j  hangot 30--40
          begyakorolt szóban".}
  \item
        A modulnak része lehet testreszabható sáv. Például egy tudományos
        kísérletező modulban a néhány gyerek a rövidtávú
        memória és a	fáradtság kapcsolatáról kutat, a másik
        csoport az esőzés és a  közlekedési dugók kialakulása
        közti kapcsolatot vizsgálja. Minden  gyerek elérheti a
        ,,\emph{Valós folyamatokat képes elemzni a folyamathoz
          tartozó függvény grafikonja alapján.}'' (forrás, STEM), de a
        ,,\emph{Környezettudatos közlekedésszemlélet.}''
        (forrás, Harmónia)  eredményt is elérheti.
  \item
        Egy-egy gyerek saját célja kiterjesztése jegyében extra lépéseket
        tesz, és olyan eredményeket is el tud érni, amit mások
        nem.	Például egy modul végén önálló prezentációt, saját
        kutatási  tervet, vagy egy kész működő modellt alkothat.
\end{itemize}

\subsubsection{Kötelező tanulási eredmények}
\label{sec:kotelezo_tanulasi_eredmenyek}
A kerettanterv kötelező tanulási eredményként definiálja mindazokat az
eredményeket, melyek a kötelező érettségi tárgyak teljesítéséhez
szükségesek. Ezeket minden mikroiskola elérhetővé kell, hogy tegye a
gyerekek számára a modulok választékában.

Ezek az 1-4 évfolyamszinteken a miniszter által kiadott kerettantervek
Magyar nyelv és irodalom, Matematika, Idegen nyelv tantárgyakból
származó tanulási eredmények, és 5. évfolyamszinttől kiegészülnek a
történelem, társadalmi és állampolgári ismeretek tantárgyak alapján
létrehozott tanulási eredményekkel. További kötelező tanulási
eredményként jelennek meg 9. évfolyamtól a választott érettségi
tantárgyhoz kapcsolódó eredmények. Ezek a tanulási eredmények
megtalálhatók a kerettanterv három tantárgyának elérhető eredményei
között.

A kötelező tanulási eredmények 80\%-át le kell fednie a Budapest School
mikroiskoláiban meghirdetett moduloknak.

\paragraph{Tanulási eredmények kiegyenlítettsége}

Szintén fontos kötöttség, hogy a modulok meghirdetésénél a kerettanterv
három tantárgyából egyenlő súllyal (plusz/minusz 20\%) legyenek
elérhetőek tanulási eredmények. Emellett a kerettanterv minden
tématerületéről (vagyis a tantárgyak eredményeit alkotó diszciplinákból)
legalább 20\% tanulási eredményt kell választani, így biztosítva, hogy a
NAT minden műveltségterülete megjelenjen a tanárok által lefedett témák
között

\subsubsection{Monitorozás}

Kötelező elérni az eredményeket? Nem tudunk hatalmi szóval tanulásra
bírni gyereket, mert lehet, hogy annyira nem akarja, vagy nincs meg
hozzá a képessége. A kerettanterv a tanároknak ad keretet. Azonban a
fenntartó által üzemeltetett rendszerrel pontosan monitorozzuk a
haladást, és ha valaki a kötelező tanulási elemekkel nem halad, akkor
szólunk neki. Mivel a többség haladni fog, ezért előre tudjuk jelezni,
hogy le fog szakadni a többiektől, és túl nagy lesz az évfolyamszint
különbség közöttük. Ezekben az esetekben a mentor tanárnak és a
gyereknek reagálnia kell a helyzetre. A fenntartó által működtetett monitorozó
és minőségfejlesztő rendszerről \aref{sec:minosegbiztositas} fejezet ír
részletesen.