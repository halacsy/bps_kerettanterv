\ketoldaltkep{pics/2A_STEM.JPG}{pics/2B_KULT.JPG}
\section{Tanulási-tanítási egységek, a modulok}
\label{sec:modulok}

\emph{A miniszter által kiadott kerettantervek minden fejlesztési szakaszban két éves időintervallumra vonatkozóan fogalmazzák meg fejlesztési céljaikat, illetve a célok megvalósítását szolgáló közműveltségi tartalmakat. Ezért a tanmenetben évekre kell lebontani a tartalmak feldolgozásának lehetséges menetét, a fejlesztési célok és a rendelkezésre álló idő konkretizálásával. A tanmenet a pedagógus egyéni terve, ezért kötelezően nem írható elő annak tartalmi, illetve formai megvalósítása.}\citep{ofi:tanmenet} Az viszont előírható, hogy kisebb, célorintált egységekekből építsék fel a tanárok a tanmenetet. Ezek a tanulási-tanítási egységek, a tanmeneteket építő egységek, a modulok.

A \emph{tanulási-tanítási egységek}, Budapest School Modell szóhasználatában a \emph{modulok}, a tanulásszervezés \emph{alapegységei}: olyan foglalkozások megtervezett sorozata, amelyek során egy meghatározott időn belül a gyerekek valamely képességüket fejlesztik, valamilyen ismeretet elsajátítanak, vagy valamilyen produktumot létrehoznak. A tanítási egységek célja sokféle lehet, de kötelező elvárás, hogy a résztvevők a portfóliójukba bejegyzésre érdemes eredményt hozzanak létre, és hogy legyen egyértelmű céljuk.


      A Budapest School Modell a tanulási-tanítási egységet, tanulási-tanítási modul és a modul szavakat szinonímaként kezeli. Egyik megközelítésben a tantárgyak tanmenetekre, a tanmenetek tanítási egységekre, tematikai egységekre, majd tanórákra bontva alakul ki a gyerekek élményének megtervezése. Ezt hívhatjuk felülről-lefele (angolul top-down) tervezésnek. A Budapest School Modell ugyanezt az eredményt akarja elérni egy lentről-felfelé (bottom-up)tervezéssel, amikor a gyerek élménytől indulunk ki, és a tanulási-tanítási egységekből építjük újra fel a tantárgyak tartalmát, a tanulási eredményeket.

      A Budapest School Modell kihangsúlyozza a tanulási-tanítási egység, a modulok szerepét, ezért nevezi a tanmenetet \emph{moduláris tanmenetnek}. Végülis ugyanezt az eredményt éri el egy tanár, aki egy tantárgy tanóráinak egyes egységeire külön foglalkozásokat tervez és meghív külső foglalkozás tartókat besegíteni, mint az a tanár, aki tanulási-tanítási egységeket, modulokat tervez, és ezekből építi fel a tanévet.

Egy tanulási-tanítási egység több tantárgy tananyagtartalmát is lefedheti. Ahogy a Nemzeti Alaptanterv fogalmaz:
\begin{quote}
      \textit{A tudományok gyors fejlődése, a szükségletek új megjelenési formái és a világ új kihívásai (köztük a gyermekek testi-lelki egészségét veszélyeztető számos tényező) a megszokottól eltérő feladatok elé állítják az iskolát, a pedagógusképzést és a pedagógustovábbképzést. Olyan tudástartalmak jelentek meg, amelyek nehezen sorolhatók be a tudományok hagyományos rendszerébe, vagy amelyek egyszerre több tudományág illetékességébe tartoznak. Így megnőtt az igény egyrészt egyes hagyományos tantárgyak összevonására és/vagy tantárgyközi megjelenítésére, másrészt új tantárgyak/tantárgyegyüttesek kialakítására.}
\end{quote}

A mindennapi tanulás a tanulás-tanítási egységek, a modulok elvégzésén keresztül történik, ezzel biztosítva, hogy rugalmas keretek között, pontosan megfogalmazott célok mentén, a gyerekek számára érthető, átlátható és sajátnak megélt tartalommal történjen a tanulás.

A tanulási-tanítási egységeket, vagyis a tanulás tartalmának és formájának alapegységét a tanulásszervezők három kötelező összetevőből állítják ösz\-sze:

\begin{enumerate}
      \item
            a kerettanterv tantárgyainak tartalmából,
      \item
            a gyerekek, tanárok érdeklődéséből, aktuális tudásából,
      \item
            és a környezetük és a világ aktuális kihívásaiból.
\end{enumerate}

A három komponensből a legelső a legstatikusabb, hiszen a kerettanterv -- összhangban a NAT-tal -- meghatározza a tantárgyakat és azok tartalmát, valamint azt, hogy milyen lehetséges eredmények elérését várjuk az ezekben való fejlődéstől. Az egyes tanítási egységekban ezek személyre, illetve a csoport igényeire szabhatóak, hiszen az elérhető eredményeket különféle gyakorlati és elméleti tanulási módszerekkel el lehet érni.

A gyerekek és tanárok érdeklődése -- ami a sajátként megélt cél és a minél nagyobb fokú bevonódás alapfeltétele -- alakítja ki a tanítási egységek témáját, a projekteket, és a gyerekek egyéni tanulási idejét is meghatározhatja.

Mindemellett az iskola szándéka, hogy a tanárok, gyerekek reagáljanak a környezetükre, a világ aktuális kihívásaira, kérdéseire. A kerettanterv meghatározza például, hogy a gyerek ,,\emph{táblázatkezelővel feladatot old meg}''. Az azonban, hogy a gyerekek milyen táblázatokat szerkesztenek szívesen, csak a tanítási egységek összeállításakor és a tanítási egységek elvégzése során derül ki. Nagyon hasonló táblázatkezelési képességeket lehet fejleszteni, ha valaki az önvezető autóktól várt csökkenő baleseti halálozási arányról, vagy ha a vegánok számának és a GDP-növekedés alakulásának arányáról készít táblázatot.

A tanulási-tanítási egységekben épülő moduláris tanmenet fő célja, hogy egyszerre képes legyen alkalmazkodni a menet közben felmerülő tanulási igényekhez, adjon átlátható struktúrát a tanulásnak, és hogy a mikroiskola minél rugalmasabban tudja támogatni a tanulást úgy, hogy a saját, a közösségi és a társadalmi célok harmóniába kerülhessenek.

Ez is mutatja, hogy bár közösek a kereteink, végtelen az elképzelhető tanítási egységek (a tanulási utak építőkövei, és így a különböző tanulási utak) száma. Ezért tartja fontosabbnak a Budapest School Modell annak meghatározását, hogy hogyan kell a tanítási egységeket, a modulokat létrehozni, mint azt, hogy a tanítási egységeket, a modulokat tételesen felsorolja.

Egy-egy tanítási-tanulási egység, azaz egy modul során a gyerekek tudnak

\begin{itemize}
      \item vagy produktum létrehozására szerveződő projektben részt venni;

      \item vagy felfedezni, feltalálni, kutatni, vizsgálni, azaz kérdésekre választ keresni;

      \item vagy egy jelenséget több nézőpontból megismerni;

      \item vagy valamely képességüket, készségüket fejleszteni;

      \item vagy adott vizsgára gyakorló feladatokkal felkészülni;

      \item vagy közösségi programokban részt venni;

      \item vagy az önismeretükkel, a tudatosságukkal, a testi-lelki jóllétükkel fog\-lal\-kozni.
\end{itemize}

\subsection{A modulok meghirdetése}
\label{sec:tanítási egységek_meghirdetese}
A tanulási-tanítási egységek kiválasztása, felkínálása a tanulásszervezők feladata, hiszen ők figyelnek és reagálnak a gyerekek, szülők céljaira és igényeire. A meghirdetett tanítási egységekból áll össze a tanulás trimeszterenkénti tanulási rendje.

A tanulásszervezők az egyes tanulási-tanítási egységek tematikáját, hosszát és feladatát a gyerekek tanulási céljainak megismerését követően és a kerettantervben meghatározott tantárgyi tanulási eredményeket figyelembe véve határozzák meg.

A nem kötelező tanulási-tanítási egységekbe, modulokba való csatlakozásról a mentor, a szülő és a gyerek közösen dönt, mindig szem előtt tartva, hogy folyamatos előrelépés legyen a már elért egyéni és tantárgyi eredményekben is. Egy modul megkezdésének lehet feltétele egy korábbi modul elvégzése, a gyerek képességszintje, a jelentkezők száma, és lehet egyedüli feltétele a gyerekek érdeklődése.

Egy modulvezető különféle tematikájú tanítási egységeket tarthat függően attól, hogy a saját célok, a tantárgyi eredmények mit kívánnak, és a tanulásszervezők, valamint a modulvezetők kapacitása mit enged.

Amikor egy gyerek moduljai befejeződnek, és újat vesz fel, a tanulásszervező feladata a gyereket segíteni abban, hogy az érdeklődési körének, tanulási céljainak, és a soron következő, még el nem ért tantárgyi eredményekben való fejlődéshez megfelelő tanítási egységek közül választhasson.

A tanulásszervezők feladata a tantárgyi eredményelvárások nyomon követése is. A tanítási egységek kidolgozásához és azok megtartásához külsős szakembereket is meghívhatnak, azonban ilyenkor is a tanulásszervezők felelnek azért, hogy a tanítási egységekkal elérni kívánt tanulási célok teljesüljenek.

\subsection{A modulok egységek formátuma}

A moduláris rendszer elég nagy szabadságot ad a tanároknak abban, hogy hogyan szervezik a mindennapokat. Ezért is fontos, hogy már a modul meghirdetése előtt néhány szempont szerint kialakítsák a tanítási egységek kereteit.

\paragraph{Célok} Minden tanulási-tanítási egységnek, modulnak előre meg kell határozni a célját. A tanulási szerződések célkitűzéseihez hasonlóan itt is minél specifikusabban és mérhetőbben kell meg\-fo\-gal\-mazni a célokat. Javasolt az OKR  (Objectives and Key Results, azaz	Cél és Kulcs Eredmények) \citep{okr} vagy a SMART (Specific, Measurable, Achievable, Relevant, Time-bound, azaz Specifikus,  Mérhető, Elérhető, Releváns és Időhöz kötött) \citep{wiki:smart} technika alkalmazása, hogy minél specifikusabb, teljesíthetőbb, tervezhetőbb és könnyen mérhető célokat tűzzenek ki.

\paragraph{Értékelés} A tanulási-tanítási egység végén minden résztvevő személyes, több szem\-pont alap\-ján készült értékelést, visszajelzést kap a tanulási-tanítási egységgel kapcsolatos tevékenységére és elért eredményeire. A visszajelzés struktúráját előre meg kell határozni és még a modul kezdete előtt meg kell osztani a résztvevőkkel\footnote{Természetesen a visszajelzés szempontjai változhatnak a tanulási-tanítási egység során, ha változik a tanulási-tanítási egység tartalma, szempontjai. Ebben az esetben ezt mindenkinek nyilvánvalóvá kell tenni.}.

\paragraph{Óraszám} Egy-egy tanulási-tanítási egység hossza és a tanulási-tanítási egységhez kapcsolódó foglalkozások száma és gyakorisága változó: egy alkalomtól legfeljebb egy teljes trimeszteren keresztül tarthat. A tanulási-tanítási egység végén a tanulásszervező és a gyerek(ek) a modult lezárják, értékelik és az elért eredményeket rögzítik a (tanulási) portfólióban. Egy tanulási-tanítási egység folytatásaként a következő trimeszterben új tanulási-tanítási egységet alakítanak ki a tanárok.

\paragraph{Módszertan, formátum} A tanulási-tanítási egységek nemcsak témájukban,
céljaikban, időtartamukban, hanem módszertanukban, folyamataikban is különbözhetnek: bizonyos tanulási-tanítási egységekban a felfedeztető (inquiry based) módszer, másokban az ismétlő (repetitív) gyakorlás a célravezető. Így mindig a tanulási-tanítási egység céljához, a tanárok és a gyerekek képességeihez és igényeihez választható a legjobb módszer. Tanulási-tanítási egységekként változhat, hogy a folyamatot a gyerekek vagy a tanárok befolyásolják-e, és milyen mértékben. Két példa az eltérésre:

\begin{enumerate}
      \item Egy digitális kézműves tanulási-tanítási egység célja, hogy építsünk valamit, ami\linebreak programozható. Annak kitalálása, hogy mit és hogyan építünk, a gyerekek feladata. Itt a tanár nem csak támogatja a tanulás folyamatát, azaz \emph{facilitál}.

      \item Egy „\emph{A vizuális kommunikáció fejlődése a XX. század második felében}'' tanulási-tanítási egység esetén a tanár előre felépíti a tanmenetet, például hogy mely alkotók munkásságát, alkotásokat fogja bemutatni, és ezeket a gyerekekkel sorban végigveszi. Ilyenkor is bővülhet azonban a tematika a gyerekek érdeklődése, felvetései mentén.

\end{enumerate}

\subsection{A tanulási-tanítási egységek helyszíne}

A tanulás az egyes mikroiskolák helyszínén, egy másik Budapest School mikroiskolában, a tanár által kiválasztott külső helyszínen, vagy akár online, virtuális térben történik. A tanulásra úgy tekintünk, mint az élethez szorosan kapcsolódó holisztikus fejlődési igényre, melynek jegyében az elsődleges szocializációs tértől és formától, a szülői, családi környezettől sem akarjuk a tanulást leválasztani. Az élethosszig tartó tanulás jegyében a tanulás tere az iskolai időszak után és az iskola terein kívül is folytatódik.

A gyerekek több ok miatt is tanulnak az iskolán kívül:

\begin{enumerate}
      \item tanítási-tanítási egységek foglalkozásai szervezhetők külső helyszínekre, úgymint múzeumokba, erdei iskolákba, parkokba, vállalatokhoz, vagy tölthetik az idejüket „kint a társadalomban''.

      \item Amennyiben ez saját céljuk elérését nem veszélyezteti, és a folyamatos fejlődés biztosított, a mentoruk tudomásával a gyerekek az önirányított tanulás elvére figyelemmel a mikroiskolán kívüli egyéb helyszínen is elvégezhetnek egy tanulási-tanítási egységet.
\end{enumerate}

A tanulási-tanítási egység lezárásaként a gyerekek és tanárok visszajelzést adnak egymásnak, aminek része, hogy megosztják saját élményeiket, reflektálnak a közös időre, összegyűjtik és értékelik az elért eredményt, és kitérnek az esetleges fejlődési lehetőségekre.

\subsection{Külső tanárok aránya}
A pedagógiai program egy fontos megkötést ad a tanulási-tanítási egységek megtartására: a mikroiskola tanulásszervezőinek kell vezetnie a gyerekek moduljainak nagy részét, másképp fogalmazva korlátozva van a külsős, nem tanulásszervezők által tartott modulok óraszáma, ahogy ezt \aref{tbl:belso_modulok}.~táblázat mutatja.  Ennek a megkötésnek az az oka, hogy
\begin{itemize}
    \item Kisebb korban szeretnénk, ha kevesebb, állandóbb felnőttekkel találkoznának a gyerekek (két tanítós rendszer mintájára).
    \item Biztosan találkozzanak elég időt a gyerekekkel az iskolát vezető, irányító tanulásszervezők.
    \item Mindenképp legyen szoros kapcsolatuk a gyerekeknek pedagógus végzettséggel bíró tanárokkal.
    \item Érettségihez közeledve legyen lehetőség minél több külsős,
      akár
      speciális szaktudással bíró embertől tanulni.
\end{itemize}

\begin{table}[ht]
    \begin{center}
        \begin{tabular}{l|c|c|c|c|c}
            évfolyamszint                  & 1--2 & 3-4 & 5--8 & 9--11 & 12   \\ \hline
            ,,belsős'' modulok min. aránya & 70\% & 60  & 55\% & 50\%  & 40\%
        \end{tabular}
    \end{center}
    \caption{A mikroiskola tanulásszervezői által vezetett modulok aránya évfolyamszintenként.}
    \label{tbl:belso_modulok}
\end{table}

\subsection{Modulok nyomonkövetése}
Minden trimeszter megkezdésekor a tanulásszervező tanárok meghirdetik a kötelező, a kötelezően válaszható és a választható modulokat, azaz rögzítik, hogy
\begin{itemize}
    \item ki a modul vezetője, és melyik pedagógus munkakörben
      alkalmazott tanulásszervező felelős (elszámoltatható a
      RACI\footnote{https://www.pmsz.hu/hirek-aktualitasok/havi-mustra/havi-mustra-a-felelosseg-hozzarendelesi-matrixrol}
      me\-nedzs-\linebreak
      ment
      rendszer értelmezésében) a modulért;
    \item mi a modul célja, keretei és várható eredményei;
    \item hol, mikor és milyen rendszerességű foglalkozások lesznek;
    \item és mi a részvétel feltétele (előzetes tudás, nivó szint,
      kor, maximum létszám), azon belül, hogy teljes folyamatra kell-e
      elköteleződni,\linebreak vagy esetileg is lehet a modult látogatni.
\end{itemize}
Ezután eldöl, hogy ki mikor  melyik modulon vesz részt. Ennek rendszerét a tanárok alakítják ki: beoszthatják a gyerekeket, ahogy ők ezt jónak látják, vagy épp hagyatkozhatnak a gyerekek választására. A lényeg, hogy alakuljon ki a rendszer a trimeszter megkezdése előtt.

\paragraph{Nyomonkövethetőség}
Az őszi első trimeszterben  -- a kerettanterv szerint -- a mikroiskola ismerkedéssel kezd. Ebben a trimeszterben október 1.~a modulrendszer felállításának határideje. A második trimeszter esetén január 1., és tavasszal április 1. a határidő. Ezektől a határidőktől kezdve kialakuló rendszerben mindennap lehet tudni, hogy ki, hol, kivel, melyik modulok keretében tanul. Azaz kialakul a gyerekek órarendje, ami nagyon hasonló a megszokott órarendekhez: \emph{mikor melyik foglalkozáson és hol vagyok}.

A Budapest School iskolában annyi a különbség, hogy a személyreszabhatóság miatt az egy mikroiskolába járó gyerekek órarendje akár nagy mértékben is eltérhet egymástól. Tehát itt nem egy osztálynak és mikroiskolának van órarendje, hanem a gyerekeknek van saját órarendje. A fenntartó felelőssége kialakítani azt a számítógépes rendszert, ami a gyerekek órarendjét a gyerekek, szülők, tanárok, mikroiskolák és a teljes Budapest School iskola szintjén átláthatóvá és nyomonkövethető teszi.

\subsection{Inkrementális fejlesztés}
A moduláris rendszer nagy szabadságot enged a tanároknak abban, hogy a mindennapokat olyan tevékenység köré szervezzék, ami szerintük a gyerekek tanulását a legjobban szolgálja. Ez a szabadság bizonytalanságot is adhat: ha bármilyen modult szervezhetünk, akkor milyen modult szervezzünk? A pedagógiai program a tanároknak azt javasolja, hogy induljanak ki egy számukra ismert, stabil rendszerből, és azt fejlesszék lépésről lépésre.

Sokaknak a legbiztosabb alap a jól ismert rendszer: a miniszter által kiadott kerettantervek tantárgyaiból létrehozott modulok, amik követik a kiadott kerettanterv tematikus egységeit. A moduláris rendszer megengedi, hogy ugyanolyan tanórákat szervezzünk, mint a miniszter által kiadott kerettantervek alapján működő iskolák. Van, amikor innen indulva, lassan, trimeszterenként változtatva érhetjük el a legjobb eredményt: ezután össze lehet vonni tantárgyakat és egy modulba szervezni például a magyar nyelv és irodalmat és a történelmet egy trimeszterre, vagy az összes természettudományi tantárgyat egy kísérletezős modulba. Lehet tömbösítve vagy epochálisan szervezni a mindennapi tanulást.

A modulrendszer le tudja fedni a szakkörök, iskola utáni foglalkozások, nyári táborok rendszerét is. Egy mikroiskola a szokásos tantárgyi modulok mellé meghirdethet más iskolákban szakkörnek nevezett modulokat: robotika, néptánc, fociedzés. A modul rendszer sajátja, hogy ezeket a szakköröket ugyanúgy tudja kezelni, mint a történelem érettségi felkészítő fakultációt.

A modulrendszer lehetővé teszi a projektpedagógiával való szabad kísérletezést is. Lehet olyan modulrendszert alkotni, ahol minden páros héten projekteken dolgoznak a gyerekek, a páratlan héten pedig klasszikus tantárgyi struktúrák mentén szervezett modulokban haladnak az akadémiai tudás elsajátításával.

\paragraph{Biztonságos felfedezés}
A tanárok bátorítva vannak egy olyan saját, rugalmas struktúra kialakításában, amely jól működik és biztonságot nyújt mind számukra, mind pedig a gyerekek és a szülők számára. A Budapest School tanulásmonitoring rendszer miatt mindig tudjuk, hogy egy gyerek egy-egy tantárgy tanulási eredményeivel hogyan haladt. És ez biztonsági hálót ad a tanároknak: mindig tudjuk, hogy a gyerekek milyen irányban haladnak, lemaradtak-e valamiből, előre szaladtak-e valami másból. Egyben folyamatos visszajelzést ad a modulstruktúráról. Ezért mondhatjuk, hogy nyugodtan kereshetjük a jobb struktúrát, a tökéleteset sose tudjuk elérni (,,better, never the best'').
