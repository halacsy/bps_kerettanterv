\section{Modulok -- a tanulásszervezés alapegységei}
\label{sec:modulok}

A \emph{modulok} a tanulásszervezés \emph{alapegységei}: olyan foglalkozások megtervezett sorozata, amelyek során egy meghatározott időn belül a gyerekek valamely képességüket fejlesztik, valamilyen ismeretet elsajátítanak, vagy valamilyen produktumot létrehoznak. A modulok célja sokféle lehet, de kötelező elvárás, hogy a résztvevők a portfóliójukba bejegyzésre érdemes eredményt hozzanak létre, és hogy legyen egyértelmű céljuk.

A mindennapi tanulás a modulok elvégzésén keresztül történik, ezzel biztosítva, hogy rugalmas keretek között, pontosan megfogalmazott célok mentén, a gyerekek számára érthető, átlátható és sajátnak megélt tartalommal történjen a tanulás.

A tanulási modulokat, vagyis a tanulás tartalmának és formájának alapegységét a tanulásszervezők három kötelező összetevőből állítják össze:

\begin{enumerate}
      \item
            a kerettanterv tantárgyainak tartalmából,
      \item
            a gyerekek, tanárok érdeklődéséből, aktuális tudásából,
      \item
            és a környezetük és a világ aktuális kihívásaiból.
\end{enumerate}

A három komponensből a legelső a legstatikusabb, hiszen a kerettanterv -- összhangban a NAT-tal -- meghatározza a tantárgyakat és azok tartalmát, valamint azt, hogy milyen lehetséges eredmények elérését várjuk az ezekben való fejlődéstől. Az egyes modulokban ezek személyre, illetve a csoport igényeire szabhatóak, hiszen az elérhető eredményeket különféle gyakorlati és elméleti tanulási módszerekkel el lehet érni.

A gyerekek és tanárok érdeklődése -- ami a sajátként megélt cél és a minél nagyobb fokú bevonódás alapfeltétele -- alakítja ki a modulok témáját, a projekteket, és a gyerekek egyéni tanulási idejét is meghatározhatja.

Mindemellett a kerettanterv szándéka, hogy a tanárok, gyerekek reagáljanak a környezetükre, a világ aktuális kihívásaira, kérdéseire. A kerettanterv meghatározza például, hogy a gyerek ,,\emph{táblázatkezelővel feladatot old meg}''. Az azonban, hogy a gyerekek milyen táblázatokat szerkesztenek szívesen, csak a modulok összeállításakor és a modulok elvégzése során derül ki. Nagyon hasonló táblázatkezelési képességeket lehet fejleszteni, ha valaki az önvezető autóktól várt csökkenő baleseti halálozási arányról, vagy ha a vegánok számának és a GDP-növekedés alakulásának arányáról készít táblázatot.

A moduláris rendszer fő célja, hogy egyszerre képes legyen alkalmazkodni a menet közben felmerülő tanulási igényekhez, adjon átlátható struktúrát a tanulásnak, és hogy a mikroiskola minél rugalmasabban tudja támogatni a tanulást úgy, hogy a saját, a közösségi és a társadalmi célok harmóniába kerülhessenek.

Ez is mutatja, hogy bár közösek a kereteink, végtelen az elképzelhető modulok (a tanulási utak építőkövei, és így a különböző tanulási utak) száma. Ezért tartja fontosabbnak a kerettanterv annak meghatározását, hogy hogyan kell a modulokat létrehozni, mint azt, hogy a modulokat tételesen felsorolja.

Modulok során a gyerekek tudnak

\begin{itemize}
      \item produktum létrehozására szerveződő projektben részt venni;

      \item felfedezni, feltalálni, kutatni, vizsgálni, azaz kérdésekre választ keresni;

      \item egy jelenséget több nézőpontból megismerni;

      \item valamely képességüket, készségüket fejleszteni;

      \item adott vizsgára gyakorló feladatokkal felkészülni;

      \item közösségi programokban részt venni;

      \item az önismeretükkel, a tudatosságukkal, a testi-lelki jóllétükkel fog\-lal\-kozni.
\end{itemize}

\subsubsection{A modulok meghirdetése}
\label{sec:modulok_meghirdetese}
A modulok kiválasztása, felkínálása a tanulásszervezők feladata, hiszen ők figyelnek és reagálnak a gyerekek, szülők céljaira és igényeire. A meghirdetett modulokból áll össze a tanulás trimeszterenkénti tanulási rendje.

A tanulásszervezők az egyes modulok tematikáját, hosszát és feladatát a gyerekek tanulási céljainak megismerését követően és a kerettantervben meghatározott tantárgyi tanulási eredményeket figyelembe véve határozzák meg.

A nem kötelező modulokba való csatlakozásról a mentor, a szülő és a gyerek közösen dönt, mindig szem előtt tartva, hogy folyamatos előrelépés legyen a már elért egyéni és tantárgyi eredményekben is. Egy modul megkezdésének lehet feltétele egy korábbi modul elvégzése, a gyerek képességszintje, a jelentkezők száma, és lehet egyedüli feltétele a gyerekek érdeklődése.

Egy modulvezető különféle tematikájú modulokat tarthat függően attól, hogy a saját célok, a tantárgyi eredmények mit kívánnak, és a tanulásszervezők, valamint a modulvezetők kapacitása mit enged.

Amikor egy gyerek moduljai befejeződnek, és újat vesz fel, a tanulásszervező feladata a gyereket segíteni abban, hogy az érdeklődési körének, tanulási céljainak, és a soron következő, még el nem ért tantárgyi eredményekben való fejlődéshez megfelelő modulok közül választhasson.

A tanulásszervezők feladata a tantárgyi eredményelvárások nyomon követése is. A modulok kidolgozásához és azok megtartásához külsős szakembereket is meghívhatnak, azonban ilyenkor is a tanulásszervezők felelnek azért, hogy a modulokkal elérni kívánt tanulási célok teljesüljenek.

\subsubsection{A modulok formátuma}

A moduláris rendszer elég nagy szabadságot ad a tanároknak abban, hogy hogyan szervezik a mindennapokat. Ezért is fontos, hogy már a modul meghirdetése előtt néhány szempont szerint kialakítsák a modulok kereteit.

\paragraph{Célok} Minden modulnak előre meg kell határozni a célját. A tanulási szerződések célkitűzéseihez hasonlóan itt is minél specifikusabban és mérhetőbben kell meg\-fo\-gal\-mazni a célokat. Javasolt az OKR  (Objectives and Key Results, azaz	Cél és Kulcs Eredmények) \citep{okr} vagy a SMART (Specific, Measurable, Achievable, Relevant, Time-bound, azaz Specifikus,  Mérhető, Elérhető, Releváns és Időhöz kötött) \citep{wiki:smart} technika alkalmazása, hogy minél specifikusabb, teljesíthetőbb, tervezhetőbb és könnyen mérhető célokat tűzzenek ki.

\paragraph{Értékelés} A modul végén minden résztvevő személyes, több szem\-pont alap\-ján készült értékelést, visszajelzést kap a modullal kapcsolatos tevékenységére és elért eredményeire. A visszajelzés struktúráját előre meg kell határozni és még a modul kezdete előtt meg kell osztani a résztvevőkkel\footnote{Természetesen a visszajelzés szempontjai változhatnak a modul során, ha változik a modul tartalma, szempontjai. Ebben az esetben ezt mindenkinek nyilvánvalóvá kell tenni.}.


\paragraph{Óraszám} Egy-egy modul hossza és a modulhoz kapcsolódó foglalkozások száma és gyakorisága változó: egy alkalomtól legfeljebb egy teljes trimeszteren keresztül tarthat. A modul végén a tanulásszervező és a gyerek(ek) a modult lezárják, értékelik és az elért eredményeket rögzítik a (tanulási) portfólióban. Egy modul folytatásaként a következő trimeszterben új modult lehet meghirdetni.

\paragraph{Módszertan, formátum} A modulok nemcsak témájukban, céljaikban, időtartamukban, hanem módszertanukban, folyamataikban is különbözhetnek: bizonyos modulokban a felfedeztető (inquiry based) módszer, másokban az ismétlő (repetitív) gyakorlás a célravezető. Így mindig a modul céljához, a tanárok és a gyerekek képességeihez és igényeihez választható a legjobb módszer. Modulonként változhat, hogy a folyamatot a gyerekek vagy a tanárok befolyásolják-e, és milyen mértékben. Két példa az eltérésre:

\begin{enumerate}
      \item Egy digitális kézműves modul célja, hogy építsünk valamit, ami programozható. Annak kitalálása, hogy mit és hogyan építünk, a gyerekek feladata. Itt a modul vezetője csak támogatja a tanulás folyamatát, azaz \emph{facilitál}.

      \item Egy „\emph{A vizuális kommunikáció fejlődése a XX. század második felében}'' modul esetén a tanár előre felépíti a tanmenetet, például hogy mely alkotók munkásságát, alkotásokat fogja bemutatni, és ezeket a gyerekekkel sorban végigveszi. Ilyenkor is bővülhet azonban a tematika a gyerekek érdeklődése, felvetései mentén.

\end{enumerate}



\subsubsection{A modulok helyszíne}

A tanulás az egyes mikroiskolák helyszínén, egy másik Budapest School mikroiskolában, a tanár által kiválasztott külső helyszínen, vagy akár online, virtuális térben történik. A tanulásra úgy tekintünk, mint az élethez szorosan kapcsolódó holisztikus fejlődési igényre, melynek jegyében az elsődleges szocializációs tértől és formától, a szülői, családi környezettől sem akarjuk a tanulást leválasztani. Az élethosszig tartó tanulás jegyében a tanulás tere az iskolai időszak után és az iskola terein kívül is folytatódik.

A gyerekek több ok miatt is tanulnak az iskolán kívül:

\begin{enumerate}
      \item Modulok vagy modulok foglalkozásai szervezhetők külső helyszínekre, úgymint múzeumokba, erdei iskolákba, parkokba, vállalatokhoz, vagy tölthetik az idejüket „kint a társadalomban''.

      \item Amennyiben ez saját céljuk elérését nem veszélyezteti, és a folyamatos fejlődés biztosított, a mentoruk tudomásával a gyerekek az önirányított tanulás elvére figyelemmel a mikroiskolán kívüli egyéb helyszínen is elvégezhetnek egy modult.
\end{enumerate}

A modul lezárásaként a gyerekek és modulvezetők visszajelzést adnak egymásnak, aminek része, hogy megosztják saját élményeiket, reflektálnak a közös időre, összegyűjtik és értékelik az elért eredményt, és kitérnek az esetleges fejlődési lehetőségekre.
