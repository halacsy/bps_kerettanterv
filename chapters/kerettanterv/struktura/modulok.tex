\section{Moduláris tanmenet és a tanulási eredmények}

\subsection{Modulok -- a tanulásszervezés alapegységei}
\label{sec:modulok}

A \emph{modulok} a tanulásszervezés \emph{alapegységei}: olyan foglalkozások megtervezett sorozata, amelyek során egy meghatározott időn belül a gyerekek valamely képességüket fejlesztik, valamilyen ismeretet elsajátítanak, vagy valamilyen produktumot létrehoznak. A modulok célja sokféle lehet, de kötelező elvárás, hogy a résztvevők a portfóliójukba bejegyzésre érdemes eredményt hozzanak létre és hogy legyen egyértelmű célja.

A mindennapi tanulás a modulok elvégzésén keresztül történik, ezzel biztosítva, hogy rugalmas keretek között, pontosan megfogalmazott célok mentén, a gyerekek számára érthető, átlátható és sajátnak megélt tartalommal történjen a tanulás.

A tanulási modulokat, vagyis a tanulás tartalmának és formájának alapegységét a tanulásszervezők három kötelező összetevőből állítják össze:

\begin{enumerate}
      \item
            a kerettanterv tantárgyainak tartalmából,
      \item
            a gyerekek, tanárok érdeklődéséből, aktuális tudásából,
      \item
            és a környezetük és a világ aktuális kihívásaiból.
\end{enumerate}

A három komponensből a legelső a legstatikusabb, hiszen a kerettanterv -- összhangban a NAT-tal -- meghatározza a tantárgyakat és azok tartalmát, valamint azt, hogy milyen lehetséges eredmények elérését várjuk az ezekben való fejlődéstől. Az egyes modulokban ezek személyre, illetve a csoport igényeire szabhatóak, hiszen az elérhető eredményeket különféle gyakorlati és elméleti tanulási módszerekkel el lehet érni.

A gyerekek és tanárok érdeklődése -- ami a sajátként megélt cél és a minél nagyobb fokú bevonódás alapfeltétele -- alakítja ki a modulok témáját, a projekteket, és a gyerekek egyéni tanulási idejét is meghatározhatja.

Mindemellett a kerettanterv szándéka, hogy a tanárok, gyerekek reagáljanak a környezetükre, a világ aktuális kihívásaira, kérdéseire. A kerettanterv meghatározza például, hogy a gyerek ,,\emph{táblázatkezelővel feladatot old meg}''. Az azonban, hogy a gyerekek milyen táblázatokat szerkesztenek szívesen, csak a modulok összeállításakor és a modulok elvégzése során derül ki. Nagyon hasonló táblázatkezelési képességeket lehet fejleszteni, ha valaki az önvezető autóktól várt csökkenő baleseti halálozási arányról, vagy ha a vegánok számának és a GDP-növekedés alakulásának arányáról készít táblázatot.

A moduláris rendszer fő célja, hogy egyszerre képes legyen alkalmazkodni a menet közben felmerülő tanulási igényekhez, adjon átlátható struktúrát a tanulásnak, és hogy a mikroiskola minél rugalmasabban tudja támogatni\break a tanulást úgy, hogy a saját, a közösségi és a társadalmi célok harmóniába kerülhessenek.

Ez is mutatja, hogy bár közösek a kereteink, végtelen az elképzelhető modulok (a tanulási utak építőkövei, és így a különböző tanulási utak) száma. Ezért tartja fontosabbnak a kerettanterv annak meghatározását, hogy hogyan kell a modulokat létrehozni, mint azt, hogy a modulokat tételesen felsorolja.

Modulok során a gyerekek tudnak

\begin{itemize}
      \item produktum létrehozására szerveződő projektben részt venni;

      \item felfedezni, feltalálni, kutatni, vizsgálni, azaz kérdésekre választ keresni;

      \item egy jelenséget több nézőpontból megismerni;

      \item valamely képességüket, készségüket fejleszteni;

      \item adott vizsgára gyakorló feladatokkal felkészülni;

      \item közösségi programokban részt venni;

      \item az önismeretükkel, a tudatosságukkal, a testi-lelki jóllétükkel fog\-lal\-kozni.
\end{itemize}

\subsubsection{A modulok meghirdetése}
\label{sec:modulok_meghirdetese}
A modulok kiválasztása, felkínálása a tanulásszervezők feladata, hiszen ők figyelnek és reagálnak a gyerekek, szülők céljaira és igényeire. A meghirdetett modulokból áll össze a tanulás trimeszterenkénti tanulási rendje.

A tanulásszervezők az egyes modulok tematikáját, azok hosszát és feladatát a gyerekek tanulási céljainak megismerését követően és a kerettantervben meghatározott tantárgyi tanulási eredményeket figyelembe véve határozzák meg.

A nem kötelező modulokba való csatlakozásról a mentor, a szülő és a gyerek közösen dönt, mindig szem előtt tartva, hogy folyamatos előrelépés legyen a már elért egyéni és tantárgyi eredményekben is. Egy modul megkezdésének lehet feltétele egy korábbi modul elvégzése, a gyerek képességszintje, a jelentkezők száma, és lehet egyedüli feltétele a gyerekek érdeklődése.

Egy modulvezető különféle tematikájú modulokat tarthat függően attól, hogy a saját célok, a tantárgyi eredmények mit kívánnak, és a tanulásszervezők, valamint a modulvezetők kapacitása mit enged.

Amikor egy gyerek moduljai befejeződnek, és újat vesz fel, a tanulásszervező feladata a gyereket segíteni abban, hogy az érdeklődési körének, tanulási céljainak, és a soron következő, még el nem ért tantárgyi eredményekben való fejlődéshez megfelelő modulok közül választhasson.

A tanulásszervezők feladata a tantárgyi eredményelvárások nyomon követése is. A modulok kidolgozásához és azok megtartásához külsős szakembereket is meghívhatnak, azonban ilyenkor is a tanulásszervezők felelnek azért, hogy a modulokkal elérni kívánt tanulási célok teljesüljenek.

\subsubsection{A modulok formátuma}

A moduláris rendszer elég nagy szabadságot ad a tanároknak abban, hogy hogyan szervezik a mindennapokat. Ezért is fontos, hogy már a modul meghírdetése előtt néhány szempont szerint kialakítsák a modulok kereteit.

\paragraph{Célok} Minden modulnak előre meg kell határozni a célját. A tanulási szerződések célkitűzéseihez hasonlóan itt is minél specifikusabban, mérhetőbben kell megfogalmazni a célokat. Javasolt az OKR  (Objectives and Key Results, azaz	Cél és Kulcs Eredmények) \citep{okr} vagy a SMART (Specific, Measurable, Achievable, Relevant, Time-bound, azaz Specifikus,  Mérhető, Elérhető, Releváns és Időhöz kötött) \citep{wiki:smart} technika alkalmazása, hogy minél specifikusabb, teljesíthetőbb, tervezhetőbb és könnyen mérhető célokat tűzzenek ki.

\paragraph{Értékelés} A modul végén minden résztvevő személyes, többszempont alapján készült értékelést, visszajelzést kap a modullal kapcsolatos tevékenységére és elért eredményeire. A visszajelzés struktúráját előre meg kell határozni és még a modul kezdete előtt meg kell osztani a résztvevőkkel\footnote{Természetesen a visszajelzés szempontjai változhatnak a modul során, ha változik a modul tartalma, szempontjai. Ebben az esetben ezt mindenkinek nyilvánvalóvá kell tenni.}.


\paragraph{Óraszám} Egy-egy modul hossza és a modulhoz kapcsolódó foglalkozások száma és gyakorisága változó: egy alkalomtól legfeljebb egy teljesen trimeszteren keresztül tarthat. A modul végén a tanulásszervező és a gyerek(ek) a modult lezárják, értékelik és az elért eredményeket rögzítik a (tanulási) portfólióban. Egy modul folytatásaként a következő trimeszterben új modult lehet meghirdetni.

\paragraph{Módszertan, formátum} A modulok nemcsak témájukban, céljaikban, időtartamukban, hanem módszertanukban, folyamataikban is különbözhetnek: bizonyos modulokban a felfedeztető (inquiry based) módszer, másokban az ismétlő (repetitív) gyakorlás a célravezető. Így mindig a modul céljához, a tanárok és a gyerekek képességeihez és igényeihez választható a legjobb módszer. Modulonként változhat, hogy a folyamatot a gyerekek vagy a tanárok befolyásolják-e, és milyen mértékben. Két példa az eltérésre:

\begin{enumerate}
      \item Egy digitális kézműves modul célja, hogy építsünk valamit, ami programozható. Annak kitalálása, hogy mit és hogyan építünk, a gyerekek feladata. Itt a modul vezetője csak támogatja a tanulás folyamatát, azaz \emph{facilitál}.

      \item Egy „\emph{A vizuális kommunikáció fejlődése a XX. század második felében}'' modul esetén a tanár előre felépíti a tanmenetet, például hogy mely alkotók munkásságát, alkotásokat fogja bemutatni, és ezeket a gyerekekkel sorban végigveszi. Ilyenkor is bővülhet azonban a tematika a gyerekek érdeklődése, felvetései mentén.

\end{enumerate}



\subsubsection{A modulok helyszíne}

A tanulás az egyes mikroiskolák helyszínén, egy másik Budapest School mikroiskolában, a tanár által kiválasztott külső helyszínen, vagy akár online, virtuális térben történik. A tanulásra úgy tekintünk, mint az élethez szorosan kapcsolódó holisztikus fejlődési igényre, melynek jegyében az elsődleges szocializációs tértől és formától, a szülői, családi környezettől sem akarjuk a tanulást leválasztani. Az élethosszig tartó tanulás jegyében a tanulás tere az iskolai időszak után és az iskola terein kívül is folytatódik.

A gyerekek több ok miatt is tanulnak az iskolán kívül:

\begin{enumerate}
      \item Modulok vagy modulok foglalkozásai szervezhetők külső helyszínekre, úgymint múzeumokba, erdei iskolákba, parkokba, vállalatokhoz, vagy tölthetik az idejüket „kint a társadalomban''.

      \item Amennyiben ez saját céljuk elérését nem veszélyezteti, és a folyamatos fejlődés biztosított, a mentoruk tudomásával a gyerekek az önirányított tanulás elvére figyelemmel a mikroiskolán kívüli egyéb helyszínen is elvégezhetnek egy modult.
\end{enumerate}

A modul lezárásaként a gyerekek és modulvezetők visszajelzést adnak egymásnak, aminek része, hogy megosztják saját élményeiket, reflektálnak a közös időre, összegyűjtik és értékelik az elért eredményt, és kitérnek az esetleges fejlődési lehetőségekre.

\subsection{Tanulási eredmények -- a formális tanulás alapegységei}
\label{sec:tanulasi_eredmenyek}
A kerettanterv három interdiszciplináris és öt kiemelt tantárgyat jelöl meg, azok témaköreit, tartalmát és követelményeit \emph{tanulási eredmények} listájaként adja meg, ezzel igazodva az Nkt.~5.~§ (5) pontjához. Tanulási eredmény (learning out\-come) lehet a kerettanterv szellemében minden olyan tudás, képesség, kompetencia, attitűd, amit a gyerek egy tanulási folyamat során elsajátított és/vagy ezt demonstrálni tudja. Az eredmény eléréséhez vezető út a modulokon keresztül történik, és a tanulás folyamata történhet az iskolában vagy azon kívül, lehet formális, non-formális vagy informális.

A tanulási eredmények több funkciót látnak el a kerettantervben.

\begin{itemize}

      \item A kerettanterv évfolyamonként meghatározza az adott tantárgy teljesítéséhez elérendő tanulási eredményeket. Egy gyerek akkor  léphet egy tantárgyból évfolyamszintet, ha a tantárgyhoz tartozó követelményeket teljesítette.

      \item A tanulási eredmények a modulok (és így a mindennapokban szervezett foglalkozások, órák stb.) építőelemei. Egy-egy modul célját a  tanulásszervezők az elérendő tanulási eredmények	összeválogatásával és saját célokkal, érdeklődéssel való	kiegészítésével adják meg, figyelembe véve az életkori  sajátosságok, az egymásra épülés és az átjárhatóság  követelményeit.
      \item A tanulási eredmények megfeleltethetőek a miniszter által kiadott kerettantervek tantárgyai (és így a kötelező érettségi tárgyai )  és a NAT műveltségi területeivel, ami biztosítja, hogy a Budapest  School tanulója más rendszerben működő iskolába is illeszkedik, így az	elért eredmények alapján mind a Budapest School tantárgyi  struktúrájával, mind (például egy esetleges iskolaváltás esetén) a	miniszter által kiadott kerettantervek tantárgystruktúrájával megfeleltethetőek.
\end{itemize}

\subsection{Modulok és tanulási eredmények}
\label{sec:modulok_es_tanulasi_eredmenyek}
A gyerekek egyik feladata az iskolában, hogy tanulási eredményeket érjenek el. Ezt megtehetik a modulok elvégzésével, vagy más tanulási helyzetekben. A tanulási eredményeket a portfólióban rögzítik. A mentor feladata, hogy folyamatosan kövesse, hogy megfelelő haladás történik-e a portfólióban a tanulási eredmények és a saját célok tekintetében. Az évfolyamszintlépés a portfólióban összegyűlt tanulási eredmények alapján történhet meg.

A modul kecsegtet a gyerekek haladásához releváns tanulási eredményekkel, a gyerekek által meghatározott saját célokkal és olyan kimenettel, amely a portfólióban rögzíthető, legyen az egy alkotás, az elért fizikai vagy szellemi eredmény dokumentációja, vagy egy értékelő visszajelzés. A modulok tehát tartalmaznak tanulási eredményeket, az önálló gondolkodás, szabad alkotás lehetőségét, és teret engednek az alkotásra, létrehozásra.

\paragraph{A modulok különféle tanulási eredmények elérését teszik elérhetővé}

Modulok tervezésekor és összeállításakor a tanulásszervezők a modulvezetővel közösen határozzák meg a modul céljait, de azok meghirdetéséért mindig a tanulásszervezők felelnek. A célok között fel kell sorolni, hogy milyen tanulási eredmények elérését várhatják el a gyerekek a modulon való részvételtől.

Például a 6--8 éves gyerekek számára megtervezett ,,\emph{3d nyomtató használata}'' modul során azon kívül, hogy megismerik a 3d nyomtatás folyamatát, a modul célja, hogy a gyerekek számára elérhetővé tegye a ,,\emph{Kocka, téglatest jellemzőit ismeri, képes őket létrehozni.}'' (Matematika tantárgy, 4. évfolyam 2. félév) tanulási eredményt is.

Lehetőség van egy modul esetében több tantárgyból való tanulási eredmény kiválasztására, ezzel biztosítva az interdiszciplinaritást, valamint a Budapest School tantárgyi fejlesztési céljaihoz való integrált kapcsolódást.

A tanulási eredmények egy időbeni egymásra épülést feltételeznek, melyben azonban van lehetőség előre- és hátrafele is lépni. Előre, amennyiben a modul meghirdetésekor az arra jelentkező gyerekcsoportnál a megfelelő előkészítés megtörtént, hátra, amennyiben ezt ismétlés/felzárkóztatás jelleggel szükségesnek ítéli a mentor vagy a modult szervező, vezető. Vagyis akkor foglalkozzon egy gyerek a 10~000-es számkörrel, ha a 100-as számkört már begyakorolta. Az egymásra épülésért a modult meghirdető tanulásszervező felel. A példát folytatva a 3d nyomtató használata modul lehetővé teszi, hogy a gyerek elérje a következő eredményeket is: \emph{,,Ismeri a számítógép
      részeinek és perifériáinak funkcióit, azokat önállóan használja.''}
(Harmónia, Informatika, 5. évfolyam 1. félév), és  \emph{,,Használati utasításokat
      értő módon olvas és tart be.''} (Harmónia, Életvitel, 4. évfolyam 2. félév)

\paragraph{Új tanulási eredmények}

A gyerekek olyan tanulási eredményt is elérhetnek, ami a modulok céljai között eredetileg nem volt megadva, mert

\begin{itemize}
      \item lehetőségük van egyénileg is tanulni;

      \item tanulási eredményekkel járnak a projektek, az iskolai lét, a közösségi élet és még számos informális és non-formális tanulási helyzet;

      \item egy modul során is alakulhatnak előre nem tervezett helyzetek, amik hozzásegíthetik a gyerekeket tanulási eredmények eléréséhez.
\end{itemize}

Az újonnan létrejövő tanulási eredmények is bekerülnek a portfólióba.

\paragraph{Tanulási eredmények dokumentációja}

Minden modul dokumentálásra kerül, hogy annak célja, elért eredményei nyilvánosak legyenek a Budapest School valamennyi mikroiskolája számára, és ha szükséges, újra meg lehessen hirdetni. A tanulási eredmények egy, a modulhoz kapcsolódó terv-tény összehasonlítás alapján kerülnek meghatározásra. Az elért eredmények újra elérhetőek, amennyiben a folyamatos fejlődés biztosítva van.

\paragraph{Egységes modulok egyedi alkalmazása}

Egy modul elvégzésével egy-egy gyerek más tanulási eredményt is elérhet.

\begin{itemize}
      \item
            Működhet a differenciálás, tehát nem minden gyerek ugyanazt és\break ugyanúgy csinálja a foglalkozásokon. Egy modulban tud együtt	tanulni az a gyerek, aki még ,,\emph{Ismeri az írott és nyomtatott  betűket''} eredményért dolgozik, és az, aki ,,\emph{Jelöli helyesen a j	hangot 30--40 begyakorolt szóban''.}
      \item
            A modulnak része lehet testre szabható sáv. Például egy tudományos kísérletező modulban néhány gyerek a rövid távú memória és a	fáradtság kapcsolatáról kutat, a másik csoport az esőzés és a	közlekedési dugók kialakulása közti kapcsolatot vizsgálja. Minden  gyerek elérheti a ,,\emph{valós folyamatokat képes elemezni a folyamathoz tartozó függvény grafikonja alapján}''  (forrás, Matematika) eredményt, de a ,,\emph{környezettudatos közlekedésszemlélet}'' (forrás, Harmónia)	eredményt is elérheti.
      \item
            Egy-egy gyerek saját tanulási célja érdekében extra lépéseket tehet, és olyan eredményeket is el tud érni, amit mások nem.	Például egy modul végén önálló prezentációt, saját kutatási  tervet, vagy egy kész működő modellt alkothat.
\end{itemize}

\subsubsection{Kötelező tanulási eredmények}
\label{sec:kotelezo_tanulasi_eredmenyek}
A kerettanterv kötelező tanulási eredményként definiálja mindazokat az eredményeket, melyek a kötelező érettségi tárgyak teljesítéséhez szükségesek. Ezeket minden mikroiskola elérhetővé kell hogy tegye a gyerekek számára a modulok választékában.

Ezek az 1--4. évfolyamszinteken a miniszter által kiadott kerettantervek \emph{Magyar nyelv és irodalom}, \emph{Matematika}, \emph{Idegen nyelv} tantárgyakból származó tantárgyak tanulási eredményei, és 5.~évfolyamszinttől a \emph{történelem, társadalmi és állampolgári ismeretek} tantárgy eredményeivel. További kötelező tanulási eredményként jelennek meg 9.~évfolyamtól a választott érettségi tantárgyhoz kapcsolódó eredmények. Ezek a tanulási eredmények megtalálhatók a kerettanterv három tantárgyának elérhető eredményei között.

A mikroiskolában meghirdetett moduloknak a kötelező tanulási eredmények 80\%-át le kell fednie.

\paragraph{Kötelező modulok}
A kerettanterv és a pedagógiai program is előírhat kötelező modulokat a mikroiskolák számára. Ilyenek például a 11.~évfolyamszinten belépő érettségire felkészítő modulok (ld.~\ref{sec:erettsegi}.~fejezet), a minden mikroiskolára egységes pedagógiai program tetszőleges kötelező modult írhat elő. Így lehet biztosítani a kötelező tartalmi elemek és foglalkozás -- például elsősegélynyújtás vagy a nemzetiségiekkel való ismerkedés -- elérhetőségét.

\subsubsection{Monitorozás}

Kötelező elérni az eredményeket? Nem tudunk hatalmi szóval tanulásra bírni gyereket, mert lehet, hogy annyira nem akarja, vagy nincs meg hozzá a képessége. A kerettanterv a tanároknak ad keretet. Azonban a fenntartó által üzemeltetett rendszerrel az iskola  monitorozza a haladást, és ha valaki a kötelező tanulási elemekkel nem halad, akkor az iskola erre felhívja a figyelmét. Mivel a többség haladni fog, ezért előre tudja az iskola jelezni, hogy le fog szakadni a többiektől, és túl nagy lesz az évfolyamszint-különbség közöttük. Ezekben az esetekben a mentortanárnak, a gyereknek és a szülőnek reagálnia kell a helyzetre. A fenntartó által működtetett monitorozó és minőségfejlesztő rendszerről \aref{sec:minosegbiztositas} fejezet ír részletesen.