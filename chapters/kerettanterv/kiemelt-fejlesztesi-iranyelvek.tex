\section{Kiemelt fejlesztési irányelvek}
\label{sec:kiemelt_fejlesztesi_iranyelvek}
A Budapest School interdiszciplináris tanulási modulok formájában szervezi az iskolai tanórákat. Ezen modulok tartalmi alapjai egy az egyben megfelelnek a miniszter által kiadott tantárgyi struktúrának. Minden modul a miniszter által kiadott kerettantervek egyes, vagy adott esetben több eredményének elérését tűzi ki célul. A tantárgyak tehát a tanulás tartalmi vázát jelölik ki, a tantárgyi struktúra pedig a hozzátartozó óraszámokkal a tanulási eredmények közötti egyensúlyt adja. 

A modulok eredményei a tantárgyakhoz a portfólión keresztül kapcsolódnak, a céljuk pedig, hogy a gyerekek folyamatosan fejlődjenek a világ tudományos megismerésében (STEM), a saját és mások kulturális közegéhez való kapcsolódásban (KULT), valamint a testi-lelki egyensúlyuk fenntartásában (Harmónia), vagyis a \emph{kiemelt tantárgyközi fejlesztési irányelvekben}. A modulok fejlesztési irányelvei útmutatóul szolgálnak a tanároknak arra, hogy az elérendő eredményekhez milyen elvek mentén szervezzenek modulokat. 

Olyan diszciplinák kerültek a fejlesztési irányelvekben összevonásra, melyek a tantárgyi struktúrára lebontva is értelmezhetőek, és melyek segítik a moduloknak a mai igényekhez való megfeleltetését. Egy modul egyszerre több fejlesztési irányelvnek is megfelelhet. Gépesíthetünk úgy egy modul során egy ökológiai tangazdaságot (STEM), hogy közben megismerjük a gazdálkodás történelmi és szocio-kulturális hátterét (KULT) és a szabad levegőn végzett munkával a testünket és a lelkünket egyaránt ápoljuk.   


\subsection[STEM]{Természet- és mérnöki	tudomány, technológia és matematika	(STEM)}\sectionmark{STEM}
\emph{A fejlesztési irányelv a miniszter által kiadott kerettanterv következő tantárgyainak fejlesztési területeire, elvárásaira épül, és azt egészíti ki a saját célokkal: biológia, egészségtan, fizika, kémia, matematika, informatika, földrajz, környezetismeret, természetismeret.}

A STEM ma már nem csak néhány tantárgy összevonása és  mozaikszóban való emlegetése miatt releváns. Az interdiszciplináris tudós-mérnöki szemlélet, a fejlesztő-kutató munka, az innováció az élet minden területén megjelenik, és nem csak eldugott laborok mérnökeinek a feladata új dolgokat felfedezni, kipróbálni, alkotni. Ezt az is mutatja, hogy a világ legnagyobb mértékben bővülő munkáinak több mint fele STEM-alapú, és ezek száma a következő évtizedekben a kutatások szerint nőni fog. A STEM a jövő nagy kérdéseit hivatott megválaszolni. Épp ezért a STEM a felfedezés és a megismerés iránti vágyat kell hogy ébren tartsa és tovább mélyítse a gyerekekben, miközben gyakorlatias alkotó módszerével fejleszti a csapatmunkát és a kitartást, valamint azt, hogy a megszerzett ismereteket új, korábban nem tapasztalt helyzetekben is alkalmazni tudják. Ez az a szemléletmód, ami egyaránt segítheti az egyén akadémiai tudásának fejlődését és a tanulás iránti hosszú távú elköteleződését.

\paragraph{STEM fejlesztési célok}
\begin{itemize}
  \item Problémamegoldási képesség fejlesztése
  \item  Matematikai gondolkodás fejlesztése
  \item  Logikus gondolkodás fejlesztése
  \item  A világ működésének mélyebb megértése a természettudományos szemléleten keresztül
  \item  Innovatív fejlesztésekhez szükséges technológiai (kódolás) és gyakorlati (maker) eszköztár megismerése és alkalmazása
\end{itemize}

A STEM komplex, tudományterületeken és a technológiai lehetőségeken átívelő megközelítése sokfelé ágazik, mégis vannak olyan alapszabályai, amelyek a tanulás során mindig jellemzik. Ahhoz, hogy folyamatos fejlődés legyen, fontos, hogy az aktuális képességekhez igazodó, nyitott, tényeken alapuló, kutatásalapú, diverz módszertanú, skálázható és releváns módon történjen a STEM területen a tanulás.

A STEM irányelv a fenti célok eléréséhez a következő egymást segítő komponensek alkalmazása javasolt:

\paragraph{Gyakorlatias tanulás elkötelezett és együttműködő közösségek hálózatához kapcsolódva}
A STEM tanulás során különösen fontos, hogy az adott területet jól ismerő modulvezetőn és a mentoron túl további innovatív és segítő közegek is támogassák a gyerekek fejlődését. Sokat hozzátehet a mindennapi élethez való kapcsolódás kialakításában, ha intézményi (pl múzeum, kutatóintézet) vagy technológiai, ipari és más STEM-hez kapcsolódó területen működő cégeknél dolgozó emberek mintákat mutatnak, vagy akár bekapcsolódnak a gyerekek mindennapos munkájába.

\paragraph{Hozzáférhető tanulási gyakorlatok, amelyek a játék és a kockáztatás eszközeit is felhasználják} A STEM tematikájú játékok segítik a gyerekeket
abban, hogy egymástól és együtt is tanuljanak, ezzel fejlesztik a kreativitásukat és a felfedezés iránti vágyukat. Ezek a felfedeztető játékok ráirányítják a fókuszt a csapatmunka fontosságára és arra, hogy miként lehet mai problémákra és kihívásokra válaszokat találni.

\paragraph{Multidiszciplináris tanulási élmények, amelyek a világ nagy kihívásaira keresnek válaszokat} A STEM legfőbb kihívása, hogy a mai kor olyan
kérdésfelvetéseit vizsgálja, amelyekre közösségi, nemzeti vagy globális szinten még nincsenek meg a válaszok. Ezek lehetnek a vízgazdálkodással, az agykutatással és annak a prevencióra, vagy épp a gyógyulásra gyakorolt hatásaival, a megújulóenergia-gazdálkodással, vagy épp az önvezető autók új generációjának technológiai irányaival kapcsolatosak. Ezek a kihívások egyúttal azt is megmutatják, hogy mely kérdések lehetnek a kultúránk szempontjából relevánsak.

\paragraph{Rugalmas és befogadó tanulási környezet} Nagyban hozzájárulhatnak a
STEM élményszerű tanulásához a nyitottan értelmezett tanulási környezetek. Az iskolai terek alkotótérré változtatásával, a természeti közegekben tett terepmunkák, vagy a korszerű technológiai platformok, mint például a VR (virtuális valóság – virtual reality) vagy AR (kiterjesztett virtuális valóság – augmented virtual reality) bevonásával a területek könnyebben megismerhetővé válnak, és újabb kérdések feltevésére sarkallhatják a gyerekeket, miközben az irányító tanárszerep helyett a kísérletező, segítő és facilitátor feladatok erősödnek.

\paragraph{A tanulási eredmények mérésének új eszköztára} A teljesítmény, a
kutatás, a kísérletezés és az alkotás kiemelt fókuszú a gyerekek STEM tanulási útja során. Kiemelt szerepet kapnak ezért a saját kutatásokra épülő prezentációk, megfigyelések, és az azokra adott értékes visszajelzések.

\subsection[KULT]{Kultúra, társadalom, kommunikáció és művészetek} (KULT)
\emph{A fejlesztési irányelv a miniszter által kiadott kerettanterv következő tantárgyainak fejlesztési területeire, elvárásaira épül, és azt egészíti ki a saját célokkal: magyar nyelv és irodalom, idegen nyelv, vizuális kultúra, dráma és tánc, hon- és népismeret, történelem, társadalmi és állampolgársági ismeretek, ének-zene.}

A művészetek és önmagunk kifejezése az őskortól segítik az embereket a túlélésben. A képzelet absztrakciós szintjének alkalmazása az egyik legfontosabb faji tulajdonsága az embernek. Ennek fejlesztése pedig globális világunk egyik legnagyobb kihívása. Ahhoz, hogy a társadalmi viták és szabályozások aktív alakítójává válhassunk, hogy megismerjük a globális világunk legnagyobb kihívásait, értenünk kell a múltunkat, a saját kultúránkat és a kultúrák szerepét általában, és ki kell tudnunk fejezni magunkat olykor szavakkal, máskor a művészet eszközeivel.

Még soha nem élt az emberiség ennyire behálózott világban, és még soha nem kellett ennyire tudatosan készülnünk arra, hogy gyerekeinknek többféle kultúrát, társadalmi hálózatot kell megérteniük, és abban eligazodniuk. Családok, munkahelyi és lakókörnyezeteink, sőt még a nemzeti és az azokat átívelő társadalmi környezetek is gyorsabban változnak ma, mint szüleink életében. Ezért szeretnénk, hogy gyerekeink stabil identitásukra építkezve képesek legyenek emberként emberekhez kapcsolódni, embertársaikat megérteni, velük együtt élni, dolgozni.

A tanulás egyik legfőbb funkciója, hogy képessé váljunk egy fenntartható élet kialakítására. Ehhez pedig önmagunk, környezetünk (Harmónia) és a világ működésének megismerésén (STEM) és fejlesztésén túl a szűk és tágabb értelemben vett kulturális tereinkhez is kapcsolódnunk kell (KULT). Meg kell ismernünk a lokális és a globális kihívásokat ahhoz, hogy értő, empatikus módon kapcsolódhassunk a saját és más kultúrákhoz, és a fenntartható fejlődés alakítójává válhassunk. A Budapest School KULT tantárgy fő célja ezért az olyan globális kompetenciák fejlesztése, amelyek a fenti cél elérését támogatják.

A KULT irányelv a következő fejlesztési területeket foglalja magába:
\begin{itemize}
  \item írott és beszélt anyanyelvi kommunikáció

  \item írott és beszélt idegen nyelvi kommunikáció

  \item mai lokális és globális kihívások megismerése a múlt kontextusában

  \item kulturális diverzitás és a kultúra mint az emberi viselkedést leíró eszköztár megismerése

  \item művészetek stílus- és formavilága

  \item művészetek mint önkifejezési eszköz a vizuális (hagyományos képzőművészetektől a digitális művészetekig) és előadó-művészetekben (dráma, tánc, zene)
\end{itemize}

Miközben világunkban egyre nő a technológia szerepe, folyamatosan nő az igény arra is, hogy képesek legyünk értelmezni és megfelelően használni az elénk táruló információt. Ennek alapja a megfelelő szövegértés, a nyelv mint írott és verbális kommunikációs stratégiának az egyéni és korosztályi képességekhez mért alkalmazása, valamint a művészeteken keresztül a kifejezés egyéb módjainak elsajátítása. Ahhoz, hogy képessé váljunk erre, megfelelő történelmi és kulturális kontextusba kell helyeznünk az elénk táruló információt, és nyitottnak kell lennünk arra, hogy ezeket befogadjuk, és a mai kor globális kihívásaihoz képest értelmezhessük.

A KULT komplex tanulási struktúrája adja meg az új információk elmélyülésének alapjait, és segít abban, hogy a mindennapos döntési stratégiáink részévé váljanak.

A KULT az alábbi képességeket fejleszti:
\begin{itemize}
  \item Empátia és együttérzés

  \item Egyéni és közösségi hiedelmek megismerése

  \item Kritikai gondolkodás

  \item A művészetek és az irodalom funkciójának megértése és a mindennapi életbe való beépítése

  \item Az etika és a morál alapkérdéseinek felfedezése és megkérdőjelezése

  \item Ismeretlen vagy komfortzónán túli ötletek befogadása

  \item Az emberiség kulturális örökségének és aktuális diverzitásának megismerése

  \item Többsíkú gondolkodás, multidiszciplináris szövegértés
\end{itemize}

A KULT irányelv a fenti célok eléréséhez a következő egymást segítő komponensek alkalmazását javasolja:

\paragraph{Helyi közösségekhez való kapcsolódás}

Tanulás során a lokális közösségek szerepe megnő, hiszen mind a saját kultúra és művészeti értékek megismerése, mind a nyelvhasználat tekintetében döntő szerepe van annak, hogy saját környezetünkben  tudjuk ezeket érteni és alkalmazni.

\paragraph{Kortárs kihívások}

A KULT alapját a kortárs művészeti értékek, a kortárs problémák és kihívások adják. Ezek ugyan mindig a múlt és a környezet kontextusában értelmezhetők, de elsődleges módszer a kapcsolat kialakítása a mai világ alkotásaival, szövegeivel, kulturális értékeivel.

\paragraph{Multidiszciplináris tanulási élmények, amelyek a világ nagy kihívásaira keresnek válaszokat}

A KULT abban segít a gyerekeknek, hogy jobban eligazodjanak a mai és holnapi világban, és olyan kérdéseket tegyenek fel, amelyek ma még megválaszolatlanok, és a jövőt formálhatják. Ehhez a művészetek, nyelvek és kultúrák találkozási pontjait, az átmeneteket is célszerű szemlélni.

\paragraph{Rugalmas és befogadó tanulási környezet}

A mai szövegek értelmezése, az online elérhető tartalmakban való tájékozódás képessége éppoly fontos, mint az idegen nyelveken történő kommunikáció elősegítése a világhoz történő kapcsolódással. Az aktuális globális kihívásokat a múltunkon keresztül érthetjük meg, amihez tanulási környezetünket folyamatosan a megismerés igényeihez képest kell formálnunk, és olyan eszközöket kell használnunk, amelyek az értelmezést segíthetik.

\paragraph{Alkotói szabadság és az alkotó szabad értelmezése}

Az önálló alkotói munka és az alkotások szabad értelmezése és befogadása az innováció és a kreativitás alapja. Ennek biztosításához a hagyományos keretek újraértelmezésére van szükség és arra, hogy egy alkotói folyamatban a saját cél kibontakozásának támogatása megtörténjen.

\paragraph{A tanulási eredmények mérésének új eszköztára}

A kommunikáció lehetőségei iránti elköteleződés, az önkifejezés és a környezetünkhöz való kapcsolódás különféle módjainak alkalmazása különösen fontos módszer. Az alkotói munka eredményei mérhetők a saját fejlődési ütemhez képest. Hasonlóképp az írott és szóbeli kifejezés, valamint a prezentáció egyéni és csoportos módjai is megfelelő mérési eszközök.

\subsection[Harmónia]{Harmónia (fizikai, lelki jóllét és kapcsolódás a környezethez)}
\emph{A fejlesztési irányelv a miniszter által kiadott kerettanterv következő tantárgyainak fejlesztési területeire, elvárásaira épül, és azt egészíti ki a saját célokkal: erkölcstan, testnevelés, technika, életvitel és gyakorlat, informatika.}

Testi és lelki egészségünk az alapja annak, hogy tanulhassunk, fejlődhessünk, és a saját életünk alakítóivá váljunk. Életünk fizikai, lelki, érzelmi és társas aspektusai határozzák meg a kapcsolódásunkat önmagunkhoz, társainkhoz és az őket körülvevő emberekhez, vagyis a tágabb értelemben vett társadalomhoz. Ez segít abban, hogy önálló döntéseket hozzunk, és jól tudjunk együtt tanulni, dolgozni csoportokban. Ahhoz, hogy a közösségünk részeként harmóniában élhessünk önmagunkkal, az épített és természeti környezetünkhöz is kapcsolódnunk kell.

A Budapest School tanulási koncepciójának középpontjában az egyén, mint a közösség jól funkcionáló, saját célokkal rendelkező tagja áll. Az iskolában való fejlődése során elsősorban azt tanulja, hogy miként tud specifikált saját célokat megfogalmazni, és hogyan tudja ezeket elérni. Ebben a folyamatban egy mentor segíti a munkáját az iskola kezdetétől a végéig. Ő figyel arra, hogy a gyerek fizikai és lelki biztonsága és fejlődése folyamatos legyen, és segíti azokban a helyzetekben, amikor biztonságérzete vagy stabilitása csökken.

A közösségben jól funkcionáló egyén belső harmóniájához ez a tantárgy a következő fejlesztési területeket határozza meg:
\begin{itemize}
  \item Érzelmi és társas intelligencia

  \item Önismeret és önbizalom

  \item Konfliktuskezelés

  \item Rugalmasság (reziliencia)

  \item Kritikai gondolkodás

  \item Közösségi szabályok alkotásában való részvétel és azok alkalmazása

  \item Csapatmunka gyakorlati fejlesztése

  \item Oldott játék

  \item Egészséges testi fejlődés

  \item Saját igényekhez képest megfelelő táplálkozás

  \item A természettel való kapcsolódás

  \item Épített falusi és városi környezetben való eligazodás

  \item A technológia világában felhasználói szintű eligazodás és annak harmonikus alkalmazása
\end{itemize}

A Harmónia irányelv a fenti célok eléréséhez a következő egymást segítő komponensek alkalmazását javasolja:

\paragraph{Közösségben, csapatban}

A Budapest School egy közösségi iskola, ahol a közösség tagjai egymással és egymástól tanulnak. A közösségekhez való tartozáshoz, a csapatban való gondolkodáshoz, és a családban való működéshez szükséges képességeket leginkább úgy tudjuk fejleszteni, ha azt kezdetektől megéljük. A közösség belső szabályainak megalkotása és az azokhoz való kapcsolódás a tanulás folyamatosságának alapfeltétele.

\paragraph{Életképességek (life skills)}

Szeretnénk, ha gyerekeink általában alkalmazkodóan (adaptívan) és pozitívan tudnának hozzáállni az élet kihívásaihoz, ha lelki és fizikai erősségük és rugalmasságuk (rezilienciájuk) megmaradna és fejlődne.	 A Egészségügyi Világszervezet \citep{oecd99lifeskills} a következőképpen definiálta  az életképességeket
\begin{itemize}
  \item Döntéshozás, problémamegoldás

  \item Kreatív gondolkodás

  \item Kommunikáció és interperszonális képességek

  \item Önismeret, empátia

  \item Magabiztosság (asszertivitás) és higgadtság

  \item Terhelhetőség és érzelmek kezelése, stressztűrés
\end{itemize}
\paragraph{Érzelmi intelligencia}

Sokszor kiemeljük az érzelmi intelligenciát, kihangsúlyozva, hogy gyerekeinknek többet kell foglalkozniuk az érzelmek felismerésével, kontrollálásával és kifejezésével, mint szüleinknek kellett.

\paragraph{Szabad mozgás és séta}

A különböző mozgásformák, sportok és a séta mindennapivá tétele természetes módon, a gyerekek saját igényei szerint kell hogy történjen.

\paragraph{Gyakorlatias, mindennapi képességek}

Ahhoz, hogy gyerekeink önállóan és hatékonyan tudják élni életüket, hogy a társakhoz való kapcsolódás ne függőség legyen, egy csomó praktikus mindennapi tudást el kell sajátítaniuk. A gyerekeknek folyamatosan fejleszteniük kell az élethez szükséges mindennapi tudást a levélszemét-kezeléstől, az internet és a szociális média tudatos használatán át, egészen a személyi költségvetés elkészítéséig.

\paragraph{Egészséges táplálkozás}

Az egészséges táplálkozás tanulható viselkedésforma, melynek alapja nemcsak a megfelelő élelmiszerek kiválasztása, hanem azok élettani hatásainak megismerése, és az étkezési szokások alakítása is.

