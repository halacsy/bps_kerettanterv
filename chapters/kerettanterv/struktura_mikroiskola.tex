\section{A Budapest School összevont osztályai, a mikroiskolák}

A Budapest School iskolában összevont osztályokban, azaz kevert
korosztályú és maximum 4 évfolyamszintű \emph{közösségben} tanulnak a gyerekek.
Az összevont osztályokat a Budapest School kerettanterv
\emph{mikroiskolának} hívja kiemelendő, hogy
\begin{itemize}
    \item Egy osztályfőnök helyett tanulásszervező tanárcsapatok vezetik a
          közösségeket.
    \item A mikroiskolák saját maguk által meghatározott házirend és szabályok
          szerint működnek.
    \item Maguk alakítják saját napirendjüket, órarendjüket, modulkínálatukat és ebben nem kell
          más osztályokhoz, mikroiskolákhoz igazodniuk.

\end{itemize}
Nem utolsó sorban a mikroiskola kifejezéssel kapcsolódunk a magyar
hagyományokban erősen élő
\emph{osztatlan kisiskolákhoz}
\citep{kisiskolak}, \citep{khan2012one}.

\paragraph{A mikroiskolákat a tanárok vezetik.}
A mikroiskolát dedikált tanulásszervezők irányítják. Ők felelnek a tanulás
tartalmáért, a
modulok meghirdetéséért és a tanulási eredmények nyomonkövetéséért. Ők döntenek
arról, hogy milyen gyereket tud befogadni a közösség.

\paragraph{Mikroiskola nem túl nagy.} Egy
mikroiskola minimális létszáma 6 maximális létszáma 60 fő. Minden
mikroiskolának megfelelő számú olyan
tanulásszervezővel kell rendelkeznie, aki mentortanárként is végzi munkáját.

\paragraph{A mikroiskolák korkülönbségei állandók.}
A mikroiskolák korhatára, mint az összevont osztályok korhatárai, a gyerekekkel
változnak. A korhatárokat tágítani és szűkíteni évente egyszer lehet, és erről
az iskola értesíti a szülőket minden tanév kezdést megelőző február 15-ig.

A Budapest School minden miroiskolát végigvisz
addig, amíg az utolsó gyerek is befejezi a 12. évfolyamot.

\paragraph{A mikroiskola állandó, a gyerekek és tanárok jöhetnek és mehetnek.}
Budapest School mikroiskolái úgy működnek, mint egy összevont osztály.
Értelemszerűen a gyerekek egy közösségben maradnak, mégha
tanárok vagy gyerekek kilépnek vagy belépnek a közösségbe. A gyerekek a
mikroiskola közösség tagjai addig, amíg ott
tudnak jól tanulni, és a közösség és a gyerek kapcsolat gyümölcsöző.

Az mikroiskolák, azaz osztályok létrehozásakor arra törekszünk, hogy
olyan gyerekek tanuljanak együtt, akik tudják egymást támogatni a tanulásban.

\paragraph{A mikroiskoláknak \emph{saját fókuszuk, stílusuk alakulhat}.}

Van olyan mikroiskola, amely a fejlesztési célok eléréséhez és az egyéni
célok
mentén már 6 éves gyerekek tanulásánál a robotika eszközeit használja,
másutt drámafoglalkozásokkal fejlesztik 12 éves gyerekek a szövegértésüket
és
éntudatukat.

A mikroiskola-rendszerben rejlik annak a lehetősége, hogy egy adott tanulási
környezetben a hangsúlyok úgy váltakozhassanak a csoport és az egyén
érdeklődését követve, hogy közben a tanulási egyensúly fennmaradjon a
tantárgyak között. A mikroiskolák nemcsak abban térnek el
egymástól, hogy kevert korcsoportban, más korosztályú gyerekek, más
érdeklődések mentén, és ily módon más célokat követve tanulnak, hanem
területileg, regionálisan is eltérőek lehetnek.

\paragraph{A mikroiskolákban a tanulók nagymértékben befolyásolják, hogy mit és
    hogyan tanulnak és alkotnak.}

A tanárok választási lehetőségeket dolgoznak ki, amikből a gyerekek (a
mentoruk és szüleik segítségével) a saját céljaikat, érdeklődésüket
leginkább
támogató \emph{egyéni tanulási tervet} alkotnak. Az iskolákban (a tanárok
által
meghatározott kereteken belül) megfér egymással több, különböző egyéni
céllal
rendelkező gyerek.

Eltérhet, hogy egy-egy gyerek mit tanul, ezért az is, hogy mikor és hogyan
sajátítja el a szükséges ismereteket: egy közösségben megfér a központi
felvételire fókuszáló 11 éves gyerek, és az olyan is, aki ekkor inkább a
Mine\-craft programozásában akar elmélyülni, ezért más képességek
fejlesztésével
lassabban halad. A tanárok feladata és felelőssége, hogy olyan közösségeket
válogassanak össze, amelyek kellően diverzek, és mégis jól működnek, a
gyerekek
igényeit és a kerettanterv céljait egyaránt megfelelően kielégítik.

\paragraph{Kisebb csoportokban tanulhatnak a gyerekek.}

A közösséget kisebb csoportokra bonthatjuk, ha a tanulásszervezés ezáltal
hatékonyabb. Egyes modulokban egy-egy projektre szerveződnek a gyerekek,
ilyenkor gyakran az eltérő képességű és életkorú gyerekek is megférnek
egymás
mellett. Más moduloknál a csoportokat általában képességszint alapján hozza
létre a tanár. Ilyen lehet a másodfokú egyenletek megoldóképletét megismerő
csoport, az írni tanulók csoportja, vagy egy angol nyelvű újság
szerkesztésére
és megírására alakult modul, ahol a nyelvismeretnek és a szövegalkotási
képességnek már egy olyan szintjén kell lenni, hogy a projektnek jól
mérhető
kimenete lehessen.

\paragraph{A mikroiskolák diverz, integratív közösségek.} A Budapest School
iskolák társadalmi,
kulturális és gazdasági értelemben is egyik fő céljuknak tartják az
integrációt addig,
amíg az a közösség céljait szolgálja.

\paragraph{A Budapest School mikroiskolái tanuló közösségek.} Mindig, minden
módszer,
folyamat fejleszthető, ezért a tanárok feladata, lehetősége, hogy az aktuális
helyzethez illő legalkalmasabb módszert válasszák a tanulás segítéséhez.

A Budapest School mikroiskolák célja, hogy jól átlátható, követhető és
folyamatosan fejlődő folyamattá váljon a tanulás mind a tanuló, mind a tanár,
mind
a szülő részéről.

\subsection{Mikroiskola születése és megszűnése}

A tanulóközösség fontos célja, hogy biztonságot, támogatást nyújtson, és
\emph{így} segítse a közösség tagjainak a minőségi tanulását. A mikroiskola a
tanárokon és a gyerekeken is túlmutató közösség. Akkor is tovább működik, ha
egy
tanár távozik a mikroiskolából, vagy ha a gyerekeknek akár a fele kicserélődik
(amikor
régiek elmennek és újak jönnek).

A mikroiskolát egy \emph{alapító}, vagy \emph{alapítócsapat} hozza létre a
Budapest School fenntartójának felkérésére.
Az alapító felelőssége, hogy segítse az iskolát megfelelő helyet és eszközöket
biztosítani a mikroiskola működtetéséhez, kijelölje a fenntartóval közösen az
adott mikroiskola fókuszát mind a korcsoport, mind a tanulás tartalmának
fókuszát illetően.

Az alapító feladata továbbá, hogy a mikroiskola tanulásszervezőivel közösen
kijelölje azt a személyt, aki a működtetés során vállalja a Budapest School
fenntartóval, valamint az intézményigazgatóval való folyamatos
kapcsolattartást, valamint a tanulásszervezőkkel és modulvezetőkkel való belső
koordinációt. Ez a koordinátor felel a jövőben azért, hogy a mikroiskola a
kerettantervi elvárásoknak megfelelő módon működjön.

A mikroiskola kétféleképpen szűnthet meg: ha összeolvad egy másik
mikroiskolával vagy minden gyerek elhagyja a mikroiskolát. Az utolsó kapcsolja
majd le a villanyt.

\subsection{Gyerek felvétele az iskolába}
Arról, hogy egy gyerek csatlakozhat-e egy mikroiskolába, arról a
mikroiskola-vezető tanulásszervező tanárok döntenek.

\subsection{Gyerek mikroiskola váltása}

Egy gyerek akkor válthat a Budapest School egyes mikroiskolái között,
amennyiben a fogadó mikroiskola őt elfogadja.

Az egyes mikroiskolák közös
modulokat is meghirdethtnek. Ebben az esetben mindkét mikroiskolában meg kell
hirdetni a modult, a vezetésének azonban egy felelőse van.
