\section{Érettségire készülés}
\label{sec:erettsegi}

Az iskola a kötelező középszintű érettségi vizsgatárgyakra való felkészítést kötelezően vállalja érettségire felkészítő modulok szervezésével. Lefordítva ezt a miniszter által kiadott kerettantervek alapján működő iskolák esetén használt terminológiára, az érettségire felkészülés érettségi tárgyak alapján szervezett fakultáció formájában történik.

A választható tantárgyak és az emelt szintű érettségi vizsgára csak akkor szervez egy mikroiskola modult, ha arra legalább a közösség 20\%-a és minimum 6 gyerek igényt tart. Abban az esetben, ha minden választható tantárgyat csak kevesebb,  mint 20\% vagy 6 gyerek választ, és így a közösség nem tud választható érettségi tárgyat választani, a fenntartó véletlenszerűen sorsol legalább egy választható tárgyat a gyerekek által megjelöltekből.

Érettségire felkészítő modulokat akkor kell meghirdetni, amikor a gyerekek elérik a 11. évfolyamszintet minden tárgyból.

Különböző mikroiskolákba járó gyerekek közös modulon készülhetnek az érettségire. A fenntartó, ha nem tudja maga megszervezni a felkészítő modulokat, akkor más iskolákkal együttműködve kell hogy biztosítsa a felkészülési lehetőséget.