\section{Minőségfejlesztés, folyamatszabályozás}
\label{sec:minosegbiztositas}
\begin{quote}
  Honnan tudjuk, hogy jól működnek a mikroiskolák? Minden gyerek azt tanulja,
  amit szeretne, és amire neki a leginkább szüksége van? Megkérdezzük az
  érintetetteket.
\end{quote}

A Budapest School iskolában a gyerekek tanulását egy komplex rendszer egy
elemének tekintjük \citep{barabasi}. Iskoláink egy, a világ felé nyitott
hálózatot alkotnak: az egy közösségben lévő gyerekek, a családok, a velük
foglalkozó tanárok, a helyi környezet, a Budapest School további
mikroiskoláinak közössége, az ország és a nemzet állapota, valamint a globális
társadalmi folyamatok is befolyásolják, hogy mi történik egy iskolában.

Az, hogy egy gyerek épp mit és mennyit tanul, nemcsak a kerettantervtől függ,
hanem számos más tényezőtől is függ, melyek befolyásolják a fejlődést és a
közösség egymás fejlődésére gyakorolt hatását: többek közt függ a gyerekek
múltjától, jelenlegi hangulatától és vágyaitól, a tanárok személyiségétől, a
családtól, a csoportdinamikától, és még a társadalomban történő változásoktól
is.

Ezért a gyermekeink fejlődését, tanulását és boldogságának alakulását egy
\emph{komplex rendszer} működésével modellezhetjük.

A tanulási folyamataink minőségfejlesztésénél a következő szempontokat vesszük
figyelembe:
\begin{enumerate}
  \item  A történéseket, eseményeket, (rész)eredményeket folyamatosan kell
        monitorozni.
  \item  A visszajelzéseket folyamatosan kell gyűjteni a rendszer minden
        tagjától: a gyerekektől, tanároktól, szülőktől és az adminisztrátoroktól.
  \item Anomália esetén a helyzetfelismerés, az eltérések okának felkutatása a
        cél.
  \item A feltárt hibák alapján a rendszert folyamatosan kell javítani.
\end{enumerate}

Az egész minőségfejlesztés célja az iskola és a teljes	Budapest School
hálózat, mint tanuló rendszer folyamatos fejlesztése. A monitorozás folyamatos,
így hamar fel tudjuk ismerni az anomáliákat, és kivizsgálás után, ha szükséges,
akkor az anomáliát meg tudjuk szüntetni, és még tanulni is tudunk belőle, és
javítani a rendszert.

A fenntartó folyamatosan monitorozza és visszajelzéseket ad a mikroiskoláknak,
ami alapján javítja a működési folyamatokat. A kerettantervben leírt működést a
fenntartó mérhető és megfigyelhető metrikákra fordítja le, és kidolgozza,
üzemelteti a metrikák 2-3 hónapnyi rendszerességű nyomon követésére alkalmas
rendszert.

A fenntartónak meg kell figyelnie legalább a következő metrikákat:
\begin{enumerate}
  \item A szülő, a gyerek és a tanár közötti egyéni célokat megfogalmazó hármas
        megállapodások időben megszülettek, nincs olyan gyerek, akinek nincs elfogadott
        saját tanulási célja. Metrika: elkészült szerződések száma

  \item A modulok végén a portfóliók bővülnek, és azok tartalma a tantárgyakhoz
        kapcsolódik. Metrika: portfólióelemeinek száma és kapcsolhatósága

  \item A szülők biztonságban érzik gyereküket, és eleget tudnak arról, hogy mit
        tanulnak. Kérdőíves vizsgálat alapján

  \item A tanárok hatékonynak tartják a munkájukat, kérdőíves felmérés alapján

  \item A gyerekek úgy érzik, folyamatosan tanulnak, támogatva vannak, vannak
        kihívásaik. Kérdőíves felmérés alapján
\end{enumerate}
