\section{Évfolyamok és osztályzatok}
A Budapest School gyerekek saját tanulási célokat tűznek ki, modulokat
választanak, tanulnak, alkotnak, trimeszterenként frissíti a portfóliójukat,
mentorukkal és a modulvezetőkkel értékelik haladásukat, és ha kell,
újraterveznek. És újra indul a tanulás ciklus.

Eközben a gyerekek a tanulás és alkotás eredményeként évfolyamszinteken
lépkednek fel,
első szintről a tizenkettedik szintig tantárgyanként.
Azt, hogy ez hogyan és mikor történik, azaz az évfolyamszintek elismerését -- a
kerettantervvel összhangban -- az iskola egy transzparens folyamata
szabályozza.

Hivatalos, tantárgyankénti érdemjegyet a gyerekek akkor és csak akkor kapnak,
ha erre iskolaváltás, továbbtanulás, ösztöndíj	    (vagy más külső rendszer)
miatt szükségük van.
Tehát osztályzatok, érdemjegyek és vizsgák nélkül is van lehetőség
évfolyamszinteket lépni.

A vizsga így teljesen átértékelődik a Budapest Schoolban.
Az évfolyamszintek elismeréséhez és (szükség esetén) az érdemjegyek
megállapításához
nem elégséges a pillanatnyi tudást vagy képességet felmérő eseményt szervezni,
hanem
a teljes portfóliót kell értékelni és figyelembe venni.
A portfólió sokkal gazdagabban dokumentálja, hogy egy gyerek mit csinált, mire
volt képes, mint egy szóbeli vagy írásbeli feladatsor: előzetes tudás-,
képességpróbák
mellett tartalmazza az alkotások, projektek, visszajelzések, versenyek, stb.
dokumentációt is.

\subsection{Évfolyamszintek}
\label{sec:evfolyamok}

A Budapest Schoolban az évfolyamokra úgy tekintünk, mint egy szerepjáték
nehézség
szintjeire \citep{wiki:game_levels}: akkor léphet egy tanuló a következőbe, ha
az
évfolyamhoz köthető tantárgyi tanulási eredményekből eleget összegyűjtött.
Ezeket \aref{sec:tantargyi_tanulasi_eredmenyek} c. fejezete határozza meg.

A Budapest School évfolyamszintjei eltérnek az iskolák többségében alkalmazott
évfolyamtól. A különbség kihangsúlyozása végett a kerettanterv az évfolyamszint
kifejezést használja. A különbségek:

\begin{itemize}
      \item Egy gyerek minden tantárgyból állhat más szinten.
      \item Nem biztos, hogy az egy korcsoportba tartozók vannak ugyanazon az
            évfolyamszinten.
      \item Nem mindig az egy évfolyamszinten lévők tanulnak együtt, sokszor
            előfordulhat az is, hogy a különböző szinten lévő tanulók tudnak
            együtt és akár egymástól is tanulni.
      \item Egy év alatt több évfolyamszintet is lehet lépni.
\end{itemize}

Bár egy gyerek, minden tantárgyból állhat más szinten, a hivatalos (azaz külső
hivatalok, rendszerek számára értelmezhető) bizonyítványába mindig csak annak
az évfolyamnak
az elvégzése kerül be, amelyből mind a három tantárgyhoz szükséges fejlesztési
célt elérte.
Formálisabban kifejezve a tantárgyankénti évfolyamszintek minimumát kell a
bizonyítványban rögzíteni.

\subsubsection{Évfolyamszint-lépés}
\label{sec:evfolyamszintlepes}
A tanulók portfóliója alapján megállapítható, hogy egy adott évfolyamhoz
köthető tantárgyi követelményeknek megfelel-e.
Ehhez a tanulók elvégzik a mentoruk segítségével a portfóliójuk (mit csináltak,
mit tanultak, mit tudnak) összehasonlítását \ifkerettanterv
      \aref{sec:tantargyi_tanulasi_eredmenyek}.
      fejezetben
\else
      a kerettanterv \emph{Tantárgyi tanulási eredmények} c. fejezetében
\fi
felsorolt tantárgyankénti bontásban megadott elvárt tanulási eredményekkel.

Ha szükséges, akkor a kapcsolódás biztosításához a portfóliójukat
kiegészíthetik tudáspróbák, tesztek, szabványos vizsgák teljesítésével, melynek
megszervezése az adott mikroiskola tanárközösségének a feladata.

Miután a gyerek (mentora és szülei) segítségével összeállította a portfólióját,
jelzi az iskolának az évfolyamszint lépési kérelmét.

Ezt az iskola által kijelölt bírálók mevizsgálják és elismerik az
évfolyamszinthez szükséges tantárgyi követelmények teljesítését.
Egy tantárgyból egy évfolyam teljesítettnek tekinthető, ha a tantárgyhoz
tartozó tanulási eredmények 50\%-ának elérése a portfólió alapján bizonyítható.

A kérelmet a gyerek digitálisan adja be.
Az elbírálás csak a portfólió alapján történhet, ami egy
online elérhető adatbázisként tartalmaz mindent, ami a döntéshez szükséges
lehet. A döntéshez így a bírálóknak és a gyereknek nem
kell egy időben és egy helyen lennie. Minden esetben szükséges a portfóliót és
a teljes folyamatot digitálisan rögzíteni.
Az iskolának két évig meg kell őriznie a portfóliót, a kérelmet és a döntéshez
használt minden dokumentációt.

Magántanuló vagy az órák látogatásáról valamilyen okkal
felmentett gyerekek ugyanígy, a portfóliójuk összeállításával és a szintlépés
kérelmezésével kérhetik az évfolyamszintek teljesítésének igazolását.

Amennyiben valamely diáknál egy adott évfolyam tantárgyi követelményei
elismerésre kerülnek, akkor az iskola igazolja, hogy a diák az adott tantárgy
vagy tantárgyak évfolyam szerinti követelményeit teljesítette.
Erről igazolást állít ki, és teljesíti a jelentési kötelezettségét az Oktatási
Hivatal felé.

\subsection{Osztályzatokra váltás}
\label{sec:osztalyzatok}
A kerettanterv lehetővé teszi, hogy a gyerekek az érdemjegyek
és osztályzatok
helyett egy több szempontot figyelembe vevő szöveges vagy értékelőtáblázat
(rubric) alapú viszajelzést kapjanak.
A gazdag információtartalmú visszajelzések és portfólió osztályzatra való
átváltására mégis szükség lehet, például iskolaváltás vagy továbbtanulás
esetén.\footnote{Azt az NKT. 54.§ (4) pontja alapján.}

Az átváltás \aref{sec:evfolyamszintlepes}.
fejezetben leírtakhoz
hasonlóan is a portfólió értékelésén alapulnak.
A gyerek (mentora és szülei segítségével) összeállítja a portfóliót, és
bizonyítja, hogy a portfólió alapján megállapítható a kívánt osztályzat, az
adott tárgyhoz az adott évfolyamban.

\paragraph{,,Szokásos'' tantárgyakra leképzés}
A gyerekek (szüleik) kérhetik NAT pedagógiai szakaszainak
végén\footnote{A NAT II.2.1 pontja három pedagógia szakaszt határoz meg: az
      alapfokó nevelés-oktatás két szakasza az 1-4. évfolyamok és 5-8.
      évfolyamok,
      a középfokú nevelés-oktatás szakasz a 9-12. évfolyam.}
a miniszter
által kiadott kerettanterv tantárgyi rendszere szerinti értékelést és az annak
történő megfeleltetést. Ez a tanulási eredmények adatbázisa alapján
egyértelműen elvégezhető, mert minden tanulási eredmény egy tanulási területhez
tartozik, amik pedig egy az egyben kapcsolódnak a miniszter által kiadott
tantárgyakkal.
