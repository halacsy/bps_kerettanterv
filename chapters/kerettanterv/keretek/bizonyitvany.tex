\section{Évfolyamok, bizonyítvány}
\label{sec:evfolyamok}

A tantárgyközi, moduláris tanulás közben a visszajelzések és a portfólióelemek
egybeállításával monitorozzuk, hogy az egyes fejlesztési célok megjelenjenek a
mindennapos tanulás során. Így a Budapest School a NAT pedagógiai szakaszainak
végén minden esetben, közben pedig a tanuló/szülő kérésére elvégzi a miniszter
által kiadott kerettanterv tantárgyi rendszere szerinti értékelést és az annak
történő megfeleltetést.

A Budapest Schoolban az évfolyamokra úgy tekintünk, mint egy játék nehézség
szintjeire: akkor léphet egy tanuló a következőbe, ha elvégezte az adott
évfolyamhoz köthető tantárgyi követelményeket. És remélhetőleg egyre nagyobb
kihívások várnak rá a következő szinten. A hagyományos évfolyamoktól eltérően

\begin{itemize}
  \item Egy tanuló minden tantárgyból állhat más szinten.
  \item Nem biztos, hogy az egy korcsoportba tartozók vannak ugyanazon az
        évfolyamszinten.
  \item Nem mindig az egy évfolyamszinten lévők tanulnak együtt, sokszor
        előfordulhat az is, hogy a különböző szinten lévő tanulók tudnak együtt és akár
        egymástól is tanulni.
  \item Egy év alatt több évfolyamszintet is lehet lépni.
\end{itemize}

A bizonyítványába azonban mindig annak az évfolyamnak az elvégzése kerül be,
amelyből minden kötelező tantárgyhoz szükséges fejlesztési célt elérte.
Formálisabban kifejezve a tantárgyankénti évfolyamszintek minimumát kell a
bizonyítványban rögzíteni.

\subsection{Előrehaladás igazolása}
A tanulók portfóliója alapján megállapítható, hogy egy adott évfolyamhoz
köthető tantárgyi követelményeknek megfelel-e. Ehhez a tanulók elvégzik a
mentoruk segítségével a portfóliójuk (mit csináltak, mit tanultak, mit tudnak)
összehasonlítását 
\ifkerettanterv 
\aref{sec:tantargyi_celok}. fejezetben 
\else
a kerettanterv \emph{Tantárgyi eredménycélok} c. fejezetében 
\fi
 felsorolt tantárgyi
eredménycélokkal. Ha szükséges, akkor a kapcsolódás biztosításához a
portfóliójukat kiegészíthetik tudáspróbák/szintfelmérő vizsgák teljesítésével,
melynek megszervezése az adott mikroiskola tanárközösségének a feladata.

Amennyiben valamely diáknál egy adott évfolyam tantárgyi követelményei
elismerésre kerülnek, akkor az iskola hivatalosan is igazolja, hogy a diák az
adott tantárgy vagy tantárgyak évfolyam szerinti követelményeit teljesítette.

Egy tantárgyból egy évfolyam teljesítettnek tekinthető, ha a tantárgyhoz
tartozó tanulási célok 50\%-ának elérése a portfólió alapján bizonyítható.

Az iskola automatikusan elkészíti a 2., 4., 6., 8., 10. és 12. évfolyamszint
követelményeinek teljesítésekor, vagy a tanuló/szülő kérésére ettől eltérő
időben is, a fenti hitelesítési, igazolási folyamatot, ami alapján bizonyítvány
állítható ki.

\subsection{Szöveges értékelés, érdemjegyek}

A Budapest Schoolban a tanulásra való visszajelzés csupán egyik eleme, hogy
érdemjegyet adunk, amikor erre van szükség. Ennek fő célja, hogy a fejlesztési
célokhoz való kapcsolódást kifejezzük. A tanulás legfőbb mércéje a gyerek
fejlődésének mérése önmagához képest a kortárs és a tanári visszajelzések
alapján, valamint a portfolióban összegyűlt tanult, tapasztalt és létrehozott
tartalmak rögzítése formájában.  Amikor a tanuló következő évfolyamba léphet,
akkor ez azt jelenti, hogy a portfóliója alapján megállapítható, hogy
\emph{elsajátította a korosztályának megfelelő fejlesztési célokat}.
