\section{A tanév ritmusa}

A tanév három trimeszter ismétlődésével írható le: a tanulási célok
tervezése után következik a tanulás, és a ciklust a visszajelzés és értékelés
zárja.	Amint egy ciklus véget ér, elkezdődik egy új.

A ciklusok állandósága adja a tanulás irányításához szükséges kereteket. Ezek
megtartásáért az egyes mikroiskolák tanulásszervezői felelnek, melynek működését a
fenntartó monitorozza. A tanév ritmusát \aref{tbl:tanevritmus}. táblázat
mutatja.

\begin{table}
  \centering
  \begin{tabular}{ l|l }
    \textbf{időszak} & \textbf{tevékenység}                \\
    \hline
    Szeptember       &
    közösségépítés                                         \\
                     & saját célok meghatározása           \\
                     & modulok kialakítása és meghirdetése
    \\ \hline

    Október          &
    tanulás, alkotás
    \\ \hline

    November         &
    tanulás, alkotás
    \\ \hline

    December         &
    portfólió frissítése                                   \\
                     & reflexiók                           \\
                     & visszajelzések                      \\
                     & célok felülvizsgálata               \\
                     & modulok változtatása igény esetén
    \\ \hline

    Jánuár           &
    tanulás, alkotás
    \\ \hline

    Február          &
    tanulás, alkotás
    \\ \hline

    Március          &
    portfólió frissítése                                   \\
                     & reflexiók                           \\
                     & visszajelzések                      \\
                     & célok felülvizsgálata               \\
                     & modulok változtatása igény esetén
    \\ \hline

    Április          &
    tanulás, alkotás
    \\ \hline

    Május            &
    tanulás, alkotás
    \\ \hline

    Fél június       &
    évzárás, értékelés, bizonyítványok
  \end{tabular}
  \caption{Egy tanévben háromszor ismételjük a célállítás, tanulás,
    reflektálás ciklust.}
  \label{tbl:tanevritmus}
\end{table}

A tanév három periódusból áll: ez a felosztás követi az üzleti világ negyedéves
tervezését, néhány egyetem trimeszterekre bontását, de leginkább az évszakokat.
Minden periódus után értékeljük az elmúlt három hónapot, ünnepeljük az
eredményeket, és megtervezzük a következő időszakot.  A trimesztereken belül az
egyes mikroiskolák között lehetnek néhány hetes eltérések, melyek a közösség
sajátosságait követik.