\section{NAT céljainak támogatása}
\label{sec:nat_celjai}
A Nemzeti Alaptantervben szereplő fejlesztési célok elérését és a
kulcskompetenciák fejlődését több minden támogatja:

Egyrészt a tantárgyak lefedik a NAT fejlesztési céljait, kulcskompetenciáit és
műveltségi területeit, mert a jelenleg érvényben lévő, a miniszter által a
\emph{51/2012. (XII. 21.) számú EMMI rendelet I-IV. mellékletében} kiadott
kerettantervek \citep{ofi:kerettanterv} tanulási, tanítási eredményeiből
indultunk ki. Mivel a rendeletben szereplő kerettantervek teljesítik a NAT
feltételeit, így a Budapest School tantárgystruktúrája is teljesíti ezeket.

Másrészt az iskola életében, folyamatában való részvétel, már önmagában
biztosítja a kulcskompentenciák fejlődését és a NAT fejlesztési céljainak
teljesülését sok esetben.

A \ref{tbl:nat_fejlesztesi} táblázat bemutatja a NAT fejlesztési területeihez
való kapcsolódást, a
\ref{tbl:nat_kulcs} táblázat pedig az illeszkedési pontokat a NAT
kulcskompetenciáihoz.

\begin{table}

  \begin{tabular}{p{5cm}|>{\raggedright}p{3cm}|p{3cm}}

    \textbf{NAT Fejlesztési célok}               & \textbf{Tantárgyak}  & \textbf{Struktúra}           \\
    \hline
    Az erkölcsi nevelés                          & kult, harmónia       & közösség                     \\ \hline
    Nemzeti öntudat, hazafias nevelés            & kult, harmónia       & projektek                    \\ \hline
    Állampolgárságra, demokráciára nevelés       & kult, harmónia       & közösség                     \\ \hline
    Az önismeret és a társas kultúra fejlesztése & kult, harmónia, stem & saját
    tanulási út, közösség                                                                              \\ \hline
    A családi életre nevelés                     & harmónia             &                              \\ \hline
    A testi és lelki egészségre nevelés          & harmónia             & közösség                     \\ \hline
    Felelősségvállalás másokért, önkéntesség     & harmónia             & közösség, pro\-jek\-tek      \\
    \hline
    Fenntarthatóság, környezettudatosság         & harmónia, stem       & projektek                    \\ \hline
    Pályaorientáció                              & kult, harmónia, stem & saját tanulási út            \\ \hline
    Gazdasági és pénzügyi nevelés                & kult, harmónia, stem & projektek                    \\ \hline
    Médiatudatosságra nevelés                    & kult                 & projektek                    \\ \hline
    A tanulás tanítása                           & kult, harmónia, stem & saját tanulási út, mentorság \\

  \end{tabular}
  \caption{A NAT fejlesztési céljainak elérését nemcsak a tantárgyak, hanem az
    iskola struktúrája is támogatja.}
  \label{tbl:nat_fejlesztesi}
\end{table}

A \emph{saját tanulási} út fogalma például önmagában segíti a tanulás
tanulását, hiszen az a gyerek, aki képes önmagának saját célt állítani (mentor
segítséggel), azt elérni, és a folyamatra való reflektálás során képességeit
javítani, az fejleszti a tanulási képességét.

Vagy másik példaként, a Budapest School iskoláiban a \emph{közösség} maga hozza
a működéséhez szükséges szabályokat, folyamatosan alakítja és fejleszti saját
működését a tagok aktív részvételével. Ez az aktív állampolgárságra, a
demokráciára való nevelés Nemzeti Alaptantervben előírt céljait is támogatja.

\begin{table}
  \centering
  \begin{tabular}{p{5cm}|>{\raggedright}p{3cm}|p{3cm}}

    \textbf{NAT kulcskompetenciái}                     & \textbf{Tantárgyak}  &
    \textbf{Struktúra}                                                                                        \\ \hline
    Anyanyelvi kommunikáció                            & kult                 & tanulási szerződés, portfólió \\ \hline
    Idegen nyelvi kommunikáció                         & kult                 & idegennyelvű modulok          \\ \hline
    Matematikai kompetencia                            & stem                 &                               \\ \hline
    Természettudományos és technikai kompetencia       & stem                 & projektek                     \\ \hline
    Digitális kompetencia                              & harmónia, stem       & digitális portfólió kezelés   \\ \hline
    Szociális és állampolgári kompetencia              & harmónia             & saját tanulási út,
    közösség                                                                                                  \\ \hline
    Kezdeményezőképesség és vállalkozói kompetencia    & kult, harmónia, stem & saját
    tanulási út, közösség                                                                                     \\ \hline
    Esztétikai-művészeti tudatosság és kifejezőkészség & harmónia, kult       &                               \\ \hline
    A hatékony, önálló tanulás                         & kult, harmónia, stem & saját tanulási út,
    mentorság                                                                                                 \\

  \end{tabular}
  \caption{NAT kulcskompetenciáinak fejlesztését támogatják a tantárgyak és
    az iskola felépítése is.}
  \label{tbl:nat_kulcs}
\end{table}