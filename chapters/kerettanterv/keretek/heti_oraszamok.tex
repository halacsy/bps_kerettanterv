\paragraph{Heti óraszámok}

A Budapest School közösségi tanulási élményeket és modulokat szervez a
gyerekeknek, egyúttal lehetőséget ad arra, hogy a gyerekek a közösen kialakított
szabályaik mentén tanulásszervezők felügyeletével a Budapest School székhelyén
vagy egyes telephelyein, vagy más, erre alkalmas tanulási környezetben
tartózkodjanak. A közösségben együtt töltött idő tanulásnak, fejlődésnek
minősül akkor is, ha az nem egy modulhoz kapcsolódik, hanem az ebéd
élvezetéhez, vagy épp a parkban a lehulló falevelek neszének megfigyeléséhez.

A gyerekek, a tanítási szüneteket leszámítva, általában naponta 8 órát
tartózkodnak az
iskolában.\footnote{Ez alól több kivétel lehet: külső foglalkozások, otthoni,
  egyéni tanulás. Ezekről külön megállapodást kell kötni a mentorral.} Ezekben
az
időkben vannak a tanítási órák, foglalkozások, szakkörök,
műhelyek. Az egyes mikroiskolák ettől 20\%-ban bármelyik irányban eltérhetnek,
ha ez segíti a tanulásszervezők munkáját és a gyerekek fejlődését. Így hetente
minimum $5 \cdot 8 \cdot 0,8 = 32$ órát, maximum 48 órát töltenek az iskolában.

Ennek az 1--4.~évfolyamszinten körülbelül a felét, azaz 16--24 órát, az 5--12.~évfolyamszinten a  kétharmad részét, azaz 21--32 órát
töltik a gyerekek előre eltervezett módon, azaz modulokkal. A többi időben a tanárok
vezetése és felügyelete mellett szabadon alkotnak, játszanak, pihennek,
közösségi életet élnek.

Mivel az elvárt kiegyensúlyozottság miatt mind a három tantárgyra körülbelül
ugyanannyi energiát kell fektetni, így az egyes tantárgyakra a teljes
rendelkezésre álló időkeret egyharmad részét kell számolni. Ettől az iskolák
$\pm$ 20\%-ban eltérhetnek, így kiszámolható, hogy minimum mennyi időt kell
egy-egy gyereknek egy héten egy tantárggyal foglalkoznia. Ezt összegzi
\aref{tbl:oraszamok}. táblázat.

\begin{table}

  \begin{tabular}{ l|l|l }

    \textbf{\small Tantárgy} & \textbf{\small 1--4. évfolyam}                               & \textbf{\small 5--12. évfolyam}
    \\ \hline
    {\small Harmónia}          & $\frac{5 \times 8 \times 0,8}{2} \times \frac{1}{3}
      \times \hbox{\small 0,8} =
    \hbox{\small 4,27}$ {\small óra}         &
    $\frac{5 \times 8 \times 0,8 \times 2}{3} \times \frac{1}{3} \times \hbox{\small 0,8} =
      \hbox{\small 5,69}$
    {\small óra}
    \\ \hline
    {\small STEM}              & {\small 4,27 óra}
                      & {\small 5,69 óra}                                                                     \\
    \hline
    {\small KULT}              & {\small 4,27 óra}
                      & {\small 5,69 óra}                                                                     \\
    \hline

  \end{tabular}
  \caption{Az elvárt kiegyensúlyozottság miatt a tantárgyakkal egyenlő
    minimális óraszámban kell foglalkozni.}
  \label{tbl:oraszamok}
\end{table}

Fontos, hogy \emph{egy-egy modul több tantárgy fejlesztési céljaihoz és
  tanulási eredményeihez is kapcsolódhat.}