\section{Heti óraszámok}

A Budapest School közösségi tanulási élményeket és modulokat szervez a gyerekeknek, egyúttal lehetőséget ad arra, hogy a gyerekek a közösen kialakított szabályaik mentén tanulásszervezők felügyeletével a Budapest School székhelyén vagy egyes telephelyein, vagy más, erre alkalmas tanulási környezetben tartózkodjanak. A közösségben együtt töltött idő tanulásnak, fejlődésnek minősül akkor is, ha az nem egy modulhoz kapcsolódik, hanem az ebéd élvezetéhez, vagy épp a parkban a lehulló falevelek neszének megfigyeléséhez.

A gyerekek, a tanítási szüneteket leszámítva, általában naponta 8 órát tartózkodnak az iskolában.\footnote{Ez alól több kivétel lehet: külső foglalkozások, otthoni, egyéni tanulás. Ezekről külön megállapodást kell kötni a mentorral.} Ezekben az időkben vannak a tanítási órák, foglalkozások, szakkörök, műhelyek. Az egyes mikroiskolák ettől 20\%-ban bármelyik irányban eltérhetnek, ha ez segíti a tanulásszervezők munkáját és a gyerekek fejlődését. Így hetente minimum $5 \cdot 8 \cdot 0,8 = 32$ órát, maximum $48$ órát töltenek az iskolában.

Ennek az 1--4.~évfolyamszinten körülbelül a felét, azaz 16--24 órát, az 5--12.~évfolyamszinten a  kétharmad részét, azaz 21--32 órát töltik a gyerekek előre eltervezett módon, azaz modulokkal. A többi időben a tanárok vezetése és felügyelete mellett szabadon alkotnak, játszanak, pihennek, közösségi életet élnek.

A NAT II.2.2. fejezete ajánlást ad a Nat műveltségi területeinek százalékos arányaira. Ezeket betartja a kerettanterv. A kiemelt tantárgyak egy az egyben megfeleltethetőek egy-egy műveltségi területnek, így kapjuk \aref{tbl:oraszamok_kiemelt}. táblázat minimális óraszámait

\begin{table}[ht]
  \begin{center}
  \begin{tabular}{l|c|c|c|c|c}
  \textbf{Tantárgy}                                   & \textbf{1-4}  & \textbf{5-6}  & \textbf{7-8 } & \textbf{9-10} & \textbf{11-12} \\ \hline \hline
  Magyar nyelv és irodalom                            & 4.3 & 2.4  & 2.1  & 2.1  & 2.1   \\  \hline 
  Idegen nyelvek                                      & 0.3 & 1.6  & 2.1  & 2.5 & 2.7  \\  \hline 
  Matematika                                          & 2.1 & 2.1 & 2.1  & 2.1  & 2.1   \\  \hline 
  Testnevelés és sport                                & 3.2  & 3.2  & 3.15 & 3.0 & 3.2  \\  \hline 
  Történelem, társadalmi és                           & 0.6 & 1.0 & 2.1  & 2.1  & 2.1   \\  
  állampolgársági ismeretek                           &      &      &      &      &       \\ 
  \end{tabular}
  \caption{Minimális heti óraszámok kiemelt tantárgyakként.  }
  \label{tbl:oraszamok_kiemelt}
\end{center}
  \end{table}

Fontos, hogy \emph{egy-egy modul több tantárgy fejlesztési céljaihoz és tanulási eredményeihez is kapcsolódhat.} Ezért külön adja meg a kerettanterv az összevont tantárgyak által követelt minimális óraszámokat \aref{tbl:oraszamok_osszevont}. táblázatban.

\begin{table}[ht]
  \begin{center}
    \begin{tabular}{l|c|c|c|c|c}
  \textbf{Tantárgy}                                   & \textbf{1-4}  & \textbf{5-6}  & \textbf{7-8 } & \textbf{9-10} & \textbf{11-12} \\  \hline  \hline 
  Termtud. Mérnöki Technológia                        & 2.7 & 3.4 & 6.1 & 6.3  & 4.2   \\  \hline 
  Kultúra                                             & 2.9 & 2.6 & 3.8 & 3.8 & 3.3  \\  \hline 
  Harmónia                                            & 3.8 & 3.8 & 4.0 & 3.8 & 3.2 
  \end{tabular}
  \caption{Minimális heti óraszámok öszzevont, integrált tantárgyakként.  }
  \label{tbl:oraszamok_osszevont}
\end{center}
  \end{table}

  