\section{Heti óraszámok}

A Budapest School közösségi tanulási élményeket és modulokat szervez a gyerekeknek, egyúttal lehetőséget ad arra, hogy a gyerekek a közösen kialakított szabályaik mentén tanulásszervezők felügyeletével a Budapest School székhelyén vagy egyes telephelyein, vagy más, erre alkalmas tanulási környezetben tartózkodjanak. A közösségben együtt töltött idő tanulásnak, fejlődésnek minősül akkor is, ha az nem egy modulhoz kapcsolódik, hanem az ebéd élvezetéhez, vagy épp a parkban a lehulló falevelek neszének megfigyeléséhez.

A gyerekek, a tanítási szüneteket leszámítva, általában naponta 8 órát tartózkodnak az iskolában.\footnote{Ez alól több kivétel lehet: külső foglalkozások, otthoni, egyéni tanulás. Ezekről külön megállapodást kell kötni a mentorral.} Ezekben az időkben vannak a tanítási órák, foglalkozások, szakkörök, műhelyek. Az egyes mikroiskolák ettől 20\%-ban bármelyik irányban eltérhetnek, ha ez segíti a tanulásszervezők munkáját és a gyerekek fejlődését. Így hetente minimum $5 \cdot 8 \cdot 0,8 = 32$ órát, maximum $48$ órát töltenek az iskolában.

Ennek az 1--4.~évfolyamszinten körülbelül a felét, azaz 16--24 órát, az 5--12.~évfolyamszinten a  kétharmad részét, azaz 21--32 órát töltik a gyerekek előre eltervezett módon, azaz modulokkal. A többi időben a tanárok vezetése és felügyelete mellett szabadon alkotnak, játszanak, pihennek, közösségi életet élnek.

A miniszter által kiadott kerettantervek óraszámait veszi a kerettanterv alapul. \emph{Minden tantárgyból hetente ennek a felét írja elő minimum óraszámnak.}

\begin{landscape}
\begin{table}[]
  \begin{tabular}{l|l|l|l|l|l|l|l|l|l|l|l|l}
  
                                                        & \multicolumn{12}{l}{\textbf{Évfolyamszintek}}                                                                                                                                                           \\ \hline
    \textbf{Tantárgyak}                                          & 1                                     & 2           & 3           & 4           & 5           & 6           & 7           & 8           & 9           & 10          & 11          & 12          \\ \hline
    Biológia - egészségtan                              &                                       &             &             &             &             &             & 2           & 1           &             & 2           & 2           & 2           \\ \hline
    Dráma és tánc/Mozgóképkultúra és médiaismeret      &                                       &             &             &             & 1           &             &             &             & 1           &             &             &             \\ \hline
    Életvitel és gyakorlat                              & 1                                     & 1           & 1           & 1           &             &             &             &             &             &             &             & 1           \\ \hline
    Ének-zene                                           & 2                                     & 2           & 2           & 2           & 1           & 1           & 1           & 1           & 1           & 1           &             &             \\ \hline
    Erkölcstan                                          & 1                                     & 1           & 1           & 1           & 1           & 1           & 1           & 1           &             &             &             &             \\ \hline
    Etika                                               &                                       &             &             &             &             &             &             &             &             &             & 1           &             \\ \hline
    Fizika                                              &                                       &             &             &             &             &             & 2           & 1           & 2           & 2           & 2           &             \\ \hline
    Földrajz                                            &                                       &             &             &             &             &             & 1           & 2           & 2           & 2           &             &             \\ \hline
    I. idegen nyelv                                     &                                       &             &             &             &             &             &             &             & 3           & 3           & 3           & 3           \\ \hline
    Idegen nyelvek                                      &                                       &             &             & 2           & 3           & 3           & 3           & 3           &             &             &             &             \\ \hline
    II. idegen nyelv                                    &                                       &             &             &             &             &             &             &             & 3           & 3           & 3           & 3           \\ \hline
    Informatika                                         &                                       &             &             &             &             & 1           & 1           & 1           & 1           & 1           &             &             \\ \hline
    Kémia                                               &                                       &             &             &             &             &             & 1           & 2           & 2           & 2           &             &             \\ \hline
    Környezetismeret                                    & 1                                     & 1           & 1           & 1           &             &             &             &             &             &             &             &             \\ \hline
    Magyar nyelv és irodalom                            & 7                                     & 7           & 6           & 6           & 4           & 4           & 3           & 4           & 4           & 4           & 4           & 4           \\ \hline
    Matematika                                          & 4                                     & 4           & 4           & 4           & 4           & 3           & 3           & 3           & 3           & 3           & 3           & 3           \\ \hline
    Művészetek                                          &                                       &             &             &             &             &             &             &             &             &             & 2           & 2           \\ \hline
    Osztályfőnöki                                       &                                       &             &             &             & 1           & 1           & 1           & 1           & 1           & 1           & 1           & 1           \\ \hline
    Szabadon tervezhető órakeret                        & 2                                     & 2           & 3           & 3           & 3           & 3           & 3           & 3           & 4           & 4           & 6           & 8           \\ \hline
    Technika. életvitel és gyakorlat                    &                                       &             &             &             & 1           & 1           & 1           &             &             &             &             &             \\ \hline
    Természetismeret                                    &                                       &             &             &             & 2           & 2           &             &             &             &             &             &             \\ \hline
    Testnevelés és sport                                & 5                                     & 5           & 5           & 5           & 5           & 5           & 5           & 5           & 5           & 5           & 5           & 5           \\ \hline
  Történelem társadalmi és állampolgársági ismeretek&                                       &             &             &             & 2           & 2           & 2           & 2           & 2           & 2           & 3           & 3           \\ \hline
    Vizuális kultúra                                    & 2                                     & 2           & 2           & 2           & 1           & 1           & 1           & 1           & 1           & 1           &             &             \\ \hline
    \textbf{Összesen}                                   & \textbf{25}                           & \textbf{25} & \textbf{25} & \textbf{27} & \textbf{29} & \textbf{28} & \textbf{31} & \textbf{31} & \textbf{35} & \textbf{36} & \textbf{35} & \textbf{35} \\ \hline
  \end{tabular}
  \caption{A minimális két hetes egységekre számolva. A táblázatban szereplő számok a miniszter által kiadott kerettanterv óraszámai. Mivel ez a kerettanterv ezeket az óraszámokat két hetes egységre adja meg, így egy-egy tárgyból a miniszter által kiadott kerettantervhez képest fele annyi tanórányi modult kell szervezni.}  
\end{table}

\end{landscape}