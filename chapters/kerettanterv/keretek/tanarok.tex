\section{Tanárok kiválasztása, tanulása, fejlődése és értékelése}

Az iskola legmeghatározóbb összetevői a tanárok.
Ezért a Budapest School külön figyelmet fordít arra, hogy ki lehet tanár az
iskolákban, és hogyan segítjük az ő fejlődésüket.

Alapelveink
\begin{enumerate}
  \item Minden tanárnak tanulnia kell. Amit ma tudunk, az nem biztos, hogy elég
        arra, hogy a holnap iskoláját működtessük. És az is biztos, hogy még sokkal
        hatékonyabban lehetne segíteni a gyerekek tanulását, mint amilyenek a ma ismert
        módszereink.

  \item A tanároknak csapatban kell dolgozniuk, mert összetett
        (interdiszciplináris) tanulást csak vegyes összetételű (diverz) csapatok tudnak
        támogatni.

  \item Szakképesítés nem szükséges és nem elégséges feltétele annak, hogy a
        Budapest Schoolban valaki jól teljesítő tanár legyen.
\end{enumerate}

\paragraph{Felvétel}
A Budapest School tanulásszervező tanárainak felvétele egy legalább háromlépcsős
folyamat, ahol vizsgálni kell a tanár egyéniségét (attitűdjét), felnőtt-felnőtt
kapcsolatokban a viselkedésmódját (társas kompetenciáit), és minden jelöltnek
próbafoglalkozást kell tartania, amit az erre kijelölt Budapest School-tanárok
megfigyelnek. A felvételi folyamatot a fenntartó felügyeli és irányítja.

\paragraph{Saját cél}

Minden tanárnak van saját, egyéni fejlődési célja: \emph{mitől tudok én jobb
  tanár lenni, jobban támogatni a gyerekek tanulását, segíteni a munkatársaimat
  és partnerként dolgozni a szülőkkel?}

\paragraph{Mentor}

Hasonlóan a gyerekekhez, minden tanárnak van egy mentora, aki segíti a saját
céljai kialakításában, és folyamatosan támogatja ezek elérésében.

\paragraph{Értékelés}

Minden tanárt évente legalább kétszer értékelnek a munkatársai. Ez az a folyamat,
amit 360 fokos értékelésnek hívnak az üzleti szférában. A visszajelzések
feldolgozása után a saját célokat frissíteni kell.

Minden tanárt értékelnek a szülők is (kifejezetten a mentorált gyerekek szülei)
és a gyerekek is legalább évente kétszer.
