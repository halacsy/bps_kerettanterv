\section{Biztonságos tanulási környezet}

A Budapest School a tanulásszervezés folyamatait befolyásolja, annak folyamatos
fejlesztését tűzte ki célul. A folyamatok kialakításakor és akkor, amikor ezek
a folyamatok valamilyen oknál fogva nem működnek, a következő szempontok
élveznek prioritást:

\begin{enumerate}

  \item Legfontosabb, hogy a tanulás és az iskolában eltöltött idő mindenki
        számára legyen biztonságos fizikai és érzelmi szempontból egyaránt.

  \item Egyensúlyban kell tartani az iskola  tagjainak egyéni fejlődését és
        tanulását és a közösség tagjainak együttműködését, kapcsolódását.

\end{enumerate}
Ez azt is jelenti, hogy nem tartható fenn az az állapot, amikor a közösség
egyik tagjának érdekei felülírják a többiek érdekeit, ahogy az sem, amikor a
közösség valakinek az igényeit nem veszi figyelembe, vagy amikor a közösségi
kapcsolódás felülírja a tanulási célokat. A közösség minden tagja számára
biztonságos és kiegyensúlyozott környezet kialakításáért, és fenntartásáért a
tanulásszervező tanárok a felelősek.