\section{Tantárgyak}
\label{sec:tantargyak}
A Budapest School a ma gyerekeinek kínál olyan oktatást, ami segíti felkészíteni őket a jövő kihívásaira. Információs társadalmunk legnagyobb kihívása az adaptációs képességünk fejlesztése, ez az alapja annak, hogy képesek legyünk eligazodni a folyamatosan változó, komplex világunkban. A tanulásunk célja, hogy boldog, hasznos és egészséges tagjai legyünk a társadalomnak. Iskolánkban a tanulás három rétege, a tudásszerzés, a megtanultakat elmélyítő önálló gondolkodás és az aktív alkotás egyszerre jelennek meg.

A kerettanterv a célok eléréséhez a miniszter által kiadott kerettantervek tantárgyi struktúráját használja  a moduláris tanulás tartalmi keretezéséhez. A keretezésen azt értjük, hogy a tantárgyak tartalma határozza meg, hogy mivel kell mindenképp a Budapest School iskolákban foglalkozni, mit kell mindenképp megtanulni. 
Az egyes modulok ezen tantárgyak tanulási eredményeinek elérését támogatják.


\subsection{Tantárgyi definiciók és a tanulási eredmények}
A tantárgyak tanulási eredmények felsorolásával adnak tartalmi szabályozást. A tanulási eredmények (learning outcomes) tudás, képesség, kompetencia kontextusában meghatározott kijelentések arra vonatkozóan, hogy a tanulónak mit kell tudnia, mit kell értenie, és mire legyen képes, miután lezárt egy tanulási folyamatot, függetlenül attól, hogy hol, hogyan, mikor szerezte meg ezeket a kompetenciákat \citep{learning_outcomes}.  Vagyis  az egyes modulok különféle tanulási eredmények elérését is támogathatják, ezzel több tantárgy részcéljait is teljesíthetik.

A kerettanterv tantárgyankénti és félévenkénti bontásban adja meg a továbbhaladáshoz elengedhetetlen tanulási eredmények listáját.

A tantárgyi definiciókhoz a miniszter által kiadott kerettanterv ,,elvárt eredmények a tanulási ciklus végén" fejezetek felsorolásait alakítottuk át egységes nyelvezetre, hogy azok valóban kompetenciákat írjanak.

A tantárgyi specifikációk nem térnek ki rész\-letesen a tematikákra. Ez szabadságot ad a tanároknak arra, hogy a tanmenet tekintetében akár jelentős eltérések legyenek addig, amig a miniszter által kiadott kerettanterv mérhető tanulási eredményei teljesülnek. A tanulási eredmény alapú szabályozás folyamatos visszacsatolást tud adni a tanulónak és a tanároknak, megmutatva, melyik tanulási eredményeket kell még elérni a következő szintre való lépéshez.

\paragraph{Tantárgyak szerepe a mindennapokban}

A Budapest School iskoláinak tantárgyi leírásai a miniszter által kiadott kerettanterv alapján készültek. Az egyes tanulási moduloknak a portfólióba való elhelyezését követően háromhavonta összevetjük az elért tanulási eredményeket és  a tantárgyi kötelezően választható tanulási eredményeket, hogy ezáltal folyamatosan monitorozni lehessen az iskolai követelmények és a gyerekek egyéni eredményei közötti egyensúlyt.

A Budapest School iskoláiban a tantárgyak ugyanúgy kapnak szerepet, mint a NAT által definiált kulcskompetenciák, fejlesztési területek: tanár sose mondja azt a gyerekeknek, hogy „most kezdeményezőképességet és vállalkozói kompetenciát fejlesztünk'', hanem a különböző feladatok elvégzése eredményeképp történik a fejlesztés. A Budapest School iskolákban a tantárgyközi tevékenységek vannak előtérben. A tantárgyak a tanulás tartalmi elemeinek forrása és keretei: a tanulandó dolgok listájaként működik. Az, hogy milyen csoportosításban történik a tanulás, az a modulvezetőkre van bízva.

A tantárgyak ezért elsősorban a modulok kiírásakor és azok kimeneti értékelésekor jelennek meg, a mindennapok struktúráját, a napi- és hetirendet azonban a modulok adják. Egyes modulok több tantárgy fejlesztési céljainak is eleget tehetnek, több tantárgy tanulási eredményének elérését is célul tűzhetik ki, összhangban a NAT-tal. A tantárgyaknak ezzel együtt fontos célja, hogy segítse a tanulás tartalmi egyensúlyának fennmaradását. A tanulásszervezők, modulvezetők szakképesítése nem köthető a Budapest School tantárgyaihoz, felelősségük, hogy a saját moduljukban megfelelően tudják szervezni a tanulást, és legfőképp, hogy saját moduljuk megtartására alkalmasak legyenek.