\section{Tantárgyak}
\label{sec:tantargyak}
A Budapest School a ma gyerekeinek kínál olyan oktatást, ami segíti felkészíteni őket a jövő kihívásaira. Információs társadalmunk legnagyobb kihívása az adaptációs képességünk fejlesztése, ez az alapja annak, hogy képesek legyünk eligazodni a folyamatosan változó, komplex világunkban. A tanulásunk célja, hogy boldog, hasznos és egészséges tagjai legyünk a társadalomnak. Iskolánkban a tanulás három rétege, a tudásszerzés, a megtanultakat elmélyítő önálló gondolkodás és az aktív alkotás egyszerre jelennek meg.

A kerettanterv a célok eléréséhez a miniszter által kiadott kerettantervek tantárgyi struktúráját használja  a moduláris tanulás tartalmi keretezéséhez. A keretezésen azt értjük, hogy a tantárgyak tartalma határozza meg, hogy mivel kell mindenképp a Budapest School iskolákban foglalkozni, mit kell mindenképp megtanulni. 
Az egyes modulok ezen tantárgyak tanulási eredményeinek elérését támogatják.

\subsection{Tanulási eredmények -- a formális tanulás alapegységei}
\label{sec:tanulasi_eredmenyek}
A kerettanterv a tantárgyak témaköreit, tartalmát és követelményeit \emph{tanulási eredmények} halmazával adja meg, ezzel igazodva az Nkt.~5.~§ (5) pontjához. A tanulási eredmények (learning out\-comes) tudás, képesség, kompetencia, attitűd kontextusában meghatározott kijelentések arra vonatkozóan, hogy a tanulónak mit kell tudnia, mit kell értenie, és mire legyen képes, miután lezárt egy tanulási folyamatot, függetlenül attól, hogy hol, hogyan, mikor szerezte meg ezeket a kompetenciákat \citep{learning_outcomes}.  Tanulási eredmény a kerettanterv szellemében minden lehet, amit a gyerek egy tanulási folyamat során elsajátított és ezt demonstrálni tudja.

Az eredmény eléréséhez vezető út a modulokon keresztül történik, és a tanulás folyamata történhet az iskolában vagy azon kívül, lehet formális, non-formális vagy informális.
Az egyes modulok különféle tanulási eredmények elérését is támogathatják, ezzel több tantárgy részcéljait is teljesíthetik.
\newpage

A tanulási eredmények több funkciót látnak el a kerettantervben.

\begin{itemize}

      \item A kerettanterv évfolyamonként meghatározza az adott tantárgy teljesítéséhez elérendő tanulási eredményeket. Egy gyerek akkor  léphet egy tantárgyból évfolyamszintet, ha a tantárgyhoz tartozó követelményeket teljesítette.

      \item A tanulási eredmények a modulok (és így a mindennapokban szervezett foglalkozások, órák stb.) építőelemei. Egy-egy modul célját a  tanulásszervezők az elérendő tanulási eredmények	összeválogatásával és saját célokkal, érdeklődéssel kiegészítve adják meg, figyelembe véve az életkori  sajátosságok, az egymásra épülés és az átjárhatóság  követelményeit.
      \item A tanulási eredmények alapján osztályzatok és évfolyamok egy átlátható és egyszerű számítás segítségével megállapíthatóak, ami biztosítja, hogy a Budapest  School tanulója más rendszerben működő iskolába is át tud menni, és a felvételikre is tud jelentkezni.
\end{itemize}

A kerettanterv tantárgyankénti és félévenkénti bontásban adja meg a továbbhaladáshoz elengedhetetlen tanulási eredmények listáját.

A tantárgyi definíciókhoz a miniszter által kiadott kerettanterv ,,elvárt eredmények a tanulási ciklus végén" fejezetek felsorolásait alakítottuk át egységes nyelvezetre, hogy azok valóban kompetenciákat írjanak.

A tantárgyi specifikációk nem térnek ki rész\-letesen a tematikákra. Ez szabadságot ad a tanároknak arra, hogy a tanmenet tekintetében akár jelentős eltérések legyenek addig, amig a miniszter által kiadott kerettanterv mérhető tanulási eredményei teljesülnek. A tanulási eredmény alapú szabályozás folyamatos visszacsatolást tud adni a tanulónak és a tanároknak, megmutatva, melyik tanulási eredményeket kell még elérni a következő szintre való lépéshez.

\subsection{Tantárgyak szerepe a mindennapokban}
A Budapest School iskoláiban a tantárgyak ugyanúgy kapnak szerepet, mint a NAT által definiált kulcskompetenciák, fejlesztési területek: tanár sose mondja azt a gyerekeknek, hogy „most kezdeményezőképességet és vállalkozói kompetenciát fejlesztünk'', hanem a különböző feladatok elvégzése eredményeképp történik a fejlesztés. A Budapest School iskolákban a tantárgyközi tevékenységek vannak előtérben. A tantárgyak a tanulás tartalmi elemeinek forrásai és keretei: a tanulandó dolgok listájaként működnek. Az, hogy milyen csoportosításban történik a tanulás, az a modulvezetőkre van bízva.

A tantárgyak ezért elsősorban a modulok kiírásakor és azok kimeneti értékelésekor jelennek meg, a mindennapok struktúráját, a napi- és hetirendet azonban a modulok adják. Egyes modulok több tantárgy fejlesztési céljainak is eleget tehetnek, több tantárgy tanulási eredményének elérését is célul tűzhetik ki, összhangban a NAT-tal. A tantárgyaknak ezzel együtt fontos célja, hogy segítse a tanulás tartalmi egyensúlyának fennmaradását. A tanulásszervezők, modulvezetők szakképesítése nem köthető a Budapest School tantárgyaihoz, felelősségük, hogy a saját moduljukban megfelelően tudják szervezni a tanulást, és legfőképp, hogy saját moduljuk megtartására alkalmasak legyenek.

\subsection{Modulok és tanulási eredmények}
\label{sec:modulok_es_tanulasi_eredmenyek}
A gyerekek egyik feladata az iskolában, hogy tanulási eredményeket érjenek el. Ezt megtehetik a modulok elvégzésével, vagy más tanulási helyzetekben. A tanulási eredményeket a portfólióban rögzítik. A mentor feladata, hogy folyamatosan kövesse, hogy megfelelő haladás történik-e a portfólióban a tanulási eredmények és a saját célok tekintetében. Az évfolyamszintlépés a portfólióban összegyűlt tanulási eredmények alapján történhet meg.

A modul kecsegtet a gyerekek haladásához releváns tanulási eredményekkel, a gyerekek által meghatározott saját célokkal és olyan kimenettel, amely a portfólióban rögzíthető, legyen az egy alkotás, az elért fizikai vagy szellemi eredmény dokumentációja, vagy egy értékelő visszajelzés. A modulok tehát tartalmaznak tanulási eredményeket, az önálló gondolkodás, szabad alkotás lehetőségét, és teret engednek az alkotásra, létrehozásra.

\paragraph{A modulok különféle tanulási eredmények elérését teszik elérhetővé}

Modulok tervezésekor és összeállításakor a tanulásszervezők a modulvezetővel közösen határozzák meg a modul céljait, de azok meghirdetéséért mindig a tanulásszervezők felelnek. A célok között fel kell sorolni, hogy milyen tanulási eredmények elérését várhatják el a gyerekek a modulon való részvételtől.

Például a 6--8 éves gyerekek számára megtervezett ,,\emph{3d nyomtató használata}'' modul során azon kívül, hogy megismerik a 3d nyomtatás folyamatát, a modul célja, hogy a gyerekek számára elérhetővé tegye a ,,\emph{Kocka, téglatest jellemzőit ismeri, képes őket létrehozni.}'' (Matematika tantárgy, 4. évfolyam 2. félév) tanulási eredményt is.

Lehetőség van egy modul esetében több tantárgyból való
tanulási\linebreak
eredmény kiválasztására, ezzel biztosítva az interdiszciplinaritást, valamint a Budapest School tantárgyi fejlesztési céljaihoz való integrált kapcsolódást.

A tanulási eredmények egy időbeni egymásra épülést feltételeznek,
melyben azonban van lehetőség előre- és hátrafele is lépni. Előre,
ameny\-nyi\-ben a modul meghirdetésekor az arra jelentkező
gyerekcsoportnál a megfelelő előkészítés megtörtént, hátra, amennyiben
ezt ismétlés\slash fel\-zár\-kóz\-ta\-tás jelleggel szükségesnek ítéli a mentor vagy a modult szervező, vezető. Vagyis akkor foglalkozzon egy gyerek a 10~000-es számkörrel, ha a 100-as számkört már begyakorolta. Az egymásra épülésért a modult meghirdető tanulásszervező felel. A példát folytatva a 3d nyomtató használata modul lehetővé teszi, hogy a gyerek elérje a következő eredményeket is: \emph{,,Ismeri a számítógép
      részeinek és perifériáinak funkcióit, azokat önállóan használja.''}
(Harmónia, Informatika, 5. évfolyam 1. félév), és  \emph{,,Használati utasításokat
      értő módon olvas és tart be.''} (Harmónia, Életvitel, 4. évfolyam 2. félév)
\newpage

\paragraph{Új tanulási eredmények}

A gyerekek olyan tanulási eredményt is elérhetnek, ami a modulok céljai között eredetileg nem volt megadva, mert

\begin{itemize}
      \item lehetőségük van egyénileg is tanulni;

      \item tanulási eredményekkel járnak a projektek, az iskolai lét, a közösségi élet és még számos informális és non-formális tanulási helyzet;

      \item egy modul során is alakulhatnak előre nem tervezett
        helyzetek,\linebreak
        amik hozzásegíthetik a gyerekeket tanulási eredmények eléréséhez.
\end{itemize}

Az újonnan létrejövő tanulási eredmények is bekerülnek a portfólióba.

\paragraph{Tanulási eredmények dokumentációja}

Minden modul dokumentálásra\linebreak
kerül, hogy annak célja, elért eredményei nyilvánosak legyenek a Budapest School valamennyi mikroiskolája számára, és ha szükséges, újra meg lehessen hirdetni. A tanulási eredmények egy, a modulhoz kapcsolódó terv-tény összehasonlítás alapján kerülnek meghatározásra. Az elért eredmények újra elérhetőek, amennyiben a folyamatos fejlődés biztosítva van.

\paragraph{Egységes modulok egyedi alkalmazása}

Egy modul elvégzésével egy-egy\linebreak
gyerek más tanulási eredményt is elérhet.

\begin{itemize}
      \item
            Működhet a differenciálás, tehát nem minden gyerek ugyanazt és ugyanúgy csinálja a foglalkozásokon. Egy modulban tud együtt	tanulni az a gyerek, aki még ,,\emph{Ismeri az írott és nyomtatott  betűket''} eredményért dolgozik, és az, aki ,,\emph{Jelöli helyesen a j	hangot 30--40 begyakorolt szóban''.}
      \item
            A modulnak része lehet testre szabható sáv. Például egy
            tudományos kísérletező modulban néhány gyerek a rövid távú
            memória és a	fáradtság kapcsolatáról kutat, a másik
            csoport az esőzés és a	közlekedési dugók kialakulása
            közti kapcsolatot vizsgálja. Minden  gyerek elérheti a
            ,,\emph{valós folyamatokat képes elemezni a folyamathoz
              tartozó függvény grafikonja alapján}''  (forrás,
            Matematika) eredményt, de a ,,\emph{környezettudatos
              közlekedésszemlélet}'' (forrás, Harmónia)\linebreak
            eredményt is elérheti.
      \item
            Egy-egy gyerek saját tanulási célja érdekében extra lépéseket tehet, és olyan eredményeket is el tud érni, amit mások nem.	Például egy modul végén önálló prezentációt, saját kutatási  tervet vagy egy kész működő modellt alkothat.
\end{itemize}

\subsubsection{Kötelező tanulási eredmények}
\label{sec:kotelezo_tanulasi_eredmenyek}
A kerettanterv kötelező tanulási eredményként definiálja mindazokat az eredményeket, melyek a kötelező érettségi tárgyak teljesítéséhez szükségesek. Ezeket minden mikroiskola elérhetővé kell hogy tegye a gyerekek számára a modulok választékában.

Ezek az 1--4. évfolyamszinteken a miniszter által kiadott kerettantervek \emph{Magyar nyelv és irodalom}, \emph{Matematika}, \emph{Idegen nyelv}, \emph{Testnevelés és sport} tantárgyakból származó tantárgyak tanulási eredményei, és 5.~évfolyamszinttől a \emph{Történelem, társadalmi és állampolgári ismeretek} tantárgy eredményei. További kötelező tanulási eredményként jelennek meg 9.~évfolyamtól a választott érettségi tantárgyhoz kapcsolódó eredmények. Ezek a tanulási eredmények megtalálhatók a kerettanterv tantárgyainak elérhető eredményei között.

\paragraph{Kötelező modulok}
A kerettanterv és a pedagógiai program is előírhat kötelező modulokat a mikroiskolák számára. Ilyenek például a 11.~évfolyamszinten belépő érettségire felkészítő modulok, a minden mikroiskolára egységes pedagógiai program tetszőleges kötelező modult írhat elő. Így lehet biztosítani a kötelező tartalmi elemek és foglalkozás -- például elsősegélynyújtás vagy a nemzetiségekkel való ismerkedés -- elérhetőségét.

\subsubsection{Monitorozás}

Kötelező elérni az eredményeket? Nem tudunk hatalmi szóval tanulásra bírni gyereket, mert lehet, hogy annyira nem akarja, vagy nincs meg hozzá a képessége. A kerettanterv a tanároknak ad keretet. Azonban a fenntartó által üzemeltetett rendszerrel az iskola  monitorozza a haladást, és ha valaki a kötelező tanulási elemekkel nem halad, akkor az iskola erre felhívja a figyelmét. Mivel a többség haladni fog, ezért előre tudja az iskola jelezni, hogy le fog szakadni a többiektől, és túl nagy lesz az évfolyamszint-különbség közöttük. Ezekben az esetekben a mentortanárnak, a gyereknek és a szülőnek reagálnia kell a helyzetre. A fenntartó által működtetett monitorozó és minőségfejlesztő rendszerről \aref{sec:minosegbiztositas} fejezet ír részletesen.


\subsection{Tantárgyi struktúra és óraszámok}
\label{sec:tantargyi_struktura}
\paragraph{Heti óraszámok} 
A Budapest School közösségi tanulási élményeket és modulokat szervez a gyerekeknek, egyúttal lehetőséget ad arra, hogy a gyerekek a közösen kialakított szabályaik mentén, a tanulásszervezők felügyeletével a Budapest School székhelyén vagy egyes telephelyein, vagy más, erre alkalmas tanulási környezetben tartózkodjanak. A közösségben együtt töltött idő tanulásnak, fejlődésnek minősül akkor is, ha az nem egy modulhoz kapcsolódik, hanem az ebéd élvezetéhez, vagy épp a parkban a lehulló falevelek neszének megfigyeléséhez.

A gyerekek, a tanítási szüneteket leszámítva, általában naponta 8 órát tartózkodnak az iskolában.\footnote{Ez alól több kivétel lehet: külső foglalkozások, otthoni, egyéni tanulás. Ezekről külön megállapodást kell kötni a mentorral.} Ezekben az időkben vannak a tanítási órák, foglalkozások, szakkörök, műhelyek. Az egyes mikroiskolák ettől 20\%-ban bármelyik irányban eltérhetnek, ha ez segíti a tanulásszervezők munkáját és a gyerekek fejlődését. Így hetente minimum $5 \cdot 8 \cdot 0,8 = 32$ órát, maximum $48$ órát töltenek az iskolában.

Ennek az 1--4.~évfolyamszinten körülbelül a felét, azaz 18--22
órát,\linebreak
majd később 3--4.~évfolyamonszinten 20--26 órát; az 5--12.~évfolyamszinten pedig kétharmad részét, azaz 24--32 órát töltik a gyerekek előre eltervezett módon, azaz modulokkal. A többi időben a tanárok vezetése és felügyelete mellett szabadon alkotnak, játszanak, pihennek, közösségi életet élnek. Ugyanezek a számok a miniszter által kiadott kerettantervekben 1. évfolyamon 25 és a 10. évfolyamon 36 órát tesznek ki.

\paragraph{Modulok és a tantárgyi óraszámok}
A modulok során több tantárgyi tananyagot is érinthetnek a modul résztvevői. Egy modul így több tantárgyi órát is lefed, ráadásul ez óraszám megtakarítással is jár. 5 óra angolul tartott dráma foglalkozás egyszerre számíthat 5 óra \emph{magyar irodalom és nyelvnek} és 5 óra \emph{idegennyelv} órának. A tantárgyköziségből spórolt óraszámokra a kerettanterv óvatos előírást ad: egy egységnyi idő alatt átlagban legalább 1,25 egységnyi tantárgyi óraszámot kell teljesíteni (szemben az előző példában szereplő kettes szorzóval).

A modulok tantárgyi óraszámát a teljes modul hosszára kell számítani, és nem hetente. Egy összevont természettudományi modul, ami érinti a kémiát és a fizikát is, nem kell, hogy minden héten járjon kémia tanulási eredménnyel. Több tantárgyat lefedő modul esetén a tantárgyi óraszámokat úgy kell számolni, hogy először meg kell állapítani, hogy átlagban az idő hány százalékában foglalkozik a modul egy-egy tantárgy anyagával, majd a modul teljes hosszából becsülhető a tantárgyi óraszám. Például egy \emph{tudományos kísérletezés} modul során az idő 30\%-ban foglalkozunk kémiával, 40\%-ban fizikával és 30\%-ában szociálpszichológiával. A modul egy trimeszteren keresztül tart, kéthetente 4 órában. Ebből számolható a modul teljes hossza, ami itt $\frac{12}{2}x4 = 24$ óra. Ez hetente $24x0,3 = 7,2$ óra kémiának felel meg. 
A példa is mutatja, hogy a \emph{kerettanterv megengedi, hogy nem egészszámú óraszámokkal dolgozzon az iskola}. 

\subsection{Tantárgyi óraszámok trimeszterenként}
A Budapest School modulalapú tanulásszervezéséhez jobban illik trimeszterenként megadni az elvárt óraszámokat, mert ahogy \aref{sec:tanev_ritmusa}.~fejezet is mutatja, nem minden hét ugyanolyan az iskolában. A kerettanterv a miniszter által kiadott kerettantervek óraszámait veszi alapul, annak heti óraszámait szorozza fel kilenccel. 

\begin{landscape}

\begin{table}[]
\scalebox{0.8}{

  \begin{tabular}{l|l|l|l|l|l|l|l|l|l|l|l|l}
  
                                                        & \multicolumn{12}{l}{\textbf{Évfolyamszintek}}                                                                                                                                                           \\ \hline
    \textbf{Tantárgyak}                                          & 1                                     & 2           & 3           & 4           & 5           & 6           & 7           & 8           & 9           & 10          & 11          & 12          \\ \hline
    Biológia - egészségtan                              &     &     &     &     &     &     & 18  & 9   &     & 18  & 18  & 18  \\\hline
    Dráma és tánc/Mozgóképkultúra és médiaismeret       &     &     &     &     & 9   &     &     &     & 9   &     &     &     \\ \hline
    Életvitel és gyakorlat                              & 9   & 9   & 9   & 9   &     &     &     &     &     &     &     & 9   \\\hline
    Ének-zene                                           & 18  & 18  & 18  & 18  & 9   & 9   & 9   & 9   & 9   & 9   &     &     \\\hline
    Erkölcstan                                          & 9   & 9   & 9   & 9   & 9   & 9   & 9   & 9   &     &     &     &     \\\hline
    Etika                                               &     &     &     &     &     &     &     &     &     &     & 9   &     \\\hline
    Fizika                                              &     &     &     &     &     &     & 18  & 9   & 18  & 18  & 18  &     \\\hline
    Földrajz                                            &     &     &     &     &     &     & 9   & 18  & 18  & 18  &     &     \\\hline
    I. idegen nyelv                                     &     &     &     & 18  & 27  & 27  & 27  & 27  & 27  & 27  & 27  & 27  \\\hline
    II. idegen nyelv                                    &     &     &     &     &     &     &     &     & 27  & 27  & 27  & 27  \\\hline
    Informatika                                         &     &     &     &     &     & 9   & 9   & 9   & 9   & 9   &     &     \\\hline
    Kémia                                               &     &     &     &     &     &     & 9   & 18  & 18  & 18  &     &     \\\hline
    Környezetismeret                                    & 9   & 9   & 9   & 9   &     &     &     &     &     &     &     &     \\\hline
    Magyar nyelv és irodalom                            & 63  & 63  & 54  & 54  & 36  & 36  & 27  & 36  & 36  & 36  & 36  & 36  \\\hline
    Matematika                                          & 36  & 36  & 36  & 36  & 36  & 27  & 27  & 27  & 27  & 27  & 27  & 27  \\\hline
    Művészetek                                          &     &     &     &     &     &     &     &     &     &     & 18  & 18  \\\hline
    Technika, életvitel és gyakorlat                    &     &     &     &     & 9   & 9   & 9   &     &     &     &     &     \\\hline
    Természetismeret                                    &     &     &     &     & 18  & 18  &     &     &     &     &     &     \\\hline
    Testnevelés és sport                                & 45  & 45  & 45  & 45  & 45  & 45  & 45  & 45  & 45  & 45  & 45  & 45  \\\hline
    Történelem, társadalmi és állampolgársági ismeretek &     &     &     &     & 18  & 18  & 18  & 18  & 18  & 18  & 27  & 27  \\\hline
    Vizuális kultúra                                    & 18  & 18  & 18  & 18  & 9   & 9   & 9   & 9   & 9   & 9   &     &     \\\hline \hline
    \textbf{Összesen}                                   & 207 & 207 & 198 & 216 & 225 & 216 & 243 & 243 & 270 & 279 & 252 & 234
    

  \end{tabular}
}
  \caption{A minimális óraszámok trimeszterekre számolva. A táblázatban szereplő számok a miniszter által kiadott kerettanterv óraszámai alapján készültek.}  
  \label{tbl:oraszamok}

\end{table}

\end{landscape}
