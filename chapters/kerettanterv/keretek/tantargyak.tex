\section{Tantárgyak}
\label{sec:tantargyak}
A Budapest School a ma gyerekeinek kínál olyan oktatást, ami segíti felkészíteni őket a jövő kihívásaira. Információs társadalmunk legnagyobb kihívása az adaptációs képességünk fejlesztése, ez az alapja annak, hogy képesek legyünk eligazodni a folyamatosan változó, komplex világunkban. A tanulásunk célja, hogy boldog, hasznos és egészséges tagjai legyünk a társadalomnak. Iskolánkban a tanulás három rétege, a tudásszerzés, a megtanultakat elmélyítő önálló gondolkodás és az aktív alkotás egyszerre jelennek meg.

A kerettanterv a célok eléréséhez a miniszter által kiadott kerettantervek tantárgyi struktúráját használja  a moduláris tanulás tartalmi keretezéséhez. A keretezésen azt értjük, hogy a tantárgyak tartalma határozza meg, hogy mivel kell mindenképp a Budapest School iskolákban foglalkozni, mit kell mindenképp megtanulni. 
Az egyes modulok ezen tantárgyak tanulási eredményeinek elérését támogatják.




\subsection{Tantárgyi definiciók és a tanulási eredmények}
A tantárgyak tanulási eredmények felsorolásával adnak tartalmi szabályozást. A tanulási eredmények (learning outcomes) tudás, képesség, kompetencia kontextusában meghatározott kijelentések arra vonatkozóan, hogy a tanulónak mit kell tudnia, mit kell értenie, és mire legyen képes, miután lezárt egy tanulási folyamatot, függetlenül attól, hogy hol, hogyan, mikor szerezte meg ezeket a kompetenciákat \citep{learning_outcomes}.  Vagyis  az egyes modulok különféle tanulási eredmények elérését is támogathatják, ezzel több tantárgy részcéljait is teljesíthetik.

A kerettanterv tantárgyankénti és félévenkénti bontásban adja meg a továbbhaladáshoz elengedhetetlen tanulási eredmények listáját.

A tantárgyi definiciókhoz a miniszter által kiadott kerettanterv ,,elvárt eredmények a tanulási ciklus végén" fejezetek felsorolásait alakítottuk át egységes nyelvezetre, hogy azok valóban kompetenciákat írjanak.

A tantárgyi specifikációk nem térnek ki rész\-letesen a tematikákra. Ez szabadságot ad a tanároknak arra, hogy a tanmenet tekintetében akár jelentős eltérések legyenek addig, amig a miniszter által kiadott kerettanterv mérhető tanulási eredményei teljesülnek. A tanulási eredmény alapú szabályozás folyamatos visszacsatolást tud adni a tanulónak és a tanároknak, megmutatva, melyik tanulási eredményeket kell még elérni a következő szintre való lépéshez.

\paragraph{Tantárgyak szerepe a mindennapokban}

A Budapest School iskoláinak tantárgyi leírásai a miniszter által kiadott kerettanterv alapján készültek. Az egyes tanulási moduloknak a portfólióba való elhelyezését követően háromhavonta összevetjük az elért tanulási eredményeket és  a tantárgyi kötelezően választható tanulási eredményeket, hogy ezáltal folyamatosan monitorozni lehessen az iskolai követelmények és a gyerekek egyéni eredményei közötti egyensúlyt.

A Budapest School iskoláiban a tantárgyak ugyanúgy kapnak szerepet, mint a NAT által definiált kulcskompetenciák, fejlesztési területek: tanár sose mondja azt a gyerekeknek, hogy „most kezdeményezőképességet és vállalkozói kompetenciát fejlesztünk'', hanem a különböző feladatok elvégzése eredményeképp történik a fejlesztés. A Budapest School iskolákban a tantárgyközi tevékenységek vannak előtérben. A tantárgyak a tanulás tartalmi elemeinek forrása és keretei: a tanulandó dolgok listájaként működik. Az, hogy milyen csoportosításban történik a tanulás, az a modulvezetőkre van bízva.

A tantárgyak ezért elsősorban a modulok kiírásakor és azok kimeneti értékelésekor jelennek meg, a mindennapok struktúráját, a napi- és hetirendet azonban a modulok adják. Egyes modulok több tantárgy fejlesztési céljainak is eleget tehetnek, több tantárgy tanulási eredményének elérését is célul tűzhetik ki, összhangban a NAT-tal. A tantárgyaknak ezzel együtt fontos célja, hogy segítse a tanulás tartalmi egyensúlyának fennmaradását. A tanulásszervezők, modulvezetők szakképesítése nem köthető a Budapest School tantárgyaihoz, felelősségük, hogy a saját moduljukban megfelelően tudják szervezni a tanulást, és legfőképp, hogy saját moduljuk megtartására alkalmasak legyenek.

\subsection{Tantárgyi struktúra és óraszámok}

\paragraph{Heti óraszámok} 
A Budapest School közösségi tanulási élményeket és modulokat szervez a gyerekeknek, egyúttal lehetőséget ad arra, hogy a gyerekek a közösen kialakított szabályaik mentén tanulásszervezők felügyeletével a Budapest School székhelyén vagy egyes telephelyein, vagy más, erre alkalmas tanulási környezetben tartózkodjanak. A közösségben együtt töltött idő tanulásnak, fejlődésnek minősül akkor is, ha az nem egy modulhoz kapcsolódik, hanem az ebéd élvezetéhez, vagy épp a parkban a lehulló falevelek neszének megfigyeléséhez.

A gyerekek, a tanítási szüneteket leszámítva, általában naponta 8 órát tartózkodnak az iskolában.\footnote{Ez alól több kivétel lehet: külső foglalkozások, otthoni, egyéni tanulás. Ezekről külön megállapodást kell kötni a mentorral.} Ezekben az időkben vannak a tanítási órák, foglalkozások, szakkörök, műhelyek. Az egyes mikroiskolák ettől 20\%-ban bármelyik irányban eltérhetnek, ha ez segíti a tanulásszervezők munkáját és a gyerekek fejlődését. Így hetente minimum $5 \cdot 8 \cdot 0,8 = 32$ órát, maximum $48$ órát töltenek az iskolában.

Ennek az 1--4.~évfolyamszinten körülbelül a felét, azaz 18--22 órát, majd később 3--4.~évfolyamonszinten 20--26 órát; az 5--12.~évfolyamszinten pedig kétharmad részét, azaz 24--32 órát töltik a gyerekek előre eltervezett módon, azaz modulokkal. A többi időben a tanárok vezetése és felügyelete mellett szabadon alkotnak, játszanak, pihennek, közösségi életet élnek. Ugyanezek a számok a miniszter által kiadott kerettantervekben 1. évfolyamon 25 és a 10. évfolyamon 36 órát tesznek ki.

\paragraph{Modulok és a tantárgyi óraszámok}
A modulok során több tantárgyi tananyagot is érinthetnek a modul résztvevői. Egy modul így több tantárgyi órát is lefed, ráadásul ez óraszám megtakarítással is jár. 5 óra angolul tartott dráma foglalkozás egyszerre számíthat 5 óra \emph{magyar irodalom és nyelvnek} és 5 óra \emph{idegennyelv} órának. A tantárgyköziségből spórolt óraszámokra a kerettanterv óvatos előírást ad: egy egységnyi idő alatt átlagban legalább 1.25 egységnyi tantárgyi óraszámot kell teljesíteni (szemben az előző példában szereplő kettes szorzóval).

A modulok tantárgyi óraszámát a teljes modul hosszára kell számítani, és nem hetente. Egy összevont természettudományi modul, ami érinti a kémiát és a fizikát is, nem kell, hogy minden héten járjon kémia tanulási eredménnyel. Több tantárgyat lefedő modul esetén a tantárgyi óraszámokat úgy kell számolni, hogy először meg kell állapítani, hogy átlagban az idő hány százalékában foglalkozik a modul egy-egy tantárgyi anyagával, majd a modul teljes hosszából és a modul hosszából becsülhető a tantárgyi óraszám. Például egy \emph{tudományos kísérletezés} modul során az idő 30\%-ban foglalkozunk kémiával, 40\%-ban fizikával és 20\%-ban szociálpszichológiával. A modul egy trimeszteren keresztül tart, kéthetente 4 órában. Ebből számolható a modul teljes hossza, ami itt $\frac{12}{2}x4 = 24$ óra. Ez hetente $24x0.3 = 7.2$ óra kémiának felel meg. 
A példa is mutatja, hogy a \emph{kerettanterv megengedi, hogy nem egészszámú óraszámokkal dolgozzon az iskola}. 

\section{Tantárgyi óraszámok trimeszterenként}
A Budapest School modulalapú tanulásszervezéhez jobban illik trimeszterenként megadni az elvárt óraszámokat, mert ahogy \aref{sec:tanev_ritmusa}.~fejezet is mutatja, nem minden hét úgyanolyan az iskolában. A kerettanterv a miniszter által kiadott kerettantervek óraszámait veszi alapul, annak heti óraszámait szorozza fel kilenccel. 

\begin{landscape}
\begin{table}[]
  \begin{tabular}{l|l|l|l|l|l|l|l|l|l|l|l|l}
  
                                                        & \multicolumn{12}{l}{\textbf{Évfolyamszintek}}                                                                                                                                                           \\ \hline
    \textbf{Tantárgyak}                                          & 1                                     & 2           & 3           & 4           & 5           & 6           & 7           & 8           & 9           & 10          & 11          & 12          \\ \hline
    Biológia - egészségtan                              &     &     &     &     &     &     & 18  & 9   &     & 18  & 18  & 18  \\\hline
    Dráma és tánc/Mozgóképkultúra és médiaismeret       &     &     &     &     & 9   &     &     &     & 9   &     &     &     \\ \hline
    Életvitel és gyakorlat                              & 9   & 9   & 9   & 9   &     &     &     &     &     &     &     & 9   \\\hline
    Ének-zene                                           & 18  & 18  & 18  & 18  & 9   & 9   & 9   & 9   & 9   & 9   &     &     \\\hline
    Erkölcstan                                          & 9   & 9   & 9   & 9   & 9   & 9   & 9   & 9   &     &     &     &     \\\hline
    Etika                                               &     &     &     &     &     &     &     &     &     &     & 9   &     \\\hline
    Fizika                                              &     &     &     &     &     &     & 18  & 9   & 18  & 18  & 18  &     \\\hline
    Földrajz                                            &     &     &     &     &     &     & 9   & 18  & 18  & 18  &     &     \\\hline
    I. idegen nyelv                                     &     &     &     & 18  & 27  & 27  & 27  & 27  & 27  & 27  & 27  & 27  \\\hline
    II. idegen nyelv                                    &     &     &     &     &     &     &     &     & 27  & 27  & 27  & 27  \\\hline
    Informatika                                         &     &     &     &     &     & 9   & 9   & 9   & 9   & 9   &     &     \\\hline
    Kémia                                               &     &     &     &     &     &     & 9   & 18  & 18  & 18  &     &     \\\hline
    Környezetismeret                                    & 9   & 9   & 9   & 9   &     &     &     &     &     &     &     &     \\\hline
    Magyar nyelv és irodalom                            & 63  & 63  & 54  & 54  & 36  & 36  & 27  & 36  & 36  & 36  & 36  & 36  \\\hline
    Matematika                                          & 36  & 36  & 36  & 36  & 36  & 27  & 27  & 27  & 27  & 27  & 27  & 27  \\\hline
    Művészetek                                          &     &     &     &     &     &     &     &     &     &     & 18  & 18  \\\hline
    Technika. életvitel és gyakorlat                    &     &     &     &     & 9   & 9   & 9   &     &     &     &     &     \\\hline
    Természetismeret                                    &     &     &     &     & 18  & 18  &     &     &     &     &     &     \\\hline
    Testnevelés és sport                                & 45  & 45  & 45  & 45  & 45  & 45  & 45  & 45  & 45  & 45  & 45  & 45  \\\hline
    Történelem, társadalmi és állampolgársági ismeretek &     &     &     &     & 18  & 18  & 18  & 18  & 18  & 18  & 27  & 27  \\\hline
    Vizuális kultúra                                    & 18  & 18  & 18  & 18  & 9   & 9   & 9   & 9   & 9   & 9   &     &     \\\hline \hline
    \textbf{Összesen}                                   & 207 & 207 & 198 & 216 & 225 & 216 & 243 & 243 & 270 & 279 & 252 & 234
    

  \end{tabular}
  \caption{A minimális óraszámok trimeszterekre számolva. A táblázatban szereplő számok a miniszter által kiadott kerettanterv óraszámai alapján készültek.}  
  \label{tbl:oraszamok}
\end{table}

\end{landscape}