\ifkerettanterv
\chapter{A Budapest School szervezete}
\else
\chapter{Az iskola struktúrája}

\fi

\begin{quote}
Tanulni bárhol lehet, ha a közeg biztonságos, és adottak a feltételek arra, hogy az egyén és a csoport fejlődhessen.
\end{quote}
\ifkerettanterv
\section{A Budapest School mikroiskola-hálózata}
\else
\section{A mikroiskola-hálózat}

\fi

 A Budapest School mikroiskolák hálózataként működik. A mikroiskola a Budapest School iskolahálózatának legkisebb egysége. Nem egy önálló intézmény, és nem a telephely szinonimája.

Az egyes mikroiskolákban a gyerekek kevert korosztályú tanulóközösségként,
érdeklődésük és képességeik alapján, együtt és egymástól tanulnak és fejlődnek.
A tanulóközösség fontos célja, hogy biztonságot, támogatást nyújtson, és \emph{így} segítse a közösség tagjainak a minőségi tanulását. Az egyes mikroiskolák egymással is kapcsolatban vannak,
egymástól tanulnak, így a gyerekek tanulási helye változhat.

 \paragraph{Épületek, tagintézménzek és telephelyek}
A Budapest School iskola egy székhellyel és több tagintézménnyel, illetve telephellyel működhet. Fontos azonban, hogy egy telephelyen több mikroiskola is működhet, és egy mikroiskola több telephely adottságait is kihasználhatja.

A kerettanterv tudatos szándéka, hogy az épületet és a
tanulás szervezeti formáját ne kösse össze szorosan. Egy-egy mikroiskola az épületére úgy gondol, mint egy átmeneti bérleményre vagy egy helyre, amit most meglátogat.
Változhat, hogy egy-egy épületet éppen melyik mikroiskola használja.

A tanulás helyszínének változtathatósága lehetővé teszi, hogy a múzeumpedagógiát, a tudományos kutatóközpontokkal való együttműködést, az erdei iskolák világát, a sportegyesületek tevékenységeit, vagy más külső helyszínen megvalósuló szakköröket a Budapest School gyerekek számára a mindennapok integrált részvé tegyük. Fontos kiemelnünk, hogy mindeközben mindenkinek szüksége van bázisra, biztonságot adó otthonra: ezért van minden Budapest School gyereknek és tanárnak egy elsődleges helye.
\ifkerettanterv
  \section{A Budapest School fenntartója}
  A Budapest School fenntartójának a nemzeti köznevelésről szóló 2011. évi CXC. törvényben (a továbbiakban: Nkt)  szabályozottakon felül feladata
  \begin{itemize}
  \item a Budapest School hálózatának építése, működési struktúrájának
  fejlesztése, az adminisztratív és szabályozási rendszer kialakítása,
  valamint az egyes mikroiskolákban a gyerekek tanulását, fejlődését segítő folyamatok megalkotása;
  \item  az adminisztratív és jogi folyamatok kezelése;
  \item  a tanárok kiválasztási folyamatának kezelése és folyamatos tanulásuk szervezése;
  \item a minőségbiztosítási és fejlesztési rendszer kialakítása és működtetése.
\end{itemize}

 Azaz a fenntartó nemcsak a fenntartásért, hanem a fenntartható fejlődésért is felel, és ebben támogatja az egyes mikroiskolákat.

  \section{A Budapest School mikroiskolái}
\fi
  A tanulási folyamat működtetéséért a Budapest School egyes mikroiskolái
  felelősek. Az iskolákat tanulásszervező-tanárok (a különböző pedagógus szerepek kibontása a \ref{sec:tanarok}. fejezetben található) egy csoportja alkotja és vezeti. Így a
  mikroiskolák vezetéséért a tanulásszervező-tanárok felelősek.

  Az egyes mikroiskolák különböznek egymástól abban, hogy az oda járó gyerekek pontosan mit és mikor
  tanulnak vagy alkotnak. Azonban a következő alapelvek az összes mikroiskolára
  érvényesek.

\paragraph{Az iskoláknak \emph{saját fókuszuk van}.}

    Van olyan mikroiskola, amely a fejlesztési célok eléréséhez és az egyéni célok
    mentén már 6 éves gyerekek tanulásánál a robotika eszközeit használja,
    másutt drámafoglalkozásokkal fejlesztik 12 éves gyerekek a szövegértésüket és
    éntudatukat. A
    mikroiskola-rendszerben rejlik annak a lehetősége, hogy egy adott tanulási
    környezetben a hangsúlyok úgy váltakozhassanak a csoport és az egyén
    érdeklődését követve, hogy közben a tanulási egyensúly fennmaradjon a
    tantárgyak fejlesztési területei között. Az iskolák nemcsak abban térnek el
    egymástól, hogy kevert korcsoportban, más korosztályú gyerekek, más
    érdeklődések mentén, és ily módon más célokat követve tanulnak, hanem
    területileg, regionálisan is eltérőek lehetnek.


  \paragraph{Az iskolákban a tanulók nagymértékben befolyásolják, hogy mit és hogyan tanulnak és alkotnak.}

    A tanárok választási lehetőségeket dolgoznak ki, amikből a gyerekek (a
    mentoruk és szüleik segítségével) a saját céljaikat, érdeklődésüket leginkább
    támogató \emph{egyéni tanulási tervet} alkotnak. Az iskolákban (a tanárok által
    meghatározott kereteken belül) megfér egymással több, különböző egyéni céllal
    rendelkező gyerek.

    Eltérhet, hogy egy-egy gyerek mit tanul, ezért az is, hogy mikor és hogyan
    sajátítja el a szükséges ismereteket: egy közösségben megfér a központi
    felvételire fókuszáló 11 éves gyerek, és az olyan is, aki ekkor inkább a
    Mine\-craft programozásában akar elmélyülni, ezért más képességek fejlesztésével
    lassabban halad. A tanárok feladata és felelőssége, hogy olyan közösségeket
    válogassanak össze, amelyek kellően diverzek, és mégis jól működnek, a gyerekek
    igényeit és a kerettanterv céljait egyaránt megfelelően kielégítik.

  \paragraph{A mikroiskolák kevert korosztályos közösségek.}

    A Budapest Schoolban a gyerekek nemcsak a saját korcsoportjukban, hanem kevert korosztályok szerinti csoportokban is tanulnak egy-egy mikroiskolában, hasonlóan a Montessori-féle kevert korosztályos csoportokhoz, vagy az \emph{önállósági szintek} (independece level  \citep{indepence_level}) alapján szervezett tanulócsoportokhoz. A csoportok létrehozásakor arra törekszünk, hogy olyan gyerekek tanuljanak együtt, akik tudják egymást támogatni a tanulásban.

  \paragraph{A mikroiskolák együtt fejlődő közösségek.}

    Úgy fejlődnek, mintha első lépésben óvodapedagógusok által vezetett óvodai csoportok
    alakulnának át kéttanítós alsós osztályokká. Majd amikor a tanuláshoz újabb
    tanárokra van szükségük, akkor bővül a tanárcsapat. Így jöhet létre akár 50 gyerek és 5-8
    tanár közössége. Amikor a fejlődésükhöz újabb tanárra van szükségük a
    gyerekeknek -- például egy speciális képesség erősítéséhez --, akkor vagy a
    Budapest School másik mikroiskolájából jövő tanártól, vagy egy külsős szakembertől
    tanulhatnak. Amikor az érettségire készülve maguk alkotnak gyakorló csoportot, akkor a tanulásukat akár már önmaguknak is megszervezhetik.

  \paragraph{Különféle tanulási struktúrák jöhetnek létre a mikroiskolákon belül.}

    A közösséget kisebb csoportokra bonthatjuk, ha a tanulásszervezés ezáltal
    hatékonyabb. Egyes modulokban egy-egy projektre szerveződnek a gyerekek,
    ilyenkor gyakran az eltérő képességű és életkorú gyerekek is megférnek egymás
    mellett. Más moduloknál a csoportokat általában képességszint alapján hozza
    létre a tanár. Ilyen lehet a másodfokú egyenletek megoldóképletét megismerő
    csoport, az írni tanulók csoportja, vagy egy angol nyelvű újság szerkesztésére
    és megírására alakult modul, ahol a nyelvismeretnek és a szövegalkotási
    képességnek már egy olyan szintjén kell lenni, hogy a projektnek jól mérhető
    kimenete lehessen.

  \paragraph{A mikroiskolák diverz, integratív közösségek.} A Budapest School iskolák társadalmi,
  kulturális és gazdasági értelemben is egyik fő céljuknak tartják az integrációt addig,
  amíg az a közösség céljait szolgálja.

  \paragraph{A Budapest School mikroiskolái tanuló közösségek.} Mindig, minden módszer,
  folyamat fejleszthető, ezért a tanárok feladata, lehetősége, hogy az aktuális
  helyzethez illő legalkalmasabb módszert válasszák a tanulás segítéséhez.

  A Budapest School mikroiskolák célja, hogy jól átlátható, követhető és
  folyamatosan fejlődő folyamattá váljon a tanulás mind a tanuló, mind a tanár, mind
  a szülő részéről. Kerettantervünk folyamatszabályozást nyújt, nem kimeneti
  szabályozást. A kimenet a gyerekek és a közösség képességeitől, céljaitól és érdeklődésétől,
  valamint a társadalmi szabályozási környezettől függ.

\ifkerettanterv
\input{chapters/kerettanterv/tanari_szerepek}
\fi
