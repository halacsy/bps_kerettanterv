A Budapest School Általános Iskola és Gimnázium 12 évfolyammal működő nevelési-oktatási intézményként ellátja az általános iskola és a gimnázium feladatait.

Az iskola a miniszter által kiadott kerettantervek tananyagtartalmát kínálja a gyerekeknek. A tanulás szervezését, azaz a pedagógiáját a motiváció \citep{pink2011drive}\footnote{Az irodalomjegyzéket lásd a \pageref{sec:bibliographyk}. oldalon}, a fejlődési szemlélet \citep{growthmindset} és az elmélyült gyakorlás \citep{ericsson2016peak} pszichológiai kutatási eredményei, az  OECD, azaz a Gazdasági Együttműködési és Fejlesztési Szervezet \emph{Az iskolázás a jövőben}  (\emph{Schooling for Tomorrow}) programjának eredményei \citep{2006schooling} és a modern tanulásszervezési paradigmák, mint az önvezérelt \citep{mitra2012beyond} és a személyre szabott \citep{khan2012one} tanulás alapján határozza meg.

Az iskola egyszerre akar a gyerekek számára egy önvezérelt és személyreszabott tanulási környezetet biztosítani, ami képes agilisen reagálni a gyerekek és a környezet igényeire és egy stabil, biztonságos, kiszámítható rendszert is adni, ami biztosítja az iskola és más iskolák közötti átjárhatóságot, a továbbtanulást. Ezt a két szándékot ötvözi a Budapest School modell.

\clearpage
\includepdf{pics/CELOK_EGYENSULYA.JPG}

\paragraph{Egyéni és közösségi szempontok} A Budapest School modellben fontos szempont, hogy
hol, milyen
módon, kitől és mit tanulhat egy Budapest Schoolba járó gyerek.
A gyerekek igényeire gyorsan reagáló mikroiskola egy buborék a gyerek körül. Ebben a körben a Budapest School Modell a tanulás folyamatát szabályozza, a \emph{mit tanulunk?} kérdés mellett a \emph{hogyan szervezzük meg a tanulást?} kérdésre ad választ. A tanulás során a gyerekek a Budapest School Modellben részletezett módon az érdeklődésüknek megfelelően specifikus tanulási egységeket, vagyis modulokat végeznek el, melyek eredményeit a saját portfóliójukban gyűjtik össze. Így a gyerekek tanulási útja a portfólió fejlődésével nyomon követhető, és a portfólió tartalma alapján megállapítható a gyerek aktuális tudása, képessége. Az iskola fő funkciója emellett mindvégig az aktív tanulás, saját fejlődésük kereteinek megtalálása, a folyamatosan újragondolt saját célok állítása, és e célok irányába történő haladás marad.
 
A Budapest School Modell a gyerekek, tanárok és a szülők közös döntésére bízza, hogy a gyerekek mit és hogyan tanulnak a modellben meghatározott kereteken belül. A modell a specifikus célkitűzés-tervezés, a tanulás, és az arra történő reflektálás módját írja le, vagyis a tanulás folyamatát rögzíti, míg annak pontos tartalmában szabadságot enged.

\paragraph{Társadalmi szempontok}  Ezt a szabadságot keretezik a társadalmi normák és jogszabályi elvárások, hogy biztosítva legyen a mikroiskolán kívüli boldogulása is a gyerekeknek. A Budapest School Modell
 a miniszter által kiadott kerettantervekre \citep{ofi:kerettanterv} épül. A helyi tanterv a tantárgyak tartalmát tanulási eredmények halmazaként adja meg. A tanulási eredmények féléves bontása, és a tantárgyi specifikációk lehetővé teszik a személyreszabott portfóliók osztályzatokra váltását egy átlátható folyamaton keresztül.

A Budapest School Modell transzparensé teszi az iskolában működő kétszíntű struktúrát: a gyerekközpontú, személyreszabott, saját tervezésen alapuló mindennapi tanulási élményt folyamatosan leképezzük a miniszter által kiadott kerettantervek tantárgyi struktúrájának, A félévenkénti osztályzatok biztosítják az átjárhatóságot és a továbbtanuláshoz szükséges feltételeket. A kettős rendszer, a személyre szabott belső buborék és a kiszámíthatóságot adó külső kör biztosítja, hogy a 12.~évfolyam végén a gyerekeknek lehetőségük van arra, hogy érettségi vizsgát tegyenek.

\paragraph{Pedagógiai módszerek}
A Budapest School Model semleges a pedagógiai módszerekkel kapcsolatban, a tanár feladatának tekinti, hogy mindig az optimálisnak tűnő tanulási, tanítási, gyakorlási módszert válassza. Ezért ez a modell nem beszél arról, hogy a modulokat (foglalkozásokat, tanórákat) milyen pedagógiai módszer alapján szervezi a tanár.

A Budapest School Model az iskolába járókat gyerekeknek hívja, nem tanulóknak és nem diákoknak. Ennek fő oka, hogy a rendszerünkben a tanárok és a szülők is tanulók, sőt az egész iskola egy tanuló szervezet, így ez a szöveg nem akarja  kizárólag az iskola egyik szereplőjére alkalmazni ezt a szót. Másodsorban a modell hangsúlyozza, hogy a \emph{család} fontos szerepet kap a Budapest School rendszerében: az iskolában a szülők, a gyerekek és a tanárok együttműködésben dolgoznak a fejlődésért. Azok a gyerekek, akik tanulmányaik vége felé felnőtté érnek a Budapest Schoolban, tanulók is maradnak, és talán egy kicsit gyerekek is, ezért a szóhasználaton miattuk sem változtatunk. Az ő esetükben a gyerek az iskolába járó tanulót jelenti.


