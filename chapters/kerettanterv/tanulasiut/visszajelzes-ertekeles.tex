\section{Visszajelzés, értékelés}
\label{sec:ertekeles}
Ahhoz, hogy hatékony legyen a tanulás, fejlődés, fontos, hogy a gyerekek, tanárok és szülők is tudják, hogy
\begin{enumerate}
      \item hol tart most egy gyerek, mit tud most,
      \item hova akar vagy kell eljutni, azaz, mi a célja,
      \item mi kell ahhoz, hogy elérje a célját.
\end{enumerate}
Ezek mellett mindenkinek hinnie kell abban, hogy odafigyeléssel, gyakorlással a gyerek meg tud tanulni egy konkrét dolgot. Fontos, hogy magas legyen a gyerekek énhatékonysága,  erős legyen az önbizalmuk, és nem szabad félniük a hibázástól, a nem-tudástól, mert a tanulás első lépése, hogy elfogadjuk, hogy valamit nem tudunk. Azaz fontos, hogy fejlődésfókuszú gondolkodásuk (growth mindset) \citep{growthmindset} legyen, azaz
\begin{enumerate}
      \setcounter{enumi}{3}
      \item hinniük kell, hogy el tudják érni a céljukat.
\end{enumerate}

Egy visszajelzés, értékelés akkor jó és hasznos, azaz hatékony, ha ebben a négy dologban segít. Mai tudásunk szerint ehhez:
\begin{itemize}
      \item Rendszeresen visszajelzést kell kapniuk és adniuk.
      \item A tanulási céloknak és visszajelzéseknek minél specifikusabbaknak kell lenniük (azaz például ne a 8. oszályos \emph{matematikatudást} értékeljük, hanem hogy mennyire képes valaki \emph{fagráfokat használni feladatmegoldások során}\footnote{Ez a konkrét példa a matematika tantárgy egyik tanulási eredménye.}).
      \item A \emph{,,hol tartok most''} diagnózisnak mindig cselekvésre, viselkedésre, aktív tevékenységre kell vonatkoznia. Ne az legyen a visszajelzés, hogy \emph{,,ügyes vagy egyenletekből''}, hanem \emph{,,gyorsan és       pontosan oldottad meg a 4 egyenletet''}. A legjobb, amikor a visszajelzés konkrét megfigyelésen alapul, és tudni, hogy mikor, hol történt az eset: \emph{,,amikor társaiddal Minecraftban házat építettél, akkor       pontosan kiszámoltad a ház területét''.}
      \item Ha a cél nem a mások legyőzése, akkor a visszajelzés se tartalmazzon olyan állítást, ami másokhoz hasonlít (így kerüljük a \emph{tehetség} szót is, aminek bevett definíciója szerint az átlagnál jobb képesség). A másokhoz való szint felmérése akkor (és csak akkor) fontos, amikor a cél egy versenyszituációban jó eredményt elérni.

      \item A gyerek legyen részese a visszajelzésnek. Értse, tudja, hogy miért kapta azt a visszajelzést, a legjobb, ha -- amikor ezt a képességei engedik -- önmaga képes elvégezni a visszajelzést, vagy annak egy részét.
      \item A visszajelzésnek transzparensen hatással kell lennie a tanulásszervezésre. Legyen része a folyamatnak, és a gyerek, tanár és a szülő is értse, hogy a visszajelzés alapján mit és hogyan csinálunk másképp.
\end{itemize}

\paragraph{Többszintű visszajelzés} A Budapest School iskolákban a gyerekek
többféle visszajelzést kapnak. \begin{enumerate}
      \item Minden modul elvégzése után a modul céljai, témája, fókusza alapján a modulvezetők visszajelzést adnak a tanulásról, eredményekről, viselkedésről.
      \item Trimeszterenként a mentorok visszajelzést adnak arról, hogy a gyerek általában hogyan haladt a tanulási célok felé.
      \item Ennek része, hogy a tantárgyi tanulási eredmények alapján hogyan haladt a gyerek a tantárgyak évfolyamszinthez tartozó követelményeinek teljesítésében. Az évfolyamok, mint elérhető szintek Budapest School-értelmezését \aref{sec:evfolyamok}. fejezet tárgyalja.
      \item A mentorok irányításával a gyerekek visszajelzést kapnak arról, hogyan működnek a közösségben.
\end{enumerate}

\paragraph{Érdemjegyek, osztályzatok helyett értékelő táblázatok} A Budapest
School visszajelzéseinek sokkal részletesebbeknek kell lenniük, mint azt a tantárgyi érdemjegyek és osztályzatok lehetővé teszik, ezért azok helyett a kerettanterv értékelő táblázatokat (angolul rubric) alkalmaz. Az értékelő táblázatban szerepelnek az értékelés szempontjai és szempontonkénti szintek, rövid leírásokkal. Ezek alapján a gyerekek maguk is láthatják, hogy hol tartanak, hogyan javíthatnak még a munkájukon. A táblázatok formája minden visszajelzés esetén (értsd modulonként, célonként) változtatható.