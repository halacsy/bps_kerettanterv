\section{Saját tanulási célok}
\label{sec:tanulasi_celok}

Minden gyerek megfogalmazza és háromhavonta újrafogalmazza a \emph{saját tanulási céljait}: eredményeket, amelyeket el akar érni, képességeket, amelyeket fejleszteni akar, szokásokat, amelyeket ki akar alakítani. A saját célok elfogadásakor a gyerek és a mentora a szülőkkel együtt \emph{tanulási szerződést} köt.

Csak olyan célok kerülhetnek a saját célok közé, amelyek minden érintettnek biztonságosak, és amelyek összhangban vannak a tantárgyi fejlesztési célokkal és tanulási eredményekkel. A szerződésben rögzíthetőek tanulási eredményekre vonatkozó megállapodások, tantárgyi évfolyamszintekre vonatkozó elvárások (pl. ,,\emph{haladjon egy évfolyamszintet egy év alatt}'' vagy ,,\emph{készüljön fel emelt szintű érettségire}''), és a tantárgyi rendszeren kívüli célok és feladatok.

%Fontos megkötés, hogy a saját tanulási célok legalább a felének \ifkerettanterv
%      \aref{sec:tantargyi_tanulasi_eredmenyek}. fejezetben
%\else
%      a kerettanterv Tantárgyi tanulási eredmények fejezetében
%\fi felsorolt tanulási eredmények elérésére kell vonatkoznia. A másik fele szabadon alakítható.

Háromhavonta a tanulásszervezők és a gyerekek megállnak, reflektálnak az elmúlt időszakra, és a tapasztalatok, valamint az elért célok ismeretében és az új célok figyelembevételével újratervezik, újraszervezik a foglalkozások rendjét, tehát azt, hogy mikor és mit csinálnak majd a gyerekek az iskolában. A mindennapi tevékenység során tapasztalt élmények, alkotások, elvégzett feladatok, kitöltött vizsgák, tehát mindaz, ami a gyerekekkel történik, bekerül a portfóliójukba. Még az is, amit nem terveztek meg előre.

A gyerekeket a mentoruk segíti a saját célok kitűzésében, a különböző választásoknál, a portfólióépítésben, a reflektálásban. A tanulási célok kitűzése az önirányított tanulás fokozatos fejlődésével és az életkor előrehaladtával folyamatosan egyre önállóbb tevékenységgé válik. Tanulási útján, céljai kitűzésében a mentor kíséri végig a gyerekeket.

A Budapest School személyre szabott tanulásszervezésének jellegzetessége, hogy a gyerekek a saját céljuk irányába haladnak, az adott célhoz az adott kontextusban leghatékonyabb úton. Tehát mindenki rendelkezik saját célokkal, még akkor is, ha egy közösség tagjainak céljai a tantárgyi tanulási eredmények azonossága, vagy a hasonló érdeklődés miatt akár  80\% átfedést mutatnak.

A NAT műveltségi területeiben megfogalmazott követelmények teljesítése is célja a tanulásnak, a tanulás fő irányítója azonban más. Mi azt kérdezzük a gyerekektől, hogy \emph{ezenfelül} mi az ő személyes céljuk.