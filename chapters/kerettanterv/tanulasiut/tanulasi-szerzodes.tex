\section{A tanulási szerződés}

A tanulási szerződés az előbbiekben említett gyerek-mentor-szülő közötti megállapodás, ami rögzíti
\begin{enumerate}
      \item a gyerek, a mentor (iskola) és a szülő igényeit, elvárásait;

            ezek lehetnek: \emph{,,szeretném, ha a gyerekem naponta olvasna''} típusú folyamatra vonatkozó kérések, vagy erősebb \emph{,,változtatnod       kell a viselkedéseden, ha a közösségben akarsz maradni''} igények, határok megfogalmazása;

      \item a gyerek céljait a következő trimeszterre, vagy a tanév végéig;

      \item a gyerek, mentorok (iskola) és szülő vállalásait, amivel támogatják a cél elérését és a felek igényének elérését.

\end{enumerate}

A tanulási szerződésre jellemző, hogy
\begin{itemize}
      \item A kitűzött célokat minél specifikusabban, mérhetőbben kell megfogalmazni. Javasolt az OKR  (Objectives and Key Results, azaz	Cél és Kulcs Eredmények) \citep{okr} vagy a SMART (Specific, Measurable, Achievable, Relevant, Time-bound, azaz Specifikus,  Mérhető, Elérhető, Releváns és Időhöz kötött) \citep{wiki:smart} technika alkalmazása, hogy minél specifikusabb, teljesíthetőbb, tervezhetőbb és könnyen mérhető célokat tűzzenek ki.

      \item A kitűzött célokban való megállapodást követően, megállapodást kell kötni arról is, hogy ki és mit tesz azért, hogy a gyerek a célokat elérje.

      \item A mentor a teljes mikroiskolát (a többi tanárt, a közösséget) képviseli a megállapodás során.
\end{itemize}

A tanulási szerződést néha hívjuk \emph{megállapodásnak} is. A megállapodás és szerződés szavakat ez a kerettanterv szinonimának tekinti. A \emph{learn\-ing con\-tract} az önirányított tanulást hangsúlyozó felnőttképzéssel foglalkozó irodalomban bevett szakkifejezés már a 80-as évektől \citep{Malcolm77}. Ennek a magyar nyelvben inkább a szerződés felel meg. Egy másik szakterületen, a pszichoterápiás munkában a terápiás szerződések megkötésekor a közös munka kereteinek kialakítását és fenntarthatóságát hangsúlyozzák \citep{pszichoterapia}. Erre is utalunk a tanulási szerződés elnevezéssel. Van, amikor a \emph{hármas szerződés} kifejezést használjuk, hangsúlyozva, hogy mind a három szereplőnek elfogadhatónak kell tartania a szerződés tartalmát.
