
\section{Harmónia tantárgy tartalma }
Harmónia tantárgy által definiált tanulási eredmények a NAT pedagógiai szakaszai, évfolyamszintek és tématerületek szerint csoportosítva.

\subsection{1--4. évfolyam}
\begin{small}
  \begin{longtable}{c | p{2cm} |  p{11cm} }
    \textbf{Évf.} & \textbf{Témater.} & \textbf{Tanulási eredmény} \\ \hline \hline
    \endhead

              1--2 & erkölcstan & Életkorának megfelelő reális képpel rendelkezik saját külső tulajdonságairól, tisztában van legfontosabb személyi adataival. \\ \hline
              1--2 & erkölcstan & Átlátja társas viszonyainak alapvető szerkezetét. \\ \hline
              1--2 & erkölcstan & Érzelmileg kötődik környezetéhez és a körülötte élő emberekhez. \\ \hline
              1--2 & erkölcstan & Kapcsolatba lép és a partner személyét figyelembe véve kommunikál környezete tagjaival, különféle beszédmódokat alkalmaz. \\ \hline
              1--2 & erkölcstan & A beszélgetés során betartja az udvarias társalgás elemi szabályait. \\ \hline
              1--2 & erkölcstan & Kifejezi a szeretetkapcsolatok fontosságát. \\ \hline
              1--2 & erkölcstan & Értelmezi a hagyományok társadalomban és a közösségekben betöltött szerepét. \\ \hline
              1--2 & erkölcstan & Érzelmeit a közösség számára elfogadható formában nyilvánítja ki. \\ \hline
              1--2 & erkölcstan & Átéli a természet szépségét, és érti, hogy felelősek vagyunk a körülöttünk lévő élővilágért. \\ \hline
              1--2 & erkölcstan & Kifejezi érzéseit, gondolatait és fantáziaképeit vizuális, mozgásos vagy szóbeli eszközökkel. \\ \hline
              1--2 & erkölcstan & Tisztában van vele és érzelmileg felfogja azt a tényt, hogy más gyerekek sokszor egészen más körülmények között élnek, mint ő. \\ \hline
              1--2 & erkölcstan & Megkülönbözteti egymástól a valóságos és a virtuális világot. \\ \hline
              1--2 & testnevelés & Az alapvető tartásos és mozgásos elemeket felismeri és pontosan végrehajtja. \\ \hline
              1--2 & testnevelés & Megnevezi a testrészeket. \\ \hline
              1--2 & testnevelés & Felismeri a fizikai terhelés és a fáradás jeleit. \\ \hline
              1--2 & testnevelés & Különbséget tesz a jó és a rossz testtartás között álló és ülő helyzetben. \\ \hline
              1--2 & testnevelés & Ismeri és betartja a mozgásórák rendszabályait, tisztában van a balesetvédelmi szempontokkal. \\ \hline
              1--2 & testnevelés & Mozgásos gyakorlás közben odafigyel a társaira, célszerűen használja az eszközöket. \\ \hline
              1--2 & testnevelés & Alkalmazza a stressz és feszültségoldás alapgyakorlatait. \\ \hline
              1--2 & testnevelés & Játékszabályokat és a játékszerepeket haszál játékos feladatok során. \\ \hline
              1--2 & testnevelés & Biztonságosan használja a sporteszközöket. \\ \hline
              1--2 & testnevelés & A feladat-végrehajtások során pontosságra, célszerűségre, biztonságra törekszik. \\ \hline
              1--2 & testnevelés & Örömmel használ sporteszközöket, vesz részt a szabad és gyakorló mozgásban. \\ \hline
              1--2 & testnevelés & Dinamikus és statikus egyensúlyi helyzetekben talajon és emelt eszközökön stabil. \\ \hline
              1--2 & testnevelés & A zenei ritmust különféle ritmikus mozgásokban egyénileg, párban és csoportban is követi. \\ \hline
              1--2 & testnevelés & Tánc közben párhoz, társakhoz térben alkalmazkodik. \\ \hline
              1--2 & testnevelés & A futó-, ugró- és dobóiskolai alapgyakorlatokat ismeri és végrehajtja. \\ \hline
              1--2 & testnevelés & Felismeri a sporthelyzetek megoldásából és a játékfolyamatból adódó örömöt, élményt és a tanulási lehetőséget. \\ \hline
              1--2 & testnevelés & Felismeri a csapatérdek szerepét az egyéni érdekkel szemben, vagyis a közös cél fontossága tudatosul. \\ \hline
              1--2 & testnevelés & Ismeri a sportszerű viselkedés néhány jellemzőjét. \\ \hline
              1--2 & testnevelés & Képes késleltetni és gátolni a mozgását. \\ \hline
              1--2 & testnevelés & Az általános uszodai rendszabályokat, baleset-megelőzési szempontokat ismeri, és betartja. \\ \hline
              1--2 & testnevelés & Hason és háton siklik, lebeg. \\ \hline
              1--2 & testnevelés & Bátran vízbe ugrik. \\ \hline
              1--2 & testnevelés & Ismeri a természeti környezetben történő sportolás néhány egészségvédelmi és környezettudatos viselkedési szabályát. \\ \hline
              1--2 & testnevelés & Kreatívan használja a sporteszközöket a játéktevékenység során. \\ \hline
              1--2 & testnevelés & A saját komfortérzetének és a közösség szabályainak megfelelően öltözködik. Ismeri és felméri az időjárás hatását az emberi szervezetre. \\ \hline
              1--2 & technika & A természeti, a társadalmi és a technikai környezet megismert jellemzőit felsorolja. \\ \hline
              1--2 & technika & Tapasztalattal rendelkezik az ember természetátalakító (építő és romboló) munkájáról. \\ \hline
              1--2 & technika & Érti a család szerepének, időbeosztásának és egészséges munkamegosztásának jelentőségét, nem alkot sztereotípiákat ezek kapcsán. \\ \hline
              1--2 & technika & Tudatosan kezeli és egészséges veszélyérzettel közelít a háztartási és a közlekedési veszélyekhez. \\ \hline
              1--2 & technika & Példákat mond az egészséges, korszerű táplálkozás és a célszerű öltözködés terén. \\ \hline
              1--2 & technika & A hétköznapjainkban használatos anyagokat felismeri, tulajdonságaikat megállapítja megfigyelések és vizsgálatok alapján. A tapasztalatokat megfogalmazza. \\ \hline
              1--2 & technika & Az anyagalakításhoz kapcsolódó foglalkozásokat megnevezi, jellemzőit ismeri. \\ \hline
              1--2 & technika & Képlékeny anyagokat, papírt, faanyagokat, fémhuzalt, szálas anyagokat, textileket magabiztosan alakít. \\ \hline
              1--2 & technika & Mintakövetéssel, önállóan épít. \\ \hline
              1--2 & technika & Az elvégzett munkáknál biztonságosan, balesetmentesen használ eszközöket. Munka közben rendet tart. \\ \hline
              1--2 & technika & Az úttesten való átkelés szabályait tudatosan alkalmazza. A kulturált és balesetmentes járműhasználat (tömegközlekedési eszközökön és személygépkocsiban történő utazás) szabályait alkalmazza. \\ \hline
              3--4 & erkölcstan & Életkorának megfelelő szinten reális képe van saját külső és belső tulajdonságairól, és késztetést érez arra, hogy fejlessze önmagát. \\ \hline
              3--4 & erkölcstan & Figyel másokra, szavakkal is kifejezi érzéseit és gondolatait, bekapcsolódik csoportos beszélgetésekbe. \\ \hline
              3--4 & erkölcstan & Tartós kapcsolatot alakít ki társaival, törekszik e kapcsolatok ápolására, és ismer olyan eljárásokat, amelyek segítségével a kapcsolati konfliktusok konstruktív módon feloldhatók. \\ \hline
              3--4 & erkölcstan & Kifejezi kötődését a saját és mások kultúrájához. \\ \hline
              3--4 & erkölcstan & Érti és elfogadja, hogy az emberek sokfélék és sokfélék a szokásaik, a hagyományaik is; tiszteletben tartja ezt a tényt és kíváncsi a sajátjától eltérő kulturális jelenségekre. \\ \hline
              3--4 & erkölcstan & Érti, hogy mi a jelentősége a szabályoknak a közösségek életében; betartja a megértett szabályokat; részt vesz a szabályok kialakításában. \\ \hline
              3--4 & erkölcstan & Érti, hogy a Föld mindannyiunk közös otthona és hogy bolygónk számos olyan értékkel rendelkezik, amit elődeink hoztak létre. \\ \hline
              3--4 & erkölcstan & A körülötte zajló eseményekre és a különféle helyzetekre a sajátjától eltérő nézőpontból is rátekint. \\ \hline
              3--4 & erkölcstan & Jelenségeket, eseményeket és helyzeteket erkölcsi nézőpontból értékel. \\ \hline
              3--4 & testnevelés & Egyszerű, általános bemelegítő gyakorlatokat hajt végre önállóan. \\ \hline
              3--4 & testnevelés & A nyújtó, erősítő, ernyesztő és légzőgyakorlatok pozitív hatásait ismeri. \\ \hline
              3--4 & testnevelés & A játékok, versenyek során személyes felelősséget vállal a magatartási szabályrendszer betartásában és a sportszerű viselkedés terén. \\ \hline
              3--4 & testnevelés & Ismer stressz- és feszültségoldó gyakorlatokat. \\ \hline
              3--4 & testnevelés & Alapvető hely- és helyzetváltoztató mozgásokat folyamatosan és magabiztosan hajt végre. \\ \hline
              3--4 & testnevelés & Alkalmaz bonyolultabb játékfeladatokat, játékszerepeket és játékszabályokat. \\ \hline
              3--4 & testnevelés & Ígényli a pontosságot, a célszerűséget és biztonságot. \\ \hline
              3--4 & testnevelés & Rendszeresen használ sporteszközöket szabadidős céllal. \\ \hline
              3--4 & testnevelés & Részben önállóan tervezetten 3-6 torna- és/vagy táncelemet összeköt zenére is. \\ \hline
              3--4 & testnevelés & Gurulások, átfordulások, fordulatok, dinamikus kar-, törzs- és lábgyakorlatok közben többnyire biztosan uralja egyensúlyi helyzetét. \\ \hline
              3--4 & testnevelés & Követi a tempóváltozásokat. \\ \hline
              3--4 & testnevelés & Ismeri a tanult táncok, dalok, játékok eredeti közösségi funkcióját. \\ \hline
              3--4 & testnevelés & A futó-, ugró- és dobóiskolai gyakorlatok vezető műveleteit ismeri, azokat végre tudja hajtani. \\ \hline
              3--4 & testnevelés & Különböző intenzitású és tartamú mozgásokat tart fenn játékos körülmények között, illetve játékban. \\ \hline
              3--4 & testnevelés & Tartósan fut egyéni tempóban, akár járások közbeiktatásával is. \\ \hline
              3--4 & testnevelés & A tanult sportjátékok alapszabályait ismeri. \\ \hline
              3--4 & testnevelés & Törekszik a játék legcélszerűbb helyzetmegoldására. \\ \hline
              3--4 & testnevelés & A csapatérdeknek megfelelő összjátékra törekszik. \\ \hline
              3--4 & testnevelés & Érték számára a sportszerű viselkedés. \\ \hline
              3--4 & testnevelés & Néhány önvédelmi fogást bemutat párban. \\ \hline
              3--4 & testnevelés & Bemutat előre-, oldalra- és hátraesést tompítással. \\ \hline
              3--4 & testnevelés & Ismeri és betartja a grundbirkózás alapszabályait. \\ \hline
              3--4 & testnevelés & Sportszerű küzdésre, asszertív viselkedésre törekszik. \\ \hline
              3--4 & testnevelés & Kezeli a saját indulatát a sportjáték során. Nem viselkedik agresszívan. \\ \hline
              3--4 & testnevelés & Érti az önvédelmi feladatok célját. \\ \hline
              3--4 & testnevelés & Egy úszásnemben 25 métert biztonságosan leúszik. \\ \hline
              3--4 & testnevelés & Legalább négy szabadidős mozgásforma alapszabályait ismeri. \\ \hline
              3--4 & testnevelés & A szabadidős mozgásformákat önszervező módon használja szabad játéktevékenység során. \\ \hline
              3--4 & technika & Mindennapokban nélkülözhetetlen praktikus ismereteket – háztartási praktikákat – gyakorol és alkalmaz. \\ \hline
              3--4 & technika & A használati utasításokat érti és betartja. \\ \hline
              3--4 & technika & A hétköznapjainkban használatos anyagokat felismeri, tulajdonságaikat megállapítja megfigyelések és vizsgálatok alapján. A tapasztalatokat megfogalmazza. \\ \hline
              3--4 & technika & Egyszerű tárgyakat készít mintakövetéssel. \\ \hline
              3--4 & technika & A munkaeszközöket célszerűen megválasztja és szakszerűen, balesetmentesen használja. \\ \hline
              3--4 & technika & A gyalogosokra vonatkozó közlekedési szabályokat tudatosan és készségszinten alkalmazza. \\ \hline
              3--4 & technika & Biciklizik. \\ \hline
      \end{longtable}
\end{small}


\subsection{5--8. évfolyam}
\begin{small}
  \begin{longtable}{c | p{2cm} |  p{11cm} }
    \textbf{Évf.} & \textbf{Témater.} & \textbf{Tanulási eredmény} \\ \hline \hline
    \endhead

              5--6 & erkölcstan & Tisztában van az egészség megőrzésének jelentőségével, és tudja, hogy maga is felelős ezért. \\ \hline
              5--6 & erkölcstan & Tudatában van annak, hogy az emberek sokfélék, elfogadja és értékeli a testi és lelki vonásokban megnyilvánuló sokszínűséget, valamint az etnikai és kulturális különbségeket. \\ \hline
              5--6 & erkölcstan & Gondolkodik saját személyiségjegyein, törekszik a megalapozott véleményalkotásra, illetve vélekedéseinek és tetteinek utólagos értékelésére. \\ \hline
              5--6 & erkölcstan & Gondolkodik rajta, hogy mit tekint értéknek; tudja, hogy ez befolyásolja a döntéseit, és hogy időnként választania kell még a számára fontos értékek között is. \\ \hline
              5--6 & erkölcstan & Képes különféle szintű kapcsolatok kialakítására és ápolására; átlátja saját kapcsolati hálójának a szerkezetét; rendelkezik a konfliktusok kezelésének és az elkövetett hibák kijavításának néhány, a gyakorlatban jól használható technikájával. \\ \hline
              5--6 & erkölcstan & Fontos számára a közösséghez való tartozás érzése; képes átlátni és elfogadni a közösségi normákat. \\ \hline
              5--6 & erkölcstan & Nyitottan fogadja a sajátjától eltérő véleményeket, szokásokat és kulturális, illetve vallási hagyományokat. \\ \hline
              5--6 & erkölcstan & Érzékeli, hogy a társadalom tagjai különféle körülmények között élnek, képes együttérzést mutatni az elesettek iránt, és lehetőségéhez mérten szerepet vállal a rászorulók segítésében. Megbecsüli a neki nyújtott segítséget. \\ \hline
              5--6 & erkölcstan & Tisztában van azzal, hogy az emberi tevékenység hatással van a környezet állapotára, és törekszik rá, hogy életvitelével minél kevésbé károsítsa a természetet. \\ \hline
              5--6 & erkölcstan & Ismeri a modern technika legfontosabb előnyeit és hátrányait, s felismeri magán a függőség kialakulásának esetleges előjeleit. \\ \hline
              5--6 & erkölcstan & Tisztában van vele, hogy a reklámok a nézők befolyásolására törekszenek, kritikusan viszonyul a különféle médiaüzenetekhez. \\ \hline
              5--6 & erkölcstan & Érti, hogy a világ megismerésének többféle útja van (különböző világképek és világnézetek), s ezek mindegyike a maga sajátos eszközeivel közelít ugyanahhoz a valósághoz. \\ \hline
              5--6 & testnevelés & Önállóan vesz részt a sportolás során felmerülő szervezési feladatok végrehajtásában. \\ \hline
              5--6 & testnevelés & Nyolc-tíz gyakorlattal, részben önállóan melegít be. A bemelegítés és a levezetés szempontjait ismeri. \\ \hline
              5--6 & testnevelés & Figyel a biomechanikailag helyes testtartás megőrzésére. \\ \hline
              5--6 & testnevelés & Stressz- és feszültségoldó módszereket alkalmaz. \\ \hline
              5--6 & testnevelés & Választott úszásnemben készségszinten, egy másikban 25 méteren vízbiztosan, folyamatosan úszik. \\ \hline
              5--6 & testnevelés & Alkalmazza a testnevelési játékokban, játékos feladatokban és a sportjátékban is a játékok technikai és taktikai készletét. \\ \hline
              5--6 & testnevelés & Szabálykövetően, fegyelmezetten, együttműködően vesz részt a sportjátékokban. \\ \hline
              5--6 & testnevelés & A rajtokat az indítási jeleknek megfelelően végrehajtja. \\ \hline
              5--6 & testnevelés & Ugrásoknál a nekifutás távolságát és sebességét kialakítja tapasztalatok felhasználásával. \\ \hline
              5--6 & testnevelés & Ismeri az atlétikai versenyek alapvető szabályait. \\ \hline
              5--6 & testnevelés & A tanult akadályleküzdési módokat és feladatokat biztonságosan végrehajtja. \\ \hline
              5--6 & testnevelés & A dinamikus és statikus egyensúlygyakorlatokat a képességnek megfelelő magasságon, szükség esetén segítségadás mellett hajtja végre. \\ \hline
              5--6 & testnevelés & A ritmikus sportgimnasztika egyszerű tartásos és mozgásos gyakorlatelemeinek bemutatása. \\ \hline
              5--6 & testnevelés & del
 \\ \hline
              5--6 & testnevelés & A gyakorlatvégzések során előforduló hibákat elismeri és a javítási megoldásokat elfogadja. \\ \hline
              5--6 & testnevelés & A balesetvédelmi utasításokat betartja. \\ \hline
              5--6 & testnevelés & Ismer alternatív környezetben űzhető sportokat. \\ \hline
              5--6 & testnevelés & Képes a tanult alternatív környezetben űzhető sportágak alaptechnikai gyakorlatainak bemutatására. \\ \hline
              5--6 & testnevelés & A sportágak űzéséhez szükséges eszközöket biztonságosan használja. \\ \hline
              5--6 & testnevelés & A természeti és környezeti hatások és a szervezet alkalmazkodó-képessége közötti összefüggéseket felismeri. \\ \hline
              5--6 & testnevelés & A természeti környezetben történő sportolás egészségvédelmi és környezettudatos viselkedési szabályait elfogadja és betartja. \\ \hline
              5--6 & testnevelés & A mostoha időjárási feltételek mellett is aktívan részt vesz a foglalkozásokon. \\ \hline
              5--6 & technika & Megfogalmazza tapasztalatait a környezet elemeiről, állapotáról, belátja a környezetátalakító tevékenységgel járó felelősséget. \\ \hline
              5--6 & technika & Vannak tapasztalatai az ételkészítéssel, élelmiszerekkel összefüggő munkatevékenységekről. \\ \hline
              5--6 & technika & Ételkészítés és tárgyalkotás során a technológiákat helyesen alkalmazza, az eszközöket szakszerűen, biztonságosan használja. \\ \hline
              5--6 & technika & Elemi műszaki rajzi ismereteit alkalmazza a tervezés és a kivitelezés során. \\ \hline
              5--6 & technika & Képes az elkészült produktumok (ételek, tárgyak, modellek) reális értékelésére, a hibák felismerésére, a javítás, fejlesztés lehetőségeinek meghatározására. \\ \hline
              5--6 & technika & Az ember közvetlen tárgyi környezetének megőrzésére, alakítására vonatkozó szükségleteket felismeri. A tevékenységek és beavatkozások következményeit előzetesen, helyesen felismeri, az azzal járó felelősséget belátja. \\ \hline
              5--6 & technika & A tárgyi környezetben végzett tevékenységeket biztonságossá, környezettudatossá, takarékossá és célszerűvé formálja. \\ \hline
              5--6 & technika & A gyalogos és kerékpáros közlekedés KRESZ szerinti szabályait, valamint a tömegközlekedés szabályait ismeri és biztonságosan alkalmazza. \\ \hline
              5--6 & technika & Elsajátította a kerékpár karbantartásához szükséges ismereteket. \\ \hline
              5--6 & technika & A vasúti közlekedésben képes biztonságosan és udvariasan részt venni. \\ \hline
              5--6 & technika & Képes tájékozódni közúti és vasúti menetrendekben, útvonaltérképeken. Útvonaltervet olvas, készít. \\ \hline
              5--6 & informatika & Ismeri a számítógép részeinek és perifériáinak funkcióit, azokat önállóan használja. \\ \hline
              5--6 & informatika & A könyvtárszerkezetben tájékozódik, mozog, könyvtárat vált és tud fájlt keresni. \\ \hline
              5--6 & informatika & Segítséggel multimédiás oktatóprogramokat használ. \\ \hline
              5--6 & informatika & Az iskolai hálózatba belép, onnan kilép, ismeri és betartja a hálózat használatának szabályait. \\ \hline
              5--6 & informatika & Ismeri egy vírusellenőrző program kezelését. \\ \hline
              5--6 & informatika & Ismeri a szövegszerkesztés alapfogalmait, önállóan elvégzi a leggyakoribb karakter- és bekezdésformázásokat. \\ \hline
              5--6 & informatika & Használja a szövegszerkesztő nyelvi segédeszközeit. \\ \hline
              5--6 & informatika & Ismeri egy bemutatókészítő program egyszerű lehetőségeit, tud rövid bemutatót készíteni. \\ \hline
              5--6 & informatika & Felismeri az összetartozó adatok közötti egyszerű összefüggéseket. \\ \hline
              5--6 & informatika & Segítséggel tantárgyi, könyvtári, hálózati adatbázisokat használ, különféle adatbázisokban keres. \\ \hline
              5--6 & informatika & Különböző dokumentumokból származó részleteket saját munkájában el tud helyezni. \\ \hline
              5--6 & informatika & A problémamegoldáshoz szükséges információt összegyűjti. \\ \hline
              5--6 & informatika & Ismeri a problémamegoldás alapvető lépéseit. \\ \hline
              5--6 & informatika & Önállóan vagy segítséggel algoritmust készít. \\ \hline
              5--6 & informatika & Egyszerű programot ír. \\ \hline
              5--6 & informatika & Egy fejlesztőrendszert alapszinten használ. \\ \hline
              5--6 & informatika & A problémamegoldás során együttműködik társaival. \\ \hline
              5--6 & informatika & Használja a böngészőprogram főbb funkcióit. \\ \hline
              5--6 & informatika & Felnőtt segítséggel megadott szempontok szerint információt tud keresni. \\ \hline
              5--6 & informatika & A keresési találatokat értelmezi. \\ \hline
              5--6 & informatika & Az elektronikus levelezőrendszert önállóan kezeli. \\ \hline
              5--6 & informatika & Elektronikus és internetes médiumokat használ. \\ \hline
              5--6 & informatika & Az interneten talált információkat menteni tudja. \\ \hline
              5--6 & informatika & Ismeri a netikett szabályait. \\ \hline
              5--6 & informatika & Ismeri az informatikai biztonsággal kapcsolatos fogalmakat. \\ \hline
              5--6 & informatika & Adatvédelemmel kapcsolatos fogalmakat ismer. \\ \hline
              5--6 & informatika & Ismeri az adatvédelem érdekében alkalmazható lehetőségeket. \\ \hline
              5--6 & informatika & Az informatikai eszközök etikus használatára vonatkozó szabályokat ismeri. \\ \hline
              5--6 & informatika & Feltünteti az információforrásokat a saját dokumentumokban. \\ \hline
              5--6 & informatika & Különböző tantárgyi feladataihoz megadott forrásokat megtalálja és további releváns forrásokat keres. \\ \hline
              5--6 & informatika & Nyomtatott és elektronikus forrásokban megtalálja a feladatok megoldásához szükséges információkat. \\ \hline
              5--6 & informatika & Eldönti, hogy az iskolai vagy a lakóhelyi könyvtár szolgáltatásait vegye igénybe. \\ \hline
              7--8 & erkölcstan & Érti, hogy az ember egyszerre biológiai és tudatos lény, akit veleszületett képességei alkalmassá tesznek a tanulásra, mások megértésre és önmaga vizsgálatára. \\ \hline
              7--8 & erkölcstan & Érti, hogy az emberek viselkedését, döntéseit tudásuk, gondolataik, érzelmeik, vágyaik, nézeteik és értékrendjük egyaránt befolyásolják. \\ \hline
              7--8 & erkölcstan & Reflektál saját maga és mások gondolataira, motívumaira és tetteire. \\ \hline
              7--8 & erkölcstan & Életkorának megfelelő szinten ismeri önmagát, hosszabb távú elképzeléseinek kialakításakor képes reálisan felmérni a lehetőségeit. \\ \hline
              7--8 & erkölcstan & Képes erkölcsi szempontok szerint mérlegelni különféle cselekedeteket, és el tudja viselni az értékek közötti választással együtt járó belső feszültséget. \\ \hline
              7--8 & erkölcstan & Képes ellenállni a csoportnyomásnak, és saját értékrendje szerinti autonóm döntéseket hozni. \\ \hline
              7--8 & erkölcstan & Tisztában van vele, hogy baráti és párkapcsolataiban felelősséggel tartozik a társaiért. \\ \hline
              7--8 & erkölcstan & Van saját identitásélménye, amely nemzethez, európaisághoz, kisebbségi léthez vagy lokalitáshoz is köthető. \\ \hline
              7--8 & erkölcstan & Nyitott más kultúrák értékeinek megismerésére és befogadására. \\ \hline
              7--8 & erkölcstan & Érti a szabályok szerepét az emberi együttélésben, és belátás alapján igyekszik alkalmazkodni hozzájuk; igényli azonban, hogy maga is alakítója lehessen a közösségi szabályoknak. \\ \hline
              7--8 & erkölcstan & Van elképzelése saját jövőjéről, és tisztában van vele, hogy céljai eléréséért erőfeszítéseket kell tennie. Képes megfogalmazni egy trimeszterre tanulási célokat, és azokat végig is viszi. \\ \hline
              7--8 & erkölcstan & Képes megfogalmazni, hogy mi okoz neki örömet, illetve rossz érzést. \\ \hline
              7--8 & erkölcstan & Tisztában van a függőséget okozó szokások súlyos következményei\-vel. \\ \hline
              7--8 & erkölcstan & Tudja, hogy ugyanazt a dolgot különböző emberek eltérő módon ítélhetik meg, ami konfliktusok forrása lehet. \\ \hline
              7--8 & testnevelés & Gyakorlott a célszerű óraszervezés megvalósításában. \\ \hline
              7--8 & testnevelés & Ismeri az egyszerű stressz- és feszültségoldó technikákat. \\ \hline
              7--8 & testnevelés & Képes egyszerű gimnasztikai gyakorlatokat önállóan összefűzni és zenére előadni. \\ \hline
              7--8 & testnevelés & Ismeri és elkerüli az erősítés és nyújtás néhány ellenjavallt gyakorlatát. \\ \hline
              7--8 & testnevelés & Kísérletet tesz az összehangolt, rendezett testtartás kritériumainak való megfelelésre. \\ \hline
              7--8 & testnevelés & A kamaszkori személyi higiénéről elemi tájékozottsággal rendelkezik. \\ \hline
              7--8 & testnevelés & A tanult két úszásnemben mennyiségi és minőségi teljesítményjavulást képes felmutatni. \\ \hline
              7--8 & testnevelés & Képes mellúszásban úszni az egyéni adottságainak és képességeinek megfelelően. \\ \hline
              7--8 & testnevelés & Rendelkezik ismeretekkel a vízben mozgás prevenciós előnyeiről és fizikai hátteréről. \\ \hline
              7--8 & testnevelés & Képes a vízből mentés alapgyakorlatainak bemutatására. \\ \hline
              7--8 & testnevelés & A vizes feladatokban kinyilvánítja felelősségérzetét és segítőkészségét. \\ \hline
              7--8 & testnevelés & Gazdag sportjáték-technikai és -taktikai készlettel rendelkezik. \\ \hline
              7--8 & testnevelés & Jártas néhány taktikai formáció, helyzet megoldásában. \\ \hline
              7--8 & testnevelés & A játékszabályok kibővített körét is megérti és alkalmazza. \\ \hline
              7--8 & testnevelés & Fejleszti a csapatjátékhoz szükséges együttműködési és kommunikációs képességet. \\ \hline
              7--8 & testnevelés & Megtapasztalja és elfogadja a sportjátékokhoz tartozó test-test elleni küzdelmet. \\ \hline
              7--8 & testnevelés & Konfliktusok, sportszerűtlenségek, deviáns magatartások esetén a gondolatait, véleményét szóban kulturáltan fejezi ki. \\ \hline
              7--8 & testnevelés & Alapvető tájékozottsága van a labdajátékok sporttörténete terén. \\ \hline
              7--8 & testnevelés & Az atlétikai cselekvésmintákat sokoldalúan és célszerűen alkalmazza. \\ \hline
              7--8 & testnevelés & Futó-, ugró- és dobógyakorlatokat végez a tanult versenyszabályoknak megfelelően. \\ \hline
              7--8 & testnevelés & Mérhető fejlődést ér el képességekben és sportági eredményekben. \\ \hline
              7--8 & testnevelés & Mozgása az atlétikai alapmozgásokban közelít a mozgásmintához. Összekapcsolja a lendületszerzést és a befejező mozgásokat. \\ \hline
              7--8 & testnevelés & Ismeri a futás, a kocogás élettani jelentőségét. \\ \hline
              7--8 & testnevelés & A helyes testtartás, a koordinált mozgás és az erőközlés összhangjára képes a torna jellegű mozgásokban. \\ \hline
              7--8 & testnevelés & Talajon és a választott tornaszeren növekvő önállóság jeleit mutatja fel a gyakorlásban, gyakorlat-összeállításban. \\ \hline
              7--8 & testnevelés & Bátran végrehajt szekrény- és a támaszugrásokat, a képességének megfelelő magasságon. \\ \hline
              7--8 & testnevelés & Látható fejlődés az aerobikgyakorlatok kivitelében és a zenével összhangban történő végrehajtása során. \\ \hline
              7--8 & testnevelés & A torna jellegű gyakorlatok végrehajtásában képes az önkontrollra, az együttműködésre és a segítségnyújtásra. \\ \hline
              7--8 & testnevelés & A rekreációs célú sportágakban és a népi hagyományokra épülő sportolási formákban bővülő gyakorlási tapasztalatot szerez, és fellelhető erősebb belső motiváció az általa választott területeken. \\ \hline
              7--8 & testnevelés & Az egészséges életmóddal kapcsolatos ismereteire támaszkodik, és terjeszti azokat. \\ \hline
              7--8 & testnevelés & Ismeri a természeti erők és a sport hasznos összekapcsolásának előnyeit, ezen a téren rutinra, jártasságra tesz szert. \\ \hline
              7--8 & testnevelés & Cselekedeteiben megjelenik a környezettudatosság. \\ \hline
              7--8 & testnevelés & Fejleszti verbális és nem verbális kommunikációját a testkultúra hagyományos és az újszerű mozgásanyagainak témájában. \\ \hline
              7--8 & testnevelés & Pozitívan viszonyul a szabadidőben végzett sportoláshoz. \\ \hline
              7--8 & testnevelés & A grundbirkózás alaptechnikáját, szabályait a gyakorlatban alkalmazza. \\ \hline
              7--8 & testnevelés & A különböző eséstechnikák, szabadulások, leszorítások és az önvédelem gyakorlatait kontrolláltan, társsal végrehajtja. \\ \hline
              7--8 & testnevelés & Vannak ismeretei a fenyegetettségi szituációkra, segítségkérésre, menekülésre vonatkozóan. \\ \hline
              7--8 & testnevelés & Sportszerűen szeretne győzni, és ezt ki is fejezi. \\ \hline
              7--8 & technika & Az egészséges, biztonságos, környezettudatos otthon működtetéséhez szükséges praktikus életvezetési ismereteit fejleszt, az ehhez szükséges készségeket kialakítja. \\ \hline
              7--8 & technika & Biztonságosan, takarékosan és felelősen kezeli a háztartás elektromos, víz-, szennyvíz-, gáz- és más tüzelőberendezéseit. A használattal járó veszélyekkel és környezeti hatásokkal tisztában van, felismeri a hibákat, működészavarokat. Képes egyszerű karbantartási, javítási munkák önálló elvégzésére. \\ \hline
              7--8 & technika & Környezettudatosan kezeli a háztartási hulladékokat. \\ \hline
              7--8 & technika & A víz- és energiafogyasztással, hulladékokkal kapcsolatos mennyiségeket és költségeket érzékeli és jól meg tudja becsülni. \\ \hline
              7--8 & technika & Elköteleződik a takarékos életvitel és a környezetkímélő technológiák mellett. \\ \hline
              7--8 & technika & A kerékpárosokra vonatkozó közlekedési szabályokat tudatosan, készségszinten alkalmazza. \\ \hline
              7--8 & technika & Tájékozott a közlekedési környezetben. \\ \hline
              7--8 & technika & Közlekedési magatartása tudatos. \\ \hline
              7--8 & technika & A közlekedési morált alkalmazza. \\ \hline
              7--8 & technika & Közlekedésszemlélete környezettudatos. \\ \hline
              7--8 & technika & Tájékozott a továbbtanulási lehetőségekről. Van elképzelése a saját felnőttkori életéről. Mérlegeli a pályaválasztási lehetőségeket. \\ \hline
              7--8 & technika & A meglátogatott munkahelyeken szerzett tapasztalatok, ismeretek alapján véleményt alkot, ezeket összeveti a személyes terveivel. \\ \hline
              7--8 & technika & Keresi az összhangot az adottságok, képességek, igények és lehetőségek között. \\ \hline
              7--8 & technika & A munkatevékenységet az önmegvalósítás részeként értékeli. \\ \hline
              7--8 & technika & A munkába álláshoz szükséges alapkészségeket és ismereteket elsajátítja. \\ \hline
              7--8 & informatika & Ismeri a különböző informatikai környezeteket. \\ \hline
              7--8 & informatika & Használja az operációs rendszer és a számítógépes hálózat alapszolgáltatásait. \\ \hline
              7--8 & informatika & Segítséggel kiválasztja az adott feladat megoldásához alkalmas hardver- és szoftvereszközöket. \\ \hline
              7--8 & informatika & Dokumentumokba különböző objektumokat beilleszt. \\ \hline
              7--8 & informatika & Készít szöveget, képet és táblázatot is tartalmazó dokumentumot minta vagy leírás alapján. \\ \hline
              7--8 & informatika & Létrehoz egyszerű táblázatot. \\ \hline
              7--8 & informatika & Diagramokat készít és módosít. \\ \hline
              7--8 & informatika & Bemutatót készít. \\ \hline
              7--8 & informatika & Látja a problémamegoldás folyamatát. \\ \hline
              7--8 & informatika & Használja az algoritmusleíró eszközöket. \\ \hline
              7--8 & informatika & Alapszintű programokat ír. \\ \hline
              7--8 & informatika & Algoritmusokat kódol. \\ \hline
              7--8 & informatika & Egyszerű vezérlési feladatokat megold fejlesztői környezetben. \\ \hline
              7--8 & informatika & Tervezési eljárásokat alkalmaz. \\ \hline
              7--8 & informatika & Meghatározza az eredményt a bemenő adatok alapján. \\ \hline
              7--8 & informatika & Tantárgyi szimulációs programokat használ. \\ \hline
              7--8 & informatika & Információt keres és talál. \\ \hline
              7--8 & informatika & Információt értékel. \\ \hline
              7--8 & informatika & Információt weben történő publikálásra előkészít. \\ \hline
              7--8 & informatika & Megkülönbözteti a publikussá tehető és védendő adatait. \\ \hline
              7--8 & informatika & Használja a legújabb infokommunikációs technológiákat, szolgáltatásokat. \\ \hline
              7--8 & informatika & Ismeri az informatikai biztonsággal és adatvédelemmel kapcsolatos fogalmakat. \\ \hline
              7--8 & informatika & Ismeri az adatokkal való visszaélésekből származó veszélyeket és következményeket. \\ \hline
              7--8 & informatika & Ismer megbízható információforrásokat. \\ \hline
              7--8 & informatika & Megállapítja az információ hitelességét. \\ \hline
              7--8 & informatika & Az informatikai eszközök etikus használatára vonatkozó szabályokat ismeri. \\ \hline
              7--8 & informatika & Ismeri az információforrások etikus felhasználási lehetőségeit. \\ \hline
              7--8 & informatika & Felismeri az informatikai eszközök használatának az emberi kapcsolatokra vonatkozó következményeit. \\ \hline
              7--8 & informatika & Ismer néhány elektronikus szolgáltatást. \\ \hline
              7--8 & informatika & Igénybe veszi, használja, lemondja a szolgáltatásokat. \\ \hline
              7--8 & informatika & A könyvtár és az internet szolgáltatásait igénybe véve önállóan talál releváns forrásokat konkrét tantárgyi feladataihoz. \\ \hline
              7--8 & informatika & A választott forrásokat alkotóan és etikusan használja feladatmegoldásnál. \\ \hline
              7--8 & informatika & Alkalmazza a más tárgyakban tanultakat (pl. informatikai eszközök használata, szövegalkotás). \\ \hline
              7--8 & informatika & Egyszerű témában önállóan végrehajtja az információs problémamegoldás folyamatát. \\ \hline
      \end{longtable}
\end{small}


\subsection{9--12. évfolyam}
\begin{small}
  \begin{longtable}{c | p{2cm} |  p{11cm} }
    \textbf{Évf.} & \textbf{Témater.} & \textbf{Tanulási eredmény} \\ \hline \hline
    \endhead

              9--10 & informatika & Felvételt készít digitális kamerával, az adatokat áttölti kameráról a számítógép adathordozójára. \\ \hline
              9--10 & informatika & Ismeri az adatvédelem hardveres és szoftveres módjait. \\ \hline
              9--10 & informatika & Ismeri az ergonómia alapjait. \\ \hline
              9--10 & informatika & Táblázatkezelővel tantárgyi feladatokat megold, egyszerű számításokat elvégez. \\ \hline
              9--10 & informatika & Körlevelet készít. \\ \hline
              9--10 & informatika & Adatbázis-kezelő programot használ. \\ \hline
              9--10 & informatika & Adattáblák között kapcsolatokat épít, adatbázisokból lekérdezéssel információt nyer ki. A kinyert adatokat esztétikus, használható formába rendezi. \\ \hline
              9--10 & informatika & Algoritmusokat készít. \\ \hline
              9--10 & informatika & A probléma megoldásához szükséges eszközöket kiválasztja. \\ \hline
              9--10 & informatika & Tantárgyi szimulációs programokat használ. \\ \hline
              9--10 & informatika & Tantárgyi mérések eredményeit kiértékeli. \\ \hline
              9--10 & informatika & Csoportban tevékenykedik. \\ \hline
              9--10 & informatika & Információt szerez, és azokat hagyományos, elektronikus vagy internetes eszközökkel publikálja. \\ \hline
              9--10 & informatika & Társaival az interneten kommunikál, közös feladatokon dolgozik. \\ \hline
              9--10 & informatika & Újabb informatikai eszközöket, információszerzési technológiákat használ. \\ \hline
              9--10 & informatika & Adatvédelemmel kapcsolatos fogalmakat ismer. \\ \hline
              9--10 & informatika & Információforrásokat értékel. \\ \hline
              9--10 & informatika & Az informatikai eszközök etikus használatára vonatkozó szabályokat ismeri. \\ \hline
              9--10 & informatika & A szerzői joggal kapcsolatos alapfogalmakat ismeri. \\ \hline
              9--10 & informatika & Az infokommunikációs publikálási szabályokat ismeri. \\ \hline
              9--10 & informatika & Az informatikai fejlesztések gazdasági, környezeti, kulturális hatásait felismeri. \\ \hline
              9--10 & informatika & Az informatikai eszközök használatának a személyiséget és az egészséget befolyásoló hatásait felismeri. \\ \hline
              9--10 & informatika & Az elektronikus szolgáltatások szerepét ismeri. \\ \hline
              9--10 & informatika & Elektronikus szolgáltatást tudatosan használ. \\ \hline
              9--10 & informatika & Az elektronikus szolgáltatások jellemzőit, előnyeit, hátrányait felismeri. \\ \hline
              9--10 & informatika & Felismeri a fogyasztói viselkedést befolyásoló módszereket a médiá\-ban. \\ \hline
              9--10 & informatika & Felismeri a tudatos vásárló jellemzőit. \\ \hline
              9--10 & informatika & A tanulmányaihoz kapcsolódó feladatai során az információs problémamegoldás folyamatát önállóan, alkotóan végrehajtja. \\ \hline
              9--10 & informatika & Saját információkeresési stratégiáival tisztában van, azokat tudatosan alkalmazza, értékeli és fejleszti. \\ \hline
              9--10 & testnevelés & Aktívan részt vesz a technikai, taktikai sportfeladatokban, ezek során együttműködik társaival. \\ \hline
              9--10 & testnevelés & Ismeri az egyes sportok szabályait, alkalmazza azokat. \\ \hline
              9--10 & testnevelés & Tudatosan kiválasztja az adott sporthoz leginkább kapcsolódó technikai és taktikai megoldásokat. \\ \hline
              9--10 & testnevelés & Elemzi a játékfolyamat taktikai megoldásait. Híve a fair és a csapatelkötelezett játéknak. \\ \hline
              9--10 & testnevelés & Sportjátékot vezet. \\ \hline
              9--10 & testnevelés & Olyan játékokat választ és játszik, amelyek ápolják a társas kapcsolatokat és alkalmasak bármilyen képességű társa számára. \\ \hline
              9--10 & testnevelés & Érvényesíti a mozgáselemek mozgásbiztonságának és a gyakorlás mennyiségének, minőségének oksági viszonyait. \\ \hline
              9--10 & testnevelés & A javító kritikát elfogadja és a mozdulatok kivitelezését javítja. Esztétikusan és harmonikusan ad elő. \\ \hline
              9--10 & testnevelés & Összeállít önálló talaj- és/vagy szergyakorlatot, egyszerű aerobik elemkapcsolatot, táncmotívumfüzért. \\ \hline
              9--10 & testnevelés & Ismer célszerű gyakorlási és gyakorlásszervezési formációkat, versenyszituációkat, versenyszabályokat. \\ \hline
              9--10 & testnevelés & A tanult mozgások versenysportja területén, a magyar sportolók sikereiről elemi ismeretei vannak. \\ \hline
              9--10 & testnevelés & Egy kijelölt táv megtételéhez szükséges időt és sebességet helyesen becsli. A becsült értékek alapján a feladatot pontosan végrehajtja. Évfolyamonként önmagához mérten javul futó-, ugró-, dobóteljesítménye. \\ \hline
              9--10 & testnevelés & Fejlődik tempóérzéke és odafigyelés a váltófutás gyakorlásában. \\ \hline
              9--10 & testnevelés & Az adott sportmozgás technikáját elfogadható cselekvésbiztonsággal hajtja végre. \\ \hline
              9--10 & testnevelés & A sportolás során használt különféle anyagok, felületek tulajdonságait és a baleseti kockázatokat ismeri. \\ \hline
              9--10 & testnevelés & Ismeri és alkalmazza az alternatív sportmozgások edzés- és balesetvédelmi alapfogalmait. \\ \hline
              9--10 & testnevelés & Önvédelmi és küzdősportot űz. \\ \hline
              9--10 & testnevelés & Betartja az önvédelmi és küzdőgyakorlatokban, harcokban a közös szabályokat, a biztonsági követelményeket és a küzdésekkel kapcsolatos rituálékat. \\ \hline
              9--10 & testnevelés & Elkerüli a sportok során a veszélyhelyzeteket; uralkodik az indulatai felett. Nem viselkedik agresszíven a sportjátékok során. \\ \hline
              9--10 & testnevelés & Ismeri a küzdősportok támadási, védekezési és önvédelmi megoldásait, kombinációit. Fogásokból szabadul. \\ \hline
              9--10 & testnevelés & 1000 m-t  úszik a választott technikával, egyéni tempóban, szabályos fordulóval. \\ \hline
              9--10 & testnevelés & Fejleszti úszóerejét és állóképességét. \\ \hline
              9--10 & testnevelés & Egy választott úszásnemhez tartozó öt szárazföldi képességfejlesztő gyakorlatot bemutat. \\ \hline
              9--10 & testnevelés & Ismer sportversenyszabályokat. \\ \hline
              9--10 & testnevelés & Szabályosan úszik. \\ \hline
              9--10 & testnevelés & Egyszerűbb feladatok, ugrások során másokkal szinkronban mozog a vízben. \\ \hline
              9--10 & testnevelés & Felsorolja a vízből mentés veszélyeit, pontos menetét. Ismerteti a passzív társ vonszolása kisebb távon (4-5 méter) történés menetét. \\ \hline
              9--10 & testnevelés & Bemelegítéssel fizikailag felkészül a sérülésmentes sporttevékenységre. \\ \hline
              9--10 & testnevelés & Ismeri és alkalmazza a biomechanikailag helyes testtartás jellemzőit, néhány jellemző deformitás kockázatait, a testtartás megőrzésének gyakorlatait. \\ \hline
              9--10 & testnevelés & Ismeri a gerinc sérüléseinek leggyakoribb fajtáit, a gerinc és az ízületek védelmének legfontosabb szempontjait. \\ \hline
              9--10 & testnevelés & Használ preventív stressz- és feszültségoldó gyakorlatokat. Ismeri és méri fittségét, ezzel kapcsolatosan önfejlesztő célokat tűz ki az egészség-edzettség érdekében. \\ \hline
              9--10 & testnevelés & A testsúlya és testtömege ismeretében egészségesen táplálkozik. \\ \hline
              9--10 & testnevelés & Ismeri és alkalmazza a rendszeres testmozgás pozitív hatásait a káros szenvedélyek leküzdésében, az érzelem- és a feszültség szabályozásában. \\ \hline
              11--12 & etika & Ismeri az erkölcsi hagyomány legfontosabb elemeit, és e tudás birtokában felismeri és kezeli a mindennapi életben felmerülő erkölcsi problémákat. \\ \hline
              11--12 & etika & Értékítéleteit észszerű érvekkel alátámasztja, felelős mérlegelésen alapuló döntést hoz. Részt vesz az etikai és közéleti vitákban, saját álláspontját megvédi, illetve továbbfejleszti. \\ \hline
              11--12 & etika & Elfogadja, megérti és tiszteli a magáétól eltérő nézeteket. \\ \hline
              11--12 & etika & Ismeri azokat az értékelveket, magatartásszabályokat és beállítódásokat, amelyeknek a közmegegyezés kitüntetett erkölcsi jelentőséget tulajdonít. \\ \hline
              11--12 & technika & Megszerez olyan gyakorlati tudást, amelynek birtokában könnyen eligazodhat a mindennapi élet számos területén. \\ \hline
              11--12 & technika & Folyamatosan fejleszti önismeretét, megoldáscentrikusan kezeli a konfliktusokat,  pozitív az életszemlélete. \\ \hline
              11--12 & technika & Pénzügyek kezelésében felelősen gondolkodik. Átgondolt döntéseket hoz a fogyasztási javak használatában, a szolgáltatások igénybevételével és a jövővel kapcsolatban. \\ \hline
              11--12 & technika & A mindennapokban nélkülözhetetlen életvezetési és háztartási ismeretek, „háztartási praktikák” ismeretében a napi munkát szakszerűen, hatékonyan, gazdaságosan végzi el. \\ \hline
              11--12 & technika & Az élő és a tárgyi környezet kapcsolatából, kölcsönhatásainak megfigyeléséből származó tapasztalatokat használ fel a problémamegoldások során, a tevékenységek gyakorlásakor. \\ \hline
              11--12 & technika & A használati utasításokat érti és betartja. \\ \hline
              11--12 & technika & Tudatos vásárló. A fogyasztóvédelem szerepét, a vásárlók jogait ismeri. \\ \hline
              11--12 & technika & Szociálisan érzékeny. A fogyatékkal élőket és az időseket segíti. Karitatív tevékenységeket végez. \\ \hline
              11--12 & technika & Hivatalos ügyekben érdekeit képviseli. Kulturált stílusban szolgáltatóknál, ügyfélszolgálatoknál ügyintéz. \\ \hline
              11--12 & technika & A korszerű pénzkezelés eszközeit ismeri. \\ \hline
              11--12 & technika & A biztonságos, balesetmentes, udvarias közlekedés szabályait betartja. \\ \hline
              11--12 & technika & Magabiztosan tájékozódik közvetlen és tágabb környezetében. \\ \hline
              11--12 & technika & A közlekedési szabályokat és a közlekedési etikát alkalmazza. \\ \hline
              11--12 & technika & Alkalmazza a balesetvédelem alapvető szabályait, felismeri és elhárítja a veszélyhelyzeteket, elsősegélyt nyújt. \\ \hline
              11--12 & technika & Tisztában van személyes ambícióival, képességeivel; mérlegeli az objektív lehetőségeket és döntést hoz a saját életpályájára vonatkozóan. \\ \hline
              11--12 & technika & Felismeri a munkakultúra szerepét és a személyes kapcsolatok fontosságát az álláskeresés és a munkahelymegtartás során. \\ \hline
              11--12 & technika & Elkötelezett a munka és az aktivitás iránt. \\ \hline
              11--12 & technika & Hisz az egész életen át tartó tanulásban. Érvényesíti szaktudását. \\ \hline
              11--12 & testnevelés & Önállóan bemelegít, gyakorol, edz, szervezi a játékot. \\ \hline
              11--12 & testnevelés & Ismeri az adott sportjáték versenykörülményeit. \\ \hline
              11--12 & testnevelés & Alkalmazza a technikai elemeket, taktikai megoldásokat, a sportverseny szabályait. \\ \hline
              11--12 & testnevelés & Csapaton belül képes a ráoszott szerepet betölteni, alkalmazza a formációkat a sporthelyzetben. \\ \hline
              11--12 & testnevelés & Ismereteinek megfelelően objektíven értékeli a csapat taktikai tervét, teljesítményét. \\ \hline
              11--12 & testnevelés & Elfogadja mások eltérő szintű játéktudását. \\ \hline
              11--12 & testnevelés & Ismer kreativitást, együttműködést, tartalmas, asszertív társas kapcsolatokat szolgáló mozgásos játéktípusokat, és azokat célszerűen használja. \\ \hline
              11--12 & testnevelés & Felismeri és kialakítja a tornagyakorlat optimális végrehajtására jellemző téri, időbeli és dinamikai sajátosságokat. \\ \hline
              11--12 & testnevelés & Koordináltan irányítja mozgását bonyolult gyakorlatelemsorok, folyamatok végrehajtása közben. \\ \hline
              11--12 & testnevelés & Tervez, gyakorol, bemutat önállóan összeállított összefüggő gyakorlatokat. \\ \hline
              11--12 & testnevelés & Önállóan választ zenét, ami illeszkedik a mozdulatsorhoz. \\ \hline
              11--12 & testnevelés & Könnyeden, plasztikusan, esztétikusan hajtja végre a táncos mozgásformákat. \\ \hline
              11--12 & testnevelés & Ismeri a versenysport előnyeit, veszélyeit, a versenyzéshez szükséges képességek fejlesztésének lehetőségeit. \\ \hline
              11--12 & testnevelés & Bemelegítő és képességfejlesztő gyakorlatokat ismeri, a célnak megfelelően használja. \\ \hline
              11--12 & testnevelés & Segítséget ad, biztosít, biztat. \\ \hline
              11--12 & testnevelés & Észreveszi és kijavítja a hibát, asszertívan kommunikál erről. \\ \hline
              11--12 & testnevelés & Ismeri az izmok mozgáshatárát bővítő aktív és passzív eljárásokat. \\ \hline
              11--12 & testnevelés & A futások, ugrások és dobások képességfejlesztő hatását felhasználja más mozgásrendszerekben. \\ \hline
              11--12 & testnevelés & Az atlétikai versenyszámok biomechanikai alapjait ismeri. \\ \hline
              11--12 & testnevelés & Fejleszti az állóképességét a lendületszerzés, az izom-előfeszítések begyakorlásával, a futó-, ugró- és a dobóteljesítmény növelése érdekében. \\ \hline
              11--12 & testnevelés & Az alapvető atlétikai versenyszabályokat ismeri. \\ \hline
              11--12 & testnevelés & Az atlétikai mozgásokhoz illeszkedően melegít be. \\ \hline
              11--12 & testnevelés & Uralja a testét a sebesség, gyorsulás, tempóváltás, gurulás, csúszás, gördülés esetén. \\ \hline
              11--12 & testnevelés & Önállóan tervez  és old meg feladatokat alternatív sporteszközökkel. \\ \hline
              11--12 & testnevelés & Az adott alternatív sportmozgáshoz szükséges edzéseket és balesetvédelmi alapfogalmakat ismeri. \\ \hline
              11--12 & testnevelés & Meglévő ismereteit alkalmazza az új sporttevékenységek során. \\ \hline
              11--12 & testnevelés & A szabályokat és rituálékat betartja. \\ \hline
              11--12 & testnevelés & Indulatait, agresszivitását kezeli. Önfegyelmmel rendelkezik. \\ \hline
              11--12 & testnevelés & Több támadási és védekezési megoldást, kombinációt ismer az álló és földharcban. \\ \hline
              11--12 & testnevelés & Magabiztosan viselkedik veszélyeztetettség esetén, elhárítja a támadást. \\ \hline
              11--12 & testnevelés & A bemelegítés szükségességének élettani okait ismeri. \\ \hline
              11--12 & testnevelés & Megtervezi az egészsége fenntartásához szükséges edzést, terhelést. Tudatosan védekezik a stresszes állapot ellen, a feszültségeket kezeli. \\ \hline
              11--12 & testnevelés & Ismeri és a gyakorlatban használja az izmok erősítését és nyújtását szolgáló gyakorlatokat. \\ \hline
              11--12 & testnevelés & Kíméli a gerincét a testnevelési és sportmozgásokban, kerti és házimunkákban, az esetleges sérüléses szituációkat megfelelően kezeli. \\ \hline
      \end{longtable}
\end{small}




% end of Harmónia
\section{STEM tantárgy tartalma }
STEM tantárgy által definiált tanulási eredmények a NAT pedagógiai szakaszai, évfolyamszintek és tématerületek szerint csoportosítva.

\subsection{1--4. évfolyam}
\begin{small}
  \begin{longtable}{c | p{2cm} |  p{11cm} }
    \textbf{Évf.} & \textbf{Témater.} & \textbf{Tanulási eredmény} \\ \hline \hline
    \endhead

              1--2 & matematika & Halmazokat összehasonlít, halmazokat alkot az elemek száma szerint. \\ \hline
              1--2 & matematika & Állításokról eldönti, hogy igaz vagy hamis. Állításokat megfogalmaz. \\ \hline
              1--2 & matematika & Több, kevesebb, ugyanannyi fogalmát használja. \\ \hline
              1--2 & matematika & Néhány elemet sorba rendez próbálgatással vagy más módszerrel. \\ \hline
              1--2 & matematika & Számokat helyesen írja, olvassa (100-as számkör). Alaki és helyi értéket ismeri, és helyesen használja. \\ \hline
              1--2 & matematika & Római számokat ismeri, és helyesen írja, olvassa (I, V, X). \\ \hline
              1--2 & matematika & Számokat elhelyezi a számegyenesen. Számszomszédokat meghatározza. Természetes számokat nagyság szerint összehasonlítja. \\ \hline
              1--2 & matematika & Matematikai jeleket: +, –, •, :, =, <, >, ( ) ismeri és használja. \\ \hline
              1--2 & matematika & Szóban és írásban összead, kivon, szoroz és oszt a százas számkörben. \\ \hline
              1--2 & matematika & A műveletek sorrendjére vonatkozó megállapodásokat ismeri, és helyesen alkalmazza. \\ \hline
              1--2 & matematika & Páros és páratlan számokat felismeri. \\ \hline
              1--2 & matematika & Szöveges feladatokat fel tud írni számokkal és matematikai jelekkel, bizonyos esetekben rajz segítségével. Az ismeretlen szimbólumot kiszámítja. \\ \hline
              1--2 & matematika & Növekvő és csökkenő számsorozatok szabályait felismeri, tudja a sorozatot folytatni. \\ \hline
              1--2 & matematika & Vonalak (egyenes, görbe) fogalmát ismeri és alkalmazza. \\ \hline
              1--2 & matematika & Képesek megkülönböztetni egymástól a testet és a síkidomot. \\ \hline
              1--2 & matematika & Képesek irányokat felismerni és megkülönböztetni egymástól. \\ \hline
              1--2 & matematika & Képes hosszúságot, űrtartalmat, tömeget és időt mérni. Ismeri a szabvány mértékegységeket: cm, dm, m, cl, dl, l, dkg, kg, perc, óra, nap, hét, hónap, év. Képes átváltani szomszédos mértékegységek között. Felismeri a mennyiségek közötti összefüggéseket és képes mérőeszközöket használni. \\ \hline
              1--2 & környezet & Az emberi test nemre és korra jellemző arányait leírja, a fő testrészeket megnevezi. Az egészséges életmód alapvető elemeit ismeri és alkalmazza. \\ \hline
              1--2 & környezet & Mesterséges és természetes életközösséget össze tudja hasonlítani. \\ \hline
              1--2 & környezet & Tiszteli az élővilág sokféleségét, felismeri a természetvédelem fontosságát. \\ \hline
              1--2 & környezet & Jól tájékozódik az iskolában és környékén. Az évszakos és napszakos változásokat felismeri és kapcsolja életmódbeli szokásokhoz. Az időjárás elemeit ismeri, az ezzel kapcsolatos piktogramokat értelmezi, az időjáráshoz illő szokásokat alkalmazza. \\ \hline
              1--2 & környezet & Használati tárgyakat és gyakori, a közvetlen környezetben előforduló anyagokat csoportosítja tulajdonságaik szerint, felismeri a kapcsolatot az anyagi tulajdonságok és a felhasználás között. A mesterséges és természetes anyagokat megkülönbözteti. A halmazállapotokat felismeri. \\ \hline
              1--2 & környezet & Megfigyeléseket végez a természetben, egyszerű vizsgálatokat és kísérleteket  folytat. Az eredményeket megfogalmazza, ábrázolja. Ok-okozati összefüggéseket keres a tapasztalatok magyarázatára. \\ \hline
              3--4 & matematika & Adott tulajdonságú elemeket halmazba rendez. \\ \hline
              3--4 & matematika & Egy adott halmazba tartozó elemek közös tulajdonságait felismeri és megnevezi. \\ \hline
              3--4 & matematika & Próbálgatással megtalálja az összes esetet. \\ \hline
              3--4 & matematika & A számokat helyesen írja és olvassa (10 000-es számkör). Alaki és helyi értéket ismeri és helyesen használja a 10 000-es számkörben. \\ \hline
              3--4 & matematika & Negatív számokat használja a mindennapi életben (hőmérséklet, adósság). \\ \hline
              3--4 & matematika & Törteket használ a mindennapi életben. 2, 3, 4, 10, 100 nevezőjű törteket képes megnevezni és előállítani hajtogatással, nyírással, rajzzal, színezéssel. \\ \hline
              3--4 & matematika & A természetes számokat nagyság szerint összehasonlítja a 10 000-es számkörben. \\ \hline
              3--4 & matematika & Képes a matematika különböző területein észszerű becslést végezni, és kerekítést alkalmazni. \\ \hline
              3--4 & matematika & Képes fejben számolni a százas számkörben. \\ \hline
              3--4 & matematika & A szorzótáblát teljes biztonsággal használja a 100-as számkörben. \\ \hline
              3--4 & matematika & Képes fejben számolni 10 000-ig nullákra végződő egyszerű esetekben. \\ \hline
              3--4 & matematika & Összeg, különbség, szorzat, hányados fogalmát ismeri. Az összeg tagjainak és a szorzat tényezőinek felcserélhetőségét ismeri és képes alkalmazni. A műveleti sorrendre vonatkozó megállapodásokat ismeri és képes alkalmazni. \\ \hline
              3--4 & matematika & Négyjegyű számokat képes írásban összeadni, kivonni, szorozni egy- és kétjegyű számmal, valamint osztani egyjegyű számmal. \\ \hline
              3--4 & matematika & Az elvégzett műveleteket ellenőrzi. \\ \hline
              3--4 & matematika & Szöveges feladat: a szöveget értelmezi, az adatokat kigyűjti, megoldási tervet készít, becslést végez, ellenőrzi az eredményt, és az eredmény realitását is vizsgálja. \\ \hline
              3--4 & matematika & Többszörös, osztó, maradék fogalmát ismeri és használja. \\ \hline
              3--4 & matematika & Egyszerű sorozatokat képes folytatni. A növekvő és a csökkenő számsorozatokat felismeri. \\ \hline
              3--4 & matematika & Összefüggéseket keres az egyszerű sorozatok elemei között. \\ \hline
              3--4 & matematika & Néhány elemmel megadott egyszerű sorozatok szabályát és a sorozat ismeretlen elemeit képes megadni. \\ \hline
              3--4 & matematika & Vizsgálja az egyenesek kölcsönös helyzetét, felismeri a metsző és párhuzamos egyeneseket. \\ \hline
              3--4 & matematika & Ismeri a szabvány mértékegységeket,például mm, km, ml, cl, hl, g, t, másodperc. Képes átváltásokat végezni a szomszédos mértékegységek között. \\ \hline
              3--4 & matematika & Hosszúságot, távolságot és időt mér (egyszerű gyakorlati példák). \\ \hline
              3--4 & matematika & A háromszöget, négyzetet, téglalapot, sokszöget, kört felismeri, és képes létrehozni egyszerű módszerekkel. Ismeri ezeknek a síkidomoknak a jellemzőit. \\ \hline
              3--4 & matematika & Érti a test és a síkidom közötti különbséget. \\ \hline
              3--4 & matematika & A kockát, téglatestet, gömböt felismeri, és képes létrehozni egyszerű módszerekkel. Ismeri ezeknek a testeknek a jellemzőit. \\ \hline
              3--4 & matematika & Tükrös alakzatokat  előállít hajtogatással, nyírással, rajzzal, színezéssel. Felismeri a tengelyes szimmetriát. \\ \hline
              3--4 & matematika & Négyzet, téglalap kerületét méri és kiszámítja. \\ \hline
              3--4 & matematika & Négyzet, téglalap területét méri és kiszámítja különféle egységekkel, területlefedéssel. \\ \hline
              3--4 & matematika & Tapasztalati adatokat lejegyez, táblázatba rendez. A táblázat adatait értelmezi. \\ \hline
              3--4 & matematika & Adatokat gyűjt és lejegyez. Diagramot készít, illetve olvas. \\ \hline
              3--4 & matematika & Valószínűségi játékokat, kísérleteket értelmezi, és a biztos, lehetetlen, lehet, de nem biztos kimeneteleket felismeri. \\ \hline
              3--4 & matematika & Felnőtt segítséggel használ az életkorának megfelelő oktatási célú programokat. \\ \hline
              3--4 & matematika & Ismer egy rajzoló programot: egyszerű ábrákat készít, színez.
 \\ \hline
              3--4 & matematika & Társaival együttműködik interaktív tábla használatánál. \\ \hline
              3--4 & környezet & Az egészséges életmód alapvető elemeit alkalmazza az egészségmegőrzés, az egészséges fejlődés és a betegségek elkerülése érdekében. \\ \hline
              3--4 & környezet & Az életkornak megfelelően és a helyzethez illően, felelősen viselkedik a segítségnyújtást igénylő helyzetekben. \\ \hline
              3--4 & környezet & A mindennapi életben előforduló távolságokat és időtartamokat megbecsli, a hosszúságot és az időt méri. \\ \hline
              3--4 & környezet & Adott szempontok alapján megfigyeléseket végez a természetben, egyszerű kísérleteken keresztül tanulmányozza a természeti jelenségeket. \\ \hline
              3--4 & környezet & A fenntartható életmód jelentőségét megmagyarázza konkrét példán kereszül és értelmezi a hagyományok szerepét a természeti környezettel való harmonikus kapcsolat kialakításában, illetve felépítésében. \\ \hline
              3--4 & környezet & Bemutatja az élőlények szerveződési szintjeit és az életközösségek kapcsolatait, csoportosítja az élőlényeket tetszőleges és adott szempontsor szerint. \\ \hline
              3--4 & környezet & Bemutat egy természetes életközösséget. \\ \hline
              3--4 & környezet & Értelmez egy technológiai folyamatot egy konkrét gyártási folyamat kapcsán és ismeri az ehhez kapcsolódó fogyasztói magatartást. \\ \hline
              3--4 & környezet & Elhelyezi Magyarországot a földrajzi térben, ismeri az ország néhány fő kulturális és természeti értékét. \\ \hline
              3--4 & környezet & Irányítottan használ informatikai és kommunikációs eszközöket az információkeresésben és a problémák megoldásában. \\ \hline
      \end{longtable}
\end{small}


\subsection{5--8. évfolyam}
\begin{small}
  \begin{longtable}{c | p{2cm} |  p{11cm} }
    \textbf{Évf.} & \textbf{Témater.} & \textbf{Tanulási eredmény} \\ \hline \hline
    \endhead

              5--6 & matematika & Elemeket halmazba rendez adott tulajdonság alapján, részhalmazokat felismer, és az elemeit megadja. \\ \hline
              5--6 & matematika & Két véges halmaz közös részét és egyesítését megadja, ábrázolja. \\ \hline
              5--6 & matematika & Néhány elemet különféle módszerekkel sorba rendez. \\ \hline
              5--6 & matematika & Állítások igazságát képes eldönteni, igaz és hamis állításokat megfogalmaz. \\ \hline
              5--6 & matematika & Néhány elem összes sorrendjét képes felírni. \\ \hline
              5--6 & matematika & Racionális számokat helyesen írja, olvassa, számegyenesen ábrázolja, valamint  egymással nagyságuk szerint összehasonlítja. \\ \hline
              5--6 & matematika & Képes felismerni és felírni számok ellentettjét, abszolút értékét, reciprokát. \\ \hline
              5--6 & matematika & Méréseket végez, melynek során használja a mértékegységeket, és azok egyszerű átváltásait. \\ \hline
              5--6 & matematika & A mindennapi életben felmerülő egyszerű arányossági feladatokat képes megoldani következtetéssel. Felismeri és alkalmazza az egyenes arányosságot. \\ \hline
              5--6 & matematika & Két-három műveletet és zárójelet is tartalmazó műveletsor eredményét képes kiszámítani. Ismeri és alkalmazza a műveleti sorrendre és a zárójel használatára vonatkozó szabályokat. \\ \hline
              5--6 & matematika & Szöveges feladatokat megold következtetéssel (az adatok közötti összefüggéseket szimbólumokkal felírja). \\ \hline
              5--6 & matematika & Képes megbecsülni a műveletek eredményeit, és ellenőrzi a kapott eredmények helyességét. \\ \hline
              5--6 & matematika & Ismeri a százalék fogalmát, és képes kiszámítani a százalékértéket. \\ \hline
              5--6 & matematika & Számok osztóit, többszöröseit felírja. Kiválasztja a közös osztókat, közös többszörösöket. Ismeri és alkalmazza az oszthatósági szabályokat (2, 3, 4, 5, 8, 9, 10, 100). \\ \hline
              5--6 & matematika & A hosszúság, terület, térfogat, űrtartalom, idő, tömeg szabvány mértékegységeit ismeri. Mértékegységek egyszerűbb átváltásait gyakorlati feladatokban alkalmazza. Algebrai kifejezéseket használ gyakorlati problémákban felmerülő  terület, kerület, felszín és térfogat számítása során. \\ \hline
              5--6 & matematika & Elsőfokú egyismeretlenes egyenleteket, egyenlőtlenségeket képes megoldani (szabadon választott módszerrel). \\ \hline
              5--6 & matematika & Koordinátákkal megadott pontot koordináta-rendszerben ábrázol, valamint koordináta-rendszerben megadott pont koordinátáit leolvassa. \\ \hline
              5--6 & matematika & Egyszerű grafikonokat készít, elemez. \\ \hline
              5--6 & matematika & Néhány elemmel megadott egyszerű sorozatok szabályát felismeri, és a sorozat ismeretlen elemeit képes megadni. Egyszerű sorozatokat adott szabály szerint tud folytatni. \\ \hline
              5--6 & matematika & Ismeri a térelemek, félegyenes, szakasz, szögtartomány, sík fogalmát. \\ \hline
              5--6 & matematika & A geometriai ismeretek segítségével a feltételeknek megfelelő ábrákat körző és vonalzó célszerű használatával pontosan szerkeszti. \\ \hline
              5--6 & matematika & Ismeri és egyszerű feladatokban alkalmazza az alapszerkesztéseket: pont és egyenes távolságának, két párhuzamos egyenes távolságának megszerkesztése, szakaszfelező merőleges, szögfelező, merőleges és párhuzamos egyenesek szerkesztése, valamint szögek másolása. \\ \hline
              5--6 & matematika & Egyszerű alakzatok tengelyes tükörképét megszerkeszti, tengelyes szimmetriát felismeri. \\ \hline
              5--6 & matematika & A tanult síkbeli és térbeli alakzatok tulajdonságait ismeri, és egyszerű feladatok megoldásában alkalmazza. \\ \hline
              5--6 & matematika & Kiszámítja a téglalap és a deltoid kerületét és területét.
 \\ \hline
              5--6 & matematika & Kiszámítja a téglatest felszínét és térfogatát. \\ \hline
              5--6 & matematika & A tanult testek térfogatszámítási módjának ismeretében gyakorlati példákban felmerülő egyszerű testek térfogatát, űrmértékét meghatározza. \\ \hline
              5--6 & matematika & Egyszerű diagramokat értelmez és  készít. Képes táblázatokat leolvasni.
 \\ \hline
              5--6 & matematika & Néhány szám számtani közepét kiszámítja. \\ \hline
              5--6 & matematika & Valószínűségi játékok, kísérletek során adatokat tervszerűen gyűjt, rendez, ábrázol. \\ \hline
              5--6 & természet & Felismeri a fizikai és kémiai változásokat anyagok kölcsönhatását követően. Értelmezi a jelenséget az energiaváltozás szempontjából. \\ \hline
              5--6 & természet & Ismeri az emberi szervezet felépítését, működését, serdülőkori változásait és okait. Tudatos az egészséget veszélyeztető hatásokkal kapcsolatban. \\ \hline
              5--6 & természet & A családi és a társas kapcsolatok jelentőségével tisztában van, társaival empatikus és segítőkész. \\ \hline
              5--6 & természet & Ismeri a Föld helyét a Világegyetemben, Magyarország helyét Európában. \\ \hline
              5--6 & természet & Átfogó képe van a hazai tájaink természetföldrajzi jellemzőiről, természeti-társadalmi erőforrásairól, gazdasági folyamatairól, környezeti állapotukról. \\ \hline
              5--6 & természet & Ismeri hazánk legjellemzőbb életközösségeit, termesztett növényeit, a házban és ház körül élő állatait. Érti az élő és élettelen környezeti tényezők kölcsönhatását. Felismeri a környezet-szervezet-életmód, valamint a szervek felépítése és működése közötti összefüggéseket. \\ \hline
              5--6 & természet & Tud tájékozódni a térképeken. Értelmezi a különböző tartalmú térképek jelrendszerét, és felhasználja az információszerzés folyamatában. \\ \hline
              5--6 & természet & Törekszik a természeti és társadalmi értékek védelmére. \\ \hline
              5--6 & természet & Felismeri a szűkebb és tágabb környezetében az emberi tevékenység környezeti hatásait. Anyag- és energiatakarékos életvitelével, tudatos vásárlási szokásaival önmaga is hozzájárul a fenntartható fejlődéshez. \\ \hline
              5--6 & természet & Egyszerű kísérleteket, megfigyeléseket, méréseket önállóan, illetve. csoportban biztonságosan elvégez, a tapasztalatokat rögzíti, következtetéséket von le. \\ \hline
              5--6 & természet & Nyitott, érdeklődő a világ megismerése iránt. Az internet és a könyvtár segítségével bővíti tudását. Saját ismeretszerzési, ismeretfeldolgozási módszereket használ. \\ \hline
              7--8 & matematika & Elemeket halmazokba rendez több szempont alapján. A halmazokat ábrázolja. \\ \hline
              7--8 & matematika & Egyszerű állítások igaz vagy hamis voltát eldönti, képes állítások tagadására. \\ \hline
              7--8 & matematika & Könnyen értelmezhető (és-sel, vagy-gyal összekötött, ha..., akkor... típusú) összetett állítások igaz vagy hamis voltát eldönti. \\ \hline
              7--8 & matematika & Kombinatorikai feladatokat az összes eset szisztematikus összeszámlálásával, és ahol lehetséges, fagráfok alkalmazásával képes megoldani. \\ \hline
              7--8 & matematika & Fagráfokat használ feladatmegoldás során. \\ \hline
              7--8 & matematika & Biztosan számol a racionális számok körében. A műveletek sorrendjére és a zárójelezésre vonatkozó szabályokat ismeri és helyesen alkalmazza. A várható eredményt megbecsüli, a kapott  eredményt ellenőrzi, valamint képes a feladat követelményeinek megfelelő, helyes kerekítésre. \\ \hline
              7--8 & matematika & Mérésekkel kapcsolatos egyszerű feladatokban a különböző mértékegységeket használja, és képes a mértékegységek szükséges és helyes átváltására. Az egyenes arányosságot és a fordított arányosságot felismeri, és egyszerű feladatokban képes az alkalmazásukra.
 \\ \hline
              7--8 & matematika & A százalékszámítás alapfogalmait ismeri, a tanult összefüggéseket alkalmazza a  feladatmegoldás során. \\ \hline
              7--8 & matematika & Kiválasztja a legnagyobb közös osztót az összes osztóból; a legkisebb pozitív közös többszöröst pedig a többszörösök közül. \\ \hline
              7--8 & matematika & Ismeri a prímszám, összetett szám, prímtényezős felbontás fogalmát; az összetett számoknak felírja a prímtényezős alakját. \\ \hline
              7--8 & matematika & Egyszerű algebrai egész kifejezések helyettesítési értékét meghatározza. Az összevonásokat elvégzi; többtagú kifejezéseket egytagúval szoroz. \\ \hline
              7--8 & matematika & Ismeri a négyzetre emelés, négyzetgyökvonás fogalmát, valamint a hatványozás fogalmát pozitív egész kitevők esetén. Ezeket a fogalmakat egyszerű feladatokban alkalmazza. \\ \hline
              7--8 & matematika & Megold elsőfokú egyismeretlenes egyenleteket és egyenlőtlenségeket. Következtetéssel vagy egyenlettel megold a matematikából és a mindennapi életből vett egyszerű szöveges feladatokat; a kapott megoldást ellenőrzi és számegyenesen ábrázolja. \\ \hline
              7--8 & matematika & A betűkifejezéseket és az azokkal végzett műveleteket alkalmazza matematikai, természettudományos és hétköznapi feladatok megoldásában. \\ \hline
              7--8 & matematika & A számológépet tudatosan használja a számolás megkönnyítésére. \\ \hline
              7--8 & matematika & Megadott sorozatokat folytat adott szabály szerint. \\ \hline
              7--8 & matematika & Az egyenes arányosság grafikonját felismeri, a lineáris kapcsolatot alkalmazza természettudományos feladatokban is. \\ \hline
              7--8 & matematika & A grafikonokat elemzi különböző szempontok szerint (növekedés, fogyás, monotonitás stb.), grafikonokat készít, grafikonokról adatokat leolvas. Táblázatok adatait értelmezi és ábrázolja különböző típusú grafikonon. \\ \hline
              7--8 & matematika & Ábrákat készít és szerkeszt geometriai ismeretei alapján. \\ \hline
              7--8 & matematika & Egyszerű geometriai alakzatok tulajdonságait ismeri (háromszögek, négyszögek belső és külső szögeinek összege, nevezetes négyszögek szimmetriatulajdonságai), ezeket alkalmazza a feladatok megoldásában. \\ \hline
              7--8 & matematika & Tengelyes és középpontos tükörkép, eltolt alakzat képét megszerkeszti. Kicsinyítést és nagyítást felismeri (szerkesztés nélkül). \\ \hline
              7--8 & matematika & A Pitagorasz-tételt ismeri és alkalmazza különböző, (köztük valós helyzeteket modellező) alakzatok ismeretlen adatainak kiszámítására. \\ \hline
              7--8 & matematika & Háromszögek, speciális négyszögek és a kör kerületét, területét kiszámítja, és tudását különböző feladatokban alkalmazza. \\ \hline
              7--8 & matematika & Egyszerűbb testek (háromszög és négyszög alapú egyenes hasáb, forgáshenger) térfogatát, űrtartalmát a térfogatképletek ismeretében kiszámítja. \\ \hline
              7--8 & matematika & Valószínűségi kísérletek eredményeit lejegyzi, relatív gyakoriságokat kiszámítja. \\ \hline
              7--8 & matematika & Konkrét feladatokban érti az esély, illetve valószínűség fogalmát, felismeri a biztos és a lehetetlen eseményeket. \\ \hline
              7--8 & matematika & Számológépet célszerűen használja statisztikai számításokban. \\ \hline
              7--8 & matematika & Néhány kiemelkedő (köztük magyar) matematikus nevét, tevékenységét ismeri. \\ \hline
              7--8 & biológia & Ismeri Magyarország legfontosabb nemzeti parkjait és a lakóhelyén vagy annak közelében található természeti értékeket (védett növények és védett természeti értékek). \\ \hline
              7--8 & biológia & Tisztában van a környezet-egészségvédelem alapjaival, a gyógy- és fűszernövényeknek a szervezetre gyakorolt hatásával. \\ \hline
              7--8 & biológia & Tudja, hogy milyen szerepe van a biológiai információnak az önfenntartásban és fajfenntartásban. \\ \hline
              7--8 & biológia & Érti a család szerepének biológiai és társadalmi jelentőségét. \\ \hline
              7--8 & biológia & Érti, hogy a párkapcsolatokból adódnak konfliktushelyzetek, és kész azokat megfelelő módszerekkel kezelni. \\ \hline
              7--8 & biológia & Összekapcsolja a tanult nem sejtes és sejtes élőlényeket az emberi szervezet működésével, az élőlények és környezetük egymásra hatásaként értelmezi őket. \\ \hline
              7--8 & biológia & Tisztában van saját szervezete működésének alapjaival. \\ \hline
              7--8 & biológia & Érti és tudja, bizonyítékokkal alátámasztja,, hogy az élővilág különböző megjelenési formáit a különböző élőhelyekhez való alkalmazkodás alakította ki. \\ \hline
              7--8 & biológia & Tudja, hogy az ember a természet része, és ennek megfelelően cselekszik. \\ \hline
              7--8 & biológia & Tudja, hogy az életmóddal nagymértékben befolyásolhatjuk szervezetünk egészséges működését. Az egészséget testi, lelki, szociális jóllétnek tekinti. \\ \hline
              7--8 & biológia & Kerüli az egészséget veszélyeztető anyagok használatát, tevékenységeket. \\ \hline
              7--8 & biológia & Szükség esetén a sérültet alapvető elsősegélynyújtásban részesíti. \\ \hline
              7--8 & biológia & Empátiával viszonyul beteg és fogyatékkal élő társaihoz. \\ \hline
              7--8 & biológia & Csoportmunkában és önállóan beszámolókat készít infokommunikációs eszközök segítségével, egyszerű kísérleteket, vizsgálódásokat végez, adatokat elemez és megoldásokat javasol valós problémákra. Projektmunkát végez tanári irányítással. \\ \hline
              7--8 & fizika & A számítógépet adatrögzítésre, információgyűjtésre használja. \\ \hline
              7--8 & fizika & Pontos, a szakszerű fogalmakat tudatosan alkalmazó, ábrákkal, irodalmi hivatkozásokkal stb. alátámasztott prezentációt tart eredményeiről. \\ \hline
              7--8 & fizika & Felismeri, hogy a természettudományos tények megismételhető megfigyelésekből, célszerűen tervezett kísérletekből nyert bizonyítékokon alapulnak. \\ \hline
              7--8 & fizika & Önállóan ismeretet szerez. \\ \hline
              7--8 & fizika & Legalább egy tudományos elmélet esetén végigkövette, hogy a társadalmi és történelmi háttér hogyan befolyásolta annak kialakulását és fejlődését. \\ \hline
              7--8 & fizika & Felhasználja ismereteit saját egészségének védelmére. \\ \hline
              7--8 & fizika & A mások által kifejtett véleményeket megérti, értékeli, azokkal szemben kulturáltan vitatkozik. \\ \hline
              7--8 & fizika & A kísérletek elemzése során kritikusan és egészséges szkepticizmussal szemléli a történéseket. Tudja, hogy ismeretei és használati készségei meglévő szintjén további tanulással túl tud lépni. \\ \hline
              7--8 & fizika & Megítéli, hogy különböző esetekben milyen módon alkalmazható a tudomány és a technika, értékeli azok előnyeit és hátrányait az egyén, a közösség és a környezet szempontjából. Törekszik a természet- és környezetvédelmi problémák enyhítésére. \\ \hline
              7--8 & fizika & Egyszerű megfigyelési és mérési folyamatokat tervez, tudományos ismeretek megszerzéséhez célzott kísérleteket végez. \\ \hline
              7--8 & fizika & Ábrák, adatsorok elemzéséből tanári irányítás alapján egyszerűbb összefüggéseket ismer fel. Megfigyelései során modelleket használ.  \\ \hline
              7--8 & fizika & Egyszerű arányossági kapcsolatokat matematikai és grafikus formában is lejegyez. Az eredmények elemzése után konklúziókat von le. \\ \hline
              7--8 & fizika & Ismeri a fényjelenségeken alapuló kutatóeszközöket, a fény alapvető tulajdonságait. \\ \hline
              7--8 & fizika & A sebesség fogalmát különböző kontextusokban is alkalmazza. \\ \hline
              7--8 & fizika & A testek közötti kölcsönhatás során a sebességük és a tömegük egyaránt fontos, és ezt konkrét példákkal bemutatja \\ \hline
              7--8 & fizika & Érti, hogy a gravitációs erő egy adott testre hat és a Föld (vagy más /égi/test) vonzása okozza. \\ \hline
              7--8 & fizika & Elemzi az energiaátalakulásokat, kapcsolatukat a hőmennyiséghez. Használja az energiafajták elnevezését, felismeri a hőmennyiség cseréjének és a hőmérséklet kiegyenlítésének kapcsolatát. \\ \hline
              7--8 & fizika & Felsorol többféle energiaforrást, ismeri alkalmazásuk környezeti hatásait. Figyel a környezettudatosságra, energiatakarékosságra. \\ \hline
              7--8 & fizika & Ismeri és azonosítja az energiaátalakítási lehetőségeket, érti a megújuló és a nem megújuló energiafajták közötti különbséget. \\ \hline
              7--8 & fizika & Elemzi az egyes energiaátalakítási lehetőségek előnyeit, hátrányait, alkalmazásuk kockázatait. Tényeket és adatokat gyűjt, vita során csoportosítja és felhasználja az érveket és ellenérveket. \\ \hline
              7--8 & fizika & Érti a nyomás fogalmát és egyszerű esetekben kiszámítja az erő és a felület hányadosaként. \\ \hline
              7--8 & fizika & Tudja, hogy nemcsak a szilárd testek fejtenek ki nyomást. \\ \hline
              7--8 & fizika & Megmagyarázza a gázok nyomását a részecskeképpel. \\ \hline
              7--8 & fizika & Tudja, hogy az áramlások oka a nyomáskülönbség. \\ \hline
              7--8 & fizika & Ismerettel rendelkezik arról, hogy a hang miként keletkezik, és hogy a részecskék sűrűségének változásával terjed a közegben. \\ \hline
              7--8 & fizika & Tudja, hogy a hang terjedési sebessége gázokban a legkisebb és szilárd anyagokban a legnagyobb. \\ \hline
              7--8 & fizika & Ismeri az áramkör részeit, összeállít egyszerű áramköröket, méri az áramerősséget. \\ \hline
              7--8 & fizika & Tudja, hogy az áramforrások kvantitatív jellemzője a feszültség. \\ \hline
              7--8 & fizika & Ismeri az elektromos fogyasztó működését. \\ \hline
              7--8 & fizika & Elmagyarázza az erőművek alapvető szerkezetét. \\ \hline
              7--8 & fizika & Tudja, hogy az elektromos energia bármilyen módon történő előállítása terheli a környezetet. \\ \hline
              7--8 & kémia & Ismeri a kémia egyszerűbb alapfogalmait (atom, kémiai és fizikai változás, elem, vegyület, keverék, halmazállapot, molekula, anyagmennyiség, tömegszázalék, kémiai egyenlet, égés, oxidáció, redukció, sav, lúg, kémhatás), alaptörvényeit, vizsgálati céljait, módszereit és kísérleti eszközeit, a mérgező anyagok jelzéseit. \\ \hline
              7--8 & kémia & Ismeri néhány, a hétköznapi élet szempontjából jelentős szervetlen és szerves vegyület tulajdonságait, egyszerűbb esetben ezen anyagok előállítását és a mindennapokban előforduló anyagok biztonságos felhasználásának módjait. \\ \hline
              7--8 & kémia & Tudja, hogy a kémia a társadalom és a gazdaság fejlődésében fontos szerepet játszik. \\ \hline
              7--8 & kémia & Érti a kémia sajátos jelrendszerét, a periódusos rendszer és a vegyértékelektron-szerkezet kapcsolatát, egyszerű vegyületek elektronszerkezeti képletét, a tanult modellek és a valóság kapcsolatát. \\ \hline
              7--8 & kémia & Érti és alkalmazza a fogalmakat és törvényeket, ez alapján magyarázatot ad a vegyületek viselkedésére, a kísérletek során tapasztalt jelenségekre. \\ \hline
              7--8 & kémia & Egy kémiával kapcsolatos témáról önállóan vagy csoportban dolgozva információt keres, és ennek eredményét másoknak változatos módszerekkel, az infokommunikációs technológia eszközeit is alkalmazva bemutatja. \\ \hline
              7--8 & kémia & Alkalmazza a megismert törvényszerűségeket egyszerűbb, a hétköznapi élethez is kapcsolódó problémák, kémiai számítási feladatok megoldása során, illetve gyakorlati szempontból jelentős kémiai reakciók egyenleteinek leírásában. \\ \hline
              7--8 & kémia & Használja a megismert egyszerű modelleket a mindennapi életben előforduló, a kémiával kapcsolatos jelenségek elemzéseskor. \\ \hline
              7--8 & kémia & Megszerzett tudását alkalmazva felelős döntéseket hoz a saját életével, egészségével kapcsolatos kérdésekben, aktív szerepet vállal a személyes környezetének megóvásában. \\ \hline
              7--8 & földrajz & Átfogó és reális képzettel rendelkezik a Föld egészéről és annak kisebb-nagyobb egységeiről (a földrészekről és a világtengerről, a kontinensek karakteres nagytájairól és tipikus tájairól, valamint a világgazdaságban kiemelkedő jelentőségű országcsoportjairól, országairól). Átfogó ismerete van földrészünk, azon belül a meghatározó és a hazánkkal szomszédos országok természet- és társadalomföldrajzi sajátosságairól, látja azok térbeli és történelmi összefüggéseit, érzékeli a földrajzi tényezők életmódot meghatározó szerepét. Reális ismereteket birtokol a Kárpát-medencében fekvő hazánk földrajzi jellemzőiről, erőforrásairól és az ország gazdasági lehetőségeiről az Európai Unió keretében. Tisztában van az Európai Unió meghatározó szerepével és jelentőségével. \\ \hline
              7--8 & földrajz & Felismeri a földrajzi övezetesség kialakulásában megnyilvánuló összefüggéseket és törvényszerűségeket. Felismeri és értelmezi az egyes földrészekre vagy országcsoportokra, tájakra jellemző természeti jelenségeket és  társadalmi-gazdasági folyamatokat. Felismeri az egyes országok, országcsoportok helyét a világ társadalmi-gazdasági folyamataiban. Érzékeli az egyes térségek, országok társadalmi-gazdasági adottságai jelentőségének időbeli változásait. Felismeri a globalizáció érvényesülését regionális példákban. Ismeri hazánk társadalmi-gazdasági fejlődésének jellemzőit összefüggésben a természeti erőforrásokkal. Érti, hogy a hazai gazdasági, társadalmi és környezeti folyamatok világméretű vagy regionális folyamatokkal függenek össze. \\ \hline
              7--8 & földrajz & Példákkal bizonyítja a társadalmi-gazdasági folyamatok környezetkárosító hatását, a lokális problémák globális következményei elvének érvényesülését. Tisztában van a Földet fenyegető veszélyekkel, érti a fenntarthatóság lényegét példák alapján. Felismeri, hogy a Föld sorsa a saját magatartásunkon is múlik. \\ \hline
              7--8 & földrajz & Valós képzetekkel rendelkezik a környezeti elemek méreteiről, a számszerűen kifejezhető adatok és az időbeli változások nagyságrendjéről. Képes nagy vonalakban tájékozódni a földtörténeti időben. Természet-, illetve társadalom- és gazdaságföldrajzi megfigyeléseket végez a különböző nyomtatott és elektronikus információhordozókból földrajzi tartalmú információk gyűjtésére, összegzésére, a lényeges elemek kiemelésére. Ezek során alkalmazza digitális ismereteit. Megadott szempontok alapján bemutatja a földrajzi öveket, földrészeket, országokat és tipikus tájakat. \\ \hline
              7--8 & földrajz & A térképet információforrásként használja. A topográfiai ismereteikhez földrajzi-környezeti tartalmakat kapcsol. Topográfiai tudása alapján biztonsággal tájékozódik a köznapi életben a földrajzi térben, illetve a térképeken, és alkalmazza topográfiai tudását más tantárgyak tanulása során is. \\ \hline
              7--8 & földrajz & Társaival együttműködik. Későbbi élete folyamán önállóan tovább gyarapítja földrajzi ismereteit. \\ \hline
      \end{longtable}
\end{small}


\subsection{9--12. évfolyam}
\begin{small}
  \begin{longtable}{c | p{2cm} |  p{11cm} }
    \textbf{Évf.} & \textbf{Témater.} & \textbf{Tanulási eredmény} \\ \hline \hline
    \endhead

              9--10 & matematika & Ismeri a halmazokkal kapcsolatos alapfogalmakat, a halmazműveleteket. Szemlélteti a halmazokat és a halmazműveleteket. Fontosabb számhalmazokat ismeri. \\ \hline
              9--10 & matematika & Ismeri és a mindennapi nyelvezetben is használja a matematikai logika alapfogalmait. \\ \hline
              9--10 & matematika & Definíciót és tételt, állítást és megfordítását felismeri; bizonyítás gondolatmenetét egyszerű esetekben követi. \\ \hline
              9--10 & matematika & Egyszerű leszámlálási feladatokat megold, a megoldás gondolatmenetét képes elmondani és írásban is rögzíteni. \\ \hline
              9--10 & matematika & Ismeri a gráffal kapcsolatos alapfogalmakat. Alkalmazza a gráfokról tanult ismereteit gondolatmenet szemléltetésére, probléma megoldására. \\ \hline
              9--10 & matematika & Egész kitevőjű hatványok fogalmát és azonosságait ismeri. Egyszerű algebrai kifejezésekkel műveleteket végez. Ismereteit matematikai problémák megoldásában (egyenlet, egyenlőtlenség felírása szöveg alapján, egyenletek megoldása, képletek értelmezése) használja. \\ \hline
              9--10 & matematika & Elsőfokú, másodfokú egyismeretlenes egyenleteket megold; ilyen egyenletre vezető szöveges és gyakorlati feladatokhoz az egyenletet felírja, megoldja, és a megoldást ellenőrzi. \\ \hline
              9--10 & matematika & Egyszerű elsőfokú, másodfokú egyismeretlenes egyenletrendszereket megold; ilyen egyenletrendszerre vezető szöveges és gyakorlati feladatokhoz az egyenletrendszert felírja, megoldja, és a megoldást ellenőrzi. \\ \hline
              9--10 & matematika & Egyismeretlenes egyszerű másodfokú egyenlőtlenséget megold. \\ \hline
              9--10 & matematika & Jártas a valós számkörben; a gyakorlati és az elvontabb feladatokban alkalmazza a valós számkör műveleteit. \\ \hline
              9--10 & matematika & Matematikai szöveget értően olvas, tankönyveket, keresőprogramokat célirányosan használ, és a szövegekből kiemeli a lényeget. \\ \hline
              9--10 & matematika & Ismeri és alkalmazza a függvények különböző megadási módjait. Tisztában van az értelmezési tartomány és az értékkészlet fogalmával. Valós függvények alaptulajdonságait ismeri. \\ \hline
              9--10 & matematika & Az alapfüggvényeket képes koordináta-rendszerben ábrázolni, tulajdonságait ismeri és feladatokban alkalmazza. \\ \hline
              9--10 & matematika & Az egyszerű függvénytranszformációkat ismeri, tisztában van a grafikonjukra gyakorolt hatásával. \\ \hline
              9--10 & matematika & Valós folyamatokat elemez a folyamathoz tartozó függvény grafikonja alapján. Lineáris függvény esetén ismeri a meredekség fogalmát és szerepét valós folyamatokban. \\ \hline
              9--10 & matematika & Függvénymodellt készít az egyenes arányosságú kapcsolatok ábrázolására; ismeri a meredekség fogalmát. \\ \hline
              9--10 & matematika & Ábrázolja az elemi függvényeket koordináta-rendszerben, és a legfontosabb függvénytulajdonságokat meghatározza, nemcsak a matematika, hanem a természettudományos tárgyak megértése kapcsán, és a különböző gyakorlati helyzetek leírásának érdekében is. \\ \hline
              9--10 & matematika & Ismeri az alapvető térelemeket, a távolság és a szög fogalmát. Távolságokat és szögeket mér. \\ \hline
              9--10 & matematika & Nevezetes ponthalmazokat felismer, és szerkeszt ilyen halmazokat. \\ \hline
              9--10 & matematika & Fontos egybevágósági és hasonlósági transzformációk fogalmát  és lényeges tulajdonságait ismeri és feladatokban alkalmazza. \\ \hline
              9--10 & matematika & Az egybevágó és a hasonló alakzatokat felismeri. Két egybevágó, illetve két hasonló alakzat különböző szempontok (például távolságok, szögek, kerület, terület, térfogat) szerinti összehasonlítására képes. Ezeket az ismereteket egyszerű feladatokban alkalmazza. \\ \hline
              9--10 & matematika & Alakzatok szimmetriáját felismeri, és egyszerű feladatokban alkalmazza. \\ \hline
              9--10 & matematika & Ismeri a háromszögek tulajdonságait (alaptulajdonságok, nevezetes vonalak, pontok, körök). \\ \hline
              9--10 & matematika & Hegyesszögek szögfüggvényeinek fogalmát ismeri.
Elvégez a derékszögű háromszögre visszavezethető (gyakorlati) számításokat Pitagorasz-tétellel és a hegyesszögek szögfüggvényeivel. A magasságtételt és a befogótételt ismeri és egyszerű feladatokban alkalmazza. \\ \hline
              9--10 & matematika & Szimmetrikus négyszögek tulajdonságait ismeri. \\ \hline
              9--10 & matematika & A vektor fogalmát és a vektorok közti műveleteket (vektorok összeadása, kivonása, vektor szorzása valós számmal) ismeri. Ismeri a vektorkoordináták fogalmát adott koordináta-rendszerben. \\ \hline
              9--10 & matematika & Képes a háromszögekről tanultak alapján számítási feladatokat elvégezni, és ezeket gyakorlati problémák megoldásában alkalmazni. \\ \hline
              9--10 & matematika & Adathalmazt megadott szempontok szerint rendezi, adat gyakoriságát és relatív gyakoriságát kiszámítja. \\ \hline
              9--10 & matematika & Táblázatot értelmez és készít; diagramot értelmez és készít. \\ \hline
              9--10 & matematika & Adathalmaz móduszának, mediánjának, átlagának fogalmát ismeri, és egyszerű feladatokban alkalmazza. \\ \hline
              9--10 & matematika & Véletlen esemény, biztos esemény, lehetetlen esemény, esély/valószínűség fogalmát ismeri, és egyszerű feladatokban alkalmazza. \\ \hline
              9--10 & matematika & Nagyszámú véletlen kísérletet tud kiértékelni, összeveti az előzetesen „jósolt” esélyeket és a relatív gyakoriságokat. \\ \hline
              9--10 & biológia & Használja a fénymikroszkóp különböző fajtáit, előkészíti a vizsgálati anyagokat. Az eredményeit dokumentálja. \\ \hline
              9--10 & biológia & Ismeri a vírusok, baktériumok biológiai egészségügyi jelentőségét, az általuk okozott emberi betegségek megelőzésének lehetőségeit, a védekezés formáit. Ismeri a féregfertőzéseket és azok megelőzési feltételeit, a kullancscsípés megelőzését, a csípés esetleges következményeit. \\ \hline
              9--10 & biológia & Képes a helyes sorrendbe állítani a biológiai szerveződés szintjeit (mikroba, növény, állat, gomba) a törzsfán. Ok-okozati összefüggéseket felismer az élőlények testfelépítése, életműködése, életmódja között. Ismeri az életmód és a környezet kölcsönhatásait. \\ \hline
              9--10 & biológia & Ismeri és példákból felismeri az állatok különböző magatartásformáit. \\ \hline
              9--10 & fizika & Fejlődik a megfigyelő, rendszerező készsége és a kísérletezési, mérési kompetenciája. \\ \hline
              9--10 & fizika & Ismeri a mozgástani alapfogalmakat, grafikusan megold feladatokat. Érti a newtoni mechanika lényegét: az erő nem a mozgás fenntartásához, hanem a mozgásállapot megváltoztatásához szükséges. \\ \hline
              9--10 & fizika & Egyszerű kinematikai és dinamikai feladatokat megold. \\ \hline
              9--10 & fizika & A kinematikát és dinamikát a mindennapokban alkalmazza. \\ \hline
              9--10 & fizika & Folyadékok és gázok sztatikájának és áramlásának alapjelenségeit felismeri a gyakorlati életben. \\ \hline
              9--10 & fizika & Az elektrosztatika alapjelenségeit és fogalmait ismeri. Az elektromos és a mágneses mezőt fizikai objektumként elfogadja. \\ \hline
              9--10 & fizika & Megold egyszerű feladatokat az áramokkal kapcsolatos ismeretei alapján és ismereteit a gyakorlatban is használja. \\ \hline
              9--10 & fizika & Ismeri a gázok makroszkopikus állapotjelzőit és összefüggéseit, az ideális gáz golyómodelljét, a nyomás és a hőmérséklet kinetikus golyómodelljét. \\ \hline
              9--10 & fizika & Ismeri a hőtani alapfogalmakat, a hőtan főtételetit, mindennapi környezetünk hőtani változásait. Ismeri annak elméletét, hogy gépeink működtetése és az élő szervezetek működése csak energia befektetése árán valósítható meg, a befektetett energia jelentős része elvész, a működésben nem hasznosul, tisztában van azzal, hogy  „örökmozgó” létezése elvileg kizárt. \\ \hline
              9--10 & fizika & Fejlődik az energiatudatossága. \\ \hline
              9--10 & kémia & Ismeri az anyag tulajdonságainak anyagszerkezeti alapokon történő magyarázatához elengedhetetlenül fontos modelleket, fogalmakat, az összefüggéseket és a törvényszerűségeket, a legfontosabb szerves és szervetlen vegyületek szerkezetét, tulajdonságait, csoportosítását, előállítását, gyakorlati jelentőségét. \\ \hline
              9--10 & kémia & Érti az alkalmazott modellek és a valóság kapcsolatát, a szerves vegyületek esetében a funkciós csoportok tulajdonságokat meghatározó szerepét, a tudományos és az áltudományos megközelítés közötti különbségeket. \\ \hline
              9--10 & kémia & Érti a fenntarthatóság fogalmát és jelentőségét. \\ \hline
              9--10 & kémia & Értelmezi az anyagi halmazok jellemzőit összetevőik szerkezete és kölcsönhatásaik alapján. \\ \hline
              9--10 & kémia & Szóbeli és írásbeli összefoglalót, prezentációt készít önállóan vagy csoportban egy kémiával kapcsolatos témáról sokféle információforrás kritikus felhasználásával; és azt érthető formában közönség előtt bemutatja. \\ \hline
              9--10 & kémia & Alkalmazza a megismert kémiai tényeket és törvényszerűségeket egyszerűbb problémák és számítási feladatok megoldása során, valamint a fenntarthatósághoz és az egészségmegőrzéshez kapcsolódó viták alkalmával. \\ \hline
              9--10 & kémia & Meglát egyszerű kémiai jelenségekben ok-okozati elemeket; tervez ezek hatását bemutató, vizsgáló egyszerű kísérletet, és ennek eredményei alapján értékeli a kísérlet alapjául szolgáló hipotéziseket. \\ \hline
              9--10 & kémia & Koherens és kritikus érvelés alkalmazásával véleményt formál kémiai tárgyú ismeretterjesztő, egyszerű tudományos, illetve áltudományos cikkekről; az abban szereplő állításokat a tanult ismereteivel összekapcsolja, mások érveivel ütközteti. \\ \hline
              9--10 & földrajz & Szintetizálja a különböző szempontból elsajátított földrajzi (általános és leíró természet-, illetve társadalom-, valamint gazdaságföldrajzi) ismereteket. Átlátja a környezeti elemek méreteit, a számszerűen kifejezhető adatok és az időbeli változások nagyságrendjeit. \\ \hline
              9--10 & földrajz & Használja a térképet információforrásként, értelmezi a leolvasott adatokat. Felismeri a Világegyetem és a Naprendszer felépítésében, a bolygók mozgásában megnyilvánuló törvényszerűségeket. \\ \hline
              9--10 & földrajz & Tájékozódik a földtörténeti korokban; ismeri a kontinenseket felépítő nagyszerkezeti egységek kialakulásának időbeli rendjét, földrajzi elhelyezkedését. \\ \hline
              9--10 & földrajz & Képes megadott szempontok alapján bemutatni az egyes geoszférák sajátosságait, jellemző folyamatait és azok összefüggéseit. Belátja, hogy az egyes geoszférákat ért környezeti károk hatása más szférákra is kiterjedhet. \\ \hline
              9--10 & földrajz & Képes a földrajzi övezetesség kialakulásában megnyilvánuló összefüggések és törvényszerűségek értelmezésére. \\ \hline
              9--10 & földrajz & Képes alapvető összefüggések és törvényszerűségek felismerésére és megfogalmazására az egész Földre jellemző társadalmi-gazdasági folyamatokkal kapcsolatosan. \\ \hline
              9--10 & földrajz & Képes elhelyezni az országokat, országcsoportokat és integrációkat a világ társadalmi-gazdasági folyamataiban, értelmezi a világgazdaságban betöltött szerepüket. \\ \hline
              9--10 & földrajz & Képes értelmezni az egyes térségek, országok eltérő társadalmi-gazdasági adottságait és az adottságok jelentőségének változását az idő függvényében. \\ \hline
              9--10 & földrajz & Ismeri a globalizáció gazdasági és társadalmi hatását, érti az ellentmondásokat. \\ \hline
              9--10 & földrajz & Ismeri a monetáris világ jellemző folyamatait, azok társadalmi-gazdasági hatásait. \\ \hline
              9--10 & földrajz & Ismeri Magyarország társadalmi-gazdasági fejlődésének jellemzőit, a gazdasági fejlettség területi különbségeit és okait. \\ \hline
              9--10 & földrajz & Példákkal alátámasztja az Európai Unio egészére kiterjedő, illetve a környező országokkal kialakult regionális együttműködések szerepét. \\ \hline
              9--10 & földrajz & Elhelyezi Magyarországot a világgazdaság folyamataiban. \\ \hline
              9--10 & földrajz & Tudja példákkal bizonyítani a társadalmi-gazdasági folyamatok környezetkárosító hatását, a lokális problémák globális következményei elvének érvényesülését. Ismeri az egész Földünket érintő globális társadalmi és gazdasági problémákat. \\ \hline
              9--10 & földrajz & Érvel a fenntarthatóságot szem előtt tartó gazdaság, illetve gazdálkodás fontossága mellett. \\ \hline
              9--10 & földrajz & Az egyén szerepét és lehetőségeit a környezeti problémák mérséklésében ismeri, és konkrét példákat nevez meg. \\ \hline
              9--10 & földrajz & Képes természet-, illetve társadalom- és gazdaságföldrajzi megfigyelésekre, a tapasztalatok lejegyzésére, értelmezésére. \\ \hline
              9--10 & földrajz & Nyomtatott és elektronikus információhordozókból földrajzi tartalmú információkat gyűjt és azokat feldolgozza. Az információkat összegzi, a lényeges elemeket kiemeli. Ennek során alkalmazza digitális ismereteit. \\ \hline
              9--10 & földrajz & Véleményét a földrajzi gondolkodásnak megfelelően megfogalmazza és logikusan érvel. \\ \hline
              9--10 & földrajz & Alkalmazza ismereteit földrajzi tartalmú problémák megoldása során a mindennapi életben. \\ \hline
              9--10 & földrajz & Földrajzi ismereteit felhasználja valódi döntéshelyzetekben. \\ \hline
              9--10 & földrajz & Társaival együttműködik a földrajzi-környezeti tartalmú feladatok megoldásakor. \\ \hline
              9--10 & földrajz & Kialakul benne az igény arra, hogy későbbi élete folyamán önállóan gyarapítsa tovább földrajzi ismereteit. \\ \hline
              9--10 & földrajz & Topográfiai tudását alkalmazza más tantárgyak tanulása során, illetve a mindennapi életben. \\ \hline
              9--10 & földrajz & Ismeretei alapján biztonsággal tájékozódik a földrajzi térben, illetve az azt megjelenítő különböző térképeken. Ismeri a tananyagban meghatározott topográfiai fogalmakhoz kapcsolódó tartalmakat. \\ \hline
              11--12 & matematika & Egyszerű kombinatorikai problémákat önállóan választott módszerrel megold. \\ \hline
              11--12 & matematika & Ismeri a gráf fogalmát. Problémamegoldások során képes gráfokat alkalmazni. \\ \hline
              11--12 & matematika & A bizonyított és nem bizonyított állítást egyszerű esetekben meg tudja különböztetni. \\ \hline
              11--12 & matematika & Egyszerű következtetésekben felismeri a feltételt és a következményt. \\ \hline
              11--12 & matematika & A szövegben található információkat önállóan kiválasztja, értékeli, rendezi problémamegoldás céljából. \\ \hline
              11--12 & matematika & Állítások tagadását képes megfogalmazni. Egyszerű esetekben összetett (és-sel, vagy-gyal összekötött, ha..., akkor... típusú) állítások igaz vagy hamis voltát meg tudja állapítani. \\ \hline
              11--12 & matematika & Alkalmaz, szerkeszt gráfokat gyakorlati, a mindennapokból vett problémák alkotta feladatok megoldásai során. \\ \hline
              11--12 & matematika & A kiterjesztett gyök- és hatványfogalmat ismeri. \\ \hline
              11--12 & matematika & A logaritmus fogalmát ismeri. \\ \hline
              11--12 & matematika & Konkrét esetekben probléma megoldása céljából  alkalmazza a gyök, a hatvány és a logaritmus azonosságait. \\ \hline
              11--12 & matematika & Egyszerű (köztük a mindennapok gyakorlatában szereplő problémák megoldására alkalmazható) exponenciális és logaritmusos egyenletet fel tud írni szöveg alapján, az egyenletet megoldja és önállóan ellenőrzi. \\ \hline
              11--12 & matematika & A számológépet tudatosan használja feladatmegoldásokban. \\ \hline
              11--12 & matematika & Ismeri a trigonometrikus függvényeket; egyszerű esetekben felismeri és megrajzolja a függvény grafikonját (függvénytranszformációk alkalmazásával is). Egyszerű (főleg gyakorlati) példákban alkalmazza az ismereteit. \\ \hline
              11--12 & matematika & Ismeri az exponenciális függvényt és a logaritmusfüggvényt. Egyszerű (főleg gyakorlati) példákban alkalmazza az ismereteit. \\ \hline
              11--12 & matematika & Ismeri a számtani és a mértani sorozat egyszerű összefüggéseit. Egyszerű (főleg gyakorlati) példákban alkalmazza az ismereteit. \\ \hline
              11--12 & matematika & Az ismert függvények jellemzése alapján képes megfogalmazni a fontosabb függvénytulajdonságokat és a függvények fontos felhasználási lehetőségeit. \\ \hline
              11--12 & matematika & Ismeri az egyszerű geometriai tételeket, alkalmazni tudja feladatmegoldásokban, és egyszerű valós problémákban megtalálja a megfelelő geometriai modellt. \\ \hline
              11--12 & matematika & Hosszúságot, szöget, kerületet, területet, felszínt és térfogatot egyszerű esetekben kiszámít. \\ \hline
              11--12 & matematika & Két vektor skaláris szorzatának fogalmát ismeri.  Egyszerű (főleg gyakorlati) példákban alkalmazza az ismereteit. \\ \hline
              11--12 & matematika & Vektorokat ábrázol koordináta-rendszerben, ismeri a helyvektor és a vektorkoordináták fogalmát. Egyszerű (főleg gyakorlati) példákban alkalmazza az ismereteit. \\ \hline
              11--12 & matematika & A geometriai és algebrai módszerek közötti kapcsolatot a koordinátageometriai ismeretek kapcsán felfogja. Egyszerű esetekben képes koordináta-rendszerben megadott alakzatokra vonatkozó távolságot, szöget kiszámítani. Ismeri a kör és egyenes néhány koordinátageometriai egyenletét. Ezeket az ismereteit alkalmazva egyszerű geometriai feladatokat megold algebrai módszerrel. \\ \hline
              11--12 & matematika & A statisztikai mutatókat alkalmazza adathalmaz elemzésében. \\ \hline
              11--12 & matematika & Ismeri a valószínűség matematikai fogalmát, és a valószínűség klasszikus kiszámítási módját. Egyszerű (főleg gyakorlati) példákban alkalmazza az ismereteit. \\ \hline
              11--12 & matematika & Mintavételre vonatkozó egyszerű feladatokban a valószínűséget kiszámítja. \\ \hline
              11--12 & matematika & A mindennapok gyakorlatában előforduló egyszerű valószínűségi problémákat tudja értelmezni és kezelni. \\ \hline
              11--12 & matematika & Megfelelő kritikával fogadja a statisztikai vizsgálatok eredményeit, egyszerű esetekben látja a vizsgálatok korlátait, érvényességi körét. \\ \hline
              11--12 & biológia & Megérti a környezet- és természetvédelem alapjait, elsajátítja az ökológiai szemléletet, és nyitottá válik a környezetkímélő gazdasági és társadalmi stratégiák befogadására. \\ \hline
              11--12 & biológia & A mindennapi életben alkalmazza a megszerzett ismereteket. \\ \hline
              11--12 & biológia & Felismeri a molekulák és a sejtalkotó részek kooperativitását, összekapcsolja  a kémia, illetve a biológia tantárgyban tanult ismereteket. \\ \hline
              11--12 & biológia & Megérti az anyag-, az energia- és az információforgalom összefüggéseit az élő rendszerekben. \\ \hline
              11--12 & biológia & Rendszerben látja a hormonális, az idegi és az immunológiai szabályozást, és összekapcsolja a szervrendszerek működését, kémiai, fizikai, műszaki és sejtbiológiai ismeretekkel. Felismeri a biológiai, a technikai és a társadalmi szabályozás analógiáit. \\ \hline
              11--12 & biológia & Az ember egészségi állapotára jellemző következtetéseket von le biológiai ismeretei alapján. \\ \hline
              11--12 & biológia & Tudja, hogy az ember szexuális életében alapvetőek a biológiai folyamatok, és ismeri azt a vélekedést, hogy a szerelemre épülő tartós párkapcsolat, az utódok tudatos vállalása, felelősségteljes felnevelése biztosít csak emberhez méltó életet. \\ \hline
              11--12 & biológia & Helyesen értelmezi az evolúciós modellt. A rendszerelvű gondolkodás alapján megérti az emberi és egyéb élő rendszerek minőségi és mennyiségi összefüggéseit. \\ \hline
              11--12 & biológia & Felismeri a biológia és a társadalmi gondolkodás közötti kapcsolatot. \\ \hline
              11--12 & biológia & Interdiszciplinárisan gondolkodik. \\ \hline
              11--12 & biológia & Felismeri saját életében a biológiai eredetű problémákat, helyesen választja meg életmódját, felelős egyéni és társadalmi döntéseket hoz megbízható szakmai ismeretei alapján. \\ \hline
              11--12 & fizika & Bővül a mechanikai fogalomtára a rezgések és hullámok témakörével, valamint a forgómozgás és a síkmozgás gyakorlatban is fontos ismereteivel. \\ \hline
              11--12 & fizika & Az elektromágneses indukcióra épülő mindennapi alkalmazások fizikai alapját ismeri: elektromos energiahálózat, elektromágneses hullámok. \\ \hline
              11--12 & fizika & A (hétköznapi) optikai jelenségeket értelmezi hármas modellezéssel (geometriai optika, hullámoptika, fotonoptika). \\ \hline
              11--12 & fizika & Bemutatja a modellalkotás jellemzőit az atommodell fejlődésén keresztül. \\ \hline
              11--12 & fizika & Ismeri a kondenzált anyagok szerkezeti és fizikai tulajdonságainak alapvető összefüggéseit. \\ \hline
              11--12 & fizika & Értelmezi a magfizika elméleti ismeretei alapján a korszerű nukleáris technikai alkalmazásokat. Ismeri a kockázatokat és azokat reálisan értékeli. \\ \hline
              11--12 & fizika & A csillagászati alapismeretek felhasználásával elhelyezi a Földet az Univerzumban; képe van  az Univerzum térbeli, időbeli méreteiről. \\ \hline
              11--12 & fizika & Érti a csillagászat és az űrkutatás fontosságát. \\ \hline
              11--12 & fizika & Önállóan ismeretet szerez a STEM világában, forrást keres, szelektál és feldolgoz. \\ \hline
      \end{longtable}
\end{small}




% end of STEM
\section{KULT tantárgy tartalma }
KULT tantárgy által definiált tanulási eredmények a NAT pedagógiai szakaszai, évfolyamszintek és tématerületek szerint csoportosítva.

\subsection{1--4. évfolyam}
\begin{small}
  \begin{longtable}{c | p{2cm} |  p{11cm} }
    \textbf{Évf.} & \textbf{Témater.} & \textbf{Tanulási eredmény} \\ \hline \hline
    \endhead

              1--2 & magyar & Érthetően beszél, tisztában van a szóbeli kommunikáció alapvető szabályaival, és alkalmazza is őket. \\ \hline
              1--2 & magyar & Megérti az egyszerű magyarázatokat, utasításokat és társai közléseit. \\ \hline
              1--2 & magyar & A kérdésekre értelmesen válaszol. Aktivizálja a szókincsét a szövegalkotó feladatokban. Használja a bemutatkozás, a felnőttek és a kortársak megszólításának és köszöntésének udvarias nyelvi formáit. Összefüggő mondatok alkot. Követhetően számol be élményeiről, olvasmányai tartalmáról. Szöveghűen mondja el a memoritereket. \\ \hline
              1--2 & magyar & Ismeri az írott és nyomtatott betűket, rendelkezik megfelelő szókinccsel. Ismert és begyakorolt szöveget folyamatosságra, pontosságra törekvően olvas fel. Tanára segítségével kiemeli az olvasottak lényegét. Írása rendezett, pontos. \\ \hline
              1--2 & magyar & Felismeri és megnevezi a tanult nyelvtani fogalmakat, szükség szerint felidézi és alkalmazza a helyesírási szabályokat a begyakorolt szókészlet szavaiban. helyesen jelöli a j hangot 30–40 begyakorolt szóban. Helyesen választja el az egyszerű szavakat. \\ \hline
              1--2 & magyar & Tisztában van a tanulás alapvető céljával. Ítélőképessége, erkölcsi, esztétikai és történeti érzéke az életkori sajátosságoknak megfelelően fejlett. \\ \hline
              1--2 & ének & Képes 60 gyermekdalt és népdalt emlékezetből, a kapcsolódó játékokkal, c’–d” hangterjedelemben előadni. \\ \hline
              1--2 & ének & Kifejezően énekel, törekszik az egységes hangzásra és a tiszta intonációra, új dalokat hallás után megtanul. \\ \hline
              1--2 & ének & Kreatívan részt vesz a generatív játékokban és feladatokban. Érzi az egyenletes lüktetést, tartja a tempót, érzi a tempóváltozást. A 2/4-es metrumot helyesen hangsúlyozza. \\ \hline
              1--2 & ének & Ismeri és lejegyzi a tanult zenei elemeket. (ritmus, dallam) \\ \hline
              1--2 & ének & Pontosan, folyamatosan szólaltatja meg a tanult ritmikai elemeket tartalmazó ritmusgyakorlatokat csoportosan és egyénileg is. \\ \hline
              1--2 & ének & Szolmizálva énekel a tanult dalok stílusában megszerkesztett rövid dallamfordulatokat kézjelről, betűkottáról és hangjegyről. Megfelelő előkészítés után hasonló dallamfordulatokat rögtönöz. \\ \hline
              1--2 & ének & Társaival együtt figyelmesen hallgat zenét. Megfigyelés útján tapasztalathoz jut, melyet az egyszerű zenei elemzés alapjaként használ. Különbséget tesz az eltérő zenei karakterek között. \\ \hline
              1--2 & ének & Hangszínhallása fejlődik. Megkülönbözteti a furulya, citera, zongora, hegedű, fuvola, fagott, gitár, dob, triangulum, réztányér, a testhangszerek és a gyermek-, női, férfihang hangszínét. Ismeri a hangszerek alapvető jellegzetességeit. \\ \hline
              1--2 & vizkult & Korának megfelelő felismerhetőségű ábrát készít. \\ \hline
              1--2 & vizkult & Figyel az alkotótevékenységnek megfelelő, rendeltetésszerű és biztonságos anyag- és eszközhasználatra. Figyelembe veszi az alkotómunka megtervezése és kivitelezése során a környezetvédelmi szempontokat. \\ \hline
              1--2 & vizkult & A felszerelést önállóan rendben tartja. \\ \hline
              1--2 & vizkult & Közvetlen környezetét megfigyeli és értelmezi. \\ \hline
              1--2 & vizkult & A képalkotó tevékenységek közül személyes, kifejező alkotásokat hoz létre. \\ \hline
              1--2 & vizkult & A téralkotó feladatok során a személyes térbeli szükségleteket felismeri. \\ \hline
              1--2 & vizkult & Az alkotótevékenység és a látványok, műalkotások szemlélése során felismer és használ néhány formát, színt, vonalat, térbeli helyet és irányt. \\ \hline
              1--2 & vizkult & Felismeri a különbségeket a szobor, festmény, tárgy, épület között. \\ \hline
              1--2 & vizkult & Megkülönbözteti a hagyományos kézműves technikával készült tárgyakat. \\ \hline
              1--2 & vizkult & Megfigyeli és befogadja a látványokat, műalkotásokat. \\ \hline
              1--2 & vizkult & Azonosítja a médiumokat, tudatosan választ a médiahasználat során, reflektív. \\ \hline
              1--2 & vizkult & Felismeri a médiaélmények változását, médiatapasztalattá alakíthatóságát. \\ \hline
              1--2 & vizkult & Azonosítja a médiaszövegek néhány elemi kódját (kép, hang, cselekmény), felismeri az ezzel kapcsolatos egyszerű összefüggéseket (pl. médiaszövegek értelmezése, kreatív kifejező eszközök hatása a médiaszövegben). \\ \hline
              1--2 & vizkult & Felismeri a személyes kommunikáció és a közvetett kommunikáció közötti alapvető különbségeket. \\ \hline
              1--2 & vizkult & Életkorához igazodóan használja az internetet és felismeri az ebben rejlő veszélyeket. \\ \hline
              1--2 & vizkult & Felismeri és kifejezi az alkotó és befogadó tevékenység során saját érzéseit. \\ \hline
              3--4 & magyar & Értelmesen és érthetően fejezi ki gondolatait. Aktivizálja a szókincsét a szövegalkotó feladatokban. Használja a mindennapi érintkezésben az udvarias nyelvi fordulatokat. Beszédstílusát a beszélge\-tőpart\-neréhez igazítja. \\ \hline
              3--4 & magyar & Bekapcsolódik a csoportos beszélgetésbe, vitába, történetalkotásba, improvizációba, közös élményekről, tevékenységekről való beszélgetésekbe, értékelésbe. A közös tevékenységeket együttműködő magatartással segíti. \\ \hline
              3--4 & magyar & Felkészülés után folyamatosan, érthetően olvas fel ismert szöveget. Életkorának megfelelő szöveget megért néma olvasás útján. Az olvasottakkal kapcsolatos véleményét értelmesen fogalmazza meg. Ismer és alkalmaz néhány olvasási stratégiát. \\ \hline
              3--4 & magyar & Tanulási tevékenységét fokozatosan növekvő időtartamban irányjtja tudatos figyelemmel. Feladatainak megoldásához szükség szerint veszi igénybe az iskola könyvtárát. A könyvekben, gyermekújságokban a tartalomjegyzék segítségével eligazodik. \\ \hline
              3--4 & magyar & A memoritereket szöveghűen elmondja. \\ \hline
              3--4 & magyar & Adott vagy választott témáról 8–10 mondatos fogalmazást készít. \\ \hline
              3--4 & magyar & Az alsó tagozaton tanult anyanyelvi ismereteit rendszerezetten alkalmazza. Biztonsággal felismeri a tanult szófajokat, és megnevezi azokat szövegben is. \\ \hline
              3--4 & magyar & A begyakorolt szókészlet körében helyesen alkalmazza a tanult helyesírási szabályokat. Írásbeli munkái rendezettek, olvashatóak. Helyesírását önellenőrzéssel vizsgálja és szükség esetén javítja. \\ \hline
              3--4 & magyar & Ismeri a tanulás alapvető céljait. Ítélőképessége, erkölcsi, esztétikai és történeti érzéke életkorának megfelel. Nyitott anyanyelvi képességei fejlesztésére. Az anyanyelvi részképességeit. \\ \hline
              3--4 & idegen nyelv & Aktívan részt vesz a célnyelvi tevékenységekben, követi a célnyelvi óravezetést, az egyszerű tanári utasításokat, megérti az ismerős kérdéseket, és válaszol ezekre; kiszűri az egyszerű, rövid szövegek lényegét. \\ \hline
              3--4 & idegen nyelv & Idegen nyelven elmond néhány verset, mondókát és néhány összefüggő mondatot önmagáról, minta alapján egyszerű párbeszédet folytat társaival. \\ \hline
              3--4 & idegen nyelv & Ismert szavakat, rövid szövegeket elolvas és megért jól ismert témában. \\ \hline
              3--4 & idegen nyelv & Tanult szavakat, ismerős mondatokat lemásol, minta alapján egyszerű, rövid szövegeket alkot. \\ \hline
              3--4 & ének & 60 népdalt, műzenei idézetet emlékezetből, a-e’ hangterjedelemben csoportosan, a népdalokat több versszakkal, csokorba rendezve is előadja. \\ \hline
              3--4 & ének & Képes kifejezően, egységes hangzással, tiszta intonációval énekelni, és új dalokat megfelelő előkészítést követően, hallás után és jelrendszerről megtanulni. \\ \hline
              3--4 & ének & Többszólamú éneklési készsége fejlődik. Képes csoportosan ismert dalokat ritmus osztinátóval énekelni, és egyszerű kétszólamú darabokat, kánonokat megszólaltatni. \\ \hline
              3--4 & ének & Kreatívan részt vesz a generatív játékokban és feladatokban. Érzi az egyenletes lüktetést, tartja a tempót, érzi a tempóváltozást. A 3/4-es és 4/4-es metrumot helyesen hangsúlyozza. \\ \hline
              3--4 & ének & Felismeri, lejegyzi és megszólaltatja a tanult zenei elemeket (metrum, dinamikai jelzések, ritmus, dallam). \\ \hline
              3--4 & ének & Az ismert dalokat kézjelről, betűkottáról, hangjegyről és emlékezetből szolmizálja. \\ \hline
              3--4 & ének & Szolmizálva énekel a tanult dalok stílusában megszerkesztett rövid dallamfordulatokat kézjelről, betűkottáról és hangjegyről. Megfelelő előkészítés után hasonló dallamfordulatokat rögtönöz. \\ \hline
              3--4 & ének & Fejlődik zenei memóriája és belső hallása. \\ \hline
              3--4 & ének & Formaérzéke fejlődik, a formai építkezés jelenségeit (azonosság, hasonlóság, különbözőség) ismeri és meg tudja fogalmazni. \\ \hline
              3--4 & ének & Fejlődik hangszínhallása. Megkülönbözteti a tekerő, duda, oboa, klarinét, kürt, trombita, üstdob, a gyermekkar, női kar, férfikar, vegyes kar hangszínét. Ismeri a hangszerek alapvető jellegzetességeit. Különbséget tesz szóló és kórus, szólóhangszer és zenekar hangzása között. \\ \hline
              3--4 & ének & Tudatosan és figyelmesen hallgat zenét. A kiválasztott művek közül 4-5 alkotást, műrészletet ismer. \\ \hline
              3--4 & vizkult & Életkorának megfelelően értelmezi az alkotó, megfigyelő és elemző jellegű feladatokat. \\ \hline
              3--4 & vizkult & Vizuálisan kifejezi élményeit, emlékeit, illusztációt készít, síkbábot és egyszerű jelmezeket alkot, jeleket és ábrákat készít, egyszerű tárgyakat alkot. \\ \hline
              3--4 & vizkult & Rendeltetésszerűen és biztonságosan használja a megismert anyagokat és eszközöket, technikákat az alkotótevékenység során. \\ \hline
              3--4 & vizkult & Téralkotó feladatok során felismeri és használja a személyes preferenciáknak és funkcióknak megfelelő térbeli szükségletet. \\ \hline
              3--4 & vizkult & Tudja  differenciálni a szobrászati, festészeti, tárgyművészeti, építészeti területek közötti különbségeket (pl. festészeten belül: arckép, csendélet, tájkép). \\ \hline
              3--4 & vizkult & Megfogalmazza a látványok, műalkotások megfigyeléseinek során kialakult gondolatai, érzéseit. \\ \hline
              3--4 & vizkult & Felismeri a különböző típusú médiaszövegeket, tudatosan választ a médiatartalmak között. \\ \hline
              3--4 & vizkult & Felismeri a médiaszövegekhez használt egyszerű kódokat, kreatív kifejezőeszközöket és azok érzelmi hatását. \\ \hline
              3--4 & vizkult & Kép- és hangrögzítő eszközöket használ, ezek segítségével saját gondolatokat, érzéseket fogalmaz meg, rövid, egyszerű történeteket formál. \\ \hline
              3--4 & vizkult & A médiaszövegek előállításával, nyelvi jellemzőivel, használatával kapcsolatos alapfogalmakat  helyesen alkalmazza élőszóban. \\ \hline
              3--4 & vizkult & Ismeri a média alapvető funkcióit (tájékoztatás, szórakoztatás, ismeretszerzés). \\ \hline
              3--4 & vizkult & A médiaszövegekben megjelenő információkat kritikusan szemléli,  valóságtartalmát felismeri. \\ \hline
              3--4 & vizkult & Az életkorához igazodóan,  biztonságosan használja az internetet és mobiltelefont.  A hálózati kommunikációban való részvétel során betartja a fontos és szükséges viselkedési szabályokat. Életkorhoz igazodóan fejlesztő, kreatív internetes tevékenységeket végez. \\ \hline
      \end{longtable}
\end{small}


\subsection{5--8. évfolyam}
\begin{small}
  \begin{longtable}{c | p{2cm} |  p{11cm} }
    \textbf{Évf.} & \textbf{Témater.} & \textbf{Tanulási eredmény} \\ \hline \hline
    \endhead

              5--6 & magyar & Gondolatait érthetően, a helyzetnek megfelelően megfogalmazza, a beszédet kísérő nem nyelvi jeleket adekvátan alkalmazza. Megért, összefoglal, továbbad rövidebb szóbeli üzeneteket, rövidebb hallott történeteket. \\ \hline
              5--6 & magyar & Ismeri és alkalmazza a legalapvetőbb anyaggyűjtési, vázlatkészítési módokat. Szöveget alkot a tanult hagyományos és internetes műfajokban (elbeszélés, leírás, jellemzés, levél, SMS, e-mail stb.). Törekszik az igényes, pontos és helyes fogalmazásra, írásra. \\ \hline
              5--6 & magyar & Megérti az írott és elektronikus felületen megjelenő olvasott szöveget.
Ismer és alkalmaz szövegértési stratégiákat.
Önállóan információt gyűjt kézikönyvek és a korosztálynak szóló ismeretterjesztő források bevonásával. \\ \hline
              5--6 & magyar & Ismeri a szövegértés folyamatát, annak megfigyelésével saját módszerét fejleszti, a hibás olvasási szokásaira megfelelő javító stratégiát talál, és azt alkalmazza. \\ \hline
              5--6 & magyar & Ismeri a tanult alapszófajok (ige, főnév, melléknév, számnév, határozószó, igenevek, névmások), valamint az igekötők általános jellemzőit, alaki sajátosságait, a hozzájuk kapcsolódó főbb helyesírási szabályokat, amelyeket az írott munkáiban igyekszik alkalmazni is. \\ \hline
              5--6 & magyar & Új szavakat, közmondásokat, szólásokat használ. \\ \hline
              5--6 & magyar & Megnevez három mesetípust példákkal, és felidéz címe vagy részlete említésével három népdalt. Különbséget tesz a népmese és a műmese között. Megfogalmazza, mi a különbség a mese és a monda között. Elkülöníti a rímes, ritmikus szöveget a prózától. Megnevezi, melyik műnem mond el történetet, melyik jelenít meg konfliktust párbeszédes formában, és melyik fejez ki érzést, élményt. Felismeri a hexameteres szövegről, hogy az időmértékes, a felező tizenkettesről, hogy az ütemhangsúlyos. Felsorol három-négy művet Petőfitől és Aranytól, megfogalmaz egyszerűbb  összehasonlításokat János vitéz és Toldi Miklós alakjáról. Értelmezi A walesi bárdokban rejlő allegóriát, és ismerteti 5–6 mondatban az Egri csillagok történelmi hátterét. Elkülöníti az egyszerűbb versekben és prózai szövegekben a nagyobb szerkezeti egységeket. Összefoglalja néhány hosszabb mű cselekményét (János vitéz, Toldi, A Pál utcai fiúk, Egri csillagok), megkülönbözteti, melyik közülük a regény és melyik az elbeszélő költemény. Értelmesen és pontosan, tisztán, tagoltan, megfelelő ritmusban olvas fel szövegeket. Részt vesz számára ismert témájú vitában, és érvel. Ismert és könnyen érthető történetben párosítja annak egyes szakaszait a konfliktus, bonyodalom, tetőpont fogalmával. Az általa jól ismert történetek szereplőit jellemzi, kapcsolatrendszerüket feltárja és ismerteti. Néhány példa közül kiválasztja az egyszerűbb metaforákat és metonímiákat. Egyszerű meghatározást ad a következő fogalmakról: líra, epika, dráma, epizód, megszemélyesítés, ballada. Néhány egyszerűbb meghatározás közül kiválasztja azt, amely a következő fogalmak valamelyikéhez illik: dal, rím, ritmus, mítosz, motívum, konfliktus. Részt vesz művek, műrészletek szöveghű felidézésében.  \\ \hline
              5--6 & magyar & Az olvasott és megtárgyalt irodalmi művek nyomán azonosít erkölcsi értékeket és álláspontokat, saját álláspontját. \\ \hline
              5--6 & idegen nyelv & A1 szintű nyelvtudás: \\ \hline
              5--6 & idegen nyelv & Érti a gazdagodó nyelvi eszközökkel megfogalmazott óravezetést, az ismert témákhoz kapcsolódó kérdéseket, rövid megnyilatkozásokat, szövegeket. \\ \hline
              5--6 & idegen nyelv & Szóban és írásban egyszerű nyelvi eszközökkel, begyakorolt beszédfordulatokkal (idiómákkal) kommunikál. \\ \hline
              5--6 & idegen nyelv & Felkészülés után elmond rövid szövegeket. \\ \hline
              5--6 & idegen nyelv & Közös feldolgozás után megérti az egyszerű olvasott szövegek lényegét, tartalmát. \\ \hline
              5--6 & idegen nyelv & Ismert témáról rövid, egyszerű mondatokat ír, mintát követve önálló írott szövegeket alkot. \\ \hline
              5--6 & történelem & Megismeri az egyetemes emberi értékeket, az ókori, középkori és kora újkori egyetemes és magyar kultúrkincsen keresztül. A családhoz, a lakóhelyhez, a nemzethez való tartozás élményét megéli. \\ \hline
              5--6 & történelem & Azonosítja a történelmet formáló alapvető folyamatokat, felismeri az összefüggéseket (pl. a munka értékteremtő ereje) és egyszerű, átélhető erkölcsi tanulságokat azonosít. \\ \hline
              5--6 & történelem & Ismeri az előző korokban élt emberek, közösségek élet-, gondolkodás- és szokásmódjait. \\ \hline
              5--6 & történelem & Felismeri, hogy a múltban való tájékozódást segítik a kulcsfogalmak és fogalmak, amelyek fejlesztik a történelmi tájékozódás és gondolkodás kialakulását. \\ \hline
              5--6 & történelem & Felismeri, hogy az utókor a történelmi személyiségek, nemzeti hősök cselekedeteit a közösségek érdekében végzett tevékenységek szempontjából értékeli. \\ \hline
              5--6 & történelem & Tudja, hogy a népeket főként vallásuk és kultúrájuk, életmódjuk alapján tudjuk megkülönböztetni. \\ \hline
              5--6 & történelem & Felismeri, hogy a vallási előírások, valamint az államok által megfogalmazott szabályok döntő mértékben befolyásolhatják a társadalmi viszonyokat és a mindennapokat. \\ \hline
              5--6 & történelem & Tudja, hogy a történelmi jelenségeket, folyamatokat társadalmi, gazdasági tényezők együttesen befolyásolják, és felismeri ezeket egy-egy történelmi probléma vagy korszak feldolgozása során. \\ \hline
              5--6 & történelem & Tudja, hogy az emberi munka nyomán elinduló termelés biztosítja az emberi közösségek létfenntartását. \\ \hline
              5--6 & történelem & Az árutermelés és pénzgazdálkodás, illetve a városiasodás kialakulásához vezető társadalmi munkamegosztás jelentőségét felismeri. \\ \hline
              5--6 & történelem & Tudja, hogy a társadalmakban eltérő jogokkal rendelkező és eltérő vagyoni helyzetű emberek alkotnak rétegeket, csoportokat. \\ \hline
              5--6 & történelem & Tudja, hogy az eredettörténetek, a közös szokások és mondák erősítik a közösség összetartozását, egyben a nemzeti öntudat kialakulásának alapjául szolgálnak. \\ \hline
              5--6 & történelem & Tudja, hogy a társadalmi, gazdasági, politikai és vallási küzdelmek számos esetben összekapcsolódnak. \\ \hline
              5--6 & történelem & Különbséget tesz a történelem különböző típusú forrásai között, felismeri a korszakra jellemző képeket, tárgyakat, épületeket. \\ \hline
              5--6 & történelem & Történetek feldolgozásánál megkülönbözteti a valós és fiktív elemeket, csoportosítja a szereplőket (fő és mellékszereplőkre), térben és időben elhelyezi őket. Ismeri a a híres történelmi személyiségek jellemzéséhez szükséges kulcsszavakat, cselekedeteket. \\ \hline
              5--6 & történelem & A hallott és olvasott szövegekből, különböző médiumok anyagából következtetéseket von le és történelmi ismeretekre tesz szert. \\ \hline
              5--6 & történelem & Értelmezi az emberi magatartásformákat, információkat rendszerez és értelmez, vizuális vázlatokat készít. \\ \hline
              5--6 & történelem & Információt gyűjt adott témához könyvtárban és múzeumban; olvasmányairól lényeget kiemelő jegyzetet készít. \\ \hline
              5--6 & történelem & Szóbeli beszámolót készít önálló gyűjtőmunkával szerzett ismereteiről, és prezit tart róla. \\ \hline
              5--6 & történelem & Ismeri az időszámítás alapelemeit (korszak, évszázad, évezred), ezek alapján kronológiai számításokat végez. Ismeri néhány kiemelkedő esemény időpontját. \\ \hline
              5--6 & történelem & Egyszerű térképeket másol, alaprajzot készít. Megtalál földrajzi és történelmi helyeket térképen, vaktérképen néhány kiemelt történelmi esemény topográfiai helyét megmutatja. Ezekhez kapcsolódóan tud távolságot becsülni és számítani történelmi térképen. \\ \hline
              5--6 & honismeret & Megismeri lakóhelye, szülőföldje természeti adottságait, hagyományos gazdasági tevékenységeit, néprajzi jellemzőit, történetének nevezetesebb eseményeit, jeles személyeit. A tanulási folyamatban kialakul az egyéni, családi, közösségi, nemzeti azonosságtudata. \\ \hline
              5--6 & honismeret & Általános képe van a hagyományos gazdálkodó életmód fontosabb területeiről, a család felépítéséről, a családon belüli munkamegosztásról. A megszerzett ismeretek birtokában értelmezi a más tantárgyakban felmerülő népismereti tartalmakat. \\ \hline
              5--6 & honismeret & Felfedezi a jeles napok, ünnepi szokások, az emberi élet fordulóihoz kapcsolódó népszokások, valamint a társas munkák, közösségi alkalmak hagyományainak jelentőségét, közösségmegtartó szerepüket a paraszti élet rendjében. Élményszerűen, hagyományhű módon elsajátítja egy-egy jeles nap, ünnepkör köszöntő vagy színjátékszerű szokását, valamint a társas munkák, közösségi alkalmak népszokásait és a hozzájuk kapcsolódó tevékenységeket. \\ \hline
              5--6 & honismeret & Ismeri a magyar nyelvterület földrajzi-néprajzi tájainak, tájegységeinek hon- és népismereti, néprajzi jellemzőit. Világos számára, hogyan függ össze egy táj természeti adottsága a gazdasági tevékenységekkel, a népi építészettel, hogyan élt harmonikus kapcsolatban az ember a természettel. \\ \hline
              5--6 & ének & 20-22 népdalt, történeti éneket több versszakkal, valamint 8-10 műzenei idézetet emlékezetből elénekel. \\ \hline
              5--6 & ének & Kifejezően, egységes hangzással, tiszta intonációval énekel. Új dalokat megfelelő előkészítést követően hallás után megtanul. \\ \hline
              5--6 & ének & Fejlődik a többszólamú éneklési készsége. Kánonban énekel társai\-val. \\ \hline
              5--6 & ének & Kreatívan részt vesz a generatív játékokban és feladatokban. Érzi az egyenletes lüktetést, tartja a tempót, érzékeli a tempóváltozást. A 6/8-os és 3/8-os metrumot helyesen hangsúlyozza. \\ \hline
              5--6 & ének & Felismeri és megszólaltatja a tanult zenei elemeket (metrum, dinamikai jelzések, ritmus, dallam). \\ \hline
              5--6 & ének & Szolmizálva énekel a tanult dalok stílusában megszerkesztett rövid dallamfordulatokat kézjelről, betűkottáról és hangjegyről. Megfelelő előkészítés után hasonló dallamfordulatokat rögtönöz. \\ \hline
              5--6 & ének & Fejlődik zenei memóriája és belső hallása. \\ \hline
              5--6 & ének & Felismeri és megfogalmazza a formai építkezés jelenségeit, fejleszti formaérzékét. \\ \hline
              5--6 & ének & Fejlődik hangszínhallása. Megkülönbözteti a tárogató, brácsa, cselló, nagybőgő, harsona, tuba, hárfa hangszínét. Ismeri a hangszerek alapvető jellegzetességeit. \\ \hline
              5--6 & ének & Különbséget tesz a népi zenekar, vonósnégyes, szimfonikus zenekar hangzása között. \\ \hline
              5--6 & ének & A két zenei korszakból zenehallgatásra kiválasztott művek közül 18-20 alkotást/műrészletet ismer. \\ \hline
              5--6 & vizkult & A vizuális nyelv és kifejezés eszközeit megfelelően alkalmazza az alkotó tevékenység során a vizuális emlékezet segítségével és megfigyelés alapján. \\ \hline
              5--6 & vizkult & Egyszerű kompozíciós alapelveket a kifejezésnek megfelelően használ a képalkotásban. \\ \hline
              5--6 & vizkult & Térbeli és időbeli változások lehetséges vizuális megjelenéseit értelmezi, egyszerű mozgásélményeket, időbeli változásokat megjelenít. \\ \hline
              5--6 & vizkult & A mindennapokban használt vizuális jeleket értelmezi, ennek analógiájára saját jelzésrendszereket alakít ki. \\ \hline
              5--6 & vizkult & Szöveg és kép együttes jelentését értelmezi különböző helyzetekben és alkalmazza különböző alkotó jellegű tevékenység során. \\ \hline
              5--6 & vizkult & Egyszerű következtetéseket fogalmaz meg az épített és tárgyi környezet elemző megfigyelése alapján. \\ \hline
              5--6 & vizkult & Alkotótevékenysége során a választott rajzi és tárgykészítési technikát megfelelően használja. \\ \hline
              5--6 & vizkult & Vizuális eszközökkel reflektál társművészeti alkotásokra. \\ \hline
              5--6 & vizkult & Azonosítja a legfontosabb művészettörténeti korokat. \\ \hline
              5--6 & vizkult & Pontosan és szabatosan megfogalmazza megfigyeléseit vizuális jelenségek, tárgyak, műalkotások elemzése során. \\ \hline
              5--6 & vizkult & Felismer szimbolikus és kulturális üzenetet közvetítő tárgyakat. \\ \hline
              5--6 & vizkult & Önállóan kérdez a vizuális megfigyelés és elemzés során. \\ \hline
              5--6 & vizkult & Önálló véleményt fogalmaz meg saját és mások munkájáról. \\ \hline
              7--8 & magyar & Képes a kulturált szociális érintkezésre, eligazodik és hatékonyan részt vesz a mindennapi páros és csoportos kommunikációs helyzetekben, vitákban. Figyeli és értelmezi partnerei kommunikációs szándékát, nem nyelvi jeleit.  \\ \hline
              7--8 & magyar & Érzelmeit kifejezi, álláspontját megfelelő érvek, bizonyítékok segítségével megvédi, ugyanakkor empatikusan beleéli magát mások gondolatvilágába, érzelmeibe, megérti mások cselekvésének mozgatórugóit. \\ \hline
              7--8 & magyar & A különböző megjelenésű és műfajú szövegeket átfogóan megérti, az adott szöveg szó szerinti jelentésén túli üzenetét is értelmezi, a szövegből információkat keres vissza. \\ \hline
              7--8 & magyar & Adott szöveg tartalmát összefoglalja, önállóan jegyzetet és vázlatot készít. Az olvasott szöveg tartalmával kapcsolatos saját véleményét szóban és írásban megfogalmazza, állításait indokolja. \\ \hline
              7--8 & magyar & Ismeri és alkalmazza a különböző mondatfajták használatát. Alkalmazza az írásbeli szövegalkotásban a mondatvégi, a tagmondatok, illetve mondatrészek közötti írásjeleket. Az összetett szavak és gyakoribb mozaikszók helyesírását ismeri, szükség szerint helyesírási segédletet használ. \\ \hline
              7--8 & magyar & Ismeri a tömegkommunikáció fogalmát, legjellemzőbb területeit. \\ \hline
              7--8 & magyar & A könnyebben besorolható műveket műfajilag azonosítja, 8-10 műfajt műnemekbe sorol, és a műnemek lényegét megfogalmazza. A különböző regénytípusok műfaji jegyeit felismeri, a szereplőket jellemezni tudja, a konfliktusok mibenlétét feltárja. Felismeri az alapvető lírai műfajok sajátosságait különböző korok alkotóinak művei alapján (elsősorban 19-20. századi alkotások). Felismeri néhány lírai mű beszédhelyzetét, a megszólító-megszólított viszony néhány jellegzetes típusát, azonosítja a művek tematikáját, meghatározó motívumait. Felfedez műfaji és tematikus-motivikus kapcsolatokat, azonosítja a zenei és ritmikai eszközök típusait, felismeri funkciójukat, hangulati hatásukat. Azonosít képeket, alakzatokat, szókincsbeli és mondattani jellegzetességeket, a lexika jelentésteremtő szerepét megérti a lírai szövegekben, megismeri a kompozíció meghatározó elemeit (pl. tematikus szerkezet, tér- és időszerkezet, logikai szerkezet, beszédhelyzet és változása). Konkrét szövegpéldán megmutatja a mindentudó és a tárgyilagos elbeszélői szerep különbözőségét, továbbá a közvetett és a közvetlen elbeszélésmód eltérését. Képes a drámákban, filmekben megjelenő emberi kapcsolatok, cselekedetek, érzelmi viszonyulások, konfliktusok összetettségének értelmezésére és megvitatására. Az olvasott, megtárgyalt művek erkölcsi kérdésfeltevéseire véleményében, érveiben tud támaszkodni. \\ \hline
              7--8 & magyar & Egyszerű meghatározását adja a következő fogalmaknak: novella, rapszódia, lírai én, hexameter, pentameter, disztichon, szinesztézia, szimbólum, tragédia, komédia, dialógus, monológ. Néhány egyszerűbb meghatározás közül kiválasztja azt, amely a következő fogalmak valamelyikéhez illik: fordulat, retorika, paródia, helyzetkomikum, jellemkomikum. \\ \hline
              7--8 & magyar & Szöveghűen felidéz műveket, műrészleteket. \\ \hline
              7--8 & magyar & Beszámolót, kiselőadást, prezentációt készít és tart írott és elektronikus forrásokból, kézikönyvekből, atlaszokból/szakmunkákból, a témától függően statisztikai táblázatokból, grafikonokból, diagramokból. \\ \hline
              7--8 & magyar & Érti a média alapvető kifejezőeszközeit, az írott és az elektronikus sajtó műfajait. Ismeri a média, kitüntetetten az audiovizuális média és az internet társadalmi szerepét, működési módjának legfőbb jellemzőit. Önálló, kritikus attitűddel, tudatosan használja a médiát. \\ \hline
              7--8 & idegen nyelv & A2 szintű nyelvtudás: \\ \hline
              7--8 & idegen nyelv & Egyszerű hangzó szövegekből kiszűri a lényeget és néhány konkrét információt. \\ \hline
              7--8 & idegen nyelv & Kérdésekre válaszol, rövid beszélgetésekben vesz részt. \\ \hline
              7--8 & idegen nyelv & Egyre bővülő szókinccsel, egyszerű nyelvi eszközökkel megfogalmazva történetet mesél el, valamint leírást ad saját magáról és közvetlen környezetéről. \\ \hline
              7--8 & idegen nyelv & Ismert témákról írt rövid szövegeket megért és értelmez. Különböző típusú, egyszerű írott szövegekben megtalálja a fontos információkat. \\ \hline
              7--8 & idegen nyelv & Összefüggő mondatokat, rövid szöveget ír hétköznapi, őt érintő témákról. \\ \hline
              7--8 & történelem & Az újkori és modern kori egyetemes és magyar történelmi jelenségek, események feldolgozásával a jelenben zajló folyamatok előzményeit felismeri. Kialakul a nemzeti azonosságtudata. \\ \hline
              7--8 & történelem & Felismeri, hogy a modern nemzetállamokat különböző kultúrájú, vallású, szokású, életmódú népek, nemzetiségek együttesen alkotják. \\ \hline
              7--8 & történelem & Felismeri a múltat és a történelmet formáló alapvető folyamatokat, összefüggéseket (pl. a technikai fejlődés hatásai a társadalomra és a gazdaságra) és azonosít egyszerű, átélhető erkölcsi tanulságokat (pl. társadalmi kirekesztés). \\ \hline
              7--8 & történelem & Azonosítja az új- és modern korban élt emberek, közösségek élet-, gondolkodás- és szokásmódjait, a hasonlóságokat és különbségeket felismeri.
 \\ \hline
              7--8 & történelem & Felismeri, hogy a múltban való tájékozódást támogatják a kulcsfogalmak és fogalmak. E fogalmak segítik a történelmi tájékozódás és gondolkodás kialakulását, fejlődését. \\ \hline
              7--8 & történelem & Felismeri, hogy az utókor a nagy történelmi személyiségek, nemzeti hősök cselekedeteit a közösségek érdekében végzett tevékenységek szempontjából értékeli, példát tud mondani ellentétes értékelésre. \\ \hline
              7--8 & történelem & Ismeri a XIX--XX. század nagy korszakainak megnevezését, illetve egy-egy korszak főbb jelenségeit, jellemzőit, szereplőit, összefüggéseit. \\ \hline
              7--8 & történelem & Ismeri a magyar történelem főbb csomópontjait egészen napjainkig. E hosszú történelmi folyamat meghatározó szereplőit azonosítja, egy-egy korszak főbb kérdéseit felismeri. \\ \hline
              7--8 & történelem & Ismeri a korszak meghatározó egyetemes és magyar történelmének eseményeit, évszámait, történelmi helyszíneit. Képes összefüggéseket találni a térben és időben eltérő fontosabb történelmi események között, különös tekintettel azokra, melyek a magyarságot közvetlenül vagy közvetetten érintik. \\ \hline
              7--8 & történelem & Tudja, hogy az egyes népek és államok a korszakban eltérő társadalmi, gazdasági és vallási körülmények között éltek, de a modern kor beköszöntével a köztük lévő kapcsolatok széles körű rendszere épült ki. \\ \hline
              7--8 & történelem & Tudja, hogy Európához köthetők a modern demokratikus viszonyokat megalapozó szellemi mozgalmak és dokumentumok. \\ \hline
              7--8 & történelem & Tudja, hogy Magyarország a trianoni békediktátum következtében elvesztette területének és lakosságának kétharmadát, és közel ötmillió magyar került kisebbségi sorba. Jelenleg közel hárommillió magyar nemzetiségű él a szomszédos államokban és a világ különböző részein, akik szintén a magyar nemzet tagjainak tekintendők. \\ \hline
              7--8 & történelem & Tudja, hogy a holokausztnak több százezer magyar áldozata is volt, és  tisztában van ennek hazai és nemzetközi történelmi, politikai előzményeivel, körülményeivel és erkölcsi vonatkozásaival. \\ \hline
              7--8 & történelem & Tudja, hogy a társadalmakban eltérő jogokkal rendelkező és vagyoni helyzetű emberek alkotnak rétegeket, csoportokat. \\ \hline
              7--8 & történelem & Különbséget tesz a demokrácia és a diktatúra között, s tud azokra példát mondani a feldolgozott történelmi korszakokból és napjainkból. \\ \hline
              7--8 & történelem & Felismeri a hazánkat és a világot fenyegető globális veszélyeket, betegségeket, terrorizmust, munkanélküliséget. \\ \hline
              7--8 & történelem & Különbséget tesz a történelem különböző típusú forrásai között, felismeri az egy-egy korszakra jellemző képeket, tárgyakat, épületeket. \\ \hline
              7--8 & történelem & Tudja, hogy hol kell a fontos események forrásait kutatni, és ha szükséges, összehasonlítja a megismert jelenségeket, eseményeket. \\ \hline
              7--8 & történelem & Ismeri a híres történelmi személyiségek jellemzéséhez szükséges adatokat, eseményeket és kulcsszavakat. \\ \hline
              7--8 & történelem & A hallott és olvasott szövegekből, különböző médiumok anyagából következtetéseket von le és történelmi ismeretekre tesz szert. \\ \hline
              7--8 & történelem & Információt gyűjt adott témához könyvtárban és múzeumban; olvasmányairól lényeget kiemelő jegyzetet készít. \\ \hline
              7--8 & történelem & Könyvtári munkával és az internet kritikus használatával forrást gyűjt, kiselőadást tart és érvel. \\ \hline
              7--8 & történelem & Felismeri a sajtó és média szerepét a nyilvánosságban, azonosítja a reklám és médiapiac jellegzetességeit. \\ \hline
              7--8 & történelem & Szóbeli beszámolót készít önálló gyűjtőmunkával szerzett ismereteiről, és kiselőadást tart. \\ \hline
              7--8 & történelem & Megfogalmazza saját véleményét, figyelembe veszi mások véleményét és reflektál azokra. \\ \hline
              7--8 & történelem & Példák segítségével értelmezi az alapvető emberi, gyermek- és diákjogokat, valamint a társadalmi szolidaritás különböző formáit. \\ \hline
              7--8 & történelem & Példák segítségével bemutatja a legfontosabb állampolgári jogokat és kötelességeket, értelmezi ezek egymáshoz való viszonyát. \\ \hline
              7--8 & történelem & Azonosítja a gazdasági és pénzügyi terület fontosabb szereplőit, egyszerű családi költségvetést készít és mérlegeli a háztartáson belüli megtakarítási lehetőségeket. \\ \hline
              7--8 & ének & Előad 14-17 népdalt, balladát, históriás éneket több versszakkal, valamint 8-10 műzenei idézetet emlékezetből, g–f” hangterjedelemben. \\ \hline
              7--8 & ének & Kifejezően, egységes hangzással, tiszta intonációval énekel. Új dalokat megfelelő előkészítést követően hallás után megtanul. \\ \hline
              7--8 & ének & Többszólamú éneklésben részt vesz. Csoportosan egyszerű orgánumokat megszólaltat. \\ \hline
              7--8 & ének & Kreatívan  részt vesz a generatív játékokban és feladatokban. Érzi az egyenletes lüktetést, tartja a tempót, érzékeli a tempóváltozást. A 5/8-os és 7/8-os és 8/8-os metrumot helyesen hangsúlyozza. \\ \hline
              7--8 & ének & Felismeri és megszólaltatja a tanult zenei elemeket (metrum, dinamikai jelzések, ritmus, dallam). \\ \hline
              7--8 & ének & Fejlődik zenei memóriája és belső hallása. \\ \hline
              7--8 & ének & Felismeri és megfogalmazza a formai építkezés jelenségeit, fejleszti formaérzékét. \\ \hline
              7--8 & ének & Fejlődik hangszínhallása. Megkülönbözteti a cimbalom, lant, csembaló, orgona, szaxofon hangszínét. Ismeri a hangszerek alapvető jellegzetességeit. \\ \hline
              7--8 & ének & A zenei korszakokból kiválasztott zeneművek közül 20-25 alkotást/műrészletet ismer. \\ \hline
              7--8 & vizkult & Önállóan alkalmaz célirányos vizuális megfigyelési szempontokat. \\ \hline
              7--8 & vizkult & A vizuális nyelv és kifejezés eszközeit tudatosan és pontosan alkalmazza az alkotótevékenység során adott célok kifejezése érdekében. \\ \hline
              7--8 & vizkult & Bonyolultabb kompozíciós alapelveket használ kölönböző célok érdekében. \\ \hline
              7--8 & vizkult & Térbeli és időbeli változásokat vizuálisan megjelenít. Kifejező vagy közlő szándékot értelmez és következtetéseket fogalmaz meg. \\ \hline
              7--8 & vizkult & Alapvetően közlő funkcióban lévő képi vagy képi és szöveges megjelenéseket egyszerűen értelmez. \\ \hline
              7--8 & vizkult & Az épített és tárgyi környezet elemző megfigyelése alapján összetettebb következtetéseket megfogalmaz. \\ \hline
              7--8 & vizkult & Több jól megkülönböztethető technikát, médiumot (pl. állókép-mozgókép, síkbeli-térbeli) tudatosan használ alkotótevékenység során. \\ \hline
              7--8 & vizkult & A vizuális kommunikációs eszközök és formák rendszerezett megismerése során megalapozza a médiatudatos gondolkodását. \\ \hline
              7--8 & vizkult & Felismeri a mozgóképi közlésmód, az írott sajtó és az online kommunikáció szövegszervező alapeszközeit. \\ \hline
              7--8 & vizkult & Megkülönböztet mozgóképi szövegeket a valóság ábrázolásához való viszonyuk, az alkotói szándék és nézői elvárás karaktere alapján. \\ \hline
              7--8 & vizkult & Önállóan értelmezi a társművészeti kapcsolatok lehetőségeit. \\ \hline
              7--8 & vizkult & Megkülönbözteti a legfontosabb kultúrákat, művészettörténeti korokat, stílusirányzatokat. A meghatározó alkotók műveit felismeri. \\ \hline
              7--8 & vizkult & Mélységükben elemzi, összehasonlítja a vizuális jelenségeket, tárgyakat, műalkotásokat. \\ \hline
              7--8 & vizkult & Önállóan kérdez a vizuális megfigyelés és elemzés során. \\ \hline
              7--8 & vizkult & Önálló véleményt fogalmaz meg saját és mások munkájáról. \\ \hline
      \end{longtable}
\end{small}


\subsection{9--12. évfolyam}
\begin{small}
  \begin{longtable}{c | p{2cm} |  p{11cm} }
    \textbf{Évf.} & \textbf{Témater.} & \textbf{Tanulási eredmény} \\ \hline \hline
    \endhead

              9--10 & idegen nyelv & Megérti nagy vonalakban az idegen nyelvű köznyelvi beszédet, ha az számára rendszeresen előforduló, ismerős témákról folyik. \\ \hline
              9--10 & idegen nyelv & A mindennapi élet legtöbb helyzetében boldogul idegen nyelven, gondolatokat cserél, véleményt mond, érzelmeit kifejezi és stílusában a kommunikációs helyzethez alkalmazkodik. \\ \hline
              9--10 & idegen nyelv & A begyakorolt szerkezetekkel érthetően, folyamatoshoz közelítően beszél. Az átadott információ lényegét megközelítő tartalmi pontossággal fejti ki. \\ \hline
              9--10 & idegen nyelv & Megérti a hétköznapi nyelven írt, érdeklődési köréhez kapcsolódó, lényegre törő, autentikus vagy kismértékben szerkesztett szövegekben az általános vagy részinformációkat. \\ \hline
              9--10 & idegen nyelv & A tanuló több műfajban is egyszerű, rövid, összefüggő szövegeket fogalmaz ismert, hétköznapi témákról. Írásbeli megnyilatkozásaiban kezdenek megjelenni műfaji sajátosságok és különböző stílusjegyek. \\ \hline
              9--10 & történelem & Megismeri az ókori, középkori és kora újkori egyetemes és magyar kultúrkincs elemeit és rendszerét, tudatosan vállalja az egyetemes emberi értékeket, felismeri és elfogadja a családhoz, lakóhelyhez, nemzethez, Európához, világhoz tartozás fontosságát. \\ \hline
              9--10 & történelem & Ismeri és felismeri a múltat és a történelmet formáló összetett folyamatokat, a látható és a háttérben meghúzódó összefüggéseket, és azonosítja ezek erkölcsi-etikai aspektusait. \\ \hline
              9--10 & történelem & Azonosítja a korábbi korokban élt emberek, közösségek élet-, gondolkodás- és szokásmódjait, felismeri a különböző államformák működési jellemzőit. \\ \hline
              9--10 & történelem & Ismeri a civilizációk történetének jellegzetes sémáját (kialakulás, virágzás, hanyatlás). \\ \hline
              9--10 & történelem & Ismeri és alkalmazza az árnyalt történelmi tájékozódást és gondolkodást segítő kulcsfogalmakat. \\ \hline
              9--10 & történelem & Felismeri, hogy az utókor a nagy történelmi személyiségek, nemzeti hősök cselekedeteit a közösségek érdekében végzett tevékenységek szempontjából értékeli, példákat hoz különböző korok eltérő értékítéleteiről egy-egy történelmi személyiség kapcsán. \\ \hline
              9--10 & történelem & Az egyes népeket vallásuk és kultúrájuk, életmódjuk alapján azonosítja és ismeri. Felismeri, hogy a vallási előírások, valamint az államok által megfogalmazott szabályok döntő mértékben befolyásolhatják a társadalmi viszonyokat és a mindennapokat. \\ \hline
              9--10 & történelem & Tudja, hogy a történelmi jelenségeket, folyamatokat társadalmi, gazdasági, szellemi tényezők együttesen befolyásolják. \\ \hline
              9--10 & történelem & Ismeri a világ és az európai kontinens eltérő fejlődési irányait, ezek társadalmi, gazdasági és szellemi hátterét. Tudja azonosítani Európa különböző régióinak eltérő fejlődési útjait. \\ \hline
              9--10 & történelem & Felismeri a meghatározó vallási, társadalmi, gazdasági, szellemi összetevőket egy-egy történelmi jelenség, folyamat értelmezésénél. \\ \hline
              9--10 & történelem & Érti és értelmezi az eltérő uralkodási formák és társadalmi, gazdasági viszonyok közötti összefüggéseket. \\ \hline
              9--10 & történelem & Ismeri a keresztény Magyar Királyság létrejöttének, virágzásának és hanyatlásának főbb állomásait, a kora újkor békés építőmunkájának eredményeit, valamint a polgári Magyarország kiépülésének meghatározó gondolatait és kulcsszereplőit. \\ \hline
              9--10 & történelem & Beszámolót, kiselőadást készít és tart különböző írott forrásokat -- történelmi kézikönyveket, atlaszokat/szakmunkákat, statisztikai táblázatokat, grafikonokat, diagramokat és az internetet -- használva. \\ \hline
              9--10 & történelem & Áttekinti és értékeli a rendelkezésre álló forrásokat, valamint rendszerezi és értelmezi a szerzett információkat. \\ \hline
              9--10 & történelem & Kérdéseket fogalmaz meg a forrás megbízhatóságára és a szerző esetleges elfogultságára vonatkozóan. \\ \hline
              9--10 & történelem & Megfigyeli a különböző magatartástípusokat és élethelyzeteket, ezekből következtetéseket von le. \\ \hline
              9--10 & történelem & Írott és hallott szövegből a lényeget kiemeli tételmondatok meghatározásával, szövegek tömörítésével és átfogalmazásával egyaránt. \\ \hline
              9--10 & történelem & Feltárja a többféleképpen értelmezhető szövegek jelentésrétegeit. \\ \hline
              9--10 & történelem & Feltevéseket fogalmaz meg történelmi személyiségek cselekedeteinek, viselkedésének mozgatórugóiról. \\ \hline
              9--10 & történelem & Elbeszél és eljátszik történelmi helyzeteket a különböző szereplők nézőpontjából. \\ \hline
              9--10 & történelem & Képes saját véleményét megfogalmazni, és közben meg tudja különböztetni egy vitában a tárgyilagos érvelést és a személyeskedést. \\ \hline
              9--10 & történelem & Képes történelmi témákat vizuálisan ábrázolni. \\ \hline
              9--10 & történelem & Kronológiai adatokat rendszerez, történelmi időszakokat meghatároz kronológiai adatok alapján. \\ \hline
              9--10 & történelem & Használja a történelmi korszakok és periódusok nevét. \\ \hline
              9--10 & történelem & Összehasonlítja a történelmi időszakokat, a változások szempontjából egybeveti az eltérő korszakok emberi sorsait. \\ \hline
              9--10 & történelem & Képes érzékelni és elemezni az egyetemes és a magyar történelem eltérő időbeli ritmusát, illetve ezek kölcsönhatásait. Az egyes korszakokat komplex módon jellemzi. \\ \hline
              9--10 & történelem & Különböző információforrásokból önálló térképvázlatokat tud rajzolni, különböző időszakok történelmi térképeit össze tudja hasonlítani. Le tudja olvasni a történelmi tér változásait és az adott témához leginkább megfelelő térképet választ. \\ \hline
              9--10 & ének & 8-10 népzenei, valamint 8-10 műzenei idézetet részben kottából, részben emlékezetből csoportosan előad. \\ \hline
              9--10 & ének & Kifejezően, egységes hangzással, tiszta intonációval énekel. Új dalokat megfelelő előkészítést követően hallás után megtanul. \\ \hline
              9--10 & ének & Egyszerű két- és háromszólamú kórusműveket vagy azok részleteit, kánonokat énekel. \\ \hline
              9--10 & ének & Felismeri és megfogalmazza a formai építkezés jelenségeit, fejleszti formaérzékét. \\ \hline
              9--10 & ének & Ismeri a hangszerek alapvető jellegzetességeit. \\ \hline
              9--10 & ének & A generatív tevékenységek eredményeként érzékeli, felismeri a zenei kifejezések, a formák, a műfajok, és a zenei eszközök közti összefüggéseket. \\ \hline
              9--10 & ének & Megismeri és értelmezi a kottakép elemeit és az alapvető zenei kifejezéseket. \\ \hline
              9--10 & ének & A zenei korszakokból kiválasztott zeneművek közül 20-25 alkotást/műrészletet ismer és felismer. \\ \hline
              9--10 & ének & A zenei műalkotások megismerése révén tájékozódik korunk kulturális sokszínűségében. \\ \hline
              9--10 & dráma & Kifejezi önmagát; dramatikus és tánchelyzetekben mások előtt megnyilatkozik és együttműködik társaival. \\ \hline
              9--10 & dráma & Részt vesz szerepjátékokban, csoportos improvizációkban. \\ \hline
              9--10 & dráma & Tudatosan és kreatívan alkalmazza a megismert munkaformákat. \\ \hline
              9--10 & dráma & Használja a megismert dramaturgiai fogalomkészletet. \\ \hline
              9--10 & dráma & A drámajáték eszközeivel csoportos munkában feldolgozza társaival a látott színházi előadást. \\ \hline
              9--10 & vizkult & Önállóan kiválaszt célirányos vizuális megfigyelési szempontokat. \\ \hline
              9--10 & vizkult & Önállóan alkalmazza a vizuális nyelv eszközeit az alkotótevékenység és a vizuális jelenségek elemzése, értelmezése során. \\ \hline
              9--10 & vizkult & Tudatosan használ bonyolultabb kompozíciós alapelveket különböző célok érdekében. \\ \hline
              9--10 & vizkult & Értelmezi a térbeli és időbeli változások vizuális megjelenítését, egyszerű mozgóképeket készít. \\ \hline
              9--10 & vizkult & Értelmezi a képi, vagy képi és szöveges megjelenéseket. \\ \hline
              9--10 & vizkult & Tudatosan gondolkodik a médiáról a tömegkommunikációs eszközök és formák rendszerező feldolgozásában. \\ \hline
              9--10 & vizkult & Tervez és makettet készít. \\ \hline
              9--10 & vizkult & Az alkotótevékenységekben tanult technikákat a célnak megfelelően tudatosan alkalmaz. \\ \hline
              9--10 & vizkult & Árnyaltan értelmezi a tárművészeti kapcsolatokat. \\ \hline
              9--10 & vizkult & Ismeri és rendszerben látja a különböző kultúrákat, művészettörténeti korokat, stílusirányzatokat, felismeri a meghatározó alkotók műveit. \\ \hline
              9--10 & vizkult & Felismeri az építészet alapvető elrendezési, szerkezeti alapelveit és egy-egy stílus meghatározó vonásait. \\ \hline
              9--10 & vizkult & Elemez, összehasonlít vizuális jelenségeket, tárgyakat, műalkotásokat; alkalmazza a műelemzés módszereit. \\ \hline
              9--10 & vizkult & Adott vizuális problémakkal kapcsolatban önálló kérdéseket fogalmaz meg. \\ \hline
              9--10 & vizkult & Kreatívan old meg problémákat. \\ \hline
              9--10 & vizkult & Önálló véleményt fogalmaz meg saját és mások munkájáról. \\ \hline
              9--10 & mozgókép\-kultúra & Felismeri és megnevezi a mozgóképi közlésmód, az írott sajtó és az online kommunikáció szövegszervező alapeszközeit. \\ \hline
              9--10 & mozgókép\-kultúra & Alkalmazza az ábrázolás megismert eszközeit egyszerű mozgóképi szövegek értelmezése során (cselekmény és történet megkülönböztetése, szemszög, nézőpont, képkivágat, kameramozgás jelentése az adott szövegben, montázsfunkciók felismerése). \\ \hline
              9--10 & mozgókép\-kultúra & A mozgóképi szövegeket megkülönbözteti a valóság ábrázolásához való viszonyuk, az alkotói szándék és a nézői elvárás karaktere szerint (dokumentumfilm és játékfilm, műfaji és szerzői beszédmód). \\ \hline
              9--10 & mozgókép\-kultúra & Életkorának megfelelően megkülönbözteti a fikció és a virtuális fogalmait egymástól.  \\ \hline
              9--10 & mozgókép\-kultúra & Ismeri a média alapfunkcióit, a kommunikáció alapfordulatait, megfogalmazza, mitől teljesül valamely kor és társadalom nyilvánossága. \\ \hline
              9--10 & mozgókép\-kultúra & Ismeri a kereskedelmi és közszolgálati médiaintézmények elsődleges céljait és eszközeit a médiaipari versenyben. \\ \hline
              9--10 & mozgókép\-kultúra & Megkülönbözteti azokat a fontosabb tényezőket, melyek alapján jellemezhetőek a médiaintézmények célközönségei. \\ \hline
              9--10 & mozgókép\-kultúra & Meghatározza és alkalmas példákkal illusztrálja a sztereotípia és a reprezentáció fogalmát; indokolja az egyszerűbb reprezentációk különbözőségeit. \\ \hline
              9--10 & mozgókép\-kultúra & Érvekkel támasztja alá álláspontját olyan vitában, amik a médiaszövegek (pl. reklám, hírműsor) valóságtartalmáról folynak. \\ \hline
              9--10 & mozgókép\-kultúra & Átlátja az internetes és a mobilkommunikáció fontosabb sajátosságait, az internethasználat biztonságának problémáit. \\ \hline
              9--10 & mozgókép\-kultúra & Életkorának megfelelően önállóan összegyűjti, rendszerezi és megfigyeli a különböző  médiumokból és médiumokról szóló ismereteket. \\ \hline
              9--10 & mozgókép\-kultúra & Alkalmazza a mozgóképi szövegekkel, a média működésével kapcsolatos ismereteit a műsorválasztás során.  \\ \hline
              9--10 & magyar & Átfogóan értelmez szövegeket (pl. szépirodalmi, dokumentum- és ismeretterjesztő szöveg), értelmezi a szöveg szó szerinti jelentésén túli üzenetét, visszakeres a szövegben információkat. \\ \hline
              9--10 & magyar & A szöveg tartalmát összefoglalja, tud önállóan jegyzetet és vázlatot készíteni. Képes az olvasott szöveg tartalmával kapcsolatos saját véleményét szóban és írásban megfogalmazni, indokolni. \\ \hline
              9--10 & magyar & Felismeri és értelmezi a szövegek kapcsolatát és különbségét mind szóbeli, mint írásbeli műfajokban. \\ \hline
              9--10 & magyar & Helyesen és tudatosan alkalmazza a szövegfonetikai eszközöket a  memoriterek szöveghű tolmácsolásánál. \\ \hline
              9--10 & magyar & Alkalmazza a művek műfaji természetének megfelelő szöveg-feldolgozási eljárásokat, megközelítési módokat. \\ \hline
              9--10 & magyar & Felismeri és alkalmazza a szépirodalmi és a nem szépirodalmi szövegekben megjelenített értékeket, erkölcsi kérdéseket, motivációkat, magatartásformákat. \\ \hline
              9--10 & magyar & Tájékozott a feldolgozott lírai alkotások különböző műfajaiban és hangnemeiben. \\ \hline
              9--10 & magyar & Bizonyítja különféle szövegek megértését a szöveg felépítésére, grammatikai jellemzőire, témahálózatára, tagolására irányuló elemzéssel; szöveghűen felolvas; kellő tempójú, olvasható, rendezett az írása.  \\ \hline
              9--10 & magyar & Szóbeli és írásbeli kommunikációs helyzetekben megválasztja a megfelelő hangnemet, nyelvváltozatot, stílusréteget. Alkalmazza a művelt köznyelv (regionális köznyelv), illetve a nyelvváltozatok nyelvhelyességi normáit, képes felismerni és értelmezni az attól eltérő nyelvváltozatokat. \\ \hline
              9--10 & magyar & Definíciót, magyarázatot és egyszerűbb értekezést készít az olvasmányaiból. Megfogalmaz a tárgyalt problémákkal összefüggésben kérdéseket és felmerülő problémákat. \\ \hline
              9--10 & magyar & Ismeri a hivatalos írásművek jellemzőit. Képes önálló szövegalkotásra ezek gyakori műfajaiban. \\ \hline
              9--10 & magyar & Alkalmazza az idézés szabályait és etikai normáit. \\ \hline
              9--10 & magyar & Az órai eszmecserékben és az irodalmi művekben megjelenő álláspontokat azonosítja, megvitatja és összehasonlítja eltérő véleményekkel. \\ \hline
              9--10 & magyar & Képes tudásanyagának megfogalmazására írásban a magyar és a világirodalom kiemelkedő alkotóiról. \\ \hline
              9--10 & magyar & Bemutatja a tanult irodalomtörténeti korszakok és stílusirányzatok sajátosságait. \\ \hline
              9--10 & magyar & Eligazodik a 2000 utáni évek kortárs irodalmában (is), és önállóan kezeli az online irodalmi felületeket. \\ \hline
              9--10 & magyar & Ismeri a digitális (szak)irodalmi szövegtárak világát, és azokban otthonosan mozog. \\ \hline
              9--10 & magyar & Ő maga is rendszeresen ír papíralapú, illetve digitális írásokat. \\ \hline
              9--10 & magyar & Részt vesz beszélgetésekben, vitákban, irodalomról folyó diskurzusokban és projektekben – ideértve a diákszínjátszást, újságírást és szerkesztést is. \\ \hline
              9--10 & magyar & Beszélgetéseiben, vitáiban tud idézni memoriterekből és egyéb irodalmi alkotásokból. \\ \hline
              11--12 & idegen nyelv & Főbb vonalaiban és egyes részleteiben is megérti a köznyelvi beszédet a számára ismerős témákról. \\ \hline
              11--12 & idegen nyelv & Idegen nyelven önállóan boldogul, véleményt mond és érvel a mindennapi élet legtöbb, akár váratlan helyzetében is. Stílusában és regiszterhasználatában alkalmazkodik a kommunikációs helyzethez. \\ \hline
              11--12 & idegen nyelv & A szintnek megfelelő szókincs és szerkezetek segítségével fejezi ki magát az ismerős témakörökben. Folyamatosan, érthetően, a főbb pontok tekintetében tartalmilag pontosan, megfelelő stílusban beszél. \\ \hline
              11--12 & idegen nyelv & Több műfajban, részleteket is tartalmazó, összefüggő szövegeket fogalmaz ismert, hétköznapi és elvontabb témákról. Írásbeli megnyilatkozásaiban műfaji sajátosságokat és különböző stílusjegyeket alkalmaz. \\ \hline
              11--12 & idegen nyelv & Megérti a gondolatmenet lényegét és egyes részinformációkat a nagyrészt közérthető nyelven írt, érdeklődési köréhez kapcsolódó, lényegre törően megfogalmazott szövegekben. \\ \hline
              11--12 & történelem & Az újkori és modern kori egyetemes és magyar történelmi jelenségek, események rendszerező feldolgozásával a jelenben zajló folyamatok előzményeit felismeri, a nemzeti öntudatra és aktív állampolgárságra törekszik. \\ \hline
              11--12 & történelem & Felismeri a múltat és a történelmet formáló, alapvető folyamatok ok-okozati összefüggéseit (pl. a globalizáció felerősödése és a lokális közösségek megerősödése) és az egyszerű, átélhető erkölcsi tanulságokat (pl. társadalmi kirekesztés) azonosítja, jelenre vonatkoztatja azokat. \\ \hline
              11--12 & történelem & Azonosítja az új- és modern korban élt emberek, közösségek sokoldalú élet-, gondolkodás- és szokásmódjait, a hasonlóságokat és különbségeket árnyaltan felismeri, több szempontból értékeli. \\ \hline
              11--12 & történelem & A civilizációk jellegzetes sémáit alkalmazza az újkori és modern kori egyetemes történelem értelmezése során. \\ \hline
              11--12 & történelem & Árnyaltan gondolkodik a történelemről, és ehhez ismeri az értelmezést segítő kulcsfogalmakat. \\ \hline
              11--12 & történelem & Tudja és érti, hogy az utókor, a történelmi emlékezet a nagy történelmi személyiségek tevékenységét többféle módon és szempont szerint értékeli, egyben saját értékítélete megfogalmazásakor a közösség hosszú távú nézőpontját tudja alkalmazni. \\ \hline
              11--12 & történelem & Ismeri a XIX--XX. század kisebb korszakainak megnevezését, egy-egy korszak főbb jelenségeit, jellemzőit, szereplőit, összefüggéseit. \\ \hline
              11--12 & történelem & Ismeri a magyar történelem főbb csomópontjait az 1848--1849-es szabadságharc leverésétől az Európai Unióhoz való csatlakozásunkig. \\ \hline
              11--12 & történelem & Beazonosítja a történelmi folyamat meghatározó összefüggéseit, bemutatja és elemzi egy-egy korszak főbb kérdéseit \\ \hline
              11--12 & történelem & Ismeri az új- és modern korban meghatározó egyetemes és magyar történelem eseményeit, évszámait, történelmi helyszíneit. \\ \hline
              11--12 & történelem & Képes összefüggéseket találni a térben és időben eltérő történelmi események között, különös tekintettel azokra, amelyek a magyarságot közvetlenül vagy közvetetten érintik. \\ \hline
              11--12 & történelem & Tudja, hogy a XIX–XX. században lényegesen átalakult Európa társadalma és gazdasága (polgárosodás, iparosodás), és ezzel párhuzamosan új eszmeáramlatok, politikai mozgalmak, pártok jelennek meg. \\ \hline
              11--12 & történelem & Felismeri, hogy az Egyesült Államok milyen körülmények között vált a mai világ vezető hatalmává, és rá tud mutatni az ebből fakadó ellentmondásokra. \\ \hline
              11--12 & történelem & Tudja a trianoni békediktátum máig tartó hatását, következményeit értékelni, és felismeri a határon túli magyarság sorskérdéseit. \\ \hline
              11--12 & történelem & Tudja a demokratikus és diktatórikus államberendezkedések közötti különbségeket, felismeri és elemzi a demokratikus berendezkedés előnyeit és működési nehézségeit. \\ \hline
              11--12 & történelem & Felismeri a globalizálódó világ problémáit (pl. túlnépesedés, betegségek, elszegényedés, munkanélküliség, élelmiszerválság, tömeges migráció). \\ \hline
              11--12 & történelem & Él a globalizáció előnyeivel, benne az európai állampolgársággal. \\ \hline
              11--12 & történelem & Ismeri az alapvető emberi jogokat, valamint állampolgári jogokat és kötelezettségeket, Magyarország politikai rendszerének legfontosabb intézményeit, érti a választási rendszer működését. \\ \hline
              11--12 & történelem & Ismereteket merít különböző forrásokból, történelmi, társadalomtudományi, filozófiai és etikai kézikönyvekből, atlaszokból, szaktudományi munkákból. Történelmi kutatást folytat ezek segítségével. \\ \hline
              11--12 & történelem & Kiselőadásokat, beszámolókat önállóan jegyzetel. \\ \hline
              11--12 & történelem & Kritikusan és tudatosan használja az internetet történelmi, filozófia- és etikatörténeti ismeretek megszerzése érdekében. \\ \hline
              11--12 & történelem & Különböző történelmi elbeszéléseket (pl. emlékiratok) hasonlít össze a narráció módja alapján. \\ \hline
              11--12 & történelem & Vizsgálja és megítéli az egyes szövegeket, hanganyagokat, filmeket a történelmi hitelesség szempontjából. \\ \hline
              11--12 & történelem & Történelmi jeleneteket el tud mesélni, néha el tudja játszani azokat különböző szempontokból. \\ \hline
              11--12 & történelem & Felismeri és értelmezi az erkölcsi kérdéseket felvető élethelyzeteket. \\ \hline
              11--12 & történelem & Önálló véleményt fogalmaz meg társadalmi, történelmi eseményekről, szereplőkről, jelenségekről, filozófiai kérdésekről. \\ \hline
              11--12 & történelem & Mások érvelését összefoglalja, értékeli és figyelembe veszi, miközben saját álláspontját gazdagítja. \\ \hline
              11--12 & történelem & A történelmi-társadalmi adatokat, modelleket és elbeszéléseket elemzi a bizonyosság, a lehetőség és a valószínűség szempontjából. \\ \hline
              11--12 & történelem & Összehasonlítja a társadalmi-történelmi jelenségeket strukturális és funkcionális szempontok szerint. Tisztázza saját értékeit, összehasonlítja másokéval. \\ \hline
              11--12 & történelem & Képes történelmi-társadalmi témákat vizuálisan ábrázolni, esszét írni (filozófiai kérdésekről is), ennek kapcsán kérdéseket fogalmaz meg. \\ \hline
              11--12 & történelem & Sokoldalóan tájékozódik a történelmi időkben. \\ \hline
              11--12 & történelem & Összehasonlítja és feltárja a különböző időszakokat bemutató történelmi térképek változásainak hátterét (területi változások, népsűrűség, vallási megosztottság stb.). \\ \hline
              11--12 & történelem & Ismeri és szakszerűen használja a nemzet, a kisebbség, a nemzetiség és a helyi társadalom fogalmát, fontos számára a társadalmi felelősségvállalás, a szolidaritás. \\ \hline
              11--12 & történelem & Tudja a nemzetgazdaság, a bankrendszer, a vállalkozási formák működésének legfontosabb szabályait. \\ \hline
              11--12 & történelem & Átlátja a munkavállalással összefüggő, a munkaviszonyhoz kapcsolódó adózási, egészség- és társadalombiztosítási kötelezettségek, illetve szolgáltatások rendszerét. \\ \hline
              11--12 & magyar & Felismeri és értő módon használja a tömegkommunikációs, illetve az audiovizuális, informatikai alapú szövegeket. Az értő, kritikus befogadáson kívül önállóan alkot szöveget  publicisztikai, audiovizuális és informatikai hátterű műfajban, a képi elemek, lehetőségek és a szöveg összekapcsolásában rejlő közlési lehetőségek kihasználásával. \\ \hline
              11--12 & magyar & Szövegelemzési, szövegértelmezési jártassággal rendelkezik a tanult leíró nyelvtani, szövegtani, jelentéstani, pragmatikai ismeretek alkalmazásában, és az elemzést kiterjeszti a szépirodalmi szövegek mellett a szakmai-tudományos, publicisztikai, közéleti (audiovizuális, informatikai alapú) szövegek feldolgozására, értelmezésére is. \\ \hline
              11--12 & magyar & Rendszeresen használja a könyvtárat, a különféle (pl. informatikai technológiákra épülő) információhordozókat, rendelkezik a képességgel, hogy kellő problémaérzékenységgel, kreativitással és önállósággal igazodjon el az információk világában; értelmesen és értékteremtően él az önképzés lehetőségeivel. \\ \hline
              11--12 & magyar & Bizonyítja különféle szövegek megértését a szöveg felépítésére, grammatikai jellemzőire, témahálózatára, tagolására irányuló elemzéssel. \\ \hline
              11--12 & magyar & Írása olvasható és rendezett. \\ \hline
              11--12 & magyar & Szóbeli és írásbeli kommunikációs helyzetekben megválasztja a megfelelő hangnemet, nyelvváltozatot, stílusréteget. Alkalmazza a művelt köznyelv (regionális köznyelv), illetve a nyelvváltozatok nyelvhelyességi normáit, képes felismerni és értelmezni az attól eltérő nyelvváltozatokat. \\ \hline
              11--12 & magyar & A hivatalos írásművek műfajaiban önállóan alkot szöveget (pl. önéletrajz, motivációs levél). \\ \hline
              11--12 & magyar & Alkalmazza az idézés szabályait és etikai normáit. \\ \hline
              11--12 & magyar & Ismeri a magyar nyelv rendszerét és történetét, önállóan felismeri és alkalmazza a grammatikai, szövegtani, jelentéstani, stilisztikai-retorikai, helyesírási jelenségeket. \\ \hline
              11--12 & magyar & Ismeri a nyelv és a társadalom viszonyát, illetve a nyelvi állandóság és annak változásának folyamatát. Anyanyelvi műveltségének fontos összetevője a tájékozottság a magyar nyelv eredetéről, rokonságáról, történetének főbb korszakairól; a magyar nyelv és a magyar művelődés kapcsolatának tudatosítása. \\ \hline
              11--12 & magyar & Szöveghűen, tudatos, kifejező szövegmondással memoritereket ad elő. \\ \hline
              11--12 & magyar & Felismeri és értelmezi szövegek kapcsolatait és különbségeit (pl. tematikus, motivikus kapcsolatok, utalások, nem irodalmi és irodalmi szövegek, tények és vélemények összevetése), alkalmazza ezeket a képességeket elemző szóbeli és írásbeli műfajokban. \\ \hline
              11--12 & magyar & Alkalmazza a művek műfaji természetének, poétikai jellemzőinek megfelelő szövegfeldolgozási eljárásokat, megközelítési módokat. \\ \hline
              11--12 & magyar & Felismeri a szépirodalmi és nem szépirodalmi szövegekben megjelenített értékeket, erkölcsi kérdéseket, álláspontokat, motivációkat, magatartásformákat, értelmezi és önállóan értékeli ezeket. \\ \hline
              11--12 & magyar & Erkölcsi kérdéseket, döntési helyzeteket megnevez, példával bemutat. Aktív résztvevője az elemző beszélgetéseknek, saját véleményét is belefűzve. Értelmezi a felismert jelenségeket, következtetéseket fogalmaz meg. \\ \hline
              11--12 & magyar & Tájékozott az olvasott, feldolgozott lírai alkotások különböző műfajaiban, poétikai megoldásaiban, kompozíciós eljárásaiban. \\ \hline
              11--12 & magyar & Definíciót nyújt, magyarázatot, értekezést (kisértekezést) készít az olvasmányaival, a felvetett  és tárgyalt problémákkal összefüggésben. Önállóan megfogalmaz kérdéseket, problémákat. \\ \hline
              11--12 & magyar & Irodalmi művekben megjelenő álláspontokat azonosít, megvitat, összehasonlítja más álláspontokkal. Képes ezzel kapcsolatban eltérő vélemények megértésére, újrafogalmazására. \\ \hline
              11--12 & magyar & A magyar és a világirodalom kiemelkedő alkotóiról való tudásáról különféle szempontok alapján írásban számot ad.
 \\ \hline
              11--12 & magyar & Bemutatja a tanult irodalomtörténeti korszakok és stílusirányzatok sajátosságait. \\ \hline
              11--12 & magyar & Ismerteti a feldolgozott epikai, lírai és drámai művek jelentését, erkölcsi tartalmát. \\ \hline
              11--12 & magyar & Műveket, alkotókat mutat be a 20. század magyar és világirodalmából, valamint a kortárs irodalomból. \\ \hline
              11--12 & magyar & Írásban és szóban ismertet alkotói pályaképeket az alkotói életút jelentős tényeinek, a művek tematikai, formabeli változatosságának bemutatásával. \\ \hline
              11--12 & magyar & Ismeri a különböző alkotók hatását az irodalmi hagyományban és felismeri az összefüggéseket egyes művek között. Bemutatja az intertextualitás példáit (evokáció, allúzió, parafrázis, palimpszesz). \\ \hline
              11--12 & magyar & Különböző korokban keletkezett alkotásokat értelmez és összevet tematika és poétikai szempontok alapján. \\ \hline
              11--12 & magyar & Olvasottsága kiterjed az online médiára és a populáris regiszter jegyében született internetes és nyomtatott sajtóra, alkotásokra is. \\ \hline
              11--12 & magyar & Tisztában van azzal, hogy az irodalmi művek a modernségben elsősorban nyelvi képződmények („fikciók”), amelyek grammatikai-poétikai összetettségük minőségétől függően fejtenek ki esztétikai hatást, hoznak létre örömérzést a lélekben. \\ \hline
              11--12 & magyar & Saját, kreatív vagy funkcionális szövegeket hoz létre. \\ \hline
              11--12 & magyar & Jár színházba, moziba, kiállításra, hangversenyre; tudatosan választ a klasszikus és a kortárs drámairodalom elérhető előadásaiból, illetve filmes adaptációiból, egyéb művészeti alkotásaiból. \\ \hline
      \end{longtable}
\end{small}




% end of KULT

