
    \section{Biológia-egészségtan}

       
           \begin{longtable}{c | p{0.8\textwidth} }
            \caption[Biológia-egészségtan 7-8.]{Biológia-egészségtan tantárgy tanulási eredményei féléves bontásokban a 7-8. évfolyamszinteken. }  \\

            \textbf{Félév} & \textbf{Tanulási Eredmény} \\
            \hline
            \endhead
                                
                                          1 &  Érti a család szerepének biológiai és társadalmi jelentőségét. \\ \hline
                                          1 &  Tisztában van saját szervezete működésének alapjaival. \\ \hline
                                          1 &  Tudja, hogy az ember a természet része, és ennek megfelelően cselekszik. \\ \hline
                                          1 &  Tudja, hogy az életmóddal nagymértékben befolyásolhatjuk szervezetünk egészséges működését. Az egészséget testi, lelki szociális jóllétnek tekinti. \\ \hline
                                          1 &  Empátiával viszonyul beteg és fogyatékkal élő társaihoz. \\ \hline
                                          1 &  Csoportmunkában és önállóan beszámolókat készít infokommunikációs eszközök segítségével, egyszerű kísérleteket, vizsgálódásokat végez, adatokat elemez és megoldásokat javasol valós problémákra. Projektmunkát végez tanári irányítással. \\ \hline
                                      
                                
                                          2 &  Tisztában van a környezet-egészségvédelem alapjaival, a gyógy- és fűszernövényeknek a szervezetre gyakorolt hatásával. \\ \hline
                                          2 &  Érti és tudja bizonyítékokkal alátámasztja,, hogy az élővilág különböző megjelenési formáit a különböző élőhelyekhez való alkalmazkodás alakította ki. \\ \hline
                                          2 &  Kerüli az egészséget veszélyeztető anyagok használatát, tevékenységeket. \\ \hline
                                      
                                
                                          3 &  Ismeri Magyarország legfontosabb nemzeti parkjait és a lakóhelyén vagy annak közelében található természeti értékeket (védett növények és védett természeti értékek). \\ \hline
                                          3 &  Összakapcsolja a tanult nem sejtes és sejtes élőlényeket az emberi szervezet működésével, az élőlények és környezetük egymásra hatásaként értelmezi őket. \\ \hline
                                          3 &  Szükség esetén a sérültet alapvető elsősegélynyújtásban részesíti. \\ \hline
                                      
                                
                                          4 &  Tudja, hogy milyen szerepe van a biológiai információnak az önfenntartásban és fajfenntartásban. \\ \hline
                                          4 &  Érti, hogy a párkapcsolatokból adódnak konfliktushelyzetek, és kész azokat megfelelő módszerekkel kezelni. \\ \hline
                                      
                        \end{longtable}
            \clearpage

       
           \begin{longtable}{c | p{0.8\textwidth} }
            \caption[Biológia-egészségtan 9-10.]{Biológia-egészségtan tantárgy tanulási eredményei féléves bontásokban a 9-10. évfolyamszinteken. }  \\

            \textbf{Félév} & \textbf{Tanulási Eredmény} \\
            \hline
            \endhead
                                
                                          2 &  Használja a fénymikroszkóp különböző fajtáit, előkészíti a vizsgálati anyagokat. Az eredményeit dokumentálja. \\ \hline
                                      
                                
                                          3 &  Ismeri a vírusok, baktériumok biológiai egészségügyi jelentőségét, az általuk okozott emberi betegségek megelőzésének lehetőségeit, a védekezés formáit. Ismeri a féregfertőzéseket és azok megelőzési feltételeit, a kullancscsípés megelőzését, a csípés esetleges következményeit. \\ \hline
                                          3 &  Ismeri és példákból felismeri az állatok különböző magatartásformáit. \\ \hline
                                      
                                
                                          4 &  Képes a helyes sorrendbe állítani a biológiai szerveződési szentijeit (mikroba, növény, állat, gomba) a törzsfán. Ok-okozati összefüggéseket felismer az élőlények testfelépítése, életműködése, életmódja között. Ismeri az életmód és a környezet kölcsönhatásait. \\ \hline
                                      
                        \end{longtable}
            \clearpage

       
           \begin{longtable}{c | p{0.8\textwidth} }
            \caption[Biológia-egészségtan 11-12.]{Biológia-egészségtan tantárgy tanulási eredményei féléves bontásokban a 11-12. évfolyamszinteken. }  \\

            \textbf{Félév} & \textbf{Tanulási Eredmény} \\
            \hline
            \endhead
                                
                                          1 &  Megérti a környezet- és természetvédelem alapjait, elsajátítja az ökológiai szemléletet, és nyitottá válik a környezetkímélő gazdasági- és társadalmi stratégiák befogadására. \\ \hline
                                          1 &  A mindennapi életben alkalmazza a megszerzett ismereteket. \\ \hline
                                          1 &  Az ember egészségi állapotára jellemző következtetéseket von le biológiai ismeretei alapján. \\ \hline
                                          1 &  Tudja, hogy az ember szexuális életében alapvetőek a biológiai folyamatok, és ismeri azt a vélekedést, hogy a szerelemre épülő tartós párkapcsolat, az utódok tudatos vállalása, felelősségteljes felnevelése biztosít csak emberhez méltó életet. \\ \hline
                                          1 &  Helyesen értelmezi az evolúciós modellt. A rendszerelvű gondolkodás alapján megérti az emberi és egyéb élő rendszerek minőségi és mennyiségi összefüggéseit. \\ \hline
                                          1 &  Interdiszciplinárisan gondolkodik. \\ \hline
                                      
                                
                                          2 &  Rendszerben látja a hormonális, az idegi és az immunológiai szabályozást, és összekapcsolja a szervrendszerek működését, kémiai, fizikai, műszaki és sejtbiológiai ismeretekkel. Felismeri a biológiai, a technikai és a társadalmi szabályozás analógiáit. \\ \hline
                                          2 &  Felismeri a biológia és a társadalmi gondolkodás közötti kapcsolatot. \\ \hline
                                      
                                
                                          3 &  Megérti az anyag-, az energia- és az információforgalom összefüggéseit az élő rendszerekben. \\ \hline
                                          3 &  Felismeri saját életében a biológiai eredetű problémákat, helyesen választja meg életmódját, felelős egyéni és társadalmi döntéseket hoz megbízható szakmai ismereteii alapján. \\ \hline
                                      
                                
                                          4 &  Felismeri a molekulák és a sejtalkotó részek kooperativitását, összekapcsolja  a kémia, illetve a biológia tantárgyban tanult ismereteket. \\ \hline
                                      
                        \end{longtable}
            \clearpage

        \section{Dráma és tánc}

       
           \begin{longtable}{c | p{0.8\textwidth} }
            \caption[Dráma és tánc 9-10.]{Dráma és tánc tantárgy tanulási eredményei féléves bontásokban a 9-10. évfolyamszinteken. }  \\

            \textbf{Félév} & \textbf{Tanulási Eredmény} \\
            \hline
            \endhead
                                
                                          1 &  Részt vesz szerepjátékokban, csoportos improvizációkban. \\ \hline
                                          1 &  A drámajáték eszközeivel csoportos munkában feldolgozza társaival a látott színházi előadást. \\ \hline
                                      
                                
                                          3 &  Tudatosan és kreatívan alkalmazza a megismert munkaformákat. \\ \hline
                                      
                                
                                          4 &  Kifejezi önmagát; dramatikus - és tánchelyzetekben mások előtt megnyilatkozik és együttműködik társaival. \\ \hline
                                          4 &  Használja a megismert dramaturgiai fogalomkészletet. \\ \hline
                                      
                        \end{longtable}
            \clearpage

        \section{Ének-zene}

       
           \begin{longtable}{c | p{0.8\textwidth} }
            \caption[Ének-zene 1-2.]{Ének-zene tantárgy tanulási eredményei féléves bontásokban a 1-2. évfolyamszinteken. }  \\

            \textbf{Félév} & \textbf{Tanulási Eredmény} \\
            \hline
            \endhead
                                
                                          1 &  Kreatívan részt vesz a generatív játékokban és feladatokban. Érzi az egyenletes lüktetést, tartja a tempót, érzi a tempóváltozást. A 2/4-es metrumot helyesen hangsúlyozza. \\ \hline
                                          1 &  Társaival együtt figyelmesen hallgat zenét. Megfigyelés útján tapasztalathoz jut, melyet az egyszerű zenei elemzés alapjaként használ. Különbséget tesz az eltérő zenei karakterek között. \\ \hline
                                      
                                
                                          3 &  Pontosan, folyamatosan szólaltatja meg a tanult ritmikai elemeket tartalmazó ritmusgyakorlatokat csoportosan és egyénileg is. \\ \hline
                                      
                                
                                          4 &  Hangszínhallása fejlődik. Megkülönbözteti a furulya, citera, zongora, hegedű, fuvola, fagott, gitár, dob, triangulum, réztányér, a testhangszerek és a gyermek-, női, férfihang hangszínét. Ismeri a hangszerek alapvető jellegzetességeit. \\ \hline
                                          4 &  Képes 60 gyermekdalt és népdalt emlékezetből, a kapcsolódó játékokkal, c’–d” hangterjedelemben előadni. \\ \hline
                                          4 &  Ismeri és lejegyzi a tanult zenei elemeket. (ritmus, dallam) \\ \hline
                                          4 &  Kifejezően énekel, törekszik az egységes hangzásra és a tiszta intonációra, új dalokat hallás után megtanul. \\ \hline
                                          4 &  Szolmizálva énekel a tanult dalok stílusában megszerkesztett rövid dallamfordulatokat kézjelről, betűkottáról és hangjegyről. Megfelelő előkészítés után hasonló dallamfordulatokat rögtönöz. \\ \hline
                                      
                        \end{longtable}
            \clearpage

       
           \begin{longtable}{c | p{0.8\textwidth} }
            \caption[Ének-zene 3-4.]{Ének-zene tantárgy tanulási eredményei féléves bontásokban a 3-4. évfolyamszinteken. }  \\

            \textbf{Félév} & \textbf{Tanulási Eredmény} \\
            \hline
            \endhead
                                
                                          1 &  Fejlődik zenei memóriája és belső hallása. \\ \hline
                                      
                                
                                          2 &  Kreatívan részt vesz a generatív játékokban és feladatokban. Érzi az egyenletes lüktetést, tartja a tempót, érzi a tempóváltozást. A 3/4-es és 4/4-es metrumot helyesen hangsúlyozza. \\ \hline
                                      
                                
                                          3 &  Többszólamú éneklési készsége fejlődik. Képes csoportosan ismert dalokat ritmus osztinátóval énekelni, és egyszerű kétszólamú darabokat, kánonokat megszólaltatni. \\ \hline
                                          3 &  Formaérzéke fejlődik, a formai építkezés jelenségeit (azonosság, hasonlóság, különbözőség) ismeri és meg tudja fogalmazni. \\ \hline
                                      
                                
                                          4 &  60 népdalt, műzenei idézetet emlékezetből, a-e’ hangterjedelemben csoportosan, a népdalokat több versszakkal, csokorba rendezve is előadja. \\ \hline
                                          4 &  Képes kifejezően, egységes hangzással, tiszta intonációval énekelni, és új dalokat megfelelő előkészítést követően, hallás után és jelrendszerről megtanulni. \\ \hline
                                          4 &  Felismeri, lejegyzi és megszólaltatja a tanult zenei elemeket (metrum, dinamikai jelzések, ritmus, dallam). \\ \hline
                                          4 &  Az ismert dalokat kézjelről, betűkottáról, hangjegyről és emlékezetből szolmizálják. \\ \hline
                                          4 &  Szolmizálva énekel a tanult dalok stílusában megszerkesztett rövid dallamfordulatokat kézjelről, betűkottáról és hangjegyről. Megfelelő előkészítés után hasonló dallamfordulatokat rögtönöz. \\ \hline
                                          4 &  Fejlődik hangszínhallása. Megkülönbözteti a tekerő, duda, oboa, klarinét, kürt, trombita, üstdob, a gyermekkar, nőikar, férfikar, vegyeskar hangszínét. Ismeri a hangszerek alapvető jellegzetességeit. Különbséget tesz szóló és kórus, szólóhangszer és zenekar hangzása között. \\ \hline
                                          4 &  Tudatosan és figyelmesen hallgat zenét. A kiválasztott művek közül 4-5 alkotást műrészletet ismernek. \\ \hline
                                      
                        \end{longtable}
            \clearpage

       
           \begin{longtable}{c | p{0.8\textwidth} }
            \caption[Ének-zene 5-6.]{Ének-zene tantárgy tanulási eredményei féléves bontásokban a 5-6. évfolyamszinteken. }  \\

            \textbf{Félév} & \textbf{Tanulási Eredmény} \\
            \hline
            \endhead
                                
                                          1 &  Fejlődik zenei memóriája és belső hallása. \\ \hline
                                          1 &  Különbséget tesz a népi zenekar, vonósnégyes, szimfonikus zenekar hangzása között. \\ \hline
                                      
                                
                                          2 &  Kreatívan részt vesz a generatív játékokban és feladatokban. Érzi az egyenletes lüktetést, tartja a tempót, érzékeli a tempóváltozást. A 6/8-os és 3/8-os metrumot helyesen hangsúlyozza. \\ \hline
                                          2 &  Felismeri és megfogalmazza a formai építkezés jelenségeit, fejleszti formaérzékét. \\ \hline
                                          2 &  Fejlődik hangszínhallása. Megkülönbözteti a tárogató, brácsa, cselló, nagybőgő, harsona, tuba, hárfa hangszínét. Ismeri a hangszerek alapvető jellegzetességeit. \\ \hline
                                      
                                
                                          3 &  Kifejezően, egységes hangzással, tiszta intonációval énekel. Új dalokat megfelelő előkészítést követően hallás után megtanul. \\ \hline
                                          3 &  Fejlődik a többszólamú éneklési készsége. Kánonban énekel társaival. \\ \hline
                                          3 &  Felismeri és megszólaltatja a tanult zenei elemeket (metrum, dinamikai jelzések, ritmus, dallam). \\ \hline
                                      
                                
                                          4 &  20-22 népdalt, történeti éneket több versszakkal, valamint 8-10 műzenei idézetet emlékezetből elénekel. \\ \hline
                                          4 &  Szolmizálva énekel a tanult dalok stílusában megszerkesztett rövid dallamfordulatokat kézjelről, betűkottáról és hangjegyről. Megfelelő előkészítés után hasonló dallamfordulatokat rögtönöz. \\ \hline
                                          4 &  A két zenei korszakból zenehallgatásra kiválasztott művek közül 18-20 alkotást/műrészletet ismer. \\ \hline
                                      
                        \end{longtable}
            \clearpage

       
           \begin{longtable}{c | p{0.8\textwidth} }
            \caption[Ének-zene 7-8.]{Ének-zene tantárgy tanulási eredményei féléves bontásokban a 7-8. évfolyamszinteken. }  \\

            \textbf{Félév} & \textbf{Tanulási Eredmény} \\
            \hline
            \endhead
                                
                                          2 &  Kifejezően, egységes hangzással, tiszta intonációval énekel. Új dalokat megfelelő előkészítést követően hallás után megtanul. \\ \hline
                                      
                                
                                          3 &  Többszólamú éneklésben részt vesz. Csoportosan egyszerű orgánumokat megszólaltat. \\ \hline
                                          3 &  Kreatívan  részt vesz a generatív játékokban és feladatokban. Érzi az egyenletes lüktetést, tartja a tempót, érzékeli a tempóváltozást. A 5/8-os és 7/8-os és 8/8-os metrumot helyesen hangsúlyozza. \\ \hline
                                          3 &  Felismeri és megszólaltatja a tanult zenei elemeket (metrum, dinamikai jelzések, ritmus, dallam). \\ \hline
                                          3 &  Fejlődik zenei memóriája és belső hallása. \\ \hline
                                          3 &  Felismeri és megfogalmazza a formai építkezés jelenségeit, fejleszti formaérzékét. \\ \hline
                                          3 &  Fejlődik hangszínhallása. Megkülönbözteti a cimbalom, lant, csembaló, orgona, szaxofon hangszínét. Ismeri a hangszerek alapvető jellegzetességeit. \\ \hline
                                      
                                
                                          4 &  Előad 14-17 népdalt, balladát, históriás éneket több versszakkal, valamint 8-10 műzenei idézetet emlékezetből, g–f” hangterjedelemben. \\ \hline
                                          4 &  A zenei korszakokból kiválasztott zeneművek közül 20-25 alkotást/műrészletet ismer. \\ \hline
                                      
                        \end{longtable}
            \clearpage

       
           \begin{longtable}{c | p{0.8\textwidth} }
            \caption[Ének-zene 9-10.]{Ének-zene tantárgy tanulási eredményei féléves bontásokban a 9-10. évfolyamszinteken. }  \\

            \textbf{Félév} & \textbf{Tanulási Eredmény} \\
            \hline
            \endhead
                                
                                          1 &  Ismeri a hangszerek alapvető jellegzetességeit. \\ \hline
                                          1 &  A zenei műalkotások megismerése révén tájékozódik korunk kulturális sokszínűségében. \\ \hline
                                      
                                
                                          2 &  Felismeri és megfogalmazza a formai építkezés jelenségeit, fejleszti formaérzékét. \\ \hline
                                      
                                
                                          3 &  Kifejezően, egységes hangzással, tiszta intonációval énekel. Új dalokat megfelelő előkészítést követően hallás után megtanul. \\ \hline
                                          3 &  A generatív tevékenységek eredményeként érzékeli, felismeri a zenei kifejezések, a formák, a műfajok, és a zenei eszközök közti összefüggéseket. \\ \hline
                                      
                                
                                          4 &  8-10 népzenei, valamint 8-10 műzenei idézetet részben kottából, részben emlékezetből csoportosan előad. \\ \hline
                                          4 &  Egyszerű két- és háromszólamú kórusműveket vagy azok részleteit, kánonokat énekel. \\ \hline
                                          4 &  Megismeri és értelmezi a kottakép elemeit és az alapvető zenei kifejezéseket. \\ \hline
                                          4 &  A zenei korszakokból kiválasztott zeneművek közül 20-25 alkotást/műrészletet ismer és felismer. \\ \hline
                                      
                        \end{longtable}
            \clearpage

        \section{Erkölcstan}

       
           \begin{longtable}{c | p{0.8\textwidth} }
            \caption[Erkölcstan 1-2.]{Erkölcstan tantárgy tanulási eredményei féléves bontásokban a 1-2. évfolyamszinteken. }  \\

            \textbf{Félév} & \textbf{Tanulási Eredmény} \\
            \hline
            \endhead
                                
                                          1 &  Életkorának megfelelő reális képpel rendelkezik saját külső tulajdonságairól, tisztában van legfontosabb személyi adataival. \\ \hline
                                          1 &  Kifejezi a szeretetkapcsolatok fontosságát. \\ \hline
                                          1 &  Megkülönbözteti egymástól a valóságos és a virtuális világot. \\ \hline
                                      
                                
                                          2 &  A beszélgetés során betartja az udvarias társalgás elemi szabályait. \\ \hline
                                          2 &  Érzelmeit a közösség számára elfogadható formában nyilvánítja ki. \\ \hline
                                      
                                
                                          3 &  Átlátja társas viszonyainak alapvető szerkezetét. \\ \hline
                                          3 &  Kapcsolatba lép és a partner személyét figyelembe véve kommunikál környezete tagjaival, különféle beszédmódokat alkalmaz. \\ \hline
                                      
                                
                                          4 &  Érzelmileg kötődik környezetéhez és a körülötte élő emberekhez. \\ \hline
                                          4 &  Értelmezi a hagyományok társadalomban és a közösségekben betöltött szerepét. \\ \hline
                                          4 &  Átéli a természet szépségét, és érti, hogy felelősek vagyunk a körülöttünk lévő élővilágért. \\ \hline
                                          4 &  Kifejezi érzéseit, gondolatait és fantáziaképeit vizuális, mozgásos vagy szóbeli eszközökkel. \\ \hline
                                          4 &  Tisztában van vele és érzelmileg felfogja azt a tényt, hogy más gyerekek sokszor egészen más körülmények között élnek, mint ő. \\ \hline
                                      
                        \end{longtable}
            \clearpage

       
           \begin{longtable}{c | p{0.8\textwidth} }
            \caption[Erkölcstan 3-4.]{Erkölcstan tantárgy tanulási eredményei féléves bontásokban a 3-4. évfolyamszinteken. }  \\

            \textbf{Félév} & \textbf{Tanulási Eredmény} \\
            \hline
            \endhead
                                
                                          1 &  Életkorának megfelelő szinten reális képe van saját külső és belső tulajdonságairól, és késztetést érez arra, hogy fejlessze önmagát. \\ \hline
                                          1 &  Érti, hogy mi a jelentősége a szabályoknak a közösségek életében; betartja a megértett szabályokat; részt vesz a szabályok kialakításában. \\ \hline
                                          1 &  A körülötte zajló eseményekre és a különféle helyzetekre a sajátjától eltérő nézőpontból is rátekint. \\ \hline
                                      
                                
                                          2 &  Figyel másokra, szavakkal is kifejezi érzéseit és gondolatait, bekapcsolódik csoportos beszélgetésekbe. \\ \hline
                                      
                                
                                          3 &  Érti, hogy a Föld mindannyiunk közös otthona és hogy bolygónk számos olyan értékkel rendelkezik, amit elődeink hoztak létre. \\ \hline
                                      
                                
                                          4 &  Tartós kapcsolatot alakít ki társaival, törekszik e kapcsolatok ápolására, és ismer olyan eljárásokat, amelyek segítségével a kapcsolati konfliktusok konstruktív módon feloldhatók. \\ \hline
                                          4 &  Kifejezi kötődését a saját és mások kultúrájához. \\ \hline
                                          4 &  Érti és elfogadja, hogy az emberek sokfélék és sokfélék a szokásaik, a hagyományaik is; tiszteletben tartja ezt a tényt és kíváncsi a sajátjától eltérő kulturális jelenségekre. \\ \hline
                                          4 &  Jelenségeket, eseményeket és helyzeteket erkölcsi nézőpontból értékel. \\ \hline
                                      
                        \end{longtable}
            \clearpage

       
           \begin{longtable}{c | p{0.8\textwidth} }
            \caption[Erkölcstan 5-6.]{Erkölcstan tantárgy tanulási eredményei féléves bontásokban a 5-6. évfolyamszinteken. }  \\

            \textbf{Félév} & \textbf{Tanulási Eredmény} \\
            \hline
            \endhead
                                
                                          1 &  Tisztában van az egészség megőrzésének jelentőségével, és tudja, hogy maga is felelős ezért. \\ \hline
                                      
                                
                                          2 &  Képes különféle szintű kapcsolatok kialakítására és ápolására; átlátja saját kapcsolati hálójának a szerkezetét; rendelkezik a konfliktusok kezelésének és az elkövetett hibák kijavításának néhány, a gyakorlatban jól használható technikájával. \\ \hline
                                          2 &  Fontos számára a közösséghez való tartozás érzése; képes átlátni és elfogadni a közösségi normákat. \\ \hline
                                          2 &  Érti, hogy a világ megismerésének többféle útja van (különböző világképek és világnézetek), s ezek mindegyike a maga sajátos eszközeivel közelít ugyanahhoz a valósághoz. \\ \hline
                                      
                                
                                          4 &  Tudatában van annak, hogy az emberek sokfélék, elfogadja és értékeli a testi és lelki vonásokban megnyilvánuló sokszínűséget, valamint az etnikai és kulturális különbségeket. \\ \hline
                                          4 &  Gondolkodik saját személyiségjegyein, törekszik a megalapozott véleményalkotásra, illetve vélekedéseinek és tetteinek utólagos értékelésére. \\ \hline
                                          4 &  Gondolkodik rajta, hogy mit tekint értéknek; tudja, hogy ez befolyásolja a döntéseit, és hogy időnként választania kell még a számára fontos értékek között is. \\ \hline
                                          4 &  Nyitottan fogadja a sajátjától eltérő véleményeket, szokásokat és kulturális, illetve vallási hagyományokat. \\ \hline
                                          4 &  Érzékeli, hogy a társadalom tagjai különféle körülmények között élnek, képes együttérzést mutatni az elesettek iránt, és lehetőségéhez mérten szerepet vállal a rászorulók segítésében. Megbecsüli a neki nyújtott segítséget. \\ \hline
                                          4 &  Tisztában van azzal, hogy az emberi tevékenység hatással van a környezet állapotára, és törekszik rá, hogy életvitelével minél kevésbé károsítsa a természetet. \\ \hline
                                          4 &  Ismeri a modern technika legfontosabb előnyeit és hátrányait, s felismeri magán a függőség kialakulásának esetleges előjeleit. \\ \hline
                                          4 &  Tisztában van vele, hogy a reklámok a nézők befolyásolására törekszenek, kritikusan viszonyul a különféle médiaüzenetekhez. \\ \hline
                                      
                        \end{longtable}
            \clearpage

       
           \begin{longtable}{c | p{0.8\textwidth} }
            \caption[Erkölcstan 7-8.]{Erkölcstan tantárgy tanulási eredményei féléves bontásokban a 7-8. évfolyamszinteken. }  \\

            \textbf{Félév} & \textbf{Tanulási Eredmény} \\
            \hline
            \endhead
                                
                                          1 &  Életkorának megfelelő szinten ismeri önmagát, hosszabb távú elképzeléseinek kialakításakor képes reálisan felmérni a lehetőségeit. \\ \hline
                                          1 &  Tisztában van vele, hogy baráti- és párkapcsolataiban felelősséggel tartozik a társaiért. \\ \hline
                                          1 &  Képes megfogalmazni, hogy mi okoz neki örömet, illetve rossz érzést. \\ \hline
                                      
                                
                                          2 &  Reflektál saját maga és mások gondolataira, motívumaira és tetteire. \\ \hline
                                          2 &  Érti a szabályok szerepét az emberi együttélésben, és belátás alapján igyekszik alkalmazkodni hozzájuk; igényli azonban, hogy maga is alakítója lehessen a közösségi szabályoknak. \\ \hline
                                      
                                
                                          3 &  Van elképzelése saját jövőjéről, és tisztában van vele, hogy céljai eléréséért erőfeszítéseket kell tennie. Képes megfogalmazni egy trimeszterre tanulási célokat és azokat végig is viszi. \\ \hline
                                          3 &  Tisztában van a függőséget okozó szokások súlyos következményeivel. \\ \hline
                                      
                                
                                          4 &  Érti, hogy az ember egyszerre biológiai és tudatos lény, akit veleszületett képességei alkalmassá tesznek a tanulásra, mások megértésre és önmaga vizsgálatára. \\ \hline
                                          4 &  Érti, hogy az emberek viselkedését, döntéseit tudásuk, gondolataik, érzelmeik, vágyaik, nézeteik és értékrendjük egyaránt befolyásolják. \\ \hline
                                          4 &  Képes erkölcsi szempontok szerint mérlegelni különféle cselekedeteket, és el tudja viselni az értékek közötti választással együtt járó belső feszültséget. \\ \hline
                                          4 &  Képes ellenállni a csoportnyomásnak, és saját értékrendje szerinti autonóm döntéseket hozni. \\ \hline
                                          4 &  Van saját identitás élménye, amely nemzethez, európaisághoz, kisebbségi léthez, vagy lokalitáshoz is köthető. \\ \hline
                                          4 &  Nyitott más kultúrák értékeinek megismerésére és befogadására. \\ \hline
                                          4 &  Tudja, hogy ugyanazt a dolgot különböző emberek eltérő módon ítélhetik meg, ami konfliktusok forrása lehet. \\ \hline
                                      
                        \end{longtable}
            \clearpage

        \section{Etika}

       
           \begin{longtable}{c | p{0.8\textwidth} }
            \caption[Etika 11-12.]{Etika tantárgy tanulási eredményei féléves bontásokban a 11-12. évfolyamszinteken. }  \\

            \textbf{Félév} & \textbf{Tanulási Eredmény} \\
            \hline
            \endhead
                                
                                          1 &  Ismeri az erkölcsi hagyomány legfontosabb elemeit, és e tudás birtokában felismeri és kezelik a mindennapi életben felmerülő erkölcsi problémákat. \\ \hline
                                          1 &  Elfogadják, megértik és tisztelik a magukétől eltérő nézeteket. \\ \hline
                                          1 &  Ismerik azokat az értékelveket, magatartásszabályokat és beállítódásokat, amelyeknek a közmegegyezés kitüntetett erkölcsi jelentőséget tulajdonít. \\ \hline
                                      
                                
                                          2 &  Értékítéleteit ésszerű érvekkel alátámasztja, felelős mérlegelésen alapuló döntést hoz. Részt vesz az etikai és közéleti vitákban, saját álláspontját megvédi, illetve továbbfejleszti. \\ \hline
                                      
                        \end{longtable}
            \clearpage

        \section{Fizika}

       
           \begin{longtable}{c | p{0.8\textwidth} }
            \caption[Fizika 7-8.]{Fizika tantárgy tanulási eredményei féléves bontásokban a 7-8. évfolyamszinteken. }  \\

            \textbf{Félév} & \textbf{Tanulási Eredmény} \\
            \hline
            \endhead
                                
                                          1 &  A számítógépet adatrögzítésre, információgyűjtésre használja. \\ \hline
                                      
                                
                                          2 &  Felismeri, hogy a természettudományos tények megismételhető megfigyelésekből, célszerűen tervezett kísérletekből nyert bizonyítékokon alapulnak. \\ \hline
                                          2 &  Legalább egy tudományos elmélet esetén végigkövette, hogy a társadalmi és történelmi háttér hogyan befolyásolta annak kialakulását és fejlődését. \\ \hline
                                          2 &  Felhasználja ismereteit saját egészségének védelmére. \\ \hline
                                          2 &  A sebesség fogalmát különböző kontextusokban is alkalmazza. \\ \hline
                                          2 &  A testek közötti kölcsönhatás során a sebességük és a tömegük egyaránt fontos, és ezt konkrét példákkal bemutatja \\ \hline
                                          2 &  Érti, hogy a gravitációs erő egy adott testre hat és a Föld (vagy más (égi)test) vonzása okozza. \\ \hline
                                          2 &  Felsorol többféle energiaforrást, ismeri alkalmazásuk környezeti hatásait. Figyel a környezettudatosságra, energiatakarékosságra. \\ \hline
                                      
                                
                                          3 &  A mások által kifejtett véleményeket megérti, értékeli, azokkal szemben kulturáltan vitatkozik. \\ \hline
                                          3 &  Egyszerű megfigyelési és mérési folyamatokat tervez, tudományos ismeretek megszerzéséhez célzott kísérleteket végez. \\ \hline
                                          3 &  Ábrák, adatsorok elemzéséből tanári irányítás alapján egyszerűbb összefüggéseket ismer fel. Megfigyelései során modelleket használ.  \\ \hline
                                          3 &  Egyszerű arányossági kapcsolatokat matematikai és grafikus formában is lejegyez. Az eredmények elemzése után konklúziókat von le. \\ \hline
                                          3 &  Ismeri a fényjelenségeken alapuló kutatóeszközöket, a fény alapvető tulajdonságait. \\ \hline
                                          3 &  Elemzi az energiaátalakulásokat, kapcsolatukat a hőmennyiséghez. Használja az energiafajták elnevezését, felismeri a hőmennyiség cseréjének és a hőmérséklet kiegyenlítésének kapcsolatát. \\ \hline
                                      
                                
                                          4 &  Pontos, a szakszerű fogalmakat tudatosan alkalmazó, ábrákkal, irodalmi hivatkozásokkal stb. alátámasztott prezentációt tart eredményeiről. \\ \hline
                                          4 &  Önállóan ismeretet szerez. \\ \hline
                                          4 &  A kísérletek elemzése során kritikusan és egészséges szkepticizmussal szemléli a történéseket. Tudja, hogy ismeretei és használati készségei meglévő szintjén további tanulással túl tud lépni. \\ \hline
                                          4 &  Megítéli, hogy különböző esetekben milyen módon alkalmazható a tudomány és a technika, értékeli azok előnyeit és hátrányait az egyén, a közösség és a környezet szempontjából. Törekszik a természet- és környezetvédelmi problémák enyhítésére. \\ \hline
                                          4 &  Ismeri és azonosítja az energiaátalakítási lehetőségeket, érti a megújuló és a nem megújuló energiafajták közötti különbséget. \\ \hline
                                          4 &  Elemzi az egyes energiaátalakítási lehetőségek előnyeit, hátrányait, alkalmazásuk kockázatait. Tényeket és adatokat gyűjt, vita során csoportosítja és felhasználja az érveket és ellenérveket. \\ \hline
                                          4 &  Érti a nyomás fogalmát és egyszerű esetekben kiszámítja az erő és a felület hányadosaként. \\ \hline
                                          4 &  Tudja, hogy nemcsak a szilárd testek fejtenek ki nyomást. \\ \hline
                                          4 &  Megmagyarázza a gázok nyomását a részecskeképpel. \\ \hline
                                          4 &  Tudja, hogy az áramlások oka a nyomáskülönbség. \\ \hline
                                          4 &  Ismerettel rendelkezik arról, hogy a hang miként keletkezik, és hogy a részecskék sűrűségének változásával terjed a közegben. \\ \hline
                                          4 &  Tudja, hogy a hang terjedési sebessége gázokban a legkisebb és szilárd anyagokban a legnagyobb. \\ \hline
                                          4 &  Ismeri az áramkör részeit, összeállít egyszerű áramköröket, méri az áramerősséget. \\ \hline
                                          4 &  Tudja, hogy az áramforrások kvantitatív jellemzője a feszültség. \\ \hline
                                          4 &  Ismeri az elektromos fogyasztó működését. \\ \hline
                                          4 &  Elmagyarázza az erőművek alapvető szerkezetét. \\ \hline
                                          4 &  Tudja, hogy az elektromos energia bármilyen módon történő előállítása terheli a környezetet. \\ \hline
                                      
                        \end{longtable}
            \clearpage

       
           \begin{longtable}{c | p{0.8\textwidth} }
            \caption[Fizika 9-10.]{Fizika tantárgy tanulási eredményei féléves bontásokban a 9-10. évfolyamszinteken. }  \\

            \textbf{Félév} & \textbf{Tanulási Eredmény} \\
            \hline
            \endhead
                                
                                          1 &  Fejlődik a megfigyelő, rendszerező készsége és a kísérletezési, mérési kompetenciája. \\ \hline
                                      
                                
                                          2 &  Ismeri a mozgástani alapfogalmakat, grafikusan megold feladatokat. Érti a newtoni mechanika lényegét: az erő nem a mozgás fenntartásához, hanem a mozgásállapot megváltoztatásához szükséges. \\ \hline
                                          2 &  Az elektrosztatika alapjelenségeit és fogalmait ismeri. Az elektromos és a mágneses mezőt fizikai objektumként elfogadja. \\ \hline
                                          2 &  Megold egyszerű feladatokat az áramokkal kapcsolatos ismeretei alapján és ismereteit a gyakorlatban is használja. \\ \hline
                                          2 &  Fejlődik az energiatudatossága. \\ \hline
                                      
                                
                                          3 &  Egyszerű kinematikai és dinamikai feladatokat megold. \\ \hline
                                          3 &  Hőtani alapfogalmak, a hőtan főtételei, hőerőgépek. Annak ismerete, hogy gépeink működtetése, az élő szervezetek működése csak energia befektetése árán valósítható meg, a befektetett energia jelentős része elvész, a működésben nem hasznosul, „örökmozgó” létezése elvileg kizárt. \\ \hline
                                      
                                
                                          4 &  A kinematikát és dinamikát a mindennapokban alkalmazza. \\ \hline
                                          4 &  Folyadékok és gázok sztatikájának és áramlásának alapjelenségeit felismeri a gyakorlati életben. \\ \hline
                                          4 &  Ismeri a gázok makroszkopikus állapotjelzőit és összefüggéseit, az ideális gáz golyómodelljét, a nyomás és a hőmérséklet kinetikus golyómodelljét. \\ \hline
                                      
                        \end{longtable}
            \clearpage

       
           \begin{longtable}{c | p{0.8\textwidth} }
            \caption[Fizika 11-12.]{Fizika tantárgy tanulási eredményei féléves bontásokban a 11-12. évfolyamszinteken. }  \\

            \textbf{Félév} & \textbf{Tanulási Eredmény} \\
            \hline
            \endhead
                                
                                          1 &  A csillagászati alapismeretek felhasználásával elhelyezi a Földet az Univerzumban; képe van az ismeri az Univerzum térbeli, időbeli méreteiről. \\ \hline
                                      
                                
                                          2 &  Bővül a mechanikai fogalomtára a rezgések és hullámok témakörével, valamint a forgómozgás és a síkmozgás gyakorlatban is fontos ismereteivel. \\ \hline
                                          2 &  Érti a csillagászat és az űrkutatás fontosságát. \\ \hline
                                          2 &  Önállóan ismeretet szerez a STEM világában, forrást keres, szelektál és feldolgoz. \\ \hline
                                      
                                
                                          3 &  Az elektromágneses indukcióra épülő mindennapi alkalmazások fizikai alapját ismeri: elektromos energiahálózat, elektromágneses hullámok. \\ \hline
                                          3 &  Ismeri a kondenzált anyagok szerkezeti és fizikai tulajdonságainak alapvető összefüggéseit. \\ \hline
                                      
                                
                                          4 &  A (hétköznapi) optikai jelenségeket értelmezi hármas modellezéssel (geometriai optika, hullámoptika, fotonoptika). \\ \hline
                                          4 &  Bemutatja a modellalkotás jellemzőit az atommodell fejlődésén keresztül. \\ \hline
                                          4 &  Értelmezi a magfizika elméleti ismeretei alapján a korszerű nukleáris technikai alkalmazásokat. Ismeri a kockázatokat és azokat reálisan értékeli. \\ \hline
                                      
                        \end{longtable}
            \clearpage

        \section{Földrajz}

       
           \begin{longtable}{c | p{0.8\textwidth} }
            \caption[Földrajz 7-8.]{Földrajz tantárgy tanulási eredményei féléves bontásokban a 7-8. évfolyamszinteken. }  \\

            \textbf{Félév} & \textbf{Tanulási Eredmény} \\
            \hline
            \endhead
                                
                                          2 &  A térképet információforrásként használja. A topográfiai ismereteikhez földrajzi-környezeti tartalmakat kapcsol. Topográfiai tudása alapján biztonsággal tájékozódik a köznapi életben a földrajzi térben, illetve a térképeken, és alkalmazza topográfiai tudását más tantárgyak tanulása során is. \\ \hline
                                      
                                
                                          3 &  Példákkal bizonyítja a társadalmi-gazdasági folyamatok környezetkárosító hatását, a lokális problémák globális következmények elvének érvényesülését. Tisztában van a Földet fenyegető veszélyekkel, érti a fenntarthatóság lényegét példák alapján. Felismeri, hogy a Föld sorsa a saját magatartásunkon is múlik. \\ \hline
                                      
                                
                                          4 &  Valós képzetekkel rendelkezik a környezeti elemek méreteiről, a számszerűen kifejezhető adatok és az időbeli változások nagyságrendjéről. Képes nagy vonalakban tájékozódni a földtörténeti időben. Természet-, illetve társadalom- és gazdaságföldrajzi megfigyeléseket végez a különböző nyomtatott és elektronikus információhordozókból földrajzi tartalmú információk gyűjtésére, összegzésére, a lényeges elemek kiemelésére. Ezek során alkalmazza digitális ismereteit. Megadott szempontok alapján bemutatja a földrajzi öveket, földrészeket, országokat és tipikus tájakat. \\ \hline
                                          4 &  Társaikkal együttműködik. Későbbi élete folyamán önállóan tovább gyarapítja földrajzi ismereteit. \\ \hline
                                      
                        \end{longtable}
            \clearpage

       
           \begin{longtable}{c | p{0.8\textwidth} }
            \caption[Földrajz 9-10.]{Földrajz tantárgy tanulási eredményei féléves bontásokban a 9-10. évfolyamszinteken. }  \\

            \textbf{Félév} & \textbf{Tanulási Eredmény} \\
            \hline
            \endhead
                                
                                          1 &  Szintetizálja a különböző szempontból elsajátított földrajzi (általános és leíró természet-, illetve társadalom-, valamint gazdaságföldrajzi) ismereteket. Átlátja a környezeti elemek méreteit, a számszerűen kifejezhető adatok és az időbeli változások nagyságrendjeit. \\ \hline
                                          1 &  Használja a térképet információforrásként, értelmezi a leolvasott adatokat. Felismeri a Világegyetem és a Naprendszer felépítésében, a bolygók mozgásában megnyilvánuló törvényszerűségeket. \\ \hline
                                          1 &  Érvel a fenntarthatóságot szem előtt tartó gazdaság, illetve gazdálkodás fontossága mellett. \\ \hline
                                          1 &  Az egyén szerepét és lehetőségeit a környezeti problémák mérséklésében ismeri, és konkrét példákat nevez meg. \\ \hline
                                          1 &  Képes természet-, illetve társadalom- és gazdaságföldrajzi megfigyelésekre, a tapasztalatok lejegyzésére, értelmezésére. \\ \hline
                                          1 &  Nyomtatott és elektronikus információhordozókból földrajzi tartalmú információkat gyűjt és azokat feldolgozza. Az információkat összegzi, a lényeges elemeket kiemeli. Ennek során alkalmazza digitális ismereteit. \\ \hline
                                          1 &  Véleményét a földrajzi gondolkodásnak megfelelően megfogalmazza és logikusan érvel. \\ \hline
                                          1 &  Alkalmazza ismereteit földrajzi tartalmú problémák megoldása során a mindennapi életben. \\ \hline
                                          1 &  Földrajzi ismereteit felhasználja valódi döntéshelyzetekben. \\ \hline
                                      
                                
                                          2 &  Ismeri a globalizáció gazdasági és társadalmi hatását, érti az ellentmondásokat. \\ \hline
                                          2 &  Társaival együttműködik a földrajzi-környezeti tartalmú feladatok megoldásakor. \\ \hline
                                          2 &  Ismeretei alapján biztonsággal tájékozódik a földrajzi térben, illetve az azt megjelenítő különböző térképeken. Ismeri a tananyagban meghatározott topográfiai fogalmakhoz kapcsolódó tartalmakat. \\ \hline
                                      
                                
                                          3 &  Képes elhelyezni az országokat, országcsoportokat és integrációkat a világ társadalmi-gazdasági folyamataiban, értelmezi a világgazdaságban betöltött szerepüket. \\ \hline
                                          3 &  Képes értelmezni az egyes térségek, országok eltérő társadalmi-gazdasági adottságait és az adottságok jelentőségének változását az idő függvényében. \\ \hline
                                          3 &  Ismeri a monetáris világ jellemző folyamatait, azok társadalmi-gazdasági hatásait. \\ \hline
                                          3 &  Ismeri Magyarország társadalmi-gazdasági fejlődésének jellemzőit, a gazdasági fejlettség területi különbségeit és okait. \\ \hline
                                          3 &  Elhelyezi Magyarországot a világgazdaság folyamataiban. \\ \hline
                                          3 &  Tudja példákkal bizonyítani a társadalmi-gazdasági folyamatok környezetkárosító hatását, a lokális problémák globális következmények elvének érvényesülését. Ismeri az egész Földünket érintő globális társadalmi és gazdasági problémákat. \\ \hline
                                      
                                
                                          4 &  Tájékozódik a földtörténeti korokban; ismeri a kontinenseket felépítő nagyszerkezeti egységek kialakulásának időbeli rendjét, földrajzi elhelyezkedését. \\ \hline
                                          4 &  Képes megadott szempontok alapján bemutatni az egyes geoszférák sajátosságait, jellemző folyamatait és azok összefüggéseit. Belátja, hogy az egyes geoszférákat ért környezeti károk hatása más szférákra is kiterjedhet. \\ \hline
                                          4 &  Képes a földrajzi övezetesség kialakulásában megnyilvánuló összefüggések és törvényszerűségek értelmezésére. \\ \hline
                                          4 &  Képes alapvető összefüggések és törvényszerűségek felismerésére és megfogalmazására az egész Földre jellemző társadalmi-gazdasági folyamatokkal kapcsolatosan. \\ \hline
                                          4 &  Példákkal alátámasztja az Európai Unio egészére kiterjedő illetve a környező országokkal kialakult regionális együttműködések szerepét. \\ \hline
                                          4 &  Kialakul benne az igény arra, hogy későbbi élete folyamán önállóan gyarapítsa tovább földrajzi ismereteit. \\ \hline
                                          4 &  Topográfiai tudását alkalmazza más tantárgyak tanulása során, illetve a mindennapi életben. \\ \hline
                                      
                        \end{longtable}
            \clearpage

        \section{Hon- és népismeret}

       
           \begin{longtable}{c | p{0.8\textwidth} }
            \caption[Hon- és népismeret 5-6.]{Hon- és népismeret tantárgy tanulási eredményei féléves bontásokban a 5-6. évfolyamszinteken. }  \\

            \textbf{Félév} & \textbf{Tanulási Eredmény} \\
            \hline
            \endhead
                                
                                          1 &  Megismeri lakóhelye, szülőföldje természeti adottságait, hagyományos gazdasági tevékenységeit, néprajzi jellemzőit, történetének nevezetesebb eseményeit, jeles személyeit. A tanulási folyamatban kialakul az egyéni, családi, közösségi, nemzeti azonosságtudata. \\ \hline
                                          1 &  Általános képe van a hagyományos gazdálkodó életmód fontosabb területeiről, a család felépítéséről, a családon belüli munkamegosztásról. A megszerzett ismeretek birtokában értelmezi a más tantárgyakban felmerülő népismereti tartalmakat. \\ \hline
                                      
                                
                                          2 &  Felfedezi a jeles napok, ünnepi szokások, az emberi élet fordulóihoz kapcsolódó népszokások, valamint a társas munkák, közösségi alkalmak hagyományainak jelentőségét, közösségmegtartó szerepüket a paraszti élet rendjében. Élményszerűen, hagyományhű módon elsajátítja egy-egy jeles nap, ünnepkör köszöntő vagy színjátékszerű szokását, valamint a társas munkák, közösségi alkalmak népszokásait és a hozzájuk kapcsolódó tevékenységeket. \\ \hline
                                      
                                
                                          4 &  Ismeri a magyar nyelvterület földrajzi-néprajzi tájainak, tájegységeinek hon- és népismereti, néprajzi jellemzőit. Világos számára, hogyan függ össze egy táj természeti adottsága a gazdasági tevékenységekkel, a népi építészettel, hogyan élt harmonikus kapcsolatban az ember a természettel. \\ \hline
                                      
                        \end{longtable}
            \clearpage

        \section{Idegen nyelv}

       
           \begin{longtable}{c | p{0.8\textwidth} }
            \caption[Idegen nyelv 3-4.]{Idegen nyelv tantárgy tanulási eredményei féléves bontásokban a 3-4. évfolyamszinteken. }  \\

            \textbf{Félév} & \textbf{Tanulási Eredmény} \\
            \hline
            \endhead
                                
                                          4 &  Aktívan részt vesz a célnyelvi tevékenységekben, követi a célnyelvi óravezetést, az egyszerű tanári utasításokat, megérti az ismerős kérdéseket, és válaszol ezekre; kiszűri az egyszerű, rövid szövegek lényegét. \\ \hline
                                          4 &  Idegen nyelven elmond néhány verset, mondókát és néhány összefüggő mondatot önmagáról, minta alapján egyszerű párbeszédet folytat társaival. \\ \hline
                                          4 &  Ismert szavakat, rövid szövegeket elolvas és megért jól ismert témában. \\ \hline
                                          4 &  Tanult szavakat, ismerős mondatokat lemásol, minta alapján egyszerű, rövid szövegeket alkot. \\ \hline
                                      
                        \end{longtable}
            \clearpage

       
           \begin{longtable}{c | p{0.8\textwidth} }
            \caption[Idegen nyelv 5-6.]{Idegen nyelv tantárgy tanulási eredményei féléves bontásokban a 5-6. évfolyamszinteken. }  \\

            \textbf{Félév} & \textbf{Tanulási Eredmény} \\
            \hline
            \endhead
                                
                                          4 &  A1 szintű nyelvtudás: \\ \hline
                                          4 &  Érti a gazdagodó nyelvi eszközökkel megfogalmazott óravezetést, az ismert témákhoz kapcsolódó kérdéseket, rövid megnyilatkozásokat, szövegeket. \\ \hline
                                          4 &  Szóban és írásban egyszerű nyelvi eszközökkel, begyakorolt beszédfordulatokkal (idiómákkal) kommunikál. \\ \hline
                                          4 &  Felkészülés után elmond rövid szövegeket. \\ \hline
                                          4 &  Közös feldolgozás után megérti az egyszerű olvasott szövegek lényegét, tartalmát. \\ \hline
                                          4 &  Ismert témáról rövid, egyszerű mondatokat ír, mintát követve önálló írott szövegeket alkot. \\ \hline
                                      
                        \end{longtable}
            \clearpage

       
           \begin{longtable}{c | p{0.8\textwidth} }
            \caption[Idegen nyelv 7-8.]{Idegen nyelv tantárgy tanulási eredményei féléves bontásokban a 7-8. évfolyamszinteken. }  \\

            \textbf{Félév} & \textbf{Tanulási Eredmény} \\
            \hline
            \endhead
                                
                                          4 &  A2 szintű nyelvtudás: \\ \hline
                                          4 &  Egyszerű hangzó szövegekből kiszűri a lényeget és néhány konkrét információt. \\ \hline
                                          4 &  Kérdésekre válaszol, rövid beszélgetésekben vesz részt. \\ \hline
                                          4 &  Egyre bővülő szókinccsel, egyszerű nyelvi eszközökkel megfogalmazva történetet mesél el, valamint leírást ad saját magáról és közvetlen környezetéről. \\ \hline
                                          4 &  Ismert témákról írt rövid szövegeket megért és értelmez. Különböző típusú, egyszerű írott szövegekben megtalálja a fontos információkat. \\ \hline
                                          4 &  Összefüggő mondatokat, rövid szöveget ír hétköznapi, őt érintő témákról. \\ \hline
                                      
                        \end{longtable}
            \clearpage

       
           \begin{longtable}{c | p{0.8\textwidth} }
            \caption[Idegen nyelv 9-10.]{Idegen nyelv tantárgy tanulási eredményei féléves bontásokban a 9-10. évfolyamszinteken. }  \\

            \textbf{Félév} & \textbf{Tanulási Eredmény} \\
            \hline
            \endhead
                                
                                          4 &  Megérti nagy vonalakban az idegennyelvű köznyelvi beszédet, ha az számára rendszeresen előforduló, ismerős témákról folyik. \\ \hline
                                          4 &  A mindennapi élet legtöbb helyzetében boldogul idegen nyelven, gondolatokat cserél, véleményt mond, érzelmeit kifejezi és stílusában a kommunikációs helyzethez alkalmazkodik. \\ \hline
                                          4 &  A begyakorolt szerkezetekkel érthetően, folyamatoshoz közelítően beszél. Az átadott információ lényegét megközelítő tartalmi pontossággal fejti ki. \\ \hline
                                          4 &  Megérti a hétköznapi nyelven írt, érdeklődési köréhez kapcsolódó, lényegre törő, autentikus vagy kismértékben szerkesztett szövegekben az általános vagy részinformációkat. \\ \hline
                                          4 &  A tanuló több műfajban is egyszerű, rövid, összefüggő szövegeket fogalmaz ismert, hétköznapi témákról. Írásbeli megnyilatkozásaiban kezdenek megjelenni műfaji sajátosságok és különböző stílusjegyek. \\ \hline
                                      
                        \end{longtable}
            \clearpage

       
           \begin{longtable}{c | p{0.8\textwidth} }
            \caption[Idegen nyelv 11-12.]{Idegen nyelv tantárgy tanulási eredményei féléves bontásokban a 11-12. évfolyamszinteken. }  \\

            \textbf{Félév} & \textbf{Tanulási Eredmény} \\
            \hline
            \endhead
                                
                                          4 &  Főbb vonalaiban és egyes részleteiben is megérti a köznyelvi beszédet a számára ismerős témákról. \\ \hline
                                          4 &  Idegen nyelven önállóan boldogul, véleményt mond és érvel a mindennapi élet legtöbb, akár váratlan helyzetében is. Stílusában és regiszterhasználatában alkalmazkodik a kommunikációs helyzethez. \\ \hline
                                          4 &  A szintnek megfelelő szókincs és szerkezetek segítségével fejezi ki magát az ismerős témakörökben. Folyamatosan, érthetően, a főbb pontok tekintetében tartalmilag pontosan, megfelelő stílusban beszél. \\ \hline
                                          4 &  Több műfajban, részleteket is tartalmazó, összefüggő szövegeket fogalmaz ismert, hétköznapi és elvontabb témákról. Írásbeli megnyilatkozásaiban műfaji sajátosságokat és különböző stílusjegyeket alkalmaz. \\ \hline
                                          4 &  Megérti a gondolatmenet lényegét és egyes részinformációkat a nagyrészt közérthető nyelven írt, érdeklődési köréhez kapcsolódó, lényegre törően megfogalmazott szövegekben. \\ \hline
                                      
                        \end{longtable}
            \clearpage

        \section{Informatika}

       
           \begin{longtable}{c | p{0.8\textwidth} }
            \caption[Informatika 5-6.]{Informatika tantárgy tanulási eredményei féléves bontásokban a 5-6. évfolyamszinteken. }  \\

            \textbf{Félév} & \textbf{Tanulási Eredmény} \\
            \hline
            \endhead
                                
                                          1 &  Ismeri a számítógép részeinek és perifériáinak funkcióit, azokat önállóan használja. \\ \hline
                                          1 &  A könyvtárszerkezetben tájékozódik, mozog, könyvtárat vált és tud fájlt keresni. \\ \hline
                                          1 &  Segítséggel multimédiás oktatóprogramokat használ. \\ \hline
                                          1 &  Az iskolai hálózatba belép, onnan kilép, ismeri és betartja a hálózat használatának szabályait. \\ \hline
                                          1 &  Ismeri egy vírusellenőrző program kezelését. \\ \hline
                                          1 &  Ismeri a szövegszerkesztés alapfogalmait, önállóan elvégzi a leggyakoribb karakter- és bekezdésformázásokat. \\ \hline
                                          1 &  Használja a szövegszerkesztő nyelvi segédeszközeit. \\ \hline
                                          1 &  Ismeri egy bemutatókészítő program egyszerű lehetőségeit, tud rövid bemutatót készíteni. \\ \hline
                                          1 &  A problémamegoldás során együttműködik társaival. \\ \hline
                                          1 &  Használja a böngészőprogram főbb funkcióit. \\ \hline
                                          1 &  Felnőtt segítséggel megadott szempontok szerint információt tud keresni. \\ \hline
                                          1 &  A keresési találatokat értelmezi. \\ \hline
                                          1 &  Az elektronikus levelezőrendszert önállóan kezeli. \\ \hline
                                          1 &  Elektronikus és internetes médiumokat használ. \\ \hline
                                          1 &  Az interneten talált információkat menteni tudja. \\ \hline
                                          1 &  Ismeri a netikett szabályait. \\ \hline
                                          1 &  Különböző tantárgyi feladataihoz megadott forrásokat megtalálja és további releváns forrásokat keres. \\ \hline
                                      
                                
                                          2 &  Különböző dokumentumokból származó részleteket saját munkájában el tud helyezni. \\ \hline
                                          2 &  Ismeri az informatikai biztonsággal kapcsolatos fogalmakat. \\ \hline
                                          2 &  Adatvédelemmel kapcsolatos fogalmakat ismer. \\ \hline
                                          2 &  Ismeri az adatvédelem érdekében alkalmazható lehetőségeket. \\ \hline
                                          2 &  Az informatikai eszközök etikus használatára vonatkozó szabályokat ismeri. \\ \hline
                                          2 &  Feltünteti az információforrásokat a saját dokumentumokban. \\ \hline
                                          2 &  Nyomtatott és elektronikus forrásokban megtalálja a feladatok megoldásához szükséges információkat. \\ \hline
                                          2 &  Eldönti, hogy az iskolai vagy a lakóhelyi könyvtár szolgáltatásait vegye igénybe. \\ \hline
                                      
                                
                                          3 &  Felismeri az összetartozó adatok közötti egyszerű összefüggéseket. \\ \hline
                                          3 &  A problémamegoldáshoz szükséges információt összegyűjti. \\ \hline
                                          3 &  Önállóan vagy segítséggel algoritmust készít. \\ \hline
                                          3 &  Egyszerű programot ír. \\ \hline
                                      
                                
                                          4 &  Segítséggel tantárgyi, könyvtári, hálózati adatbázisokat használ, különféle adatbázisokban keres. \\ \hline
                                          4 &  Ismeri a problémamegoldás alapvető lépéseit. \\ \hline
                                          4 &  Egy fejlesztőrendszert alapszinten használ. \\ \hline
                                      
                        \end{longtable}
            \clearpage

       
           \begin{longtable}{c | p{0.8\textwidth} }
            \caption[Informatika 7-8.]{Informatika tantárgy tanulási eredményei féléves bontásokban a 7-8. évfolyamszinteken. }  \\

            \textbf{Félév} & \textbf{Tanulási Eredmény} \\
            \hline
            \endhead
                                
                                          1 &  Ismeri a különböző informatikai környezeteket. \\ \hline
                                          1 &  Használja az operációs rendszer és a számítógépes hálózat alapszolgáltatásait. \\ \hline
                                          1 &  Segítséggel kiválasztja az adott feladat megoldásához alkalmas hardver- és szoftvereszközöket. \\ \hline
                                          1 &  Dokumentumokba különböző objektumokat beilleszt. \\ \hline
                                          1 &  Készít szöveget, képet és táblázatot is tartalmazó dokumentumot minta vagy leírás alapján. \\ \hline
                                          1 &  Létrehoz egyszerű táblázatot. \\ \hline
                                          1 &  Diagramokat készít és módosít. \\ \hline
                                          1 &  Bemutatót készít. \\ \hline
                                          1 &  Információt keres és talál. \\ \hline
                                          1 &  Információt értékel. \\ \hline
                                          1 &  Használja a legújabb infokommunikációs technológiákat, szolgáltatásokat. \\ \hline
                                          1 &  Ismeri az informatikai biztonsággal és adatvédelemmel kapcsolatos fogalmakat. \\ \hline
                                          1 &  Ismeri az adatokkal való visszaélésekből származó veszélyeket és következményeket. \\ \hline
                                          1 &  Ismer megbízható információforrásokat. \\ \hline
                                          1 &  Megállapítja az információ hitelességét. \\ \hline
                                          1 &  Az informatikai eszközök etikus használatára vonatkozó szabályokat ismeri. \\ \hline
                                          1 &  Ismeri az információforrások etikus felhasználási lehetőségeit. \\ \hline
                                          1 &  Felismeri az informatikai eszközök használatának az emberi kapcsolatokra vonatkozó következményeit. \\ \hline
                                          1 &  Ismer néhány elektronikus szolgáltatást. \\ \hline
                                          1 &  Igénybeveszi, használja, lemondja a szolgáltatásokat. \\ \hline
                                          1 &  A könyvtár és az internet szolgáltatásait igénybe véve önállóan talál releváns forrásokat konkrét tantárgyi feladataihoz. \\ \hline
                                          1 &  A választott forrásokat alkotóan és etikusan használja feladatmegoldásnál. \\ \hline
                                          1 &  Alkalmazza a más tárgyakban tanultakat (pl. informatikai eszközök használata, szövegalkotás). \\ \hline
                                          1 &  Egyszerű témában önállóan végrehajtja az információs problémamegoldás folyamatát. \\ \hline
                                      
                                
                                          2 &  Látja a problémamegoldás folyamatát. \\ \hline
                                          2 &  Használja az algoritmusleíró eszközöket. \\ \hline
                                          2 &  Alapszintű programokat ír. \\ \hline
                                          2 &  Információt weben történő publikálásra előkészít. \\ \hline
                                          2 &  Megkülönbözteti a publikussá tehető és védendő adatait. \\ \hline
                                      
                                
                                          3 &  Algoritmusokat kódol. \\ \hline
                                          3 &  Egyszerű vezérlési feladatokat megold fejlesztői környezetben. \\ \hline
                                          3 &  Tervezési eljárásokat alkalmaz. \\ \hline
                                          3 &  Meghatározza az eredményt a bemenő adatok alapján. \\ \hline
                                          3 &  Tantárgyi szimulációs programokat használ. \\ \hline
                                      
                        \end{longtable}
            \clearpage

       
           \begin{longtable}{c | p{0.8\textwidth} }
            \caption[Informatika 9-10.]{Informatika tantárgy tanulási eredményei féléves bontásokban a 9-10. évfolyamszinteken. }  \\

            \textbf{Félév} & \textbf{Tanulási Eredmény} \\
            \hline
            \endhead
                                
                                          1 &  Felvételt készít digitális kamerával, adatokat áttölti kameráról a számítógép adathordozójára. \\ \hline
                                          1 &  Ismeri az adatvédelem hardveres és szoftveres módjait. \\ \hline
                                          1 &  Ismeri az ergonómia alapjait. \\ \hline
                                          1 &  Táblázatkezelővel tantárgyi feladatokat megold, egyszerű számításokat elvégez. \\ \hline
                                          1 &  Körlevelet készít. \\ \hline
                                          1 &  Adatbázis-kezelő programot használ. \\ \hline
                                          1 &  Adattáblák között kapcsolatokat épít, adatbázisokból lekérdezéssel információt nyer ki. A kinyert adatokat esztétikus, használható formába rendezi. \\ \hline
                                          1 &  Algoritmikusokat készít. \\ \hline
                                          1 &  A probléma megoldásához szükséges eszközöket kiválasztja. \\ \hline
                                          1 &  Csoportban tevékenykedik. \\ \hline
                                      
                                
                                          2 &  Információt szerez és azokat hagyományos, elektronikus vagy internetes eszközökkel publikálja. \\ \hline
                                          2 &  Társaival az interneten kommunikál, közös feladatokon dolgozik. \\ \hline
                                          2 &  Újabb informatikai eszközöket, információszerzési technológiákat használ. \\ \hline
                                          2 &  Adatvédelemmel kapcsolatos fogalmakat ismer. \\ \hline
                                          2 &  Információforrásokat értékel. \\ \hline
                                          2 &  Az informatikai eszközök etikus használatára vonatkozó szabályokat ismeri. \\ \hline
                                      
                                
                                          3 &  Tantárgyi szimulációs programokat használ. \\ \hline
                                          3 &  A szerzői joggal kapcsolatos alapfogalmakat ismeri. \\ \hline
                                          3 &  Az infokommunikációs publikálási szabályokat ismeri. \\ \hline
                                          3 &  Az informatikai fejlesztések gazdasági, környezeti, kulturális hatásait felismeri. \\ \hline
                                          3 &  Az informatikai eszközök használatának a személyiséget és az egészséget befolyásoló hatásait felismeri. \\ \hline
                                          3 &  Az elektronikus szolgáltatások szerepét ismeri. \\ \hline
                                          3 &  Saját információkeresési stratégiáival tisztában van, azokat tudatosan alkalmazza, értékeli és fejleszti. \\ \hline
                                      
                                
                                          4 &  Tantárgyi mérések eredményeit kiértékeli. \\ \hline
                                          4 &  Elektronikus szolgáltatást tudatosan használ. \\ \hline
                                          4 &  Az elektronikus szolgáltatások jellemzőit, előnyeit, hátrányait felismeri. \\ \hline
                                          4 &  Felismeri a fogyasztói viselkedést befolyásoló módszereket a médiában. \\ \hline
                                          4 &  Felismeri a tudatos vásárló jellemzőit. \\ \hline
                                          4 &  A tanulmányaihoz kapcsolódó feladatai során az információs problémamegoldás folyamatát önállóan, alkotóan végrehajtja. \\ \hline
                                      
                        \end{longtable}
            \clearpage

        \section{Kémia}

       
           \begin{longtable}{c | p{0.8\textwidth} }
            \caption[Kémia 7-8.]{Kémia tantárgy tanulási eredményei féléves bontásokban a 7-8. évfolyamszinteken. }  \\

            \textbf{Félév} & \textbf{Tanulási Eredmény} \\
            \hline
            \endhead
                                
                                          1 &  Ismeri a kémia egyszerűbb alapfogalmait (atom, kémiai és fizikai változás, elem, vegyület, keverék, halmazállapot, molekula, anyagmennyiség, tömegszázalék, kémiai egyenlet, égés, oxidáció, redukció, sav, lúg, kémhatás), alaptörvényeit, vizsgálati céljait, módszereit és kísérleti eszközeit, a mérgező anyagok jelzéseit. \\ \hline
                                          1 &  Ismeri néhány, a hétköznapi élet szempontjából jelentős szervetlen és szerves vegyület tulajdonságait, egyszerűbb esetben ezen anyagok előállítását és a mindennapokban előforduló anyagok biztonságos felhasználásának módjait. \\ \hline
                                          1 &  Egy kémiával kapcsolatos témáról önállóan vagy csoportban dolgozva információt keres és ennek eredményét másoknak változatos módszerekkel, az infokommunikációs technológia eszközeit is alkalmazva bemutatja. \\ \hline
                                          1 &  Alkalmazza a megismert törvényszerűségeket egyszerűbb, a hétköznapi élethez is kapcsolódó problémák, kémiai számítási feladatok megoldása során, illetve gyakorlati szempontból jelentős kémiai reakciók egyenleteinek leírásában. \\ \hline
                                      
                                
                                          2 &  Érti és alkalmazza a fogalmakat és törvényeket, ez alapján magyarázatot ad a vegyületek viselkedésére, a kísérletek során tapasztalt jelenségekre. \\ \hline
                                          2 &  Használja a megismert egyszerű modelleket a mindennapi életben előforduló, a kémiával kapcsolatos jelenségek elemzéseskor. \\ \hline
                                      
                                
                                          3 &  Érti a kémia sajátos jelrendszerét, a periódusos rendszer és a vegyértékelektron-szerkezet kapcsolatát, egyszerű vegyületek elektronszerkezeti képletét, a tanult modellek és a valóság kapcsolatát. \\ \hline
                                      
                                
                                          4 &  Tudja, hogy a kémia a társadalom és a gazdaság fejlődésében fontos szerepet játszik. \\ \hline
                                          4 &  Megszerzett tudását alkalmazva felelős döntéseket hoz a saját életével, egészségével kapcsolatos kérdésekben, aktív szerepet vállal a személyes környezetének megóvásában. \\ \hline
                                      
                        \end{longtable}
            \clearpage

       
           \begin{longtable}{c | p{0.8\textwidth} }
            \caption[Kémia 9-10.]{Kémia tantárgy tanulási eredményei féléves bontásokban a 9-10. évfolyamszinteken. }  \\

            \textbf{Félév} & \textbf{Tanulási Eredmény} \\
            \hline
            \endhead
                                
                                          1 &  Érti a fenntarthatóság fogalmát és jelentőségét. \\ \hline
                                          1 &  Értelmezi az anyagi halmazok jellemzőit összetevőik szerkezete és kölcsönhatásaik alapján. \\ \hline
                                          1 &  Meglát egyszerű kémiai jelenségekben ok-okozati elemeket; tervez ezek hatását bemutató, vizsgáló egyszerű kísérletet, és ennek eredményei alapján értékeli a kísérlet alapjául szolgáló hipotéziseket. \\ \hline
                                      
                                
                                          2 &  Szóbeli és írásbeli összefoglalót, prezentációt készít önállóan vagy csoportban egy kémiával kapcsolatos témáról sokféle információforrás kritikus felhasználásával; és azt érthető formában közönség előtt bemutatja. \\ \hline
                                          2 &  Alkalmazza a megismert kémiai tényeket és törvényszerűségeket egyszerűbb problémák és számítási feladatok megoldása során, valamint a fenntarthatósághoz és az egészségmegőrzéshez kapcsolódó viták alkalmával. \\ \hline
                                          2 &  Koherens és kritikus érvelés alkalmazásával véleményt formál kémiai tárgyú ismeretterjesztő, egyszerű tudományos, illetve áltudományos cikkekről; az abban szereplő állításokat a tanult ismereteivel összekapcsolja, mások érveivel ütközteti. \\ \hline
                                      
                                
                                          3 &  Ismeri az anyag tulajdonságainak anyagszerkezeti alapokon történő magyarázatához elengedhetetlenül fontos modelleket, fogalmakat, az összefüggéseket és a törvényszerűségeket, a legfontosabb szerves és szervetlen vegyületek szerkezetét, tulajdonságait, csoportosítását, előállítását, gyakorlati jelentőségét. \\ \hline
                                      
                                
                                          4 &  Érti az alkalmazott modellek és a valóság kapcsolatát, a szerves vegyületek esetében a funkciós csoportok tulajdonságokat meghatározó szerepét, a tudományos és az áltudományos megközelítés közötti különbségeket. \\ \hline
                                      
                        \end{longtable}
            \clearpage

        \section{Környezetismeret}

       
           \begin{longtable}{c | p{0.8\textwidth} }
            \caption[Környezetismeret 1-2.]{Környezetismeret tantárgy tanulási eredményei féléves bontásokban a 1-2. évfolyamszinteken. }  \\

            \textbf{Félév} & \textbf{Tanulási Eredmény} \\
            \hline
            \endhead
                                
                                          1 &  Tiszteli az élővilág sokféleségét, felismeri a természetvédelem fontosságát. \\ \hline
                                      
                                
                                          2 &  Jól tájékozódik az iskolában és környékén. Az évszakos és napszakos változásokat felismeri és kapcsolja életmódbeli szokásokhoz. Az időjárás elemeit ismeri, az ezzel kapcsolatos piktogramokat értelmezi, az időjáráshoz illő szokásokat alkalmazza. \\ \hline
                                      
                                
                                          3 &  Az emberi test nemre és korra jellemző arányait leírja, a fő testrészeket megnevezi. Az egészséges életmód alapvető elemeit ismeri és alkalmazza. \\ \hline
                                          3 &  Használati tárgyakat és gyakori, a közvetlen környezetben előforduló anyagokat csoportosítja tulajdonságaik szerint, felismeri a kapcsolatot az anyagi tulajdonságok és a felhasználás között. A mesterséges és természetes anyagokat megkülönbözteti. A halmazállapotokat felismeri. \\ \hline
                                          3 &  Megfigyeléseket végez a természetben, egyszerű vizsgálatokat és kísérleteket  folytat. Az eredményeket megfogalmazza, ábrázolja. Ok-okozati összefüggéseket keres a tapasztalatok magyarázatára. \\ \hline
                                      
                                
                                          4 &  Mesterséges és természetes életközösséget össze tudja hasonlítani. \\ \hline
                                      
                        \end{longtable}
            \clearpage

       
           \begin{longtable}{c | p{0.8\textwidth} }
            \caption[Környezetismeret 3-4.]{Környezetismeret tantárgy tanulási eredményei féléves bontásokban a 3-4. évfolyamszinteken. }  \\

            \textbf{Félév} & \textbf{Tanulási Eredmény} \\
            \hline
            \endhead
                                
                                          2 &  Az egészséges életmód alapvető elemeit alkalmazza az egészségmegőrzés, az egészséges fejlődés és a betegségek elkerülése érdekében. \\ \hline
                                          2 &  Adott szempontok alapján megfigyeléseket végez a természetben, egyszerű kísérleteken keresztül tanulmányozza a természeti jelenségeket. \\ \hline
                                      
                                
                                          3 &  A mindennapi életben előforduló távolságokat és időtartamokat megbecsli, a hosszúságot és az időt méri. \\ \hline
                                          3 &  Bemutat egy természetes életközösséget. \\ \hline
                                          3 &  Elhelyezi Magyarországot a földrajzi térben, ismeri az ország néhány fő kulturális és természeti értékét. \\ \hline
                                      
                                
                                          4 &  Az életkornak megfelelően és a helyzethez illően, felelősen viselkedik a segítségnyújtást igénylő helyzetekben. \\ \hline
                                          4 &  A fenntartható életmód jelentőségét megmagyarázza konkrét példán kereszül és értelmezi a hagyományok szerepét a természeti környezettel való harmonikus kapcsolat kialakításában, illetve felépítésében. \\ \hline
                                          4 &  Bemutatja az élőlények szerveződési szintjeit és az életközösségek kapcsolatait, csoportosítja az élőlényeket tetszőleges és adott szempontsor szerint. \\ \hline
                                          4 &  Értelmez egy technológiai folyamatot egy konkrét gyártási folyamat kapcsán és ismeri az ehhez kapcsolódó fogyasztói magatartást. \\ \hline
                                          4 &  Irányítottan használ informatikai és kommunikációs eszközöket az információkeresésben és a problémák megoldásában. \\ \hline
                                      
                        \end{longtable}
            \clearpage

        \section{Magyar nyelv és irodalom}

       
           \begin{longtable}{c | p{0.8\textwidth} }
            \caption[Magyar nyelv és irodalom 1-2.]{Magyar nyelv és irodalom tantárgy tanulási eredményei féléves bontásokban a 1-2. évfolyamszinteken. }  \\

            \textbf{Félév} & \textbf{Tanulási Eredmény} \\
            \hline
            \endhead
                                
                                          1 &  Érthetően beszél, tisztában van a szóbeli kommunikáció alapvető szabályaival és alkalmazza is őket. \\ \hline
                                          1 &  Megérti az egyszerű magyarázatokat, utasításokat és társai közléseit. \\ \hline
                                      
                                
                                          4 &  A kérdésekre értelmesen válaszol. Aktivizálja a szókincsét a szövegalkotó feladatokban. Használja a bemutatkozás, a felnőttek és a kortársak megszólításának és köszöntésének udvarias nyelvi formáit. Összefüggő mondatok alkot. Követhetően számol be élményeiről, olvasmányai tartalmáról. Szöveghűen mond el a memoritereket. \\ \hline
                                          4 &  Ismeri az írott és nyomtatott betűket, rendelkezik megfelelő szókinccsel. Ismert és begyakorolt szöveget folyamatosságra, pontosságra törekvően olvas fel. Tanára segítségével kiemeli az olvasottak lényegét. Írása rendezett, pontos. \\ \hline
                                          4 &  Felismeri és megnevezi a tanult nyelvtani fogalmakat, szükség szerint felidézi és alkalmazza a helyesírási szabályokat a begyakorolt szókészlet szavaiban. helyesen jelöli a j hangot 30–40 begyakorolt szóban. Helyesen választja el az egyszerű szavakat. \\ \hline
                                          4 &  Tisztában van a tanulás alapvető céljával. Ítélőképessége, erkölcsi, esztétikai és történeti érzéke az életkori sajátosságoknak megfelelően fejlett. \\ \hline
                                      
                        \end{longtable}
            \clearpage

       
           \begin{longtable}{c | p{0.8\textwidth} }
            \caption[Magyar nyelv és irodalom 3-4.]{Magyar nyelv és irodalom tantárgy tanulási eredményei féléves bontásokban a 3-4. évfolyamszinteken. }  \\

            \textbf{Félév} & \textbf{Tanulási Eredmény} \\
            \hline
            \endhead
                                
                                          1 &  Tanulási tevékenységét fokozatosan növekvő időtartamban irányjtja tudatos figyelemmel. Feladatainak megoldásához szükség szerint veszi igénybe az iskola könyvtárát. A könyvekben, gyermekújságokban a tartalomjegyzék segítségével eligazodik. \\ \hline
                                      
                                
                                          2 &  Értelmesen és érthetően fejezi ki gondolatait. Aktivizálja a szókincsét a szövegalkotó feladatokban. Használja a mindennapi érintkezésben az udvarias nyelvi fordulatokat. Beszédstílusát a beszélgető partneréhez igazítja. \\ \hline
                                          2 &  A memoritereket szöveghűen elmondja. \\ \hline
                                          2 &  Ismeri a tanulás alapvető céljait. Ítélőképessége, erkölcsi, esztétikai és történeti érzéke életkorának megfelel. Nyitott anyanyelvi képességei fejlesztésére. Az anyanyelvi részképességeit. \\ \hline
                                      
                                
                                          4 &  Bekapcsolódik a csoportos beszélgetésbe, vitába, történetalkotásba, improvizációba, közös élményekről, tevékenységekről való beszélgetésekbe, értékelésbe. A közös tevékenységeket együttműködő magatartással segíti. \\ \hline
                                          4 &  Felkészülés után folyamatosan, érthetően olvas fel ismert szöveget. Életkorának megfelelő szöveget megért néma olvasás útján. Az olvasottakkal kapcsolatos véleményét értelmesen fogalmazza meg. Ismer és alkalmaz néhány olvasási stratégiát. \\ \hline
                                          4 &  Adott vagy választott témáról 8–10 mondatos fogalmazást készít. \\ \hline
                                          4 &  Az alsó tagozaton tanult anyanyelvi ismereteit rendszerezetten alkalmazza. Biztonsággal felismeri a tanult szófajokat, és megnevezi azokat szövegben is. \\ \hline
                                          4 &  A begyakorolt szókészlet körében helyesen alkalmazza a tanult helyesírási szabályokat. Írásbeli munkái rendezettek, olvashatóak. Helyesírását önellenőrzéssel vizsgálja és szükség esetén javítja. \\ \hline
                                      
                        \end{longtable}
            \clearpage

       
           \begin{longtable}{c | p{0.8\textwidth} }
            \caption[Magyar nyelv és irodalom 5-6.]{Magyar nyelv és irodalom tantárgy tanulási eredményei féléves bontásokban a 5-6. évfolyamszinteken. }  \\

            \textbf{Félév} & \textbf{Tanulási Eredmény} \\
            \hline
            \endhead
                                
                                          1 &  Gondolatait érthetően, a helyzetnek megfelelően megfogalmazza, a beszédet kísérő nem nyelvi jeleket adekvátan alkalmazza. Megért, összefoglal, továbbad rövidebb szóbeli üzeneteket, rövidebb hallott történeteket. \\ \hline
                                      
                                
                                          2 &  Ismeri és alkalmazza a legalapvetőbb anyaggyűjtési, vázlatkészítési módokat. Szöveget alkot a tanult hagyományos és internetes műfajokban (elbeszélés, leírás, jellemzés, levél, SMS, e-mail stb.). Törekszik az igényes, pontos és helyes fogalmazásra, írásra. \\ \hline
                                          2 &  Új szavakat, közmondásokat, szólásokat használ. \\ \hline
                                          2 &  Az olvasott és megtárgyalt irodalmi művek nyomán azonosít erkölcsi értékeket és álláspontokat, saját álláspontját. \\ \hline
                                      
                                
                                          4 &  Ismeri a szövegértés folyamatát, annak megfigyelésével saját módszerét fejleszti, a hibás olvasási szokásaira megfelelő javító stratégiát talál, és azt alkalmazza. \\ \hline
                                          4 &  Ismeri a tanult alapszófajok (ige, főnév, melléknév, számnév, határozószó, igenevek, névmások), valamint az igekötők általános jellemzőit, alaki sajátosságait, a hozzájuk kapcsolódó főbb helyesírási szabályokat, amelyeket az írott munkáiban igyekszik alkalmazni is. \\ \hline
                                          4 &  Megnevez három mesetípust példákkal, és felidéz címe vagy részlete említésével három népdalt. \\ \hline
                                          4 &  Különbséget tesz a népmese és a műmese között. \\ \hline
                                          4 &  Megfogalmazza, mi a különbség a mese és a monda között. \\ \hline
                                          4 &  Elkülöníti a rímes, ritmikus szöveget a prózától. \\ \hline
                                          4 &  Megnevezi, melyik mű nem mond el történetet, melyik jelenít meg konfliktust párbeszédes formában, és melyik fejez ki érzést, élményt. \\ \hline
                                          4 &  Felismeri a hexameteres szövegről, hogy az időmértékes, a felező tizenkettesről, hogy az ütemhangsúlyos. \\ \hline
                                          4 &  Felsorol három-négy művet Petőfitől és Aranytól, megfogalmaz egyszerűbb  összehasonlításokat János vitéz és Toldi Miklós alakjáról. \\ \hline
                                          4 &  Értelmezi A walesi bárdokban rejlő allegóriát, és ismerteti 5–6 mondatban az Egri csillagok történelmi hátterét. \\ \hline
                                          4 &  Elkülöníti az egyszerűbb versekben és prózai szövegekben a nagyobb szerkezeti egységeket. \\ \hline
                                          4 &  Összefoglalja néhány hosszabb mű cselekményét (János vitéz, Toldi, A Pál utcai fiúk, Egri csillagok), megkülönbözteti, melyik közülük a regény és melyik az elbeszélő költemény. \\ \hline
                                          4 &  Értelmesen és pontosan, tisztán, tagoltan, megfelelő ritmusban olvas fel szövegeket. \\ \hline
                                          4 &  Részt vesz számára ismert témájú vitában, és érvel. \\ \hline
                                          4 &  Ismert és könnyen érthető történetben párosítja annak egyes szakaszait a konfliktus, bonyodalom, tetőpont fogalmával. \\ \hline
                                          4 &  Az általa jól ismert történetek szereplőit jellemzi, kapcsolatrendszerüket feltárja és ismerteti. \\ \hline
                                          4 &  Néhány példa közül kiválasztja az egyszerűbb metaforákat és metonímiákat. \\ \hline
                                          4 &  Egyszerű meghatározást ad a következő fogalmakról: líra, epika, dráma, epizód, megszemélyesítés, ballada. \\ \hline
                                          4 &  Néhány egyszerűbb meghatározás közül kiválasztja azt, amely a következő fogalmak valamelyikéhez illik: dal, rím, ritmus, mítosz, motívum, konfliktus. \\ \hline
                                          4 &  Részt vesz művek, műrészletek szöveghű felidézésében. \\ \hline
                                      
                        \end{longtable}
            \clearpage

       
           \begin{longtable}{c | p{0.8\textwidth} }
            \caption[Magyar nyelv és irodalom 7-8.]{Magyar nyelv és irodalom tantárgy tanulási eredményei féléves bontásokban a 7-8. évfolyamszinteken. }  \\

            \textbf{Félév} & \textbf{Tanulási Eredmény} \\
            \hline
            \endhead
                                
                                          1 &  Képes a kulturált szociális érintkezésre, eligazodik és hatékonyan részt vesz a mindennapi páros és csoportos kommunikációs helyzetekben, vitákban. Figyeli és értelmezi partnerei kommunikációs szándékát, nem nyelvi jeleit.  \\ \hline
                                          1 &  Érzelmeit kifejezi, álláspontját megfelelő érvek, bizonyítékok segítségével megvédi, ugyanakkor empatikusan beleéli magát mások gondolatvilágába, érzelmeibe, megérti mások cselekvésének mozgatórugóit. \\ \hline
                                          1 &  A különböző megjelenésű és műfajú szövegeket átfogóan megérti, az adott szöveg szó szerinti jelentésén túli üzenetét is értelmezi, a szövegből információkat keres vissza. \\ \hline
                                          1 &  Érti a média alapvető kifejezőeszközeit, az írott és az elektronikus sajtó műfajait. Ismeri a média, kitüntetetten az audiovizuális média és az internet társadalmi szerepét, működési módjának legfőbb jellemzőit. Önálló, kritikus attitűddel, tudatosan használja a médiát. \\ \hline
                                      
                                
                                          2 &  Felismeri a földrajzi övezetesség kialakulásában megnyilvánuló összefüggéseket és törvényszerűségeket. \\ \hline
                                          2 &  Felismeri és értelmezi az egyes földrészekre vagy országcsoportokra, tájakra jellemző természeti jelenségeket és  társadalmi-gazdasági folyamatokat. \\ \hline
                                          2 &  Felismeri az egyes országok, országcsoportok helyét a világ társadalmi-gazdasági folyamataiban. \\ \hline
                                          2 &  Érzékeli az egyes térségek, országok társadalmi-gazdasági adottságai jelentőségének időbeli változásait. \\ \hline
                                          2 &  Felismeri a globalizáció érvényesülését regionális példákban. \\ \hline
                                          2 &  Ismeri hazánk társadalmi-gazdasági fejlődésének jellemzőit összefüggésben a természeti erőforrásokkal. \\ \hline
                                          2 &  Érti, hogy a hazai gazdasági, társadalmi és környezeti folyamatok világméretű vagy regionális folyamatokkal függenek össze. \\ \hline
                                          2 &  Átfogó és reális képzettel rendelkezik a Föld egészéről és annak kisebb-nagyobb egységeiről (a földrészekről és a világtengerről, a kontinensek karakteres nagytájairól és tipikus tájairól, valamint a világgazdaságban kiemelkedő jelentőségű országcsoportjairól, országairól). \\ \hline
                                          2 &  Átfogó ismerete van földrészünk, azon belül a meghatározó és a hazánkkal szomszédos országok természet- és társadalomföldrajzi sajátosságairól, látja azok térbeli és történelmi összefüggéseit, érzékeli a földrajzi tényezők életmódot meghatározó szerepét. \\ \hline
                                          2 &  Reális ismereteket birtokol a Kárpát-medencében fekvő hazánk földrajzi jellemzőiről, erőforrásairól és az ország gazdasági lehetőségeiről az Európai Unió keretében. \\ \hline
                                          2 &  Tisztában van az Európai Unió meghatározó szerepével és jelentőségével. \\ \hline
                                          2 &  Megérti az írott és elektronikus felületen megjelenő olvasott szöveget. \\ \hline
                                          2 &  Ismer és alkalmaz szövegértési stratégiákat. \\ \hline
                                          2 &  Önállóan információt gyűjt kézikönyvek és a korosztálynak szóló ismeretterjesztő források bevonásával. \\ \hline
                                          2 &  Adott szöveg tartalmát összefoglalja, önállóan jegyzetet és vázlatot készít. Az olvasott szöveg tartalmával kapcsolatos saját véleményét szóban és írásban megfogalmazza, állításait indokolja. \\ \hline
                                          2 &  Ismeri és alkalmazza a különböző mondatfajták használatát. Alkalmazza az írásbeli szövegalkotásban a mondatvégi, a tagmondatok, illetve mondatrészek közötti írásjeleket. Az összetett szavak és gyakoribb mozaikszók helyesírását ismeri, szükség szerint helyesírási segédletet használ. \\ \hline
                                          2 &  Ismeri a tömegkommunikáció fogalmát, legjellemzőbb területeit. \\ \hline
                                          2 &  A könnyebben besorolható műveket műfajilag azonosítja, 8-10 műfajt műnemekbe sorol, és a műnemek lényegét megfogalmazza. \\ \hline
                                          2 &  A különböző regénytípusok műfaji jegyeit felismeri, a szereplőket jellemezni tudja, a konfliktusok mibenlétét feltárja. \\ \hline
                                          2 &  Felismeri az alapvető lírai műfajok sajátosságait különböző korok alkotóinak művei alapján (elsősorban 19-20. századi alkotások). \\ \hline
                                          2 &  Felismeri néhány lírai mű beszédhelyzetét, a megszólító-megszólított viszony néhány jellegzetes típusát, azonosítja a művek tematikáját, meghatározó motívumait. \\ \hline
                                          2 &  Felfedez műfaji és tematikus-motivikus kapcsolatokat, azonosítja a zenei és ritmikai eszközök típusait, felismeri funkciójukat, hangulati hatásukat. \\ \hline
                                          2 &  Azonosít képeket, alakzatokat, szókincsbeli és mondattani jellegzetességeket, a lexika jelentésteremtő szerepét megérti a lírai szövegekben, megismeri a kompozíció meghatározó elemeit (pl. tematikus szerkezet, tér- és időszerkezet, logikai szerkezet, beszédhelyzet és változása). \\ \hline
                                          2 &  Konkrét szövegpéldán megmutatja a mindentudó és a tárgyilagos elbeszélői szerep különbözőségét, továbbá a közvetett és a közvetlen elbeszélésmód eltérését. \\ \hline
                                          2 &  Képes a drámákban, filmekben megjelenő emberi kapcsolatok, cselekedetek, érzelmi viszonyulások, konfliktusok összetettségének értelmezésére és megvitatására. \\ \hline
                                          2 &  Az olvasott, megtárgyalt művek erkölcsi kérdésfeltevéseire véleményében, érveiben tud támaszkodni. \\ \hline
                                          2 &  Szöveghűen felidéz műveket, műrészleteket. \\ \hline
                                      
                                
                                          3 &  Beszámolót, kiselőadást, prezentációt készít és tart írott és elektronikus forrásokból, kézikönyvekből, atlaszokból/szakmunkákból, a témától függően statisztikai táblázatokból, grafikonokból, diagramokból. \\ \hline
                                      
                                
                                          4 &  Egyszerű meghatározását adja a következő fogalmaknak: novella, rapszódia, lírai én, hexameter, pentameter, disztichon, szinesztézia, szimbólum, tragédia, komédia, dialógus, monológ. Néhány egyszerűbb meghatározás közül kiválasztja azt, amely a következő fogalmak valamelyikéhez illik: fordulat, retorika, paródia, helyzetkomikum, jellemkomikum. \\ \hline
                                      
                        \end{longtable}
            \clearpage

       
           \begin{longtable}{c | p{0.8\textwidth} }
            \caption[Magyar nyelv és irodalom 9-10.]{Magyar nyelv és irodalom tantárgy tanulási eredményei féléves bontásokban a 9-10. évfolyamszinteken. }  \\

            \textbf{Félév} & \textbf{Tanulási Eredmény} \\
            \hline
            \endhead
                                
                                          1 &  Helyesen és tudatosan alkalmazza a szövegfonetikai eszközöket a képes memoriterek szöveghű tolmácsolásánál. \\ \hline
                                          1 &  Szóbeli és írásbeli kommunikációs helyzetekben megválasztja a megfelelő hangnemet, nyelvváltozatot, stílusréteget. Alkalmazza a művelt köznyelv (regionális köznyelv), illetve a nyelvváltozatok nyelvhelyességi normáit, képes felismerni és értelmezni az attól eltérő nyelvváltozatokat. \\ \hline
                                          1 &  Az órai eszmecserékben és az irodalmi művekben megjelenő álláspontokat azonosítja, megvitatja és összehasonlítja eltérő véleményekkel. \\ \hline
                                      
                                
                                          2 &  Átfogóan értelmez szövegeket (pl. szépirodalmi, dokumentum- és ismeretterjesztő szöveg), értelmezi a szöveg szó szerinti jelentésén túli üzenetét, visszakeres a szövegben információkat. \\ \hline
                                          2 &  A szöveg tartalmát összefoglalja, tud önállóan jegyzetet és vázlatot készíteni. Képes az olvasott szöveg tartalmával kapcsolatos saját véleményét szóban és írásban megfogalmazni, indokolni. \\ \hline
                                          2 &  Felismeri és értelmezi a szövegek kapcsolatát és különbségét mind szóbeli, mint írásbeli műfajokban. \\ \hline
                                          2 &  Felismeri és alkalmazza a szépirodalmi és a nem szépirodalmi szövegekben megjelenített értékeket, erkölcsi kérdéseket, motivációkat, magatartásformákat. \\ \hline
                                          2 &  Definíciót, magyarázatot és egyszerűbb értekezést készít az olvasmányaiból. Megfogalmaz a tárgyalt problémákkal összefüggésben kérdéseket és felmerülő problémákat. \\ \hline
                                          2 &  Alkalmazza az idézés szabályait és etikai normáit. \\ \hline
                                      
                                
                                          3 &  Alkalmazza a művek műfaji természetének megfelelő szöveg-feldolgozási eljárásokat, megközelítési módokat. \\ \hline
                                          3 &  Tájékozott a feldolgozott lírai alkotások különböző műfajaiban és hangnemeiben. \\ \hline
                                          3 &  Bizonyítja különféle szövegek megértését a szöveg felépítésére, grammatikai jellemzőire, témahálózatára, tagolására irányuló elemzéssel; szöveghűen felolvas; kellő tempójú, olvasható, rendezett az írása.  \\ \hline
                                          3 &  Képes tudásanyagának megfogalmazására írásban a magyar és a világirodalom kiemelkedő alkotóiról. \\ \hline
                                      
                                
                                          4 &  Ismeri a hivatalos írásművek jellemzőit. Képes önálló szövegalkotásra ezek gyakori műfajaiban. \\ \hline
                                          4 &  Bemutatja a tanult irodalomtörténeti korszakok és stílusirányzatok sajátosságait. \\ \hline
                                          4 &  Eligazodik a 2000 utáni évek kortárs irodalmában (is), és önállóan kezeli az online irodalmi felületeket. \\ \hline
                                          4 &  Ismeri a digitális (szak)irodalmi szövegtárak világát, és azokban otthonosan mozog. \\ \hline
                                          4 &  Ő maga is rendszeresen ír papíralapú, illetve digitális írásokat. \\ \hline
                                          4 &  Részt vesz beszélgetésekben, vitákban, irodalomról folyó diskurzusokban és projektekben – ideértve a diákszínjátszást, -újságírást és -szerkesztést is. \\ \hline
                                          4 &  Beszélgetéseiben, vitáiban tud idézni memoriterekből és egyéb irodalmi alkotásokból. \\ \hline
                                      
                        \end{longtable}
            \clearpage

       
           \begin{longtable}{c | p{0.8\textwidth} }
            \caption[Magyar nyelv és irodalom 11-12.]{Magyar nyelv és irodalom tantárgy tanulási eredményei féléves bontásokban a 11-12. évfolyamszinteken. }  \\

            \textbf{Félév} & \textbf{Tanulási Eredmény} \\
            \hline
            \endhead
                                
                                          1 &  Rendszeresen használja a könyvtárat, a különféle (pl. informatikai technológiákra épülő) információhordozókat, rendelkezik a képességgel, hogy kellő problémaérzékenységgel, kreativitással és önállósággal igazodjon el az információk világában; értelmesen és értékteremtően él az önképzés lehetőségeivel. \\ \hline
                                          1 &  Írása olvasható és rendezett. \\ \hline
                                          1 &  Szóbeli és írásbeli kommunikációs helyzetekben megválasztja a megfelelő hangnemet, nyelvváltozatot, stílusréteget. Alkalmazza a művelt köznyelv (regionális köznyelv), illetve a nyelvváltozatok nyelvhelyességi normáit, képes felismerni és értelmezni az attól eltérő nyelvváltozatokat. \\ \hline
                                          1 &  Alkalmazza az idézés szabályait és etikai normáit. \\ \hline
                                          1 &  Szöveghűen, tudatos, kifejező szövegmondással memoritereket ad elő. \\ \hline
                                      
                                
                                          2 &  Bizonyítja különféle szövegek megértését a szöveg felépítésére, grammatikai jellemzőire, témahálózatára, tagolására irányuló elemzéssel. \\ \hline
                                          2 &  Felismeri a szépirodalmi és nem szépirodalmi szövegekben megjelenített értékeket, erkölcsi kérdéseket, álláspontokat, motivációkat, magatartásformákat, értelmezi és önálló értékeli ezeket. \\ \hline
                                          2 &  Erkölcsi kérdéseket, döntési helyzeteket megnevez, példával bemutat. Aktív résztvevője az elemző beszélgetéseknek, saját véleményét is belefűzve. Értelmezi a felismert jelenségeket, következtetések fogalmaz meg. \\ \hline
                                          2 &  Definíciót nyújt, magyarázatot, értekezést (kisértekezést) készít az olvasmányaival, a felvetett  és tárgyalt problémákkal összefüggésben. Önállóan megfogalmaz kérdéseket, problémákat. \\ \hline
                                          2 &  Irodalmi művekben megjelenő álláspontokat azonosít, megvitat, összehasonlítja más álláspontokkal. Képes ezzel kapcsolatban eltérő vélemények megértésére, újrafogalmazására. \\ \hline
                                      
                                
                                          3 &  A hivatalos írásművek műfajaiban önállóan alkot szöveget (pl. önéletrajz, motivációs levél). \\ \hline
                                          3 &  Ismeri a magyar nyelv rendszerét és történetét, önállóan felismeri és alkalmazza a grammatikai, szövegtani, jelentéstani, stilisztikai-retorikai, helyesírási jelenségeket. \\ \hline
                                          3 &  Tájékozott az olvasott, feldolgozott lírai alkotások különböző műfajaiban, poétikai megoldásaiban, kompozíciós eljárásaiban. \\ \hline
                                          3 &  Ismerteti a feldolgozott epikai, lírai és drámai művek jelentését, erkölcsi tartalmát. \\ \hline
                                          3 &  Írásban és szóban ismertet alkotói pályaképeket az alkotói életút jelentős tényeinek, a művek tematikai, formabeli változatosságának bemutatásával. \\ \hline
                                          3 &  Olvasottsága kiterjed az online médiára és a populáris regiszter jegyében született internetes és nyomtatott sajtóra, alkotásokra is. \\ \hline
                                          3 &  Tisztában van azzal, hogy az irodalmi művek a modernségben elsősorban nyelvi képződmények („fikciók”), amelyek grammatikai-poétikai összetettségük minőségétől függően fejtenek ki esztétikai hatást, hoznak létre örömérzést a lélekben. \\ \hline
                                      
                                
                                          4 &  Felismeri és értő módon használja a tömegkommunikációs, illetve az audiovizuális, informatikai alapú szövegeket. Az értő, kritikus befogadáson kívül önállóan alkot szöveget  publicisztikai, audiovizuális és informatikai hátterű műfajban, a képi elemek, lehetőségek és a szöveg összekapcsolásában rejlő közlési lehetőségek kihasználásával. \\ \hline
                                          4 &  Szövegelemzési, szövegértelmezési jártassággal rendelkezik a tanult leíró nyelvtani, szövegtani, jelentéstani, pragmatikai ismeretek alkalmazásában és az elemzést kiterjeszti a szépirodalmi szövegek mellett a szakmai-tudományos, publicisztikai, közéleti (audiovizuális, informatikai alapú) szövegek feldolgozására, értelmezésére is. \\ \hline
                                          4 &  Ismeri a nyelv és a társadalom viszonyát, illetve a nyelvi állandóság és annak változásának folyamatát. Anyanyelvi műveltségének fontos összetevője a tájékozottság a magyar nyelv eredetéről, rokonságáról, történetének főbb korszakairól; a magyar nyelv és a magyar művelődés kapcsolatának tudatosítása. \\ \hline
                                          4 &  Felismeri és értelmezi szövegek kapcsolatait és különbségeit (pl. tematikus, motivikus kapcsolatok, utalások, nem irodalmi és irodalmi szövegek, tények és vélemények összevetése), alkalmazza ezeket a képsességeket elemző szóbeli és írásbeli műfajokban. \\ \hline
                                          4 &  Alkalmazza a művek műfaji természetének, poétikai jellemzőinek megfelelő szövegfeldolgozási eljárásokat, megközelítési módokat. \\ \hline
                                          4 &  A magyar és a világirodalom kiemelkedő alkotóiról való tudásáról különféle szempontok alapján írásban számot ad.
 \\ \hline
                                          4 &  Bemutatja a tanult irodalomtörténeti korszakok és stílusirányzatok sajátosságait. \\ \hline
                                          4 &  Műveket, alkotókat mutat be a 20. század magyar és világirodalmából, valamint a kortárs irodalomból. \\ \hline
                                          4 &  Ismeri a különböző alkotók hatását az irodalmi hagyományban és felismeri az összefüggéseket egyes művek között. Bemutatja az intertextualitás példáit (evokáció, allúzió, parafrázis, palimpszesz). \\ \hline
                                          4 &  Különböző korokban keletkezett alkotásokat értelmez és összevet tematika és poétikai szempontok alapján. \\ \hline
                                          4 &  Saját, kreatív vagy funkcionális szövegeket hoz létre. \\ \hline
                                          4 &  Jár színházba, moziba, kiállításra, hangversenyre; tudatosan választ a klasszikus és a kortárs drámairodalom elérhető előadásaiból, illetve filmes adaptációiból, egyéb művészeti alkotásaiból. \\ \hline
                                      
                        \end{longtable}
            \clearpage

        \section{Matematika}

       
           \begin{longtable}{c | p{0.8\textwidth} }
            \caption[Matematika 1-2.]{Matematika tantárgy tanulási eredményei féléves bontásokban a 1-2. évfolyamszinteken. }  \\

            \textbf{Félév} & \textbf{Tanulási Eredmény} \\
            \hline
            \endhead
                                
                                          1 &  Több, kevesebb, ugyanannyi fogalmát használja. \\ \hline
                                          1 &  Néhány elemet sorbarendez próbálgatással vagy más módszerrel. \\ \hline
                                          1 &  Páros és páratlan számokat felismeri. \\ \hline
                                          1 &  Képesek irányokat felismerni és megkülönböztetni egymástól. \\ \hline
                                      
                                
                                          2 &  Számokat elhelyezi a számegyenesen. Számszomszédokat meghatározza. Természetes számokat nagyság szerint összehasonlítja. \\ \hline
                                          2 &  Matematikai jeleket: +, –, •, :, =, <, >, ( ) ismeri és használja. \\ \hline
                                          2 &  Növekvő és csökkenő számsorozatok szabályait felismeri, tudja a sorozatot folytatni. \\ \hline
                                      
                                
                                          3 &  Római számokat ismeri, és helyesen írja, olvassa (I, V, X). \\ \hline
                                          3 &  Szöveges feladatokat fel tud írni számokkal és matematikai jelekkel, bizonyos esetekben rajz segítségével. Az ismeretlen szimbólumot kiszámítja. \\ \hline
                                          3 &  Vonalak (egyenes, görbe) fogalmát ismeri és alkalmazza. \\ \hline
                                      
                                
                                          4 &  Számokat helyesen írja, olvassa (100-as számkör). Alaki és helyi értéket ismeri, és helyesen használja. \\ \hline
                                          4 &  Szóban és írásban összead, kivon, szoroz és oszt a százas számkörben. \\ \hline
                                          4 &  A műveletek sorrendjére vonatkozó megállapodásokat ismeri, és helyesen alkalmazza. \\ \hline
                                          4 &  Képesek megkülönböztetni egymástól a testet és a síkidomot. \\ \hline
                                          4 &  Képes hosszúságot, űrtartalmat, tömeget és időt mérni. Ismeri a szabvány mértékegységeket: cm, dm, m, cl, dl, l, dkg, kg, perc, óra, nap, hét, hónap, év. Képes átváltani szomszédos mértékegységek között. Felismeri a mennyiségek közötti összefüggéseket és képes mérőeszközöket használni. \\ \hline
                                          4 &  Halmazokat összehasonlít, halmazokat alkot az elemek száma szerint. \\ \hline
                                          4 &  Állításokról eldönti, hogy igaz vagy hamis. Állításokat megfogalmaz. \\ \hline
                                      
                        \end{longtable}
            \clearpage

       
           \begin{longtable}{c | p{0.8\textwidth} }
            \caption[Matematika 3-4.]{Matematika tantárgy tanulási eredményei féléves bontásokban a 3-4. évfolyamszinteken. }  \\

            \textbf{Félév} & \textbf{Tanulási Eredmény} \\
            \hline
            \endhead
                                
                                          1 &  Képes a matematika különböző területein észszerű becslést végezni, és kerekítést alkalmazni. \\ \hline
                                          1 &  Képes fejben számolni a százas számkörben. \\ \hline
                                          1 &  A szorzótáblát teljes biztonsággal használja a 100-as számkörben. \\ \hline
                                          1 &  Egyszerű sorozatokat képes folytatni. A növekvő és a csökkenő számsorozatokat felismeri. \\ \hline
                                          1 &  Összefüggéseket keres az egyszerű sorozatok elemei között. \\ \hline
                                          1 &  Néhány elemmel megadott egyszerű sorozatok szabályát és a sorozat ismeretlen elemeit képes megadni. \\ \hline
                                      
                                
                                          2 &  Összeg, különbség, szorzat, hányados fogalmát ismeri. Az összeg tagjainak és a szorzat tényezőinek felcserélhetőségét ismeri és képes alkalmazni. A műveleti sorrendre vonatkozó megállapodásokat ismeri és képes alkalmazni. \\ \hline
                                          2 &  Szöveges feladat: a szöveget értelmezi, az adatokat kigyűjti, megoldási tervet készít, becslést végez, ellenőrzi az eredményt, és az eredmény realitását is vizsgálja. \\ \hline
                                          2 &  Többszörös, osztó, maradék fogalmát ismeri és használja. \\ \hline
                                          2 &  Hosszúságot, távolságot és időt mér (egyszerű gyakorlati példák). \\ \hline
                                          2 &  A háromszöget, négyzetet, téglalapot, sokszöget, kört felismeri, és képes létrehozni egyszerű módszerekkel. Ismeri ezeknek a síkidomoknak a jellemzőit. \\ \hline
                                          2 &  Érti a test és a síkidom közötti különbséget. \\ \hline
                                      
                                
                                          3 &  A számokat helyesen írja és olvassa (10 000-es számkör). Alaki és helyi értéket ismeri és helyesen használja a 10 000-es számkörben. \\ \hline
                                          3 &  Negatív számokat használja a mindennapi életben (hőmérséklet, adósság). \\ \hline
                                          3 &  Törteket használ a mindennapi életben. 2, 3, 4, 10, 100 nevezőjű törteket képes megnevezni és előállítani hajtogatással, nyírással, rajzzal, színezéssel. \\ \hline
                                          3 &  A természetes számokat nagyság szerint összehasonlítja a 10 000-es számkörben. \\ \hline
                                          3 &  Képes fejben számolni 10 000-ig nullákra végződő egyszerű esetekben. \\ \hline
                                          3 &  Négyjegyű számokat képes írásban összeadni, kivonni, szorozni egy- és kétjegyű számmal, valamint osztani egyjegyű számmal. \\ \hline
                                          3 &  Az elvégzett műveleteket ellenőrzi. \\ \hline
                                          3 &  Felnőtt segítséggel használ az életkorának megfelelő oktatási célú programokat. \\ \hline
                                      
                                
                                          4 &  Adott tulajdonságú elemeket halmazba rendez. \\ \hline
                                          4 &  Egy adott halmazba tartozó elemek közös tulajdonságait felismeri és megnevezi. \\ \hline
                                          4 &  Próbálgatással megtalálja az összes esetet. \\ \hline
                                          4 &  Vizsgálja az egyenesek kölcsönös helyzetét, felismeri a metsző és párhuzamos egyeneseket. \\ \hline
                                          4 &  Ismeri a szabvány mértékegységeket,például mm, km, ml, cl, hl, g, t, másodperc. Képes átváltásokat végezni a szomszédos mértékegységek között. \\ \hline
                                          4 &  A kockát, téglatestet, gömböt felismeri, és képes létrehozni egyszerű módszerekkel. Ismeri ezeknek a testeknek a jellemzőit. \\ \hline
                                          4 &  Tükrös alakzatokat  előállít hajtogatással, nyírással, rajzzal, színezéssel. Felismeri a tengelyes szimmetriát. \\ \hline
                                          4 &  Négyzet, téglalap kerületét méri és kiszámítja. \\ \hline
                                          4 &  Négyzet, téglalap területét méri és kiszámítja különféle egységekkel, területlefedéssel. \\ \hline
                                          4 &  Tapasztalati adatokat lejegyez, táblázatba rendez. A táblázat adatait értelmezi. \\ \hline
                                          4 &  Adatokat gyűjt és lejegyez. Diagramot készít, illetve olvas. \\ \hline
                                          4 &  Valószínűségi játékokat, kísérleteket értelmezi, és a biztos, lehetetlen, lehet, de nem biztos kimeneteleket felismeri. \\ \hline
                                          4 &  Ismer egy rajzoló programot: egyszerű ábrákat készít, színez.
 \\ \hline
                                      
                        \end{longtable}
            \clearpage

       
           \begin{longtable}{c | p{0.8\textwidth} }
            \caption[Matematika 5-6.]{Matematika tantárgy tanulási eredményei féléves bontásokban a 5-6. évfolyamszinteken. }  \\

            \textbf{Félév} & \textbf{Tanulási Eredmény} \\
            \hline
            \endhead
                                
                                          1 &  Méréseket végez, melynek során használja a mértékegységeket, és azok egyszerű átváltásait. \\ \hline
                                          1 &  Egyszerű grafikonokat készít, elemez. \\ \hline
                                          1 &  Ismeri a térelemek, félegyenes, szakasz, szögtartomány, sík fogalmát. \\ \hline
                                          1 &  Ismeri és egyszerű feladatokban alkalmazza az alapszerkesztéseket: pont és egyenes távolságának, két párhuzamos egyenes távolságának megszerkesztése, szakaszfelező merőleges, szögfelező, merőleges és párhuzamos egyenesek szerkesztése, valamint szögek másolása. \\ \hline
                                          1 &  Egyszerű diagramokat értelmez és  készít. Képes táblázatokat leolvasni.
 \\ \hline
                                      
                                
                                          2 &  Racionális számokat helyesen írja, olvassa, számegyenesen ábrázolja, valamint  egymással nagyságuk szerint összehasonlítja. \\ \hline
                                          2 &  Képes felismerni és felírni számok ellentettjét, abszolút értékét, reciprokát. \\ \hline
                                          2 &  A hosszúság, terület, térfogat, űrtartalom, idő, tömeg szabványmértékegységeit ismeri. Mértékegységek egyszerűbb átváltásait gyakorlati feladatokban alkalmazza. Algebrai kifejezéseket használ gyakorlati problémákban felmerülő  terület, kerület, felszín és térfogat számítása során. \\ \hline
                                          2 &  Kiszámítja a téglalap és a deltoid kerületét és területét.
 \\ \hline
                                          2 &  Kiszámítja a téglatest felszínét és térfogatát. \\ \hline
                                          2 &  A tanult testek térfogatszámítási módjának ismeretében gyakorlati példákban felmerülő egyszerű testek térfogatát, űrmértékét meghatározza. \\ \hline
                                          2 &  Néhány szám számtani közepét kiszámítja. \\ \hline
                                      
                                
                                          3 &  Szöveges feladatokat megold következtetéssel (az adatok közötti összefüggéseket szimbólumokkal felírja). \\ \hline
                                          3 &  Képes megbecsülni a műveletek eredményeit, és ellenőrzi a kapott eredmények helyességét. \\ \hline
                                          3 &  Számok osztóit, többszöröseit felírja. Kiválasztja a közös osztókat, közös többszörösöket. Ismeri és alkalmazza az oszthatósági szabályokat (2, 3, 4, 5, 8, 9, 10, 100). \\ \hline
                                          3 &  Koordinátákkal megadott pontot koordináta-rendszerben ábrázol, valamint koordináta-rendszerben megadott pont koordinátáit leolvassa. \\ \hline
                                          3 &  A geometriai ismeretek segítségével a feltételeknek megfelelő ábrákat körző és vonalzó célszerű használatával pontosan szerkeszti. \\ \hline
                                          3 &  Egyszerű alakzatok tengelyes tükörképét megszerkeszti, tengelyes szimmetriát felismeri. \\ \hline
                                          3 &  A tanult síkbeli és térbeli alakzatok tulajdonságait ismeri, és egyszerű feladatok megoldásában alkalmazza. \\ \hline
                                      
                                
                                          4 &  Elemeket halmazba rendez adott tulajdonság alapján, részhalmazokat felismer, és az elemeit megadja. \\ \hline
                                          4 &  Két véges halmaz közös részét és egyesítését megadja, ábrázolja. \\ \hline
                                          4 &  Néhány elemet különféle módszerekkel sorba rendez. \\ \hline
                                          4 &  Állítások igazságát képes eldönteni, igaz és hamis állításokat megfogalmaz. \\ \hline
                                          4 &  Néhány elem összes sorrendjét képes felírni. \\ \hline
                                          4 &  A mindennapi életben felmerülő egyszerű arányossági feladatokat képes megoldani következtetéssel. Felismeri és alkalmazza az egyenes arányosságot. \\ \hline
                                          4 &  Két-három műveletet és zárójelet is tartalmazó műveletsor eredményét képes kiszámítani. Ismeri és alkalmazza a műveleti sorrendre és a zárójel használatára vonatkozó szabályokat. \\ \hline
                                          4 &  Ismeri a százalék fogalmát, és képes kiszámítani a százalékértéket. \\ \hline
                                          4 &  Elsőfokú egyismeretlenes egyenleteket, egyenlőtlenségeket képes megoldani (szabadon választott módszerrel). \\ \hline
                                          4 &  Néhány elemmel megadott egyszerű sorozatok szabályát felismeri, és a sorozat ismeretlen elemeit képes megadni. Egyszerű sorozatokat adott szabály szerint tud folytatni. \\ \hline
                                          4 &  Valószínűségi játékok, kísérletek során adatokat tervszerűen gyűjt, rendez, ábrázol. \\ \hline
                                      
                        \end{longtable}
            \clearpage

       
           \begin{longtable}{c | p{0.8\textwidth} }
            \caption[Matematika 7-8.]{Matematika tantárgy tanulási eredményei féléves bontásokban a 7-8. évfolyamszinteken. }  \\

            \textbf{Félév} & \textbf{Tanulási Eredmény} \\
            \hline
            \endhead
                                
                                          1 &  Mérésekkel kapcsolatos egyszerű feladatokban a különböző mértékegységeket használja, és képes a mértékegységek szükséges és helyes átváltására. Az egyenes arányosságot és a fordított arányosságot felismeri, és egyszerű feladatokban képes az alkalmazásukra.
 \\ \hline
                                          1 &  A százalékszámítás alapfogalmait ismeri, a tanult összefüggéseket alkalmazza a  feladatmegoldás során. \\ \hline
                                          1 &  Kiválasztja a legnagyobb közös osztót az összes osztóból; a legkisebb pozitív közös többszöröst pedig a többszörösök közül. \\ \hline
                                          1 &  Ismeri a prímszám, összetett szám, prímtényezős felbontás fogalmát; az összetett számoknak felírja a prímtényezős alakját. \\ \hline
                                          1 &  A számológépet tudatosan használja a számolás megkönnyítésére. \\ \hline
                                      
                                
                                          2 &  Biztosan számol a racionális számok körében. A műveletek sorrendjére és a zárójelezésre vonatkozó szabályokat ismeri és helyesen alkalmazza. A várható eredményt megbecsüli, a kapott  eredményt ellenőrzi, valamint képes a feladat követelményeinek megfelelő, helyes kerekítésre. \\ \hline
                                          2 &  Ábrákat készít és szerkeszt geometriai ismeretei alapján. \\ \hline
                                          2 &  Egyszerű geometriai alakzatok tulajdonságait ismeri (háromszögek, négyszögek belső és külső szögeinek összege, nevezetes négyszögek szimmetriatulajdonságai), ezeket alkalmazza a feladatok megoldásában. \\ \hline
                                          2 &  Tengelyes és középpontos tükörkép, eltolt alakzat képét megszerkeszti. Kicsinyítést és nagyítást felismeri (szerkesztés nélkül). \\ \hline
                                          2 &  Háromszögek, speciális négyszögek és a kör kerületét, területét kiszámítja, és tudását különböző feladatokban alkalmazza. \\ \hline
                                          2 &  Egyszerűbb testek (háromszög és négyszög alapú egyenes hasáb, forgáshenger) térfogatát, űrtartalmát a térfogatképletek ismeretében kiszámítja. \\ \hline
                                      
                                
                                          3 &  Egyszerű algebrai egész kifejezések helyettesítési értékét meghatározza. Az összevonásokat elvégzi; többtagú kifejezéseket egytagúval szoroz. \\ \hline
                                          3 &  Ismeri a négyzetre emelés, négyzetgyökvonás fogalmát, valamint a hatványozás fogalmát pozitív egész kitevők esetén. Ezeket a fogalmakat egyszerű feladatokban alkalmazza. \\ \hline
                                          3 &  Megold elsőfokú egyismeretlenes egyenleteket és egyenlőtlenségeket. Következtetéssel vagy egyenlettel megold a matematikából és a mindennapi életből vett egyszerű szöveges feladatokat; a kapott megoldást ellenőrzi és számegyenesen ábrázolja. \\ \hline
                                          3 &  A betűkifejezéseket és az azokkal végzett műveleteket alkalmazza matematikai, természettudományos és hétköznapi feladatok megoldásában. \\ \hline
                                          3 &  A Pitagorasz-tételt ismeri és alkalmazza különböző, (köztük valós helyzeteket modellező) alakzatok ismeretlen adatainak kiszámítására. \\ \hline
                                      
                                
                                          4 &  Elemeket halmazokba rendez több szempont alapján. A halmazokat ábrázolja. \\ \hline
                                          4 &  Egyszerű állítások igaz vagy hamis voltát eldönti, képes állítások tagadására. \\ \hline
                                          4 &  Könnyen értelmezhető (és-sel, vagy-gyal összekötött, ha...akkor...típusú) összetett állítások igaz vagy hamis voltát eldönti. \\ \hline
                                          4 &  Kombinatorikai feladatokat az összes eset szisztematikus összeszámlálásával, és ahol lehetséges, fagráfok alkalmazásával képes megoldani. \\ \hline
                                          4 &  Fagráfokat használ feladatmegoldás során. \\ \hline
                                          4 &  Megadott sorozatokat folytat adott szabály szerint. \\ \hline
                                          4 &  Az egyenes arányosság grafikonját felismeri, a lineáris kapcsolatot alkalmazza természettudományos feladatokban is. \\ \hline
                                          4 &  A grafikonokat elemzi különböző szempontok szerint (növekedés, fogyás, monotonitás stb.), grafikonokat készít, grafikonokról adatokat leolvas. Táblázatok adatait értelmezi és ábrázolja különböző típusú grafikonon. \\ \hline
                                          4 &  Valószínűségi kísérletek eredményeit lejegyzi, relatív gyakoriságokat kiszámítja. \\ \hline
                                          4 &  Konkrét feladatokban érti az esély, illetve valószínűség fogalmát, felismeri a biztos és a lehetetlen eseményeket. \\ \hline
                                          4 &  Számológépet célszerűen használja statisztikai számításokban. \\ \hline
                                          4 &  Néhány kiemelkedő (köztük magyar) matematikus nevét, tevékenységét ismeri. \\ \hline
                                      
                        \end{longtable}
            \clearpage

       
           \begin{longtable}{c | p{0.8\textwidth} }
            \caption[Matematika 9-10.]{Matematika tantárgy tanulási eredményei féléves bontásokban a 9-10. évfolyamszinteken. }  \\

            \textbf{Félév} & \textbf{Tanulási Eredmény} \\
            \hline
            \endhead
                                
                                          1 &  Ismeri a halmazokkal kapcsolatos alapfogalmakat, a halmazműveleteket. Szemlélteti a halmazokat és a halmazműveleteket. Fontosabb számhalmazokat ismeri. \\ \hline
                                          1 &  Egész kitevőjű hatványok fogalmát és azonosságait ismeri. Egyszerű algebrai kifejezésekkel műveleteket végez. Ismereteit matematikai problémák megoldásában (egyenlet, egyenlőtlenség felírása szöveg alapján, egyenletek megoldása, képletek értelmezése) használja. \\ \hline
                                          1 &  Ismeri és alkalmazza a függvények különböző megadási módjait.Tisztában van az értelmezési tartomány és az értékkészlet fogalmával. Valós függvények alaptulajdonságait ismeri. \\ \hline
                                          1 &  Az alapfüggvényeket képes koordináta-rendszerben ábrázolni, tulajdonságait ismeri és feladatokban alkalmazza. \\ \hline
                                          1 &  Az egyszerű függvénytranszformációkat ismeri, tisztában van a grafikonjukra gyakorolt hatásával. \\ \hline
                                          1 &  Valós folyamatokat elemez a folyamathoz tartozó függvény grafikonja alapján. Lineáris függvény esetén ismeri a meredekség fogalmát és szerepét valós folyamatokban. \\ \hline
                                          1 &  Függvénymodellt készít az egyenes arányosságú kapcsolatok ábrázolására; ismeri a meredekség fogalmát. \\ \hline
                                          1 &  Ábrázolja az elemi függvényeket koordináta-rendszerben, és a legfontosabb függvénytulajdonságokat meghatározza, nem csak a matematika, hanem a természettudományos tárgyak megértése kapcsán, és a különböző gyakorlati helyzetek leírásának érdekében is. \\ \hline
                                      
                                
                                          2 &  Ismeri az alapvető térelemeket, a távolság és a szög fogalmát. Távolságokat és szögeket mér. \\ \hline
                                          2 &  Nevezetes ponthalmazokat felismer, és szerkeszt ilyen halmazokat. \\ \hline
                                          2 &  Alakzatok szimmetriáját felismeri, és egyszerű feladatokban alkalmazza. \\ \hline
                                          2 &  Ismeri a háromszögek tulajdonságait (alaptulajdonságok, nevezetes vonalak, pontok, körök). \\ \hline
                                          2 &  Hegyesszögek szögfüggvényeinek fogalmát ismeri.
Elvégez a derékszögű háromszögre visszavezethető (gyakorlati) számításokat Pitagorasz-tétellel és a hegyesszögek szögfüggvényeivel. A magasságtételt és a befogótételt ismeri és egyszerű feladatokban alkalmazza. \\ \hline
                                          2 &  Szimmetrikus négyszögek tulajdonságait ismeri. \\ \hline
                                          2 &  Képes a háromszögekről tanultak alapján számítási feladatokat elvégezni, és ezeket gyakorlati problémák megoldásában alkalmazni. \\ \hline
                                      
                                
                                          3 &  Ismeri és a mindennapi nyelvezetben is használja a matematikai logika alapfogalmait. \\ \hline
                                          3 &  Elsőfokú, másodfokú egyismeretlenes egyenleteket megold; ilyen egyenletre vezető szöveges és gyakorlati feladatokhoz az egyenletet felírja, megoldja, és a megoldást ellenőrzi. \\ \hline
                                          3 &  Egyszerű elsőfokú, másodfokú egyismeretlenes egyenletrendszereket megold; ilyen egyenletrendszerre vezető szöveges és gyakorlati feladatokhoz az egyenletrendszert felírja, megoldja, és a megoldást ellenőrzi. \\ \hline
                                          3 &  Egyismeretlenes egyszerű másodfokú egyenlőtlenséget megold. \\ \hline
                                      
                                
                                          4 &  Definíciót és tételt, állítást és megfordítását felismeri; bizonyítás gondolatmenetét egyszerű esetekben követi. \\ \hline
                                          4 &  Egyszerű leszámlálási feladatokat megold, a megoldás gondolatmenetét képes elmondani és írásban is rögzíteni. \\ \hline
                                          4 &  Ismeri a gráffal kapcsolatos alapfogalmakat. Alkalmazza a gráfokról tanult ismereteit gondolatmenet szemléltetésére, probléma megoldására. \\ \hline
                                          4 &  Jártas a valós számkörben; a gyakorlati és az elvontabb feladatokban alkalmazza a valós számkör műveleteit. \\ \hline
                                          4 &  Matematikai szöveget értően olvas, tankönyveket, keresőprogramokat célirányosan használ, és a szövegekből kiemeli a lényeget. \\ \hline
                                          4 &  Fontos egybevágósági és hasonlósági transzformációk fogalmát  és lényeges tulajdonságait ismeri és feladatokban alkalmazza. \\ \hline
                                          4 &  Az egybevágó és a hasonló alakzatokat felismeri. Két egybevágó, illetve két hasonló alakzat különböző szempontok (például távolságok, szögek, kerület, terület, térfogat) szerinti összehasonlítására képes. Ezeket az ismereteket egyszerű feladatokban alkalmazza. \\ \hline
                                          4 &  A vektor fogalmát és a vektorok közti műveleteket (vektorok összeadása, kivonása, vektor szorzása valós számmal) ismeri. Ismeri a vektorkoordináták fogalmát adott koordináta-rendszerben. \\ \hline
                                          4 &  Adathalmazt megadott szempontok szerint rendezi, adat gyakoriságát és relatív gyakoriságát kiszámítja. \\ \hline
                                          4 &  Táblázatot értelmez és készít; diagramot értelmez és készít. \\ \hline
                                          4 &  Adathalmaz móduszának, mediánjának, átlagának fogalmát ismeri, és egyszerű feladatokban alkalmazza. \\ \hline
                                          4 &  Véletlen esemény, biztos esemény, lehetetlen esemény, esély/valószínűség fogalmát ismeri, és egyszerű feladatokban alkalmazza. \\ \hline
                                          4 &  Nagyszámú véletlen kísérletet tud kiértékelni, összeveti az előzetesen „jósolt” esélyeket és a relatív gyakoriságokat. \\ \hline
                                      
                        \end{longtable}
            \clearpage

       
           \begin{longtable}{c | p{0.8\textwidth} }
            \caption[Matematika 11-12.]{Matematika tantárgy tanulási eredményei féléves bontásokban a 11-12. évfolyamszinteken. }  \\

            \textbf{Félév} & \textbf{Tanulási Eredmény} \\
            \hline
            \endhead
                                
                                          1 &  Egyszerű kombinatorikai problémákat önállóan választott módszerrel megold. \\ \hline
                                          1 &  Ismeri a gráf fogalmát. Problémamegoldások során képes gráfokat alkalmazni. \\ \hline
                                          1 &  A kiterjesztett gyök- és hatványfogalmat ismeri. \\ \hline
                                          1 &  A logaritmus fogalmát ismeri. \\ \hline
                                          1 &  Konkrét esetekben probléma megoldása céljából  alkalmazza a gyök, a hatvány és a logaritmus azonosságait. \\ \hline
                                      
                                
                                          2 &  Egyszerű (köztük a mindennapok gyakorlatában szereplő problémák megoldására alkalmazható) exponenciális és logaritmusos egyenletet fel tud írni szöveg alapján, az egyenletet megoldja és önállóan ellenőrzi. \\ \hline
                                          2 &  A számológépet tudatosan használja feladatmegoldásokban. \\ \hline
                                          2 &  Hosszúságot, szöget, kerületet, területet, felszínt és térfogatot egyszerű esetekben kiszámít. \\ \hline
                                          2 &  A geometriai és algebrai módszerek közötti kapcsolatot a koordinátageometriai ismeretek kapcsán felfogja. Egyszerű esetekben képes koordináta-rendszerben megadott alakzatokra vonatkozó távolságot, szöget kiszámítani. Ismeri a kör és egyenes néhány koordinátageometriai egyenletét. Ezeket az ismereteit alkalmazva egyszerű geometriai feladatokat megold algebrai módszerrel. \\ \hline
                                      
                                
                                          3 &  Alkalmaz, szerkeszt gráfokat gyakorlati, a mindennapokból vett problémák alkotta feladatok megoldásai során. \\ \hline
                                          3 &  Ismeri a trigonometrikus függvényeket; egyszerű esetekben felismeri és megrajzolja a függvény grafikonját (függvénytranszformációk alkalmazásával is). Egyszerű (főleg gyakorlati példákban) alkalmazza az ismereteit. \\ \hline
                                          3 &  Ismeri az exponenciális függvényt és a logaritmusfüggvényt. Egyszerű (főleg gyakorlati példákban) alkalmazza az ismereteit. \\ \hline
                                          3 &  Az ismert függvények jellemzése alapján képes megfogalmazni a fontosabb függvénytulajdonságokat és a függvények fontos felhasználási lehetőségeit. \\ \hline
                                      
                                
                                          4 &  A bizonyított és nem bizonyított állítást egyszerű esetekben meg tudja különböztetni. \\ \hline
                                          4 &  Egyszerű következtetésekben felismeri a feltételt és a következményt. \\ \hline
                                          4 &  A szövegben található információkat önállóan kiválasztja, értékeleli, rendezi problémamegoldás céljából. \\ \hline
                                          4 &  Állítások tagadását képes megfogalmazni. Egyszerű esetekben összetett (és-sel, vagy-gyal összekötött, ha..., akkor... típusú) állítások igaz vagy hamis voltát meg tudja állapítani. \\ \hline
                                          4 &  Ismeri a számtani és a mértani sorozat egyszerű összefüggéseit. Egyszerű (főleg gyakorlati példákban) alkalmazza az ismereteit. \\ \hline
                                          4 &  Ismeri az egyszerű geometriai tételeket, alkalmazni tudja feladatmegoldásokban, és egyszerű valós problémákban megtalálja a megfelelő geometriai modellt. \\ \hline
                                          4 &  Két vektor skaláris szorzatának fogalmát ismeri.  Egyszerű (főleg gyakorlati példákban) alkalmazza az ismereteit. \\ \hline
                                          4 &  Vektorokat ábrázol koordináta-rendszerben, ismeri a helyvektor és a vektorkoordináták fogalmát. Egyszerű (főleg gyakorlati példákban) alkalmazza az ismereteit. \\ \hline
                                          4 &  A statisztikai mutatókat alkalmazza adathalmaz elemzésében. \\ \hline
                                          4 &  Ismeri a valószínűség matematikai fogalmát, és a valószínűség klasszikus kiszámítási módját. Egyszerű (főleg gyakorlati példákban) alkalmazza az ismereteit. \\ \hline
                                          4 &  Mintavételre vonatkozó egyszerű feladatokban a valószínűséget kiszámítja. \\ \hline
                                          4 &  A mindennapok gyakorlatában előforduló egyszerű valószínűségi problémákat tudja értelmezni és kezelni. \\ \hline
                                          4 &  Megfelelő kritikával fogadja a statisztikai vizsgálatok eredményeit, egyszerű esetekben látja a vizsgálatok korlátait, érvényességi körét. \\ \hline
                                      
                        \end{longtable}
            \clearpage

        \section{Mozgóképkultúra és médiaismeret}

       
           \begin{longtable}{c | p{0.8\textwidth} }
            \caption[Mozgóképkultúra és médiaismeret 9-10.]{Mozgóképkultúra és médiaismeret tantárgy tanulási eredményei féléves bontásokban a 9-10. évfolyamszinteken. }  \\

            \textbf{Félév} & \textbf{Tanulási Eredmény} \\
            \hline
            \endhead
                                
                                          2 &  Felismeri és megnevezi a mozgóképi közlésmód, az írott sajtó és az online kommunikáció szövegszervező alapeszközeit. \\ \hline
                                          2 &  Életkorának megfelelően megkülönbözteti a fikció és a virtuális fogalmait egymástól.  \\ \hline
                                          2 &  Átlátja az internetes és a mobilkommunikáció fontosabb sajátosságait, az internethasználat biztonságának problémáit. \\ \hline
                                          2 &  Életkorának megfelelően önállóan összegyűjti, rendszerezi és megfigyeli a különböző  médiumokból és médiumokról szóló ismereteket. \\ \hline
                                          2 &  Alkalmazza a mozgóképi szövegekkel, a média működésével kapcsolatos ismereteit a műsorválasztás során.  \\ \hline
                                      
                                
                                          3 &  A mozgóképi szövegeket megkülönbözteti a valóság ábrázolásához való viszonyuk, az alkotói szándék és a nézői elvárás karaktere szerint (dokumentumfilm és játékfilm, műfaji és szerzői beszédmód). \\ \hline
                                      
                                
                                          4 &  Alkalmazza az ábrázolás megismert eszközeit egyszerű mozgóképi szövegek értelmezése során (cselekmény és történet megkülönböztetése, szemszög, nézőpont, képkivágat, kameramozgás jelentése az adott szövegben, montázsfunkciók felismerése). \\ \hline
                                          4 &  Ismeri a média alapfunkcióit, a kommunikáció alapfordulatait, megfogalmazza, mitől teljesül valamely kor és társadalom nyilvánossága. \\ \hline
                                          4 &  Ismeri a kereskedelmi és közszolgálati médiaintézmények elsődleges céljait és eszközeit a médiaipari versenyben. \\ \hline
                                          4 &  Megkülönbözteti azokat a fontosabb tényezőket, melyek alapján jellemezhetőek a médiaintézmények célközönségei. \\ \hline
                                          4 &  Meghatározza és alkalmas példákkal illusztrálja a sztereotípia és a reprezentáció fogalmát; indokolja az egyszerűbb reprezentációk különbözőségeit. \\ \hline
                                          4 &  Érvekkel támasztja alá álláspontját olyan vitában, amik a médiaszövegek (pl. reklám, hírműsor) valóságtartalmáról folynak. \\ \hline
                                      
                        \end{longtable}
            \clearpage

        \section{Technika, életvitel és gyakorlat}

       
           \begin{longtable}{c | p{0.8\textwidth} }
            \caption[Technika, életvitel és gyakorlat 1-2.]{Technika, életvitel és gyakorlat tantárgy tanulási eredményei féléves bontásokban a 1-2. évfolyamszinteken. }  \\

            \textbf{Félév} & \textbf{Tanulási Eredmény} \\
            \hline
            \endhead
                                
                                          1 &  Érti a család szerepének, időbeosztásának és egészséges munkamegosztásának jelentőségét, nem alkot sztereotípiákat ezek kapcsán. \\ \hline
                                          1 &  Példákat mond az egészséges, korszerű táplálkozás és a célszerű öltözködés terén. \\ \hline
                                      
                                
                                          2 &  Tudatosan kezeli és egészséges veszélyérzettel közelít a háztartási és a közlekedési veszélyekhez. \\ \hline
                                          2 &  A hétköznapjainkban használatos anyagokat felismeri, tulajdonságaikat megállapítja megfigyelések és vizsgálatok alapján. A tapasztalatokat megfogalmazza. \\ \hline
                                          2 &  Mintakövetéssel, önállóan épít. \\ \hline
                                          2 &  Az elvégzett munkáknál biztonságosan, balesetmentesen használ eszközöket. Munka közben rendet tart. \\ \hline
                                      
                                
                                          3 &  Képlékeny anyagokat, papírt, faanyagokat, fémhuzalt, szálas anyagokat, textileket magabiztosan alakít. \\ \hline
                                      
                                
                                          4 &  A természeti, a társadalmi és a technikai környezet megismert jellemzőit felsorolja. \\ \hline
                                          4 &  Tapasztalattal rendelkezik az ember természetátalakító (építő és romboló) munkájáról. \\ \hline
                                          4 &  Az anyagalakításhoz kapcsolódó foglalkozásokat megnevezi, jellemzőit ismeri. \\ \hline
                                          4 &  Az úttesten való átkelés szabályait tudatosan alkalmazza. A kulturált és balesetmentes járműhasználat (tömegközlekedési eszközökön és személygépkocsiban történő utazás) szabályait alkalmazza. \\ \hline
                                      
                        \end{longtable}
            \clearpage

       
           \begin{longtable}{c | p{0.8\textwidth} }
            \caption[Technika, életvitel és gyakorlat 3-4.]{Technika, életvitel és gyakorlat tantárgy tanulási eredményei féléves bontásokban a 3-4. évfolyamszinteken. }  \\

            \textbf{Félév} & \textbf{Tanulási Eredmény} \\
            \hline
            \endhead
                                
                                          1 &  Egyszerű tárgyakat készít mintakövetéssel. \\ \hline
                                      
                                
                                          2 &  A gyalogosokra vonatkozó közlekedési szabályokat tudatosan és készségszinten alkalmazza. \\ \hline
                                          2 &  Biciklizik. \\ \hline
                                      
                                
                                          3 &  A hétköznapjainkban használatos anyagokat felismeri, tulajdonságaikat megállapítja megfigyelések és vizsgálatok alapján. A tapasztalatokat megfogalmazza. \\ \hline
                                          3 &  A Munkaeszközöket célszerűen megválasztja és szakszerűen, balesetmentesen használja. \\ \hline
                                      
                                
                                          4 &  Mindennapokban nélkülözhetetlen praktikus ismereteket – háztartási praktikákat – gyakorol és alkalmaz. \\ \hline
                                          4 &  A használati utasításokat érti és betartja. \\ \hline
                                      
                        \end{longtable}
            \clearpage

       
           \begin{longtable}{c | p{0.8\textwidth} }
            \caption[Technika, életvitel és gyakorlat 5-6.]{Technika, életvitel és gyakorlat tantárgy tanulási eredményei féléves bontásokban a 5-6. évfolyamszinteken. }  \\

            \textbf{Félév} & \textbf{Tanulási Eredmény} \\
            \hline
            \endhead
                                
                                          1 &  A vasúti közlekedésben képes biztonságosan és udvariasan részt venni. \\ \hline
                                      
                                
                                          2 &  Vannak tapasztalatai az ételkészítéssel, élelmiszerekkel összefüggő munkatevékenységekről. \\ \hline
                                          2 &  A gyalogos és kerékpáros közlekedés KRESZ szerinti szabályait, valamint a tömegközlekedés szabályait ismeri és biztonságosan alkalmazza. \\ \hline
                                      
                                
                                          3 &  Ételkészítés és tárgyalkotás során a technológiákat helyesen alkalmazza, az eszközöket szakszerűen, biztonságosan használja. \\ \hline
                                          3 &  Képes tájékozódni közúti és vasúti menetrendekben, útvonaltérképeken. Útvonaltervet olvas, készít. \\ \hline
                                      
                                
                                          4 &  Megfogalmazza tapasztalatait a környezet elemeiről, állapotáról, belátja a környezetátalakító tevékenységgel járó felelősséget. \\ \hline
                                          4 &  Elemi műszaki rajzi ismereteit alkalmazza a tervezés és a kivitelezés során. \\ \hline
                                          4 &  Képes az elkészült produktumok (ételek, tárgyak, modellek) reális értékelésére, a hibák felismerésére, a javítás, fejlesztés lehetőségeinek meghatározására. \\ \hline
                                          4 &  Az ember közvetlen tárgyi környezetének megőrzésére, alakítására vonatkozó szükségleteket felismeri. A tevékenységek és beavatkozások következményeit előzetesen, helyesen felismeri, az azzal járó felelősséget belátja. \\ \hline
                                          4 &  A tárgyi környezetben végzett tevékenységeket biztonságossá, környezettudatossá, takarékossá és célszerűvé formálja. \\ \hline
                                          4 &  Kerékpárját karban tudja tartani. \\ \hline
                                      
                        \end{longtable}
            \clearpage

       
           \begin{longtable}{c | p{0.8\textwidth} }
            \caption[Technika, életvitel és gyakorlat 7-8.]{Technika, életvitel és gyakorlat tantárgy tanulási eredményei féléves bontásokban a 7-8. évfolyamszinteken. }  \\

            \textbf{Félév} & \textbf{Tanulási Eredmény} \\
            \hline
            \endhead
                                
                                          1 &  Környezettudatosan kezeli a háztartási hulladékokat. \\ \hline
                                          1 &  A kerékpárosokra vonatkozó közlekedési szabályokat tudatosan, készségszinten alkalmazza. \\ \hline
                                          1 &  Tájékozott a közlekedési környezetben. \\ \hline
                                          1 &  Közlekedési magatartása tudatos. \\ \hline
                                          1 &  A közlekedési morált alkalmazza. \\ \hline
                                          1 &  Közlekedésszemlélete környezettudatos. \\ \hline
                                      
                                
                                          2 &  Elköteleződik a takarékos életvitel és a környezetkímélő technológiák mellett. \\ \hline
                                          2 &  Tájékozott a továbbtanulási lehetőségekről. Van elképzelése a saját felnőttkori életéről. Mérlegeli a pályaválasztási lehetőségeket. \\ \hline
                                          2 &  A meglátogatott munkahelyeken szerzett tapasztalatok, ismeretek alapján véleményt alkot, ezeket összeveti a személyes terveivel. \\ \hline
                                      
                                
                                          3 &  A víz- és energiafogyasztással, hulladékokkal kapcsolatos mennyiségeket és költségeket érzékeli és jól meg tudja becsülni. \\ \hline
                                          3 &  Keresi az összhangot az adottságok, képességek, igények és lehetőségek között. \\ \hline
                                      
                                
                                          4 &  Az egészséges, biztonságos, környezettudatos otthon működtetéséhez szükséges praktikus életvezetési ismereteit fejleszt, az ehhez szükséges készségeket kialakítja. \\ \hline
                                          4 &  Biztonságosan, takarékosan és felelősen kezeli a háztartás elektromos, víz-, szennyvíz-, gáz- és más tüzelőberendezéseit. A használattal járó veszélyekkel és környezeti hatásokkal tisztában van, felismeri a hibákat, működészavarokat. Képes egyszerű karbantartási, javítási munkák önálló elvégzésére. \\ \hline
                                          4 &  A munkatevékenységet az önmegvalósítás részeként értékeli. \\ \hline
                                          4 &  A munkába álláshoz szükséges alapkészségeket és ismereteket elsajátítja. \\ \hline
                                      
                        \end{longtable}
            \clearpage

       
           \begin{longtable}{c | p{0.8\textwidth} }
            \caption[Technika, életvitel és gyakorlat 11-12.]{Technika, életvitel és gyakorlat tantárgy tanulási eredményei féléves bontásokban a 11-12. évfolyamszinteken. }  \\

            \textbf{Félév} & \textbf{Tanulási Eredmény} \\
            \hline
            \endhead
                                
                                          1 &  Az élő és a tárgyi környezet kapcsolatából, kölcsönhatásainak megfigyeléséből származó tapasztalatokat használ fel a problémamegoldások során, a tevékenységek gyakorlásakor. \\ \hline
                                          1 &  A használati utasításokat érti és betartja. \\ \hline
                                          1 &  Tudatos vásárló. A fogyasztóvédelem szerepét, a vásárlók jogait ismeri. \\ \hline
                                          1 &  Szociálisan érzékeny. A fogyatékkal élőket és az időseket segíti. Karitatív tevékenységeket végez. \\ \hline
                                          1 &  A biztonságos, balesetmentes, udvarias közlekedés szabályait betartja. \\ \hline
                                          1 &  Magabiztosan tájékozódik közvetlen és tágabb környezetében. \\ \hline
                                          1 &  A közlekedési szabályokat és a közlekedési etikát alkalmazza. \\ \hline
                                          1 &  Alkalmazza a balesetvédelem alapvető szabályait, felismeri és elhárítja a veszélyhelyzeteket, elsősegélyt nyújt. \\ \hline
                                      
                                
                                          2 &  Megszerez olyan gyakorlati tudást, amelynek birtokában könnyen eligazodhat a mindennapi élet számos területén. \\ \hline
                                          2 &  A mindennapokban nélkülözhetetlen életvezetési és háztartási ismeretek, „háztartási praktikák” ismeretében a napi munkát szakszerűen, hatékonyan, gazdaságosan végzi el. \\ \hline
                                          2 &  A korszerű pénzkezelés eszközeit ismeri. \\ \hline
                                          2 &  Elkötelezett a munka és az aktivitás iránt. \\ \hline
                                          2 &  Hisz az egész életen át tartó tanulásban. Érvényesíti szaktudását. \\ \hline
                                      
                                
                                          3 &  Hivatalos ügyekben érdekeit képviseli. Kulturált stílusban szolgáltatóknál, ügyfélszolgálatoknál ügyintéz. \\ \hline
                                          3 &  Tisztában van személyes ambícióival, képességeivel; mérlegeli az objektív lehetőségeket és döntést hoz a saját életpályájára vonatkozóan. \\ \hline
                                      
                                
                                          4 &  Folyamatosan fejleszti önismeretét, megoldáscentrikusan kezeli a konfliktusokat,  pozitív az életszemlélete. \\ \hline
                                          4 &  Pénzügyek kezelésében felelősen gondolkodik. Átgondolt döntéseket hoz a fogyasztási javak használatában, a szolgáltatások igénybevételével és a jövővel kapcsolatban. \\ \hline
                                          4 &  Felismeri a munkakultúra szerepét és a személyes kapcsolatok fontosságát az álláskeresés és a munkahely-megtartás során. \\ \hline
                                      
                        \end{longtable}
            \clearpage

        \section{Természetismeret}

       
           \begin{longtable}{c | p{0.8\textwidth} }
            \caption[Természetismeret 5-6.]{Természetismeret tantárgy tanulási eredményei féléves bontásokban a 5-6. évfolyamszinteken. }  \\

            \textbf{Félév} & \textbf{Tanulási Eredmény} \\
            \hline
            \endhead
                                
                                          1 &  Ismeri a Föld helyét a Világegyetemben, Magyarország helyét Európában. \\ \hline
                                          1 &  Törekszik a természeti és társadalmi értékek védelmére. \\ \hline
                                          1 &  Nyitott, érdeklődő a világ megismerése iránt. Az internet és a könyvtár segítségével bővíti tudását. Saját ismeretszerzési, ismeretfeldolgozási módszereket használ. \\ \hline
                                      
                                
                                          2 &  Ismeri hazánk legjellemzőbb életközösségeit, termesztett növényeit, a házban és ház körül élő állatait. Érti az élő és élettelen környezeti tényezők kölcsönhatását. Felismeri a környezet-szervezet-életmód, valamint a szervek felépítése és működése közötti összefüggéseket. \\ \hline
                                          2 &  Tud tájékozódni a térképeken. Értelmezi a különböző tartalmú térképek jelrendszerét, és felhasználja az információszerzés folyamatában. \\ \hline
                                      
                                
                                          3 &  Átfogó képe van a hazai tájaink természetföldrajzi jellemzőiről, természeti-társadalmi erőforrásairól, gazdasági folyamatairól, környezeti állapotukról. \\ \hline
                                          3 &  Felismeri a szűkebb és tágabb környezetében az emberi tevékenység környezeti hatásait. Anyag- és energiatakarékos életvitelével, tudatos vásárlási szokásaival önmaga is hozzájárul a fenntartható fejlődéshez. \\ \hline
                                          3 &  Egyszerű kísérleteket, megfigyeléseket, méréseket önállóan, illetve. csoportban biztonságosan elvégez, a tapasztalatokat rögzíti, következtetéséket von le. \\ \hline
                                      
                                
                                          4 &  Felismeri a fizikai és kémiai változásokat anyagok kölcsönhatását követően. Értelmezi a jelenséget az energiaváltozás szempontjából. \\ \hline
                                          4 &  Ismeri az emberi szervezet felépítését, működését, serdülőkori változásait és okait. Tudatos az egészséget veszélyeztető hatásokkal kapcsolatban. \\ \hline
                                          4 &  A családi és a társas kapcsolatok jelentőségével tisztában van, társaival empatikus és segítőkész. \\ \hline
                                      
                        \end{longtable}
            \clearpage

        \section{Testnevelés és sport}

       
           \begin{longtable}{c | p{0.8\textwidth} }
            \caption[Testnevelés és sport 1-2.]{Testnevelés és sport tantárgy tanulási eredményei féléves bontásokban a 1-2. évfolyamszinteken. }  \\

            \textbf{Félév} & \textbf{Tanulási Eredmény} \\
            \hline
            \endhead
                                
                                          1 &  Az alapvető tartásos és mozgásos elemeket felismeri és pontosan végrehajtja. \\ \hline
                                          1 &  Megnevezi a testrészeket. \\ \hline
                                          1 &  Ismeri és betartja a mozgás órák rendszabályait, tisztában van a balesetvédelmi szempontokkal. \\ \hline
                                          1 &  Felismeri a sporthelyzetek megoldásából és a játékfolyamatból adódó örömöt, élményt és a tanulási lehetőséget. \\ \hline
                                          1 &  Ismeri a sportszerű viselkedés néhány jellemzőjét. \\ \hline
                                      
                                
                                          2 &  Mozgásos gyakorlás közben odafigyel a társaira, célszerűen használja az eszközöket. \\ \hline
                                          2 &  Játékszabályokat és a játékszerepeket haszál játékos feladatok során. \\ \hline
                                          2 &  Biztonságosan használja a sporteszközöket. \\ \hline
                                          2 &  A feladat-végrehajtások során pontosságra, célszerűségre, biztonságra törekszik. \\ \hline
                                          2 &  Ismeri a természeti környezetben történő sportolás néhány egészségvédelmi és környezettudatos viselkedési szabályát. \\ \hline
                                          2 &  A saját komfortérzetének és a közösség szabályainak megfelelően öltözködik. Ismeri és felméri az időjárás hatását az emberi szervezetre. \\ \hline
                                      
                                
                                          3 &  Felismeri a fizikai terhelés és a fáradás jeleit. \\ \hline
                                          3 &  Különbséget tesz a jó és a rossz testtartás között álló és ülő helyzetben. \\ \hline
                                          3 &  Alkalmazza a stressz és feszültségoldás alapgyakorlatait. \\ \hline
                                          3 &  Örömmel használ sporteszközöket, vesz részt a szabad és gyakorló mozgásban. \\ \hline
                                          3 &  Dinamikus és statikus egyensúlyi helyzetekben talajon és emelt eszközökön stabil. \\ \hline
                                          3 &  A futó-, ugró- és dobóiskolai alapgyakorlatokat ismeri és végrehajtja. \\ \hline
                                          3 &  Az általános uszodai rendszabályokat, baleset-megelőzési szempontokat ismeri, és betartja. \\ \hline
                                          3 &  Hason és háton siklik, lebeg. \\ \hline
                                          3 &  Bátran vízbeugrik. \\ \hline
                                      
                                
                                          4 &  A zenei ritmust különféle ritmikus mozgásokban egyénileg, párban és csoportban is követi. \\ \hline
                                          4 &  Tánc közben párhoz, társakhoz térben alkalmazkodik. \\ \hline
                                          4 &  Felismeri a csapatérdek szerepét az egyéni érdekkel szemben, vagyis a közös cél fontossága tudatosul. \\ \hline
                                          4 &  Képes késleltetni és gátolni a mozgását. \\ \hline
                                          4 &  Kreatívan használja a sporteszközöket a játéktevékenység során. \\ \hline
                                      
                        \end{longtable}
            \clearpage

       
           \begin{longtable}{c | p{0.8\textwidth} }
            \caption[Testnevelés és sport 3-4.]{Testnevelés és sport tantárgy tanulási eredményei féléves bontásokban a 3-4. évfolyamszinteken. }  \\

            \textbf{Félév} & \textbf{Tanulási Eredmény} \\
            \hline
            \endhead
                                
                                          1 &  Egyszerű, általános bemelegítő gyakorlatokat hajt végre önállóan. \\ \hline
                                          1 &  A játékok, versenyek során személyes felelősséget vállal a magatartási szabályrendszer betartásában és a sportszerű viselkedés terén. \\ \hline
                                          1 &  Alapvető hely- és helyzetváltoztató mozgásokat folyamatosan és magabiztosan hajt végre. \\ \hline
                                          1 &  Ígényli a pontosságot, a célszerűséget és biztonságot. \\ \hline
                                          1 &  Követi a tempóváltozásokat. \\ \hline
                                          1 &  A futó-, ugró- és dobóiskolai gyakorlatok vezető műveleteit ismeri, azokat végre tudja hajtani. \\ \hline
                                          1 &  Érték számára a sportszerű viselkedés. \\ \hline
                                          1 &  Kezeli a saját indulatát a sportjáték során. Nem viselkedik agresszívan. \\ \hline
                                          1 &  Érti az önvédelmi feladatok célját. \\ \hline
                                          1 &  Egy úszásnemben 25 métert biztonságosan leúszik. \\ \hline
                                          1 &  Legalább négy szabadidős mozgásforma alapszabályait ismeri. \\ \hline
                                      
                                
                                          2 &  A nyújtó, erősítő, ernyesztő és légzőgyakorlatok pozitív hatásait ismeri. \\ \hline
                                          2 &  Ismer stressz- és feszültségoldó gyakorlatokat. \\ \hline
                                          2 &  Rendszeresen használ sporteszközöket szabadidős céllal. \\ \hline
                                          2 &  Gurulások, átfordulások, fordulatok, dinamikus kar-, törzs- és lábgyakorlatok közben többnyire biztosan uralja egyensúlyi helyzetét. \\ \hline
                                          2 &  Különböző intenzitású és tartamú mozgásokat tart fenn játékos körülmények között, illetve játékban. \\ \hline
                                          2 &  Sportszerű küzdésre, asszertív viselkedésre törekszik. \\ \hline
                                      
                                
                                          3 &  Alkalmaz bonyolultabb játékfeladatokat, játékszerepeket és játékszabályokat. \\ \hline
                                          3 &  Ismeri a tanult táncok, dalok, játékok eredeti közösségi funkcióját. \\ \hline
                                          3 &  Tartósan fut egyéni tempóban, akár járások közbeiktatásával is. \\ \hline
                                          3 &  Törekszik a játék legcélszerűbb helyzetmegoldására. \\ \hline
                                          3 &  Bemutat előre-, oldalra- és hátraesést tompítással. \\ \hline
                                      
                                
                                          4 &  Részben önállóan tervezetten 3-6 torna- és/vagy táncelemet összeköt zenére is. \\ \hline
                                          4 &  A tanult sportjátékok alapszabályait ismeri. \\ \hline
                                          4 &  A csapatérdeknek megfelelő összjátékra törekszik. \\ \hline
                                          4 &  Néhány önvédelmi fogást bemutat párban. \\ \hline
                                          4 &  Ismeri és betartja a grundbirkózás alapszabályait. \\ \hline
                                          4 &  A szabadidős mozgásformákat önszervező módon használja szabad játéktevékenység során. \\ \hline
                                      
                        \end{longtable}
            \clearpage

       
           \begin{longtable}{c | p{0.8\textwidth} }
            \caption[Testnevelés és sport 5-6.]{Testnevelés és sport tantárgy tanulási eredményei féléves bontásokban a 5-6. évfolyamszinteken. }  \\

            \textbf{Félév} & \textbf{Tanulási Eredmény} \\
            \hline
            \endhead
                                
                                          1 &  Önállóan vesz részt a sportolás során felmerülő szervezési feladatok végrehajtásában. \\ \hline
                                          1 &  Nyolc-tíz gyakorlattal, részben önállóan melegít be. A bemelegítés és a levezetés szempontjait ismeri. \\ \hline
                                          1 &  Figyel a biomechanikailag helyes testtartás megőrzésére. \\ \hline
                                          1 &  Szabálykövetően, fegyelmezetten, együttműködően vesz részt a sportjátékokban. \\ \hline
                                          1 &  A balesetvédelmi utasításokat betartja. \\ \hline
                                          1 &  A sportágak űzéséhez szükséges eszközöket biztonságosan használja. \\ \hline
                                          1 &  A természeti és környezeti hatások és a szervezet alkalmazkodó képessége közötti összefüggéseket felismeri. \\ \hline
                                      
                                
                                          2 &  Stressz- és feszültségoldó módszereket alkalmaz. \\ \hline
                                          2 &  Választott úszásnemben készségszinten, egy másikban 25 méteren vízbiztosan, folyamatosan úszik. \\ \hline
                                          2 &  A tanult akadályleküzdési módokat és feladatokat biztonságosan végrehajtja. \\ \hline
                                          2 &  Ismer alternatív környezetben űzhető sportokat. \\ \hline
                                          2 &  Képes a tanult alternatív környezetben űzhető sportágak alaptechnikai gyakorlatainak bemutatására. \\ \hline
                                          2 &  A természeti környezetben történő sportolás egészségvédelmi és környezettudatos viselkedési szabályait elfogadja és betartja. \\ \hline
                                      
                                
                                          3 &  Alkalmazza a testnevelési játékokban, játékos feladatokban és a sportjátékban is a játékok technikai és taktikai készletét. \\ \hline
                                          3 &  Ugrásoknál a nekifutás távolságát és sebességét kialakítja tapasztalatok felhasználásával. \\ \hline
                                          3 &  Ismeri az atlétikai versenyek alapvető szabályait. \\ \hline
                                          3 &  A dinamikus és statikus egyensúlygyakorlatokat a képességnek megfelelő magasságon, szükség esetén segítségadás mellett hajtja végre. \\ \hline
                                          3 &  A gyakorlatvégzések során előforduló hibákat elismeri és a javítási megoldásokat elfogadja. \\ \hline
                                      
                                
                                          4 &  A rajtokat az indítási jeleknek megfelelően végrehajtja. \\ \hline
                                          4 &  A ritmikus sportgimnasztika egyszerű tartásos és mozgásos gyakorlatelemeinek bemutatása. \\ \hline
                                          4 &  A mostoha időjárási feltételek mellett is aktívan részt vesz a foglalkozásokon. \\ \hline
                                      
                        \end{longtable}
            \clearpage

       
           \begin{longtable}{c | p{0.8\textwidth} }
            \caption[Testnevelés és sport 7-8.]{Testnevelés és sport tantárgy tanulási eredményei féléves bontásokban a 7-8. évfolyamszinteken. }  \\

            \textbf{Félév} & \textbf{Tanulási Eredmény} \\
            \hline
            \endhead
                                
                                          1 &  Ismeri az egyszerű stressz- és feszültségoldó technikákat. \\ \hline
                                          1 &  Ismeri és elkerüli az erősítés és nyújtás néhány ellenjavallt gyakorlatát. \\ \hline
                                          1 &  A kamaszkori személyi higiénéről elemi tájékozottsággal rendelkezik. \\ \hline
                                          1 &  Rendelkezik ismeretekkel a vízben mozgás prevenciós előnyeiről és fizikai hátteréről. \\ \hline
                                          1 &  Képes a vízből mentés alapgyakorlatainak bemutatására. \\ \hline
                                          1 &  A vizes feladatokban kinyilvánítja felelősségérzetét és segítőkészségét. \\ \hline
                                          1 &  Konfliktusok, sportszerűtlenségek, deviáns magatartások esetén a gondolatait, véleményét szóban kulturáltan fejezi ki. \\ \hline
                                          1 &  Az egészséges életmóddal kapcsolatos ismereteire támaszkodik és terjeszti azokat. \\ \hline
                                          1 &  Cselekedeteiben megjelenik a környezettudatosság. \\ \hline
                                          1 &  Fejleszti verbális és nem verbális kommunikációját a testkultúra hagyományos és az újszerű mozgásanyagainak témájába. \\ \hline
                                          1 &  Sportszerűen szeretne győzni és ezt ki is fejezi. \\ \hline
                                      
                                
                                          2 &  Kísérletet tesz az összehangolt, rendezett testtartás kritériumainak való megfelelésre. \\ \hline
                                          2 &  Képes mellúszásban úszni az egyéni adottságainak és képességeinek megfelelően. \\ \hline
                                          2 &  Gazdag sportjáték-technikai és -taktikai készlettel rendelkezik. \\ \hline
                                          2 &  Fejleszti a csapatjátékhoz szükséges együttműködési és kommunikációs képességet. \\ \hline
                                          2 &  Megtapasztalja és elfogadja a sportjátékokhoz tartozó test-test elleni küzdelmet. \\ \hline
                                          2 &  Ismeri a futás, a kocogás élettani jelentőségét. \\ \hline
                                          2 &  Bátran végrehajt szekrény- és a támaszugrásokat, a képességének megfelelő magasságon. \\ \hline
                                          2 &  A rekreációs célú sportágakban és a népi hagyományokra épülő sportolási formákban bővülő gyakorlási tapasztalatot szerez és fellelhető erősebb belső motiváció az általa választott területeken. \\ \hline
                                          2 &  Ismeri a természeti erők és a sport hasznos összekapcsolásának előnyeit, ezen a téren rutinra, jártasságra tesz szert. \\ \hline
                                          2 &  Pozitívan viszonyul a szabadidőben végzett sportoláshoz. \\ \hline
                                      
                                
                                          3 &  A tanult két úszásnemben mennyiségi és minőségi teljesítményjavulást képes felmutatni. \\ \hline
                                          3 &  Jártas néhány taktikai formáció, helyzet megoldásában. \\ \hline
                                          3 &  A játékszabályok kibővített körét is megérti és alkalmazza. \\ \hline
                                          3 &  Alapvető tájékozottsága van a labdajátékok sporttörténete terén. \\ \hline
                                          3 &  Az atlétikai cselekvésmintákat sokoldalúan és célszerűen alkalmazza. \\ \hline
                                          3 &  Futó-, ugró- és dobógyakorlatokat végez a tanult versenyszabályoknak megfelelően. \\ \hline
                                          3 &  Mozgása az atlétikai alapmozgásokban közelít a mozgásmintához. Összekapcsolja a lendületszerzést és a befejező mozgásokat. \\ \hline
                                          3 &  Talajon és a választott tornaszeren növekvő önállóság jeleit mutatja fel a gyakorlásban, gyakorlat-összeállításban. \\ \hline
                                          3 &  Látható fejlődés az aerobikgyakorlatok kivitelében és a zenével összhangban történő végrehajtása során. \\ \hline
                                          3 &  A torna jellegű gyakorlatok végrehajtásában képes az önkontrollra, az együttműködésre és a segítségnyújtásra. \\ \hline
                                          3 &  Vannak ismeretei a fenyegetettségi szituációkra, segítségkérésre, menekülésre vonatkozóan. \\ \hline
                                      
                                
                                          4 &  Gyakorlott a célszerű óraszervezés megvalósításában. \\ \hline
                                          4 &  Képes egyszerű gimnasztikai gyakorlatokat önállóan összefűzni és zenére előadni. \\ \hline
                                          4 &  Mérhető fejlődést ér el képességekben és sportági eredményekben. \\ \hline
                                          4 &  A helyes testtartás, a koordinált mozgás és az erőközlés összhangjára képes a torna jellegű mozgásokban. \\ \hline
                                          4 &  A grundbirkózás alaptechnikáját, szabályait a gyakorlatban alkalmazza. \\ \hline
                                          4 &  A különböző eséstechnikák, szabadulások, leszorítások és az önvédelmi gyakorlatait kontrolláltan, társsal végrehajtja. \\ \hline
                                      
                        \end{longtable}
            \clearpage

       
           \begin{longtable}{c | p{0.8\textwidth} }
            \caption[Testnevelés és sport 9-10.]{Testnevelés és sport tantárgy tanulási eredményei féléves bontásokban a 9-10. évfolyamszinteken. }  \\

            \textbf{Félév} & \textbf{Tanulási Eredmény} \\
            \hline
            \endhead
                                
                                          1 &  Sportjátékot vezet. \\ \hline
                                          1 &  Olyan játékokat választ és játszik, amelyek ápolják a társas kapcsolatokat és alkalmasak bármilyen képességű társa számára. \\ \hline
                                          1 &  A javító kritikát elfogadja és a mozdulatok kivitelezését javítja. Esztétikusan és harmonikusan ad elő. \\ \hline
                                          1 &  A sportolás során használt különféle anyagok, felületek tulajdonságait és a baleseti kockázatokat ismeri. \\ \hline
                                          1 &  Ismer sportversenyszabályokat. \\ \hline
                                          1 &  Bemelegítéssel fizikailag felkészül a sérülésmentes sporttevékenységre. \\ \hline
                                          1 &  Ismeri és alkalmazza a biomechanikailag helyes testtartás jellemzőit, néhány jellemző deformitás kockázatait, a testtartás megőrzésének gyakorlatait. \\ \hline
                                      
                                
                                          2 &  Tudatosan kiválasztja az adott sporthoz leginkább kapcsolódó technikai és taktikai megoldásokat. \\ \hline
                                          2 &  Elemzi a játékfolyamat taktikai megoldásait. Híve a fair és a csapatelkötelezett játéknak. \\ \hline
                                          2 &  Érvényesíti a mozgáselemek mozgásbiztonságának és a gyakorlás mennyiségének, minőségének oksági viszonyait. \\ \hline
                                          2 &  Ismer célszerű gyakorlási és gyakorlásszervezési formációkat, versenyszituációkat, versenyszabályokat. \\ \hline
                                          2 &  Egy kijelölt táv megtételéhez szükséges időt és sebességet helyesen becsli. A becsült értékek alapján a feladatot pontosan végrehajtja. Évfolyamonként önmagához mérten javul futó-, ugró-, dobóteljesítménye. \\ \hline
                                          2 &  Ismeri a gerinc sérüléseinek leggyakoribb fajtáit, a gerinc és az izületek védelmének legfontosabb szempontjait. \\ \hline
                                          2 &  Használ preventív stressz és feszültségoldó gyakorlatokat. Ismeri és méri fittségét, ezzel kapcsolatosan önfejlesztő célokat tűz ki az egészség-edzettség érdekében. \\ \hline
                                          2 &  Ismeri és alkalmazza a rendszeres testmozgás pozitív hatásait a káros szenvedélyek leküzdésében, az érzelem- és a feszültségszabályozásában. \\ \hline
                                      
                                
                                          3 &  Aktívan részt vesz a technikai, taktikai sportfeladatokban, ezek során együttműködik társaival. \\ \hline
                                          3 &  Ismeri az egyes sportok szabályait, alkalmazza azokat. \\ \hline
                                          3 &  Összeállít önálló talaj és/vagy szergyakorlatot, egyszerű aerobik elemkapcsolatot, táncmotívumfüzért. \\ \hline
                                          3 &  A tanult mozgások versenysportja területén, a magyar sportolók sikereiről elemi ismeretei vannak. \\ \hline
                                          3 &  Fejlődik tempóérzéke és odafigyelés a váltófutás gyakorlásában. \\ \hline
                                          3 &  Az adott sportmozgás technikáját elfogadható cselekvésbiztonsággal hajtja végre. \\ \hline
                                          3 &  Ismeri és alkalmazza az alternatív sportmozgások edzés - és balesetvédelmi alapfogalmait. \\ \hline
                                          3 &  Elkerüli a sportok során a veszélyhelyzeteket; uralkodik az indulatai felett. Nem viselkedik agresszíven a sportjátékok során. \\ \hline
                                          3 &  Egy választott úszásnemhez tartozó öt szárazföldi képességfejlesztő gyakorlatot bemutat. \\ \hline
                                          3 &  Szabályosan úszik. \\ \hline
                                          3 &  Egyszerűbb feladatok, ugrások során másokkal szinkronban mozog a vízben. \\ \hline
                                          3 &  Felsorolja a vízből mentés veszélyeit, pontos menetét. Ismerteti a passzív társ vonszolása kisebb távon (4-5 méter) történés menetét. \\ \hline
                                      
                                
                                          4 &  Önvédelmi és küzdősportot űz. \\ \hline
                                          4 &  Betartja az önvédelmi és küzdőgyakorlatokban, harcokban a közös szabályokat, a biztonsági követelményeket és a küzdésekkel kapcsolatos rituálékat. \\ \hline
                                          4 &  Ismer küzdősportok támadási, védekezési és önvédelmi megoldásait, kombinációit. Fogásokból szabadul. \\ \hline
                                          4 &  1000 m-t  úszik a választott technikával, egyéni tempóban, szabályos fordulóval. \\ \hline
                                          4 &  Fejleszti úszóerejét és állóképességét. \\ \hline
                                          4 &  A testsúlya és testtömege ismeretében egészségesen táplálkozik. \\ \hline
                                      
                        \end{longtable}
            \clearpage

       
           \begin{longtable}{c | p{0.8\textwidth} }
            \caption[Testnevelés és sport 11-12.]{Testnevelés és sport tantárgy tanulási eredményei féléves bontásokban a 11-12. évfolyamszinteken. }  \\

            \textbf{Félév} & \textbf{Tanulási Eredmény} \\
            \hline
            \endhead
                                
                                          1 &  Önállóan bemelegít, gyakorol, edz, szervezi a játékot. \\ \hline
                                          1 &  Elfogadja mások eltérő szintű játéktudását. \\ \hline
                                          1 &  Bemelegítő és képességfejlesztő gyakorlatokat ismeri, a célnak megfelelően használja. \\ \hline
                                          1 &  Segítséget ad, biztosít, biztat. \\ \hline
                                          1 &  Észreveszi és kijavítja a hibát, asszertívan kommunikál erről. \\ \hline
                                          1 &  Az atlétikai versenyszámok biomechanikai alapjait ismeri. \\ \hline
                                          1 &  Az alapvető atlétikai versenyszabályokat ismeri. \\ \hline
                                          1 &  Az atlétikai mozgásokhoz illeszkedően melegít be. \\ \hline
                                          1 &  A szabályokat és rituálékat betartja. \\ \hline
                                          1 &  Indulatait, agresszivitását kezeli. Önfegyelmmel rendelkezik. \\ \hline
                                          1 &  A bemelegítés szükségességének élettani okait ismeri. \\ \hline
                                          1 &  Kíméli a gerincét a testnevelési és sportmozgásokban, kerti és házimunkákban, az esetleges sérüléses szituációkat megfelelően kezeli. \\ \hline
                                      
                                
                                          2 &  Ismeri az adott sportjáték versenykörülményeit. \\ \hline
                                          2 &  Alkalmazza a technikai elemeket, taktikai megoldásokat, a sportverseny szabályait. \\ \hline
                                          2 &  Csapaton belül képes a ráoszott szerepet betölteni, alkalmazza a formációkat a sporthelyzetben. \\ \hline
                                          2 &  Ismer kreativitást, együttműködést, tartalmas, asszertív társas kapcsolatokat szolgáló mozgásos játéktípusokat és azokat célszerűen használja. \\ \hline
                                          2 &  Ismeri az izmok mozgáshatárát bővítő aktív és passzív eljárásokat. \\ \hline
                                          2 &  A futások, ugrások és dobások képességfejlesztő hatását felhasználja más mozgásrendszerekben. \\ \hline
                                          2 &  Uralja a testét a sebesség, gyorsulás, tempóváltás, gurulás, csúszás, gördülés esetén. \\ \hline
                                          2 &  Több támadási és védekezési megoldást, kombinációt ismer az álló és földharcban. \\ \hline
                                          2 &  Megtervezi az egészsége fenntartásához szükséges edzést, terhelést. Tudatosan védekezik a stresszes állapot ellen, a feszültségeket kezeli. \\ \hline
                                          2 &  Ismeri és a gyakorlatban használja az izmok erősítését és nyújtását szolgáló gyakorlatokat. \\ \hline
                                      
                                
                                          3 &  Ismereteinek megfelelően objektíven értékeli a csapat taktikai tervét, teljesítményét. \\ \hline
                                          3 &  Felismeri és kialakítja a tornagyakorlat optimális végrehajtására jellemző téri, időbeli és dinamikai sajátosságokat. \\ \hline
                                          3 &  Koordináltan irányítja mozgását bonyolult gyakorlatelem sorok, folyamatok végrehajtása közben. \\ \hline
                                          3 &  Önállóan választ zenét, ami illeszkedik a mozdulatsorhoz. \\ \hline
                                          3 &  Fejleszti az állóképességét a lendületszerzés, az izom-előfeszítések begyakorlásával, a futó-, ugró- és a dobóteljesítmény növelése érdekében. \\ \hline
                                          3 &  Önállóan tervez  és old meg feladatokat alternatív sporteszközökkel. \\ \hline
                                          3 &  Az adott alternatív sportmozgáshoz szükséges edzés és balesetvédelmi alapfogalmakat ismeri. \\ \hline
                                          3 &  Magabiztosan viselkedik veszélyeztetettség esetén, elhárítja a támadást. \\ \hline
                                      
                                
                                          4 &  Tervez, gyakorol, bemutat önállóan összeállított összefüggő gyakorlatokat. \\ \hline
                                          4 &  Könnyeden, plasztikusan, esztétikusan hajtja végre a táncos mozgásformákat. \\ \hline
                                          4 &  Ismeri a versenysport előnyeit, veszélyeit, a versenyzéshez szükségek képességek fejlesztésének lehetőségeit. \\ \hline
                                          4 &  Meglévő ismereteit alkalmazza az új sporttevékenységek során. \\ \hline
                                      
                        \end{longtable}
            \clearpage

        \section{Történelem, társadalmi és állampolgári ismeretek}

       
           \begin{longtable}{c | p{0.8\textwidth} }
            \caption[Történelem, társadalmi és állampolgári ismeretek 5-6.]{Történelem, társadalmi és állampolgári ismeretek tantárgy tanulási eredményei féléves bontásokban a 5-6. évfolyamszinteken. }  \\

            \textbf{Félév} & \textbf{Tanulási Eredmény} \\
            \hline
            \endhead
                                
                                          1 &  Felismeri, hogy a múltban való tájékozódást segítik a kulcsfogalmak és fogalmak, amelyek fejlesztik a történelmi tájékozódás és gondolkodás kialakulását. \\ \hline
                                          1 &  Tudja, hogy a népeket főként vallásuk és kultúrájuk, életmódjuk alapján tudjuk megkülönböztetni. \\ \hline
                                          1 &  Tudja, hogy az eredettörténetek, a közös szokások és mondák erősítik a közösség összetartozását, egyben a nemzeti öntudat kialakulásának alapjául szolgálnak. \\ \hline
                                          1 &  Értelmezi az emberi magatartásformákat, információkat rendszerez és értelmez, vizuális vázlatokat készít. \\ \hline
                                      
                                
                                          2 &  Azonosítja a történelmet formáló alapvető folyamatokat, felismeri az összefüggéseket (pl. a munka értékteremtő ereje) és egyszerű, átélhető erkölcsi tanulságokat azonosít. \\ \hline
                                          2 &  Ismeri az előző korokban élt emberek, közösségek élet-, gondolkodás- és szokásmódjait. \\ \hline
                                          2 &  Felismeri, hogy az utókor a történelmi személyiségek, nemzeti hősök cselekedeteit a közösségek érdekében végzett tevékenységek szempontjából értékeli. \\ \hline
                                          2 &  Felismeri, hogy a vallási előírások, valamint az államok által megfogalmazott szabályok döntő mértékben befolyásolhatják a társadalmi viszonyokat és a mindennapokat. \\ \hline
                                          2 &  Tudja, hogy az emberi munka nyomán elinduló termelés biztosítja az emberi közösségek létfenntartását. \\ \hline
                                          2 &  Az árutermelés és pénzgazdálkodás, illetve a városiasodás kialakulásához vezető társadalmi munkamegosztás jelentőségét felismeri. \\ \hline
                                          2 &  Tudja, hogy a társadalmakban eltérő jogokkal rendelkező és eltérő vagyoni helyzetű emberek alkotnak rétegeket, csoportokat. \\ \hline
                                          2 &  Tudja, hogy a társadalmi, gazdasági, politikai és vallási küzdelmek számos esetben összekapcsolódnak. \\ \hline
                                          2 &  A hallott és olvasott szövegekből, különböző médiumok anyagából következtetéseket von le és történelmi ismeretekre tesz szert. \\ \hline
                                          2 &  Információt gyűjt adott témához könyvtárban és múzeumban; olvasmányairól lényeget kiemelő jegyzetet készít. \\ \hline
                                          2 &  Szóbeli beszámolót készít önálló gyűjtőmunkával szerzett ismereteiről, és Prezit tart róla. \\ \hline
                                          2 &  Ismeri az időszámítás alapelemeit (korszak, évszázad, évezred), ezek alapján kronológiai számításokat végez. Ismeri néhány kiemelkedő esemény időpontját. \\ \hline
                                      
                                
                                          3 &  Történetek feldolgozásánál megkülönbözteti a valós és fiktív elemeket, csoportosítja a szereplőket (fő és mellékszereplőkre), térben és időben elhelyezi őket. Ismeri a a híres történelmi személyiségek jellemzéséhez szükséges kulcsszavakat, cselekedeteket. \\ \hline
                                          3 &  Egyszerű térképeket másol, alaprajzot készít. Megtalál földrajzi és történelmi helyeket térképen, vaktérképen néhány kiemelt történelmi esemény topográfiai helyét megmutatja. Ezekhez kapcsolódóan tud távolságot becsülni és számítani történelmi térképen. \\ \hline
                                      
                                
                                          4 &  Megismeri az egyetemes emberi értékeket, az ókori, középkori és kora újkori egyetemes és magyar kultúrkincsen keresztül. A családhoz, a lakóhelyhez, a nemzethez való tartozás élményét megéli. \\ \hline
                                          4 &  Tudja, hogy a történelmi jelenségeket, folyamatokat társadalmi, gazdasági tényezők együttesen befolyásolják és felismeri ezeket egy-egy történelmi probléma vagy korszak feldolgozása során. \\ \hline
                                          4 &  Különbséget tesz a történelem különböző típusú forrásai között, felismeri a korszakra jellemző képeket, tárgyakat, épületeket. \\ \hline
                                      
                        \end{longtable}
            \clearpage

       
           \begin{longtable}{c | p{0.8\textwidth} }
            \caption[Történelem, társadalmi és állampolgári ismeretek 7-8.]{Történelem, társadalmi és állampolgári ismeretek tantárgy tanulási eredményei féléves bontásokban a 7-8. évfolyamszinteken. }  \\

            \textbf{Félév} & \textbf{Tanulási Eredmény} \\
            \hline
            \endhead
                                
                                          1 &  Felismeri, hogy a modern nemzetállamokat különböző kultúrájú, vallású, szokású, életmódú népek, nemzetiségek együttesen alkotják. \\ \hline
                                          1 &  Felismeri a múltat és a történelmet formáló alapvető folyamatokat, összefüggéseket (pl. a technikai fejlődés hatásai a társadalomra és a gazdaságra) és azonosít egyszerű, átélhető erkölcsi tanulságokat (pl. társadalmi kirekesztés). \\ \hline
                                          1 &  Felismeri, hogy a múltban való tájékozódást támogatják a kulcsfogalmak és fogalmak. E fogalmak segítik a történelmi tájékozódás és gondolkodás kialakulását, fejlődését. \\ \hline
                                          1 &  Felismeri, hogy az utókor a nagy történelmi személyiségek, nemzeti hősök cselekedeteit a közösségek érdekében végzett tevékenységek szempontjából értékeli, példát tud mondani ellentétes értékelésre. \\ \hline
                                          1 &  Megfogalmazza saját véleményét, figyelmbe veszi mások véleményét és reflektál azokra. \\ \hline
                                      
                                
                                          2 &  Azonosítja az új- és modern korban élt emberek, közösségek élet-, gondolkodás- és szokásmódjait, a hasonlóságokat és különbségeket felismeri.
 \\ \hline
                                          2 &  Ismeri a XIX-XX. század nagy korszakainak megnevezését, illetve egy-egy korszak főbb jelenségeit, jellemzőit, szereplőit, összefüggéseit. \\ \hline
                                          2 &  Tudja, hogy Európához köthetők a modern demokratikus viszonyokat megalapozó szellemi mozgalmak és dokumentumok. \\ \hline
                                          2 &  Tudja, hogy Magyarország a trianoni békediktátum következtében elvesztette területének és lakosságának kétharmadát, és közel öt millió magyar került kisebbségi sorba. Jelenleg közel hárommillió magyar nemzetiségű él a szomszédos államokban és a világ különböző részein, akik szintén a magyar nemzet tagjainak tekintendők. \\ \hline
                                          2 &  Tudja, hogy a társadalmakban eltérő jogokkal rendelkező és vagyoni helyzetű emberek alkotnak rétegeket, csoportokat. \\ \hline
                                          2 &  Különbséget tesz a demokrácia és a diktatúra között, s tud azokra példát mondani a feldolgozott történelmi korszakokból és napjainkból. \\ \hline
                                          2 &  Felismeri a hazánkat és a világot fenyegető globális veszélyeket, betegségeket, terrorizmust, munkanélküliséget. \\ \hline
                                          2 &  Különbséget tesz a történelem különböző típusú forrásai között, felismeri az egy-egy korszakra jellemző képeket, tárgyakat, épületeket. \\ \hline
                                          2 &  Tudja, hogy hol kell a fontos események forrásait kutatni, és ha szükséges, összehasonlítja a megismert jelenségeket, eseményeket. \\ \hline
                                          2 &  A hallott és olvasott szövegekből, különböző médiumok anyagából következtetéseket von le és történelmi ismeretekre tesz szert. \\ \hline
                                          2 &  Információt gyűjt adott témához könyvtárban és múzeumban; olvasmányairól lényeget kiemelő jegyzetet készít. \\ \hline
                                          2 &  Könyvtári munkával és az internet kritikus használatával forrást gyűjt, kiselőadást tart és érvel. \\ \hline
                                          2 &  Szóbeli beszámolót készít önálló gyűjtőmunkával szerzett ismereteiről, és kiselőadást tart. \\ \hline
                                      
                                
                                          3 &  Tudja, hogy az egyes népek és államok a korszakban eltérő társadalmi, gazdasági és vallási körülmények között éltek, de a modern kor beköszöntével a köztük lévő kapcsolatok széleskörű rendszere épült ki. \\ \hline
                                          3 &  Tudja, hogy a holokausztnak többszázezer magyar áldozata is volt, és  tisztában van ennek hazai és nemzetközi történelmi, politikai előzményeivel, körülményeivel és erkölcsi vonatkozásaival. \\ \hline
                                          3 &  Felismeri a sajtó és média szerepét a nyilvánosságban, azonosítja a reklám és médiapiac jellegzetességeit. \\ \hline
                                          3 &  Példák segítségével értelmezi az alapvető emberi, gyermek- és diákjogokat, valamint a társadalmi szolidaritás különböző formáit. \\ \hline
                                      
                                
                                          4 &  Az újkori és modern kori egyetemes és magyar történelmi jelenségek, események feldolgozásával a jelenben zajló folyamatok előzményeit felismeri. Kialakul a nemzeti azonosságtudata. \\ \hline
                                          4 &  Ismeri a magyar történelem főbb csomópontjait egészen napjainkig. E hosszú történelmi folyamat meghatározó szereplőit azonosítja, egy-egy korszak főbb kérdéseit felismeri. \\ \hline
                                          4 &  Ismeri a korszak meghatározó egyetemes és magyar történelmének eseményeit, évszámait, történelmi helyszíneit. Képes összefüggéseket találni a térben és időben eltérő fontosabb történelmi események között, különös tekintettel azokra, melyek a magyarságot közvetlenül vagy közvetetten érintik. \\ \hline
                                          4 &  Ismeri a híres történelmi személyiségek jellemzéséhez szükséges adatokat, eseményeket és kulcsszavakat. \\ \hline
                                          4 &  Példák segítségével bemutatja a legfontosabb állampolgári jogokat és kötelességeket, értelmezi ezek egymáshoz való viszonyát. \\ \hline
                                          4 &  Azonosítja a gazdasági és pénzügyi terület fontosabb szereplőit, egyszerű családi költségvetést készít és mérlegeli a háztartáson belüli megtakarítási lehetőségeket. \\ \hline
                                      
                        \end{longtable}
            \clearpage

       
           \begin{longtable}{c | p{0.8\textwidth} }
            \caption[Történelem, társadalmi és állampolgári ismeretek 9-10.]{Történelem, társadalmi és állampolgári ismeretek tantárgy tanulási eredményei féléves bontásokban a 9-10. évfolyamszinteken. }  \\

            \textbf{Félév} & \textbf{Tanulási Eredmény} \\
            \hline
            \endhead
                                
                                          1 &  Ismeri a civilizációk történetének jellegzetes sémáját (kialakulás, virágzás, hanyatlás). \\ \hline
                                          1 &  Felismeri, hogy az utókor a nagy történelmi személyiségek, nemzeti hősök cselekedeteit a közösségek érdekében végzett tevékenységek szempontjából értékeli, példákat hoz különböző korok eltérő értékítéleteiről egy-egy történelmi személyiség kapcsán. \\ \hline
                                          1 &  Áttekinti és értékeli a rendelkezésre álló forrásokat, valamint rendszerezi és értelmezi a szerzett információkat. \\ \hline
                                          1 &  Kérdéseket fogalmaz meg a forrás megbízhatóságára és a szerző esetleges elfogultságára vonatkozóan. \\ \hline
                                          1 &  Írott és hallott szövegből a lényeget kiemeli tételmondatok meghatározásával, szövegek tömörítésével és átfogalmazásával egyaránt. \\ \hline
                                      
                                
                                          2 &  Tudja, hogy a történelmi jelenségeket, folyamatokat társadalmi, gazdasági, szellemi tényezők együttesen befolyásolják. \\ \hline
                                          2 &  Ismeri a keresztény Magyar Királyság létrejöttének, virágzásának és hanyatlásának főbb állomásait, a kora újkor békés építőmunkájának eredményeit, valamint a polgári Magyarország kiépülésének meghatározó gondolatait és kulcsszereplőit. \\ \hline
                                          2 &  Beszámolót, kiselőadást készít és tart különböző írott forrásokat - történelmi kézikönyveket, atlaszokat/szakmunkákat, statisztikai táblázatokat, grafikonokat, diagramokat és az internetet - használva. \\ \hline
                                          2 &  Megfigyeli a különböző magatartástípusokat és élethelyzeteket, ezekből következtetéseket von le. \\ \hline
                                          2 &  Feltevéseket fogalmaz meg történelmi személyiségek cselekedeteinek, viselkedésének mozgatórugóiról. \\ \hline
                                          2 &  Elbeszél és eljátszik történelmi helyzeteket a különböző szereplők nézőpontjából. \\ \hline
                                          2 &  Képes saját véleményét megfogalmazni, és közben meg tudja különböztetni egy vitában a tárgyilagos érvelést és a személyeskedést. \\ \hline
                                          2 &  Használja a történelmi korszakok és periódusok nevét. \\ \hline
                                          2 &  Különböző információforrásokból önálló térképvázlatokat tud rajzolni, különböző időszakok történelmi térképeit össze tudja hasonlítására. Le tudja olvasni a történelmi tér változásait és az adott témához leginkább megfelelő térképet választ. \\ \hline
                                      
                                
                                          3 &  Feltárja a többféleképpen értelmezhető szövegek jelentésrétegeit. \\ \hline
                                          3 &  Képes történelmi témákat vizuálisan ábrázolni. \\ \hline
                                          3 &  Kronológiai adatokat rendszerez, történelmi időszakokat meghatároz kronológiai adatok alapján. \\ \hline
                                      
                                
                                          4 &  Megismeri az ókori, középkori és kora újkori egyetemes és magyar kultúrkincs elemeit és rendszerét, tudatosan vállalja az egyetemes emberi értékeket, felismeri és elfogadja a családhoz, lakóhelyhez, nemzethez, Európához, világhoz tartozás fontosságát. \\ \hline
                                          4 &  Ismeri és felismeri a múltat és a történelmet formáló összetett folyamatokat, a látható és a háttérben meghúzódó összefüggéseket, és azonosítja ezek erkölcsi-etikai aspektusait. \\ \hline
                                          4 &  Azonosítja a korábbi korokban élt emberek, közösségek élet-, gondolkodás- és szokásmódjait, felismeri a különböző államformák működési jellemzőit. \\ \hline
                                          4 &  Ismeri és alkalmazza az árnyalt történelmi tájékozódást és gondolkodást segítő kulcsfogalmakat. \\ \hline
                                          4 &  Az egyes népeket vallásuk és kultúrájuk, életmódjuk alapján azonosítja és ismeri. Felismeri, hogy a vallási előírások, valamint az államok által megfogalmazott szabályok döntő mértékben befolyásolhatják a társadalmi viszonyokat és a mindennapokat. \\ \hline
                                          4 &  Ismeri a világ és az európai kontinens eltérő fejlődési irányait, ezek társadalmi, gazdasági és szellemi hátterét. Tudja azonosítani Európa különböző régióinak eltérő fejlődési útjait. \\ \hline
                                          4 &  Felismeri a meghatározó vallási, társadalmi, gazdasági, szellemi összetevőket egy-egy történelmi jelenség, folyamat értelmezésénél. \\ \hline
                                          4 &  Érti és értelmezi az eltérő uralkodási formák és társadalmi, gazdasági viszonyok közötti összefüggéseket. \\ \hline
                                          4 &  Összehasonlítja a történelmi időszakokat, a változások szempontjából egybeveti az eltérő korszakok emberi sorsait. \\ \hline
                                          4 &  Képes érzékelni és elemezni az egyetemes és a magyar történelem eltérő időbeli ritmusát, illetve ezek kölcsönhatásait. Az egyes korszakokat komplex módon jellemzi. \\ \hline
                                      
                        \end{longtable}
            \clearpage

       
           \begin{longtable}{c | p{0.8\textwidth} }
            \caption[Történelem, társadalmi és állampolgári ismeretek 11-12.]{Történelem, társadalmi és állampolgári ismeretek tantárgy tanulási eredményei féléves bontásokban a 11-12. évfolyamszinteken. }  \\

            \textbf{Félév} & \textbf{Tanulási Eredmény} \\
            \hline
            \endhead
                                
                                          1 &  Árnyaltan gondolkodik a történelemről és ehhez ismeri az értelmezést segítő kulcsfogalmakat. \\ \hline
                                          1 &  Tudja és érti, hogy az utókor, a történelmi emlékezet a nagy történelmi személyiségek tevékenységét többféle módon és szempont szerint értékeli, egyben saját értékítélete megfogalmazásakor a közösség hosszú távú nézőpontját tudja alkalmazni. \\ \hline
                                      
                                
                                          2 &  Ismeri a XIX-XX. század kisebb korszakainak megnevezését, egy-egy korszak főbb jelenségeit, jellemzőit, szereplőit, összefüggéseit. \\ \hline
                                          2 &  Ismeri a magyar történelem főbb csomópontjait az 1848-1849-es szabadságharc leverésétől az Európai Unióhoz való csatlakozásunkig. \\ \hline
                                      
                                
                                          3 &  Az újkori és modern kori egyetemes és magyar történelmi jelenségek, események rendszerező feldolgozásával a jelenben zajló folyamatok előzményeit felismeri, a nemzeti öntudatra és aktív állampolgárságra törekszik. \\ \hline
                                          3 &  Beazonosítja a történelmi folyamat meghatározó összefüggéseit, bemutatja és elemzi egy-egy korszak főbb kérdéseit \\ \hline
                                          3 &  Ismeri az új- és modern korban meghatározó egyetemes és magyar történelem eseményeit, évszámait, történelmi helyszíneit. \\ \hline
                                          3 &  Képes összefüggéseket találni a térben és időben eltérő történelmi események között, különös tekintettel azokra, amelyek a magyarságot közvetlenül vagy közvetetten érintik. \\ \hline
                                          3 &  Tudja, hogy a XIX–XX. században lényegesen átalakult Európa társadalma és gazdasága (polgárosodás, iparosodás), és ezzel párhuzamosan új eszmeáramlatok, politikai mozgalmak, pártok jelennek meg. \\ \hline
                                          3 &  Felismeri, hogy az Egyesült Államok milyen körülmények között vált a mai világ vezető hatalmává, és rá tud mutatni az ebből fakadó ellentmondásokra. \\ \hline
                                          3 &  Tudja a trianoni békediktátum máig tartó hatását, következményeit értékelni, és felismeri a határon túli magyarság sorskérdéseit. \\ \hline
                                          3 &  Tudja a demokratikus és diktatórikus államberendezkedések közötti különbségeket, felismeri és elemzi a demokratikus berendezkedés előnyeit és működési nehézségeit. \\ \hline
                                          3 &  Ismereteket merít különböző forrásokból, történelmi, társadalomtudományi, filozófiai és etikai kézikönyvekből, atlaszokból, szaktudományi munkákból. Történelmi kutatást folytat ezek segítségével. \\ \hline
                                          3 &  Kiselőadásokat, beszámolókat önállóan jegyzetel. \\ \hline
                                          3 &  Kritikusan és tudatosan használja az internetet történelmi, filozófia- és etikatörténeti ismeretek megszerzése érdekében. \\ \hline
                                          3 &  Különböző történelmi elbeszéléseket (pl. emlékiratok) hasonlít össze a narráció módja alapján. \\ \hline
                                          3 &  Vizsgálja és megítéli az egyes szövegeket, hanganyagokat, filmeket a történelmi hitelesség szempontjából. \\ \hline
                                          3 &  Történelmi jeleneteket el tud mesélni, néha el tudja játszani azokat különböző szempontokból. \\ \hline
                                          3 &  Felismeri és értelmezi az erkölcsi kérdéseket felvető élethelyzeteket. \\ \hline
                                          3 &  Sokoldalóan tájékozódik a történelmi időkben. \\ \hline
                                          3 &  Összehasonlítja és feltárja a különböző időszakokat bemutató történelmi térképek változásainak hátterét (területi változások, népsűrűség, vallási megosztottság stb.). \\ \hline
                                          3 &  Ismeri és szakszerűen használja a nemzet, a kisebbség, a nemzetiség és a helyi társadalom fogalmát, fontos számára a társadalmi felelősségvállalás, a szolidaritás. \\ \hline
                                      
                                
                                          4 &  Felismeri a múltat és a történelmet formáló, alapvető folyamatok ok-okozati összefüggéseit (pl. a globalizáció felerősödése és a lokális közösségek megerősödése) és az egyszerű, átélhető erkölcsi tanulságokat (pl. társadalmi kirekesztés) azonosítja, jelenre vonatkoztatja azokat. \\ \hline
                                          4 &  Azonosítja az új- és modern korban élt emberek, közösségek sokoldalú élet-, gondolkodás- és szokásmódjait, a hasonlóságokat és különbségeket árnyaltan felismeri, több szempontból értékeli. \\ \hline
                                          4 &  A civilizációk jellegzetes sémáit alkalmazza az újkori és modern kori egyetemes történelem értelmezése során. \\ \hline
                                          4 &  Felismeri a globalizálódó világ problémáit (pl. túlnépesedés, betegségek, elszegényesedés, munkanélküliség, élelmiszerválság, tömeges migráció). \\ \hline
                                          4 &  Él a globalizáció előnyeivel, benne az európai állampolgársággal. \\ \hline
                                          4 &  Ismeri az alapvető emberi jogokat, valamint állampolgári jogokat és kötelezettségeket, Magyarország politikai rendszerének legfontosabb intézményeit, érti a választási rendszer működését. \\ \hline
                                          4 &  Önálló véleményt fogalmaz meg társadalmi, történelmi eseményekről, szereplőkről, jelenségekről, filozófiai kérdésekről. \\ \hline
                                          4 &  Mások érvelését összefoglalja, értékeli és figyelembe veszi, miközben saját álláspontját gazdagítja. \\ \hline
                                          4 &  A történelmi-társadalmi adatokat, modelleket és elbeszéléseket elemzi a bizonyosság, a lehetőség és a valószínűség szempontjából. \\ \hline
                                          4 &  Összehasonlítja a társadalmi-történelmi jelenségeket strukturális és funkcionális szempontok szerint. Tisztázza saját értékeit, összehasonlítja másokéval. \\ \hline
                                          4 &  Képes történelmi-társadalmi témákat vizuálisan ábrázolni, esszét írni (filozófiai kérdésekről is), ennek kapcsán kérdéseket fogalmaz meg. \\ \hline
                                          4 &  Tudja a nemzetgazdaság, a bankrendszer, a vállalkozási formák működésének legfontosabb szabályait. \\ \hline
                                          4 &  Átlátja a munkavállalással összefüggő, a munkaviszonyhoz kapcsolódó adózási, egészség- és társadalombiztosítási kötelezettségek, illetve szolgáltatások rendszerét. \\ \hline
                                      
                        \end{longtable}
            \clearpage

        \section{Vizuális kultúra}

       
           \begin{longtable}{c | p{0.8\textwidth} }
            \caption[Vizuális kultúra 1-2.]{Vizuális kultúra tantárgy tanulási eredményei féléves bontásokban a 1-2. évfolyamszinteken. }  \\

            \textbf{Félév} & \textbf{Tanulási Eredmény} \\
            \hline
            \endhead
                                
                                          1 &  A felszerelést önállóan rendben tartja. \\ \hline
                                      
                                
                                          2 &  Korának megfelelő felismerhetőségű ábrát készít. \\ \hline
                                          2 &  Közvetlen környezetét megfigyeli és értelmezi. \\ \hline
                                          2 &  A képalkotó tevékenységek közül személyes, kifejező alkotásokat hoz létre. \\ \hline
                                          2 &  A téralkotó feladatok során a személyes térbeli szükségeleteket felismeri. \\ \hline
                                          2 &  Az alkotótevékenység és a látványok, műalkotások szemlélése során felismer és használ néhány formát, színt, vonalat, térbeli helyet és irányt. \\ \hline
                                          2 &  Felismeri a különbségeket a szobor, festmény, tárgy, épület között. \\ \hline
                                      
                                
                                          3 &  Figyel az alkotótevékenységnek megfelelő, rendeltetésszerű és biztonságos anyag- és eszközhasználatra. Figyelembe veszi az alkotómunka megtervezése és kivitelezése során a környezetvédelmi szempontokat. \\ \hline
                                          3 &  Megkülönbözteti a hagyományos kézműves technikával készült tárgyakat. \\ \hline
                                          3 &  Megfigyeli és befogadja a látványokat, műalkotásokat. \\ \hline
                                      
                                
                                          4 &  Azonosítja a médiumokat, tudatosan választ a médiahasználat során, reflektív. \\ \hline
                                          4 &  Felismeri a médiaélmények változását, médiatapasztalattá alakíthatóságát. \\ \hline
                                          4 &  Azonosítja a médiaszövegek néhány elemi kódját (kép, hang, cselekmény), felismeri az ezzel kapcsolatos egyszerű összefüggéseket (pl. médiaszövegek értelmezése, kreatív kifejező eszközök hatása a médiaszövegben). \\ \hline
                                          4 &  Felismeri a személyes kommunikáció és a közvetett kommunikáció közötti alapvető különbségeket. \\ \hline
                                          4 &  Életkorához igazodóan használja az internetet és felismeri az ebben rejlő veszélyeket. \\ \hline
                                          4 &  Felismeri és kifejezi az alkotó és befogadó tevékenység során saját érzéseit. \\ \hline
                                      
                        \end{longtable}
            \clearpage

       
           \begin{longtable}{c | p{0.8\textwidth} }
            \caption[Vizuális kultúra 3-4.]{Vizuális kultúra tantárgy tanulási eredményei féléves bontásokban a 3-4. évfolyamszinteken. }  \\

            \textbf{Félév} & \textbf{Tanulási Eredmény} \\
            \hline
            \endhead
                                
                                          1 &  Életkorának megfelelően értelmezi az alkotó, megfigyelő, és elemző jellegű feladatokat. \\ \hline
                                      
                                
                                          2 &  Rendeltetésszerűen és biztonságosan használja a megismert anyagokat és eszközöket, technikákat az alkotótevékenység során. \\ \hline
                                          2 &  Megfogalmazza a látványok, műalkotások megfigyeléseinek során kialakult gondolatai, érzéseit. \\ \hline
                                      
                                
                                          3 &  Vizuálisan kifejezi élményeit, emlékeit, illusztációt készít, síkbábot és egyszerű jelmezeket alkot, jeleket és ábrákat készít, egyszerű tárgyakat alkot. \\ \hline
                                          3 &  Téralkotó feladatok során felismeri és használja a személyes preferenciáknak és funkcióknak megfelelő térbeli szükségletet. \\ \hline
                                          3 &  Tudja  differenciálni a szobrászati, festészeti, tárgyművészeti, építészeti területek közötti különbségeket (pl. festészeten belül: arckép, csendélet, tájkép). \\ \hline
                                      
                                
                                          4 &  Felismeri a különböző típusú médiaszövegeket, tudatosan választ a médiatartalmak között. \\ \hline
                                          4 &  Felismeri a médiaszövegekhez használt egyszerű kódokat, kreatív kifejezőeszközöket és azok érzelmi hatását. \\ \hline
                                          4 &  Kép- és hangrögzítő eszközöket használ, ezek segítségével saját gondolatokat, érzéseket fogalmaz meg, rövid, egyszerű történeteket formál. \\ \hline
                                          4 &  A médiaszövegek előállításával, nyelvi jellemzőivel, használatával kapcsolatos alapfogalmakat  helyesen alkalmazza élőszóban. \\ \hline
                                          4 &  Ismeri a média alapvető funkcióit (tájékoztatás, szórakoztatás, ismeretszerzés). \\ \hline
                                          4 &  A médiaszövegekben megjelenő információkat kritikusan szemléli,  valóságtartalmát felismeri. \\ \hline
                                          4 &  Az életkorához igazodóan,  biztonságosan használja az internetet és mobiltelefont.  A hálózati kommunikációban való részvétel során betartja a fontos és szükséges viselkedési szabályokat. Életkorhoz igazodóan fejlesztő, kreatív internetes tevékenységeket végez. \\ \hline
                                      
                        \end{longtable}
            \clearpage

       
           \begin{longtable}{c | p{0.8\textwidth} }
            \caption[Vizuális kultúra 5-6.]{Vizuális kultúra tantárgy tanulási eredményei féléves bontásokban a 5-6. évfolyamszinteken. }  \\

            \textbf{Félév} & \textbf{Tanulási Eredmény} \\
            \hline
            \endhead
                                
                                          1 &  A vizuális nyelv és kifejezés eszközeit megfelelően alkalmazza az alkotó tevékenység során a vizuális emlékezet segítségével és megfigyelés alapján. \\ \hline
                                          1 &  Egyszerű kompozíciós alapelveket a kifejezésnek megfelelően használ a képalkotásban. \\ \hline
                                          1 &  Önálló véleményt fogalmaz meg saját és mások munkájáról. \\ \hline
                                      
                                
                                          2 &  Térbeli és időbeli változások lehetséges vizuális megjelenéseit értelmezi, egyszerű mozgásélményeket, időbeli változásokat megjelenít. \\ \hline
                                          2 &  A mindennapokban használt vizuális jeleket értelmezi, ennek analógiájára saját jelzésrendszereket alakít ki. \\ \hline
                                          2 &  Szöveg és kép együttes jelentését értelmezi különböző helyzetekben és alkalmazza különböző alkotó jellegű tevékenység során. \\ \hline
                                          2 &  Egyszerű következtetéseket fogalmaz meg az épített és tárgyi környezet elemző megfigyelése alapján. \\ \hline
                                      
                                
                                          3 &  Alkotótevékenysége során a választott rajzi és tárgykészítési technikát megfelelően használja. \\ \hline
                                          3 &  Vizuális eszközökkel reflektál társművészeti alkotásokra. \\ \hline
                                          3 &  Önállóan kérdez a vizuális megfigyelés és elemzés során. \\ \hline
                                      
                                
                                          4 &  Azonosítja a legfontosabb művészettörténeti korokat. \\ \hline
                                          4 &  Pontosan és szabatosan megfogalmazza megfigyeléseit vizuális jelenségek, tárgyak, műalkotások elemzése során. \\ \hline
                                          4 &  Felismer szimbolikus és kultárális üzenetet közvetítő tárgyakat. \\ \hline
                                      
                        \end{longtable}
            \clearpage

       
           \begin{longtable}{c | p{0.8\textwidth} }
            \caption[Vizuális kultúra 7-8.]{Vizuális kultúra tantárgy tanulási eredményei féléves bontásokban a 7-8. évfolyamszinteken. }  \\

            \textbf{Félév} & \textbf{Tanulási Eredmény} \\
            \hline
            \endhead
                                
                                          1 &  Önállóan alkalmaz célirányos vizuális megfigyelési szempontokat. \\ \hline
                                          1 &  Önállóan kérdez a vizuális megfigyelés és elemzés során. \\ \hline
                                          1 &  Önálló véleményt fogalmaz meg saját és mások munkájáról. \\ \hline
                                      
                                
                                          2 &  A vizuális nyelv és kifejezés eszközeit tudatosan és pontosan alkalmazza az alkotótevékenység során adott célok kifejezése érdekében. \\ \hline
                                          2 &  Térbeli és időbeli változásokat vizuálisan megjelenít. Kifejező vagy közlő szándékot értelmez és következtetéseket fogalmaz meg. \\ \hline
                                          2 &  Mélységükben elemzi, összehasonlítja a vizuális jelenségeket, tárgyakat, műalkotásokat. \\ \hline
                                      
                                
                                          3 &  Bonyolultabb kompozíciós alapelveket használ kölönböző célok érdekében. \\ \hline
                                          3 &  Alapvetően közlő funkcióban lévő képi vagy képi és szöveges megjelenéseket egyszerűen értelmez. \\ \hline
                                          3 &  Az épített és tárgyi környezet elemző megfigyelése alapján összetettebb következtetések megfogalmaz. \\ \hline
                                      
                                
                                          4 &  Több jól megkülönböztethető technikát, médiumot (pl. állókép-mozgókép, síkbeli-térbeli) tudatosan használ alkotótevékenység során. \\ \hline
                                          4 &  A vizuális kommunikációs eszközök és formák rendszerezett megismerése során megalapozza a médiatudatos gondolkodását. \\ \hline
                                          4 &  Felismeri a mozgóképi közlésmód, az írott sajtó és az online kommunikáció szövegszervező alapeszközeit. \\ \hline
                                          4 &  Megkülönböztet mozgóképi szövegeket a valóság ábrázolásához való viszonyuk, az alkotói szándék és nézői elvárás karaktere alapján. \\ \hline
                                          4 &  Önállóan értelmezi a társművészeti kapcsolatok lehetőségeit. \\ \hline
                                          4 &  Megkülönbözteti a legfontosabb kultúrákat, művészettörténeti korokat, stílusirányzatokat. A meghatározó alkotók műveit felismeri. \\ \hline
                                      
                        \end{longtable}
            \clearpage

       
           \begin{longtable}{c | p{0.8\textwidth} }
            \caption[Vizuális kultúra 9-10.]{Vizuális kultúra tantárgy tanulási eredményei féléves bontásokban a 9-10. évfolyamszinteken. }  \\

            \textbf{Félév} & \textbf{Tanulási Eredmény} \\
            \hline
            \endhead
                                
                                          1 &  Önállóan kiválaszt célirányos vizuális megfigyelési szempontokat. \\ \hline
                                          1 &  Önállóan alkalmazza a vizuális nyelv eszközeit az alkotótevékenység és a vizuális jelenségek elemzése, értelmezése során. \\ \hline
                                          1 &  Adott vizuális problémakkal kapcsolatban önálló kérdéseket fogalmaz meg. \\ \hline
                                          1 &  Önálló véleményt fogalmaz meg saját és mások munkájáról. \\ \hline
                                      
                                
                                          2 &  Tudatosan használ bonyolultabb kompozíciós alapelveket különböző célok érdekében. \\ \hline
                                          2 &  Értelmezi a térbeli és időbeli változások vizuális megjelenítését, egyszerű mozgóképeket készít. \\ \hline
                                          2 &  Az alkotótevékenységekben tanult technikákat a célnak megfelelően tudatosan alkalmaz. \\ \hline
                                          2 &  Kreatívan old meg problémákat. \\ \hline
                                      
                                
                                          3 &  Értelmezi a képi, vagy képi és szöveges megjelenéseket. \\ \hline
                                          3 &  Tervez és makettet készít. \\ \hline
                                          3 &  Árnyaltan értelmezi a tárművészeti kapcsolatokat. \\ \hline
                                      
                                
                                          4 &  Tudatosan gondolkodik a médiáról a tömegkommunikációs eszközök és formák rendszerező feldolgozásában. \\ \hline
                                          4 &  Ismeri és rendszerben látja a különböző kultúrákat, művészettörténeti korokat, stílusirányzatokat, felismeri a meghatározó alkotók műveit. \\ \hline
                                          4 &  Felismeri az építészet alapvető elrendezési, szerkezeti alapelveit és egy-egy stílus meghatározó vonásait. \\ \hline
                                          4 &  Elemez, összehasonlít vizuális jelenségeket, tárgyakat, műalkotásokat; alkalmazza a műelemzés módszereit. \\ \hline
                                      
                        \end{longtable}
            \clearpage

    



