\chapter{A tanulás keretei}


\section{Biztonságos tanulási környezet}

A Budapest School a tanulásszervezés folyamatait befolyásolja, annak folyamatos fejlesztését tűzte ki célul. A folyamatok kialakításakor és akkor, amikor ezek a folyamatok valamilyen oknál fogva nem működnek, a következő szempontok élveznek prioritást:

\begin{itemize}
\item A tanulás és az iskolában eltöltött idő mindenki számára legyen biztonságos fizikai és érzelmi szempontból egyaránt.

\item Ha több olyan megoldás van, amely megőrzi mindenki biztonságérzetét, akkor azt választjuk, ami leginkább szolgálja egy iskola tagjainak fejlődését.

\item Ha minden megoldás azonos, akkor azt választjuk, ami esetén a tanulás még inkább magától tud megtörténni, és még jobban tudjuk segíteni a tanulókat önállóan tanulniuk.

\item Ha ez már nem kérdés, akkor az a megoldás a jobb, amely jobban elősegíti a szülők, gyerekek és tanárok együttműködését, kapcsolódását.
\end{itemize}


\section{Tantárgyak}
\label{sec:tantargyak}
A Budapest School a ma gyerekeinek kínál olyan oktatást, ami segíti felkészíteni őket a jövő kihívásaira. Információs társadalmunk legnagyobb kihívása az adaptációs képességünk fejlesztése, ez az alapja annak, hogy képesek legyünk eligazodni a folyamatosan változó, komplex világunkban. A tanulásunk célja, hogy boldog, hasznos és egészséges tagjai legyünk a társadalomnak. Tanulásunk során a tanulás három rétege, a tudásszerzés, a megtanultakat elmélyítő önálló gondolkodás és az aktív alkotás egyszerre jelennek meg. Céljaink eléréséhez három fő tantárgy köré csoportosítjuk a Budapest School moduláris tanulási struktúráját. Tantárgyainkhoz kapcsolódóan specifikált modulok során alkotnak, kísérleteznek és gyakorolnak a Budapest School tanulói. Az egyes tantárgyakhoz kapcsolódnak a gyerekek egyéni tanulási tervei és később az eredményeket bemutató portfóliói, melyek az általuk elvégzett modulok (legkisebb tanulási egységek) eredményeit tartalmazzák.

A tantárgyak fő funkciója, hogy egyensúlyt teremtsenek egyrészt a gyerekek egyéni céljai és a tantárgyspecifikus fejlesztési célok között. Másfelől az, hogy a Budapest School három tantárgya között is folyamatos egyensúly állhasson fel, ezzel a fizikai és lelki jóllét, a tudomány és a humánum területein való fejlődés arányosan történjen.

Kötelező modulokként jelennek meg a Budapest School tantárgyaihoz kapcsolódóan azok a tanulási tartalmak, melyek az érettségi elvégzéséhez szükségesek.

A Budapest School alábbi tantárgyai összegyűjtik és újracsoportosítják a miniszter által kiadott kerettantervekben szereplő tantárgyi specifikációkat, használják azok fejlesztési irányait és eredményelvárásait, de nem térnek ki részletesen a tematikákra, ezzel szabadságot adva a tanároknak arra, hogy a tanmenet tekintetében akár jelentős eltérések legyenek abban az esetben, ha a miniszter által kiadott kerettanterv mérhető tanulási céljai legalább 50\%-ban teljesülnek és azok ugyanilyen arányban kiegészülnek saját célokkal is.

\paragraph{Tantárgyak szerepe a mindennapokban}

A Budapest School iskoláinak tantárgyai a miniszter által kiadott kerettanterv alapján készültek, azok összevonásával, tömörítésével, valamint céljainak újragondolásával. Az egyes tanulási moduloknak a portfólióba való elhelyezését követően háromhavonta összevetjük az egyéni tanulási tervben szereplő saját célokat és a tantárgyi kötelezően választható fejlesztési célokat a modulok eredményeivel, hogy ezáltal folyamatosan monitorozni lehessen az iskolai követelmények és a tanulók egyéni céljai közötti egyensúlyt.

A Budapest School iskoláiban a tantárgyak ugyanúgy kapnak szerepet, mint a Nemzeti Alaptanterv által definiált kulcskompetenciák, fejlesztési területek és műveltségi területek minden iskolában: pedagógus sose mondja azt a tanulóknak, hogy „most kezdeményezőképességet és vállalkozói kompetenciát fejlesztünk", hanem a különböző feladatok elvégzése eredményeképp történik a fejlesztés. Vagyis a tantárgyak a tanulás tartalmi elemeinek ernyőjét adják, a tanulás tartalma az ehhez kapcsolódó egyes modulok konkrét tematikáján keresztül válik láthatóvá. Egy adott modul kiírása és egyéni célba emelése a tanulási tervet mutatja, míg annak portfólióba helyezése a tanulás tényét és a gyerek pilllanatnyi állapotát teszi láthatóvá az adott tanulási egységben.

Kerettantervünk a tanulásnak ezeket a specifikus, moduláris tanulási elemeit emeli és koncentrálja három tantárgy köré tematikusan az interdiszciplináris tudásszerzés, valamint a megértést és az összefüggések keresését segítő gondolkodásmód fejlesztése jegyében. Ahogy a Nemzeti Alaptanterv fogalmaz, „olyan tudástartalmak jelentek meg, amelyek nehezen sorolhatók be a tudományok hagyományos rendszerébe, vagy amelyek egyszerre több tudományág illetékességébe tartoznak". Ezen tantárgyközi tudások és képességek szerepe annyira megnőtt a XXI. században, hogy akárhogyan is definiálnánk a tantárgyakat, a pedagógusok és a tanulók tevékenységének nagy része nem lenne besorolható egyetlen tantárgy alá. Iskoláinkban a tantárgyak ezért elsősorban a modulok kiírásakor és azok kimeneti értékeléskor jelennek meg, a mindennapok struktúráját, a napi- és hetirendet azonban a modulok adják. Egyes modulok több tantárgy fejlesztési céljainak is eleget tehetnek, összhangban a NAT-tal. A tantárgyaknak ezzel együtt fontos célja, hogy segítse a tanulás tartalmi egyensúlyának fennmaradását. A tanárok, modulvezetők szakképesítése nem köthető a Budapest School tantárgyaihoz, felelősségük, hogy a saját moduljukban megfelelően tudják szervezni a tanulást és legfőképp, hogy saját moduljuk megtartására alkalmasak legyenek.

\subsection[STEM]{STEM  (Természet- és mérnöki  tudomány, technológia és matematika)\sectionmark{STEM}}
\emph{A tantárgy a miniszter által kiadott kerettanterv következő tantárgyainak fejlesztési területeire épül, és azt egészíti ki a saját célokkal és a tantárgyhoz kapcsolódó egyéb modulokkal:
Biológia, egészségtan, fizika, kémia, matematika, informatika, földrajz, környezetismeret, természetismeret.}


A Földön ma már több mint 4 milliárd ember használ telefont, a NASA 20 éven belülre tervezi az első Marsra szállást, az első önvezető autókat pedig már tesztelik szerte a világban. A jövő megérkezett és ahhoz, hogy formálóivá válhassunk, a tudományos gondolkodás és alkotás egységes megismerésére van szükség. A Földünket és az univerzumot alakító alap-törvényszerűségek megértése és az új technológiákban való jártasság éppoly fontos, mint az, hogy megértsük és jól használjuk az internetet, ezt a világot tükröző és összekapcsoló komplex hálózatot.

A STEM ma már nem csak néhány hangzatos tantárgy összevonása miatt releváns. Ma már a fejlesztés, az innováció az élet minden területén megjelenik, és nem csak eldugott laborok mérnökeinek a feladata új dolgokat felfedezni, kipróbálni, alkotni. Ezt az is mutatja, hogy a világ legnagyobb mértékben bővülő munkáinak több mint fele STEM-alapú és ezek száma a következő évtizedekben a kutatások szerint nőni fog. A STEM a jövő nagy kérdéseit hivatott megválaszolni. Épp ezért a STEM a felfedezés és a megismerés iránti vágyat kell, hogy ébren tartsa, és tovább mélyítse a tanulókban, miközben gyakorlatias alkotó módszerével fejleszti a csapatmunkát és a kitartást, valamint azt, hogy a megszerzett ismereteket új, korábban nem tapasztalt helyzetekben is alkalmazni tudják. Ez az a szemléletmód, ami egyaránt segítheti az egyén akadémiai tudásának fejlődését és a tanulás iránti hosszú távú elköteleződését.

\paragraph{STEM fejlesztési célok}
\begin{itemize}
\item Problémamegoldási képesség
\item  Matematikai gondolkodás
\item  Logikus gondolkodás
\item  A világ működésének mélyebb megértése a természettudományos szemléleten keresztül
\item  Innovatív fejlesztésekhez szükséges technológiai (kódolás) és gyakorlati (maker) eszköztár megismerése és alkalmazása
\end{itemize}

A STEM komplex, tudományterületeken és a technológiai lehetőségeken átivelő megközelítése sokfelé ágazik, mégis vannak olyan alapszabályai, amelyek a tanulás során mindig jellemzik. Ahhoz, hogy folyamatos fejlődés legyen, fontos, hogy az aktuális képességekhez igazodó, nyitott, tényeken alapuló, kutatás alapú, diverz módszertanú, skálázható és releváns módon történjen a STEM területen a tanulás.

A STEM tantárgy során a fenti célok eléréséhez a következő egymást segítő komponensek alkalmazása javasolt


\paragraph{Gyakorlatias tanulás elkötelezett és együttműködő közösségek hálózatához kapcsolódva.}
A STEM tanulás során különösen fontos, hogy az adott területet jól ismerő tanáron és a mentoron túl további innovatív és segítő közegek is támogassák a tanulók fejlődését. Sokat hozzá tehet a mindennapi élethez való kapcsolódás kialakításában, ha intézményi (pl múzeum, kutatóintézet), vagy technológiai, ipari és más STEM-hez kapcsolódó területen működő cégeknél dolgozó emberek mintákat mutatnak, vagy akár bekapcsolódnak a tanulók mindennapos munkájába.

\paragraph{Hozzáférhető tanulási gyakorlatok, amelyek a játék és a kockáztatás eszközeit is felhasználják.} A STEM tematikájú játékok segítik a tanulókat abban, hogy egymástól és együtt is tanuljanak, ezzel fejlesztik a kreativitásukat és a felfedezés iránti vágyukat. Ezek a felfedeztető játékok ráirányítják a fókuszt a csapatmunka fontosságára és arra, hogy miként lehet mai problémákra és kihívásokra válaszokat találni.

\paragraph{Multidiszciplináris tanulási élmények, amelyek a világ nagy kihívásaira keresnek válaszokat.} A STEM legfőbb kihívása, hogy a mai kor olyan kérdésfelvetéseit vizsgálja, amelyekre közösségi, nemzeti, vagy globális szinten még nincsenek meg a válaszok. Ezek lehetnek a vízgazdálkodással, az agykutatással és annak a prevencióra, vagy épp a gyógyulásra gyakorolt hatásaival, a megújuló energiagazdálkodással, vagy épp az önvezető autók új generációjának technológiai irányaival kapcsolatosak. Ezek a kihívások egyúttal azt is megmutatják, hogy mely kérdések lehetnek a kultúránk szempontjából relevánsak.

\paragraph{Rugalmas és befogadó tanulási környezet.} Nagyban hozzájárulhatnak a STEM élményszerű tanulásához a nyitottan értelmezett tanulási környezetek. Az iskolai terek alkotótérré változtatásával, a természeti közegekben tett terepmunkák, vagy a korszerű technológiai platformok, mint például a VR (virtuális valóság – virtual reality), vagy AR (kiterjesztett virtuális valóság – augmented virtual reality) bevonásával a területek könnyebben megismerhetővé válnak és újabb kérdések feltevésére sarkallhatják a tanulókat, miközben az irányító tanárszerep helyett a kísérletező, segítő és facilitátor feladatok erősödnek.


\paragraph{A tanulási eredmények mérésének új eszköztára.} A teljesítmény, a kutatás, kísérletezés és az alkotás kiemelt fókuszú a gyerekek STEM tanulási útja során. Kiemelt szerepet kapnak ezért a saját kutatásokra épülő prezentációk, megfigyelések, és az azokra adott értékes visszajelzések.

\subsection[KULT]{KULT (Kultúra, nyelv és művészet)}
\emph{A tantárgy a miniszter által kiadott kerettanterv következő tantárgyainak fejlesztési területeire épül, és azt egészíti ki a saját célokkal és a tantárgyhoz kapcsolódó egyéb modulokkal: Magyar nyelv és irodalom, idegen nyelv, vizuális kultúra, dráma és tánc, hon- és népismeret, történelem, társadalmi és állampolgársági ismeretek, ének-zene.}

A művészetek és önmagunk kifejezése az őskortól segítik az embereket a túlélésben. A képzelet absztrakciós szintjének alkalmazása az egyik legfontosabb faji tulajdonsága az embernek. Ennek fejlesztése pedig globális világunk egyik legnagyobb kihívása. Ahhoz, hogy a társadalmi viták és szabályozások aktív alakítójává válhassunk, hogy megismerjük a globális világunk legnagyobb kihívásait, értenünk kell a múltunkat, a saját kultúránkat és a kultúrák szerepét általában, és ki kell tudnunk fejezni magunkat olykor szavakkal, máskor a művészet eszközeivel.

Még soha nem élt az emberiség ennyire behálózott világban, és még soha nem kellett ennyire tudatosan készülnünk arra, hogy gyerekeinknek többféle kultúrát, társadalmi hálózatot kell megérteniük, és abban eligazodniuk. Családok, munkahelyi és lakókörnyezeteink, sőt még a nemzeti és az azokat átívelő társadalmi környezetek is gyorsabban változnak ma, mint szüleink életében. Ezért szeretnénk, hogy gyerekeink stabil identitásukra épitkezve képesek legyenek emberként emberekhez kapcsolódni, embertársaikat megérteni, velük együtt élni, dolgozni.

Ahhoz, hogy gyerekeink a társadalmi folyamatok aktív alakítójává válhassanak, hogy megismerjék a globálisan összefonódó világunk legnagyobb kihívásait, érteniük kell a múltunkat, a saját kultúránkat és a kultúrák szerepét általában, és ki kell tudnunk fejezni magunkat olykor szavakkal, máskor a művészet eszközeivel.

A tanulás egyik legfőbb funkciója, hogy képessé váljunk egy fenntartható élet kialakítására. Ehhez pedig önmagunk, környezetünk (Harmónia) és a világ működésének megismerésén (STEM) és fejlesztésén túl a szűk és tágabb értelemben vett kulturális élettereinkhez is kapcsolódnunk kell (KULT). Meg kell ismernünk a lokális és a globális kihívásokat ahhoz, hogy értő, empatikus módon kapcsolódhassunk a saját és más kultúrákhoz és a fenntartható fejlődés alakítójává válhassunk. A Budapest School KULT tantárgy fő célja ezért az olyan globális kompetenciák fejlesztése, amelyek a fenti cél elérését támogatják.

KULT tantárgy fejlesztési céljai:
\begin{itemize}
\item írott és beszélt anyanyelvi kommunikáció

\item írott és beszélt idegen nyelvi kommunikáció

\item mai lokális és globális kihívások megismerése a múlt kontextusában

\item kulturális diverzitás és a kultúra mint az emberi viselkedést leíró eszköztár megismerése

\item művészetek stílus és formavilága,

\item művészetek mint önkifejezési eszköz a vizuális (hagyományos képzőművészetektől a digitális művészetekig) és előadóművészetekben (dráma, tánc, zene)
\end{itemize}

Miközben világunkban egyre nő a technológia szerepe, folyamatosan nő az igény arra is, hogy képesek legyünk értelmezni és megfelelően használni az elénk táruló információt. Ennek alapja a megfelelő szövegértés, a nyelv mint írott és verbális kommunikációs stratégiának az egyéni és korosztályi képességekhez mért alkalmazása, valamint a művészeteken keresztül a kifejezés egyéb módjainak elsajátítása. Ahhoz, hogy képessé váljunk erre, megfelelő történelmi és kulturális kontextusba kell helyeznünk az elénk táruló információt, és nyitottnak kell lennünk arra, hogy ezeket befogadjuk, és a mai kor globális kihívásaihoz képest értelmezhessük.

A KULT komplex tanulási struktúrája adja meg az új információk elmélyülésének alapjait, és segít abban, hogy a mindennapos döntési stratégiáink részévé váljanak.

A KULT tantárggyal az alábbi fejlesztésekre törekszünk:
\begin{itemize}
\item Empátia és együttérzés

\item Egyéni és közösségi hiedelmek megismerése

\item Kritikai gondolkodás

\item A művészetek és az irodalom funkciójának megértése és a mindennapi életbe való beépítése

\item Az etika és a morál alapkérdéseinek felfedezése és megkérdőjelezése

\item Ismeretlen vagy komfortzónán túli ötletek befogadása

\item Az emberiség kulturális örökségének és aktuális diverzitásának megismerése

\item Többsíkú gondolkodás, multidiszciplináris szövegértés
\end{itemize}

A KULT tantárgy során a fenti célok eléréséhez a következő egymást segítő komponensek alkalmazása javasolt

\paragraph{Helyi közösségekhez való kapcsolódás}

A KULT tanulás során a lokális közösségek szerepe megnő, hiszen mind a saját kultúra és művészeti értékek megismerése, mind a nyelvhasználat tekintetében döntő szerepe van annak, hogy saját környezetünkben  tudjuk ezeket érteni és alkalmazni.

\paragraph{Kortárs kihívások}

A KULT alapját a kortárs művészeti értékek, a kortárs problémák és kihívások adják. Ezek ugyan mindig a múlt és a környezet kontextusában értelmezhetők, de elsődleges módszer a kapcsolat kialakítása a mai világ alkotásaival, szövegeivel, kulturális értékeivel.

\paragraph{Multidiszciplináris tanulási élmények, amelyek a világ nagy kihívásaira keresnek válaszokat}

A KULT abban segíti a tanulót,  hogy a mai világban eligazodhasson, és olyan kérdéseket tegyen fel, amelyek ma még megválaszolatlanok, és a jövőt formálhatják. Ehhez a művészetek, nyelvek és kultúrák találkozási pontjait, az átmeneteket is célszerű szemlélni, ami a világ egyik fő értéke.

\paragraph{Rugalmas és befogadó tanulási környezet}

A KULT tanulása során a mai szövegek értelmezése, az online elérhető tartalmakban való eligazodás éppoly fontos, mint az idegen nyelveken történő kommunikáció elősegítése a világhoz történő kapcsolódással. Az aktuális globális kihívásokat a múltunkon keresztül érthetjük meg, amihez tanulási környezetünket folyamatosan a megismerés igényeihez képest kell formálnunk és olyan eszközöket kell használnunk, amelyek az értelmezést segíthetik.

\paragraph{Alkotói szabadság és az alkotó szabad értelmezése}

Az önálló alkotói munka és az alkotások szabad értelmezése és befogadása az innováció és a kreativitás alapja. Ennek biztosításához a hagyományos keretek újraértelmezésére van szükség és arra, hogy egy alkotói folyamatban a saját cél kibontakozásának támogatása történjen meg.

\paragraph{A tanulási eredmények mérésének új eszköztára}

A kommunikáció lehetőségei iránti elköteleződés, az önkifejezés és a környezetünkhöz való kapcsolódás különféle módjainak alkalmazása különösen fontos módszer. Az alkotói munka eredményei mérhetők a saját fejlődési ütemhez képest. Hasonlóképp az írott és szóbeli kifejezés és a prezentáció egyéni és csoportos módjai is megfelelő mérési eszközök.

\subsection[Harmónia]{Harmónia (fizikai, lelki jóllét és kapcsolódás a környezethez)}
\emph{A tantárgy a miniszter által kiadott kerettanterv következő tantárgyainak fejlesztési területeire épül, és azt egészíti ki a saját célokkal és a tantárgyhoz kapcsolódó egyéb modulokkal: erkölcstan, testnevelés, technika, életvitel és gyakorlat, informatika.}

Testi és lelki egészségünk az alapja annak, hogy tanulhassunk, fejlődhessünk, és a saját életünk alakítóivá váljunk. Életünk fizikai, lelki, érzelmi és társas aspektusai határozzák meg a kapcsolódásunkat önmagunkhoz, társainkhoz és az őket körülvevő emberekhez, vagyis a tágabb értelemben vett társadalomhoz. Ez segít abban, hogy önálló döntéseket hozzunk, és jól tudjunk együtt tanulni, dolgozni csoportokban. Ahhoz, hogy a közösségünk részeként harmóniában élhessünk önmagunkkal, az épített és természeti környezetünkhöz is kapcsolódnunk kell.


A Budapest School tanulási koncepciójának középpontjában az egyén mint a közösség jól funkcionáló, saját célokkal rendelkező tagja áll. Az iskolában való fejlődése során elsősorban azt tanulja, hogy miként tud specifikált saját célokat megfogalmazni, és hogyan tudja ezeket elérni. Ebben a folyamatban egy mentor segíti a munkáját az iskola kezdetétől a végéig. Ő figyel arra, hogy a gyerek fizikai és lelki biztonsága és fejlődése folyamatos legyen, és segíti azokban a helyzetekben, amikor biztonságérzete vagy stabilitása csökken. A közösségben jól funkcionáló egyén belső harmóniájához ez a tantárgy a következő fejlesztési célokat határozza meg:
\begin{itemize}
\item Érzelmi és társas intelligencia

\item Önismeret és önbizalom

\item Konfliktuskezelés

\item Rugalmasság (rezíliencia)

\item Kritikai gondolkodás

\item Közösségi szabályok alkotásában való részvétel és azok alkalmazása

\item Csapatmunka gyakorlati fejlesztése

\item Oldott játék

\item Egészséges testi fejlődés

\item Saját igényekhez képest megfelelő táplálkozás

\item A természettel való kapcsolódás

\item Épített falusi és városi környezetben való eligazodás

\item A technológia világában felhasználói szintű eligazodás és annak harmonikus alkalmazása
\end{itemize}

A Harmónia tantárgy során a fenti célok eléréséhez a következő egymást segítő komponensek alkalmazása javasolt:

\paragraph{Közösségben, csapatban}

A Budapest School egy közösségi iskola, ahol a közösség tagjai egymással és egymástól tanulnak. A közösségekhez való tartozáshoz, a csapatban való gondolkodáshoz, és a családban való működéshez szükséges képességeket leginkább úgy tudjuk fejleszteni, ha azt kezdetektől megéljük. A közösség belső szabályainak megalkotása és az azokhoz való kapcsolódás a tanulás folyamatosságának alapfeltétele.

\paragraph{Életképességek (life skills)}

Szeretnénk, ha gyerekeink általában alkalmazkodóan (adaptívan) és pozitívan tudnának hozzáállni az élet kihívásaihoz , ha lelki és fizikai erősségük és rugalmasságuk (rezílienciájuk) megmaradna és fejlődne.   A Egészségügyi Világszervezet (WHO) a következőképpen definiálta\cite{oecd99lifeskills} az életképességeket
\begin{itemize}
\item Döntéshozás, problémamegoldás

\item Kreatív gondolkodás

\item Kommunikáció és interperszonális képességek

\item Önismeret, empátia

\item Magabiztosság (asszertivitás) és higgadtság

\item Terhelhetőség és érzelmek kezelése, stressztűrés
\end{itemize}
\paragraph{Érzelmi intelligencia}

Sokszor kiemeljük az érzelmi intelligenciát, kihangsúlyozva, hogy gyerekeinknek többet kell foglalkozniuk az érzelmek felismerésével, kontrollálásával és kifejezésével, mint szüleinknek kellett.

\paragraph{Szabad mozgás és séta}

A különböző mozgásformák, sportok és a séta mindennapossá tétele természetes módon, a gyerekek saját igényei szerint kell hogy történjenek.

\paragraph{Gyakorlatias, mindennapi képességek}

Ahhoz, hogy gyerekeink önállóan és hatékonyan tudják élni életüket, hogy a társakhoz való kapcsolódás ne függőség legyen, egy csomó praktikus mindennapi tudást el kell sajátítaniuk. A gyerekek folyamatosan kell, hogy fejlesszék az élethez szükséges mindennapi tudást a levélszemét kezeléstől, a facebook profil tudatos használatán át, egészen a személyi költségvetés készítéséig.

\paragraph{Egészséges táplálkozás}

Az egészséges táplálkozás tanulható viselkedésforma, melynek alapja nem csupán a megfelelő élelmiszerek kiválasztása, hanem azok élettani hatásainak megismerése, és az étkezési szokások alakítása is.

\section{NAT céljainak támogatása}

A Nemzeti Alaptantervben szereplő fejlesztési célok elérését és a kulcskompetenciák fejlődését több minden támogatja:

Egyrészt a tantárgyak lefedik a NAT fejlesztési céljait, kulcskompetenciáit és műveltségi területeit, mert a jelenleg érvényben lévő, a miniszter által a \emph{51/2012. (XII. 21.) számú EMMI rendelet I-IV. mellékletében} kiadott kerettantervek\cite{ofi:kerettanterv} tanulási, tanítási eredményeiből indultunk ki. Mivel a rendeletben szereplő kerettantervek teljesítik a NAT feltételeit, így a Budapest School tantárgystruktúrája is teljesíti ezeket.

Másrészt az iskola életében, folyamatában való részvétel, már önmagában biztosítja a kulcskompentenciák fejlődését és a NAT fejlesztési céljainak teljesülését sok esetben.

A \ref{tbl:nat_fejlesztesi} táblázat bemutatja a NAT fejlesztési területeihez való kapcsolódást, a
\ref{tbl:nat_kulcs} táblázat pedig az illeszkedési pontokat a NAT kulcskompetenciáihoz.

\begin{table}

  \begin{tabular}{p{5cm}|>{\raggedright}p{3cm}|p{3cm}}


    \textbf{NAT Fejlesztési célok} & \textbf{Tantárgyak} & \textbf{Struktúra}\\ \hline
Az erkölcsi nevelés & kult, harmónia & közösség\\ \hline
Nemzeti öntudat, hazafias nevelés & kult, harmónia & projektek\\ \hline
Állampolgárságra, demokráciára nevelés & kult, harmónia & közösség\\ \hline
Az önismeret és a társas kultúra fejlesztése & kult, harmónia, stem & saját tanulási út, közösség\\ \hline
A családi életre nevelés & harmónia &  \\ \hline
A testi és lelki egészségre nevelés & harmónia & közösség\\ \hline
Felelősségvállalás másokért, önkéntesség & harmónia & közösség, projektek\\ \hline
Fenntarthatóság, környezettudatosság & harmónia, stem & projektek\\ \hline
Pályaorientáció & kult, harmónia, stem & saját tanulási út\\ \hline
Gazdasági és pénzügyi nevelés & kult, harmónia, stem & projektek\\ \hline
Médiatudatosságra nevelés & kult & projektek\\ \hline
A tanulás tanítása & kult, harmónia, stem & saját tanulási út, mentorság\\

  \end{tabular}
  \caption{A NAT fejlesztési céljainak elérését nem csak a tantárgyak, hanem az iskola struktúrája is támogatja.}
  \label{tbl:nat_fejlesztesi}
\end{table}


A \emph{saját tanulási} út fogalma például önmagában segíti a tanulás tanulását, hiszen az a tanuló, aki képes önmagának saját célt állítani (mentor segítséggel), azt elérni, és a folyamatra való reflektálás során képességeit javítani, az fejleszti a tanulási képességét.

Vagy másik példaként, a Budapest School iskoláiban a \emph{közösség} maga hozza a működéséhez szükséges szabályokat, folyamatosan alakítja és fejleszti saját működését a tagok aktív részvételével. Ez az aktív állampolgárságra, a demokráciára való nevelés Nemzeti Alaptantervben előírt céljait is támogatja.

\begin{table}
  \centering
      \begin{tabular}{p{5cm}|>{\raggedright}p{3cm}|p{3cm}}


        \textbf{NAT kulcskompetenciái} & \textbf{Tantárgyak} & \textbf{Struktúra}\\ \hline
Anyanyelvi kommunikáció & kult & tanulási szerződés, portfólió\\ \hline
Idegen nyelvi kommunikáció & kult & Idegennyelvű modulok\\ \hline
Matematikai kompetencia & stem & \\ \hline
Természettudományos és technikai kompetencia & stem & \\ \hline
Digitális kompetencia & Harmónia, stem & Digitális portfólió kezelés\\ \hline
Szociális és állampolgári kompetencia & harmónia & saját tanulási út, közösség\\ \hline
Kezdeményezőképesség és vállalkozói kompetencia & kult, harmónia, stem & saját tanulási út, közösség\\ \hline
Esztétikai-művészeti tudatosság és kifejezőkészség & Harmónia, kult & \\ \hline
A hatékony, önálló tanulás & kult, harmónia, stem & saját tanulási út, mentorság\\

      \end{tabular}
      \caption{NAT kulcskompetenciáinak fejlesztését támogatják a tantárgyak és az iskola felépítése is.}
      \label{tbl:nat_kulcs}
    \end{table}

\section{A tanév ritmusa}

A tanév egy hármas ciklus háromszori ismétlődésével írható le. A tanulási célok tervezése, majd a tanulás, végül a visszajelzés és értékelés köre adja az állandóságot és a kereteket a tanulás irányításához. Amint egy ciklus véget ér, elkezdődik egy új. A keretek megtartásáért az egyes mikroiskolák tanárcsapatai felelnek, melynek működését a fenntartó monitorozza. A tanév ritmusát a \ref{tbl:tanevritmus} ábra mutatja.



			\begin{table}
\centering
				\begin{tabular}{ l|l }
    \textbf{időszak} & \textbf{tevékenység}\\
    \hline
    Szeptember&
    Közösségépítés \\
  & Saját célok meghatározása \\
  & Modulok kialakítása és meghirdetése
  \\ \hline

    Október&
    Tanulás, alkotás
  \\ \hline

    November&
    Tanulás, alkotás
  \\ \hline

    December&
    Portfólió frissítése\\
    & reflexiók, \\
    & visszajelzések \\
    & célok felülvizsgálata \\
    & modulok változtatása igény esetén
  \\ \hline

    Jánuár&
    Tanulás, alkotás
  \\ \hline

    Február&
    Tanulás, alkotás
  \\ \hline

    Március&
    Portfólió frissítése, \\
    & reflexiók,\\
    & visszajelzések \\
    & célok felülvizsgálata \\
    & modulok változtatása igény esetén
  \\ \hline

    Április&
    Tanulás, alkotás
  \\ \hline

    Május&
    Tanulás, alkotás
  \\ \hline

    Fél június&
    Évzárás, értékelés, bizonyítványok
    \end{tabular}
    \caption{Egy tanévben háromszor ismételjük a célállítás, tanulás, reflektálás ciklust.}
    \label{tbl:tanevritmus}
  \end{table}


A tanév három periódusból áll: ez a felosztás követi az üzleti világ negyedéves tervezését, néhány egyetem trimeszterekre bontását, de leginkább az évszakokat. Minden periódus után értékeljük az elmúlt három hónapot, ünnepeljük az eredményeket, és megtervezzük a következő időszakot.  A trimesztereken belül az egyes mikroiskolák között lehetnek néhány hetes eltérések, melyek a közösség sajátosságait követik.

\paragraph{Heti óraszámok}

A Budapest School közösségi tanulási élményeket és modulokat szervez a tanulóinak, egyúttal lehetőséget ad arra, hogy a gyerekek a közösen kialakított szabályaik mentén tanulásszervezők felügyeletével a Budapest School székhelyén vagy egyes telephelyein, vagy más, erre alkalmas tanulási környezetben tartózkodjanak. A közösségben együtt töltött idő tanulásnak, fejlődésnek minősül akkor is, ha az nem egy modulhoz kapcsolódik, hanem az ebéd élvezetéhez, vagy épp a parkban a lehulló falevelek sistergésének megfigyeléséhez.

A gyerekek, a tanítási szüneteket leszámítva, naponta 8 órát tartózkodnak az iskolában. Ezekben az időkben vannak a tanítási órák, foglalkozások, szakkörök, műhelyek. Az egyes mikroiskolák ettől 20\%-ban bármelyik irányban eltérhetnek, ha ez segíti a tanulásszervezők munkáját és a gyerekek fejlődését. Így hetente minimum $5 \cdot 8 \cdot 0.8 = 32$ órát, maximum 48 órát töltenek az iskolában.

Ennek alsó tagozatban körülbelül a felét, a felső tagozatban kétharmad részét töltik előre eltervezett módon, azaz modulokkal. A többi időben a tanárok vezetése és felügyelete mellett szabadon alkotnak, játszanak, pihennek, közösségi életet élnek. Azaz modulokra, alsó tagozatban 16--24, míg felső tagozatban 21--32 órát töltenek modulokkal.

Mivel az elvárt kiegyensúlyozottság miatt mind a három tantárgyra körülbelül ugyanannyi energiát kell fektetni, így az egyes tantárgyakra a teljes rendelkezésre álló időkeret egyharmadrészét kell számolni. Ettől az iskolák $\pm$ 20\%-ban eltérhetnek, így kiszámolható, hogy tantárgyanként minimum mennyi időt kell egy-egy gyereknek egy héten foglalkoznia. Ezt összegzi a \ref{tbl:oraszamok} táblázat.


			\begin{table}

				\begin{tabular}{ l|l|l }

					\textbf{Tantárgy} & \textbf{Alsó tagozat} & \textbf{Felső tagozat}\\ \hline
Harmónia & $\frac{5 \times 8 \times 0.8}{2} \times \frac{1}{3} \times 0.8 = 4.25$ óra &
$\frac{5 \times 8 \times 0.8 \times 2}{3} \times \frac{1}{3} \times 0.8 = 5.68$   óra\\ \hline
STEM & 4.25 óra &  5.68 óra\\ \hline
KULT  & 4.25 óra & 5.68 óra\\ \hline

				\end{tabular}
        \caption{Az elvárt kiegyensúlyozottság miatt a tantárgyakkal egyenlő minimális óraszámban kell foglalkozni.}
        \label{tbl:oraszamok}
			\end{table}

Fontos, hogy \emph{egy-egy modul több tantárgyként is beszámítható.}

\section{Évfolyamok, bizonyítvány}
\label{sec:evfolyamok}

A tantárgyközi, moduláris tanulás közben a visszajelzések és a portfolió elemek egybeállítása formájában monitorozzuk, hogy az egyes fejlesztési célok megjelenjenek a mindennapos tanulás során. Így a Budapest School a NAT pedagógiai szakaszainak végén minden esetben, közben pedig a tanuló/szülő kérésére elvégzi a miniszter által kiadott kerettanterv tantárgyi rendszere szerinti értékelést és az annak történő megfeleltetést.

A Budapest Schoolban az évfolyamokra úgy tekintünk, mint egy játék nehézség szintjeire: akkor léphet egy tanuló a következőbe, ha elvégezte az adott évfolyamhoz köthető tantárgyi követelményeket. És remélhetőleg egyre nagyobb kihívások várnak rá a következő szinten. A hagyományos évfolyamoktól eltérően

\begin{itemize}
\item Egy tanuló minden tantárgyból állhat más szinten.
\item Nem biztos, hogy az egy korcsoportba tartozók vannak ugyanazon az évfolyam-szinten.
\item Nem mindig az egy évfolyam-szinten lévők tanulnak együtt, sokszor előfordulhat az is, hogy a különböző szinten lévő tanulók tudnak együtt és akár egymástól is tanulni.
\item Egy év alatt több évfolyam-szintet is lehet lépni.
\end{itemize}

 A bizonyítványába azonban mindig annak az évfolyamnak az elvégzése kerül be, amelyből minden kötelező tantárgyhoz szükséges fejlesztési célt elérte. Formálisabban kivejezve a tantárgyankénti évfolyam-szintek minimumát kell a bizonyítványban rögzíteni.

\subsection{Előre haladás igazolása}
A tanulók portfóliója alapján megállapítható, hogy egy adott évfolyamhoz köthető tantárgyi követelményeknek megfelel-e. Ehhez a tanulók elvégzik a mentoruk segítségével a portfóliójuk (mit csináltak, mit tanultak, mit tudnak) összehasonlítását a \ref{sec:tantargyi_celok} fejezetben felsorolt tantárgyi eredménycélokkal. Ha szükséges, akkor a kapcsolódás biztosításához a portfóliójukat kiegészíthetik tudáspróbák/szintfelmérő vizsgák teljesítésével, melynek megszervezése az adott mikroiskola tanárközösségének a feladata.

Amennyiben valamely diáknál egy adott évfolyam tantárgyi követelményei elismerésre kerülnek, akkor az iskola hivatalosan is igazolja, hogy a diák az adott tantárgy vagy tantárgyak évfolyam szerinti követelményeit teljesítette.

Egy tantárgyból egy évfolyam teljesítettnek tekinthető, ha a tantárgyhoz tartozó tanulási célok 50\%-ának elérése a portfólió alapján bizonyítható.

Az iskola automatikusan elkészíti a 2., 4., 6. és 8., 10. és 12. évfolyam/szint követelményeinek teljesítésekor, vagy a tanuló/szülő kérésére ettől eltérő időben is, a fenti hitelesítési, igazolási folyamatot, ami alapján bizonyítvány állítható ki.

\subsection{Szöveges értékelés, érdemjegyek}
TODO
A Budapest Schoolban a tanulásra való visszajelzés csupán egyik eleme, hogy érdemjegyet adunk. Ennek fő célja, hogy a fejlesztési célokhoz való kapcsolódást kifejezzük. A tanulás legfőbb mércéje a gyerek önmagához képesti fejlődésének mérése a kortárs és a tanári visszajelzések alapján, valamint a portfolióban összegyűlt tanult, tapasztalt és létrehozott tartalmak rögzítése formájában.  Amikor a tanuló következő évfolyamba léphet, akkor ez azt jelenti, hogy a portfóliója alapján megállapítható, hogy \emph{elsajátította a korosztályának megfelelő fejlesztési célokat}.


\section{Minőségbiztosítás, folyamatszabályozás}
\label{sec:minosegbiztositas}
\begin{quote}
Honnan tudjuk, hogy jól működnek a mikroiskolák? Minden gyerek azt tanulja, amit szeretne, és amire neki a leginkább szüksége van?
\end{quote}

A XX. század iskoláit sokszor a hagyományos gyárak működésével modelleztük: bejön az input (gyerek), amivel az előre meghatározott munkafolyamatokat elvégezzük (tantárgyak tananyagát a tanmenetet alapján leadjuk), és kijön az eredmény (gyerek tudással, képességgel, attitűddel felvértezve). Ezek a gyárak zárt rendszerként működtek: a kimenet (output) minősége csak a bemenet minőségétől és a gyár folyamataitól függött. Így a minőségbiztosításnak nem volt más dolga, mint szabályozni és keretek között tartani a gyártósor gépeit (itt a tanmeneteket), és folyamatosan ellenőrizni az outputot, és a hibákat detektálni.

A Budapest School iskoláiban eltérő modellt használunk a gyerekek tanulásának leírására. Iskoláink egy, a világ felé nyitott hálózatot alkotnak: az egy közösségben lévő gyerekek, a családok, a velük foglalkozó tanárok, a helyi környezet, a Budapest School további mikroiskoláinak közössége, az ország és a nemzet állapota, valamint a globális társadalmi folyamatok is befolyásolják, hogy mi történik egy iskolában.

Az hogy, egy gyerek épp mit és mennyit tanul, nemcsak a kerettantervtől függ, hanem számos más tényező is befolyásolja a fejlődést és a közösség egymás fejlődésére gyakorolt hatását: többek közt függ a gyerekek múltjától, jelenlegi hangulatától és vágyaitól, a tanárok személyiségétől, függ a családtól, a csoportdinamikától, és még a társadalomban történő változásoktól is.

Ezért a gyermekeink fejlődését, tanulását és boldogságának alakulását egy \emph{komplex rendszer} működésével modellezhetjük.

\begin{quote}
Példa: a 11 éves Hanna gyakran kerül konfliktusokba az iskolában, ezért nincs kedve résztvenni a közösségi programokon. Sokat akar egyedül lenni. Épp kooperatív technikával dolgoznak fel STEM témákat az egyik modulban, és Hanna a társas aktivitásokat kerüli. Így az emberi test modelljének csoportos megépítése most nem érdekli.  Szintén nem vesz részt a közösség által megálmodott videósorozat létrehozásában, melynek célja, hogy  a youtube videók hatására rászoruló gyerekeknek gyűjtsenek adventi ajándékokat, így lemarad a média és művészet területeken is. Egyetlen dolog érdekli: a matek feladatokat, és a mindenféle matematikával kapcsolatos könyveket falja. Mindig egyedül. 3 hónap után a mentoridejében\footnote{Amikor a tanuló és mentora minőségi időt tölt együtt és a mentorált tanuló életéről, haladásáról beszélgetnek.} beszél először arról, hogy a szülei válása mennyire rosszul érinti. Ebben az esetben a tanárok, az iskola támogathatja  Hannát, de mindenki érzi, hogy az ő tanulmányi haladásában most lassulás várható.
\end{quote}
Belátható, hogy Hanna természettudományos fejlődése jobb lett volna, ha a tanár éppen hagyományosabb, frontális tanítási módszert választ, vagy ha nem válnának a szülei.

Egy komplex rendszerben nem lehet hagyományos módszerekkel biztosítani az eredményt és annak minőségét, mert attól, hogy a tanár mindig, mindent ugyanúgy csinál, nagyon különböző eredményeket érhetünk el egy-egy gyereknél.



Ezért a tanulási folyamataink minőségbiztosításánál a következő szempontokat vesszük figyelembe:
\begin{enumerate}
\item  A történések, események, (rész)eredmények folyamatos monitorozása,
\item folyamatos visszajelzés gyűjtése a rendszer a Budapest School tagjaitól: a tanulóktól, tanároktól, szülőktől és az adminisztrátoroktól,
\item anomália esetén helyzetfelismerés, az eltérések okának megvizsgálása,
\item a feltárt hibák alapján a rendszer javítása.
\end{enumerate}

Az egész minőségbiztosítás célja az iskola és a teljes  Budapest School hálózat, mint tanuló rendszer folyamatos fejlesztése. A monitorozás folyamatos, így hamar fel tudjuk ismerni az anomáliákat, és kivizsgálás után, ha szükséges, akkor az anomáliát meg tudjuk szüntetni, de még tanulni is tudunk belőle, és javítani a rendszert.
\begin{quote}
Példa: Az egyik iskolai csoportban hirtelen azt tapasztaljuk, mert számítógépes programmal monitorozzuk, hogy a trimesztert záró értékeléskor a tanulók nagy része nem teljesítette a saját tanulási célját. Önmagában a nem teljesítés még nem lehet gond, mert sokszor nem érjük el elsőre a kitűzött célt, de ebben az iskolában hirtelen nagyon megugrott a nem teljesítők aránya. A fenntartó munkatársai nem tudnak mást tenni, mint elolvassák a tanulási szerződéseket, és beszélnek a résztvevőkkel (diagnózis). Hipotézisként azt fogalmazzák meg, hogy a tanulási célok túl általánosan, félreérthetően voltak megfogalmazva, így nem segítették a fókusztartást a trimeszter során. Első és legfontosabb teendő: segítőt vonnak be, hogy a következő trimeszterben úgynevezett SMART\cite{wiki:smart} célokat állítsanak fel.

Elemző, úgynevezett retrospektív folyamat során felmerül, hogy az egyik új, Budapest Schoolban még kezdő mentortanár nem tudta eléggé segíteni a tanulók és szülők célállítását, ezért az új munkatársak kiképzésébe bekerül a SMART célállítás technikájáról egy rész.

Ezenkívül egy mesterséges intelligencia alapú szövegelemző fejlesztésébe kezdünk, ami az elkészült tanulási szerződéseket osztályozza az alapján, hogy mennyire specifikus, mennyire teljesíthető az adott gyereknek, és mennyire van összhangban a Budapest School kerettantervének elvárásaival.
\end{quote}




A fenntartó folyamatosan monitorozza és visszajelzéseket ad a mikroiskoláknak, ami alapján javítja a működési folyamatokat. A kerettantervben leírt működést a fenntartó mérhető és megfigyelhető metrikákra fordítja le, és kidolgozza, üzemelteti a metrikák 2-3 hónapnyi rendszerességű nyomon követésére alkalmas rendszert.

A fenntartónak meg kell figyelnie legalább a következő metrikákat:
\begin{enumerate}
\item A szülő, a gyerek és a tanár közötti egyéni célokat megfogalmazó hármas megállapodások időben megszülettek, nincs olyan gyerek, akinek nincs elfogadott saját tanulási célja. Metrika: elkészült szerződések száma

\item A modulok végén a portfóliók bővülnek, és azok tartalma a tantárgyakhoz kapcsolódik. Metrika: portfólió elemeinek száma és kapcsolhatósága

\item A szülők biztonságban érzik gyereküket, és eleget tudnak arról, hogy mit tanulnak. Kérdőíves vizsgálat alapján

\item A tanárok hatékonynak tartják a munkájukat, kérdőíves felmérés alapján

\item A gyerekek úgy érzik, folyamatosan tanulnak, támogatva vannak, vannak kihívásaik. Kérdőíves felmérés alapján
\end{enumerate}

\section{Tanárok kiválasztása, tanulása, fejlődése és értékelése}

Mindenki egyetért abban, hogy az iskola legmeghatározobb összetevői a tanárok. Ezért a Budapest School külön figyelmet fordít arra, hogy ki lehet tanár az iskolákban, és hogyan segítjük az ő fejlődésüket.

Alapelveink
\begin{enumerate}
\item Minden tanárnak tanulnia kell. Amit ma tudunk, az nem biztos, hogy elég arra, hogy a holnap iskoláját működtessük. És az is biztos, hogy még sokkal hatékonyabban lehetne segíteni gyerekek tanulását, mint amilyenek a ma ismert módszereink.

\item Tanároknak csapatban kell dolgozniuk, mert összetett (interdiszciplináris) tanulást, csak vegyes összetételű (diverz) csapatok tudnak támogatni.

\item Szakképesítés nem szükséges és nem elégséges feltétele annak, hogy a Budapest Schoolban jól teljesítő tanár legyen valaki.
\end{enumerate}
\paragraph{Bekerülés}
Budapest School tanár felvétele egy minimum háromlépcsős folyamat, ahol vizsgálni kell a tanár egyéniségét (attitűdjét), felnőtt-felnőtt kapcsolatokban a viselkedésmódját (társas kompetenciáit) és minden jelöltnek próbafoglalkozást kell tartania, amit erre kijelölt Budapest School tanárok megfigyelnek. A felvételi folyamatot, a Budapest School központ felügyeli és írányítja.

\paragraph{Saját cél}

Minden tanárnak van saját, egyéni fejlődési célja: \emph{mitől tudok én jobb tanár lenni, jobban támogatni a gyerekek tanulását, segíteni a munkatársaimat és partnerként dolgozni a szülőkkel?}

\paragraph{Mentor}

Hasonlóan a gyerekekhez, minden tanárnak van egy mentora, aki segíti a saját céljai kialakításában, és folyamatosan támogatja a célja elérésében.

\paragraph{Értékelés}

Minden tanárt évente legalább kétszer értékelnek a munkatársai. Ez a folyamat, amit 360 fokos értékelésnek hívnak az üzleti szférában. A visszajelzések feldolgozása után a saját célokat frissíteni kell.

Minden tanárt értékelnek a szülők is (kifejezetten a mentorált gyerekek szülei) és a a gyerekek is legalább évente kétszer.
