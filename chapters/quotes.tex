\newlength\longest
\cleardoublepage
\thispagestyle{empty}
\null\vfill
{\centering
\parbox{0.8\textwidth} {%
    \raggedright{\large\itshape%
    Ha egy gyerek nem érett meg az algebrára, hiába kínozzuk vele,  
    ha meg igen: nem kell hosszú évekig kanalas orvosságként adagolni neki. \\
    Az osztályban oktatás természete szerint idomítással gyötri az éretlent 
    és kedvét szegi az érettnek. A ,,csoportban'' olyan gyerekek kerülnek össze, akik egy-egy téren 
    nagyjából egyformán érettek. S addig rágják az előttük lévő anyagot, amíg erejükből futja\ldots\\ 
    Két-három hónap múlva a résztvevő új csoportba kerül, s más képességét edzi tovább.

        \par\bigskip
    }
    \raggedleft\Large\MakeUppercase{Németh László, 1945}\par%
    \raggedleft\normalsize{író (eredetileg fogorvos)}\par%
}\par%
}
\vfill\vfill
\bigskip
\null\vfill
{\centering
\parbox{0.8\textwidth} {%
    \raggedright{\large\itshape%
    Kutatások bizonyítják, hogy az önvezérelt csoportokban tanuló gyerekek, az internet segítségével, évekkel azelőtt képesek tudományos kérdések megválaszolására, mielőtt azok az iskolai tananyagukban megjelennek. Úgy tűnik, hogy valójában épp a felnőttek támogatásának hiányát élvezik, és kellően magabiztosak abban, hogy saját maguktól találjanak helyes válaszokat.

        \par\bigskip
    }
    \raggedleft\Large\MakeUppercase{Sugata Mitra, 2015}\par%
    \raggedleft\normalsize{oktatáskutató (eredetileg fizikus)}\par%
}\par%
}

\vfill\vfill
\clearpage

