\chapter{Jogszabályok által igényelt elvárások rövid összefoglalója}
\chaptermark{Egyéb jogszabályi elvárások}
\paragraph{A nevelés-oktatás célja}

Minél több olyan dolgot tanuljanak az iskola tanulói, amit szeretnek, vagy amire szükségük van úgy, hogy a mindenkori Nemzeti Alaptanterv követelményeit teljesítsék, és saját céljaikat is el tudják érni.

\paragraph{A tantárgyi rendszer}

A miniszter által kiadott kerettanterv tantárgyainak összevonásából létrejött három Budapest School tantárgy a mindennapokban nem jelenik meg önálló tanórákként az iskola életében, ezek funkciója a fejlesztési célok elérése és az egyensúlytartás. A tudástartalom elsajátítása a tanulási modulokon keresztül történik. A tantárgy elvégzésének feltétele, hogy a NAT pedagógiai szakaszainak befejeztével a gyerekek egyéni portfóliója és az itt szereplő követelmények kapcsolódjanak egymáshoz úgy, hogy a fejlesztési célokat legalább 50\%-ban eléri a tanuló és további 50\%-ban kiegészül a saját céljaival.

\paragraph{A tantárgyközi tudás- és képességterületek fejlesztésének feladata}

A Nemzeti Alaptanterv kulcskompetenciáit, fejlesztési területeit és még a műveltségi területek fejezetben megtalálható fejlesztési feladatokat is az iskola feladatának tekintjük. A modulok többsége nem tantárgyak alapján szerveződik, így nálunk a tantárgyközi tudás az alapértelmezett.

\paragraph{A követelmények teljesítéséhez rendelkezésre álló kötelező, továbbá ajánlott időkeret}

A \pageref{tbl:oraszamok}. oldalon található \ref{tbl:oraszamok} táblázat alapján a három tantárgy által leírt területekkel az idő egyharmad részében kell foglalkozni, ami minimum heti 4.25 óra alsótagozatban és 5.68 óra felsőtagozatban.

\paragraph{Kötött és kötetlen munkaidő szabályozása}

A Budapest School  iskolákban a tanárok teljesen kötetlen munkaidőben, sok esetben egyéni vállalkozóként dolgoznak, mert a mikroiskola tanárcsapatának a feladata a szükséges óraszámok, beosztások kialakítása és a megfelelő modulvezetők megtalálása. Amennyiben egy tanár maga kevesebb modult tart, és több munkát tölt a tanulás megszervezésével, akár több mikroiskola tanáraként is dolgozhat egy időben.

\section{Elfogadott pedagógus végzettség és szakképzettség}

A Budapest School iskolában tanulásszervező tanárok, mentortanárok és modulvezető tanárok dolgoznak együtt, hogy a gyerekek számára megfelelő tanulási környezetet, kihívásokat, kereteket, stb. biztosítsanak. A tanárok lehetnek alkalmazotti vagy megbízási jogviszonyban az iskolával. Sőt az is lehetséges, hogy egy külsős modulvezetőt harmadik fél (például a családok, önkormányzat, tanulásszervező tanárok) közvetlenül megbíznak, és így a modulvezető a Budapest Schooltől termet bérel csak.

Fontos alapelv, hogy a végzettségnél fontosabb a tanárok képessége és kompetenciái. Mindenki lehet tanár, aki képes segíteni a tanulókat a tanulásban. A végzettség egy lehetséges indikátora tudásuknak.

\paragraph{Tanulásszervezők és mentorok elvárt képesítések} Tanulásszervezők és a mentorok a mikro iskolák irányítói, a gyerekek tanulási környezetének kialakítói és menedzserei. Elvárt, hogy csoportos tanulási és fejlesztési helyzet vezetéséhez szüksége képességeik meglegyenek. Ehhez szükséges a következő végzettségek közül valamelyik, vagy ezeket kiváltó releváns, igazolható szakmai tapasztalat.
\begin{itemize}
\item andragógus
\item család- és gyermekvédő tanár
\item drámainstruktor
\item drámapedagógus
\item egészségügyi szakoktató
\item egészségügyi tanár
\item együttnevelést segítő pedagógus
\item fejlesztőpedagógus
\item felnőttoktató
\item gazdaságismeret tanár
\item gyakorlati oktató
\item gyakorlatvezető
\item gyerekcoach
\item gyermektanulmányi szakember
\item gyógypedagógiai tanár
\item gyógypedagógiai terapeuta
\item gyógypedagógus
\item humán segítő tereptanár
\item interkulturális pszichológia és pedagógia
\item iparművészeti oktató
\item képzési szakasszisztens
\item képzőművészeti oktató
\item kézműves, hagyományismeret oktató
\item konduktor
\item közgazdász tanár
\item közművelődési szakember
\item mentor
\item mérnöktanár
\item mezőgazdasági szakoktató
\item műszaki szakoktató
\item művelődésszervező
\item múzeumpedagógiai szaktanácsadó
\item népi játék és kismesterségek oktatója
\item neveléstudomány szakos bölcsész
\item nyelv-és beszédfejlesztő tanár
\item óvodapedagógus
\item pedagógia szakos bölcsész
\item pedagógiai asszisztens
\item pszichológus
\item sportedző
\item szabadidő-szervező
\item szakoktató
\item szociológus
\item társművészetek pedagógusa
\item tehetségfejlesztő tanár
\item teológus
\item trainer
\item üzleti szakoktató
\item zenei kultúrát fejlesztő pedagógus
\end{itemize}

\paragraph{Modulvezetőktől elvárt végzettségek}
Modulvezetőktől azt várjuk el, hogy a magasszinten ismerjék a modul tematikája által lefedett területet. Szakképesítést csak az érettségi felkészítő modulok esetén várunk el. Itt a tantárgyhoz tartozó tanári diploma elengedhetetlen.

A többi modul esetén a modul témájához kapcsolatos legalább 5 éves szakmai tapasztalat szükséges.

\paragraph{Asszisztensek}
A Budapest School iskolában megjelennek az asszisztensek, akik a tanulásszervező, mentor és modulvezető tanárok munkáját segítik, kiegészítik. Minden asszisztens minden foglalkozáson egy Budapest School tanárhoz van rendelve. Az asszisztensek munkájáért, visszajelzésért ilyenkor ez a tanár vállalja a felelősséget.


\section{Épületekre vonatkozó előírások}
\Aref{tbl:helyisegek}. táblázat meghatározza Budapest School Általános Iskola és Gimnázium munkájához kötelezően szükséges helyiségeket -- az Nkt. 9.§ (8) f pont felhatalmazása alapján -- a kerettantervben részletezett strukturális, szervezeti és tanulásszervezési elvek és folyamatok, különösen a mikroiskola hálózatos működési jelleget figyelembe véve.


\begin{table}[thb]

\begin{center}
\begin{tabular}{@{}p{4cm}|p{4cm}|p{6cm}@{}}

\textbf{helyiség megnevezése}   & \textbf{mennyiségi mutató}                       & \textbf{megjegyzés}                                                                                      \\ \hline
tanterem               & 16 gyerekenként egy                       & 1,5 nm / gyerek előírás figyelembe vételével                                                    \\ \hline
tornaterem             & iskolánként egy                           & kiváltható szerződéssel                                                                         \\ \hline
tornaszoba             & székhelyenként, telephelyenként egy       & ha a székhelynek/telephelynek nincs saját tornaterme                                            \\ \hline
sportudvar             & székhelyen, telephelyeken egy             & kiváltható szerződéssel, vagy helyettesíthető alkalmas szabad területtel                        \\ \hline
sportszertár           & iskolánként egy                           & tornateremhez kapcsolódóan (kiváltható szerződéssel)                                            \\ \hline
iskolatitkári iroda    & iskolánként egy                           & tantestületi szobával közös helyiségben  is kialakítható                                        \\ \hline
intézményvezetői iroda & iskolánként egy                           & iskolatitkári irodával közös helyiségben is kialakítható                                        \\ \hline
nevelőtestületi szoba  & iskolánként egy                           &                                                                                                 \\ \hline
könyvtár               & iskolánként egy                           & nyilvános könyvtár elláthatja a funkcót, megállapodás alapján                                   \\ \hline
orvosi szoba           & iskolánként egy                           & amennyiben egészségügyi intézményben megoldható a gyerekek ellátása megoldható, nem kötelező    \\ \hline
ebédlő                 & székhelyen, telephelyeken egy             & gyerek és felnőtt étkező közös helyiségben                                                      \\ \hline
főzőkonyha             & székhelyen, telephelyeken egy             & ha helyben főznek                                                                               \\ \hline
melegítőkonyha         & székhelyen, telephelyeken egy             & ha nem helyben főznek, de helyben étkeznek                                                      \\ \hline
tálaló-mosogató        & székhelyen, telephelyeken egy             & ha nem helyben főznek, de helyben étkeznek; melegítőkonyhával közös helyiségben is kialakítható \\ \hline
éléskamra              & székhelyen, telephelyeken egy             & ha helyben főznek                                                                               \\ \hline
szárazáru raktár       & székhelyen, telephelyeken egy             & ha helyben főznek                                                                               \\ \hline
földesáru raktár       & székhelyen, telephelyeken egy             & ha helyben főznek                                                                               \\ \hline
öltöző                 & székhelyen, telephelyeken egy             & gyerek öltözővel együtt kialakítható                                                            \\ \hline
személyzeti wc         & székhelyen, telephelyeken egy             &                                                                                                 \\ \hline
gyerek wc              & székhelyen, telephelyeken szintenként egy & tanulói létszám figyelembe vételével                                                            \\
\end{tabular}
\caption{Kötelező helységek listája.}
\label{tbl:helyisegek}
\end{center}
\end{table}
