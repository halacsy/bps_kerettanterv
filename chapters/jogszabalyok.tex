\chapter{Jogszabályok által igényelt elvárások rövid összefoglalója}
\chaptermark{Egyéb jogszabályi elvárások}
\paragraph{A nevelés-oktatás célja}

Minél több olyan dolgot tanuljanak az iskola tanulói, amit szeretnek, vagy amire szükségük van úgy, hogy a mindenkori Nemzeti Alaptanterv követelményeit teljesítsék, és saját céljaikat is el tudják érni.

\paragraph{A tantárgyi rendszer}

A miniszter által kiadott kerettanterv tantárgyainak összevonásából létrejött három Budapest School tantárgy a mindennapokban nem jelenik meg önálló tanórákként az iskola életében, ezek funkciója a fejlesztési célok elérése és az egyensúlytartás. A tudástartalom elsajátítása a tanulási modulokon keresztül történik. A tantárgy elvégzésének feltétele, hogy a NAT pedagógiai szakaszainak befejeztével a gyerekek egyéni portfóliója és az itt szereplő követelmények kapcsolódjanak egymáshoz úgy, hogy a fejlesztési célokat legalább 50\%-ban eléri a tanuló és további 50\%-ban kiegészül a saját céljaival.

\paragraph{A tantárgyközi tudás- és képességterületek fejlesztésének feladata}

A Nemzeti Alaptanterv kulcskompetenciáit, fejlesztési területeit és még a műveltségi területek fejezetben megtalálható fejlesztési feladatokat is az iskola feladatának tekintjük. A modulok többsége nem tantárgyak alapján szerveződik, így nálunk a tantárgyközi tudás az alapértelmezett.

\paragraph{A követelmények teljesítéséhez rendelkezésre álló kötelező, továbbá ajánlott időkeret}

A \pageref{tbl:oraszamok}. oldalon található \ref{tbl:oraszamok} táblázat alapján a három tantárgy által leírt területekkel az idő egyharmad részében kell foglalkozni, ami minimum heti 4.25 óra alsótagozatban és 5.68 óra felsőtagozatban.

\paragraph{Kötött és kötetlen munkaidő szabályozása}

A Budapest School  iskolákban a tanárok teljesen kötetlen munkaidőben, sok esetben egyéni vállalkozóként dolgoznak, mert a mikroiskola tanárcsapatának a feladata a szükséges óraszámok, beosztások kialakítása és a megfelelő modulvezetők megtalálása. Amennyiben egy tanár maga kevesebb modult tart, és több munkát tölt a tanulás megszervezésével, akár több mikroiskola tanáraként is dolgozhat egy időben.

\paragraph{Elfogadott pedagógus végzettség és szakképzettség}

A Budapest School iskolákban a tanárok, tanulásszervezők, mentorok képessége és kompetenciái a mérvadóak. Mindenki lehet tanár, aki képes segíteni a tanulókat a tanulásban. Ahhoz, hogy a tantárgyi fejlesztési céloknak megfelelő módon is segíthessék tanáraink a gyerekek fejlődését, a Budapest School alkalmaz szaktanárokat is, akik a portfólióknak a fejlesztési célokhoz való kapcsolását megfelelően el tudják végezni.
